\documentclass{artikel3}

% If you're new to LaTeX, here's some short tutorials:
% https://www.overleaf.com/learn/latex/Learn_LaTeX_in_30_minutes
% https://en.wikibooks.org/wiki/LaTeX/Basics

% Formatting
\usepackage[utf8]{inputenc}
\usepackage[margin=1in]{geometry}
\usepackage[titletoc,title]{appendix}
\usepackage[T1]{fontenc}
\usepackage{lmodern}


% Math
% https://www.overleaf.com/learn/latex/Mathematical_expressions
% https://en.wikibooks.org/wiki/LaTeX/Mathematics
\usepackage{amsmath,amsfonts,amssymb,mathtools}

% Images
% https://www.overleaf.com/learn/latex/Inserting_Images
% https://en.wikibooks.org/wiki/LaTeX/Floats,_Figures_and_Captions
\usepackage{graphicx,float}

% Tables
% https://www.overleaf.com/learn/latex/Tables
% https://en.wikibooks.org/wiki/LaTeX/Tables

% Code syntax highlighting
% https://www.overleaf.com/learn/latex/Code_Highlighting_with_minted
\usepackage{minted}
\usepackage{circuitikz}
\usemintedstyle{trac}

% % References
% % https://www.overleaf.com/learn/latex/Bibliography_management_in_LaTeX
% % https://en.wikibooks.org/wiki/LaTeX/Bibliography_Management
% \usepackage{biblatex}

% \addbibresource{references.bib}

% Title content
\title{My text title}
\author{Mateusz Kojro}
\date{\today}

\begin{document}

\maketitle

% Abstract
\begin{abstract}
    Add your abstract here.
\end{abstract}


  \section{Matplotlib import}
    \begin{figure}[h!]
        \begin{center}
            %% Creator: Matplotlib, PGF backend
%%
%% To include the figure in your LaTeX document, write
%%   \input{<filename>.pgf}
%%
%% Make sure the required packages are loaded in your preamble
%%   \usepackage{pgf}
%%
%% and, on pdftex
%%   \usepackage[utf8]{inputenc}\DeclareUnicodeCharacter{2212}{-}
%%
%% or, on luatex and xetex
%%   \usepackage{unicode-math}
%%
%% Figures using additional raster images can only be included by \input if
%% they are in the same directory as the main LaTeX file. For loading figures
%% from other directories you can use the `import` package
%%   \usepackage{import}
%%
%% and then include the figures with
%%   \import{<path to file>}{<filename>.pgf}
%%
%% Matplotlib used the following preamble
%%
\begingroup%
\makeatletter%
\begin{pgfpicture}%
\pgfpathrectangle{\pgfpointorigin}{\pgfqpoint{6.400000in}{4.800000in}}%
\pgfusepath{use as bounding box, clip}%
\begin{pgfscope}%
\pgfsetbuttcap%
\pgfsetmiterjoin%
\definecolor{currentfill}{rgb}{1.000000,1.000000,1.000000}%
\pgfsetfillcolor{currentfill}%
\pgfsetlinewidth{0.000000pt}%
\definecolor{currentstroke}{rgb}{1.000000,1.000000,1.000000}%
\pgfsetstrokecolor{currentstroke}%
\pgfsetdash{}{0pt}%
\pgfpathmoveto{\pgfqpoint{0.000000in}{0.000000in}}%
\pgfpathlineto{\pgfqpoint{6.400000in}{0.000000in}}%
\pgfpathlineto{\pgfqpoint{6.400000in}{4.800000in}}%
\pgfpathlineto{\pgfqpoint{0.000000in}{4.800000in}}%
\pgfpathclose%
\pgfusepath{fill}%
\end{pgfscope}%
\begin{pgfscope}%
\pgfsetbuttcap%
\pgfsetmiterjoin%
\definecolor{currentfill}{rgb}{1.000000,1.000000,1.000000}%
\pgfsetfillcolor{currentfill}%
\pgfsetlinewidth{0.000000pt}%
\definecolor{currentstroke}{rgb}{0.000000,0.000000,0.000000}%
\pgfsetstrokecolor{currentstroke}%
\pgfsetstrokeopacity{0.000000}%
\pgfsetdash{}{0pt}%
\pgfpathmoveto{\pgfqpoint{0.500000in}{3.638271in}}%
\pgfpathlineto{\pgfqpoint{2.030942in}{3.638271in}}%
\pgfpathlineto{\pgfqpoint{2.030942in}{4.531482in}}%
\pgfpathlineto{\pgfqpoint{0.500000in}{4.531482in}}%
\pgfpathclose%
\pgfusepath{fill}%
\end{pgfscope}%
\begin{pgfscope}%
\pgfsetbuttcap%
\pgfsetroundjoin%
\definecolor{currentfill}{rgb}{0.000000,0.000000,0.000000}%
\pgfsetfillcolor{currentfill}%
\pgfsetlinewidth{0.803000pt}%
\definecolor{currentstroke}{rgb}{0.000000,0.000000,0.000000}%
\pgfsetstrokecolor{currentstroke}%
\pgfsetdash{}{0pt}%
\pgfsys@defobject{currentmarker}{\pgfqpoint{0.000000in}{-0.048611in}}{\pgfqpoint{0.000000in}{0.000000in}}{%
\pgfpathmoveto{\pgfqpoint{0.000000in}{0.000000in}}%
\pgfpathlineto{\pgfqpoint{0.000000in}{-0.048611in}}%
\pgfusepath{stroke,fill}%
}%
\begin{pgfscope}%
\pgfsys@transformshift{0.569589in}{3.638271in}%
\pgfsys@useobject{currentmarker}{}%
\end{pgfscope}%
\end{pgfscope}%
\begin{pgfscope}%
\definecolor{textcolor}{rgb}{0.000000,0.000000,0.000000}%
\pgfsetstrokecolor{textcolor}%
\pgfsetfillcolor{textcolor}%
\pgftext[x=0.569589in,y=3.541049in,,top]{\color{textcolor}\rmfamily\fontsize{10.000000}{12.000000}\selectfont \(\displaystyle {0}\)}%
\end{pgfscope}%
\begin{pgfscope}%
\pgfsetbuttcap%
\pgfsetroundjoin%
\definecolor{currentfill}{rgb}{0.000000,0.000000,0.000000}%
\pgfsetfillcolor{currentfill}%
\pgfsetlinewidth{0.803000pt}%
\definecolor{currentstroke}{rgb}{0.000000,0.000000,0.000000}%
\pgfsetstrokecolor{currentstroke}%
\pgfsetdash{}{0pt}%
\pgfsys@defobject{currentmarker}{\pgfqpoint{0.000000in}{-0.048611in}}{\pgfqpoint{0.000000in}{0.000000in}}{%
\pgfpathmoveto{\pgfqpoint{0.000000in}{0.000000in}}%
\pgfpathlineto{\pgfqpoint{0.000000in}{-0.048611in}}%
\pgfusepath{stroke,fill}%
}%
\begin{pgfscope}%
\pgfsys@transformshift{1.268968in}{3.638271in}%
\pgfsys@useobject{currentmarker}{}%
\end{pgfscope}%
\end{pgfscope}%
\begin{pgfscope}%
\definecolor{textcolor}{rgb}{0.000000,0.000000,0.000000}%
\pgfsetstrokecolor{textcolor}%
\pgfsetfillcolor{textcolor}%
\pgftext[x=1.268968in,y=3.541049in,,top]{\color{textcolor}\rmfamily\fontsize{10.000000}{12.000000}\selectfont \(\displaystyle {2000}\)}%
\end{pgfscope}%
\begin{pgfscope}%
\pgfsetbuttcap%
\pgfsetroundjoin%
\definecolor{currentfill}{rgb}{0.000000,0.000000,0.000000}%
\pgfsetfillcolor{currentfill}%
\pgfsetlinewidth{0.803000pt}%
\definecolor{currentstroke}{rgb}{0.000000,0.000000,0.000000}%
\pgfsetstrokecolor{currentstroke}%
\pgfsetdash{}{0pt}%
\pgfsys@defobject{currentmarker}{\pgfqpoint{0.000000in}{-0.048611in}}{\pgfqpoint{0.000000in}{0.000000in}}{%
\pgfpathmoveto{\pgfqpoint{0.000000in}{0.000000in}}%
\pgfpathlineto{\pgfqpoint{0.000000in}{-0.048611in}}%
\pgfusepath{stroke,fill}%
}%
\begin{pgfscope}%
\pgfsys@transformshift{1.968347in}{3.638271in}%
\pgfsys@useobject{currentmarker}{}%
\end{pgfscope}%
\end{pgfscope}%
\begin{pgfscope}%
\definecolor{textcolor}{rgb}{0.000000,0.000000,0.000000}%
\pgfsetstrokecolor{textcolor}%
\pgfsetfillcolor{textcolor}%
\pgftext[x=1.968347in,y=3.541049in,,top]{\color{textcolor}\rmfamily\fontsize{10.000000}{12.000000}\selectfont \(\displaystyle {4000}\)}%
\end{pgfscope}%
\begin{pgfscope}%
\definecolor{textcolor}{rgb}{0.000000,0.000000,0.000000}%
\pgfsetstrokecolor{textcolor}%
\pgfsetfillcolor{textcolor}%
\pgftext[x=1.265471in,y=3.362037in,,top]{\color{textcolor}\rmfamily\fontsize{10.000000}{12.000000}\selectfont Czas [\(\displaystyle \mu s\)] }%
\end{pgfscope}%
\begin{pgfscope}%
\pgfsetbuttcap%
\pgfsetroundjoin%
\definecolor{currentfill}{rgb}{0.000000,0.000000,0.000000}%
\pgfsetfillcolor{currentfill}%
\pgfsetlinewidth{0.803000pt}%
\definecolor{currentstroke}{rgb}{0.000000,0.000000,0.000000}%
\pgfsetstrokecolor{currentstroke}%
\pgfsetdash{}{0pt}%
\pgfsys@defobject{currentmarker}{\pgfqpoint{-0.048611in}{0.000000in}}{\pgfqpoint{-0.000000in}{0.000000in}}{%
\pgfpathmoveto{\pgfqpoint{-0.000000in}{0.000000in}}%
\pgfpathlineto{\pgfqpoint{-0.048611in}{0.000000in}}%
\pgfusepath{stroke,fill}%
}%
\begin{pgfscope}%
\pgfsys@transformshift{0.500000in}{3.678872in}%
\pgfsys@useobject{currentmarker}{}%
\end{pgfscope}%
\end{pgfscope}%
\begin{pgfscope}%
\definecolor{textcolor}{rgb}{0.000000,0.000000,0.000000}%
\pgfsetstrokecolor{textcolor}%
\pgfsetfillcolor{textcolor}%
\pgftext[x=0.333333in, y=3.630647in, left, base]{\color{textcolor}\rmfamily\fontsize{10.000000}{12.000000}\selectfont \(\displaystyle {0}\)}%
\end{pgfscope}%
\begin{pgfscope}%
\pgfsetbuttcap%
\pgfsetroundjoin%
\definecolor{currentfill}{rgb}{0.000000,0.000000,0.000000}%
\pgfsetfillcolor{currentfill}%
\pgfsetlinewidth{0.803000pt}%
\definecolor{currentstroke}{rgb}{0.000000,0.000000,0.000000}%
\pgfsetstrokecolor{currentstroke}%
\pgfsetdash{}{0pt}%
\pgfsys@defobject{currentmarker}{\pgfqpoint{-0.048611in}{0.000000in}}{\pgfqpoint{-0.000000in}{0.000000in}}{%
\pgfpathmoveto{\pgfqpoint{-0.000000in}{0.000000in}}%
\pgfpathlineto{\pgfqpoint{-0.048611in}{0.000000in}}%
\pgfusepath{stroke,fill}%
}%
\begin{pgfscope}%
\pgfsys@transformshift{0.500000in}{4.010304in}%
\pgfsys@useobject{currentmarker}{}%
\end{pgfscope}%
\end{pgfscope}%
\begin{pgfscope}%
\definecolor{textcolor}{rgb}{0.000000,0.000000,0.000000}%
\pgfsetstrokecolor{textcolor}%
\pgfsetfillcolor{textcolor}%
\pgftext[x=0.263889in, y=3.962079in, left, base]{\color{textcolor}\rmfamily\fontsize{10.000000}{12.000000}\selectfont \(\displaystyle {20}\)}%
\end{pgfscope}%
\begin{pgfscope}%
\pgfsetbuttcap%
\pgfsetroundjoin%
\definecolor{currentfill}{rgb}{0.000000,0.000000,0.000000}%
\pgfsetfillcolor{currentfill}%
\pgfsetlinewidth{0.803000pt}%
\definecolor{currentstroke}{rgb}{0.000000,0.000000,0.000000}%
\pgfsetstrokecolor{currentstroke}%
\pgfsetdash{}{0pt}%
\pgfsys@defobject{currentmarker}{\pgfqpoint{-0.048611in}{0.000000in}}{\pgfqpoint{-0.000000in}{0.000000in}}{%
\pgfpathmoveto{\pgfqpoint{-0.000000in}{0.000000in}}%
\pgfpathlineto{\pgfqpoint{-0.048611in}{0.000000in}}%
\pgfusepath{stroke,fill}%
}%
\begin{pgfscope}%
\pgfsys@transformshift{0.500000in}{4.341737in}%
\pgfsys@useobject{currentmarker}{}%
\end{pgfscope}%
\end{pgfscope}%
\begin{pgfscope}%
\definecolor{textcolor}{rgb}{0.000000,0.000000,0.000000}%
\pgfsetstrokecolor{textcolor}%
\pgfsetfillcolor{textcolor}%
\pgftext[x=0.263889in, y=4.293511in, left, base]{\color{textcolor}\rmfamily\fontsize{10.000000}{12.000000}\selectfont \(\displaystyle {40}\)}%
\end{pgfscope}%
\begin{pgfscope}%
\definecolor{textcolor}{rgb}{0.000000,0.000000,0.000000}%
\pgfsetstrokecolor{textcolor}%
\pgfsetfillcolor{textcolor}%
\pgftext[x=0.208333in,y=4.084877in,,bottom,rotate=90.000000]{\color{textcolor}\rmfamily\fontsize{10.000000}{12.000000}\selectfont Napiecie [\(\displaystyle V\)] }%
\end{pgfscope}%
\begin{pgfscope}%
\pgfpathrectangle{\pgfqpoint{0.500000in}{3.638271in}}{\pgfqpoint{1.530941in}{0.893210in}}%
\pgfusepath{clip}%
\pgfsetrectcap%
\pgfsetroundjoin%
\pgfsetlinewidth{1.505625pt}%
\definecolor{currentstroke}{rgb}{0.121569,0.466667,0.705882}%
\pgfsetstrokecolor{currentstroke}%
\pgfsetdash{}{0pt}%
\pgfpathmoveto{\pgfqpoint{0.569589in}{4.474310in}}%
\pgfpathlineto{\pgfqpoint{0.611551in}{4.474310in}}%
\pgfpathlineto{\pgfqpoint{0.618545in}{4.490881in}}%
\pgfpathlineto{\pgfqpoint{0.625539in}{4.474310in}}%
\pgfpathlineto{\pgfqpoint{0.695477in}{4.474310in}}%
\pgfpathlineto{\pgfqpoint{0.702471in}{4.490881in}}%
\pgfpathlineto{\pgfqpoint{0.723452in}{4.490881in}}%
\pgfpathlineto{\pgfqpoint{0.730446in}{4.474310in}}%
\pgfpathlineto{\pgfqpoint{0.737440in}{4.474310in}}%
\pgfpathlineto{\pgfqpoint{0.744433in}{3.695444in}}%
\pgfpathlineto{\pgfqpoint{0.779402in}{3.695444in}}%
\pgfpathlineto{\pgfqpoint{0.786396in}{3.712015in}}%
\pgfpathlineto{\pgfqpoint{0.793390in}{3.678872in}}%
\pgfpathlineto{\pgfqpoint{0.800384in}{3.695444in}}%
\pgfpathlineto{\pgfqpoint{0.807378in}{3.678872in}}%
\pgfpathlineto{\pgfqpoint{0.814371in}{3.695444in}}%
\pgfpathlineto{\pgfqpoint{0.835353in}{3.695444in}}%
\pgfpathlineto{\pgfqpoint{0.842347in}{3.678872in}}%
\pgfpathlineto{\pgfqpoint{0.849340in}{3.678872in}}%
\pgfpathlineto{\pgfqpoint{0.863328in}{3.712015in}}%
\pgfpathlineto{\pgfqpoint{0.870322in}{3.695444in}}%
\pgfpathlineto{\pgfqpoint{0.898297in}{3.695444in}}%
\pgfpathlineto{\pgfqpoint{0.905291in}{3.678872in}}%
\pgfpathlineto{\pgfqpoint{0.912285in}{3.678872in}}%
\pgfpathlineto{\pgfqpoint{0.919278in}{4.474310in}}%
\pgfpathlineto{\pgfqpoint{0.961241in}{4.474310in}}%
\pgfpathlineto{\pgfqpoint{0.968235in}{4.490881in}}%
\pgfpathlineto{\pgfqpoint{0.975229in}{4.474310in}}%
\pgfpathlineto{\pgfqpoint{1.010198in}{4.474310in}}%
\pgfpathlineto{\pgfqpoint{1.017191in}{4.490881in}}%
\pgfpathlineto{\pgfqpoint{1.031179in}{4.490881in}}%
\pgfpathlineto{\pgfqpoint{1.038173in}{4.474310in}}%
\pgfpathlineto{\pgfqpoint{1.045167in}{4.474310in}}%
\pgfpathlineto{\pgfqpoint{1.052160in}{4.490881in}}%
\pgfpathlineto{\pgfqpoint{1.059154in}{4.474310in}}%
\pgfpathlineto{\pgfqpoint{1.066148in}{4.474310in}}%
\pgfpathlineto{\pgfqpoint{1.073142in}{4.490881in}}%
\pgfpathlineto{\pgfqpoint{1.087129in}{4.490881in}}%
\pgfpathlineto{\pgfqpoint{1.094123in}{3.695444in}}%
\pgfpathlineto{\pgfqpoint{1.115105in}{3.695444in}}%
\pgfpathlineto{\pgfqpoint{1.122098in}{3.712015in}}%
\pgfpathlineto{\pgfqpoint{1.136086in}{3.678872in}}%
\pgfpathlineto{\pgfqpoint{1.143080in}{3.678872in}}%
\pgfpathlineto{\pgfqpoint{1.150073in}{3.695444in}}%
\pgfpathlineto{\pgfqpoint{1.171055in}{3.695444in}}%
\pgfpathlineto{\pgfqpoint{1.178049in}{3.678872in}}%
\pgfpathlineto{\pgfqpoint{1.185042in}{3.695444in}}%
\pgfpathlineto{\pgfqpoint{1.192036in}{3.678872in}}%
\pgfpathlineto{\pgfqpoint{1.199030in}{3.678872in}}%
\pgfpathlineto{\pgfqpoint{1.206024in}{3.695444in}}%
\pgfpathlineto{\pgfqpoint{1.220011in}{3.695444in}}%
\pgfpathlineto{\pgfqpoint{1.227005in}{3.678872in}}%
\pgfpathlineto{\pgfqpoint{1.233999in}{3.678872in}}%
\pgfpathlineto{\pgfqpoint{1.240993in}{3.695444in}}%
\pgfpathlineto{\pgfqpoint{1.247987in}{3.695444in}}%
\pgfpathlineto{\pgfqpoint{1.254980in}{3.678872in}}%
\pgfpathlineto{\pgfqpoint{1.261974in}{3.695444in}}%
\pgfpathlineto{\pgfqpoint{1.268968in}{4.474310in}}%
\pgfpathlineto{\pgfqpoint{1.338906in}{4.474310in}}%
\pgfpathlineto{\pgfqpoint{1.345900in}{4.490881in}}%
\pgfpathlineto{\pgfqpoint{1.352893in}{4.474310in}}%
\pgfpathlineto{\pgfqpoint{1.359887in}{4.490881in}}%
\pgfpathlineto{\pgfqpoint{1.366881in}{4.490881in}}%
\pgfpathlineto{\pgfqpoint{1.373875in}{4.474310in}}%
\pgfpathlineto{\pgfqpoint{1.429825in}{4.474310in}}%
\pgfpathlineto{\pgfqpoint{1.436819in}{4.490881in}}%
\pgfpathlineto{\pgfqpoint{1.443813in}{3.712015in}}%
\pgfpathlineto{\pgfqpoint{1.450807in}{3.712015in}}%
\pgfpathlineto{\pgfqpoint{1.457800in}{3.695444in}}%
\pgfpathlineto{\pgfqpoint{1.464794in}{3.695444in}}%
\pgfpathlineto{\pgfqpoint{1.471788in}{3.712015in}}%
\pgfpathlineto{\pgfqpoint{1.478782in}{3.695444in}}%
\pgfpathlineto{\pgfqpoint{1.485776in}{3.695444in}}%
\pgfpathlineto{\pgfqpoint{1.492769in}{3.678872in}}%
\pgfpathlineto{\pgfqpoint{1.499763in}{3.695444in}}%
\pgfpathlineto{\pgfqpoint{1.520745in}{3.695444in}}%
\pgfpathlineto{\pgfqpoint{1.527738in}{3.678872in}}%
\pgfpathlineto{\pgfqpoint{1.534732in}{3.695444in}}%
\pgfpathlineto{\pgfqpoint{1.541726in}{3.678872in}}%
\pgfpathlineto{\pgfqpoint{1.548720in}{3.695444in}}%
\pgfpathlineto{\pgfqpoint{1.555713in}{3.678872in}}%
\pgfpathlineto{\pgfqpoint{1.562707in}{3.695444in}}%
\pgfpathlineto{\pgfqpoint{1.569701in}{3.695444in}}%
\pgfpathlineto{\pgfqpoint{1.576695in}{3.678872in}}%
\pgfpathlineto{\pgfqpoint{1.583689in}{3.678872in}}%
\pgfpathlineto{\pgfqpoint{1.590682in}{3.695444in}}%
\pgfpathlineto{\pgfqpoint{1.604670in}{3.695444in}}%
\pgfpathlineto{\pgfqpoint{1.611664in}{3.678872in}}%
\pgfpathlineto{\pgfqpoint{1.618658in}{4.474310in}}%
\pgfpathlineto{\pgfqpoint{1.709577in}{4.474310in}}%
\pgfpathlineto{\pgfqpoint{1.716571in}{4.490881in}}%
\pgfpathlineto{\pgfqpoint{1.723565in}{4.474310in}}%
\pgfpathlineto{\pgfqpoint{1.730558in}{4.474310in}}%
\pgfpathlineto{\pgfqpoint{1.737552in}{4.490881in}}%
\pgfpathlineto{\pgfqpoint{1.744546in}{4.490881in}}%
\pgfpathlineto{\pgfqpoint{1.751540in}{4.474310in}}%
\pgfpathlineto{\pgfqpoint{1.758533in}{4.474310in}}%
\pgfpathlineto{\pgfqpoint{1.765527in}{4.490881in}}%
\pgfpathlineto{\pgfqpoint{1.772521in}{4.474310in}}%
\pgfpathlineto{\pgfqpoint{1.779515in}{4.490881in}}%
\pgfpathlineto{\pgfqpoint{1.786509in}{4.490881in}}%
\pgfpathlineto{\pgfqpoint{1.793502in}{3.712015in}}%
\pgfpathlineto{\pgfqpoint{1.800496in}{3.712015in}}%
\pgfpathlineto{\pgfqpoint{1.807490in}{3.695444in}}%
\pgfpathlineto{\pgfqpoint{1.814484in}{3.695444in}}%
\pgfpathlineto{\pgfqpoint{1.821478in}{3.712015in}}%
\pgfpathlineto{\pgfqpoint{1.828471in}{3.695444in}}%
\pgfpathlineto{\pgfqpoint{1.835465in}{3.712015in}}%
\pgfpathlineto{\pgfqpoint{1.842459in}{3.712015in}}%
\pgfpathlineto{\pgfqpoint{1.849453in}{3.678872in}}%
\pgfpathlineto{\pgfqpoint{1.856447in}{3.695444in}}%
\pgfpathlineto{\pgfqpoint{1.912397in}{3.695444in}}%
\pgfpathlineto{\pgfqpoint{1.919391in}{3.678872in}}%
\pgfpathlineto{\pgfqpoint{1.926384in}{3.678872in}}%
\pgfpathlineto{\pgfqpoint{1.933378in}{3.695444in}}%
\pgfpathlineto{\pgfqpoint{1.940372in}{3.678872in}}%
\pgfpathlineto{\pgfqpoint{1.961353in}{3.678872in}}%
\pgfpathlineto{\pgfqpoint{1.961353in}{3.678872in}}%
\pgfusepath{stroke}%
\end{pgfscope}%
\begin{pgfscope}%
\pgfpathrectangle{\pgfqpoint{0.500000in}{3.638271in}}{\pgfqpoint{1.530941in}{0.893210in}}%
\pgfusepath{clip}%
\pgfsetrectcap%
\pgfsetroundjoin%
\pgfsetlinewidth{1.505625pt}%
\definecolor{currentstroke}{rgb}{1.000000,0.498039,0.054902}%
\pgfsetstrokecolor{currentstroke}%
\pgfsetdash{}{0pt}%
\pgfpathmoveto{\pgfqpoint{0.000000in}{0.000000in}}%
\pgfusepath{stroke}%
\end{pgfscope}%
\begin{pgfscope}%
\pgfpathrectangle{\pgfqpoint{0.500000in}{3.638271in}}{\pgfqpoint{1.530941in}{0.893210in}}%
\pgfusepath{clip}%
\pgfsetrectcap%
\pgfsetroundjoin%
\pgfsetlinewidth{1.505625pt}%
\definecolor{currentstroke}{rgb}{0.172549,0.627451,0.172549}%
\pgfsetstrokecolor{currentstroke}%
\pgfsetdash{}{0pt}%
\pgfpathmoveto{\pgfqpoint{0.569589in}{3.993733in}}%
\pgfpathlineto{\pgfqpoint{0.576582in}{4.010304in}}%
\pgfpathlineto{\pgfqpoint{0.583576in}{4.010304in}}%
\pgfpathlineto{\pgfqpoint{0.590570in}{4.026876in}}%
\pgfpathlineto{\pgfqpoint{0.604558in}{4.026876in}}%
\pgfpathlineto{\pgfqpoint{0.611551in}{4.043448in}}%
\pgfpathlineto{\pgfqpoint{0.618545in}{4.043448in}}%
\pgfpathlineto{\pgfqpoint{0.625539in}{4.076591in}}%
\pgfpathlineto{\pgfqpoint{0.632533in}{4.076591in}}%
\pgfpathlineto{\pgfqpoint{0.639527in}{4.093162in}}%
\pgfpathlineto{\pgfqpoint{0.660508in}{4.093162in}}%
\pgfpathlineto{\pgfqpoint{0.674496in}{4.126306in}}%
\pgfpathlineto{\pgfqpoint{0.688483in}{4.126306in}}%
\pgfpathlineto{\pgfqpoint{0.695477in}{4.142877in}}%
\pgfpathlineto{\pgfqpoint{0.702471in}{4.142877in}}%
\pgfpathlineto{\pgfqpoint{0.709465in}{4.159449in}}%
\pgfpathlineto{\pgfqpoint{0.730446in}{4.159449in}}%
\pgfpathlineto{\pgfqpoint{0.744433in}{4.192592in}}%
\pgfpathlineto{\pgfqpoint{0.751427in}{4.159449in}}%
\pgfpathlineto{\pgfqpoint{0.758421in}{4.176021in}}%
\pgfpathlineto{\pgfqpoint{0.765415in}{4.159449in}}%
\pgfpathlineto{\pgfqpoint{0.772409in}{4.126306in}}%
\pgfpathlineto{\pgfqpoint{0.779402in}{4.142877in}}%
\pgfpathlineto{\pgfqpoint{0.800384in}{4.093162in}}%
\pgfpathlineto{\pgfqpoint{0.828359in}{4.093162in}}%
\pgfpathlineto{\pgfqpoint{0.842347in}{4.060019in}}%
\pgfpathlineto{\pgfqpoint{0.849340in}{4.060019in}}%
\pgfpathlineto{\pgfqpoint{0.863328in}{4.026876in}}%
\pgfpathlineto{\pgfqpoint{0.884309in}{4.026876in}}%
\pgfpathlineto{\pgfqpoint{0.891303in}{4.010304in}}%
\pgfpathlineto{\pgfqpoint{0.898297in}{4.026876in}}%
\pgfpathlineto{\pgfqpoint{0.912285in}{3.993733in}}%
\pgfpathlineto{\pgfqpoint{0.919278in}{3.993733in}}%
\pgfpathlineto{\pgfqpoint{0.933266in}{4.026876in}}%
\pgfpathlineto{\pgfqpoint{0.947253in}{4.026876in}}%
\pgfpathlineto{\pgfqpoint{0.954247in}{4.043448in}}%
\pgfpathlineto{\pgfqpoint{0.961241in}{4.043448in}}%
\pgfpathlineto{\pgfqpoint{0.968235in}{4.060019in}}%
\pgfpathlineto{\pgfqpoint{0.975229in}{4.060019in}}%
\pgfpathlineto{\pgfqpoint{0.982222in}{4.076591in}}%
\pgfpathlineto{\pgfqpoint{0.989216in}{4.076591in}}%
\pgfpathlineto{\pgfqpoint{0.996210in}{4.093162in}}%
\pgfpathlineto{\pgfqpoint{1.010198in}{4.093162in}}%
\pgfpathlineto{\pgfqpoint{1.017191in}{4.109734in}}%
\pgfpathlineto{\pgfqpoint{1.024185in}{4.109734in}}%
\pgfpathlineto{\pgfqpoint{1.038173in}{4.142877in}}%
\pgfpathlineto{\pgfqpoint{1.045167in}{4.142877in}}%
\pgfpathlineto{\pgfqpoint{1.052160in}{4.159449in}}%
\pgfpathlineto{\pgfqpoint{1.080136in}{4.159449in}}%
\pgfpathlineto{\pgfqpoint{1.087129in}{4.176021in}}%
\pgfpathlineto{\pgfqpoint{1.094123in}{4.176021in}}%
\pgfpathlineto{\pgfqpoint{1.101117in}{4.159449in}}%
\pgfpathlineto{\pgfqpoint{1.115105in}{4.159449in}}%
\pgfpathlineto{\pgfqpoint{1.122098in}{4.142877in}}%
\pgfpathlineto{\pgfqpoint{1.129092in}{4.142877in}}%
\pgfpathlineto{\pgfqpoint{1.150073in}{4.093162in}}%
\pgfpathlineto{\pgfqpoint{1.171055in}{4.093162in}}%
\pgfpathlineto{\pgfqpoint{1.178049in}{4.076591in}}%
\pgfpathlineto{\pgfqpoint{1.185042in}{4.076591in}}%
\pgfpathlineto{\pgfqpoint{1.192036in}{4.060019in}}%
\pgfpathlineto{\pgfqpoint{1.199030in}{4.060019in}}%
\pgfpathlineto{\pgfqpoint{1.206024in}{4.043448in}}%
\pgfpathlineto{\pgfqpoint{1.213018in}{4.043448in}}%
\pgfpathlineto{\pgfqpoint{1.220011in}{4.026876in}}%
\pgfpathlineto{\pgfqpoint{1.247987in}{4.026876in}}%
\pgfpathlineto{\pgfqpoint{1.261974in}{3.993733in}}%
\pgfpathlineto{\pgfqpoint{1.268968in}{4.010304in}}%
\pgfpathlineto{\pgfqpoint{1.275962in}{3.993733in}}%
\pgfpathlineto{\pgfqpoint{1.289949in}{4.026876in}}%
\pgfpathlineto{\pgfqpoint{1.310931in}{4.026876in}}%
\pgfpathlineto{\pgfqpoint{1.324918in}{4.060019in}}%
\pgfpathlineto{\pgfqpoint{1.331912in}{4.060019in}}%
\pgfpathlineto{\pgfqpoint{1.338906in}{4.093162in}}%
\pgfpathlineto{\pgfqpoint{1.359887in}{4.093162in}}%
\pgfpathlineto{\pgfqpoint{1.366881in}{4.126306in}}%
\pgfpathlineto{\pgfqpoint{1.380869in}{4.126306in}}%
\pgfpathlineto{\pgfqpoint{1.387862in}{4.142877in}}%
\pgfpathlineto{\pgfqpoint{1.394856in}{4.126306in}}%
\pgfpathlineto{\pgfqpoint{1.408844in}{4.159449in}}%
\pgfpathlineto{\pgfqpoint{1.436819in}{4.159449in}}%
\pgfpathlineto{\pgfqpoint{1.443813in}{4.176021in}}%
\pgfpathlineto{\pgfqpoint{1.450807in}{4.159449in}}%
\pgfpathlineto{\pgfqpoint{1.464794in}{4.159449in}}%
\pgfpathlineto{\pgfqpoint{1.471788in}{4.142877in}}%
\pgfpathlineto{\pgfqpoint{1.485776in}{4.142877in}}%
\pgfpathlineto{\pgfqpoint{1.492769in}{4.109734in}}%
\pgfpathlineto{\pgfqpoint{1.499763in}{4.109734in}}%
\pgfpathlineto{\pgfqpoint{1.506757in}{4.093162in}}%
\pgfpathlineto{\pgfqpoint{1.520745in}{4.093162in}}%
\pgfpathlineto{\pgfqpoint{1.534732in}{4.060019in}}%
\pgfpathlineto{\pgfqpoint{1.541726in}{4.060019in}}%
\pgfpathlineto{\pgfqpoint{1.555713in}{4.026876in}}%
\pgfpathlineto{\pgfqpoint{1.583689in}{4.026876in}}%
\pgfpathlineto{\pgfqpoint{1.590682in}{4.010304in}}%
\pgfpathlineto{\pgfqpoint{1.597676in}{4.010304in}}%
\pgfpathlineto{\pgfqpoint{1.604670in}{3.993733in}}%
\pgfpathlineto{\pgfqpoint{1.618658in}{3.993733in}}%
\pgfpathlineto{\pgfqpoint{1.632645in}{4.026876in}}%
\pgfpathlineto{\pgfqpoint{1.653627in}{4.026876in}}%
\pgfpathlineto{\pgfqpoint{1.667614in}{4.060019in}}%
\pgfpathlineto{\pgfqpoint{1.674608in}{4.060019in}}%
\pgfpathlineto{\pgfqpoint{1.688596in}{4.093162in}}%
\pgfpathlineto{\pgfqpoint{1.709577in}{4.093162in}}%
\pgfpathlineto{\pgfqpoint{1.716571in}{4.109734in}}%
\pgfpathlineto{\pgfqpoint{1.723565in}{4.109734in}}%
\pgfpathlineto{\pgfqpoint{1.730558in}{4.126306in}}%
\pgfpathlineto{\pgfqpoint{1.737552in}{4.126306in}}%
\pgfpathlineto{\pgfqpoint{1.751540in}{4.159449in}}%
\pgfpathlineto{\pgfqpoint{1.779515in}{4.159449in}}%
\pgfpathlineto{\pgfqpoint{1.793502in}{4.192592in}}%
\pgfpathlineto{\pgfqpoint{1.807490in}{4.159449in}}%
\pgfpathlineto{\pgfqpoint{1.814484in}{4.159449in}}%
\pgfpathlineto{\pgfqpoint{1.821478in}{4.142877in}}%
\pgfpathlineto{\pgfqpoint{1.828471in}{4.142877in}}%
\pgfpathlineto{\pgfqpoint{1.835465in}{4.126306in}}%
\pgfpathlineto{\pgfqpoint{1.842459in}{4.126306in}}%
\pgfpathlineto{\pgfqpoint{1.849453in}{4.093162in}}%
\pgfpathlineto{\pgfqpoint{1.870434in}{4.093162in}}%
\pgfpathlineto{\pgfqpoint{1.884422in}{4.060019in}}%
\pgfpathlineto{\pgfqpoint{1.891416in}{4.076591in}}%
\pgfpathlineto{\pgfqpoint{1.905403in}{4.043448in}}%
\pgfpathlineto{\pgfqpoint{1.912397in}{4.043448in}}%
\pgfpathlineto{\pgfqpoint{1.919391in}{4.026876in}}%
\pgfpathlineto{\pgfqpoint{1.940372in}{4.026876in}}%
\pgfpathlineto{\pgfqpoint{1.947366in}{4.010304in}}%
\pgfpathlineto{\pgfqpoint{1.961353in}{4.010304in}}%
\pgfpathlineto{\pgfqpoint{1.961353in}{4.010304in}}%
\pgfusepath{stroke}%
\end{pgfscope}%
\begin{pgfscope}%
\pgfpathrectangle{\pgfqpoint{0.500000in}{3.638271in}}{\pgfqpoint{1.530941in}{0.893210in}}%
\pgfusepath{clip}%
\pgfsetrectcap%
\pgfsetroundjoin%
\pgfsetlinewidth{1.505625pt}%
\definecolor{currentstroke}{rgb}{0.839216,0.152941,0.156863}%
\pgfsetstrokecolor{currentstroke}%
\pgfsetdash{}{0pt}%
\pgfpathmoveto{\pgfqpoint{0.000000in}{0.000000in}}%
\pgfusepath{stroke}%
\end{pgfscope}%
\begin{pgfscope}%
\pgfsetrectcap%
\pgfsetmiterjoin%
\pgfsetlinewidth{0.803000pt}%
\definecolor{currentstroke}{rgb}{0.000000,0.000000,0.000000}%
\pgfsetstrokecolor{currentstroke}%
\pgfsetdash{}{0pt}%
\pgfpathmoveto{\pgfqpoint{0.500000in}{3.638271in}}%
\pgfpathlineto{\pgfqpoint{0.500000in}{4.531482in}}%
\pgfusepath{stroke}%
\end{pgfscope}%
\begin{pgfscope}%
\pgfsetrectcap%
\pgfsetmiterjoin%
\pgfsetlinewidth{0.803000pt}%
\definecolor{currentstroke}{rgb}{0.000000,0.000000,0.000000}%
\pgfsetstrokecolor{currentstroke}%
\pgfsetdash{}{0pt}%
\pgfpathmoveto{\pgfqpoint{2.030942in}{3.638271in}}%
\pgfpathlineto{\pgfqpoint{2.030942in}{4.531482in}}%
\pgfusepath{stroke}%
\end{pgfscope}%
\begin{pgfscope}%
\pgfsetrectcap%
\pgfsetmiterjoin%
\pgfsetlinewidth{0.803000pt}%
\definecolor{currentstroke}{rgb}{0.000000,0.000000,0.000000}%
\pgfsetstrokecolor{currentstroke}%
\pgfsetdash{}{0pt}%
\pgfpathmoveto{\pgfqpoint{0.500000in}{3.638271in}}%
\pgfpathlineto{\pgfqpoint{2.030942in}{3.638271in}}%
\pgfusepath{stroke}%
\end{pgfscope}%
\begin{pgfscope}%
\pgfsetrectcap%
\pgfsetmiterjoin%
\pgfsetlinewidth{0.803000pt}%
\definecolor{currentstroke}{rgb}{0.000000,0.000000,0.000000}%
\pgfsetstrokecolor{currentstroke}%
\pgfsetdash{}{0pt}%
\pgfpathmoveto{\pgfqpoint{0.500000in}{4.531482in}}%
\pgfpathlineto{\pgfqpoint{2.030942in}{4.531482in}}%
\pgfusepath{stroke}%
\end{pgfscope}%
\begin{pgfscope}%
\definecolor{textcolor}{rgb}{0.000000,0.000000,0.000000}%
\pgfsetstrokecolor{textcolor}%
\pgfsetfillcolor{textcolor}%
\pgftext[x=1.265471in,y=4.614815in,,base]{\color{textcolor}\rmfamily\fontsize{12.000000}{14.400000}\selectfont 0}%
\end{pgfscope}%
\begin{pgfscope}%
\pgfsetbuttcap%
\pgfsetmiterjoin%
\definecolor{currentfill}{rgb}{1.000000,1.000000,1.000000}%
\pgfsetfillcolor{currentfill}%
\pgfsetlinewidth{0.000000pt}%
\definecolor{currentstroke}{rgb}{0.000000,0.000000,0.000000}%
\pgfsetstrokecolor{currentstroke}%
\pgfsetstrokeopacity{0.000000}%
\pgfsetdash{}{0pt}%
\pgfpathmoveto{\pgfqpoint{2.610186in}{3.638271in}}%
\pgfpathlineto{\pgfqpoint{4.141127in}{3.638271in}}%
\pgfpathlineto{\pgfqpoint{4.141127in}{4.531482in}}%
\pgfpathlineto{\pgfqpoint{2.610186in}{4.531482in}}%
\pgfpathclose%
\pgfusepath{fill}%
\end{pgfscope}%
\begin{pgfscope}%
\pgfsetbuttcap%
\pgfsetroundjoin%
\definecolor{currentfill}{rgb}{0.000000,0.000000,0.000000}%
\pgfsetfillcolor{currentfill}%
\pgfsetlinewidth{0.803000pt}%
\definecolor{currentstroke}{rgb}{0.000000,0.000000,0.000000}%
\pgfsetstrokecolor{currentstroke}%
\pgfsetdash{}{0pt}%
\pgfsys@defobject{currentmarker}{\pgfqpoint{0.000000in}{-0.048611in}}{\pgfqpoint{0.000000in}{0.000000in}}{%
\pgfpathmoveto{\pgfqpoint{0.000000in}{0.000000in}}%
\pgfpathlineto{\pgfqpoint{0.000000in}{-0.048611in}}%
\pgfusepath{stroke,fill}%
}%
\begin{pgfscope}%
\pgfsys@transformshift{2.679774in}{3.638271in}%
\pgfsys@useobject{currentmarker}{}%
\end{pgfscope}%
\end{pgfscope}%
\begin{pgfscope}%
\definecolor{textcolor}{rgb}{0.000000,0.000000,0.000000}%
\pgfsetstrokecolor{textcolor}%
\pgfsetfillcolor{textcolor}%
\pgftext[x=2.679774in,y=3.541049in,,top]{\color{textcolor}\rmfamily\fontsize{10.000000}{12.000000}\selectfont \(\displaystyle {0}\)}%
\end{pgfscope}%
\begin{pgfscope}%
\pgfsetbuttcap%
\pgfsetroundjoin%
\definecolor{currentfill}{rgb}{0.000000,0.000000,0.000000}%
\pgfsetfillcolor{currentfill}%
\pgfsetlinewidth{0.803000pt}%
\definecolor{currentstroke}{rgb}{0.000000,0.000000,0.000000}%
\pgfsetstrokecolor{currentstroke}%
\pgfsetdash{}{0pt}%
\pgfsys@defobject{currentmarker}{\pgfqpoint{0.000000in}{-0.048611in}}{\pgfqpoint{0.000000in}{0.000000in}}{%
\pgfpathmoveto{\pgfqpoint{0.000000in}{0.000000in}}%
\pgfpathlineto{\pgfqpoint{0.000000in}{-0.048611in}}%
\pgfusepath{stroke,fill}%
}%
\begin{pgfscope}%
\pgfsys@transformshift{3.379153in}{3.638271in}%
\pgfsys@useobject{currentmarker}{}%
\end{pgfscope}%
\end{pgfscope}%
\begin{pgfscope}%
\definecolor{textcolor}{rgb}{0.000000,0.000000,0.000000}%
\pgfsetstrokecolor{textcolor}%
\pgfsetfillcolor{textcolor}%
\pgftext[x=3.379153in,y=3.541049in,,top]{\color{textcolor}\rmfamily\fontsize{10.000000}{12.000000}\selectfont \(\displaystyle {2000}\)}%
\end{pgfscope}%
\begin{pgfscope}%
\pgfsetbuttcap%
\pgfsetroundjoin%
\definecolor{currentfill}{rgb}{0.000000,0.000000,0.000000}%
\pgfsetfillcolor{currentfill}%
\pgfsetlinewidth{0.803000pt}%
\definecolor{currentstroke}{rgb}{0.000000,0.000000,0.000000}%
\pgfsetstrokecolor{currentstroke}%
\pgfsetdash{}{0pt}%
\pgfsys@defobject{currentmarker}{\pgfqpoint{0.000000in}{-0.048611in}}{\pgfqpoint{0.000000in}{0.000000in}}{%
\pgfpathmoveto{\pgfqpoint{0.000000in}{0.000000in}}%
\pgfpathlineto{\pgfqpoint{0.000000in}{-0.048611in}}%
\pgfusepath{stroke,fill}%
}%
\begin{pgfscope}%
\pgfsys@transformshift{4.078532in}{3.638271in}%
\pgfsys@useobject{currentmarker}{}%
\end{pgfscope}%
\end{pgfscope}%
\begin{pgfscope}%
\definecolor{textcolor}{rgb}{0.000000,0.000000,0.000000}%
\pgfsetstrokecolor{textcolor}%
\pgfsetfillcolor{textcolor}%
\pgftext[x=4.078532in,y=3.541049in,,top]{\color{textcolor}\rmfamily\fontsize{10.000000}{12.000000}\selectfont \(\displaystyle {4000}\)}%
\end{pgfscope}%
\begin{pgfscope}%
\definecolor{textcolor}{rgb}{0.000000,0.000000,0.000000}%
\pgfsetstrokecolor{textcolor}%
\pgfsetfillcolor{textcolor}%
\pgftext[x=3.375656in,y=3.362037in,,top]{\color{textcolor}\rmfamily\fontsize{10.000000}{12.000000}\selectfont Czas [\(\displaystyle \mu s\)] }%
\end{pgfscope}%
\begin{pgfscope}%
\pgfsetbuttcap%
\pgfsetroundjoin%
\definecolor{currentfill}{rgb}{0.000000,0.000000,0.000000}%
\pgfsetfillcolor{currentfill}%
\pgfsetlinewidth{0.803000pt}%
\definecolor{currentstroke}{rgb}{0.000000,0.000000,0.000000}%
\pgfsetstrokecolor{currentstroke}%
\pgfsetdash{}{0pt}%
\pgfsys@defobject{currentmarker}{\pgfqpoint{-0.048611in}{0.000000in}}{\pgfqpoint{-0.000000in}{0.000000in}}{%
\pgfpathmoveto{\pgfqpoint{-0.000000in}{0.000000in}}%
\pgfpathlineto{\pgfqpoint{-0.048611in}{0.000000in}}%
\pgfusepath{stroke,fill}%
}%
\begin{pgfscope}%
\pgfsys@transformshift{2.610186in}{3.678872in}%
\pgfsys@useobject{currentmarker}{}%
\end{pgfscope}%
\end{pgfscope}%
\begin{pgfscope}%
\definecolor{textcolor}{rgb}{0.000000,0.000000,0.000000}%
\pgfsetstrokecolor{textcolor}%
\pgfsetfillcolor{textcolor}%
\pgftext[x=2.443519in, y=3.630647in, left, base]{\color{textcolor}\rmfamily\fontsize{10.000000}{12.000000}\selectfont \(\displaystyle {0}\)}%
\end{pgfscope}%
\begin{pgfscope}%
\pgfsetbuttcap%
\pgfsetroundjoin%
\definecolor{currentfill}{rgb}{0.000000,0.000000,0.000000}%
\pgfsetfillcolor{currentfill}%
\pgfsetlinewidth{0.803000pt}%
\definecolor{currentstroke}{rgb}{0.000000,0.000000,0.000000}%
\pgfsetstrokecolor{currentstroke}%
\pgfsetdash{}{0pt}%
\pgfsys@defobject{currentmarker}{\pgfqpoint{-0.048611in}{0.000000in}}{\pgfqpoint{-0.000000in}{0.000000in}}{%
\pgfpathmoveto{\pgfqpoint{-0.000000in}{0.000000in}}%
\pgfpathlineto{\pgfqpoint{-0.048611in}{0.000000in}}%
\pgfusepath{stroke,fill}%
}%
\begin{pgfscope}%
\pgfsys@transformshift{2.610186in}{4.010304in}%
\pgfsys@useobject{currentmarker}{}%
\end{pgfscope}%
\end{pgfscope}%
\begin{pgfscope}%
\definecolor{textcolor}{rgb}{0.000000,0.000000,0.000000}%
\pgfsetstrokecolor{textcolor}%
\pgfsetfillcolor{textcolor}%
\pgftext[x=2.374074in, y=3.962079in, left, base]{\color{textcolor}\rmfamily\fontsize{10.000000}{12.000000}\selectfont \(\displaystyle {20}\)}%
\end{pgfscope}%
\begin{pgfscope}%
\pgfsetbuttcap%
\pgfsetroundjoin%
\definecolor{currentfill}{rgb}{0.000000,0.000000,0.000000}%
\pgfsetfillcolor{currentfill}%
\pgfsetlinewidth{0.803000pt}%
\definecolor{currentstroke}{rgb}{0.000000,0.000000,0.000000}%
\pgfsetstrokecolor{currentstroke}%
\pgfsetdash{}{0pt}%
\pgfsys@defobject{currentmarker}{\pgfqpoint{-0.048611in}{0.000000in}}{\pgfqpoint{-0.000000in}{0.000000in}}{%
\pgfpathmoveto{\pgfqpoint{-0.000000in}{0.000000in}}%
\pgfpathlineto{\pgfqpoint{-0.048611in}{0.000000in}}%
\pgfusepath{stroke,fill}%
}%
\begin{pgfscope}%
\pgfsys@transformshift{2.610186in}{4.341737in}%
\pgfsys@useobject{currentmarker}{}%
\end{pgfscope}%
\end{pgfscope}%
\begin{pgfscope}%
\definecolor{textcolor}{rgb}{0.000000,0.000000,0.000000}%
\pgfsetstrokecolor{textcolor}%
\pgfsetfillcolor{textcolor}%
\pgftext[x=2.374074in, y=4.293511in, left, base]{\color{textcolor}\rmfamily\fontsize{10.000000}{12.000000}\selectfont \(\displaystyle {40}\)}%
\end{pgfscope}%
\begin{pgfscope}%
\definecolor{textcolor}{rgb}{0.000000,0.000000,0.000000}%
\pgfsetstrokecolor{textcolor}%
\pgfsetfillcolor{textcolor}%
\pgftext[x=2.318519in,y=4.084877in,,bottom,rotate=90.000000]{\color{textcolor}\rmfamily\fontsize{10.000000}{12.000000}\selectfont Napiecie [\(\displaystyle V\)] }%
\end{pgfscope}%
\begin{pgfscope}%
\pgfpathrectangle{\pgfqpoint{2.610186in}{3.638271in}}{\pgfqpoint{1.530941in}{0.893210in}}%
\pgfusepath{clip}%
\pgfsetrectcap%
\pgfsetroundjoin%
\pgfsetlinewidth{1.505625pt}%
\definecolor{currentstroke}{rgb}{0.121569,0.466667,0.705882}%
\pgfsetstrokecolor{currentstroke}%
\pgfsetdash{}{0pt}%
\pgfpathmoveto{\pgfqpoint{2.679774in}{3.678872in}}%
\pgfpathlineto{\pgfqpoint{2.728730in}{3.678872in}}%
\pgfpathlineto{\pgfqpoint{2.735724in}{4.474310in}}%
\pgfpathlineto{\pgfqpoint{2.742718in}{4.457738in}}%
\pgfpathlineto{\pgfqpoint{2.749712in}{4.474310in}}%
\pgfpathlineto{\pgfqpoint{2.805662in}{4.474310in}}%
\pgfpathlineto{\pgfqpoint{2.812656in}{4.490881in}}%
\pgfpathlineto{\pgfqpoint{2.819650in}{4.474310in}}%
\pgfpathlineto{\pgfqpoint{2.826643in}{4.490881in}}%
\pgfpathlineto{\pgfqpoint{2.833637in}{4.474310in}}%
\pgfpathlineto{\pgfqpoint{2.847625in}{4.474310in}}%
\pgfpathlineto{\pgfqpoint{2.854619in}{4.490881in}}%
\pgfpathlineto{\pgfqpoint{2.861612in}{4.474310in}}%
\pgfpathlineto{\pgfqpoint{2.868606in}{4.490881in}}%
\pgfpathlineto{\pgfqpoint{2.896581in}{4.490881in}}%
\pgfpathlineto{\pgfqpoint{2.903575in}{4.474310in}}%
\pgfpathlineto{\pgfqpoint{2.910569in}{3.695444in}}%
\pgfpathlineto{\pgfqpoint{2.917563in}{3.712015in}}%
\pgfpathlineto{\pgfqpoint{2.924557in}{3.712015in}}%
\pgfpathlineto{\pgfqpoint{2.931550in}{3.695444in}}%
\pgfpathlineto{\pgfqpoint{2.938544in}{3.695444in}}%
\pgfpathlineto{\pgfqpoint{2.945538in}{3.678872in}}%
\pgfpathlineto{\pgfqpoint{2.952532in}{3.695444in}}%
\pgfpathlineto{\pgfqpoint{2.966519in}{3.695444in}}%
\pgfpathlineto{\pgfqpoint{2.973513in}{3.678872in}}%
\pgfpathlineto{\pgfqpoint{2.994495in}{3.678872in}}%
\pgfpathlineto{\pgfqpoint{3.001488in}{3.695444in}}%
\pgfpathlineto{\pgfqpoint{3.008482in}{3.695444in}}%
\pgfpathlineto{\pgfqpoint{3.015476in}{3.678872in}}%
\pgfpathlineto{\pgfqpoint{3.043451in}{3.678872in}}%
\pgfpathlineto{\pgfqpoint{3.050445in}{3.695444in}}%
\pgfpathlineto{\pgfqpoint{3.057439in}{3.678872in}}%
\pgfpathlineto{\pgfqpoint{3.078420in}{3.678872in}}%
\pgfpathlineto{\pgfqpoint{3.085414in}{4.441166in}}%
\pgfpathlineto{\pgfqpoint{3.092408in}{4.474310in}}%
\pgfpathlineto{\pgfqpoint{3.099401in}{4.457738in}}%
\pgfpathlineto{\pgfqpoint{3.106395in}{4.474310in}}%
\pgfpathlineto{\pgfqpoint{3.134370in}{4.474310in}}%
\pgfpathlineto{\pgfqpoint{3.141364in}{4.490881in}}%
\pgfpathlineto{\pgfqpoint{3.148358in}{4.474310in}}%
\pgfpathlineto{\pgfqpoint{3.155352in}{4.474310in}}%
\pgfpathlineto{\pgfqpoint{3.162346in}{4.490881in}}%
\pgfpathlineto{\pgfqpoint{3.169339in}{4.474310in}}%
\pgfpathlineto{\pgfqpoint{3.204308in}{4.474310in}}%
\pgfpathlineto{\pgfqpoint{3.211302in}{4.490881in}}%
\pgfpathlineto{\pgfqpoint{3.218296in}{4.474310in}}%
\pgfpathlineto{\pgfqpoint{3.225290in}{4.474310in}}%
\pgfpathlineto{\pgfqpoint{3.232283in}{4.490881in}}%
\pgfpathlineto{\pgfqpoint{3.253265in}{4.490881in}}%
\pgfpathlineto{\pgfqpoint{3.260259in}{3.712015in}}%
\pgfpathlineto{\pgfqpoint{3.274246in}{3.712015in}}%
\pgfpathlineto{\pgfqpoint{3.281240in}{3.695444in}}%
\pgfpathlineto{\pgfqpoint{3.302221in}{3.695444in}}%
\pgfpathlineto{\pgfqpoint{3.309215in}{3.712015in}}%
\pgfpathlineto{\pgfqpoint{3.316209in}{3.695444in}}%
\pgfpathlineto{\pgfqpoint{3.330197in}{3.695444in}}%
\pgfpathlineto{\pgfqpoint{3.337190in}{3.678872in}}%
\pgfpathlineto{\pgfqpoint{3.344184in}{3.695444in}}%
\pgfpathlineto{\pgfqpoint{3.351178in}{3.678872in}}%
\pgfpathlineto{\pgfqpoint{3.372159in}{3.678872in}}%
\pgfpathlineto{\pgfqpoint{3.379153in}{3.695444in}}%
\pgfpathlineto{\pgfqpoint{3.386147in}{3.695444in}}%
\pgfpathlineto{\pgfqpoint{3.393141in}{3.678872in}}%
\pgfpathlineto{\pgfqpoint{3.414122in}{3.678872in}}%
\pgfpathlineto{\pgfqpoint{3.421116in}{3.695444in}}%
\pgfpathlineto{\pgfqpoint{3.428110in}{3.695444in}}%
\pgfpathlineto{\pgfqpoint{3.435103in}{4.457738in}}%
\pgfpathlineto{\pgfqpoint{3.442097in}{4.457738in}}%
\pgfpathlineto{\pgfqpoint{3.449091in}{4.474310in}}%
\pgfpathlineto{\pgfqpoint{3.477066in}{4.474310in}}%
\pgfpathlineto{\pgfqpoint{3.484060in}{4.490881in}}%
\pgfpathlineto{\pgfqpoint{3.491054in}{4.474310in}}%
\pgfpathlineto{\pgfqpoint{3.505041in}{4.474310in}}%
\pgfpathlineto{\pgfqpoint{3.512035in}{4.490881in}}%
\pgfpathlineto{\pgfqpoint{3.519029in}{4.474310in}}%
\pgfpathlineto{\pgfqpoint{3.540010in}{4.474310in}}%
\pgfpathlineto{\pgfqpoint{3.547004in}{4.490881in}}%
\pgfpathlineto{\pgfqpoint{3.553998in}{4.474310in}}%
\pgfpathlineto{\pgfqpoint{3.560992in}{4.490881in}}%
\pgfpathlineto{\pgfqpoint{3.567986in}{4.474310in}}%
\pgfpathlineto{\pgfqpoint{3.574979in}{4.490881in}}%
\pgfpathlineto{\pgfqpoint{3.581973in}{4.490881in}}%
\pgfpathlineto{\pgfqpoint{3.588967in}{4.474310in}}%
\pgfpathlineto{\pgfqpoint{3.595961in}{4.490881in}}%
\pgfpathlineto{\pgfqpoint{3.602955in}{4.474310in}}%
\pgfpathlineto{\pgfqpoint{3.609948in}{3.712015in}}%
\pgfpathlineto{\pgfqpoint{3.630930in}{3.712015in}}%
\pgfpathlineto{\pgfqpoint{3.637923in}{3.695444in}}%
\pgfpathlineto{\pgfqpoint{3.644917in}{3.695444in}}%
\pgfpathlineto{\pgfqpoint{3.651911in}{3.678872in}}%
\pgfpathlineto{\pgfqpoint{3.658905in}{3.695444in}}%
\pgfpathlineto{\pgfqpoint{3.665899in}{3.678872in}}%
\pgfpathlineto{\pgfqpoint{3.672892in}{3.695444in}}%
\pgfpathlineto{\pgfqpoint{3.679886in}{3.695444in}}%
\pgfpathlineto{\pgfqpoint{3.686880in}{3.678872in}}%
\pgfpathlineto{\pgfqpoint{3.693874in}{3.695444in}}%
\pgfpathlineto{\pgfqpoint{3.700868in}{3.678872in}}%
\pgfpathlineto{\pgfqpoint{3.749824in}{3.678872in}}%
\pgfpathlineto{\pgfqpoint{3.756818in}{3.695444in}}%
\pgfpathlineto{\pgfqpoint{3.763812in}{3.678872in}}%
\pgfpathlineto{\pgfqpoint{3.770806in}{3.695444in}}%
\pgfpathlineto{\pgfqpoint{3.777799in}{3.695444in}}%
\pgfpathlineto{\pgfqpoint{3.784793in}{4.457738in}}%
\pgfpathlineto{\pgfqpoint{3.791787in}{4.474310in}}%
\pgfpathlineto{\pgfqpoint{3.847737in}{4.474310in}}%
\pgfpathlineto{\pgfqpoint{3.854731in}{4.490881in}}%
\pgfpathlineto{\pgfqpoint{3.861725in}{4.474310in}}%
\pgfpathlineto{\pgfqpoint{3.896694in}{4.474310in}}%
\pgfpathlineto{\pgfqpoint{3.903688in}{4.490881in}}%
\pgfpathlineto{\pgfqpoint{3.910681in}{4.474310in}}%
\pgfpathlineto{\pgfqpoint{3.917675in}{4.474310in}}%
\pgfpathlineto{\pgfqpoint{3.924669in}{4.490881in}}%
\pgfpathlineto{\pgfqpoint{3.952644in}{4.490881in}}%
\pgfpathlineto{\pgfqpoint{3.959638in}{3.695444in}}%
\pgfpathlineto{\pgfqpoint{3.966632in}{3.712015in}}%
\pgfpathlineto{\pgfqpoint{3.973626in}{3.695444in}}%
\pgfpathlineto{\pgfqpoint{3.987613in}{3.695444in}}%
\pgfpathlineto{\pgfqpoint{3.994607in}{3.712015in}}%
\pgfpathlineto{\pgfqpoint{4.001601in}{3.695444in}}%
\pgfpathlineto{\pgfqpoint{4.022582in}{3.695444in}}%
\pgfpathlineto{\pgfqpoint{4.029576in}{3.678872in}}%
\pgfpathlineto{\pgfqpoint{4.036570in}{3.695444in}}%
\pgfpathlineto{\pgfqpoint{4.043563in}{3.678872in}}%
\pgfpathlineto{\pgfqpoint{4.050557in}{3.695444in}}%
\pgfpathlineto{\pgfqpoint{4.057551in}{3.678872in}}%
\pgfpathlineto{\pgfqpoint{4.064545in}{3.695444in}}%
\pgfpathlineto{\pgfqpoint{4.071539in}{3.678872in}}%
\pgfpathlineto{\pgfqpoint{4.071539in}{3.678872in}}%
\pgfusepath{stroke}%
\end{pgfscope}%
\begin{pgfscope}%
\pgfpathrectangle{\pgfqpoint{2.610186in}{3.638271in}}{\pgfqpoint{1.530941in}{0.893210in}}%
\pgfusepath{clip}%
\pgfsetrectcap%
\pgfsetroundjoin%
\pgfsetlinewidth{1.505625pt}%
\definecolor{currentstroke}{rgb}{1.000000,0.498039,0.054902}%
\pgfsetstrokecolor{currentstroke}%
\pgfsetdash{}{0pt}%
\pgfpathmoveto{\pgfqpoint{0.000000in}{0.000000in}}%
\pgfusepath{stroke}%
\end{pgfscope}%
\begin{pgfscope}%
\pgfpathrectangle{\pgfqpoint{2.610186in}{3.638271in}}{\pgfqpoint{1.530941in}{0.893210in}}%
\pgfusepath{clip}%
\pgfsetrectcap%
\pgfsetroundjoin%
\pgfsetlinewidth{1.505625pt}%
\definecolor{currentstroke}{rgb}{0.172549,0.627451,0.172549}%
\pgfsetstrokecolor{currentstroke}%
\pgfsetdash{}{0pt}%
\pgfpathmoveto{\pgfqpoint{2.679774in}{3.993733in}}%
\pgfpathlineto{\pgfqpoint{2.686768in}{3.960589in}}%
\pgfpathlineto{\pgfqpoint{2.693761in}{3.960589in}}%
\pgfpathlineto{\pgfqpoint{2.700755in}{3.944018in}}%
\pgfpathlineto{\pgfqpoint{2.714743in}{3.944018in}}%
\pgfpathlineto{\pgfqpoint{2.721737in}{3.910875in}}%
\pgfpathlineto{\pgfqpoint{2.728730in}{3.894303in}}%
\pgfpathlineto{\pgfqpoint{2.735724in}{3.894303in}}%
\pgfpathlineto{\pgfqpoint{2.749712in}{3.960589in}}%
\pgfpathlineto{\pgfqpoint{2.756706in}{3.960589in}}%
\pgfpathlineto{\pgfqpoint{2.763699in}{3.993733in}}%
\pgfpathlineto{\pgfqpoint{2.819650in}{4.126306in}}%
\pgfpathlineto{\pgfqpoint{2.826643in}{4.126306in}}%
\pgfpathlineto{\pgfqpoint{2.840631in}{4.159449in}}%
\pgfpathlineto{\pgfqpoint{2.847625in}{4.159449in}}%
\pgfpathlineto{\pgfqpoint{2.854619in}{4.176021in}}%
\pgfpathlineto{\pgfqpoint{2.861612in}{4.209164in}}%
\pgfpathlineto{\pgfqpoint{2.868606in}{4.192592in}}%
\pgfpathlineto{\pgfqpoint{2.875600in}{4.225735in}}%
\pgfpathlineto{\pgfqpoint{2.882594in}{4.225735in}}%
\pgfpathlineto{\pgfqpoint{2.896581in}{4.258879in}}%
\pgfpathlineto{\pgfqpoint{2.903575in}{4.258879in}}%
\pgfpathlineto{\pgfqpoint{2.910569in}{4.292022in}}%
\pgfpathlineto{\pgfqpoint{2.924557in}{4.225735in}}%
\pgfpathlineto{\pgfqpoint{2.931550in}{4.209164in}}%
\pgfpathlineto{\pgfqpoint{2.938544in}{4.176021in}}%
\pgfpathlineto{\pgfqpoint{2.959526in}{4.126306in}}%
\pgfpathlineto{\pgfqpoint{2.966519in}{4.126306in}}%
\pgfpathlineto{\pgfqpoint{2.973513in}{4.093162in}}%
\pgfpathlineto{\pgfqpoint{3.001488in}{4.026876in}}%
\pgfpathlineto{\pgfqpoint{3.008482in}{4.026876in}}%
\pgfpathlineto{\pgfqpoint{3.015476in}{4.010304in}}%
\pgfpathlineto{\pgfqpoint{3.022470in}{4.010304in}}%
\pgfpathlineto{\pgfqpoint{3.029463in}{3.960589in}}%
\pgfpathlineto{\pgfqpoint{3.036457in}{3.977161in}}%
\pgfpathlineto{\pgfqpoint{3.043451in}{3.960589in}}%
\pgfpathlineto{\pgfqpoint{3.050445in}{3.960589in}}%
\pgfpathlineto{\pgfqpoint{3.057439in}{3.927446in}}%
\pgfpathlineto{\pgfqpoint{3.071426in}{3.927446in}}%
\pgfpathlineto{\pgfqpoint{3.078420in}{3.894303in}}%
\pgfpathlineto{\pgfqpoint{3.092408in}{3.927446in}}%
\pgfpathlineto{\pgfqpoint{3.099401in}{3.960589in}}%
\pgfpathlineto{\pgfqpoint{3.106395in}{3.960589in}}%
\pgfpathlineto{\pgfqpoint{3.113389in}{3.993733in}}%
\pgfpathlineto{\pgfqpoint{3.127377in}{4.026876in}}%
\pgfpathlineto{\pgfqpoint{3.134370in}{4.060019in}}%
\pgfpathlineto{\pgfqpoint{3.141364in}{4.060019in}}%
\pgfpathlineto{\pgfqpoint{3.155352in}{4.093162in}}%
\pgfpathlineto{\pgfqpoint{3.162346in}{4.093162in}}%
\pgfpathlineto{\pgfqpoint{3.169339in}{4.126306in}}%
\pgfpathlineto{\pgfqpoint{3.183327in}{4.159449in}}%
\pgfpathlineto{\pgfqpoint{3.190321in}{4.159449in}}%
\pgfpathlineto{\pgfqpoint{3.204308in}{4.192592in}}%
\pgfpathlineto{\pgfqpoint{3.211302in}{4.192592in}}%
\pgfpathlineto{\pgfqpoint{3.225290in}{4.225735in}}%
\pgfpathlineto{\pgfqpoint{3.232283in}{4.225735in}}%
\pgfpathlineto{\pgfqpoint{3.239277in}{4.258879in}}%
\pgfpathlineto{\pgfqpoint{3.260259in}{4.258879in}}%
\pgfpathlineto{\pgfqpoint{3.274246in}{4.225735in}}%
\pgfpathlineto{\pgfqpoint{3.288234in}{4.159449in}}%
\pgfpathlineto{\pgfqpoint{3.302221in}{4.159449in}}%
\pgfpathlineto{\pgfqpoint{3.316209in}{4.093162in}}%
\pgfpathlineto{\pgfqpoint{3.323203in}{4.093162in}}%
\pgfpathlineto{\pgfqpoint{3.337190in}{4.060019in}}%
\pgfpathlineto{\pgfqpoint{3.344184in}{4.060019in}}%
\pgfpathlineto{\pgfqpoint{3.351178in}{4.026876in}}%
\pgfpathlineto{\pgfqpoint{3.358172in}{4.026876in}}%
\pgfpathlineto{\pgfqpoint{3.379153in}{3.977161in}}%
\pgfpathlineto{\pgfqpoint{3.386147in}{3.977161in}}%
\pgfpathlineto{\pgfqpoint{3.400135in}{3.944018in}}%
\pgfpathlineto{\pgfqpoint{3.414122in}{3.944018in}}%
\pgfpathlineto{\pgfqpoint{3.428110in}{3.910875in}}%
\pgfpathlineto{\pgfqpoint{3.435103in}{3.910875in}}%
\pgfpathlineto{\pgfqpoint{3.442097in}{3.927446in}}%
\pgfpathlineto{\pgfqpoint{3.449091in}{3.960589in}}%
\pgfpathlineto{\pgfqpoint{3.456085in}{3.960589in}}%
\pgfpathlineto{\pgfqpoint{3.463079in}{3.993733in}}%
\pgfpathlineto{\pgfqpoint{3.477066in}{4.026876in}}%
\pgfpathlineto{\pgfqpoint{3.484060in}{4.026876in}}%
\pgfpathlineto{\pgfqpoint{3.491054in}{4.060019in}}%
\pgfpathlineto{\pgfqpoint{3.505041in}{4.093162in}}%
\pgfpathlineto{\pgfqpoint{3.512035in}{4.093162in}}%
\pgfpathlineto{\pgfqpoint{3.519029in}{4.126306in}}%
\pgfpathlineto{\pgfqpoint{3.526023in}{4.126306in}}%
\pgfpathlineto{\pgfqpoint{3.540010in}{4.159449in}}%
\pgfpathlineto{\pgfqpoint{3.547004in}{4.159449in}}%
\pgfpathlineto{\pgfqpoint{3.553998in}{4.192592in}}%
\pgfpathlineto{\pgfqpoint{3.560992in}{4.192592in}}%
\pgfpathlineto{\pgfqpoint{3.567986in}{4.225735in}}%
\pgfpathlineto{\pgfqpoint{3.588967in}{4.225735in}}%
\pgfpathlineto{\pgfqpoint{3.595961in}{4.258879in}}%
\pgfpathlineto{\pgfqpoint{3.602955in}{4.258879in}}%
\pgfpathlineto{\pgfqpoint{3.609948in}{4.275450in}}%
\pgfpathlineto{\pgfqpoint{3.616942in}{4.258879in}}%
\pgfpathlineto{\pgfqpoint{3.630930in}{4.192592in}}%
\pgfpathlineto{\pgfqpoint{3.679886in}{4.076591in}}%
\pgfpathlineto{\pgfqpoint{3.686880in}{4.076591in}}%
\pgfpathlineto{\pgfqpoint{3.693874in}{4.043448in}}%
\pgfpathlineto{\pgfqpoint{3.700868in}{4.026876in}}%
\pgfpathlineto{\pgfqpoint{3.707861in}{4.026876in}}%
\pgfpathlineto{\pgfqpoint{3.714855in}{4.010304in}}%
\pgfpathlineto{\pgfqpoint{3.721849in}{4.010304in}}%
\pgfpathlineto{\pgfqpoint{3.728843in}{3.993733in}}%
\pgfpathlineto{\pgfqpoint{3.735837in}{3.960589in}}%
\pgfpathlineto{\pgfqpoint{3.742830in}{3.960589in}}%
\pgfpathlineto{\pgfqpoint{3.749824in}{3.944018in}}%
\pgfpathlineto{\pgfqpoint{3.756818in}{3.944018in}}%
\pgfpathlineto{\pgfqpoint{3.763812in}{3.927446in}}%
\pgfpathlineto{\pgfqpoint{3.770806in}{3.927446in}}%
\pgfpathlineto{\pgfqpoint{3.784793in}{3.894303in}}%
\pgfpathlineto{\pgfqpoint{3.791787in}{3.927446in}}%
\pgfpathlineto{\pgfqpoint{3.805775in}{3.960589in}}%
\pgfpathlineto{\pgfqpoint{3.812768in}{3.993733in}}%
\pgfpathlineto{\pgfqpoint{3.840743in}{4.060019in}}%
\pgfpathlineto{\pgfqpoint{3.847737in}{4.093162in}}%
\pgfpathlineto{\pgfqpoint{3.854731in}{4.093162in}}%
\pgfpathlineto{\pgfqpoint{3.861725in}{4.109734in}}%
\pgfpathlineto{\pgfqpoint{3.868719in}{4.109734in}}%
\pgfpathlineto{\pgfqpoint{3.875712in}{4.142877in}}%
\pgfpathlineto{\pgfqpoint{3.882706in}{4.142877in}}%
\pgfpathlineto{\pgfqpoint{3.889700in}{4.159449in}}%
\pgfpathlineto{\pgfqpoint{3.896694in}{4.159449in}}%
\pgfpathlineto{\pgfqpoint{3.924669in}{4.225735in}}%
\pgfpathlineto{\pgfqpoint{3.931663in}{4.225735in}}%
\pgfpathlineto{\pgfqpoint{3.938657in}{4.258879in}}%
\pgfpathlineto{\pgfqpoint{3.945650in}{4.258879in}}%
\pgfpathlineto{\pgfqpoint{3.952644in}{4.275450in}}%
\pgfpathlineto{\pgfqpoint{3.973626in}{4.225735in}}%
\pgfpathlineto{\pgfqpoint{3.980619in}{4.192592in}}%
\pgfpathlineto{\pgfqpoint{4.029576in}{4.076591in}}%
\pgfpathlineto{\pgfqpoint{4.036570in}{4.076591in}}%
\pgfpathlineto{\pgfqpoint{4.043563in}{4.043448in}}%
\pgfpathlineto{\pgfqpoint{4.050557in}{4.026876in}}%
\pgfpathlineto{\pgfqpoint{4.057551in}{4.026876in}}%
\pgfpathlineto{\pgfqpoint{4.071539in}{3.993733in}}%
\pgfpathlineto{\pgfqpoint{4.071539in}{3.993733in}}%
\pgfusepath{stroke}%
\end{pgfscope}%
\begin{pgfscope}%
\pgfpathrectangle{\pgfqpoint{2.610186in}{3.638271in}}{\pgfqpoint{1.530941in}{0.893210in}}%
\pgfusepath{clip}%
\pgfsetrectcap%
\pgfsetroundjoin%
\pgfsetlinewidth{1.505625pt}%
\definecolor{currentstroke}{rgb}{0.839216,0.152941,0.156863}%
\pgfsetstrokecolor{currentstroke}%
\pgfsetdash{}{0pt}%
\pgfpathmoveto{\pgfqpoint{0.000000in}{0.000000in}}%
\pgfusepath{stroke}%
\end{pgfscope}%
\begin{pgfscope}%
\pgfsetrectcap%
\pgfsetmiterjoin%
\pgfsetlinewidth{0.803000pt}%
\definecolor{currentstroke}{rgb}{0.000000,0.000000,0.000000}%
\pgfsetstrokecolor{currentstroke}%
\pgfsetdash{}{0pt}%
\pgfpathmoveto{\pgfqpoint{2.610186in}{3.638271in}}%
\pgfpathlineto{\pgfqpoint{2.610186in}{4.531482in}}%
\pgfusepath{stroke}%
\end{pgfscope}%
\begin{pgfscope}%
\pgfsetrectcap%
\pgfsetmiterjoin%
\pgfsetlinewidth{0.803000pt}%
\definecolor{currentstroke}{rgb}{0.000000,0.000000,0.000000}%
\pgfsetstrokecolor{currentstroke}%
\pgfsetdash{}{0pt}%
\pgfpathmoveto{\pgfqpoint{4.141127in}{3.638271in}}%
\pgfpathlineto{\pgfqpoint{4.141127in}{4.531482in}}%
\pgfusepath{stroke}%
\end{pgfscope}%
\begin{pgfscope}%
\pgfsetrectcap%
\pgfsetmiterjoin%
\pgfsetlinewidth{0.803000pt}%
\definecolor{currentstroke}{rgb}{0.000000,0.000000,0.000000}%
\pgfsetstrokecolor{currentstroke}%
\pgfsetdash{}{0pt}%
\pgfpathmoveto{\pgfqpoint{2.610186in}{3.638271in}}%
\pgfpathlineto{\pgfqpoint{4.141127in}{3.638271in}}%
\pgfusepath{stroke}%
\end{pgfscope}%
\begin{pgfscope}%
\pgfsetrectcap%
\pgfsetmiterjoin%
\pgfsetlinewidth{0.803000pt}%
\definecolor{currentstroke}{rgb}{0.000000,0.000000,0.000000}%
\pgfsetstrokecolor{currentstroke}%
\pgfsetdash{}{0pt}%
\pgfpathmoveto{\pgfqpoint{2.610186in}{4.531482in}}%
\pgfpathlineto{\pgfqpoint{4.141127in}{4.531482in}}%
\pgfusepath{stroke}%
\end{pgfscope}%
\begin{pgfscope}%
\definecolor{textcolor}{rgb}{0.000000,0.000000,0.000000}%
\pgfsetstrokecolor{textcolor}%
\pgfsetfillcolor{textcolor}%
\pgftext[x=3.375656in,y=4.614815in,,base]{\color{textcolor}\rmfamily\fontsize{12.000000}{14.400000}\selectfont 1}%
\end{pgfscope}%
\begin{pgfscope}%
\pgfsetbuttcap%
\pgfsetmiterjoin%
\definecolor{currentfill}{rgb}{1.000000,1.000000,1.000000}%
\pgfsetfillcolor{currentfill}%
\pgfsetlinewidth{0.000000pt}%
\definecolor{currentstroke}{rgb}{0.000000,0.000000,0.000000}%
\pgfsetstrokecolor{currentstroke}%
\pgfsetstrokeopacity{0.000000}%
\pgfsetdash{}{0pt}%
\pgfpathmoveto{\pgfqpoint{4.720371in}{3.638271in}}%
\pgfpathlineto{\pgfqpoint{6.251312in}{3.638271in}}%
\pgfpathlineto{\pgfqpoint{6.251312in}{4.531482in}}%
\pgfpathlineto{\pgfqpoint{4.720371in}{4.531482in}}%
\pgfpathclose%
\pgfusepath{fill}%
\end{pgfscope}%
\begin{pgfscope}%
\pgfsetbuttcap%
\pgfsetroundjoin%
\definecolor{currentfill}{rgb}{0.000000,0.000000,0.000000}%
\pgfsetfillcolor{currentfill}%
\pgfsetlinewidth{0.803000pt}%
\definecolor{currentstroke}{rgb}{0.000000,0.000000,0.000000}%
\pgfsetstrokecolor{currentstroke}%
\pgfsetdash{}{0pt}%
\pgfsys@defobject{currentmarker}{\pgfqpoint{0.000000in}{-0.048611in}}{\pgfqpoint{0.000000in}{0.000000in}}{%
\pgfpathmoveto{\pgfqpoint{0.000000in}{0.000000in}}%
\pgfpathlineto{\pgfqpoint{0.000000in}{-0.048611in}}%
\pgfusepath{stroke,fill}%
}%
\begin{pgfscope}%
\pgfsys@transformshift{4.789959in}{3.638271in}%
\pgfsys@useobject{currentmarker}{}%
\end{pgfscope}%
\end{pgfscope}%
\begin{pgfscope}%
\definecolor{textcolor}{rgb}{0.000000,0.000000,0.000000}%
\pgfsetstrokecolor{textcolor}%
\pgfsetfillcolor{textcolor}%
\pgftext[x=4.789959in,y=3.541049in,,top]{\color{textcolor}\rmfamily\fontsize{10.000000}{12.000000}\selectfont \(\displaystyle {0}\)}%
\end{pgfscope}%
\begin{pgfscope}%
\pgfsetbuttcap%
\pgfsetroundjoin%
\definecolor{currentfill}{rgb}{0.000000,0.000000,0.000000}%
\pgfsetfillcolor{currentfill}%
\pgfsetlinewidth{0.803000pt}%
\definecolor{currentstroke}{rgb}{0.000000,0.000000,0.000000}%
\pgfsetstrokecolor{currentstroke}%
\pgfsetdash{}{0pt}%
\pgfsys@defobject{currentmarker}{\pgfqpoint{0.000000in}{-0.048611in}}{\pgfqpoint{0.000000in}{0.000000in}}{%
\pgfpathmoveto{\pgfqpoint{0.000000in}{0.000000in}}%
\pgfpathlineto{\pgfqpoint{0.000000in}{-0.048611in}}%
\pgfusepath{stroke,fill}%
}%
\begin{pgfscope}%
\pgfsys@transformshift{5.489338in}{3.638271in}%
\pgfsys@useobject{currentmarker}{}%
\end{pgfscope}%
\end{pgfscope}%
\begin{pgfscope}%
\definecolor{textcolor}{rgb}{0.000000,0.000000,0.000000}%
\pgfsetstrokecolor{textcolor}%
\pgfsetfillcolor{textcolor}%
\pgftext[x=5.489338in,y=3.541049in,,top]{\color{textcolor}\rmfamily\fontsize{10.000000}{12.000000}\selectfont \(\displaystyle {2000}\)}%
\end{pgfscope}%
\begin{pgfscope}%
\pgfsetbuttcap%
\pgfsetroundjoin%
\definecolor{currentfill}{rgb}{0.000000,0.000000,0.000000}%
\pgfsetfillcolor{currentfill}%
\pgfsetlinewidth{0.803000pt}%
\definecolor{currentstroke}{rgb}{0.000000,0.000000,0.000000}%
\pgfsetstrokecolor{currentstroke}%
\pgfsetdash{}{0pt}%
\pgfsys@defobject{currentmarker}{\pgfqpoint{0.000000in}{-0.048611in}}{\pgfqpoint{0.000000in}{0.000000in}}{%
\pgfpathmoveto{\pgfqpoint{0.000000in}{0.000000in}}%
\pgfpathlineto{\pgfqpoint{0.000000in}{-0.048611in}}%
\pgfusepath{stroke,fill}%
}%
\begin{pgfscope}%
\pgfsys@transformshift{6.188718in}{3.638271in}%
\pgfsys@useobject{currentmarker}{}%
\end{pgfscope}%
\end{pgfscope}%
\begin{pgfscope}%
\definecolor{textcolor}{rgb}{0.000000,0.000000,0.000000}%
\pgfsetstrokecolor{textcolor}%
\pgfsetfillcolor{textcolor}%
\pgftext[x=6.188718in,y=3.541049in,,top]{\color{textcolor}\rmfamily\fontsize{10.000000}{12.000000}\selectfont \(\displaystyle {4000}\)}%
\end{pgfscope}%
\begin{pgfscope}%
\definecolor{textcolor}{rgb}{0.000000,0.000000,0.000000}%
\pgfsetstrokecolor{textcolor}%
\pgfsetfillcolor{textcolor}%
\pgftext[x=5.485841in,y=3.362037in,,top]{\color{textcolor}\rmfamily\fontsize{10.000000}{12.000000}\selectfont Czas [\(\displaystyle \mu s\)] }%
\end{pgfscope}%
\begin{pgfscope}%
\pgfsetbuttcap%
\pgfsetroundjoin%
\definecolor{currentfill}{rgb}{0.000000,0.000000,0.000000}%
\pgfsetfillcolor{currentfill}%
\pgfsetlinewidth{0.803000pt}%
\definecolor{currentstroke}{rgb}{0.000000,0.000000,0.000000}%
\pgfsetstrokecolor{currentstroke}%
\pgfsetdash{}{0pt}%
\pgfsys@defobject{currentmarker}{\pgfqpoint{-0.048611in}{0.000000in}}{\pgfqpoint{-0.000000in}{0.000000in}}{%
\pgfpathmoveto{\pgfqpoint{-0.000000in}{0.000000in}}%
\pgfpathlineto{\pgfqpoint{-0.048611in}{0.000000in}}%
\pgfusepath{stroke,fill}%
}%
\begin{pgfscope}%
\pgfsys@transformshift{4.720371in}{3.678872in}%
\pgfsys@useobject{currentmarker}{}%
\end{pgfscope}%
\end{pgfscope}%
\begin{pgfscope}%
\definecolor{textcolor}{rgb}{0.000000,0.000000,0.000000}%
\pgfsetstrokecolor{textcolor}%
\pgfsetfillcolor{textcolor}%
\pgftext[x=4.553704in, y=3.630647in, left, base]{\color{textcolor}\rmfamily\fontsize{10.000000}{12.000000}\selectfont \(\displaystyle {0}\)}%
\end{pgfscope}%
\begin{pgfscope}%
\pgfsetbuttcap%
\pgfsetroundjoin%
\definecolor{currentfill}{rgb}{0.000000,0.000000,0.000000}%
\pgfsetfillcolor{currentfill}%
\pgfsetlinewidth{0.803000pt}%
\definecolor{currentstroke}{rgb}{0.000000,0.000000,0.000000}%
\pgfsetstrokecolor{currentstroke}%
\pgfsetdash{}{0pt}%
\pgfsys@defobject{currentmarker}{\pgfqpoint{-0.048611in}{0.000000in}}{\pgfqpoint{-0.000000in}{0.000000in}}{%
\pgfpathmoveto{\pgfqpoint{-0.000000in}{0.000000in}}%
\pgfpathlineto{\pgfqpoint{-0.048611in}{0.000000in}}%
\pgfusepath{stroke,fill}%
}%
\begin{pgfscope}%
\pgfsys@transformshift{4.720371in}{4.003676in}%
\pgfsys@useobject{currentmarker}{}%
\end{pgfscope}%
\end{pgfscope}%
\begin{pgfscope}%
\definecolor{textcolor}{rgb}{0.000000,0.000000,0.000000}%
\pgfsetstrokecolor{textcolor}%
\pgfsetfillcolor{textcolor}%
\pgftext[x=4.484259in, y=3.955450in, left, base]{\color{textcolor}\rmfamily\fontsize{10.000000}{12.000000}\selectfont \(\displaystyle {20}\)}%
\end{pgfscope}%
\begin{pgfscope}%
\pgfsetbuttcap%
\pgfsetroundjoin%
\definecolor{currentfill}{rgb}{0.000000,0.000000,0.000000}%
\pgfsetfillcolor{currentfill}%
\pgfsetlinewidth{0.803000pt}%
\definecolor{currentstroke}{rgb}{0.000000,0.000000,0.000000}%
\pgfsetstrokecolor{currentstroke}%
\pgfsetdash{}{0pt}%
\pgfsys@defobject{currentmarker}{\pgfqpoint{-0.048611in}{0.000000in}}{\pgfqpoint{-0.000000in}{0.000000in}}{%
\pgfpathmoveto{\pgfqpoint{-0.000000in}{0.000000in}}%
\pgfpathlineto{\pgfqpoint{-0.048611in}{0.000000in}}%
\pgfusepath{stroke,fill}%
}%
\begin{pgfscope}%
\pgfsys@transformshift{4.720371in}{4.328479in}%
\pgfsys@useobject{currentmarker}{}%
\end{pgfscope}%
\end{pgfscope}%
\begin{pgfscope}%
\definecolor{textcolor}{rgb}{0.000000,0.000000,0.000000}%
\pgfsetstrokecolor{textcolor}%
\pgfsetfillcolor{textcolor}%
\pgftext[x=4.484259in, y=4.280254in, left, base]{\color{textcolor}\rmfamily\fontsize{10.000000}{12.000000}\selectfont \(\displaystyle {40}\)}%
\end{pgfscope}%
\begin{pgfscope}%
\definecolor{textcolor}{rgb}{0.000000,0.000000,0.000000}%
\pgfsetstrokecolor{textcolor}%
\pgfsetfillcolor{textcolor}%
\pgftext[x=4.428704in,y=4.084877in,,bottom,rotate=90.000000]{\color{textcolor}\rmfamily\fontsize{10.000000}{12.000000}\selectfont Napiecie [\(\displaystyle V\)] }%
\end{pgfscope}%
\begin{pgfscope}%
\pgfpathrectangle{\pgfqpoint{4.720371in}{3.638271in}}{\pgfqpoint{1.530941in}{0.893210in}}%
\pgfusepath{clip}%
\pgfsetrectcap%
\pgfsetroundjoin%
\pgfsetlinewidth{1.505625pt}%
\definecolor{currentstroke}{rgb}{0.121569,0.466667,0.705882}%
\pgfsetstrokecolor{currentstroke}%
\pgfsetdash{}{0pt}%
\pgfpathmoveto{\pgfqpoint{4.789959in}{3.695112in}}%
\pgfpathlineto{\pgfqpoint{4.796953in}{3.678872in}}%
\pgfpathlineto{\pgfqpoint{4.838916in}{3.678872in}}%
\pgfpathlineto{\pgfqpoint{4.845909in}{4.458401in}}%
\pgfpathlineto{\pgfqpoint{4.908854in}{4.458401in}}%
\pgfpathlineto{\pgfqpoint{4.915847in}{4.474641in}}%
\pgfpathlineto{\pgfqpoint{4.922841in}{4.458401in}}%
\pgfpathlineto{\pgfqpoint{4.929835in}{4.458401in}}%
\pgfpathlineto{\pgfqpoint{4.936829in}{4.474641in}}%
\pgfpathlineto{\pgfqpoint{4.943822in}{4.458401in}}%
\pgfpathlineto{\pgfqpoint{4.950816in}{4.474641in}}%
\pgfpathlineto{\pgfqpoint{4.957810in}{4.474641in}}%
\pgfpathlineto{\pgfqpoint{4.964804in}{4.458401in}}%
\pgfpathlineto{\pgfqpoint{4.971798in}{4.474641in}}%
\pgfpathlineto{\pgfqpoint{4.992779in}{4.474641in}}%
\pgfpathlineto{\pgfqpoint{4.999773in}{4.458401in}}%
\pgfpathlineto{\pgfqpoint{5.006767in}{4.474641in}}%
\pgfpathlineto{\pgfqpoint{5.027748in}{4.474641in}}%
\pgfpathlineto{\pgfqpoint{5.034742in}{4.458401in}}%
\pgfpathlineto{\pgfqpoint{5.041736in}{4.490881in}}%
\pgfpathlineto{\pgfqpoint{5.048729in}{4.474641in}}%
\pgfpathlineto{\pgfqpoint{5.062717in}{4.474641in}}%
\pgfpathlineto{\pgfqpoint{5.069711in}{4.458401in}}%
\pgfpathlineto{\pgfqpoint{5.076705in}{4.474641in}}%
\pgfpathlineto{\pgfqpoint{5.083698in}{4.474641in}}%
\pgfpathlineto{\pgfqpoint{5.090692in}{4.490881in}}%
\pgfpathlineto{\pgfqpoint{5.097686in}{4.474641in}}%
\pgfpathlineto{\pgfqpoint{5.104680in}{4.474641in}}%
\pgfpathlineto{\pgfqpoint{5.111674in}{3.711352in}}%
\pgfpathlineto{\pgfqpoint{5.118667in}{3.711352in}}%
\pgfpathlineto{\pgfqpoint{5.125661in}{3.695112in}}%
\pgfpathlineto{\pgfqpoint{5.132655in}{3.695112in}}%
\pgfpathlineto{\pgfqpoint{5.139649in}{3.711352in}}%
\pgfpathlineto{\pgfqpoint{5.146642in}{3.695112in}}%
\pgfpathlineto{\pgfqpoint{5.153636in}{3.711352in}}%
\pgfpathlineto{\pgfqpoint{5.160630in}{3.711352in}}%
\pgfpathlineto{\pgfqpoint{5.167624in}{3.695112in}}%
\pgfpathlineto{\pgfqpoint{5.195599in}{3.695112in}}%
\pgfpathlineto{\pgfqpoint{5.202593in}{3.678872in}}%
\pgfpathlineto{\pgfqpoint{5.209587in}{3.695112in}}%
\pgfpathlineto{\pgfqpoint{5.216580in}{3.695112in}}%
\pgfpathlineto{\pgfqpoint{5.223574in}{3.678872in}}%
\pgfpathlineto{\pgfqpoint{5.230568in}{3.678872in}}%
\pgfpathlineto{\pgfqpoint{5.237562in}{3.695112in}}%
\pgfpathlineto{\pgfqpoint{5.244556in}{3.695112in}}%
\pgfpathlineto{\pgfqpoint{5.251549in}{3.678872in}}%
\pgfpathlineto{\pgfqpoint{5.279525in}{3.678872in}}%
\pgfpathlineto{\pgfqpoint{5.286518in}{3.695112in}}%
\pgfpathlineto{\pgfqpoint{5.293512in}{3.678872in}}%
\pgfpathlineto{\pgfqpoint{5.363450in}{3.678872in}}%
\pgfpathlineto{\pgfqpoint{5.370444in}{4.442161in}}%
\pgfpathlineto{\pgfqpoint{5.377438in}{4.458401in}}%
\pgfpathlineto{\pgfqpoint{5.391425in}{4.458401in}}%
\pgfpathlineto{\pgfqpoint{5.398419in}{4.442161in}}%
\pgfpathlineto{\pgfqpoint{5.405413in}{4.458401in}}%
\pgfpathlineto{\pgfqpoint{5.482345in}{4.458401in}}%
\pgfpathlineto{\pgfqpoint{5.489338in}{4.474641in}}%
\pgfpathlineto{\pgfqpoint{5.517313in}{4.474641in}}%
\pgfpathlineto{\pgfqpoint{5.524307in}{4.458401in}}%
\pgfpathlineto{\pgfqpoint{5.531301in}{4.474641in}}%
\pgfpathlineto{\pgfqpoint{5.573264in}{4.474641in}}%
\pgfpathlineto{\pgfqpoint{5.580258in}{4.490881in}}%
\pgfpathlineto{\pgfqpoint{5.587251in}{4.474641in}}%
\pgfpathlineto{\pgfqpoint{5.594245in}{4.490881in}}%
\pgfpathlineto{\pgfqpoint{5.601239in}{4.490881in}}%
\pgfpathlineto{\pgfqpoint{5.608233in}{4.474641in}}%
\pgfpathlineto{\pgfqpoint{5.629214in}{4.474641in}}%
\pgfpathlineto{\pgfqpoint{5.636208in}{3.711352in}}%
\pgfpathlineto{\pgfqpoint{5.664183in}{3.711352in}}%
\pgfpathlineto{\pgfqpoint{5.671177in}{3.695112in}}%
\pgfpathlineto{\pgfqpoint{5.713140in}{3.695112in}}%
\pgfpathlineto{\pgfqpoint{5.720133in}{3.678872in}}%
\pgfpathlineto{\pgfqpoint{5.727127in}{3.695112in}}%
\pgfpathlineto{\pgfqpoint{5.762096in}{3.695112in}}%
\pgfpathlineto{\pgfqpoint{5.769090in}{3.678872in}}%
\pgfpathlineto{\pgfqpoint{5.776084in}{3.695112in}}%
\pgfpathlineto{\pgfqpoint{5.783078in}{3.695112in}}%
\pgfpathlineto{\pgfqpoint{5.790071in}{3.678872in}}%
\pgfpathlineto{\pgfqpoint{5.797065in}{3.695112in}}%
\pgfpathlineto{\pgfqpoint{5.804059in}{3.678872in}}%
\pgfpathlineto{\pgfqpoint{5.825040in}{3.678872in}}%
\pgfpathlineto{\pgfqpoint{5.832034in}{3.695112in}}%
\pgfpathlineto{\pgfqpoint{5.839028in}{3.678872in}}%
\pgfpathlineto{\pgfqpoint{5.867003in}{3.678872in}}%
\pgfpathlineto{\pgfqpoint{5.873997in}{3.695112in}}%
\pgfpathlineto{\pgfqpoint{5.880991in}{3.695112in}}%
\pgfpathlineto{\pgfqpoint{5.887985in}{3.678872in}}%
\pgfpathlineto{\pgfqpoint{5.894978in}{4.442161in}}%
\pgfpathlineto{\pgfqpoint{5.908966in}{4.442161in}}%
\pgfpathlineto{\pgfqpoint{5.915960in}{4.458401in}}%
\pgfpathlineto{\pgfqpoint{5.964916in}{4.458401in}}%
\pgfpathlineto{\pgfqpoint{5.971910in}{4.474641in}}%
\pgfpathlineto{\pgfqpoint{5.978904in}{4.474641in}}%
\pgfpathlineto{\pgfqpoint{5.985898in}{4.458401in}}%
\pgfpathlineto{\pgfqpoint{5.992891in}{4.474641in}}%
\pgfpathlineto{\pgfqpoint{5.999885in}{4.458401in}}%
\pgfpathlineto{\pgfqpoint{6.020867in}{4.458401in}}%
\pgfpathlineto{\pgfqpoint{6.027860in}{4.474641in}}%
\pgfpathlineto{\pgfqpoint{6.055836in}{4.474641in}}%
\pgfpathlineto{\pgfqpoint{6.062829in}{4.458401in}}%
\pgfpathlineto{\pgfqpoint{6.069823in}{4.474641in}}%
\pgfpathlineto{\pgfqpoint{6.097798in}{4.474641in}}%
\pgfpathlineto{\pgfqpoint{6.104792in}{4.458401in}}%
\pgfpathlineto{\pgfqpoint{6.118780in}{4.490881in}}%
\pgfpathlineto{\pgfqpoint{6.125773in}{4.490881in}}%
\pgfpathlineto{\pgfqpoint{6.132767in}{4.474641in}}%
\pgfpathlineto{\pgfqpoint{6.153749in}{4.474641in}}%
\pgfpathlineto{\pgfqpoint{6.160742in}{3.711352in}}%
\pgfpathlineto{\pgfqpoint{6.167736in}{3.711352in}}%
\pgfpathlineto{\pgfqpoint{6.174730in}{3.695112in}}%
\pgfpathlineto{\pgfqpoint{6.181724in}{3.695112in}}%
\pgfpathlineto{\pgfqpoint{6.181724in}{3.695112in}}%
\pgfusepath{stroke}%
\end{pgfscope}%
\begin{pgfscope}%
\pgfpathrectangle{\pgfqpoint{4.720371in}{3.638271in}}{\pgfqpoint{1.530941in}{0.893210in}}%
\pgfusepath{clip}%
\pgfsetrectcap%
\pgfsetroundjoin%
\pgfsetlinewidth{1.505625pt}%
\definecolor{currentstroke}{rgb}{1.000000,0.498039,0.054902}%
\pgfsetstrokecolor{currentstroke}%
\pgfsetdash{}{0pt}%
\pgfpathmoveto{\pgfqpoint{0.000000in}{0.000000in}}%
\pgfusepath{stroke}%
\end{pgfscope}%
\begin{pgfscope}%
\pgfpathrectangle{\pgfqpoint{4.720371in}{3.638271in}}{\pgfqpoint{1.530941in}{0.893210in}}%
\pgfusepath{clip}%
\pgfsetrectcap%
\pgfsetroundjoin%
\pgfsetlinewidth{1.505625pt}%
\definecolor{currentstroke}{rgb}{0.172549,0.627451,0.172549}%
\pgfsetstrokecolor{currentstroke}%
\pgfsetdash{}{0pt}%
\pgfpathmoveto{\pgfqpoint{4.789959in}{4.214798in}}%
\pgfpathlineto{\pgfqpoint{4.796953in}{3.873754in}}%
\pgfpathlineto{\pgfqpoint{4.803947in}{3.857514in}}%
\pgfpathlineto{\pgfqpoint{4.817934in}{3.857514in}}%
\pgfpathlineto{\pgfqpoint{4.831922in}{3.825034in}}%
\pgfpathlineto{\pgfqpoint{4.845909in}{3.825034in}}%
\pgfpathlineto{\pgfqpoint{4.859897in}{3.889994in}}%
\pgfpathlineto{\pgfqpoint{4.873885in}{3.922475in}}%
\pgfpathlineto{\pgfqpoint{4.880878in}{3.954955in}}%
\pgfpathlineto{\pgfqpoint{4.887872in}{3.954955in}}%
\pgfpathlineto{\pgfqpoint{4.901860in}{4.019916in}}%
\pgfpathlineto{\pgfqpoint{4.908854in}{4.019916in}}%
\pgfpathlineto{\pgfqpoint{4.950816in}{4.117357in}}%
\pgfpathlineto{\pgfqpoint{4.957810in}{4.149837in}}%
\pgfpathlineto{\pgfqpoint{4.978791in}{4.149837in}}%
\pgfpathlineto{\pgfqpoint{4.992779in}{4.182318in}}%
\pgfpathlineto{\pgfqpoint{4.999773in}{4.214798in}}%
\pgfpathlineto{\pgfqpoint{5.020754in}{4.214798in}}%
\pgfpathlineto{\pgfqpoint{5.048729in}{4.279759in}}%
\pgfpathlineto{\pgfqpoint{5.069711in}{4.279759in}}%
\pgfpathlineto{\pgfqpoint{5.083698in}{4.312239in}}%
\pgfpathlineto{\pgfqpoint{5.097686in}{4.312239in}}%
\pgfpathlineto{\pgfqpoint{5.104680in}{4.328479in}}%
\pgfpathlineto{\pgfqpoint{5.111674in}{4.312239in}}%
\pgfpathlineto{\pgfqpoint{5.118667in}{4.279759in}}%
\pgfpathlineto{\pgfqpoint{5.125661in}{4.263519in}}%
\pgfpathlineto{\pgfqpoint{5.132655in}{4.231038in}}%
\pgfpathlineto{\pgfqpoint{5.153636in}{4.182318in}}%
\pgfpathlineto{\pgfqpoint{5.160630in}{4.149837in}}%
\pgfpathlineto{\pgfqpoint{5.167624in}{4.149837in}}%
\pgfpathlineto{\pgfqpoint{5.174618in}{4.117357in}}%
\pgfpathlineto{\pgfqpoint{5.188605in}{4.084877in}}%
\pgfpathlineto{\pgfqpoint{5.195599in}{4.084877in}}%
\pgfpathlineto{\pgfqpoint{5.202593in}{4.052396in}}%
\pgfpathlineto{\pgfqpoint{5.209587in}{4.052396in}}%
\pgfpathlineto{\pgfqpoint{5.216580in}{4.019916in}}%
\pgfpathlineto{\pgfqpoint{5.230568in}{4.019916in}}%
\pgfpathlineto{\pgfqpoint{5.237562in}{4.003676in}}%
\pgfpathlineto{\pgfqpoint{5.244556in}{3.971195in}}%
\pgfpathlineto{\pgfqpoint{5.251549in}{3.954955in}}%
\pgfpathlineto{\pgfqpoint{5.265537in}{3.954955in}}%
\pgfpathlineto{\pgfqpoint{5.279525in}{3.922475in}}%
\pgfpathlineto{\pgfqpoint{5.286518in}{3.922475in}}%
\pgfpathlineto{\pgfqpoint{5.293512in}{3.906235in}}%
\pgfpathlineto{\pgfqpoint{5.300506in}{3.906235in}}%
\pgfpathlineto{\pgfqpoint{5.307500in}{3.889994in}}%
\pgfpathlineto{\pgfqpoint{5.314494in}{3.889994in}}%
\pgfpathlineto{\pgfqpoint{5.321487in}{3.873754in}}%
\pgfpathlineto{\pgfqpoint{5.335475in}{3.873754in}}%
\pgfpathlineto{\pgfqpoint{5.342469in}{3.857514in}}%
\pgfpathlineto{\pgfqpoint{5.349462in}{3.857514in}}%
\pgfpathlineto{\pgfqpoint{5.356456in}{3.825034in}}%
\pgfpathlineto{\pgfqpoint{5.370444in}{3.825034in}}%
\pgfpathlineto{\pgfqpoint{5.377438in}{3.857514in}}%
\pgfpathlineto{\pgfqpoint{5.391425in}{3.889994in}}%
\pgfpathlineto{\pgfqpoint{5.398419in}{3.938715in}}%
\pgfpathlineto{\pgfqpoint{5.405413in}{3.938715in}}%
\pgfpathlineto{\pgfqpoint{5.412407in}{3.954955in}}%
\pgfpathlineto{\pgfqpoint{5.426394in}{4.019916in}}%
\pgfpathlineto{\pgfqpoint{5.433388in}{4.019916in}}%
\pgfpathlineto{\pgfqpoint{5.461363in}{4.084877in}}%
\pgfpathlineto{\pgfqpoint{5.468357in}{4.084877in}}%
\pgfpathlineto{\pgfqpoint{5.475351in}{4.117357in}}%
\pgfpathlineto{\pgfqpoint{5.482345in}{4.117357in}}%
\pgfpathlineto{\pgfqpoint{5.496332in}{4.149837in}}%
\pgfpathlineto{\pgfqpoint{5.503326in}{4.149837in}}%
\pgfpathlineto{\pgfqpoint{5.510320in}{4.182318in}}%
\pgfpathlineto{\pgfqpoint{5.517313in}{4.182318in}}%
\pgfpathlineto{\pgfqpoint{5.531301in}{4.214798in}}%
\pgfpathlineto{\pgfqpoint{5.545289in}{4.214798in}}%
\pgfpathlineto{\pgfqpoint{5.552282in}{4.247279in}}%
\pgfpathlineto{\pgfqpoint{5.559276in}{4.247279in}}%
\pgfpathlineto{\pgfqpoint{5.566270in}{4.279759in}}%
\pgfpathlineto{\pgfqpoint{5.594245in}{4.279759in}}%
\pgfpathlineto{\pgfqpoint{5.601239in}{4.295999in}}%
\pgfpathlineto{\pgfqpoint{5.608233in}{4.295999in}}%
\pgfpathlineto{\pgfqpoint{5.615227in}{4.312239in}}%
\pgfpathlineto{\pgfqpoint{5.622220in}{4.312239in}}%
\pgfpathlineto{\pgfqpoint{5.629214in}{4.328479in}}%
\pgfpathlineto{\pgfqpoint{5.636208in}{4.312239in}}%
\pgfpathlineto{\pgfqpoint{5.643202in}{4.279759in}}%
\pgfpathlineto{\pgfqpoint{5.650196in}{4.263519in}}%
\pgfpathlineto{\pgfqpoint{5.657189in}{4.231038in}}%
\pgfpathlineto{\pgfqpoint{5.664183in}{4.214798in}}%
\pgfpathlineto{\pgfqpoint{5.671177in}{4.182318in}}%
\pgfpathlineto{\pgfqpoint{5.678171in}{4.182318in}}%
\pgfpathlineto{\pgfqpoint{5.685165in}{4.149837in}}%
\pgfpathlineto{\pgfqpoint{5.692158in}{4.149837in}}%
\pgfpathlineto{\pgfqpoint{5.699152in}{4.117357in}}%
\pgfpathlineto{\pgfqpoint{5.713140in}{4.084877in}}%
\pgfpathlineto{\pgfqpoint{5.720133in}{4.084877in}}%
\pgfpathlineto{\pgfqpoint{5.727127in}{4.052396in}}%
\pgfpathlineto{\pgfqpoint{5.734121in}{4.052396in}}%
\pgfpathlineto{\pgfqpoint{5.748109in}{4.019916in}}%
\pgfpathlineto{\pgfqpoint{5.755102in}{4.019916in}}%
\pgfpathlineto{\pgfqpoint{5.783078in}{3.954955in}}%
\pgfpathlineto{\pgfqpoint{5.790071in}{3.954955in}}%
\pgfpathlineto{\pgfqpoint{5.797065in}{3.938715in}}%
\pgfpathlineto{\pgfqpoint{5.804059in}{3.938715in}}%
\pgfpathlineto{\pgfqpoint{5.811053in}{3.922475in}}%
\pgfpathlineto{\pgfqpoint{5.818047in}{3.922475in}}%
\pgfpathlineto{\pgfqpoint{5.825040in}{3.889994in}}%
\pgfpathlineto{\pgfqpoint{5.846022in}{3.889994in}}%
\pgfpathlineto{\pgfqpoint{5.853016in}{3.873754in}}%
\pgfpathlineto{\pgfqpoint{5.860009in}{3.873754in}}%
\pgfpathlineto{\pgfqpoint{5.867003in}{3.857514in}}%
\pgfpathlineto{\pgfqpoint{5.873997in}{3.857514in}}%
\pgfpathlineto{\pgfqpoint{5.887985in}{3.825034in}}%
\pgfpathlineto{\pgfqpoint{5.894978in}{3.825034in}}%
\pgfpathlineto{\pgfqpoint{5.901972in}{3.857514in}}%
\pgfpathlineto{\pgfqpoint{5.915960in}{3.889994in}}%
\pgfpathlineto{\pgfqpoint{5.929947in}{3.954955in}}%
\pgfpathlineto{\pgfqpoint{5.936941in}{3.954955in}}%
\pgfpathlineto{\pgfqpoint{5.943935in}{3.987436in}}%
\pgfpathlineto{\pgfqpoint{5.971910in}{4.052396in}}%
\pgfpathlineto{\pgfqpoint{5.978904in}{4.084877in}}%
\pgfpathlineto{\pgfqpoint{5.992891in}{4.084877in}}%
\pgfpathlineto{\pgfqpoint{5.999885in}{4.117357in}}%
\pgfpathlineto{\pgfqpoint{6.006879in}{4.117357in}}%
\pgfpathlineto{\pgfqpoint{6.013873in}{4.149837in}}%
\pgfpathlineto{\pgfqpoint{6.020867in}{4.149837in}}%
\pgfpathlineto{\pgfqpoint{6.027860in}{4.166078in}}%
\pgfpathlineto{\pgfqpoint{6.034854in}{4.166078in}}%
\pgfpathlineto{\pgfqpoint{6.055836in}{4.214798in}}%
\pgfpathlineto{\pgfqpoint{6.069823in}{4.214798in}}%
\pgfpathlineto{\pgfqpoint{6.076817in}{4.247279in}}%
\pgfpathlineto{\pgfqpoint{6.083811in}{4.247279in}}%
\pgfpathlineto{\pgfqpoint{6.090805in}{4.263519in}}%
\pgfpathlineto{\pgfqpoint{6.097798in}{4.263519in}}%
\pgfpathlineto{\pgfqpoint{6.104792in}{4.279759in}}%
\pgfpathlineto{\pgfqpoint{6.125773in}{4.279759in}}%
\pgfpathlineto{\pgfqpoint{6.132767in}{4.312239in}}%
\pgfpathlineto{\pgfqpoint{6.139761in}{4.312239in}}%
\pgfpathlineto{\pgfqpoint{6.146755in}{4.328479in}}%
\pgfpathlineto{\pgfqpoint{6.153749in}{4.328479in}}%
\pgfpathlineto{\pgfqpoint{6.160742in}{4.295999in}}%
\pgfpathlineto{\pgfqpoint{6.181724in}{4.247279in}}%
\pgfpathlineto{\pgfqpoint{6.181724in}{4.247279in}}%
\pgfusepath{stroke}%
\end{pgfscope}%
\begin{pgfscope}%
\pgfpathrectangle{\pgfqpoint{4.720371in}{3.638271in}}{\pgfqpoint{1.530941in}{0.893210in}}%
\pgfusepath{clip}%
\pgfsetrectcap%
\pgfsetroundjoin%
\pgfsetlinewidth{1.505625pt}%
\definecolor{currentstroke}{rgb}{0.839216,0.152941,0.156863}%
\pgfsetstrokecolor{currentstroke}%
\pgfsetdash{}{0pt}%
\pgfpathmoveto{\pgfqpoint{0.000000in}{0.000000in}}%
\pgfusepath{stroke}%
\end{pgfscope}%
\begin{pgfscope}%
\pgfsetrectcap%
\pgfsetmiterjoin%
\pgfsetlinewidth{0.803000pt}%
\definecolor{currentstroke}{rgb}{0.000000,0.000000,0.000000}%
\pgfsetstrokecolor{currentstroke}%
\pgfsetdash{}{0pt}%
\pgfpathmoveto{\pgfqpoint{4.720371in}{3.638271in}}%
\pgfpathlineto{\pgfqpoint{4.720371in}{4.531482in}}%
\pgfusepath{stroke}%
\end{pgfscope}%
\begin{pgfscope}%
\pgfsetrectcap%
\pgfsetmiterjoin%
\pgfsetlinewidth{0.803000pt}%
\definecolor{currentstroke}{rgb}{0.000000,0.000000,0.000000}%
\pgfsetstrokecolor{currentstroke}%
\pgfsetdash{}{0pt}%
\pgfpathmoveto{\pgfqpoint{6.251312in}{3.638271in}}%
\pgfpathlineto{\pgfqpoint{6.251312in}{4.531482in}}%
\pgfusepath{stroke}%
\end{pgfscope}%
\begin{pgfscope}%
\pgfsetrectcap%
\pgfsetmiterjoin%
\pgfsetlinewidth{0.803000pt}%
\definecolor{currentstroke}{rgb}{0.000000,0.000000,0.000000}%
\pgfsetstrokecolor{currentstroke}%
\pgfsetdash{}{0pt}%
\pgfpathmoveto{\pgfqpoint{4.720371in}{3.638271in}}%
\pgfpathlineto{\pgfqpoint{6.251312in}{3.638271in}}%
\pgfusepath{stroke}%
\end{pgfscope}%
\begin{pgfscope}%
\pgfsetrectcap%
\pgfsetmiterjoin%
\pgfsetlinewidth{0.803000pt}%
\definecolor{currentstroke}{rgb}{0.000000,0.000000,0.000000}%
\pgfsetstrokecolor{currentstroke}%
\pgfsetdash{}{0pt}%
\pgfpathmoveto{\pgfqpoint{4.720371in}{4.531482in}}%
\pgfpathlineto{\pgfqpoint{6.251312in}{4.531482in}}%
\pgfusepath{stroke}%
\end{pgfscope}%
\begin{pgfscope}%
\definecolor{textcolor}{rgb}{0.000000,0.000000,0.000000}%
\pgfsetstrokecolor{textcolor}%
\pgfsetfillcolor{textcolor}%
\pgftext[x=5.485841in,y=4.614815in,,base]{\color{textcolor}\rmfamily\fontsize{12.000000}{14.400000}\selectfont 2}%
\end{pgfscope}%
\begin{pgfscope}%
\pgfsetbuttcap%
\pgfsetmiterjoin%
\definecolor{currentfill}{rgb}{1.000000,1.000000,1.000000}%
\pgfsetfillcolor{currentfill}%
\pgfsetlinewidth{0.000000pt}%
\definecolor{currentstroke}{rgb}{0.000000,0.000000,0.000000}%
\pgfsetstrokecolor{currentstroke}%
\pgfsetstrokeopacity{0.000000}%
\pgfsetdash{}{0pt}%
\pgfpathmoveto{\pgfqpoint{0.500000in}{2.061420in}}%
\pgfpathlineto{\pgfqpoint{2.030942in}{2.061420in}}%
\pgfpathlineto{\pgfqpoint{2.030942in}{2.954630in}}%
\pgfpathlineto{\pgfqpoint{0.500000in}{2.954630in}}%
\pgfpathclose%
\pgfusepath{fill}%
\end{pgfscope}%
\begin{pgfscope}%
\pgfsetbuttcap%
\pgfsetroundjoin%
\definecolor{currentfill}{rgb}{0.000000,0.000000,0.000000}%
\pgfsetfillcolor{currentfill}%
\pgfsetlinewidth{0.803000pt}%
\definecolor{currentstroke}{rgb}{0.000000,0.000000,0.000000}%
\pgfsetstrokecolor{currentstroke}%
\pgfsetdash{}{0pt}%
\pgfsys@defobject{currentmarker}{\pgfqpoint{0.000000in}{-0.048611in}}{\pgfqpoint{0.000000in}{0.000000in}}{%
\pgfpathmoveto{\pgfqpoint{0.000000in}{0.000000in}}%
\pgfpathlineto{\pgfqpoint{0.000000in}{-0.048611in}}%
\pgfusepath{stroke,fill}%
}%
\begin{pgfscope}%
\pgfsys@transformshift{0.569589in}{2.061420in}%
\pgfsys@useobject{currentmarker}{}%
\end{pgfscope}%
\end{pgfscope}%
\begin{pgfscope}%
\definecolor{textcolor}{rgb}{0.000000,0.000000,0.000000}%
\pgfsetstrokecolor{textcolor}%
\pgfsetfillcolor{textcolor}%
\pgftext[x=0.569589in,y=1.964197in,,top]{\color{textcolor}\rmfamily\fontsize{10.000000}{12.000000}\selectfont \(\displaystyle {0}\)}%
\end{pgfscope}%
\begin{pgfscope}%
\pgfsetbuttcap%
\pgfsetroundjoin%
\definecolor{currentfill}{rgb}{0.000000,0.000000,0.000000}%
\pgfsetfillcolor{currentfill}%
\pgfsetlinewidth{0.803000pt}%
\definecolor{currentstroke}{rgb}{0.000000,0.000000,0.000000}%
\pgfsetstrokecolor{currentstroke}%
\pgfsetdash{}{0pt}%
\pgfsys@defobject{currentmarker}{\pgfqpoint{0.000000in}{-0.048611in}}{\pgfqpoint{0.000000in}{0.000000in}}{%
\pgfpathmoveto{\pgfqpoint{0.000000in}{0.000000in}}%
\pgfpathlineto{\pgfqpoint{0.000000in}{-0.048611in}}%
\pgfusepath{stroke,fill}%
}%
\begin{pgfscope}%
\pgfsys@transformshift{1.268968in}{2.061420in}%
\pgfsys@useobject{currentmarker}{}%
\end{pgfscope}%
\end{pgfscope}%
\begin{pgfscope}%
\definecolor{textcolor}{rgb}{0.000000,0.000000,0.000000}%
\pgfsetstrokecolor{textcolor}%
\pgfsetfillcolor{textcolor}%
\pgftext[x=1.268968in,y=1.964197in,,top]{\color{textcolor}\rmfamily\fontsize{10.000000}{12.000000}\selectfont \(\displaystyle {2000}\)}%
\end{pgfscope}%
\begin{pgfscope}%
\pgfsetbuttcap%
\pgfsetroundjoin%
\definecolor{currentfill}{rgb}{0.000000,0.000000,0.000000}%
\pgfsetfillcolor{currentfill}%
\pgfsetlinewidth{0.803000pt}%
\definecolor{currentstroke}{rgb}{0.000000,0.000000,0.000000}%
\pgfsetstrokecolor{currentstroke}%
\pgfsetdash{}{0pt}%
\pgfsys@defobject{currentmarker}{\pgfqpoint{0.000000in}{-0.048611in}}{\pgfqpoint{0.000000in}{0.000000in}}{%
\pgfpathmoveto{\pgfqpoint{0.000000in}{0.000000in}}%
\pgfpathlineto{\pgfqpoint{0.000000in}{-0.048611in}}%
\pgfusepath{stroke,fill}%
}%
\begin{pgfscope}%
\pgfsys@transformshift{1.968347in}{2.061420in}%
\pgfsys@useobject{currentmarker}{}%
\end{pgfscope}%
\end{pgfscope}%
\begin{pgfscope}%
\definecolor{textcolor}{rgb}{0.000000,0.000000,0.000000}%
\pgfsetstrokecolor{textcolor}%
\pgfsetfillcolor{textcolor}%
\pgftext[x=1.968347in,y=1.964197in,,top]{\color{textcolor}\rmfamily\fontsize{10.000000}{12.000000}\selectfont \(\displaystyle {4000}\)}%
\end{pgfscope}%
\begin{pgfscope}%
\definecolor{textcolor}{rgb}{0.000000,0.000000,0.000000}%
\pgfsetstrokecolor{textcolor}%
\pgfsetfillcolor{textcolor}%
\pgftext[x=1.265471in,y=1.785185in,,top]{\color{textcolor}\rmfamily\fontsize{10.000000}{12.000000}\selectfont Czas [\(\displaystyle \mu s\)] }%
\end{pgfscope}%
\begin{pgfscope}%
\pgfsetbuttcap%
\pgfsetroundjoin%
\definecolor{currentfill}{rgb}{0.000000,0.000000,0.000000}%
\pgfsetfillcolor{currentfill}%
\pgfsetlinewidth{0.803000pt}%
\definecolor{currentstroke}{rgb}{0.000000,0.000000,0.000000}%
\pgfsetstrokecolor{currentstroke}%
\pgfsetdash{}{0pt}%
\pgfsys@defobject{currentmarker}{\pgfqpoint{-0.048611in}{0.000000in}}{\pgfqpoint{-0.000000in}{0.000000in}}{%
\pgfpathmoveto{\pgfqpoint{-0.000000in}{0.000000in}}%
\pgfpathlineto{\pgfqpoint{-0.048611in}{0.000000in}}%
\pgfusepath{stroke,fill}%
}%
\begin{pgfscope}%
\pgfsys@transformshift{0.500000in}{2.102020in}%
\pgfsys@useobject{currentmarker}{}%
\end{pgfscope}%
\end{pgfscope}%
\begin{pgfscope}%
\definecolor{textcolor}{rgb}{0.000000,0.000000,0.000000}%
\pgfsetstrokecolor{textcolor}%
\pgfsetfillcolor{textcolor}%
\pgftext[x=0.333333in, y=2.053795in, left, base]{\color{textcolor}\rmfamily\fontsize{10.000000}{12.000000}\selectfont \(\displaystyle {0}\)}%
\end{pgfscope}%
\begin{pgfscope}%
\pgfsetbuttcap%
\pgfsetroundjoin%
\definecolor{currentfill}{rgb}{0.000000,0.000000,0.000000}%
\pgfsetfillcolor{currentfill}%
\pgfsetlinewidth{0.803000pt}%
\definecolor{currentstroke}{rgb}{0.000000,0.000000,0.000000}%
\pgfsetstrokecolor{currentstroke}%
\pgfsetdash{}{0pt}%
\pgfsys@defobject{currentmarker}{\pgfqpoint{-0.048611in}{0.000000in}}{\pgfqpoint{-0.000000in}{0.000000in}}{%
\pgfpathmoveto{\pgfqpoint{-0.000000in}{0.000000in}}%
\pgfpathlineto{\pgfqpoint{-0.048611in}{0.000000in}}%
\pgfusepath{stroke,fill}%
}%
\begin{pgfscope}%
\pgfsys@transformshift{0.500000in}{2.426824in}%
\pgfsys@useobject{currentmarker}{}%
\end{pgfscope}%
\end{pgfscope}%
\begin{pgfscope}%
\definecolor{textcolor}{rgb}{0.000000,0.000000,0.000000}%
\pgfsetstrokecolor{textcolor}%
\pgfsetfillcolor{textcolor}%
\pgftext[x=0.263889in, y=2.378599in, left, base]{\color{textcolor}\rmfamily\fontsize{10.000000}{12.000000}\selectfont \(\displaystyle {20}\)}%
\end{pgfscope}%
\begin{pgfscope}%
\pgfsetbuttcap%
\pgfsetroundjoin%
\definecolor{currentfill}{rgb}{0.000000,0.000000,0.000000}%
\pgfsetfillcolor{currentfill}%
\pgfsetlinewidth{0.803000pt}%
\definecolor{currentstroke}{rgb}{0.000000,0.000000,0.000000}%
\pgfsetstrokecolor{currentstroke}%
\pgfsetdash{}{0pt}%
\pgfsys@defobject{currentmarker}{\pgfqpoint{-0.048611in}{0.000000in}}{\pgfqpoint{-0.000000in}{0.000000in}}{%
\pgfpathmoveto{\pgfqpoint{-0.000000in}{0.000000in}}%
\pgfpathlineto{\pgfqpoint{-0.048611in}{0.000000in}}%
\pgfusepath{stroke,fill}%
}%
\begin{pgfscope}%
\pgfsys@transformshift{0.500000in}{2.751628in}%
\pgfsys@useobject{currentmarker}{}%
\end{pgfscope}%
\end{pgfscope}%
\begin{pgfscope}%
\definecolor{textcolor}{rgb}{0.000000,0.000000,0.000000}%
\pgfsetstrokecolor{textcolor}%
\pgfsetfillcolor{textcolor}%
\pgftext[x=0.263889in, y=2.703402in, left, base]{\color{textcolor}\rmfamily\fontsize{10.000000}{12.000000}\selectfont \(\displaystyle {40}\)}%
\end{pgfscope}%
\begin{pgfscope}%
\definecolor{textcolor}{rgb}{0.000000,0.000000,0.000000}%
\pgfsetstrokecolor{textcolor}%
\pgfsetfillcolor{textcolor}%
\pgftext[x=0.208333in,y=2.508025in,,bottom,rotate=90.000000]{\color{textcolor}\rmfamily\fontsize{10.000000}{12.000000}\selectfont Napiecie [\(\displaystyle V\)] }%
\end{pgfscope}%
\begin{pgfscope}%
\pgfpathrectangle{\pgfqpoint{0.500000in}{2.061420in}}{\pgfqpoint{1.530941in}{0.893210in}}%
\pgfusepath{clip}%
\pgfsetrectcap%
\pgfsetroundjoin%
\pgfsetlinewidth{1.505625pt}%
\definecolor{currentstroke}{rgb}{0.121569,0.466667,0.705882}%
\pgfsetstrokecolor{currentstroke}%
\pgfsetdash{}{0pt}%
\pgfpathmoveto{\pgfqpoint{0.569589in}{2.914029in}}%
\pgfpathlineto{\pgfqpoint{0.576582in}{2.897789in}}%
\pgfpathlineto{\pgfqpoint{0.583576in}{2.897789in}}%
\pgfpathlineto{\pgfqpoint{0.590570in}{2.914029in}}%
\pgfpathlineto{\pgfqpoint{0.618545in}{2.914029in}}%
\pgfpathlineto{\pgfqpoint{0.625539in}{2.134500in}}%
\pgfpathlineto{\pgfqpoint{0.639527in}{2.134500in}}%
\pgfpathlineto{\pgfqpoint{0.646520in}{2.118260in}}%
\pgfpathlineto{\pgfqpoint{0.653514in}{2.134500in}}%
\pgfpathlineto{\pgfqpoint{0.660508in}{2.118260in}}%
\pgfpathlineto{\pgfqpoint{0.674496in}{2.118260in}}%
\pgfpathlineto{\pgfqpoint{0.681489in}{2.134500in}}%
\pgfpathlineto{\pgfqpoint{0.688483in}{2.118260in}}%
\pgfpathlineto{\pgfqpoint{0.709465in}{2.118260in}}%
\pgfpathlineto{\pgfqpoint{0.716458in}{2.102020in}}%
\pgfpathlineto{\pgfqpoint{0.723452in}{2.118260in}}%
\pgfpathlineto{\pgfqpoint{0.730446in}{2.102020in}}%
\pgfpathlineto{\pgfqpoint{0.737440in}{2.102020in}}%
\pgfpathlineto{\pgfqpoint{0.744433in}{2.118260in}}%
\pgfpathlineto{\pgfqpoint{0.758421in}{2.118260in}}%
\pgfpathlineto{\pgfqpoint{0.765415in}{2.102020in}}%
\pgfpathlineto{\pgfqpoint{0.772409in}{2.118260in}}%
\pgfpathlineto{\pgfqpoint{0.779402in}{2.102020in}}%
\pgfpathlineto{\pgfqpoint{0.786396in}{2.118260in}}%
\pgfpathlineto{\pgfqpoint{0.793390in}{2.102020in}}%
\pgfpathlineto{\pgfqpoint{0.968235in}{2.102020in}}%
\pgfpathlineto{\pgfqpoint{0.975229in}{2.865309in}}%
\pgfpathlineto{\pgfqpoint{0.989216in}{2.865309in}}%
\pgfpathlineto{\pgfqpoint{0.996210in}{2.881549in}}%
\pgfpathlineto{\pgfqpoint{1.066148in}{2.881549in}}%
\pgfpathlineto{\pgfqpoint{1.073142in}{2.897789in}}%
\pgfpathlineto{\pgfqpoint{1.080136in}{2.897789in}}%
\pgfpathlineto{\pgfqpoint{1.087129in}{2.881549in}}%
\pgfpathlineto{\pgfqpoint{1.094123in}{2.881549in}}%
\pgfpathlineto{\pgfqpoint{1.101117in}{2.897789in}}%
\pgfpathlineto{\pgfqpoint{1.108111in}{2.881549in}}%
\pgfpathlineto{\pgfqpoint{1.115105in}{2.897789in}}%
\pgfpathlineto{\pgfqpoint{1.150073in}{2.897789in}}%
\pgfpathlineto{\pgfqpoint{1.157067in}{2.881549in}}%
\pgfpathlineto{\pgfqpoint{1.164061in}{2.897789in}}%
\pgfpathlineto{\pgfqpoint{1.178049in}{2.897789in}}%
\pgfpathlineto{\pgfqpoint{1.185042in}{2.881549in}}%
\pgfpathlineto{\pgfqpoint{1.192036in}{2.897789in}}%
\pgfpathlineto{\pgfqpoint{1.220011in}{2.897789in}}%
\pgfpathlineto{\pgfqpoint{1.227005in}{2.914029in}}%
\pgfpathlineto{\pgfqpoint{1.233999in}{2.897789in}}%
\pgfpathlineto{\pgfqpoint{1.247987in}{2.897789in}}%
\pgfpathlineto{\pgfqpoint{1.254980in}{2.914029in}}%
\pgfpathlineto{\pgfqpoint{1.261974in}{2.914029in}}%
\pgfpathlineto{\pgfqpoint{1.268968in}{2.897789in}}%
\pgfpathlineto{\pgfqpoint{1.275962in}{2.914029in}}%
\pgfpathlineto{\pgfqpoint{1.282956in}{2.897789in}}%
\pgfpathlineto{\pgfqpoint{1.289949in}{2.897789in}}%
\pgfpathlineto{\pgfqpoint{1.296943in}{2.914029in}}%
\pgfpathlineto{\pgfqpoint{1.303937in}{2.914029in}}%
\pgfpathlineto{\pgfqpoint{1.310931in}{2.897789in}}%
\pgfpathlineto{\pgfqpoint{1.317925in}{2.897789in}}%
\pgfpathlineto{\pgfqpoint{1.324918in}{2.134500in}}%
\pgfpathlineto{\pgfqpoint{1.352893in}{2.134500in}}%
\pgfpathlineto{\pgfqpoint{1.359887in}{2.118260in}}%
\pgfpathlineto{\pgfqpoint{1.366881in}{2.134500in}}%
\pgfpathlineto{\pgfqpoint{1.373875in}{2.118260in}}%
\pgfpathlineto{\pgfqpoint{1.429825in}{2.118260in}}%
\pgfpathlineto{\pgfqpoint{1.436819in}{2.102020in}}%
\pgfpathlineto{\pgfqpoint{1.443813in}{2.118260in}}%
\pgfpathlineto{\pgfqpoint{1.450807in}{2.102020in}}%
\pgfpathlineto{\pgfqpoint{1.457800in}{2.118260in}}%
\pgfpathlineto{\pgfqpoint{1.464794in}{2.102020in}}%
\pgfpathlineto{\pgfqpoint{1.485776in}{2.102020in}}%
\pgfpathlineto{\pgfqpoint{1.492769in}{2.118260in}}%
\pgfpathlineto{\pgfqpoint{1.499763in}{2.102020in}}%
\pgfpathlineto{\pgfqpoint{1.527738in}{2.102020in}}%
\pgfpathlineto{\pgfqpoint{1.534732in}{2.118260in}}%
\pgfpathlineto{\pgfqpoint{1.541726in}{2.102020in}}%
\pgfpathlineto{\pgfqpoint{1.576695in}{2.102020in}}%
\pgfpathlineto{\pgfqpoint{1.583689in}{2.118260in}}%
\pgfpathlineto{\pgfqpoint{1.590682in}{2.102020in}}%
\pgfpathlineto{\pgfqpoint{1.667614in}{2.102020in}}%
\pgfpathlineto{\pgfqpoint{1.674608in}{2.865309in}}%
\pgfpathlineto{\pgfqpoint{1.688596in}{2.865309in}}%
\pgfpathlineto{\pgfqpoint{1.695589in}{2.881549in}}%
\pgfpathlineto{\pgfqpoint{1.716571in}{2.881549in}}%
\pgfpathlineto{\pgfqpoint{1.723565in}{2.865309in}}%
\pgfpathlineto{\pgfqpoint{1.730558in}{2.881549in}}%
\pgfpathlineto{\pgfqpoint{1.758533in}{2.881549in}}%
\pgfpathlineto{\pgfqpoint{1.765527in}{2.897789in}}%
\pgfpathlineto{\pgfqpoint{1.772521in}{2.881549in}}%
\pgfpathlineto{\pgfqpoint{1.779515in}{2.881549in}}%
\pgfpathlineto{\pgfqpoint{1.786509in}{2.897789in}}%
\pgfpathlineto{\pgfqpoint{1.793502in}{2.897789in}}%
\pgfpathlineto{\pgfqpoint{1.800496in}{2.881549in}}%
\pgfpathlineto{\pgfqpoint{1.807490in}{2.881549in}}%
\pgfpathlineto{\pgfqpoint{1.814484in}{2.897789in}}%
\pgfpathlineto{\pgfqpoint{1.821478in}{2.881549in}}%
\pgfpathlineto{\pgfqpoint{1.835465in}{2.914029in}}%
\pgfpathlineto{\pgfqpoint{1.842459in}{2.897789in}}%
\pgfpathlineto{\pgfqpoint{1.898409in}{2.897789in}}%
\pgfpathlineto{\pgfqpoint{1.905403in}{2.914029in}}%
\pgfpathlineto{\pgfqpoint{1.912397in}{2.914029in}}%
\pgfpathlineto{\pgfqpoint{1.919391in}{2.897789in}}%
\pgfpathlineto{\pgfqpoint{1.926384in}{2.897789in}}%
\pgfpathlineto{\pgfqpoint{1.933378in}{2.914029in}}%
\pgfpathlineto{\pgfqpoint{1.940372in}{2.897789in}}%
\pgfpathlineto{\pgfqpoint{1.961353in}{2.897789in}}%
\pgfpathlineto{\pgfqpoint{1.961353in}{2.897789in}}%
\pgfusepath{stroke}%
\end{pgfscope}%
\begin{pgfscope}%
\pgfpathrectangle{\pgfqpoint{0.500000in}{2.061420in}}{\pgfqpoint{1.530941in}{0.893210in}}%
\pgfusepath{clip}%
\pgfsetrectcap%
\pgfsetroundjoin%
\pgfsetlinewidth{1.505625pt}%
\definecolor{currentstroke}{rgb}{1.000000,0.498039,0.054902}%
\pgfsetstrokecolor{currentstroke}%
\pgfsetdash{}{0pt}%
\pgfpathmoveto{\pgfqpoint{0.000000in}{0.000000in}}%
\pgfusepath{stroke}%
\end{pgfscope}%
\begin{pgfscope}%
\pgfpathrectangle{\pgfqpoint{0.500000in}{2.061420in}}{\pgfqpoint{1.530941in}{0.893210in}}%
\pgfusepath{clip}%
\pgfsetrectcap%
\pgfsetroundjoin%
\pgfsetlinewidth{1.505625pt}%
\definecolor{currentstroke}{rgb}{0.172549,0.627451,0.172549}%
\pgfsetstrokecolor{currentstroke}%
\pgfsetdash{}{0pt}%
\pgfpathmoveto{\pgfqpoint{0.569589in}{2.767868in}}%
\pgfpathlineto{\pgfqpoint{0.576582in}{2.767868in}}%
\pgfpathlineto{\pgfqpoint{0.583576in}{2.784108in}}%
\pgfpathlineto{\pgfqpoint{0.590570in}{2.767868in}}%
\pgfpathlineto{\pgfqpoint{0.597564in}{2.784108in}}%
\pgfpathlineto{\pgfqpoint{0.604558in}{2.784108in}}%
\pgfpathlineto{\pgfqpoint{0.611551in}{2.800348in}}%
\pgfpathlineto{\pgfqpoint{0.625539in}{2.800348in}}%
\pgfpathlineto{\pgfqpoint{0.632533in}{2.767868in}}%
\pgfpathlineto{\pgfqpoint{0.639527in}{2.751628in}}%
\pgfpathlineto{\pgfqpoint{0.646520in}{2.719147in}}%
\pgfpathlineto{\pgfqpoint{0.653514in}{2.702907in}}%
\pgfpathlineto{\pgfqpoint{0.660508in}{2.670427in}}%
\pgfpathlineto{\pgfqpoint{0.667502in}{2.654186in}}%
\pgfpathlineto{\pgfqpoint{0.674496in}{2.621706in}}%
\pgfpathlineto{\pgfqpoint{0.681489in}{2.621706in}}%
\pgfpathlineto{\pgfqpoint{0.688483in}{2.589226in}}%
\pgfpathlineto{\pgfqpoint{0.716458in}{2.524265in}}%
\pgfpathlineto{\pgfqpoint{0.723452in}{2.491785in}}%
\pgfpathlineto{\pgfqpoint{0.730446in}{2.491785in}}%
\pgfpathlineto{\pgfqpoint{0.737440in}{2.475544in}}%
\pgfpathlineto{\pgfqpoint{0.744433in}{2.443064in}}%
\pgfpathlineto{\pgfqpoint{0.758421in}{2.443064in}}%
\pgfpathlineto{\pgfqpoint{0.786396in}{2.378103in}}%
\pgfpathlineto{\pgfqpoint{0.800384in}{2.378103in}}%
\pgfpathlineto{\pgfqpoint{0.807378in}{2.361863in}}%
\pgfpathlineto{\pgfqpoint{0.814371in}{2.361863in}}%
\pgfpathlineto{\pgfqpoint{0.821365in}{2.329383in}}%
\pgfpathlineto{\pgfqpoint{0.828359in}{2.329383in}}%
\pgfpathlineto{\pgfqpoint{0.835353in}{2.313143in}}%
\pgfpathlineto{\pgfqpoint{0.842347in}{2.313143in}}%
\pgfpathlineto{\pgfqpoint{0.849340in}{2.296902in}}%
\pgfpathlineto{\pgfqpoint{0.856334in}{2.296902in}}%
\pgfpathlineto{\pgfqpoint{0.863328in}{2.280662in}}%
\pgfpathlineto{\pgfqpoint{0.877316in}{2.280662in}}%
\pgfpathlineto{\pgfqpoint{0.891303in}{2.248182in}}%
\pgfpathlineto{\pgfqpoint{0.919278in}{2.248182in}}%
\pgfpathlineto{\pgfqpoint{0.926272in}{2.231942in}}%
\pgfpathlineto{\pgfqpoint{0.947253in}{2.231942in}}%
\pgfpathlineto{\pgfqpoint{0.961241in}{2.199461in}}%
\pgfpathlineto{\pgfqpoint{0.968235in}{2.215701in}}%
\pgfpathlineto{\pgfqpoint{0.975229in}{2.199461in}}%
\pgfpathlineto{\pgfqpoint{0.982222in}{2.231942in}}%
\pgfpathlineto{\pgfqpoint{0.989216in}{2.248182in}}%
\pgfpathlineto{\pgfqpoint{0.996210in}{2.280662in}}%
\pgfpathlineto{\pgfqpoint{1.010198in}{2.313143in}}%
\pgfpathlineto{\pgfqpoint{1.017191in}{2.361863in}}%
\pgfpathlineto{\pgfqpoint{1.024185in}{2.378103in}}%
\pgfpathlineto{\pgfqpoint{1.031179in}{2.378103in}}%
\pgfpathlineto{\pgfqpoint{1.045167in}{2.443064in}}%
\pgfpathlineto{\pgfqpoint{1.052160in}{2.443064in}}%
\pgfpathlineto{\pgfqpoint{1.066148in}{2.475544in}}%
\pgfpathlineto{\pgfqpoint{1.073142in}{2.508025in}}%
\pgfpathlineto{\pgfqpoint{1.087129in}{2.508025in}}%
\pgfpathlineto{\pgfqpoint{1.094123in}{2.540505in}}%
\pgfpathlineto{\pgfqpoint{1.108111in}{2.572986in}}%
\pgfpathlineto{\pgfqpoint{1.115105in}{2.572986in}}%
\pgfpathlineto{\pgfqpoint{1.143080in}{2.637946in}}%
\pgfpathlineto{\pgfqpoint{1.164061in}{2.637946in}}%
\pgfpathlineto{\pgfqpoint{1.192036in}{2.702907in}}%
\pgfpathlineto{\pgfqpoint{1.206024in}{2.702907in}}%
\pgfpathlineto{\pgfqpoint{1.213018in}{2.719147in}}%
\pgfpathlineto{\pgfqpoint{1.227005in}{2.719147in}}%
\pgfpathlineto{\pgfqpoint{1.233999in}{2.735387in}}%
\pgfpathlineto{\pgfqpoint{1.240993in}{2.735387in}}%
\pgfpathlineto{\pgfqpoint{1.247987in}{2.751628in}}%
\pgfpathlineto{\pgfqpoint{1.254980in}{2.751628in}}%
\pgfpathlineto{\pgfqpoint{1.261974in}{2.767868in}}%
\pgfpathlineto{\pgfqpoint{1.282956in}{2.767868in}}%
\pgfpathlineto{\pgfqpoint{1.289949in}{2.784108in}}%
\pgfpathlineto{\pgfqpoint{1.296943in}{2.767868in}}%
\pgfpathlineto{\pgfqpoint{1.303937in}{2.784108in}}%
\pgfpathlineto{\pgfqpoint{1.310931in}{2.784108in}}%
\pgfpathlineto{\pgfqpoint{1.317925in}{2.800348in}}%
\pgfpathlineto{\pgfqpoint{1.324918in}{2.800348in}}%
\pgfpathlineto{\pgfqpoint{1.331912in}{2.767868in}}%
\pgfpathlineto{\pgfqpoint{1.338906in}{2.751628in}}%
\pgfpathlineto{\pgfqpoint{1.345900in}{2.702907in}}%
\pgfpathlineto{\pgfqpoint{1.352893in}{2.702907in}}%
\pgfpathlineto{\pgfqpoint{1.366881in}{2.637946in}}%
\pgfpathlineto{\pgfqpoint{1.373875in}{2.637946in}}%
\pgfpathlineto{\pgfqpoint{1.387862in}{2.572986in}}%
\pgfpathlineto{\pgfqpoint{1.401850in}{2.572986in}}%
\pgfpathlineto{\pgfqpoint{1.415838in}{2.508025in}}%
\pgfpathlineto{\pgfqpoint{1.422831in}{2.508025in}}%
\pgfpathlineto{\pgfqpoint{1.450807in}{2.443064in}}%
\pgfpathlineto{\pgfqpoint{1.457800in}{2.443064in}}%
\pgfpathlineto{\pgfqpoint{1.464794in}{2.426824in}}%
\pgfpathlineto{\pgfqpoint{1.471788in}{2.426824in}}%
\pgfpathlineto{\pgfqpoint{1.478782in}{2.410584in}}%
\pgfpathlineto{\pgfqpoint{1.485776in}{2.378103in}}%
\pgfpathlineto{\pgfqpoint{1.499763in}{2.378103in}}%
\pgfpathlineto{\pgfqpoint{1.527738in}{2.313143in}}%
\pgfpathlineto{\pgfqpoint{1.548720in}{2.313143in}}%
\pgfpathlineto{\pgfqpoint{1.562707in}{2.280662in}}%
\pgfpathlineto{\pgfqpoint{1.576695in}{2.280662in}}%
\pgfpathlineto{\pgfqpoint{1.590682in}{2.248182in}}%
\pgfpathlineto{\pgfqpoint{1.611664in}{2.248182in}}%
\pgfpathlineto{\pgfqpoint{1.618658in}{2.231942in}}%
\pgfpathlineto{\pgfqpoint{1.625651in}{2.248182in}}%
\pgfpathlineto{\pgfqpoint{1.639639in}{2.215701in}}%
\pgfpathlineto{\pgfqpoint{1.660620in}{2.215701in}}%
\pgfpathlineto{\pgfqpoint{1.667614in}{2.199461in}}%
\pgfpathlineto{\pgfqpoint{1.674608in}{2.199461in}}%
\pgfpathlineto{\pgfqpoint{1.681602in}{2.231942in}}%
\pgfpathlineto{\pgfqpoint{1.688596in}{2.248182in}}%
\pgfpathlineto{\pgfqpoint{1.702583in}{2.313143in}}%
\pgfpathlineto{\pgfqpoint{1.709577in}{2.313143in}}%
\pgfpathlineto{\pgfqpoint{1.723565in}{2.378103in}}%
\pgfpathlineto{\pgfqpoint{1.730558in}{2.378103in}}%
\pgfpathlineto{\pgfqpoint{1.737552in}{2.410584in}}%
\pgfpathlineto{\pgfqpoint{1.800496in}{2.556745in}}%
\pgfpathlineto{\pgfqpoint{1.807490in}{2.556745in}}%
\pgfpathlineto{\pgfqpoint{1.814484in}{2.572986in}}%
\pgfpathlineto{\pgfqpoint{1.821478in}{2.572986in}}%
\pgfpathlineto{\pgfqpoint{1.835465in}{2.605466in}}%
\pgfpathlineto{\pgfqpoint{1.842459in}{2.605466in}}%
\pgfpathlineto{\pgfqpoint{1.849453in}{2.637946in}}%
\pgfpathlineto{\pgfqpoint{1.856447in}{2.637946in}}%
\pgfpathlineto{\pgfqpoint{1.863440in}{2.670427in}}%
\pgfpathlineto{\pgfqpoint{1.870434in}{2.670427in}}%
\pgfpathlineto{\pgfqpoint{1.877428in}{2.686667in}}%
\pgfpathlineto{\pgfqpoint{1.884422in}{2.686667in}}%
\pgfpathlineto{\pgfqpoint{1.891416in}{2.702907in}}%
\pgfpathlineto{\pgfqpoint{1.912397in}{2.702907in}}%
\pgfpathlineto{\pgfqpoint{1.919391in}{2.719147in}}%
\pgfpathlineto{\pgfqpoint{1.926384in}{2.719147in}}%
\pgfpathlineto{\pgfqpoint{1.940372in}{2.751628in}}%
\pgfpathlineto{\pgfqpoint{1.947366in}{2.751628in}}%
\pgfpathlineto{\pgfqpoint{1.954360in}{2.767868in}}%
\pgfpathlineto{\pgfqpoint{1.961353in}{2.767868in}}%
\pgfpathlineto{\pgfqpoint{1.961353in}{2.767868in}}%
\pgfusepath{stroke}%
\end{pgfscope}%
\begin{pgfscope}%
\pgfpathrectangle{\pgfqpoint{0.500000in}{2.061420in}}{\pgfqpoint{1.530941in}{0.893210in}}%
\pgfusepath{clip}%
\pgfsetrectcap%
\pgfsetroundjoin%
\pgfsetlinewidth{1.505625pt}%
\definecolor{currentstroke}{rgb}{0.839216,0.152941,0.156863}%
\pgfsetstrokecolor{currentstroke}%
\pgfsetdash{}{0pt}%
\pgfpathmoveto{\pgfqpoint{0.000000in}{0.000000in}}%
\pgfusepath{stroke}%
\end{pgfscope}%
\begin{pgfscope}%
\pgfsetrectcap%
\pgfsetmiterjoin%
\pgfsetlinewidth{0.803000pt}%
\definecolor{currentstroke}{rgb}{0.000000,0.000000,0.000000}%
\pgfsetstrokecolor{currentstroke}%
\pgfsetdash{}{0pt}%
\pgfpathmoveto{\pgfqpoint{0.500000in}{2.061420in}}%
\pgfpathlineto{\pgfqpoint{0.500000in}{2.954630in}}%
\pgfusepath{stroke}%
\end{pgfscope}%
\begin{pgfscope}%
\pgfsetrectcap%
\pgfsetmiterjoin%
\pgfsetlinewidth{0.803000pt}%
\definecolor{currentstroke}{rgb}{0.000000,0.000000,0.000000}%
\pgfsetstrokecolor{currentstroke}%
\pgfsetdash{}{0pt}%
\pgfpathmoveto{\pgfqpoint{2.030942in}{2.061420in}}%
\pgfpathlineto{\pgfqpoint{2.030942in}{2.954630in}}%
\pgfusepath{stroke}%
\end{pgfscope}%
\begin{pgfscope}%
\pgfsetrectcap%
\pgfsetmiterjoin%
\pgfsetlinewidth{0.803000pt}%
\definecolor{currentstroke}{rgb}{0.000000,0.000000,0.000000}%
\pgfsetstrokecolor{currentstroke}%
\pgfsetdash{}{0pt}%
\pgfpathmoveto{\pgfqpoint{0.500000in}{2.061420in}}%
\pgfpathlineto{\pgfqpoint{2.030942in}{2.061420in}}%
\pgfusepath{stroke}%
\end{pgfscope}%
\begin{pgfscope}%
\pgfsetrectcap%
\pgfsetmiterjoin%
\pgfsetlinewidth{0.803000pt}%
\definecolor{currentstroke}{rgb}{0.000000,0.000000,0.000000}%
\pgfsetstrokecolor{currentstroke}%
\pgfsetdash{}{0pt}%
\pgfpathmoveto{\pgfqpoint{0.500000in}{2.954630in}}%
\pgfpathlineto{\pgfqpoint{2.030942in}{2.954630in}}%
\pgfusepath{stroke}%
\end{pgfscope}%
\begin{pgfscope}%
\definecolor{textcolor}{rgb}{0.000000,0.000000,0.000000}%
\pgfsetstrokecolor{textcolor}%
\pgfsetfillcolor{textcolor}%
\pgftext[x=1.265471in,y=3.037963in,,base]{\color{textcolor}\rmfamily\fontsize{12.000000}{14.400000}\selectfont 3}%
\end{pgfscope}%
\begin{pgfscope}%
\pgfsetbuttcap%
\pgfsetmiterjoin%
\definecolor{currentfill}{rgb}{1.000000,1.000000,1.000000}%
\pgfsetfillcolor{currentfill}%
\pgfsetlinewidth{0.000000pt}%
\definecolor{currentstroke}{rgb}{0.000000,0.000000,0.000000}%
\pgfsetstrokecolor{currentstroke}%
\pgfsetstrokeopacity{0.000000}%
\pgfsetdash{}{0pt}%
\pgfpathmoveto{\pgfqpoint{2.610186in}{2.061420in}}%
\pgfpathlineto{\pgfqpoint{4.141127in}{2.061420in}}%
\pgfpathlineto{\pgfqpoint{4.141127in}{2.954630in}}%
\pgfpathlineto{\pgfqpoint{2.610186in}{2.954630in}}%
\pgfpathclose%
\pgfusepath{fill}%
\end{pgfscope}%
\begin{pgfscope}%
\pgfsetbuttcap%
\pgfsetroundjoin%
\definecolor{currentfill}{rgb}{0.000000,0.000000,0.000000}%
\pgfsetfillcolor{currentfill}%
\pgfsetlinewidth{0.803000pt}%
\definecolor{currentstroke}{rgb}{0.000000,0.000000,0.000000}%
\pgfsetstrokecolor{currentstroke}%
\pgfsetdash{}{0pt}%
\pgfsys@defobject{currentmarker}{\pgfqpoint{0.000000in}{-0.048611in}}{\pgfqpoint{0.000000in}{0.000000in}}{%
\pgfpathmoveto{\pgfqpoint{0.000000in}{0.000000in}}%
\pgfpathlineto{\pgfqpoint{0.000000in}{-0.048611in}}%
\pgfusepath{stroke,fill}%
}%
\begin{pgfscope}%
\pgfsys@transformshift{2.679774in}{2.061420in}%
\pgfsys@useobject{currentmarker}{}%
\end{pgfscope}%
\end{pgfscope}%
\begin{pgfscope}%
\definecolor{textcolor}{rgb}{0.000000,0.000000,0.000000}%
\pgfsetstrokecolor{textcolor}%
\pgfsetfillcolor{textcolor}%
\pgftext[x=2.679774in,y=1.964197in,,top]{\color{textcolor}\rmfamily\fontsize{10.000000}{12.000000}\selectfont \(\displaystyle {0}\)}%
\end{pgfscope}%
\begin{pgfscope}%
\pgfsetbuttcap%
\pgfsetroundjoin%
\definecolor{currentfill}{rgb}{0.000000,0.000000,0.000000}%
\pgfsetfillcolor{currentfill}%
\pgfsetlinewidth{0.803000pt}%
\definecolor{currentstroke}{rgb}{0.000000,0.000000,0.000000}%
\pgfsetstrokecolor{currentstroke}%
\pgfsetdash{}{0pt}%
\pgfsys@defobject{currentmarker}{\pgfqpoint{0.000000in}{-0.048611in}}{\pgfqpoint{0.000000in}{0.000000in}}{%
\pgfpathmoveto{\pgfqpoint{0.000000in}{0.000000in}}%
\pgfpathlineto{\pgfqpoint{0.000000in}{-0.048611in}}%
\pgfusepath{stroke,fill}%
}%
\begin{pgfscope}%
\pgfsys@transformshift{3.379153in}{2.061420in}%
\pgfsys@useobject{currentmarker}{}%
\end{pgfscope}%
\end{pgfscope}%
\begin{pgfscope}%
\definecolor{textcolor}{rgb}{0.000000,0.000000,0.000000}%
\pgfsetstrokecolor{textcolor}%
\pgfsetfillcolor{textcolor}%
\pgftext[x=3.379153in,y=1.964197in,,top]{\color{textcolor}\rmfamily\fontsize{10.000000}{12.000000}\selectfont \(\displaystyle {2000}\)}%
\end{pgfscope}%
\begin{pgfscope}%
\pgfsetbuttcap%
\pgfsetroundjoin%
\definecolor{currentfill}{rgb}{0.000000,0.000000,0.000000}%
\pgfsetfillcolor{currentfill}%
\pgfsetlinewidth{0.803000pt}%
\definecolor{currentstroke}{rgb}{0.000000,0.000000,0.000000}%
\pgfsetstrokecolor{currentstroke}%
\pgfsetdash{}{0pt}%
\pgfsys@defobject{currentmarker}{\pgfqpoint{0.000000in}{-0.048611in}}{\pgfqpoint{0.000000in}{0.000000in}}{%
\pgfpathmoveto{\pgfqpoint{0.000000in}{0.000000in}}%
\pgfpathlineto{\pgfqpoint{0.000000in}{-0.048611in}}%
\pgfusepath{stroke,fill}%
}%
\begin{pgfscope}%
\pgfsys@transformshift{4.078532in}{2.061420in}%
\pgfsys@useobject{currentmarker}{}%
\end{pgfscope}%
\end{pgfscope}%
\begin{pgfscope}%
\definecolor{textcolor}{rgb}{0.000000,0.000000,0.000000}%
\pgfsetstrokecolor{textcolor}%
\pgfsetfillcolor{textcolor}%
\pgftext[x=4.078532in,y=1.964197in,,top]{\color{textcolor}\rmfamily\fontsize{10.000000}{12.000000}\selectfont \(\displaystyle {4000}\)}%
\end{pgfscope}%
\begin{pgfscope}%
\definecolor{textcolor}{rgb}{0.000000,0.000000,0.000000}%
\pgfsetstrokecolor{textcolor}%
\pgfsetfillcolor{textcolor}%
\pgftext[x=3.375656in,y=1.785185in,,top]{\color{textcolor}\rmfamily\fontsize{10.000000}{12.000000}\selectfont Czas [\(\displaystyle \mu s\)] }%
\end{pgfscope}%
\begin{pgfscope}%
\pgfsetbuttcap%
\pgfsetroundjoin%
\definecolor{currentfill}{rgb}{0.000000,0.000000,0.000000}%
\pgfsetfillcolor{currentfill}%
\pgfsetlinewidth{0.803000pt}%
\definecolor{currentstroke}{rgb}{0.000000,0.000000,0.000000}%
\pgfsetstrokecolor{currentstroke}%
\pgfsetdash{}{0pt}%
\pgfsys@defobject{currentmarker}{\pgfqpoint{-0.048611in}{0.000000in}}{\pgfqpoint{-0.000000in}{0.000000in}}{%
\pgfpathmoveto{\pgfqpoint{-0.000000in}{0.000000in}}%
\pgfpathlineto{\pgfqpoint{-0.048611in}{0.000000in}}%
\pgfusepath{stroke,fill}%
}%
\begin{pgfscope}%
\pgfsys@transformshift{2.610186in}{2.117942in}%
\pgfsys@useobject{currentmarker}{}%
\end{pgfscope}%
\end{pgfscope}%
\begin{pgfscope}%
\definecolor{textcolor}{rgb}{0.000000,0.000000,0.000000}%
\pgfsetstrokecolor{textcolor}%
\pgfsetfillcolor{textcolor}%
\pgftext[x=2.443519in, y=2.069717in, left, base]{\color{textcolor}\rmfamily\fontsize{10.000000}{12.000000}\selectfont \(\displaystyle {0}\)}%
\end{pgfscope}%
\begin{pgfscope}%
\pgfsetbuttcap%
\pgfsetroundjoin%
\definecolor{currentfill}{rgb}{0.000000,0.000000,0.000000}%
\pgfsetfillcolor{currentfill}%
\pgfsetlinewidth{0.803000pt}%
\definecolor{currentstroke}{rgb}{0.000000,0.000000,0.000000}%
\pgfsetstrokecolor{currentstroke}%
\pgfsetdash{}{0pt}%
\pgfsys@defobject{currentmarker}{\pgfqpoint{-0.048611in}{0.000000in}}{\pgfqpoint{-0.000000in}{0.000000in}}{%
\pgfpathmoveto{\pgfqpoint{-0.000000in}{0.000000in}}%
\pgfpathlineto{\pgfqpoint{-0.048611in}{0.000000in}}%
\pgfusepath{stroke,fill}%
}%
\begin{pgfscope}%
\pgfsys@transformshift{2.610186in}{2.436377in}%
\pgfsys@useobject{currentmarker}{}%
\end{pgfscope}%
\end{pgfscope}%
\begin{pgfscope}%
\definecolor{textcolor}{rgb}{0.000000,0.000000,0.000000}%
\pgfsetstrokecolor{textcolor}%
\pgfsetfillcolor{textcolor}%
\pgftext[x=2.374074in, y=2.388152in, left, base]{\color{textcolor}\rmfamily\fontsize{10.000000}{12.000000}\selectfont \(\displaystyle {20}\)}%
\end{pgfscope}%
\begin{pgfscope}%
\pgfsetbuttcap%
\pgfsetroundjoin%
\definecolor{currentfill}{rgb}{0.000000,0.000000,0.000000}%
\pgfsetfillcolor{currentfill}%
\pgfsetlinewidth{0.803000pt}%
\definecolor{currentstroke}{rgb}{0.000000,0.000000,0.000000}%
\pgfsetstrokecolor{currentstroke}%
\pgfsetdash{}{0pt}%
\pgfsys@defobject{currentmarker}{\pgfqpoint{-0.048611in}{0.000000in}}{\pgfqpoint{-0.000000in}{0.000000in}}{%
\pgfpathmoveto{\pgfqpoint{-0.000000in}{0.000000in}}%
\pgfpathlineto{\pgfqpoint{-0.048611in}{0.000000in}}%
\pgfusepath{stroke,fill}%
}%
\begin{pgfscope}%
\pgfsys@transformshift{2.610186in}{2.754812in}%
\pgfsys@useobject{currentmarker}{}%
\end{pgfscope}%
\end{pgfscope}%
\begin{pgfscope}%
\definecolor{textcolor}{rgb}{0.000000,0.000000,0.000000}%
\pgfsetstrokecolor{textcolor}%
\pgfsetfillcolor{textcolor}%
\pgftext[x=2.374074in, y=2.706587in, left, base]{\color{textcolor}\rmfamily\fontsize{10.000000}{12.000000}\selectfont \(\displaystyle {40}\)}%
\end{pgfscope}%
\begin{pgfscope}%
\definecolor{textcolor}{rgb}{0.000000,0.000000,0.000000}%
\pgfsetstrokecolor{textcolor}%
\pgfsetfillcolor{textcolor}%
\pgftext[x=2.318519in,y=2.508025in,,bottom,rotate=90.000000]{\color{textcolor}\rmfamily\fontsize{10.000000}{12.000000}\selectfont Napiecie [\(\displaystyle V\)] }%
\end{pgfscope}%
\begin{pgfscope}%
\pgfpathrectangle{\pgfqpoint{2.610186in}{2.061420in}}{\pgfqpoint{1.530941in}{0.893210in}}%
\pgfusepath{clip}%
\pgfsetrectcap%
\pgfsetroundjoin%
\pgfsetlinewidth{1.505625pt}%
\definecolor{currentstroke}{rgb}{0.121569,0.466667,0.705882}%
\pgfsetstrokecolor{currentstroke}%
\pgfsetdash{}{0pt}%
\pgfpathmoveto{\pgfqpoint{2.679774in}{2.914029in}}%
\pgfpathlineto{\pgfqpoint{2.686768in}{2.133864in}}%
\pgfpathlineto{\pgfqpoint{2.693761in}{2.117942in}}%
\pgfpathlineto{\pgfqpoint{2.721737in}{2.117942in}}%
\pgfpathlineto{\pgfqpoint{2.728730in}{2.133864in}}%
\pgfpathlineto{\pgfqpoint{2.735724in}{2.117942in}}%
\pgfpathlineto{\pgfqpoint{2.798668in}{2.117942in}}%
\pgfpathlineto{\pgfqpoint{2.805662in}{2.102020in}}%
\pgfpathlineto{\pgfqpoint{2.812656in}{2.117942in}}%
\pgfpathlineto{\pgfqpoint{2.875600in}{2.117942in}}%
\pgfpathlineto{\pgfqpoint{2.882594in}{2.102020in}}%
\pgfpathlineto{\pgfqpoint{2.889588in}{2.102020in}}%
\pgfpathlineto{\pgfqpoint{2.903575in}{2.133864in}}%
\pgfpathlineto{\pgfqpoint{2.910569in}{2.850342in}}%
\pgfpathlineto{\pgfqpoint{2.924557in}{2.882186in}}%
\pgfpathlineto{\pgfqpoint{2.931550in}{2.882186in}}%
\pgfpathlineto{\pgfqpoint{2.938544in}{2.866264in}}%
\pgfpathlineto{\pgfqpoint{2.945538in}{2.882186in}}%
\pgfpathlineto{\pgfqpoint{2.980507in}{2.882186in}}%
\pgfpathlineto{\pgfqpoint{2.987501in}{2.898108in}}%
\pgfpathlineto{\pgfqpoint{2.994495in}{2.882186in}}%
\pgfpathlineto{\pgfqpoint{3.022470in}{2.882186in}}%
\pgfpathlineto{\pgfqpoint{3.029463in}{2.898108in}}%
\pgfpathlineto{\pgfqpoint{3.043451in}{2.898108in}}%
\pgfpathlineto{\pgfqpoint{3.050445in}{2.914029in}}%
\pgfpathlineto{\pgfqpoint{3.057439in}{2.882186in}}%
\pgfpathlineto{\pgfqpoint{3.064432in}{2.898108in}}%
\pgfpathlineto{\pgfqpoint{3.071426in}{2.882186in}}%
\pgfpathlineto{\pgfqpoint{3.078420in}{2.882186in}}%
\pgfpathlineto{\pgfqpoint{3.085414in}{2.898108in}}%
\pgfpathlineto{\pgfqpoint{3.183327in}{2.898108in}}%
\pgfpathlineto{\pgfqpoint{3.190321in}{2.914029in}}%
\pgfpathlineto{\pgfqpoint{3.197315in}{2.898108in}}%
\pgfpathlineto{\pgfqpoint{3.211302in}{2.898108in}}%
\pgfpathlineto{\pgfqpoint{3.218296in}{2.914029in}}%
\pgfpathlineto{\pgfqpoint{3.225290in}{2.898108in}}%
\pgfpathlineto{\pgfqpoint{3.232283in}{2.914029in}}%
\pgfpathlineto{\pgfqpoint{3.239277in}{2.898108in}}%
\pgfpathlineto{\pgfqpoint{3.267252in}{2.898108in}}%
\pgfpathlineto{\pgfqpoint{3.274246in}{2.914029in}}%
\pgfpathlineto{\pgfqpoint{3.281240in}{2.898108in}}%
\pgfpathlineto{\pgfqpoint{3.288234in}{2.914029in}}%
\pgfpathlineto{\pgfqpoint{3.295228in}{2.914029in}}%
\pgfpathlineto{\pgfqpoint{3.302221in}{2.898108in}}%
\pgfpathlineto{\pgfqpoint{3.309215in}{2.914029in}}%
\pgfpathlineto{\pgfqpoint{3.316209in}{2.898108in}}%
\pgfpathlineto{\pgfqpoint{3.323203in}{2.914029in}}%
\pgfpathlineto{\pgfqpoint{3.337190in}{2.914029in}}%
\pgfpathlineto{\pgfqpoint{3.344184in}{2.898108in}}%
\pgfpathlineto{\pgfqpoint{3.351178in}{2.149785in}}%
\pgfpathlineto{\pgfqpoint{3.379153in}{2.149785in}}%
\pgfpathlineto{\pgfqpoint{3.386147in}{2.133864in}}%
\pgfpathlineto{\pgfqpoint{3.435103in}{2.133864in}}%
\pgfpathlineto{\pgfqpoint{3.442097in}{2.149785in}}%
\pgfpathlineto{\pgfqpoint{3.449091in}{2.117942in}}%
\pgfpathlineto{\pgfqpoint{3.456085in}{2.117942in}}%
\pgfpathlineto{\pgfqpoint{3.463079in}{2.133864in}}%
\pgfpathlineto{\pgfqpoint{3.477066in}{2.133864in}}%
\pgfpathlineto{\pgfqpoint{3.484060in}{2.117942in}}%
\pgfpathlineto{\pgfqpoint{3.505041in}{2.117942in}}%
\pgfpathlineto{\pgfqpoint{3.512035in}{2.149785in}}%
\pgfpathlineto{\pgfqpoint{3.519029in}{2.117942in}}%
\pgfpathlineto{\pgfqpoint{3.728843in}{2.117942in}}%
\pgfpathlineto{\pgfqpoint{3.735837in}{2.102020in}}%
\pgfpathlineto{\pgfqpoint{3.742830in}{2.117942in}}%
\pgfpathlineto{\pgfqpoint{3.770806in}{2.117942in}}%
\pgfpathlineto{\pgfqpoint{3.777799in}{2.102020in}}%
\pgfpathlineto{\pgfqpoint{3.784793in}{2.850342in}}%
\pgfpathlineto{\pgfqpoint{3.791787in}{2.866264in}}%
\pgfpathlineto{\pgfqpoint{3.798781in}{2.866264in}}%
\pgfpathlineto{\pgfqpoint{3.805775in}{2.882186in}}%
\pgfpathlineto{\pgfqpoint{3.903688in}{2.882186in}}%
\pgfpathlineto{\pgfqpoint{3.910681in}{2.898108in}}%
\pgfpathlineto{\pgfqpoint{3.917675in}{2.882186in}}%
\pgfpathlineto{\pgfqpoint{3.931663in}{2.882186in}}%
\pgfpathlineto{\pgfqpoint{3.938657in}{2.898108in}}%
\pgfpathlineto{\pgfqpoint{3.945650in}{2.898108in}}%
\pgfpathlineto{\pgfqpoint{3.952644in}{2.882186in}}%
\pgfpathlineto{\pgfqpoint{3.966632in}{2.882186in}}%
\pgfpathlineto{\pgfqpoint{3.973626in}{2.898108in}}%
\pgfpathlineto{\pgfqpoint{3.994607in}{2.898108in}}%
\pgfpathlineto{\pgfqpoint{4.001601in}{2.914029in}}%
\pgfpathlineto{\pgfqpoint{4.008595in}{2.882186in}}%
\pgfpathlineto{\pgfqpoint{4.015588in}{2.898108in}}%
\pgfpathlineto{\pgfqpoint{4.022582in}{2.898108in}}%
\pgfpathlineto{\pgfqpoint{4.029576in}{2.882186in}}%
\pgfpathlineto{\pgfqpoint{4.036570in}{2.898108in}}%
\pgfpathlineto{\pgfqpoint{4.043563in}{2.882186in}}%
\pgfpathlineto{\pgfqpoint{4.050557in}{2.914029in}}%
\pgfpathlineto{\pgfqpoint{4.057551in}{2.914029in}}%
\pgfpathlineto{\pgfqpoint{4.064545in}{2.898108in}}%
\pgfpathlineto{\pgfqpoint{4.071539in}{2.898108in}}%
\pgfpathlineto{\pgfqpoint{4.071539in}{2.898108in}}%
\pgfusepath{stroke}%
\end{pgfscope}%
\begin{pgfscope}%
\pgfpathrectangle{\pgfqpoint{2.610186in}{2.061420in}}{\pgfqpoint{1.530941in}{0.893210in}}%
\pgfusepath{clip}%
\pgfsetrectcap%
\pgfsetroundjoin%
\pgfsetlinewidth{1.505625pt}%
\definecolor{currentstroke}{rgb}{1.000000,0.498039,0.054902}%
\pgfsetstrokecolor{currentstroke}%
\pgfsetdash{}{0pt}%
\pgfpathmoveto{\pgfqpoint{0.000000in}{0.000000in}}%
\pgfusepath{stroke}%
\end{pgfscope}%
\begin{pgfscope}%
\pgfpathrectangle{\pgfqpoint{2.610186in}{2.061420in}}{\pgfqpoint{1.530941in}{0.893210in}}%
\pgfusepath{clip}%
\pgfsetrectcap%
\pgfsetroundjoin%
\pgfsetlinewidth{1.505625pt}%
\definecolor{currentstroke}{rgb}{0.172549,0.627451,0.172549}%
\pgfsetstrokecolor{currentstroke}%
\pgfsetdash{}{0pt}%
\pgfpathmoveto{\pgfqpoint{2.679774in}{2.754812in}}%
\pgfpathlineto{\pgfqpoint{2.686768in}{2.324925in}}%
\pgfpathlineto{\pgfqpoint{2.700755in}{2.324925in}}%
\pgfpathlineto{\pgfqpoint{2.707749in}{2.309003in}}%
\pgfpathlineto{\pgfqpoint{2.721737in}{2.309003in}}%
\pgfpathlineto{\pgfqpoint{2.742718in}{2.261238in}}%
\pgfpathlineto{\pgfqpoint{2.763699in}{2.261238in}}%
\pgfpathlineto{\pgfqpoint{2.770693in}{2.245316in}}%
\pgfpathlineto{\pgfqpoint{2.791675in}{2.245316in}}%
\pgfpathlineto{\pgfqpoint{2.798668in}{2.229394in}}%
\pgfpathlineto{\pgfqpoint{2.812656in}{2.229394in}}%
\pgfpathlineto{\pgfqpoint{2.819650in}{2.213472in}}%
\pgfpathlineto{\pgfqpoint{2.833637in}{2.213472in}}%
\pgfpathlineto{\pgfqpoint{2.840631in}{2.197551in}}%
\pgfpathlineto{\pgfqpoint{2.889588in}{2.197551in}}%
\pgfpathlineto{\pgfqpoint{2.896581in}{2.181629in}}%
\pgfpathlineto{\pgfqpoint{2.910569in}{2.181629in}}%
\pgfpathlineto{\pgfqpoint{2.917563in}{2.197551in}}%
\pgfpathlineto{\pgfqpoint{2.938544in}{2.293081in}}%
\pgfpathlineto{\pgfqpoint{2.952532in}{2.324925in}}%
\pgfpathlineto{\pgfqpoint{2.959526in}{2.356768in}}%
\pgfpathlineto{\pgfqpoint{2.980507in}{2.404533in}}%
\pgfpathlineto{\pgfqpoint{2.987501in}{2.436377in}}%
\pgfpathlineto{\pgfqpoint{2.994495in}{2.452299in}}%
\pgfpathlineto{\pgfqpoint{3.001488in}{2.452299in}}%
\pgfpathlineto{\pgfqpoint{3.008482in}{2.484142in}}%
\pgfpathlineto{\pgfqpoint{3.022470in}{2.515986in}}%
\pgfpathlineto{\pgfqpoint{3.029463in}{2.515986in}}%
\pgfpathlineto{\pgfqpoint{3.036457in}{2.547829in}}%
\pgfpathlineto{\pgfqpoint{3.050445in}{2.579673in}}%
\pgfpathlineto{\pgfqpoint{3.071426in}{2.579673in}}%
\pgfpathlineto{\pgfqpoint{3.078420in}{2.611516in}}%
\pgfpathlineto{\pgfqpoint{3.092408in}{2.643360in}}%
\pgfpathlineto{\pgfqpoint{3.099401in}{2.643360in}}%
\pgfpathlineto{\pgfqpoint{3.106395in}{2.659281in}}%
\pgfpathlineto{\pgfqpoint{3.113389in}{2.659281in}}%
\pgfpathlineto{\pgfqpoint{3.127377in}{2.691125in}}%
\pgfpathlineto{\pgfqpoint{3.134370in}{2.691125in}}%
\pgfpathlineto{\pgfqpoint{3.141364in}{2.707047in}}%
\pgfpathlineto{\pgfqpoint{3.162346in}{2.707047in}}%
\pgfpathlineto{\pgfqpoint{3.176333in}{2.738890in}}%
\pgfpathlineto{\pgfqpoint{3.190321in}{2.738890in}}%
\pgfpathlineto{\pgfqpoint{3.197315in}{2.754812in}}%
\pgfpathlineto{\pgfqpoint{3.204308in}{2.754812in}}%
\pgfpathlineto{\pgfqpoint{3.211302in}{2.770734in}}%
\pgfpathlineto{\pgfqpoint{3.239277in}{2.770734in}}%
\pgfpathlineto{\pgfqpoint{3.246271in}{2.786655in}}%
\pgfpathlineto{\pgfqpoint{3.253265in}{2.786655in}}%
\pgfpathlineto{\pgfqpoint{3.260259in}{2.802577in}}%
\pgfpathlineto{\pgfqpoint{3.281240in}{2.802577in}}%
\pgfpathlineto{\pgfqpoint{3.288234in}{2.818499in}}%
\pgfpathlineto{\pgfqpoint{3.302221in}{2.818499in}}%
\pgfpathlineto{\pgfqpoint{3.309215in}{2.834421in}}%
\pgfpathlineto{\pgfqpoint{3.323203in}{2.834421in}}%
\pgfpathlineto{\pgfqpoint{3.330197in}{2.818499in}}%
\pgfpathlineto{\pgfqpoint{3.337190in}{2.834421in}}%
\pgfpathlineto{\pgfqpoint{3.344184in}{2.834421in}}%
\pgfpathlineto{\pgfqpoint{3.351178in}{2.818499in}}%
\pgfpathlineto{\pgfqpoint{3.358172in}{2.786655in}}%
\pgfpathlineto{\pgfqpoint{3.365166in}{2.770734in}}%
\pgfpathlineto{\pgfqpoint{3.379153in}{2.707047in}}%
\pgfpathlineto{\pgfqpoint{3.393141in}{2.675203in}}%
\pgfpathlineto{\pgfqpoint{3.400135in}{2.643360in}}%
\pgfpathlineto{\pgfqpoint{3.428110in}{2.579673in}}%
\pgfpathlineto{\pgfqpoint{3.435103in}{2.547829in}}%
\pgfpathlineto{\pgfqpoint{3.449091in}{2.515986in}}%
\pgfpathlineto{\pgfqpoint{3.456085in}{2.515986in}}%
\pgfpathlineto{\pgfqpoint{3.484060in}{2.452299in}}%
\pgfpathlineto{\pgfqpoint{3.491054in}{2.452299in}}%
\pgfpathlineto{\pgfqpoint{3.519029in}{2.388612in}}%
\pgfpathlineto{\pgfqpoint{3.526023in}{2.388612in}}%
\pgfpathlineto{\pgfqpoint{3.540010in}{2.356768in}}%
\pgfpathlineto{\pgfqpoint{3.547004in}{2.372690in}}%
\pgfpathlineto{\pgfqpoint{3.553998in}{2.340846in}}%
\pgfpathlineto{\pgfqpoint{3.560992in}{2.324925in}}%
\pgfpathlineto{\pgfqpoint{3.581973in}{2.324925in}}%
\pgfpathlineto{\pgfqpoint{3.595961in}{2.293081in}}%
\pgfpathlineto{\pgfqpoint{3.609948in}{2.293081in}}%
\pgfpathlineto{\pgfqpoint{3.623936in}{2.261238in}}%
\pgfpathlineto{\pgfqpoint{3.644917in}{2.261238in}}%
\pgfpathlineto{\pgfqpoint{3.651911in}{2.245316in}}%
\pgfpathlineto{\pgfqpoint{3.665899in}{2.245316in}}%
\pgfpathlineto{\pgfqpoint{3.672892in}{2.229394in}}%
\pgfpathlineto{\pgfqpoint{3.679886in}{2.229394in}}%
\pgfpathlineto{\pgfqpoint{3.686880in}{2.213472in}}%
\pgfpathlineto{\pgfqpoint{3.693874in}{2.213472in}}%
\pgfpathlineto{\pgfqpoint{3.700868in}{2.229394in}}%
\pgfpathlineto{\pgfqpoint{3.714855in}{2.197551in}}%
\pgfpathlineto{\pgfqpoint{3.763812in}{2.197551in}}%
\pgfpathlineto{\pgfqpoint{3.770806in}{2.181629in}}%
\pgfpathlineto{\pgfqpoint{3.777799in}{2.197551in}}%
\pgfpathlineto{\pgfqpoint{3.784793in}{2.197551in}}%
\pgfpathlineto{\pgfqpoint{3.798781in}{2.229394in}}%
\pgfpathlineto{\pgfqpoint{3.812768in}{2.293081in}}%
\pgfpathlineto{\pgfqpoint{3.819762in}{2.293081in}}%
\pgfpathlineto{\pgfqpoint{3.833750in}{2.356768in}}%
\pgfpathlineto{\pgfqpoint{3.854731in}{2.404533in}}%
\pgfpathlineto{\pgfqpoint{3.861725in}{2.436377in}}%
\pgfpathlineto{\pgfqpoint{3.868719in}{2.452299in}}%
\pgfpathlineto{\pgfqpoint{3.875712in}{2.452299in}}%
\pgfpathlineto{\pgfqpoint{3.882706in}{2.484142in}}%
\pgfpathlineto{\pgfqpoint{3.896694in}{2.515986in}}%
\pgfpathlineto{\pgfqpoint{3.903688in}{2.515986in}}%
\pgfpathlineto{\pgfqpoint{3.910681in}{2.547829in}}%
\pgfpathlineto{\pgfqpoint{3.917675in}{2.547829in}}%
\pgfpathlineto{\pgfqpoint{3.931663in}{2.579673in}}%
\pgfpathlineto{\pgfqpoint{3.938657in}{2.579673in}}%
\pgfpathlineto{\pgfqpoint{3.945650in}{2.611516in}}%
\pgfpathlineto{\pgfqpoint{3.959638in}{2.611516in}}%
\pgfpathlineto{\pgfqpoint{3.966632in}{2.643360in}}%
\pgfpathlineto{\pgfqpoint{3.980619in}{2.643360in}}%
\pgfpathlineto{\pgfqpoint{3.987613in}{2.675203in}}%
\pgfpathlineto{\pgfqpoint{3.994607in}{2.691125in}}%
\pgfpathlineto{\pgfqpoint{4.008595in}{2.691125in}}%
\pgfpathlineto{\pgfqpoint{4.015588in}{2.707047in}}%
\pgfpathlineto{\pgfqpoint{4.029576in}{2.707047in}}%
\pgfpathlineto{\pgfqpoint{4.043563in}{2.738890in}}%
\pgfpathlineto{\pgfqpoint{4.064545in}{2.738890in}}%
\pgfpathlineto{\pgfqpoint{4.071539in}{2.770734in}}%
\pgfpathlineto{\pgfqpoint{4.071539in}{2.770734in}}%
\pgfusepath{stroke}%
\end{pgfscope}%
\begin{pgfscope}%
\pgfpathrectangle{\pgfqpoint{2.610186in}{2.061420in}}{\pgfqpoint{1.530941in}{0.893210in}}%
\pgfusepath{clip}%
\pgfsetrectcap%
\pgfsetroundjoin%
\pgfsetlinewidth{1.505625pt}%
\definecolor{currentstroke}{rgb}{0.839216,0.152941,0.156863}%
\pgfsetstrokecolor{currentstroke}%
\pgfsetdash{}{0pt}%
\pgfpathmoveto{\pgfqpoint{0.000000in}{0.000000in}}%
\pgfusepath{stroke}%
\end{pgfscope}%
\begin{pgfscope}%
\pgfsetrectcap%
\pgfsetmiterjoin%
\pgfsetlinewidth{0.803000pt}%
\definecolor{currentstroke}{rgb}{0.000000,0.000000,0.000000}%
\pgfsetstrokecolor{currentstroke}%
\pgfsetdash{}{0pt}%
\pgfpathmoveto{\pgfqpoint{2.610186in}{2.061420in}}%
\pgfpathlineto{\pgfqpoint{2.610186in}{2.954630in}}%
\pgfusepath{stroke}%
\end{pgfscope}%
\begin{pgfscope}%
\pgfsetrectcap%
\pgfsetmiterjoin%
\pgfsetlinewidth{0.803000pt}%
\definecolor{currentstroke}{rgb}{0.000000,0.000000,0.000000}%
\pgfsetstrokecolor{currentstroke}%
\pgfsetdash{}{0pt}%
\pgfpathmoveto{\pgfqpoint{4.141127in}{2.061420in}}%
\pgfpathlineto{\pgfqpoint{4.141127in}{2.954630in}}%
\pgfusepath{stroke}%
\end{pgfscope}%
\begin{pgfscope}%
\pgfsetrectcap%
\pgfsetmiterjoin%
\pgfsetlinewidth{0.803000pt}%
\definecolor{currentstroke}{rgb}{0.000000,0.000000,0.000000}%
\pgfsetstrokecolor{currentstroke}%
\pgfsetdash{}{0pt}%
\pgfpathmoveto{\pgfqpoint{2.610186in}{2.061420in}}%
\pgfpathlineto{\pgfqpoint{4.141127in}{2.061420in}}%
\pgfusepath{stroke}%
\end{pgfscope}%
\begin{pgfscope}%
\pgfsetrectcap%
\pgfsetmiterjoin%
\pgfsetlinewidth{0.803000pt}%
\definecolor{currentstroke}{rgb}{0.000000,0.000000,0.000000}%
\pgfsetstrokecolor{currentstroke}%
\pgfsetdash{}{0pt}%
\pgfpathmoveto{\pgfqpoint{2.610186in}{2.954630in}}%
\pgfpathlineto{\pgfqpoint{4.141127in}{2.954630in}}%
\pgfusepath{stroke}%
\end{pgfscope}%
\begin{pgfscope}%
\definecolor{textcolor}{rgb}{0.000000,0.000000,0.000000}%
\pgfsetstrokecolor{textcolor}%
\pgfsetfillcolor{textcolor}%
\pgftext[x=3.375656in,y=3.037963in,,base]{\color{textcolor}\rmfamily\fontsize{12.000000}{14.400000}\selectfont 4}%
\end{pgfscope}%
\begin{pgfscope}%
\pgfsetbuttcap%
\pgfsetmiterjoin%
\definecolor{currentfill}{rgb}{1.000000,1.000000,1.000000}%
\pgfsetfillcolor{currentfill}%
\pgfsetlinewidth{0.000000pt}%
\definecolor{currentstroke}{rgb}{0.000000,0.000000,0.000000}%
\pgfsetstrokecolor{currentstroke}%
\pgfsetstrokeopacity{0.000000}%
\pgfsetdash{}{0pt}%
\pgfpathmoveto{\pgfqpoint{4.720371in}{2.061420in}}%
\pgfpathlineto{\pgfqpoint{6.251312in}{2.061420in}}%
\pgfpathlineto{\pgfqpoint{6.251312in}{2.954630in}}%
\pgfpathlineto{\pgfqpoint{4.720371in}{2.954630in}}%
\pgfpathclose%
\pgfusepath{fill}%
\end{pgfscope}%
\begin{pgfscope}%
\pgfsetbuttcap%
\pgfsetroundjoin%
\definecolor{currentfill}{rgb}{0.000000,0.000000,0.000000}%
\pgfsetfillcolor{currentfill}%
\pgfsetlinewidth{0.803000pt}%
\definecolor{currentstroke}{rgb}{0.000000,0.000000,0.000000}%
\pgfsetstrokecolor{currentstroke}%
\pgfsetdash{}{0pt}%
\pgfsys@defobject{currentmarker}{\pgfqpoint{0.000000in}{-0.048611in}}{\pgfqpoint{0.000000in}{0.000000in}}{%
\pgfpathmoveto{\pgfqpoint{0.000000in}{0.000000in}}%
\pgfpathlineto{\pgfqpoint{0.000000in}{-0.048611in}}%
\pgfusepath{stroke,fill}%
}%
\begin{pgfscope}%
\pgfsys@transformshift{4.789959in}{2.061420in}%
\pgfsys@useobject{currentmarker}{}%
\end{pgfscope}%
\end{pgfscope}%
\begin{pgfscope}%
\definecolor{textcolor}{rgb}{0.000000,0.000000,0.000000}%
\pgfsetstrokecolor{textcolor}%
\pgfsetfillcolor{textcolor}%
\pgftext[x=4.789959in,y=1.964197in,,top]{\color{textcolor}\rmfamily\fontsize{10.000000}{12.000000}\selectfont \(\displaystyle {0}\)}%
\end{pgfscope}%
\begin{pgfscope}%
\pgfsetbuttcap%
\pgfsetroundjoin%
\definecolor{currentfill}{rgb}{0.000000,0.000000,0.000000}%
\pgfsetfillcolor{currentfill}%
\pgfsetlinewidth{0.803000pt}%
\definecolor{currentstroke}{rgb}{0.000000,0.000000,0.000000}%
\pgfsetstrokecolor{currentstroke}%
\pgfsetdash{}{0pt}%
\pgfsys@defobject{currentmarker}{\pgfqpoint{0.000000in}{-0.048611in}}{\pgfqpoint{0.000000in}{0.000000in}}{%
\pgfpathmoveto{\pgfqpoint{0.000000in}{0.000000in}}%
\pgfpathlineto{\pgfqpoint{0.000000in}{-0.048611in}}%
\pgfusepath{stroke,fill}%
}%
\begin{pgfscope}%
\pgfsys@transformshift{5.489338in}{2.061420in}%
\pgfsys@useobject{currentmarker}{}%
\end{pgfscope}%
\end{pgfscope}%
\begin{pgfscope}%
\definecolor{textcolor}{rgb}{0.000000,0.000000,0.000000}%
\pgfsetstrokecolor{textcolor}%
\pgfsetfillcolor{textcolor}%
\pgftext[x=5.489338in,y=1.964197in,,top]{\color{textcolor}\rmfamily\fontsize{10.000000}{12.000000}\selectfont \(\displaystyle {2000}\)}%
\end{pgfscope}%
\begin{pgfscope}%
\pgfsetbuttcap%
\pgfsetroundjoin%
\definecolor{currentfill}{rgb}{0.000000,0.000000,0.000000}%
\pgfsetfillcolor{currentfill}%
\pgfsetlinewidth{0.803000pt}%
\definecolor{currentstroke}{rgb}{0.000000,0.000000,0.000000}%
\pgfsetstrokecolor{currentstroke}%
\pgfsetdash{}{0pt}%
\pgfsys@defobject{currentmarker}{\pgfqpoint{0.000000in}{-0.048611in}}{\pgfqpoint{0.000000in}{0.000000in}}{%
\pgfpathmoveto{\pgfqpoint{0.000000in}{0.000000in}}%
\pgfpathlineto{\pgfqpoint{0.000000in}{-0.048611in}}%
\pgfusepath{stroke,fill}%
}%
\begin{pgfscope}%
\pgfsys@transformshift{6.188718in}{2.061420in}%
\pgfsys@useobject{currentmarker}{}%
\end{pgfscope}%
\end{pgfscope}%
\begin{pgfscope}%
\definecolor{textcolor}{rgb}{0.000000,0.000000,0.000000}%
\pgfsetstrokecolor{textcolor}%
\pgfsetfillcolor{textcolor}%
\pgftext[x=6.188718in,y=1.964197in,,top]{\color{textcolor}\rmfamily\fontsize{10.000000}{12.000000}\selectfont \(\displaystyle {4000}\)}%
\end{pgfscope}%
\begin{pgfscope}%
\definecolor{textcolor}{rgb}{0.000000,0.000000,0.000000}%
\pgfsetstrokecolor{textcolor}%
\pgfsetfillcolor{textcolor}%
\pgftext[x=5.485841in,y=1.785185in,,top]{\color{textcolor}\rmfamily\fontsize{10.000000}{12.000000}\selectfont Czas [\(\displaystyle \mu s\)] }%
\end{pgfscope}%
\begin{pgfscope}%
\pgfsetbuttcap%
\pgfsetroundjoin%
\definecolor{currentfill}{rgb}{0.000000,0.000000,0.000000}%
\pgfsetfillcolor{currentfill}%
\pgfsetlinewidth{0.803000pt}%
\definecolor{currentstroke}{rgb}{0.000000,0.000000,0.000000}%
\pgfsetstrokecolor{currentstroke}%
\pgfsetdash{}{0pt}%
\pgfsys@defobject{currentmarker}{\pgfqpoint{-0.048611in}{0.000000in}}{\pgfqpoint{-0.000000in}{0.000000in}}{%
\pgfpathmoveto{\pgfqpoint{-0.000000in}{0.000000in}}%
\pgfpathlineto{\pgfqpoint{-0.048611in}{0.000000in}}%
\pgfusepath{stroke,fill}%
}%
\begin{pgfscope}%
\pgfsys@transformshift{4.720371in}{2.102020in}%
\pgfsys@useobject{currentmarker}{}%
\end{pgfscope}%
\end{pgfscope}%
\begin{pgfscope}%
\definecolor{textcolor}{rgb}{0.000000,0.000000,0.000000}%
\pgfsetstrokecolor{textcolor}%
\pgfsetfillcolor{textcolor}%
\pgftext[x=4.553704in, y=2.053795in, left, base]{\color{textcolor}\rmfamily\fontsize{10.000000}{12.000000}\selectfont \(\displaystyle {0}\)}%
\end{pgfscope}%
\begin{pgfscope}%
\pgfsetbuttcap%
\pgfsetroundjoin%
\definecolor{currentfill}{rgb}{0.000000,0.000000,0.000000}%
\pgfsetfillcolor{currentfill}%
\pgfsetlinewidth{0.803000pt}%
\definecolor{currentstroke}{rgb}{0.000000,0.000000,0.000000}%
\pgfsetstrokecolor{currentstroke}%
\pgfsetdash{}{0pt}%
\pgfsys@defobject{currentmarker}{\pgfqpoint{-0.048611in}{0.000000in}}{\pgfqpoint{-0.000000in}{0.000000in}}{%
\pgfpathmoveto{\pgfqpoint{-0.000000in}{0.000000in}}%
\pgfpathlineto{\pgfqpoint{-0.048611in}{0.000000in}}%
\pgfusepath{stroke,fill}%
}%
\begin{pgfscope}%
\pgfsys@transformshift{4.720371in}{2.426824in}%
\pgfsys@useobject{currentmarker}{}%
\end{pgfscope}%
\end{pgfscope}%
\begin{pgfscope}%
\definecolor{textcolor}{rgb}{0.000000,0.000000,0.000000}%
\pgfsetstrokecolor{textcolor}%
\pgfsetfillcolor{textcolor}%
\pgftext[x=4.484259in, y=2.378599in, left, base]{\color{textcolor}\rmfamily\fontsize{10.000000}{12.000000}\selectfont \(\displaystyle {20}\)}%
\end{pgfscope}%
\begin{pgfscope}%
\pgfsetbuttcap%
\pgfsetroundjoin%
\definecolor{currentfill}{rgb}{0.000000,0.000000,0.000000}%
\pgfsetfillcolor{currentfill}%
\pgfsetlinewidth{0.803000pt}%
\definecolor{currentstroke}{rgb}{0.000000,0.000000,0.000000}%
\pgfsetstrokecolor{currentstroke}%
\pgfsetdash{}{0pt}%
\pgfsys@defobject{currentmarker}{\pgfqpoint{-0.048611in}{0.000000in}}{\pgfqpoint{-0.000000in}{0.000000in}}{%
\pgfpathmoveto{\pgfqpoint{-0.000000in}{0.000000in}}%
\pgfpathlineto{\pgfqpoint{-0.048611in}{0.000000in}}%
\pgfusepath{stroke,fill}%
}%
\begin{pgfscope}%
\pgfsys@transformshift{4.720371in}{2.751628in}%
\pgfsys@useobject{currentmarker}{}%
\end{pgfscope}%
\end{pgfscope}%
\begin{pgfscope}%
\definecolor{textcolor}{rgb}{0.000000,0.000000,0.000000}%
\pgfsetstrokecolor{textcolor}%
\pgfsetfillcolor{textcolor}%
\pgftext[x=4.484259in, y=2.703402in, left, base]{\color{textcolor}\rmfamily\fontsize{10.000000}{12.000000}\selectfont \(\displaystyle {40}\)}%
\end{pgfscope}%
\begin{pgfscope}%
\definecolor{textcolor}{rgb}{0.000000,0.000000,0.000000}%
\pgfsetstrokecolor{textcolor}%
\pgfsetfillcolor{textcolor}%
\pgftext[x=4.428704in,y=2.508025in,,bottom,rotate=90.000000]{\color{textcolor}\rmfamily\fontsize{10.000000}{12.000000}\selectfont Napiecie [\(\displaystyle V\)] }%
\end{pgfscope}%
\begin{pgfscope}%
\pgfpathrectangle{\pgfqpoint{4.720371in}{2.061420in}}{\pgfqpoint{1.530941in}{0.893210in}}%
\pgfusepath{clip}%
\pgfsetrectcap%
\pgfsetroundjoin%
\pgfsetlinewidth{1.505625pt}%
\definecolor{currentstroke}{rgb}{0.121569,0.466667,0.705882}%
\pgfsetstrokecolor{currentstroke}%
\pgfsetdash{}{0pt}%
\pgfpathmoveto{\pgfqpoint{4.789959in}{2.102020in}}%
\pgfpathlineto{\pgfqpoint{4.796953in}{2.897789in}}%
\pgfpathlineto{\pgfqpoint{4.803947in}{2.914029in}}%
\pgfpathlineto{\pgfqpoint{4.810940in}{2.897789in}}%
\pgfpathlineto{\pgfqpoint{4.817934in}{2.914029in}}%
\pgfpathlineto{\pgfqpoint{4.824928in}{2.897789in}}%
\pgfpathlineto{\pgfqpoint{4.831922in}{2.914029in}}%
\pgfpathlineto{\pgfqpoint{4.838916in}{2.897789in}}%
\pgfpathlineto{\pgfqpoint{4.845909in}{2.897789in}}%
\pgfpathlineto{\pgfqpoint{4.852903in}{2.914029in}}%
\pgfpathlineto{\pgfqpoint{4.873885in}{2.914029in}}%
\pgfpathlineto{\pgfqpoint{4.880878in}{2.897789in}}%
\pgfpathlineto{\pgfqpoint{4.887872in}{2.914029in}}%
\pgfpathlineto{\pgfqpoint{4.894866in}{2.914029in}}%
\pgfpathlineto{\pgfqpoint{4.901860in}{2.897789in}}%
\pgfpathlineto{\pgfqpoint{4.908854in}{2.134500in}}%
\pgfpathlineto{\pgfqpoint{4.936829in}{2.134500in}}%
\pgfpathlineto{\pgfqpoint{4.950816in}{2.102020in}}%
\pgfpathlineto{\pgfqpoint{4.957810in}{2.118260in}}%
\pgfpathlineto{\pgfqpoint{4.971798in}{2.118260in}}%
\pgfpathlineto{\pgfqpoint{4.978791in}{2.102020in}}%
\pgfpathlineto{\pgfqpoint{4.985785in}{2.102020in}}%
\pgfpathlineto{\pgfqpoint{4.992779in}{2.118260in}}%
\pgfpathlineto{\pgfqpoint{4.999773in}{2.118260in}}%
\pgfpathlineto{\pgfqpoint{5.006767in}{2.102020in}}%
\pgfpathlineto{\pgfqpoint{5.020754in}{2.102020in}}%
\pgfpathlineto{\pgfqpoint{5.027748in}{2.118260in}}%
\pgfpathlineto{\pgfqpoint{5.034742in}{2.102020in}}%
\pgfpathlineto{\pgfqpoint{5.160630in}{2.102020in}}%
\pgfpathlineto{\pgfqpoint{5.167624in}{2.865309in}}%
\pgfpathlineto{\pgfqpoint{5.181611in}{2.865309in}}%
\pgfpathlineto{\pgfqpoint{5.188605in}{2.881549in}}%
\pgfpathlineto{\pgfqpoint{5.202593in}{2.881549in}}%
\pgfpathlineto{\pgfqpoint{5.209587in}{2.897789in}}%
\pgfpathlineto{\pgfqpoint{5.216580in}{2.881549in}}%
\pgfpathlineto{\pgfqpoint{5.244556in}{2.881549in}}%
\pgfpathlineto{\pgfqpoint{5.251549in}{2.897789in}}%
\pgfpathlineto{\pgfqpoint{5.293512in}{2.897789in}}%
\pgfpathlineto{\pgfqpoint{5.300506in}{2.914029in}}%
\pgfpathlineto{\pgfqpoint{5.307500in}{2.897789in}}%
\pgfpathlineto{\pgfqpoint{5.314494in}{2.897789in}}%
\pgfpathlineto{\pgfqpoint{5.321487in}{2.914029in}}%
\pgfpathlineto{\pgfqpoint{5.328481in}{2.897789in}}%
\pgfpathlineto{\pgfqpoint{5.335475in}{2.897789in}}%
\pgfpathlineto{\pgfqpoint{5.342469in}{2.914029in}}%
\pgfpathlineto{\pgfqpoint{5.356456in}{2.914029in}}%
\pgfpathlineto{\pgfqpoint{5.363450in}{2.897789in}}%
\pgfpathlineto{\pgfqpoint{5.370444in}{2.914029in}}%
\pgfpathlineto{\pgfqpoint{5.384431in}{2.914029in}}%
\pgfpathlineto{\pgfqpoint{5.391425in}{2.897789in}}%
\pgfpathlineto{\pgfqpoint{5.398419in}{2.897789in}}%
\pgfpathlineto{\pgfqpoint{5.405413in}{2.914029in}}%
\pgfpathlineto{\pgfqpoint{5.419400in}{2.914029in}}%
\pgfpathlineto{\pgfqpoint{5.426394in}{2.881549in}}%
\pgfpathlineto{\pgfqpoint{5.433388in}{2.134500in}}%
\pgfpathlineto{\pgfqpoint{5.447376in}{2.134500in}}%
\pgfpathlineto{\pgfqpoint{5.454369in}{2.118260in}}%
\pgfpathlineto{\pgfqpoint{5.461363in}{2.134500in}}%
\pgfpathlineto{\pgfqpoint{5.468357in}{2.118260in}}%
\pgfpathlineto{\pgfqpoint{5.489338in}{2.118260in}}%
\pgfpathlineto{\pgfqpoint{5.496332in}{2.102020in}}%
\pgfpathlineto{\pgfqpoint{5.503326in}{2.118260in}}%
\pgfpathlineto{\pgfqpoint{5.510320in}{2.102020in}}%
\pgfpathlineto{\pgfqpoint{5.517313in}{2.102020in}}%
\pgfpathlineto{\pgfqpoint{5.524307in}{2.118260in}}%
\pgfpathlineto{\pgfqpoint{5.531301in}{2.102020in}}%
\pgfpathlineto{\pgfqpoint{5.538295in}{2.118260in}}%
\pgfpathlineto{\pgfqpoint{5.545289in}{2.118260in}}%
\pgfpathlineto{\pgfqpoint{5.552282in}{2.102020in}}%
\pgfpathlineto{\pgfqpoint{5.685165in}{2.102020in}}%
\pgfpathlineto{\pgfqpoint{5.692158in}{2.865309in}}%
\pgfpathlineto{\pgfqpoint{5.699152in}{2.881549in}}%
\pgfpathlineto{\pgfqpoint{5.706146in}{2.881549in}}%
\pgfpathlineto{\pgfqpoint{5.713140in}{2.897789in}}%
\pgfpathlineto{\pgfqpoint{5.720133in}{2.881549in}}%
\pgfpathlineto{\pgfqpoint{5.734121in}{2.881549in}}%
\pgfpathlineto{\pgfqpoint{5.741115in}{2.897789in}}%
\pgfpathlineto{\pgfqpoint{5.748109in}{2.881549in}}%
\pgfpathlineto{\pgfqpoint{5.755102in}{2.881549in}}%
\pgfpathlineto{\pgfqpoint{5.762096in}{2.897789in}}%
\pgfpathlineto{\pgfqpoint{5.769090in}{2.881549in}}%
\pgfpathlineto{\pgfqpoint{5.776084in}{2.881549in}}%
\pgfpathlineto{\pgfqpoint{5.783078in}{2.897789in}}%
\pgfpathlineto{\pgfqpoint{5.790071in}{2.897789in}}%
\pgfpathlineto{\pgfqpoint{5.797065in}{2.914029in}}%
\pgfpathlineto{\pgfqpoint{5.804059in}{2.897789in}}%
\pgfpathlineto{\pgfqpoint{5.811053in}{2.897789in}}%
\pgfpathlineto{\pgfqpoint{5.818047in}{2.914029in}}%
\pgfpathlineto{\pgfqpoint{5.825040in}{2.897789in}}%
\pgfpathlineto{\pgfqpoint{5.832034in}{2.914029in}}%
\pgfpathlineto{\pgfqpoint{5.860009in}{2.914029in}}%
\pgfpathlineto{\pgfqpoint{5.867003in}{2.897789in}}%
\pgfpathlineto{\pgfqpoint{5.873997in}{2.914029in}}%
\pgfpathlineto{\pgfqpoint{5.880991in}{2.897789in}}%
\pgfpathlineto{\pgfqpoint{5.901972in}{2.897789in}}%
\pgfpathlineto{\pgfqpoint{5.908966in}{2.914029in}}%
\pgfpathlineto{\pgfqpoint{5.950929in}{2.914029in}}%
\pgfpathlineto{\pgfqpoint{5.957922in}{2.134500in}}%
\pgfpathlineto{\pgfqpoint{5.971910in}{2.134500in}}%
\pgfpathlineto{\pgfqpoint{5.978904in}{2.118260in}}%
\pgfpathlineto{\pgfqpoint{6.006879in}{2.118260in}}%
\pgfpathlineto{\pgfqpoint{6.013873in}{2.134500in}}%
\pgfpathlineto{\pgfqpoint{6.020867in}{2.118260in}}%
\pgfpathlineto{\pgfqpoint{6.027860in}{2.118260in}}%
\pgfpathlineto{\pgfqpoint{6.034854in}{2.102020in}}%
\pgfpathlineto{\pgfqpoint{6.041848in}{2.118260in}}%
\pgfpathlineto{\pgfqpoint{6.048842in}{2.118260in}}%
\pgfpathlineto{\pgfqpoint{6.055836in}{2.102020in}}%
\pgfpathlineto{\pgfqpoint{6.083811in}{2.102020in}}%
\pgfpathlineto{\pgfqpoint{6.090805in}{2.118260in}}%
\pgfpathlineto{\pgfqpoint{6.097798in}{2.102020in}}%
\pgfpathlineto{\pgfqpoint{6.181724in}{2.102020in}}%
\pgfpathlineto{\pgfqpoint{6.181724in}{2.102020in}}%
\pgfusepath{stroke}%
\end{pgfscope}%
\begin{pgfscope}%
\pgfpathrectangle{\pgfqpoint{4.720371in}{2.061420in}}{\pgfqpoint{1.530941in}{0.893210in}}%
\pgfusepath{clip}%
\pgfsetrectcap%
\pgfsetroundjoin%
\pgfsetlinewidth{1.505625pt}%
\definecolor{currentstroke}{rgb}{1.000000,0.498039,0.054902}%
\pgfsetstrokecolor{currentstroke}%
\pgfsetdash{}{0pt}%
\pgfpathmoveto{\pgfqpoint{0.000000in}{0.000000in}}%
\pgfusepath{stroke}%
\end{pgfscope}%
\begin{pgfscope}%
\pgfpathrectangle{\pgfqpoint{4.720371in}{2.061420in}}{\pgfqpoint{1.530941in}{0.893210in}}%
\pgfusepath{clip}%
\pgfsetrectcap%
\pgfsetroundjoin%
\pgfsetlinewidth{1.505625pt}%
\definecolor{currentstroke}{rgb}{0.172549,0.627451,0.172549}%
\pgfsetstrokecolor{currentstroke}%
\pgfsetdash{}{0pt}%
\pgfpathmoveto{\pgfqpoint{4.789959in}{2.150741in}}%
\pgfpathlineto{\pgfqpoint{4.796953in}{2.767868in}}%
\pgfpathlineto{\pgfqpoint{4.803947in}{2.767868in}}%
\pgfpathlineto{\pgfqpoint{4.817934in}{2.800348in}}%
\pgfpathlineto{\pgfqpoint{4.831922in}{2.800348in}}%
\pgfpathlineto{\pgfqpoint{4.838916in}{2.816588in}}%
\pgfpathlineto{\pgfqpoint{4.845909in}{2.816588in}}%
\pgfpathlineto{\pgfqpoint{4.852903in}{2.832829in}}%
\pgfpathlineto{\pgfqpoint{4.859897in}{2.832829in}}%
\pgfpathlineto{\pgfqpoint{4.866891in}{2.849069in}}%
\pgfpathlineto{\pgfqpoint{4.873885in}{2.832829in}}%
\pgfpathlineto{\pgfqpoint{4.880878in}{2.849069in}}%
\pgfpathlineto{\pgfqpoint{4.887872in}{2.832829in}}%
\pgfpathlineto{\pgfqpoint{4.894866in}{2.865309in}}%
\pgfpathlineto{\pgfqpoint{4.901860in}{2.849069in}}%
\pgfpathlineto{\pgfqpoint{4.908854in}{2.816588in}}%
\pgfpathlineto{\pgfqpoint{4.915847in}{2.767868in}}%
\pgfpathlineto{\pgfqpoint{4.922841in}{2.702907in}}%
\pgfpathlineto{\pgfqpoint{4.936829in}{2.637946in}}%
\pgfpathlineto{\pgfqpoint{4.943822in}{2.589226in}}%
\pgfpathlineto{\pgfqpoint{4.950816in}{2.556745in}}%
\pgfpathlineto{\pgfqpoint{4.957810in}{2.508025in}}%
\pgfpathlineto{\pgfqpoint{4.964804in}{2.475544in}}%
\pgfpathlineto{\pgfqpoint{4.971798in}{2.459304in}}%
\pgfpathlineto{\pgfqpoint{4.978791in}{2.426824in}}%
\pgfpathlineto{\pgfqpoint{4.985785in}{2.410584in}}%
\pgfpathlineto{\pgfqpoint{4.992779in}{2.378103in}}%
\pgfpathlineto{\pgfqpoint{5.034742in}{2.280662in}}%
\pgfpathlineto{\pgfqpoint{5.041736in}{2.248182in}}%
\pgfpathlineto{\pgfqpoint{5.048729in}{2.248182in}}%
\pgfpathlineto{\pgfqpoint{5.062717in}{2.215701in}}%
\pgfpathlineto{\pgfqpoint{5.069711in}{2.231942in}}%
\pgfpathlineto{\pgfqpoint{5.090692in}{2.183221in}}%
\pgfpathlineto{\pgfqpoint{5.118667in}{2.183221in}}%
\pgfpathlineto{\pgfqpoint{5.125661in}{2.150741in}}%
\pgfpathlineto{\pgfqpoint{5.132655in}{2.166981in}}%
\pgfpathlineto{\pgfqpoint{5.139649in}{2.150741in}}%
\pgfpathlineto{\pgfqpoint{5.167624in}{2.150741in}}%
\pgfpathlineto{\pgfqpoint{5.181611in}{2.248182in}}%
\pgfpathlineto{\pgfqpoint{5.188605in}{2.313143in}}%
\pgfpathlineto{\pgfqpoint{5.195599in}{2.345623in}}%
\pgfpathlineto{\pgfqpoint{5.202593in}{2.394343in}}%
\pgfpathlineto{\pgfqpoint{5.209587in}{2.426824in}}%
\pgfpathlineto{\pgfqpoint{5.216580in}{2.443064in}}%
\pgfpathlineto{\pgfqpoint{5.223574in}{2.491785in}}%
\pgfpathlineto{\pgfqpoint{5.237562in}{2.556745in}}%
\pgfpathlineto{\pgfqpoint{5.244556in}{2.572986in}}%
\pgfpathlineto{\pgfqpoint{5.258543in}{2.637946in}}%
\pgfpathlineto{\pgfqpoint{5.265537in}{2.637946in}}%
\pgfpathlineto{\pgfqpoint{5.272531in}{2.670427in}}%
\pgfpathlineto{\pgfqpoint{5.286518in}{2.702907in}}%
\pgfpathlineto{\pgfqpoint{5.293512in}{2.702907in}}%
\pgfpathlineto{\pgfqpoint{5.300506in}{2.719147in}}%
\pgfpathlineto{\pgfqpoint{5.307500in}{2.751628in}}%
\pgfpathlineto{\pgfqpoint{5.314494in}{2.767868in}}%
\pgfpathlineto{\pgfqpoint{5.328481in}{2.767868in}}%
\pgfpathlineto{\pgfqpoint{5.335475in}{2.800348in}}%
\pgfpathlineto{\pgfqpoint{5.342469in}{2.784108in}}%
\pgfpathlineto{\pgfqpoint{5.349462in}{2.800348in}}%
\pgfpathlineto{\pgfqpoint{5.356456in}{2.800348in}}%
\pgfpathlineto{\pgfqpoint{5.363450in}{2.816588in}}%
\pgfpathlineto{\pgfqpoint{5.370444in}{2.816588in}}%
\pgfpathlineto{\pgfqpoint{5.377438in}{2.832829in}}%
\pgfpathlineto{\pgfqpoint{5.391425in}{2.832829in}}%
\pgfpathlineto{\pgfqpoint{5.398419in}{2.865309in}}%
\pgfpathlineto{\pgfqpoint{5.405413in}{2.849069in}}%
\pgfpathlineto{\pgfqpoint{5.419400in}{2.849069in}}%
\pgfpathlineto{\pgfqpoint{5.426394in}{2.865309in}}%
\pgfpathlineto{\pgfqpoint{5.433388in}{2.832829in}}%
\pgfpathlineto{\pgfqpoint{5.440382in}{2.767868in}}%
\pgfpathlineto{\pgfqpoint{5.454369in}{2.670427in}}%
\pgfpathlineto{\pgfqpoint{5.461363in}{2.637946in}}%
\pgfpathlineto{\pgfqpoint{5.468357in}{2.589226in}}%
\pgfpathlineto{\pgfqpoint{5.496332in}{2.459304in}}%
\pgfpathlineto{\pgfqpoint{5.510320in}{2.426824in}}%
\pgfpathlineto{\pgfqpoint{5.517313in}{2.378103in}}%
\pgfpathlineto{\pgfqpoint{5.559276in}{2.280662in}}%
\pgfpathlineto{\pgfqpoint{5.566270in}{2.280662in}}%
\pgfpathlineto{\pgfqpoint{5.573264in}{2.248182in}}%
\pgfpathlineto{\pgfqpoint{5.580258in}{2.248182in}}%
\pgfpathlineto{\pgfqpoint{5.594245in}{2.215701in}}%
\pgfpathlineto{\pgfqpoint{5.601239in}{2.215701in}}%
\pgfpathlineto{\pgfqpoint{5.615227in}{2.183221in}}%
\pgfpathlineto{\pgfqpoint{5.636208in}{2.183221in}}%
\pgfpathlineto{\pgfqpoint{5.643202in}{2.166981in}}%
\pgfpathlineto{\pgfqpoint{5.657189in}{2.166981in}}%
\pgfpathlineto{\pgfqpoint{5.664183in}{2.150741in}}%
\pgfpathlineto{\pgfqpoint{5.678171in}{2.150741in}}%
\pgfpathlineto{\pgfqpoint{5.685165in}{2.134500in}}%
\pgfpathlineto{\pgfqpoint{5.692158in}{2.150741in}}%
\pgfpathlineto{\pgfqpoint{5.720133in}{2.345623in}}%
\pgfpathlineto{\pgfqpoint{5.727127in}{2.378103in}}%
\pgfpathlineto{\pgfqpoint{5.734121in}{2.426824in}}%
\pgfpathlineto{\pgfqpoint{5.748109in}{2.491785in}}%
\pgfpathlineto{\pgfqpoint{5.755102in}{2.508025in}}%
\pgfpathlineto{\pgfqpoint{5.762096in}{2.556745in}}%
\pgfpathlineto{\pgfqpoint{5.776084in}{2.589226in}}%
\pgfpathlineto{\pgfqpoint{5.783078in}{2.621706in}}%
\pgfpathlineto{\pgfqpoint{5.790071in}{2.637946in}}%
\pgfpathlineto{\pgfqpoint{5.797065in}{2.670427in}}%
\pgfpathlineto{\pgfqpoint{5.811053in}{2.702907in}}%
\pgfpathlineto{\pgfqpoint{5.818047in}{2.702907in}}%
\pgfpathlineto{\pgfqpoint{5.825040in}{2.719147in}}%
\pgfpathlineto{\pgfqpoint{5.832034in}{2.719147in}}%
\pgfpathlineto{\pgfqpoint{5.839028in}{2.735387in}}%
\pgfpathlineto{\pgfqpoint{5.846022in}{2.767868in}}%
\pgfpathlineto{\pgfqpoint{5.853016in}{2.767868in}}%
\pgfpathlineto{\pgfqpoint{5.860009in}{2.784108in}}%
\pgfpathlineto{\pgfqpoint{5.867003in}{2.784108in}}%
\pgfpathlineto{\pgfqpoint{5.873997in}{2.800348in}}%
\pgfpathlineto{\pgfqpoint{5.887985in}{2.800348in}}%
\pgfpathlineto{\pgfqpoint{5.894978in}{2.832829in}}%
\pgfpathlineto{\pgfqpoint{5.908966in}{2.832829in}}%
\pgfpathlineto{\pgfqpoint{5.915960in}{2.849069in}}%
\pgfpathlineto{\pgfqpoint{5.922953in}{2.832829in}}%
\pgfpathlineto{\pgfqpoint{5.929947in}{2.832829in}}%
\pgfpathlineto{\pgfqpoint{5.936941in}{2.849069in}}%
\pgfpathlineto{\pgfqpoint{5.943935in}{2.849069in}}%
\pgfpathlineto{\pgfqpoint{5.950929in}{2.865309in}}%
\pgfpathlineto{\pgfqpoint{5.978904in}{2.670427in}}%
\pgfpathlineto{\pgfqpoint{5.985898in}{2.637946in}}%
\pgfpathlineto{\pgfqpoint{5.992891in}{2.589226in}}%
\pgfpathlineto{\pgfqpoint{5.999885in}{2.556745in}}%
\pgfpathlineto{\pgfqpoint{6.006879in}{2.508025in}}%
\pgfpathlineto{\pgfqpoint{6.013873in}{2.475544in}}%
\pgfpathlineto{\pgfqpoint{6.027860in}{2.443064in}}%
\pgfpathlineto{\pgfqpoint{6.041848in}{2.378103in}}%
\pgfpathlineto{\pgfqpoint{6.048842in}{2.378103in}}%
\pgfpathlineto{\pgfqpoint{6.062829in}{2.313143in}}%
\pgfpathlineto{\pgfqpoint{6.069823in}{2.313143in}}%
\pgfpathlineto{\pgfqpoint{6.097798in}{2.248182in}}%
\pgfpathlineto{\pgfqpoint{6.104792in}{2.248182in}}%
\pgfpathlineto{\pgfqpoint{6.118780in}{2.215701in}}%
\pgfpathlineto{\pgfqpoint{6.125773in}{2.215701in}}%
\pgfpathlineto{\pgfqpoint{6.132767in}{2.199461in}}%
\pgfpathlineto{\pgfqpoint{6.139761in}{2.199461in}}%
\pgfpathlineto{\pgfqpoint{6.146755in}{2.183221in}}%
\pgfpathlineto{\pgfqpoint{6.160742in}{2.183221in}}%
\pgfpathlineto{\pgfqpoint{6.167736in}{2.166981in}}%
\pgfpathlineto{\pgfqpoint{6.174730in}{2.166981in}}%
\pgfpathlineto{\pgfqpoint{6.181724in}{2.150741in}}%
\pgfpathlineto{\pgfqpoint{6.181724in}{2.150741in}}%
\pgfusepath{stroke}%
\end{pgfscope}%
\begin{pgfscope}%
\pgfpathrectangle{\pgfqpoint{4.720371in}{2.061420in}}{\pgfqpoint{1.530941in}{0.893210in}}%
\pgfusepath{clip}%
\pgfsetrectcap%
\pgfsetroundjoin%
\pgfsetlinewidth{1.505625pt}%
\definecolor{currentstroke}{rgb}{0.839216,0.152941,0.156863}%
\pgfsetstrokecolor{currentstroke}%
\pgfsetdash{}{0pt}%
\pgfpathmoveto{\pgfqpoint{0.000000in}{0.000000in}}%
\pgfusepath{stroke}%
\end{pgfscope}%
\begin{pgfscope}%
\pgfsetrectcap%
\pgfsetmiterjoin%
\pgfsetlinewidth{0.803000pt}%
\definecolor{currentstroke}{rgb}{0.000000,0.000000,0.000000}%
\pgfsetstrokecolor{currentstroke}%
\pgfsetdash{}{0pt}%
\pgfpathmoveto{\pgfqpoint{4.720371in}{2.061420in}}%
\pgfpathlineto{\pgfqpoint{4.720371in}{2.954630in}}%
\pgfusepath{stroke}%
\end{pgfscope}%
\begin{pgfscope}%
\pgfsetrectcap%
\pgfsetmiterjoin%
\pgfsetlinewidth{0.803000pt}%
\definecolor{currentstroke}{rgb}{0.000000,0.000000,0.000000}%
\pgfsetstrokecolor{currentstroke}%
\pgfsetdash{}{0pt}%
\pgfpathmoveto{\pgfqpoint{6.251312in}{2.061420in}}%
\pgfpathlineto{\pgfqpoint{6.251312in}{2.954630in}}%
\pgfusepath{stroke}%
\end{pgfscope}%
\begin{pgfscope}%
\pgfsetrectcap%
\pgfsetmiterjoin%
\pgfsetlinewidth{0.803000pt}%
\definecolor{currentstroke}{rgb}{0.000000,0.000000,0.000000}%
\pgfsetstrokecolor{currentstroke}%
\pgfsetdash{}{0pt}%
\pgfpathmoveto{\pgfqpoint{4.720371in}{2.061420in}}%
\pgfpathlineto{\pgfqpoint{6.251312in}{2.061420in}}%
\pgfusepath{stroke}%
\end{pgfscope}%
\begin{pgfscope}%
\pgfsetrectcap%
\pgfsetmiterjoin%
\pgfsetlinewidth{0.803000pt}%
\definecolor{currentstroke}{rgb}{0.000000,0.000000,0.000000}%
\pgfsetstrokecolor{currentstroke}%
\pgfsetdash{}{0pt}%
\pgfpathmoveto{\pgfqpoint{4.720371in}{2.954630in}}%
\pgfpathlineto{\pgfqpoint{6.251312in}{2.954630in}}%
\pgfusepath{stroke}%
\end{pgfscope}%
\begin{pgfscope}%
\definecolor{textcolor}{rgb}{0.000000,0.000000,0.000000}%
\pgfsetstrokecolor{textcolor}%
\pgfsetfillcolor{textcolor}%
\pgftext[x=5.485841in,y=3.037963in,,base]{\color{textcolor}\rmfamily\fontsize{12.000000}{14.400000}\selectfont 5}%
\end{pgfscope}%
\begin{pgfscope}%
\pgfsetbuttcap%
\pgfsetmiterjoin%
\definecolor{currentfill}{rgb}{1.000000,1.000000,1.000000}%
\pgfsetfillcolor{currentfill}%
\pgfsetlinewidth{0.000000pt}%
\definecolor{currentstroke}{rgb}{0.000000,0.000000,0.000000}%
\pgfsetstrokecolor{currentstroke}%
\pgfsetstrokeopacity{0.000000}%
\pgfsetdash{}{0pt}%
\pgfpathmoveto{\pgfqpoint{0.500000in}{0.484568in}}%
\pgfpathlineto{\pgfqpoint{2.030942in}{0.484568in}}%
\pgfpathlineto{\pgfqpoint{2.030942in}{1.377778in}}%
\pgfpathlineto{\pgfqpoint{0.500000in}{1.377778in}}%
\pgfpathclose%
\pgfusepath{fill}%
\end{pgfscope}%
\begin{pgfscope}%
\pgfsetbuttcap%
\pgfsetroundjoin%
\definecolor{currentfill}{rgb}{0.000000,0.000000,0.000000}%
\pgfsetfillcolor{currentfill}%
\pgfsetlinewidth{0.803000pt}%
\definecolor{currentstroke}{rgb}{0.000000,0.000000,0.000000}%
\pgfsetstrokecolor{currentstroke}%
\pgfsetdash{}{0pt}%
\pgfsys@defobject{currentmarker}{\pgfqpoint{0.000000in}{-0.048611in}}{\pgfqpoint{0.000000in}{0.000000in}}{%
\pgfpathmoveto{\pgfqpoint{0.000000in}{0.000000in}}%
\pgfpathlineto{\pgfqpoint{0.000000in}{-0.048611in}}%
\pgfusepath{stroke,fill}%
}%
\begin{pgfscope}%
\pgfsys@transformshift{0.569589in}{0.484568in}%
\pgfsys@useobject{currentmarker}{}%
\end{pgfscope}%
\end{pgfscope}%
\begin{pgfscope}%
\definecolor{textcolor}{rgb}{0.000000,0.000000,0.000000}%
\pgfsetstrokecolor{textcolor}%
\pgfsetfillcolor{textcolor}%
\pgftext[x=0.569589in,y=0.387346in,,top]{\color{textcolor}\rmfamily\fontsize{10.000000}{12.000000}\selectfont \(\displaystyle {0}\)}%
\end{pgfscope}%
\begin{pgfscope}%
\pgfsetbuttcap%
\pgfsetroundjoin%
\definecolor{currentfill}{rgb}{0.000000,0.000000,0.000000}%
\pgfsetfillcolor{currentfill}%
\pgfsetlinewidth{0.803000pt}%
\definecolor{currentstroke}{rgb}{0.000000,0.000000,0.000000}%
\pgfsetstrokecolor{currentstroke}%
\pgfsetdash{}{0pt}%
\pgfsys@defobject{currentmarker}{\pgfqpoint{0.000000in}{-0.048611in}}{\pgfqpoint{0.000000in}{0.000000in}}{%
\pgfpathmoveto{\pgfqpoint{0.000000in}{0.000000in}}%
\pgfpathlineto{\pgfqpoint{0.000000in}{-0.048611in}}%
\pgfusepath{stroke,fill}%
}%
\begin{pgfscope}%
\pgfsys@transformshift{1.268968in}{0.484568in}%
\pgfsys@useobject{currentmarker}{}%
\end{pgfscope}%
\end{pgfscope}%
\begin{pgfscope}%
\definecolor{textcolor}{rgb}{0.000000,0.000000,0.000000}%
\pgfsetstrokecolor{textcolor}%
\pgfsetfillcolor{textcolor}%
\pgftext[x=1.268968in,y=0.387346in,,top]{\color{textcolor}\rmfamily\fontsize{10.000000}{12.000000}\selectfont \(\displaystyle {2000}\)}%
\end{pgfscope}%
\begin{pgfscope}%
\pgfsetbuttcap%
\pgfsetroundjoin%
\definecolor{currentfill}{rgb}{0.000000,0.000000,0.000000}%
\pgfsetfillcolor{currentfill}%
\pgfsetlinewidth{0.803000pt}%
\definecolor{currentstroke}{rgb}{0.000000,0.000000,0.000000}%
\pgfsetstrokecolor{currentstroke}%
\pgfsetdash{}{0pt}%
\pgfsys@defobject{currentmarker}{\pgfqpoint{0.000000in}{-0.048611in}}{\pgfqpoint{0.000000in}{0.000000in}}{%
\pgfpathmoveto{\pgfqpoint{0.000000in}{0.000000in}}%
\pgfpathlineto{\pgfqpoint{0.000000in}{-0.048611in}}%
\pgfusepath{stroke,fill}%
}%
\begin{pgfscope}%
\pgfsys@transformshift{1.968347in}{0.484568in}%
\pgfsys@useobject{currentmarker}{}%
\end{pgfscope}%
\end{pgfscope}%
\begin{pgfscope}%
\definecolor{textcolor}{rgb}{0.000000,0.000000,0.000000}%
\pgfsetstrokecolor{textcolor}%
\pgfsetfillcolor{textcolor}%
\pgftext[x=1.968347in,y=0.387346in,,top]{\color{textcolor}\rmfamily\fontsize{10.000000}{12.000000}\selectfont \(\displaystyle {4000}\)}%
\end{pgfscope}%
\begin{pgfscope}%
\definecolor{textcolor}{rgb}{0.000000,0.000000,0.000000}%
\pgfsetstrokecolor{textcolor}%
\pgfsetfillcolor{textcolor}%
\pgftext[x=1.265471in,y=0.208333in,,top]{\color{textcolor}\rmfamily\fontsize{10.000000}{12.000000}\selectfont Czas [\(\displaystyle \mu s\)] }%
\end{pgfscope}%
\begin{pgfscope}%
\pgfsetbuttcap%
\pgfsetroundjoin%
\definecolor{currentfill}{rgb}{0.000000,0.000000,0.000000}%
\pgfsetfillcolor{currentfill}%
\pgfsetlinewidth{0.803000pt}%
\definecolor{currentstroke}{rgb}{0.000000,0.000000,0.000000}%
\pgfsetstrokecolor{currentstroke}%
\pgfsetdash{}{0pt}%
\pgfsys@defobject{currentmarker}{\pgfqpoint{-0.048611in}{0.000000in}}{\pgfqpoint{-0.000000in}{0.000000in}}{%
\pgfpathmoveto{\pgfqpoint{-0.000000in}{0.000000in}}%
\pgfpathlineto{\pgfqpoint{-0.048611in}{0.000000in}}%
\pgfusepath{stroke,fill}%
}%
\begin{pgfscope}%
\pgfsys@transformshift{0.500000in}{0.541090in}%
\pgfsys@useobject{currentmarker}{}%
\end{pgfscope}%
\end{pgfscope}%
\begin{pgfscope}%
\definecolor{textcolor}{rgb}{0.000000,0.000000,0.000000}%
\pgfsetstrokecolor{textcolor}%
\pgfsetfillcolor{textcolor}%
\pgftext[x=0.333333in, y=0.492865in, left, base]{\color{textcolor}\rmfamily\fontsize{10.000000}{12.000000}\selectfont \(\displaystyle {0}\)}%
\end{pgfscope}%
\begin{pgfscope}%
\pgfsetbuttcap%
\pgfsetroundjoin%
\definecolor{currentfill}{rgb}{0.000000,0.000000,0.000000}%
\pgfsetfillcolor{currentfill}%
\pgfsetlinewidth{0.803000pt}%
\definecolor{currentstroke}{rgb}{0.000000,0.000000,0.000000}%
\pgfsetstrokecolor{currentstroke}%
\pgfsetdash{}{0pt}%
\pgfsys@defobject{currentmarker}{\pgfqpoint{-0.048611in}{0.000000in}}{\pgfqpoint{-0.000000in}{0.000000in}}{%
\pgfpathmoveto{\pgfqpoint{-0.000000in}{0.000000in}}%
\pgfpathlineto{\pgfqpoint{-0.048611in}{0.000000in}}%
\pgfusepath{stroke,fill}%
}%
\begin{pgfscope}%
\pgfsys@transformshift{0.500000in}{0.859525in}%
\pgfsys@useobject{currentmarker}{}%
\end{pgfscope}%
\end{pgfscope}%
\begin{pgfscope}%
\definecolor{textcolor}{rgb}{0.000000,0.000000,0.000000}%
\pgfsetstrokecolor{textcolor}%
\pgfsetfillcolor{textcolor}%
\pgftext[x=0.263889in, y=0.811300in, left, base]{\color{textcolor}\rmfamily\fontsize{10.000000}{12.000000}\selectfont \(\displaystyle {20}\)}%
\end{pgfscope}%
\begin{pgfscope}%
\pgfsetbuttcap%
\pgfsetroundjoin%
\definecolor{currentfill}{rgb}{0.000000,0.000000,0.000000}%
\pgfsetfillcolor{currentfill}%
\pgfsetlinewidth{0.803000pt}%
\definecolor{currentstroke}{rgb}{0.000000,0.000000,0.000000}%
\pgfsetstrokecolor{currentstroke}%
\pgfsetdash{}{0pt}%
\pgfsys@defobject{currentmarker}{\pgfqpoint{-0.048611in}{0.000000in}}{\pgfqpoint{-0.000000in}{0.000000in}}{%
\pgfpathmoveto{\pgfqpoint{-0.000000in}{0.000000in}}%
\pgfpathlineto{\pgfqpoint{-0.048611in}{0.000000in}}%
\pgfusepath{stroke,fill}%
}%
\begin{pgfscope}%
\pgfsys@transformshift{0.500000in}{1.177960in}%
\pgfsys@useobject{currentmarker}{}%
\end{pgfscope}%
\end{pgfscope}%
\begin{pgfscope}%
\definecolor{textcolor}{rgb}{0.000000,0.000000,0.000000}%
\pgfsetstrokecolor{textcolor}%
\pgfsetfillcolor{textcolor}%
\pgftext[x=0.263889in, y=1.129735in, left, base]{\color{textcolor}\rmfamily\fontsize{10.000000}{12.000000}\selectfont \(\displaystyle {40}\)}%
\end{pgfscope}%
\begin{pgfscope}%
\definecolor{textcolor}{rgb}{0.000000,0.000000,0.000000}%
\pgfsetstrokecolor{textcolor}%
\pgfsetfillcolor{textcolor}%
\pgftext[x=0.208333in,y=0.931173in,,bottom,rotate=90.000000]{\color{textcolor}\rmfamily\fontsize{10.000000}{12.000000}\selectfont Napiecie [\(\displaystyle V\)] }%
\end{pgfscope}%
\begin{pgfscope}%
\pgfpathrectangle{\pgfqpoint{0.500000in}{0.484568in}}{\pgfqpoint{1.530941in}{0.893210in}}%
\pgfusepath{clip}%
\pgfsetrectcap%
\pgfsetroundjoin%
\pgfsetlinewidth{1.505625pt}%
\definecolor{currentstroke}{rgb}{0.121569,0.466667,0.705882}%
\pgfsetstrokecolor{currentstroke}%
\pgfsetdash{}{0pt}%
\pgfpathmoveto{\pgfqpoint{0.569589in}{0.541090in}}%
\pgfpathlineto{\pgfqpoint{0.576582in}{1.337178in}}%
\pgfpathlineto{\pgfqpoint{0.583576in}{1.321256in}}%
\pgfpathlineto{\pgfqpoint{0.597564in}{1.321256in}}%
\pgfpathlineto{\pgfqpoint{0.604558in}{1.337178in}}%
\pgfpathlineto{\pgfqpoint{0.611551in}{1.321256in}}%
\pgfpathlineto{\pgfqpoint{0.618545in}{1.337178in}}%
\pgfpathlineto{\pgfqpoint{0.639527in}{1.337178in}}%
\pgfpathlineto{\pgfqpoint{0.646520in}{1.321256in}}%
\pgfpathlineto{\pgfqpoint{0.653514in}{1.337178in}}%
\pgfpathlineto{\pgfqpoint{0.667502in}{1.337178in}}%
\pgfpathlineto{\pgfqpoint{0.674496in}{1.321256in}}%
\pgfpathlineto{\pgfqpoint{0.681489in}{1.337178in}}%
\pgfpathlineto{\pgfqpoint{0.688483in}{0.588855in}}%
\pgfpathlineto{\pgfqpoint{0.695477in}{0.572934in}}%
\pgfpathlineto{\pgfqpoint{0.702471in}{0.572934in}}%
\pgfpathlineto{\pgfqpoint{0.709465in}{0.557012in}}%
\pgfpathlineto{\pgfqpoint{0.716458in}{0.572934in}}%
\pgfpathlineto{\pgfqpoint{0.723452in}{0.572934in}}%
\pgfpathlineto{\pgfqpoint{0.737440in}{0.541090in}}%
\pgfpathlineto{\pgfqpoint{0.744433in}{0.557012in}}%
\pgfpathlineto{\pgfqpoint{0.751427in}{0.541090in}}%
\pgfpathlineto{\pgfqpoint{0.758421in}{0.557012in}}%
\pgfpathlineto{\pgfqpoint{0.765415in}{0.541090in}}%
\pgfpathlineto{\pgfqpoint{0.779402in}{0.541090in}}%
\pgfpathlineto{\pgfqpoint{0.786396in}{0.557012in}}%
\pgfpathlineto{\pgfqpoint{0.793390in}{0.541090in}}%
\pgfpathlineto{\pgfqpoint{0.821365in}{0.541090in}}%
\pgfpathlineto{\pgfqpoint{0.828359in}{0.557012in}}%
\pgfpathlineto{\pgfqpoint{0.835353in}{0.541090in}}%
\pgfpathlineto{\pgfqpoint{0.933266in}{0.541090in}}%
\pgfpathlineto{\pgfqpoint{0.940260in}{0.525168in}}%
\pgfpathlineto{\pgfqpoint{0.947253in}{1.289412in}}%
\pgfpathlineto{\pgfqpoint{0.954247in}{1.305334in}}%
\pgfpathlineto{\pgfqpoint{0.989216in}{1.305334in}}%
\pgfpathlineto{\pgfqpoint{0.996210in}{1.321256in}}%
\pgfpathlineto{\pgfqpoint{1.017191in}{1.321256in}}%
\pgfpathlineto{\pgfqpoint{1.024185in}{1.305334in}}%
\pgfpathlineto{\pgfqpoint{1.031179in}{1.337178in}}%
\pgfpathlineto{\pgfqpoint{1.038173in}{1.321256in}}%
\pgfpathlineto{\pgfqpoint{1.045167in}{1.337178in}}%
\pgfpathlineto{\pgfqpoint{1.052160in}{1.305334in}}%
\pgfpathlineto{\pgfqpoint{1.059154in}{1.321256in}}%
\pgfpathlineto{\pgfqpoint{1.066148in}{1.321256in}}%
\pgfpathlineto{\pgfqpoint{1.073142in}{1.337178in}}%
\pgfpathlineto{\pgfqpoint{1.080136in}{1.321256in}}%
\pgfpathlineto{\pgfqpoint{1.087129in}{1.337178in}}%
\pgfpathlineto{\pgfqpoint{1.094123in}{1.321256in}}%
\pgfpathlineto{\pgfqpoint{1.101117in}{1.321256in}}%
\pgfpathlineto{\pgfqpoint{1.108111in}{1.337178in}}%
\pgfpathlineto{\pgfqpoint{1.115105in}{1.321256in}}%
\pgfpathlineto{\pgfqpoint{1.129092in}{1.321256in}}%
\pgfpathlineto{\pgfqpoint{1.136086in}{1.337178in}}%
\pgfpathlineto{\pgfqpoint{1.143080in}{1.321256in}}%
\pgfpathlineto{\pgfqpoint{1.150073in}{1.337178in}}%
\pgfpathlineto{\pgfqpoint{1.157067in}{1.337178in}}%
\pgfpathlineto{\pgfqpoint{1.164061in}{1.321256in}}%
\pgfpathlineto{\pgfqpoint{1.171055in}{1.321256in}}%
\pgfpathlineto{\pgfqpoint{1.178049in}{1.337178in}}%
\pgfpathlineto{\pgfqpoint{1.206024in}{1.337178in}}%
\pgfpathlineto{\pgfqpoint{1.213018in}{0.572934in}}%
\pgfpathlineto{\pgfqpoint{1.247987in}{0.572934in}}%
\pgfpathlineto{\pgfqpoint{1.254980in}{0.557012in}}%
\pgfpathlineto{\pgfqpoint{1.261974in}{0.557012in}}%
\pgfpathlineto{\pgfqpoint{1.268968in}{0.541090in}}%
\pgfpathlineto{\pgfqpoint{1.275962in}{0.557012in}}%
\pgfpathlineto{\pgfqpoint{1.289949in}{0.557012in}}%
\pgfpathlineto{\pgfqpoint{1.296943in}{0.541090in}}%
\pgfpathlineto{\pgfqpoint{1.303937in}{0.541090in}}%
\pgfpathlineto{\pgfqpoint{1.310931in}{0.557012in}}%
\pgfpathlineto{\pgfqpoint{1.317925in}{0.541090in}}%
\pgfpathlineto{\pgfqpoint{1.422831in}{0.541090in}}%
\pgfpathlineto{\pgfqpoint{1.429825in}{0.525168in}}%
\pgfpathlineto{\pgfqpoint{1.436819in}{0.541090in}}%
\pgfpathlineto{\pgfqpoint{1.464794in}{0.541090in}}%
\pgfpathlineto{\pgfqpoint{1.471788in}{1.289412in}}%
\pgfpathlineto{\pgfqpoint{1.478782in}{1.289412in}}%
\pgfpathlineto{\pgfqpoint{1.485776in}{1.305334in}}%
\pgfpathlineto{\pgfqpoint{1.520745in}{1.305334in}}%
\pgfpathlineto{\pgfqpoint{1.527738in}{1.321256in}}%
\pgfpathlineto{\pgfqpoint{1.534732in}{1.305334in}}%
\pgfpathlineto{\pgfqpoint{1.541726in}{1.305334in}}%
\pgfpathlineto{\pgfqpoint{1.548720in}{1.321256in}}%
\pgfpathlineto{\pgfqpoint{1.555713in}{1.305334in}}%
\pgfpathlineto{\pgfqpoint{1.562707in}{1.321256in}}%
\pgfpathlineto{\pgfqpoint{1.569701in}{1.305334in}}%
\pgfpathlineto{\pgfqpoint{1.576695in}{1.337178in}}%
\pgfpathlineto{\pgfqpoint{1.583689in}{1.321256in}}%
\pgfpathlineto{\pgfqpoint{1.604670in}{1.321256in}}%
\pgfpathlineto{\pgfqpoint{1.611664in}{1.337178in}}%
\pgfpathlineto{\pgfqpoint{1.618658in}{1.321256in}}%
\pgfpathlineto{\pgfqpoint{1.625651in}{1.321256in}}%
\pgfpathlineto{\pgfqpoint{1.632645in}{1.337178in}}%
\pgfpathlineto{\pgfqpoint{1.639639in}{1.337178in}}%
\pgfpathlineto{\pgfqpoint{1.646633in}{1.321256in}}%
\pgfpathlineto{\pgfqpoint{1.653627in}{1.337178in}}%
\pgfpathlineto{\pgfqpoint{1.660620in}{1.321256in}}%
\pgfpathlineto{\pgfqpoint{1.674608in}{1.321256in}}%
\pgfpathlineto{\pgfqpoint{1.681602in}{1.337178in}}%
\pgfpathlineto{\pgfqpoint{1.730558in}{1.337178in}}%
\pgfpathlineto{\pgfqpoint{1.737552in}{0.572934in}}%
\pgfpathlineto{\pgfqpoint{1.758533in}{0.572934in}}%
\pgfpathlineto{\pgfqpoint{1.765527in}{0.557012in}}%
\pgfpathlineto{\pgfqpoint{1.800496in}{0.557012in}}%
\pgfpathlineto{\pgfqpoint{1.807490in}{0.541090in}}%
\pgfpathlineto{\pgfqpoint{1.961353in}{0.541090in}}%
\pgfpathlineto{\pgfqpoint{1.961353in}{0.541090in}}%
\pgfusepath{stroke}%
\end{pgfscope}%
\begin{pgfscope}%
\pgfpathrectangle{\pgfqpoint{0.500000in}{0.484568in}}{\pgfqpoint{1.530941in}{0.893210in}}%
\pgfusepath{clip}%
\pgfsetrectcap%
\pgfsetroundjoin%
\pgfsetlinewidth{1.505625pt}%
\definecolor{currentstroke}{rgb}{1.000000,0.498039,0.054902}%
\pgfsetstrokecolor{currentstroke}%
\pgfsetdash{}{0pt}%
\pgfpathmoveto{\pgfqpoint{0.000000in}{0.000000in}}%
\pgfusepath{stroke}%
\end{pgfscope}%
\begin{pgfscope}%
\pgfpathrectangle{\pgfqpoint{0.500000in}{0.484568in}}{\pgfqpoint{1.530941in}{0.893210in}}%
\pgfusepath{clip}%
\pgfsetrectcap%
\pgfsetroundjoin%
\pgfsetlinewidth{1.505625pt}%
\definecolor{currentstroke}{rgb}{0.172549,0.627451,0.172549}%
\pgfsetstrokecolor{currentstroke}%
\pgfsetdash{}{0pt}%
\pgfpathmoveto{\pgfqpoint{0.569589in}{0.588855in}}%
\pgfpathlineto{\pgfqpoint{0.576582in}{1.193882in}}%
\pgfpathlineto{\pgfqpoint{0.583576in}{1.193882in}}%
\pgfpathlineto{\pgfqpoint{0.590570in}{1.209804in}}%
\pgfpathlineto{\pgfqpoint{0.597564in}{1.209804in}}%
\pgfpathlineto{\pgfqpoint{0.604558in}{1.225725in}}%
\pgfpathlineto{\pgfqpoint{0.611551in}{1.225725in}}%
\pgfpathlineto{\pgfqpoint{0.625539in}{1.257569in}}%
\pgfpathlineto{\pgfqpoint{0.632533in}{1.241647in}}%
\pgfpathlineto{\pgfqpoint{0.639527in}{1.257569in}}%
\pgfpathlineto{\pgfqpoint{0.646520in}{1.257569in}}%
\pgfpathlineto{\pgfqpoint{0.653514in}{1.273491in}}%
\pgfpathlineto{\pgfqpoint{0.660508in}{1.257569in}}%
\pgfpathlineto{\pgfqpoint{0.667502in}{1.257569in}}%
\pgfpathlineto{\pgfqpoint{0.674496in}{1.289412in}}%
\pgfpathlineto{\pgfqpoint{0.681489in}{1.289412in}}%
\pgfpathlineto{\pgfqpoint{0.695477in}{1.193882in}}%
\pgfpathlineto{\pgfqpoint{0.702471in}{1.130195in}}%
\pgfpathlineto{\pgfqpoint{0.716458in}{1.066508in}}%
\pgfpathlineto{\pgfqpoint{0.723452in}{1.018743in}}%
\pgfpathlineto{\pgfqpoint{0.730446in}{1.002821in}}%
\pgfpathlineto{\pgfqpoint{0.737440in}{0.955056in}}%
\pgfpathlineto{\pgfqpoint{0.751427in}{0.891369in}}%
\pgfpathlineto{\pgfqpoint{0.758421in}{0.875447in}}%
\pgfpathlineto{\pgfqpoint{0.765415in}{0.843603in}}%
\pgfpathlineto{\pgfqpoint{0.779402in}{0.811760in}}%
\pgfpathlineto{\pgfqpoint{0.786396in}{0.779916in}}%
\pgfpathlineto{\pgfqpoint{0.828359in}{0.684386in}}%
\pgfpathlineto{\pgfqpoint{0.835353in}{0.684386in}}%
\pgfpathlineto{\pgfqpoint{0.849340in}{0.652542in}}%
\pgfpathlineto{\pgfqpoint{0.856334in}{0.652542in}}%
\pgfpathlineto{\pgfqpoint{0.870322in}{0.620699in}}%
\pgfpathlineto{\pgfqpoint{0.898297in}{0.620699in}}%
\pgfpathlineto{\pgfqpoint{0.905291in}{0.604777in}}%
\pgfpathlineto{\pgfqpoint{0.912285in}{0.604777in}}%
\pgfpathlineto{\pgfqpoint{0.919278in}{0.588855in}}%
\pgfpathlineto{\pgfqpoint{0.947253in}{0.588855in}}%
\pgfpathlineto{\pgfqpoint{0.975229in}{0.779916in}}%
\pgfpathlineto{\pgfqpoint{0.982222in}{0.811760in}}%
\pgfpathlineto{\pgfqpoint{0.989216in}{0.875447in}}%
\pgfpathlineto{\pgfqpoint{0.996210in}{0.891369in}}%
\pgfpathlineto{\pgfqpoint{1.003204in}{0.939134in}}%
\pgfpathlineto{\pgfqpoint{1.010198in}{0.955056in}}%
\pgfpathlineto{\pgfqpoint{1.017191in}{0.986899in}}%
\pgfpathlineto{\pgfqpoint{1.024185in}{1.002821in}}%
\pgfpathlineto{\pgfqpoint{1.031179in}{1.002821in}}%
\pgfpathlineto{\pgfqpoint{1.038173in}{1.050586in}}%
\pgfpathlineto{\pgfqpoint{1.045167in}{1.066508in}}%
\pgfpathlineto{\pgfqpoint{1.059154in}{1.130195in}}%
\pgfpathlineto{\pgfqpoint{1.066148in}{1.130195in}}%
\pgfpathlineto{\pgfqpoint{1.073142in}{1.146117in}}%
\pgfpathlineto{\pgfqpoint{1.080136in}{1.146117in}}%
\pgfpathlineto{\pgfqpoint{1.087129in}{1.162038in}}%
\pgfpathlineto{\pgfqpoint{1.094123in}{1.193882in}}%
\pgfpathlineto{\pgfqpoint{1.108111in}{1.193882in}}%
\pgfpathlineto{\pgfqpoint{1.115105in}{1.209804in}}%
\pgfpathlineto{\pgfqpoint{1.122098in}{1.209804in}}%
\pgfpathlineto{\pgfqpoint{1.129092in}{1.225725in}}%
\pgfpathlineto{\pgfqpoint{1.136086in}{1.225725in}}%
\pgfpathlineto{\pgfqpoint{1.150073in}{1.257569in}}%
\pgfpathlineto{\pgfqpoint{1.185042in}{1.257569in}}%
\pgfpathlineto{\pgfqpoint{1.199030in}{1.289412in}}%
\pgfpathlineto{\pgfqpoint{1.206024in}{1.289412in}}%
\pgfpathlineto{\pgfqpoint{1.233999in}{1.098351in}}%
\pgfpathlineto{\pgfqpoint{1.240993in}{1.066508in}}%
\pgfpathlineto{\pgfqpoint{1.247987in}{1.018743in}}%
\pgfpathlineto{\pgfqpoint{1.254980in}{0.986899in}}%
\pgfpathlineto{\pgfqpoint{1.261974in}{0.939134in}}%
\pgfpathlineto{\pgfqpoint{1.268968in}{0.939134in}}%
\pgfpathlineto{\pgfqpoint{1.275962in}{0.891369in}}%
\pgfpathlineto{\pgfqpoint{1.289949in}{0.859525in}}%
\pgfpathlineto{\pgfqpoint{1.296943in}{0.827682in}}%
\pgfpathlineto{\pgfqpoint{1.303937in}{0.811760in}}%
\pgfpathlineto{\pgfqpoint{1.310931in}{0.779916in}}%
\pgfpathlineto{\pgfqpoint{1.324918in}{0.748073in}}%
\pgfpathlineto{\pgfqpoint{1.331912in}{0.716229in}}%
\pgfpathlineto{\pgfqpoint{1.338906in}{0.716229in}}%
\pgfpathlineto{\pgfqpoint{1.352893in}{0.684386in}}%
\pgfpathlineto{\pgfqpoint{1.359887in}{0.684386in}}%
\pgfpathlineto{\pgfqpoint{1.373875in}{0.652542in}}%
\pgfpathlineto{\pgfqpoint{1.387862in}{0.652542in}}%
\pgfpathlineto{\pgfqpoint{1.401850in}{0.620699in}}%
\pgfpathlineto{\pgfqpoint{1.422831in}{0.620699in}}%
\pgfpathlineto{\pgfqpoint{1.429825in}{0.604777in}}%
\pgfpathlineto{\pgfqpoint{1.443813in}{0.604777in}}%
\pgfpathlineto{\pgfqpoint{1.457800in}{0.572934in}}%
\pgfpathlineto{\pgfqpoint{1.464794in}{0.572934in}}%
\pgfpathlineto{\pgfqpoint{1.471788in}{0.588855in}}%
\pgfpathlineto{\pgfqpoint{1.478782in}{0.620699in}}%
\pgfpathlineto{\pgfqpoint{1.485776in}{0.684386in}}%
\pgfpathlineto{\pgfqpoint{1.492769in}{0.732151in}}%
\pgfpathlineto{\pgfqpoint{1.499763in}{0.795838in}}%
\pgfpathlineto{\pgfqpoint{1.513751in}{0.859525in}}%
\pgfpathlineto{\pgfqpoint{1.520745in}{0.875447in}}%
\pgfpathlineto{\pgfqpoint{1.527738in}{0.939134in}}%
\pgfpathlineto{\pgfqpoint{1.534732in}{0.939134in}}%
\pgfpathlineto{\pgfqpoint{1.548720in}{1.002821in}}%
\pgfpathlineto{\pgfqpoint{1.555713in}{1.018743in}}%
\pgfpathlineto{\pgfqpoint{1.562707in}{1.066508in}}%
\pgfpathlineto{\pgfqpoint{1.569701in}{1.066508in}}%
\pgfpathlineto{\pgfqpoint{1.576695in}{1.098351in}}%
\pgfpathlineto{\pgfqpoint{1.583689in}{1.114273in}}%
\pgfpathlineto{\pgfqpoint{1.590682in}{1.114273in}}%
\pgfpathlineto{\pgfqpoint{1.597676in}{1.146117in}}%
\pgfpathlineto{\pgfqpoint{1.604670in}{1.146117in}}%
\pgfpathlineto{\pgfqpoint{1.632645in}{1.209804in}}%
\pgfpathlineto{\pgfqpoint{1.646633in}{1.209804in}}%
\pgfpathlineto{\pgfqpoint{1.660620in}{1.241647in}}%
\pgfpathlineto{\pgfqpoint{1.667614in}{1.241647in}}%
\pgfpathlineto{\pgfqpoint{1.674608in}{1.257569in}}%
\pgfpathlineto{\pgfqpoint{1.709577in}{1.257569in}}%
\pgfpathlineto{\pgfqpoint{1.716571in}{1.273491in}}%
\pgfpathlineto{\pgfqpoint{1.723565in}{1.273491in}}%
\pgfpathlineto{\pgfqpoint{1.730558in}{1.289412in}}%
\pgfpathlineto{\pgfqpoint{1.744546in}{1.193882in}}%
\pgfpathlineto{\pgfqpoint{1.751540in}{1.130195in}}%
\pgfpathlineto{\pgfqpoint{1.765527in}{1.066508in}}%
\pgfpathlineto{\pgfqpoint{1.772521in}{1.002821in}}%
\pgfpathlineto{\pgfqpoint{1.779515in}{0.986899in}}%
\pgfpathlineto{\pgfqpoint{1.786509in}{0.955056in}}%
\pgfpathlineto{\pgfqpoint{1.793502in}{0.939134in}}%
\pgfpathlineto{\pgfqpoint{1.800496in}{0.891369in}}%
\pgfpathlineto{\pgfqpoint{1.814484in}{0.859525in}}%
\pgfpathlineto{\pgfqpoint{1.821478in}{0.811760in}}%
\pgfpathlineto{\pgfqpoint{1.828471in}{0.811760in}}%
\pgfpathlineto{\pgfqpoint{1.842459in}{0.748073in}}%
\pgfpathlineto{\pgfqpoint{1.849453in}{0.732151in}}%
\pgfpathlineto{\pgfqpoint{1.856447in}{0.732151in}}%
\pgfpathlineto{\pgfqpoint{1.863440in}{0.716229in}}%
\pgfpathlineto{\pgfqpoint{1.870434in}{0.684386in}}%
\pgfpathlineto{\pgfqpoint{1.891416in}{0.684386in}}%
\pgfpathlineto{\pgfqpoint{1.905403in}{0.652542in}}%
\pgfpathlineto{\pgfqpoint{1.912397in}{0.620699in}}%
\pgfpathlineto{\pgfqpoint{1.940372in}{0.620699in}}%
\pgfpathlineto{\pgfqpoint{1.947366in}{0.604777in}}%
\pgfpathlineto{\pgfqpoint{1.961353in}{0.604777in}}%
\pgfpathlineto{\pgfqpoint{1.961353in}{0.604777in}}%
\pgfusepath{stroke}%
\end{pgfscope}%
\begin{pgfscope}%
\pgfpathrectangle{\pgfqpoint{0.500000in}{0.484568in}}{\pgfqpoint{1.530941in}{0.893210in}}%
\pgfusepath{clip}%
\pgfsetrectcap%
\pgfsetroundjoin%
\pgfsetlinewidth{1.505625pt}%
\definecolor{currentstroke}{rgb}{0.839216,0.152941,0.156863}%
\pgfsetstrokecolor{currentstroke}%
\pgfsetdash{}{0pt}%
\pgfpathmoveto{\pgfqpoint{0.000000in}{0.000000in}}%
\pgfusepath{stroke}%
\end{pgfscope}%
\begin{pgfscope}%
\pgfsetrectcap%
\pgfsetmiterjoin%
\pgfsetlinewidth{0.803000pt}%
\definecolor{currentstroke}{rgb}{0.000000,0.000000,0.000000}%
\pgfsetstrokecolor{currentstroke}%
\pgfsetdash{}{0pt}%
\pgfpathmoveto{\pgfqpoint{0.500000in}{0.484568in}}%
\pgfpathlineto{\pgfqpoint{0.500000in}{1.377778in}}%
\pgfusepath{stroke}%
\end{pgfscope}%
\begin{pgfscope}%
\pgfsetrectcap%
\pgfsetmiterjoin%
\pgfsetlinewidth{0.803000pt}%
\definecolor{currentstroke}{rgb}{0.000000,0.000000,0.000000}%
\pgfsetstrokecolor{currentstroke}%
\pgfsetdash{}{0pt}%
\pgfpathmoveto{\pgfqpoint{2.030942in}{0.484568in}}%
\pgfpathlineto{\pgfqpoint{2.030942in}{1.377778in}}%
\pgfusepath{stroke}%
\end{pgfscope}%
\begin{pgfscope}%
\pgfsetrectcap%
\pgfsetmiterjoin%
\pgfsetlinewidth{0.803000pt}%
\definecolor{currentstroke}{rgb}{0.000000,0.000000,0.000000}%
\pgfsetstrokecolor{currentstroke}%
\pgfsetdash{}{0pt}%
\pgfpathmoveto{\pgfqpoint{0.500000in}{0.484568in}}%
\pgfpathlineto{\pgfqpoint{2.030942in}{0.484568in}}%
\pgfusepath{stroke}%
\end{pgfscope}%
\begin{pgfscope}%
\pgfsetrectcap%
\pgfsetmiterjoin%
\pgfsetlinewidth{0.803000pt}%
\definecolor{currentstroke}{rgb}{0.000000,0.000000,0.000000}%
\pgfsetstrokecolor{currentstroke}%
\pgfsetdash{}{0pt}%
\pgfpathmoveto{\pgfqpoint{0.500000in}{1.377778in}}%
\pgfpathlineto{\pgfqpoint{2.030942in}{1.377778in}}%
\pgfusepath{stroke}%
\end{pgfscope}%
\begin{pgfscope}%
\definecolor{textcolor}{rgb}{0.000000,0.000000,0.000000}%
\pgfsetstrokecolor{textcolor}%
\pgfsetfillcolor{textcolor}%
\pgftext[x=1.265471in,y=1.461111in,,base]{\color{textcolor}\rmfamily\fontsize{12.000000}{14.400000}\selectfont 6}%
\end{pgfscope}%
\begin{pgfscope}%
\pgfsetbuttcap%
\pgfsetmiterjoin%
\definecolor{currentfill}{rgb}{1.000000,1.000000,1.000000}%
\pgfsetfillcolor{currentfill}%
\pgfsetlinewidth{0.000000pt}%
\definecolor{currentstroke}{rgb}{0.000000,0.000000,0.000000}%
\pgfsetstrokecolor{currentstroke}%
\pgfsetstrokeopacity{0.000000}%
\pgfsetdash{}{0pt}%
\pgfpathmoveto{\pgfqpoint{2.610186in}{0.484568in}}%
\pgfpathlineto{\pgfqpoint{4.141127in}{0.484568in}}%
\pgfpathlineto{\pgfqpoint{4.141127in}{1.377778in}}%
\pgfpathlineto{\pgfqpoint{2.610186in}{1.377778in}}%
\pgfpathclose%
\pgfusepath{fill}%
\end{pgfscope}%
\begin{pgfscope}%
\pgfsetbuttcap%
\pgfsetroundjoin%
\definecolor{currentfill}{rgb}{0.000000,0.000000,0.000000}%
\pgfsetfillcolor{currentfill}%
\pgfsetlinewidth{0.803000pt}%
\definecolor{currentstroke}{rgb}{0.000000,0.000000,0.000000}%
\pgfsetstrokecolor{currentstroke}%
\pgfsetdash{}{0pt}%
\pgfsys@defobject{currentmarker}{\pgfqpoint{0.000000in}{-0.048611in}}{\pgfqpoint{0.000000in}{0.000000in}}{%
\pgfpathmoveto{\pgfqpoint{0.000000in}{0.000000in}}%
\pgfpathlineto{\pgfqpoint{0.000000in}{-0.048611in}}%
\pgfusepath{stroke,fill}%
}%
\begin{pgfscope}%
\pgfsys@transformshift{2.679774in}{0.484568in}%
\pgfsys@useobject{currentmarker}{}%
\end{pgfscope}%
\end{pgfscope}%
\begin{pgfscope}%
\definecolor{textcolor}{rgb}{0.000000,0.000000,0.000000}%
\pgfsetstrokecolor{textcolor}%
\pgfsetfillcolor{textcolor}%
\pgftext[x=2.679774in,y=0.387346in,,top]{\color{textcolor}\rmfamily\fontsize{10.000000}{12.000000}\selectfont \(\displaystyle {0}\)}%
\end{pgfscope}%
\begin{pgfscope}%
\pgfsetbuttcap%
\pgfsetroundjoin%
\definecolor{currentfill}{rgb}{0.000000,0.000000,0.000000}%
\pgfsetfillcolor{currentfill}%
\pgfsetlinewidth{0.803000pt}%
\definecolor{currentstroke}{rgb}{0.000000,0.000000,0.000000}%
\pgfsetstrokecolor{currentstroke}%
\pgfsetdash{}{0pt}%
\pgfsys@defobject{currentmarker}{\pgfqpoint{0.000000in}{-0.048611in}}{\pgfqpoint{0.000000in}{0.000000in}}{%
\pgfpathmoveto{\pgfqpoint{0.000000in}{0.000000in}}%
\pgfpathlineto{\pgfqpoint{0.000000in}{-0.048611in}}%
\pgfusepath{stroke,fill}%
}%
\begin{pgfscope}%
\pgfsys@transformshift{3.379153in}{0.484568in}%
\pgfsys@useobject{currentmarker}{}%
\end{pgfscope}%
\end{pgfscope}%
\begin{pgfscope}%
\definecolor{textcolor}{rgb}{0.000000,0.000000,0.000000}%
\pgfsetstrokecolor{textcolor}%
\pgfsetfillcolor{textcolor}%
\pgftext[x=3.379153in,y=0.387346in,,top]{\color{textcolor}\rmfamily\fontsize{10.000000}{12.000000}\selectfont \(\displaystyle {2000}\)}%
\end{pgfscope}%
\begin{pgfscope}%
\pgfsetbuttcap%
\pgfsetroundjoin%
\definecolor{currentfill}{rgb}{0.000000,0.000000,0.000000}%
\pgfsetfillcolor{currentfill}%
\pgfsetlinewidth{0.803000pt}%
\definecolor{currentstroke}{rgb}{0.000000,0.000000,0.000000}%
\pgfsetstrokecolor{currentstroke}%
\pgfsetdash{}{0pt}%
\pgfsys@defobject{currentmarker}{\pgfqpoint{0.000000in}{-0.048611in}}{\pgfqpoint{0.000000in}{0.000000in}}{%
\pgfpathmoveto{\pgfqpoint{0.000000in}{0.000000in}}%
\pgfpathlineto{\pgfqpoint{0.000000in}{-0.048611in}}%
\pgfusepath{stroke,fill}%
}%
\begin{pgfscope}%
\pgfsys@transformshift{4.078532in}{0.484568in}%
\pgfsys@useobject{currentmarker}{}%
\end{pgfscope}%
\end{pgfscope}%
\begin{pgfscope}%
\definecolor{textcolor}{rgb}{0.000000,0.000000,0.000000}%
\pgfsetstrokecolor{textcolor}%
\pgfsetfillcolor{textcolor}%
\pgftext[x=4.078532in,y=0.387346in,,top]{\color{textcolor}\rmfamily\fontsize{10.000000}{12.000000}\selectfont \(\displaystyle {4000}\)}%
\end{pgfscope}%
\begin{pgfscope}%
\definecolor{textcolor}{rgb}{0.000000,0.000000,0.000000}%
\pgfsetstrokecolor{textcolor}%
\pgfsetfillcolor{textcolor}%
\pgftext[x=3.375656in,y=0.208333in,,top]{\color{textcolor}\rmfamily\fontsize{10.000000}{12.000000}\selectfont Czas [\(\displaystyle \mu s\)] }%
\end{pgfscope}%
\begin{pgfscope}%
\pgfsetbuttcap%
\pgfsetroundjoin%
\definecolor{currentfill}{rgb}{0.000000,0.000000,0.000000}%
\pgfsetfillcolor{currentfill}%
\pgfsetlinewidth{0.803000pt}%
\definecolor{currentstroke}{rgb}{0.000000,0.000000,0.000000}%
\pgfsetstrokecolor{currentstroke}%
\pgfsetdash{}{0pt}%
\pgfsys@defobject{currentmarker}{\pgfqpoint{-0.048611in}{0.000000in}}{\pgfqpoint{-0.000000in}{0.000000in}}{%
\pgfpathmoveto{\pgfqpoint{-0.000000in}{0.000000in}}%
\pgfpathlineto{\pgfqpoint{-0.048611in}{0.000000in}}%
\pgfusepath{stroke,fill}%
}%
\begin{pgfscope}%
\pgfsys@transformshift{2.610186in}{0.541090in}%
\pgfsys@useobject{currentmarker}{}%
\end{pgfscope}%
\end{pgfscope}%
\begin{pgfscope}%
\definecolor{textcolor}{rgb}{0.000000,0.000000,0.000000}%
\pgfsetstrokecolor{textcolor}%
\pgfsetfillcolor{textcolor}%
\pgftext[x=2.443519in, y=0.492865in, left, base]{\color{textcolor}\rmfamily\fontsize{10.000000}{12.000000}\selectfont \(\displaystyle {0}\)}%
\end{pgfscope}%
\begin{pgfscope}%
\pgfsetbuttcap%
\pgfsetroundjoin%
\definecolor{currentfill}{rgb}{0.000000,0.000000,0.000000}%
\pgfsetfillcolor{currentfill}%
\pgfsetlinewidth{0.803000pt}%
\definecolor{currentstroke}{rgb}{0.000000,0.000000,0.000000}%
\pgfsetstrokecolor{currentstroke}%
\pgfsetdash{}{0pt}%
\pgfsys@defobject{currentmarker}{\pgfqpoint{-0.048611in}{0.000000in}}{\pgfqpoint{-0.000000in}{0.000000in}}{%
\pgfpathmoveto{\pgfqpoint{-0.000000in}{0.000000in}}%
\pgfpathlineto{\pgfqpoint{-0.048611in}{0.000000in}}%
\pgfusepath{stroke,fill}%
}%
\begin{pgfscope}%
\pgfsys@transformshift{2.610186in}{0.859525in}%
\pgfsys@useobject{currentmarker}{}%
\end{pgfscope}%
\end{pgfscope}%
\begin{pgfscope}%
\definecolor{textcolor}{rgb}{0.000000,0.000000,0.000000}%
\pgfsetstrokecolor{textcolor}%
\pgfsetfillcolor{textcolor}%
\pgftext[x=2.374074in, y=0.811300in, left, base]{\color{textcolor}\rmfamily\fontsize{10.000000}{12.000000}\selectfont \(\displaystyle {20}\)}%
\end{pgfscope}%
\begin{pgfscope}%
\pgfsetbuttcap%
\pgfsetroundjoin%
\definecolor{currentfill}{rgb}{0.000000,0.000000,0.000000}%
\pgfsetfillcolor{currentfill}%
\pgfsetlinewidth{0.803000pt}%
\definecolor{currentstroke}{rgb}{0.000000,0.000000,0.000000}%
\pgfsetstrokecolor{currentstroke}%
\pgfsetdash{}{0pt}%
\pgfsys@defobject{currentmarker}{\pgfqpoint{-0.048611in}{0.000000in}}{\pgfqpoint{-0.000000in}{0.000000in}}{%
\pgfpathmoveto{\pgfqpoint{-0.000000in}{0.000000in}}%
\pgfpathlineto{\pgfqpoint{-0.048611in}{0.000000in}}%
\pgfusepath{stroke,fill}%
}%
\begin{pgfscope}%
\pgfsys@transformshift{2.610186in}{1.177960in}%
\pgfsys@useobject{currentmarker}{}%
\end{pgfscope}%
\end{pgfscope}%
\begin{pgfscope}%
\definecolor{textcolor}{rgb}{0.000000,0.000000,0.000000}%
\pgfsetstrokecolor{textcolor}%
\pgfsetfillcolor{textcolor}%
\pgftext[x=2.374074in, y=1.129735in, left, base]{\color{textcolor}\rmfamily\fontsize{10.000000}{12.000000}\selectfont \(\displaystyle {40}\)}%
\end{pgfscope}%
\begin{pgfscope}%
\definecolor{textcolor}{rgb}{0.000000,0.000000,0.000000}%
\pgfsetstrokecolor{textcolor}%
\pgfsetfillcolor{textcolor}%
\pgftext[x=2.318519in,y=0.931173in,,bottom,rotate=90.000000]{\color{textcolor}\rmfamily\fontsize{10.000000}{12.000000}\selectfont Napiecie [\(\displaystyle V\)] }%
\end{pgfscope}%
\begin{pgfscope}%
\pgfpathrectangle{\pgfqpoint{2.610186in}{0.484568in}}{\pgfqpoint{1.530941in}{0.893210in}}%
\pgfusepath{clip}%
\pgfsetrectcap%
\pgfsetroundjoin%
\pgfsetlinewidth{1.505625pt}%
\definecolor{currentstroke}{rgb}{0.121569,0.466667,0.705882}%
\pgfsetstrokecolor{currentstroke}%
\pgfsetdash{}{0pt}%
\pgfpathmoveto{\pgfqpoint{2.679774in}{0.525168in}}%
\pgfpathlineto{\pgfqpoint{2.686768in}{0.572934in}}%
\pgfpathlineto{\pgfqpoint{2.693761in}{0.557012in}}%
\pgfpathlineto{\pgfqpoint{2.714743in}{0.557012in}}%
\pgfpathlineto{\pgfqpoint{2.721737in}{0.541090in}}%
\pgfpathlineto{\pgfqpoint{2.728730in}{0.557012in}}%
\pgfpathlineto{\pgfqpoint{2.735724in}{0.541090in}}%
\pgfpathlineto{\pgfqpoint{2.798668in}{0.541090in}}%
\pgfpathlineto{\pgfqpoint{2.805662in}{0.557012in}}%
\pgfpathlineto{\pgfqpoint{2.812656in}{0.541090in}}%
\pgfpathlineto{\pgfqpoint{2.882594in}{0.541090in}}%
\pgfpathlineto{\pgfqpoint{2.889588in}{0.525168in}}%
\pgfpathlineto{\pgfqpoint{2.896581in}{0.541090in}}%
\pgfpathlineto{\pgfqpoint{2.903575in}{0.525168in}}%
\pgfpathlineto{\pgfqpoint{2.910569in}{0.541090in}}%
\pgfpathlineto{\pgfqpoint{2.917563in}{0.525168in}}%
\pgfpathlineto{\pgfqpoint{2.931550in}{0.525168in}}%
\pgfpathlineto{\pgfqpoint{2.938544in}{0.541090in}}%
\pgfpathlineto{\pgfqpoint{2.959526in}{0.541090in}}%
\pgfpathlineto{\pgfqpoint{2.966519in}{0.525168in}}%
\pgfpathlineto{\pgfqpoint{2.973513in}{1.273491in}}%
\pgfpathlineto{\pgfqpoint{2.980507in}{1.273491in}}%
\pgfpathlineto{\pgfqpoint{2.987501in}{1.305334in}}%
\pgfpathlineto{\pgfqpoint{2.994495in}{1.289412in}}%
\pgfpathlineto{\pgfqpoint{3.001488in}{1.305334in}}%
\pgfpathlineto{\pgfqpoint{3.057439in}{1.305334in}}%
\pgfpathlineto{\pgfqpoint{3.064432in}{1.321256in}}%
\pgfpathlineto{\pgfqpoint{3.071426in}{1.305334in}}%
\pgfpathlineto{\pgfqpoint{3.078420in}{1.321256in}}%
\pgfpathlineto{\pgfqpoint{3.113389in}{1.321256in}}%
\pgfpathlineto{\pgfqpoint{3.120383in}{1.337178in}}%
\pgfpathlineto{\pgfqpoint{3.127377in}{1.321256in}}%
\pgfpathlineto{\pgfqpoint{3.134370in}{1.337178in}}%
\pgfpathlineto{\pgfqpoint{3.141364in}{1.321256in}}%
\pgfpathlineto{\pgfqpoint{3.148358in}{1.321256in}}%
\pgfpathlineto{\pgfqpoint{3.155352in}{1.337178in}}%
\pgfpathlineto{\pgfqpoint{3.162346in}{1.337178in}}%
\pgfpathlineto{\pgfqpoint{3.169339in}{1.321256in}}%
\pgfpathlineto{\pgfqpoint{3.176333in}{1.337178in}}%
\pgfpathlineto{\pgfqpoint{3.197315in}{1.337178in}}%
\pgfpathlineto{\pgfqpoint{3.204308in}{1.321256in}}%
\pgfpathlineto{\pgfqpoint{3.211302in}{1.337178in}}%
\pgfpathlineto{\pgfqpoint{3.218296in}{1.321256in}}%
\pgfpathlineto{\pgfqpoint{3.225290in}{1.337178in}}%
\pgfpathlineto{\pgfqpoint{3.253265in}{1.337178in}}%
\pgfpathlineto{\pgfqpoint{3.260259in}{1.321256in}}%
\pgfpathlineto{\pgfqpoint{3.267252in}{1.337178in}}%
\pgfpathlineto{\pgfqpoint{3.274246in}{1.321256in}}%
\pgfpathlineto{\pgfqpoint{3.281240in}{0.572934in}}%
\pgfpathlineto{\pgfqpoint{3.295228in}{0.572934in}}%
\pgfpathlineto{\pgfqpoint{3.302221in}{0.557012in}}%
\pgfpathlineto{\pgfqpoint{3.309215in}{0.572934in}}%
\pgfpathlineto{\pgfqpoint{3.316209in}{0.572934in}}%
\pgfpathlineto{\pgfqpoint{3.330197in}{0.541090in}}%
\pgfpathlineto{\pgfqpoint{3.337190in}{0.557012in}}%
\pgfpathlineto{\pgfqpoint{3.344184in}{0.541090in}}%
\pgfpathlineto{\pgfqpoint{3.351178in}{0.557012in}}%
\pgfpathlineto{\pgfqpoint{3.358172in}{0.541090in}}%
\pgfpathlineto{\pgfqpoint{3.372159in}{0.541090in}}%
\pgfpathlineto{\pgfqpoint{3.379153in}{0.557012in}}%
\pgfpathlineto{\pgfqpoint{3.386147in}{0.541090in}}%
\pgfpathlineto{\pgfqpoint{3.547004in}{0.541090in}}%
\pgfpathlineto{\pgfqpoint{3.553998in}{0.525168in}}%
\pgfpathlineto{\pgfqpoint{3.560992in}{0.541090in}}%
\pgfpathlineto{\pgfqpoint{3.567986in}{0.525168in}}%
\pgfpathlineto{\pgfqpoint{3.574979in}{0.541090in}}%
\pgfpathlineto{\pgfqpoint{3.581973in}{1.273491in}}%
\pgfpathlineto{\pgfqpoint{3.595961in}{1.305334in}}%
\pgfpathlineto{\pgfqpoint{3.623936in}{1.305334in}}%
\pgfpathlineto{\pgfqpoint{3.630930in}{1.321256in}}%
\pgfpathlineto{\pgfqpoint{3.637923in}{1.305334in}}%
\pgfpathlineto{\pgfqpoint{3.658905in}{1.305334in}}%
\pgfpathlineto{\pgfqpoint{3.665899in}{1.321256in}}%
\pgfpathlineto{\pgfqpoint{3.707861in}{1.321256in}}%
\pgfpathlineto{\pgfqpoint{3.714855in}{1.337178in}}%
\pgfpathlineto{\pgfqpoint{3.721849in}{1.321256in}}%
\pgfpathlineto{\pgfqpoint{3.728843in}{1.321256in}}%
\pgfpathlineto{\pgfqpoint{3.735837in}{1.337178in}}%
\pgfpathlineto{\pgfqpoint{3.742830in}{1.321256in}}%
\pgfpathlineto{\pgfqpoint{3.749824in}{1.337178in}}%
\pgfpathlineto{\pgfqpoint{3.756818in}{1.337178in}}%
\pgfpathlineto{\pgfqpoint{3.763812in}{1.321256in}}%
\pgfpathlineto{\pgfqpoint{3.770806in}{1.321256in}}%
\pgfpathlineto{\pgfqpoint{3.777799in}{1.337178in}}%
\pgfpathlineto{\pgfqpoint{3.791787in}{1.337178in}}%
\pgfpathlineto{\pgfqpoint{3.798781in}{1.321256in}}%
\pgfpathlineto{\pgfqpoint{3.805775in}{1.321256in}}%
\pgfpathlineto{\pgfqpoint{3.812768in}{1.337178in}}%
\pgfpathlineto{\pgfqpoint{3.819762in}{1.337178in}}%
\pgfpathlineto{\pgfqpoint{3.826756in}{1.321256in}}%
\pgfpathlineto{\pgfqpoint{3.833750in}{1.337178in}}%
\pgfpathlineto{\pgfqpoint{3.861725in}{1.337178in}}%
\pgfpathlineto{\pgfqpoint{3.868719in}{1.321256in}}%
\pgfpathlineto{\pgfqpoint{3.875712in}{1.337178in}}%
\pgfpathlineto{\pgfqpoint{3.882706in}{1.321256in}}%
\pgfpathlineto{\pgfqpoint{3.889700in}{0.572934in}}%
\pgfpathlineto{\pgfqpoint{3.896694in}{0.572934in}}%
\pgfpathlineto{\pgfqpoint{3.903688in}{0.557012in}}%
\pgfpathlineto{\pgfqpoint{3.910681in}{0.557012in}}%
\pgfpathlineto{\pgfqpoint{3.917675in}{0.572934in}}%
\pgfpathlineto{\pgfqpoint{3.924669in}{0.557012in}}%
\pgfpathlineto{\pgfqpoint{3.952644in}{0.557012in}}%
\pgfpathlineto{\pgfqpoint{3.959638in}{0.541090in}}%
\pgfpathlineto{\pgfqpoint{4.008595in}{0.541090in}}%
\pgfpathlineto{\pgfqpoint{4.015588in}{0.557012in}}%
\pgfpathlineto{\pgfqpoint{4.022582in}{0.541090in}}%
\pgfpathlineto{\pgfqpoint{4.071539in}{0.541090in}}%
\pgfpathlineto{\pgfqpoint{4.071539in}{0.541090in}}%
\pgfusepath{stroke}%
\end{pgfscope}%
\begin{pgfscope}%
\pgfpathrectangle{\pgfqpoint{2.610186in}{0.484568in}}{\pgfqpoint{1.530941in}{0.893210in}}%
\pgfusepath{clip}%
\pgfsetrectcap%
\pgfsetroundjoin%
\pgfsetlinewidth{1.505625pt}%
\definecolor{currentstroke}{rgb}{1.000000,0.498039,0.054902}%
\pgfsetstrokecolor{currentstroke}%
\pgfsetdash{}{0pt}%
\pgfpathmoveto{\pgfqpoint{0.000000in}{0.000000in}}%
\pgfusepath{stroke}%
\end{pgfscope}%
\begin{pgfscope}%
\pgfpathrectangle{\pgfqpoint{2.610186in}{0.484568in}}{\pgfqpoint{1.530941in}{0.893210in}}%
\pgfusepath{clip}%
\pgfsetrectcap%
\pgfsetroundjoin%
\pgfsetlinewidth{1.505625pt}%
\definecolor{currentstroke}{rgb}{0.172549,0.627451,0.172549}%
\pgfsetstrokecolor{currentstroke}%
\pgfsetdash{}{0pt}%
\pgfpathmoveto{\pgfqpoint{2.679774in}{0.636621in}}%
\pgfpathlineto{\pgfqpoint{2.686768in}{1.130195in}}%
\pgfpathlineto{\pgfqpoint{2.693761in}{1.066508in}}%
\pgfpathlineto{\pgfqpoint{2.721737in}{0.939134in}}%
\pgfpathlineto{\pgfqpoint{2.728730in}{0.923212in}}%
\pgfpathlineto{\pgfqpoint{2.735724in}{0.875447in}}%
\pgfpathlineto{\pgfqpoint{2.749712in}{0.843603in}}%
\pgfpathlineto{\pgfqpoint{2.756706in}{0.811760in}}%
\pgfpathlineto{\pgfqpoint{2.770693in}{0.779916in}}%
\pgfpathlineto{\pgfqpoint{2.777687in}{0.748073in}}%
\pgfpathlineto{\pgfqpoint{2.784681in}{0.748073in}}%
\pgfpathlineto{\pgfqpoint{2.791675in}{0.732151in}}%
\pgfpathlineto{\pgfqpoint{2.798668in}{0.700308in}}%
\pgfpathlineto{\pgfqpoint{2.805662in}{0.684386in}}%
\pgfpathlineto{\pgfqpoint{2.812656in}{0.684386in}}%
\pgfpathlineto{\pgfqpoint{2.819650in}{0.668464in}}%
\pgfpathlineto{\pgfqpoint{2.826643in}{0.668464in}}%
\pgfpathlineto{\pgfqpoint{2.833637in}{0.652542in}}%
\pgfpathlineto{\pgfqpoint{2.840631in}{0.652542in}}%
\pgfpathlineto{\pgfqpoint{2.847625in}{0.620699in}}%
\pgfpathlineto{\pgfqpoint{2.854619in}{0.636621in}}%
\pgfpathlineto{\pgfqpoint{2.861612in}{0.620699in}}%
\pgfpathlineto{\pgfqpoint{2.875600in}{0.620699in}}%
\pgfpathlineto{\pgfqpoint{2.882594in}{0.604777in}}%
\pgfpathlineto{\pgfqpoint{2.889588in}{0.604777in}}%
\pgfpathlineto{\pgfqpoint{2.896581in}{0.588855in}}%
\pgfpathlineto{\pgfqpoint{2.938544in}{0.588855in}}%
\pgfpathlineto{\pgfqpoint{2.945538in}{0.557012in}}%
\pgfpathlineto{\pgfqpoint{2.966519in}{0.557012in}}%
\pgfpathlineto{\pgfqpoint{2.980507in}{0.652542in}}%
\pgfpathlineto{\pgfqpoint{2.987501in}{0.684386in}}%
\pgfpathlineto{\pgfqpoint{2.994495in}{0.748073in}}%
\pgfpathlineto{\pgfqpoint{3.001488in}{0.795838in}}%
\pgfpathlineto{\pgfqpoint{3.008482in}{0.811760in}}%
\pgfpathlineto{\pgfqpoint{3.015476in}{0.875447in}}%
\pgfpathlineto{\pgfqpoint{3.022470in}{0.891369in}}%
\pgfpathlineto{\pgfqpoint{3.029463in}{0.939134in}}%
\pgfpathlineto{\pgfqpoint{3.036457in}{0.970977in}}%
\pgfpathlineto{\pgfqpoint{3.050445in}{1.002821in}}%
\pgfpathlineto{\pgfqpoint{3.064432in}{1.066508in}}%
\pgfpathlineto{\pgfqpoint{3.071426in}{1.066508in}}%
\pgfpathlineto{\pgfqpoint{3.078420in}{1.098351in}}%
\pgfpathlineto{\pgfqpoint{3.099401in}{1.146117in}}%
\pgfpathlineto{\pgfqpoint{3.106395in}{1.146117in}}%
\pgfpathlineto{\pgfqpoint{3.113389in}{1.177960in}}%
\pgfpathlineto{\pgfqpoint{3.120383in}{1.177960in}}%
\pgfpathlineto{\pgfqpoint{3.134370in}{1.209804in}}%
\pgfpathlineto{\pgfqpoint{3.148358in}{1.209804in}}%
\pgfpathlineto{\pgfqpoint{3.162346in}{1.241647in}}%
\pgfpathlineto{\pgfqpoint{3.176333in}{1.241647in}}%
\pgfpathlineto{\pgfqpoint{3.183327in}{1.257569in}}%
\pgfpathlineto{\pgfqpoint{3.204308in}{1.257569in}}%
\pgfpathlineto{\pgfqpoint{3.211302in}{1.273491in}}%
\pgfpathlineto{\pgfqpoint{3.225290in}{1.273491in}}%
\pgfpathlineto{\pgfqpoint{3.232283in}{1.289412in}}%
\pgfpathlineto{\pgfqpoint{3.246271in}{1.289412in}}%
\pgfpathlineto{\pgfqpoint{3.253265in}{1.305334in}}%
\pgfpathlineto{\pgfqpoint{3.260259in}{1.305334in}}%
\pgfpathlineto{\pgfqpoint{3.267252in}{1.321256in}}%
\pgfpathlineto{\pgfqpoint{3.274246in}{1.305334in}}%
\pgfpathlineto{\pgfqpoint{3.281240in}{1.257569in}}%
\pgfpathlineto{\pgfqpoint{3.288234in}{1.193882in}}%
\pgfpathlineto{\pgfqpoint{3.316209in}{1.002821in}}%
\pgfpathlineto{\pgfqpoint{3.323203in}{0.986899in}}%
\pgfpathlineto{\pgfqpoint{3.330197in}{0.955056in}}%
\pgfpathlineto{\pgfqpoint{3.337190in}{0.939134in}}%
\pgfpathlineto{\pgfqpoint{3.344184in}{0.891369in}}%
\pgfpathlineto{\pgfqpoint{3.351178in}{0.875447in}}%
\pgfpathlineto{\pgfqpoint{3.365166in}{0.811760in}}%
\pgfpathlineto{\pgfqpoint{3.372159in}{0.811760in}}%
\pgfpathlineto{\pgfqpoint{3.386147in}{0.748073in}}%
\pgfpathlineto{\pgfqpoint{3.393141in}{0.748073in}}%
\pgfpathlineto{\pgfqpoint{3.407128in}{0.716229in}}%
\pgfpathlineto{\pgfqpoint{3.414122in}{0.684386in}}%
\pgfpathlineto{\pgfqpoint{3.428110in}{0.684386in}}%
\pgfpathlineto{\pgfqpoint{3.435103in}{0.668464in}}%
\pgfpathlineto{\pgfqpoint{3.442097in}{0.668464in}}%
\pgfpathlineto{\pgfqpoint{3.456085in}{0.636621in}}%
\pgfpathlineto{\pgfqpoint{3.463079in}{0.636621in}}%
\pgfpathlineto{\pgfqpoint{3.470072in}{0.620699in}}%
\pgfpathlineto{\pgfqpoint{3.484060in}{0.620699in}}%
\pgfpathlineto{\pgfqpoint{3.491054in}{0.604777in}}%
\pgfpathlineto{\pgfqpoint{3.505041in}{0.604777in}}%
\pgfpathlineto{\pgfqpoint{3.512035in}{0.588855in}}%
\pgfpathlineto{\pgfqpoint{3.547004in}{0.588855in}}%
\pgfpathlineto{\pgfqpoint{3.553998in}{0.557012in}}%
\pgfpathlineto{\pgfqpoint{3.574979in}{0.557012in}}%
\pgfpathlineto{\pgfqpoint{3.581973in}{0.572934in}}%
\pgfpathlineto{\pgfqpoint{3.616942in}{0.811760in}}%
\pgfpathlineto{\pgfqpoint{3.658905in}{1.002821in}}%
\pgfpathlineto{\pgfqpoint{3.665899in}{1.002821in}}%
\pgfpathlineto{\pgfqpoint{3.679886in}{1.066508in}}%
\pgfpathlineto{\pgfqpoint{3.693874in}{1.098351in}}%
\pgfpathlineto{\pgfqpoint{3.700868in}{1.130195in}}%
\pgfpathlineto{\pgfqpoint{3.707861in}{1.130195in}}%
\pgfpathlineto{\pgfqpoint{3.721849in}{1.162038in}}%
\pgfpathlineto{\pgfqpoint{3.728843in}{1.162038in}}%
\pgfpathlineto{\pgfqpoint{3.735837in}{1.193882in}}%
\pgfpathlineto{\pgfqpoint{3.749824in}{1.193882in}}%
\pgfpathlineto{\pgfqpoint{3.763812in}{1.225725in}}%
\pgfpathlineto{\pgfqpoint{3.770806in}{1.225725in}}%
\pgfpathlineto{\pgfqpoint{3.777799in}{1.241647in}}%
\pgfpathlineto{\pgfqpoint{3.791787in}{1.241647in}}%
\pgfpathlineto{\pgfqpoint{3.798781in}{1.257569in}}%
\pgfpathlineto{\pgfqpoint{3.819762in}{1.257569in}}%
\pgfpathlineto{\pgfqpoint{3.833750in}{1.289412in}}%
\pgfpathlineto{\pgfqpoint{3.868719in}{1.289412in}}%
\pgfpathlineto{\pgfqpoint{3.875712in}{1.321256in}}%
\pgfpathlineto{\pgfqpoint{3.882706in}{1.305334in}}%
\pgfpathlineto{\pgfqpoint{3.889700in}{1.257569in}}%
\pgfpathlineto{\pgfqpoint{3.896694in}{1.225725in}}%
\pgfpathlineto{\pgfqpoint{3.917675in}{1.082430in}}%
\pgfpathlineto{\pgfqpoint{3.924669in}{1.050586in}}%
\pgfpathlineto{\pgfqpoint{3.931663in}{1.002821in}}%
\pgfpathlineto{\pgfqpoint{3.959638in}{0.875447in}}%
\pgfpathlineto{\pgfqpoint{3.966632in}{0.859525in}}%
\pgfpathlineto{\pgfqpoint{3.973626in}{0.827682in}}%
\pgfpathlineto{\pgfqpoint{3.980619in}{0.811760in}}%
\pgfpathlineto{\pgfqpoint{3.987613in}{0.779916in}}%
\pgfpathlineto{\pgfqpoint{3.994607in}{0.779916in}}%
\pgfpathlineto{\pgfqpoint{4.001601in}{0.748073in}}%
\pgfpathlineto{\pgfqpoint{4.008595in}{0.748073in}}%
\pgfpathlineto{\pgfqpoint{4.015588in}{0.716229in}}%
\pgfpathlineto{\pgfqpoint{4.022582in}{0.716229in}}%
\pgfpathlineto{\pgfqpoint{4.029576in}{0.684386in}}%
\pgfpathlineto{\pgfqpoint{4.043563in}{0.684386in}}%
\pgfpathlineto{\pgfqpoint{4.071539in}{0.620699in}}%
\pgfpathlineto{\pgfqpoint{4.071539in}{0.620699in}}%
\pgfusepath{stroke}%
\end{pgfscope}%
\begin{pgfscope}%
\pgfpathrectangle{\pgfqpoint{2.610186in}{0.484568in}}{\pgfqpoint{1.530941in}{0.893210in}}%
\pgfusepath{clip}%
\pgfsetrectcap%
\pgfsetroundjoin%
\pgfsetlinewidth{1.505625pt}%
\definecolor{currentstroke}{rgb}{0.839216,0.152941,0.156863}%
\pgfsetstrokecolor{currentstroke}%
\pgfsetdash{}{0pt}%
\pgfpathmoveto{\pgfqpoint{0.000000in}{0.000000in}}%
\pgfusepath{stroke}%
\end{pgfscope}%
\begin{pgfscope}%
\pgfsetrectcap%
\pgfsetmiterjoin%
\pgfsetlinewidth{0.803000pt}%
\definecolor{currentstroke}{rgb}{0.000000,0.000000,0.000000}%
\pgfsetstrokecolor{currentstroke}%
\pgfsetdash{}{0pt}%
\pgfpathmoveto{\pgfqpoint{2.610186in}{0.484568in}}%
\pgfpathlineto{\pgfqpoint{2.610186in}{1.377778in}}%
\pgfusepath{stroke}%
\end{pgfscope}%
\begin{pgfscope}%
\pgfsetrectcap%
\pgfsetmiterjoin%
\pgfsetlinewidth{0.803000pt}%
\definecolor{currentstroke}{rgb}{0.000000,0.000000,0.000000}%
\pgfsetstrokecolor{currentstroke}%
\pgfsetdash{}{0pt}%
\pgfpathmoveto{\pgfqpoint{4.141127in}{0.484568in}}%
\pgfpathlineto{\pgfqpoint{4.141127in}{1.377778in}}%
\pgfusepath{stroke}%
\end{pgfscope}%
\begin{pgfscope}%
\pgfsetrectcap%
\pgfsetmiterjoin%
\pgfsetlinewidth{0.803000pt}%
\definecolor{currentstroke}{rgb}{0.000000,0.000000,0.000000}%
\pgfsetstrokecolor{currentstroke}%
\pgfsetdash{}{0pt}%
\pgfpathmoveto{\pgfqpoint{2.610186in}{0.484568in}}%
\pgfpathlineto{\pgfqpoint{4.141127in}{0.484568in}}%
\pgfusepath{stroke}%
\end{pgfscope}%
\begin{pgfscope}%
\pgfsetrectcap%
\pgfsetmiterjoin%
\pgfsetlinewidth{0.803000pt}%
\definecolor{currentstroke}{rgb}{0.000000,0.000000,0.000000}%
\pgfsetstrokecolor{currentstroke}%
\pgfsetdash{}{0pt}%
\pgfpathmoveto{\pgfqpoint{2.610186in}{1.377778in}}%
\pgfpathlineto{\pgfqpoint{4.141127in}{1.377778in}}%
\pgfusepath{stroke}%
\end{pgfscope}%
\begin{pgfscope}%
\definecolor{textcolor}{rgb}{0.000000,0.000000,0.000000}%
\pgfsetstrokecolor{textcolor}%
\pgfsetfillcolor{textcolor}%
\pgftext[x=3.375656in,y=1.461111in,,base]{\color{textcolor}\rmfamily\fontsize{12.000000}{14.400000}\selectfont 7}%
\end{pgfscope}%
\begin{pgfscope}%
\pgfsetbuttcap%
\pgfsetmiterjoin%
\definecolor{currentfill}{rgb}{1.000000,1.000000,1.000000}%
\pgfsetfillcolor{currentfill}%
\pgfsetlinewidth{0.000000pt}%
\definecolor{currentstroke}{rgb}{0.000000,0.000000,0.000000}%
\pgfsetstrokecolor{currentstroke}%
\pgfsetstrokeopacity{0.000000}%
\pgfsetdash{}{0pt}%
\pgfpathmoveto{\pgfqpoint{4.720371in}{0.484568in}}%
\pgfpathlineto{\pgfqpoint{6.251312in}{0.484568in}}%
\pgfpathlineto{\pgfqpoint{6.251312in}{1.377778in}}%
\pgfpathlineto{\pgfqpoint{4.720371in}{1.377778in}}%
\pgfpathclose%
\pgfusepath{fill}%
\end{pgfscope}%
\begin{pgfscope}%
\pgfsetbuttcap%
\pgfsetroundjoin%
\definecolor{currentfill}{rgb}{0.000000,0.000000,0.000000}%
\pgfsetfillcolor{currentfill}%
\pgfsetlinewidth{0.803000pt}%
\definecolor{currentstroke}{rgb}{0.000000,0.000000,0.000000}%
\pgfsetstrokecolor{currentstroke}%
\pgfsetdash{}{0pt}%
\pgfsys@defobject{currentmarker}{\pgfqpoint{0.000000in}{-0.048611in}}{\pgfqpoint{0.000000in}{0.000000in}}{%
\pgfpathmoveto{\pgfqpoint{0.000000in}{0.000000in}}%
\pgfpathlineto{\pgfqpoint{0.000000in}{-0.048611in}}%
\pgfusepath{stroke,fill}%
}%
\begin{pgfscope}%
\pgfsys@transformshift{4.789959in}{0.484568in}%
\pgfsys@useobject{currentmarker}{}%
\end{pgfscope}%
\end{pgfscope}%
\begin{pgfscope}%
\definecolor{textcolor}{rgb}{0.000000,0.000000,0.000000}%
\pgfsetstrokecolor{textcolor}%
\pgfsetfillcolor{textcolor}%
\pgftext[x=4.789959in,y=0.387346in,,top]{\color{textcolor}\rmfamily\fontsize{10.000000}{12.000000}\selectfont \(\displaystyle {0}\)}%
\end{pgfscope}%
\begin{pgfscope}%
\pgfsetbuttcap%
\pgfsetroundjoin%
\definecolor{currentfill}{rgb}{0.000000,0.000000,0.000000}%
\pgfsetfillcolor{currentfill}%
\pgfsetlinewidth{0.803000pt}%
\definecolor{currentstroke}{rgb}{0.000000,0.000000,0.000000}%
\pgfsetstrokecolor{currentstroke}%
\pgfsetdash{}{0pt}%
\pgfsys@defobject{currentmarker}{\pgfqpoint{0.000000in}{-0.048611in}}{\pgfqpoint{0.000000in}{0.000000in}}{%
\pgfpathmoveto{\pgfqpoint{0.000000in}{0.000000in}}%
\pgfpathlineto{\pgfqpoint{0.000000in}{-0.048611in}}%
\pgfusepath{stroke,fill}%
}%
\begin{pgfscope}%
\pgfsys@transformshift{5.489338in}{0.484568in}%
\pgfsys@useobject{currentmarker}{}%
\end{pgfscope}%
\end{pgfscope}%
\begin{pgfscope}%
\definecolor{textcolor}{rgb}{0.000000,0.000000,0.000000}%
\pgfsetstrokecolor{textcolor}%
\pgfsetfillcolor{textcolor}%
\pgftext[x=5.489338in,y=0.387346in,,top]{\color{textcolor}\rmfamily\fontsize{10.000000}{12.000000}\selectfont \(\displaystyle {2000}\)}%
\end{pgfscope}%
\begin{pgfscope}%
\pgfsetbuttcap%
\pgfsetroundjoin%
\definecolor{currentfill}{rgb}{0.000000,0.000000,0.000000}%
\pgfsetfillcolor{currentfill}%
\pgfsetlinewidth{0.803000pt}%
\definecolor{currentstroke}{rgb}{0.000000,0.000000,0.000000}%
\pgfsetstrokecolor{currentstroke}%
\pgfsetdash{}{0pt}%
\pgfsys@defobject{currentmarker}{\pgfqpoint{0.000000in}{-0.048611in}}{\pgfqpoint{0.000000in}{0.000000in}}{%
\pgfpathmoveto{\pgfqpoint{0.000000in}{0.000000in}}%
\pgfpathlineto{\pgfqpoint{0.000000in}{-0.048611in}}%
\pgfusepath{stroke,fill}%
}%
\begin{pgfscope}%
\pgfsys@transformshift{6.188718in}{0.484568in}%
\pgfsys@useobject{currentmarker}{}%
\end{pgfscope}%
\end{pgfscope}%
\begin{pgfscope}%
\definecolor{textcolor}{rgb}{0.000000,0.000000,0.000000}%
\pgfsetstrokecolor{textcolor}%
\pgfsetfillcolor{textcolor}%
\pgftext[x=6.188718in,y=0.387346in,,top]{\color{textcolor}\rmfamily\fontsize{10.000000}{12.000000}\selectfont \(\displaystyle {4000}\)}%
\end{pgfscope}%
\begin{pgfscope}%
\definecolor{textcolor}{rgb}{0.000000,0.000000,0.000000}%
\pgfsetstrokecolor{textcolor}%
\pgfsetfillcolor{textcolor}%
\pgftext[x=5.485841in,y=0.208333in,,top]{\color{textcolor}\rmfamily\fontsize{10.000000}{12.000000}\selectfont Czas [\(\displaystyle \mu s\)] }%
\end{pgfscope}%
\begin{pgfscope}%
\pgfsetbuttcap%
\pgfsetroundjoin%
\definecolor{currentfill}{rgb}{0.000000,0.000000,0.000000}%
\pgfsetfillcolor{currentfill}%
\pgfsetlinewidth{0.803000pt}%
\definecolor{currentstroke}{rgb}{0.000000,0.000000,0.000000}%
\pgfsetstrokecolor{currentstroke}%
\pgfsetdash{}{0pt}%
\pgfsys@defobject{currentmarker}{\pgfqpoint{-0.048611in}{0.000000in}}{\pgfqpoint{-0.000000in}{0.000000in}}{%
\pgfpathmoveto{\pgfqpoint{-0.000000in}{0.000000in}}%
\pgfpathlineto{\pgfqpoint{-0.048611in}{0.000000in}}%
\pgfusepath{stroke,fill}%
}%
\begin{pgfscope}%
\pgfsys@transformshift{4.720371in}{0.541090in}%
\pgfsys@useobject{currentmarker}{}%
\end{pgfscope}%
\end{pgfscope}%
\begin{pgfscope}%
\definecolor{textcolor}{rgb}{0.000000,0.000000,0.000000}%
\pgfsetstrokecolor{textcolor}%
\pgfsetfillcolor{textcolor}%
\pgftext[x=4.553704in, y=0.492865in, left, base]{\color{textcolor}\rmfamily\fontsize{10.000000}{12.000000}\selectfont \(\displaystyle {0}\)}%
\end{pgfscope}%
\begin{pgfscope}%
\pgfsetbuttcap%
\pgfsetroundjoin%
\definecolor{currentfill}{rgb}{0.000000,0.000000,0.000000}%
\pgfsetfillcolor{currentfill}%
\pgfsetlinewidth{0.803000pt}%
\definecolor{currentstroke}{rgb}{0.000000,0.000000,0.000000}%
\pgfsetstrokecolor{currentstroke}%
\pgfsetdash{}{0pt}%
\pgfsys@defobject{currentmarker}{\pgfqpoint{-0.048611in}{0.000000in}}{\pgfqpoint{-0.000000in}{0.000000in}}{%
\pgfpathmoveto{\pgfqpoint{-0.000000in}{0.000000in}}%
\pgfpathlineto{\pgfqpoint{-0.048611in}{0.000000in}}%
\pgfusepath{stroke,fill}%
}%
\begin{pgfscope}%
\pgfsys@transformshift{4.720371in}{0.859525in}%
\pgfsys@useobject{currentmarker}{}%
\end{pgfscope}%
\end{pgfscope}%
\begin{pgfscope}%
\definecolor{textcolor}{rgb}{0.000000,0.000000,0.000000}%
\pgfsetstrokecolor{textcolor}%
\pgfsetfillcolor{textcolor}%
\pgftext[x=4.484259in, y=0.811300in, left, base]{\color{textcolor}\rmfamily\fontsize{10.000000}{12.000000}\selectfont \(\displaystyle {20}\)}%
\end{pgfscope}%
\begin{pgfscope}%
\pgfsetbuttcap%
\pgfsetroundjoin%
\definecolor{currentfill}{rgb}{0.000000,0.000000,0.000000}%
\pgfsetfillcolor{currentfill}%
\pgfsetlinewidth{0.803000pt}%
\definecolor{currentstroke}{rgb}{0.000000,0.000000,0.000000}%
\pgfsetstrokecolor{currentstroke}%
\pgfsetdash{}{0pt}%
\pgfsys@defobject{currentmarker}{\pgfqpoint{-0.048611in}{0.000000in}}{\pgfqpoint{-0.000000in}{0.000000in}}{%
\pgfpathmoveto{\pgfqpoint{-0.000000in}{0.000000in}}%
\pgfpathlineto{\pgfqpoint{-0.048611in}{0.000000in}}%
\pgfusepath{stroke,fill}%
}%
\begin{pgfscope}%
\pgfsys@transformshift{4.720371in}{1.177960in}%
\pgfsys@useobject{currentmarker}{}%
\end{pgfscope}%
\end{pgfscope}%
\begin{pgfscope}%
\definecolor{textcolor}{rgb}{0.000000,0.000000,0.000000}%
\pgfsetstrokecolor{textcolor}%
\pgfsetfillcolor{textcolor}%
\pgftext[x=4.484259in, y=1.129735in, left, base]{\color{textcolor}\rmfamily\fontsize{10.000000}{12.000000}\selectfont \(\displaystyle {40}\)}%
\end{pgfscope}%
\begin{pgfscope}%
\definecolor{textcolor}{rgb}{0.000000,0.000000,0.000000}%
\pgfsetstrokecolor{textcolor}%
\pgfsetfillcolor{textcolor}%
\pgftext[x=4.428704in,y=0.931173in,,bottom,rotate=90.000000]{\color{textcolor}\rmfamily\fontsize{10.000000}{12.000000}\selectfont Napiecie [\(\displaystyle V\)] }%
\end{pgfscope}%
\begin{pgfscope}%
\pgfpathrectangle{\pgfqpoint{4.720371in}{0.484568in}}{\pgfqpoint{1.530941in}{0.893210in}}%
\pgfusepath{clip}%
\pgfsetrectcap%
\pgfsetroundjoin%
\pgfsetlinewidth{1.505625pt}%
\definecolor{currentstroke}{rgb}{0.121569,0.466667,0.705882}%
\pgfsetstrokecolor{currentstroke}%
\pgfsetdash{}{0pt}%
\pgfpathmoveto{\pgfqpoint{4.789959in}{0.541090in}}%
\pgfpathlineto{\pgfqpoint{4.859897in}{0.541090in}}%
\pgfpathlineto{\pgfqpoint{4.866891in}{0.525168in}}%
\pgfpathlineto{\pgfqpoint{4.873885in}{0.541090in}}%
\pgfpathlineto{\pgfqpoint{4.894866in}{0.541090in}}%
\pgfpathlineto{\pgfqpoint{4.901860in}{0.525168in}}%
\pgfpathlineto{\pgfqpoint{4.908854in}{0.541090in}}%
\pgfpathlineto{\pgfqpoint{4.985785in}{0.541090in}}%
\pgfpathlineto{\pgfqpoint{4.992779in}{1.273491in}}%
\pgfpathlineto{\pgfqpoint{5.006767in}{1.305334in}}%
\pgfpathlineto{\pgfqpoint{5.048729in}{1.305334in}}%
\pgfpathlineto{\pgfqpoint{5.055723in}{1.321256in}}%
\pgfpathlineto{\pgfqpoint{5.062717in}{1.305334in}}%
\pgfpathlineto{\pgfqpoint{5.069711in}{1.321256in}}%
\pgfpathlineto{\pgfqpoint{5.104680in}{1.321256in}}%
\pgfpathlineto{\pgfqpoint{5.111674in}{1.305334in}}%
\pgfpathlineto{\pgfqpoint{5.125661in}{1.337178in}}%
\pgfpathlineto{\pgfqpoint{5.132655in}{1.321256in}}%
\pgfpathlineto{\pgfqpoint{5.139649in}{1.337178in}}%
\pgfpathlineto{\pgfqpoint{5.146642in}{1.337178in}}%
\pgfpathlineto{\pgfqpoint{5.153636in}{1.321256in}}%
\pgfpathlineto{\pgfqpoint{5.174618in}{1.321256in}}%
\pgfpathlineto{\pgfqpoint{5.181611in}{1.337178in}}%
\pgfpathlineto{\pgfqpoint{5.230568in}{1.337178in}}%
\pgfpathlineto{\pgfqpoint{5.237562in}{1.321256in}}%
\pgfpathlineto{\pgfqpoint{5.244556in}{1.321256in}}%
\pgfpathlineto{\pgfqpoint{5.251549in}{1.337178in}}%
\pgfpathlineto{\pgfqpoint{5.293512in}{1.337178in}}%
\pgfpathlineto{\pgfqpoint{5.300506in}{1.321256in}}%
\pgfpathlineto{\pgfqpoint{5.307500in}{1.337178in}}%
\pgfpathlineto{\pgfqpoint{5.335475in}{1.337178in}}%
\pgfpathlineto{\pgfqpoint{5.342469in}{0.572934in}}%
\pgfpathlineto{\pgfqpoint{5.363450in}{0.572934in}}%
\pgfpathlineto{\pgfqpoint{5.370444in}{0.557012in}}%
\pgfpathlineto{\pgfqpoint{5.377438in}{0.557012in}}%
\pgfpathlineto{\pgfqpoint{5.384431in}{0.572934in}}%
\pgfpathlineto{\pgfqpoint{5.391425in}{0.572934in}}%
\pgfpathlineto{\pgfqpoint{5.405413in}{0.541090in}}%
\pgfpathlineto{\pgfqpoint{5.412407in}{0.557012in}}%
\pgfpathlineto{\pgfqpoint{5.426394in}{0.557012in}}%
\pgfpathlineto{\pgfqpoint{5.433388in}{0.541090in}}%
\pgfpathlineto{\pgfqpoint{5.461363in}{0.541090in}}%
\pgfpathlineto{\pgfqpoint{5.468357in}{0.557012in}}%
\pgfpathlineto{\pgfqpoint{5.475351in}{0.541090in}}%
\pgfpathlineto{\pgfqpoint{5.510320in}{0.541090in}}%
\pgfpathlineto{\pgfqpoint{5.517313in}{0.557012in}}%
\pgfpathlineto{\pgfqpoint{5.524307in}{0.525168in}}%
\pgfpathlineto{\pgfqpoint{5.531301in}{0.525168in}}%
\pgfpathlineto{\pgfqpoint{5.538295in}{0.541090in}}%
\pgfpathlineto{\pgfqpoint{5.552282in}{0.541090in}}%
\pgfpathlineto{\pgfqpoint{5.559276in}{0.525168in}}%
\pgfpathlineto{\pgfqpoint{5.566270in}{0.541090in}}%
\pgfpathlineto{\pgfqpoint{5.573264in}{0.541090in}}%
\pgfpathlineto{\pgfqpoint{5.580258in}{0.525168in}}%
\pgfpathlineto{\pgfqpoint{5.587251in}{0.541090in}}%
\pgfpathlineto{\pgfqpoint{5.636208in}{0.541090in}}%
\pgfpathlineto{\pgfqpoint{5.643202in}{0.525168in}}%
\pgfpathlineto{\pgfqpoint{5.650196in}{0.541090in}}%
\pgfpathlineto{\pgfqpoint{5.671177in}{0.541090in}}%
\pgfpathlineto{\pgfqpoint{5.678171in}{0.525168in}}%
\pgfpathlineto{\pgfqpoint{5.685165in}{0.525168in}}%
\pgfpathlineto{\pgfqpoint{5.692158in}{1.273491in}}%
\pgfpathlineto{\pgfqpoint{5.706146in}{1.305334in}}%
\pgfpathlineto{\pgfqpoint{5.713140in}{1.305334in}}%
\pgfpathlineto{\pgfqpoint{5.720133in}{1.289412in}}%
\pgfpathlineto{\pgfqpoint{5.727127in}{1.305334in}}%
\pgfpathlineto{\pgfqpoint{5.755102in}{1.305334in}}%
\pgfpathlineto{\pgfqpoint{5.762096in}{1.321256in}}%
\pgfpathlineto{\pgfqpoint{5.769090in}{1.305334in}}%
\pgfpathlineto{\pgfqpoint{5.776084in}{1.337178in}}%
\pgfpathlineto{\pgfqpoint{5.783078in}{1.321256in}}%
\pgfpathlineto{\pgfqpoint{5.790071in}{1.321256in}}%
\pgfpathlineto{\pgfqpoint{5.797065in}{1.305334in}}%
\pgfpathlineto{\pgfqpoint{5.804059in}{1.321256in}}%
\pgfpathlineto{\pgfqpoint{5.853016in}{1.321256in}}%
\pgfpathlineto{\pgfqpoint{5.860009in}{1.337178in}}%
\pgfpathlineto{\pgfqpoint{5.867003in}{1.337178in}}%
\pgfpathlineto{\pgfqpoint{5.873997in}{1.321256in}}%
\pgfpathlineto{\pgfqpoint{5.880991in}{1.321256in}}%
\pgfpathlineto{\pgfqpoint{5.887985in}{1.337178in}}%
\pgfpathlineto{\pgfqpoint{5.894978in}{1.321256in}}%
\pgfpathlineto{\pgfqpoint{5.901972in}{1.337178in}}%
\pgfpathlineto{\pgfqpoint{5.929947in}{1.337178in}}%
\pgfpathlineto{\pgfqpoint{5.936941in}{1.321256in}}%
\pgfpathlineto{\pgfqpoint{5.943935in}{1.337178in}}%
\pgfpathlineto{\pgfqpoint{5.950929in}{1.337178in}}%
\pgfpathlineto{\pgfqpoint{5.957922in}{1.321256in}}%
\pgfpathlineto{\pgfqpoint{5.964916in}{1.337178in}}%
\pgfpathlineto{\pgfqpoint{6.034854in}{1.337178in}}%
\pgfpathlineto{\pgfqpoint{6.041848in}{0.572934in}}%
\pgfpathlineto{\pgfqpoint{6.062829in}{0.572934in}}%
\pgfpathlineto{\pgfqpoint{6.069823in}{0.557012in}}%
\pgfpathlineto{\pgfqpoint{6.097798in}{0.557012in}}%
\pgfpathlineto{\pgfqpoint{6.104792in}{0.541090in}}%
\pgfpathlineto{\pgfqpoint{6.146755in}{0.541090in}}%
\pgfpathlineto{\pgfqpoint{6.153749in}{0.525168in}}%
\pgfpathlineto{\pgfqpoint{6.160742in}{0.557012in}}%
\pgfpathlineto{\pgfqpoint{6.167736in}{0.541090in}}%
\pgfpathlineto{\pgfqpoint{6.181724in}{0.541090in}}%
\pgfpathlineto{\pgfqpoint{6.181724in}{0.541090in}}%
\pgfusepath{stroke}%
\end{pgfscope}%
\begin{pgfscope}%
\pgfpathrectangle{\pgfqpoint{4.720371in}{0.484568in}}{\pgfqpoint{1.530941in}{0.893210in}}%
\pgfusepath{clip}%
\pgfsetrectcap%
\pgfsetroundjoin%
\pgfsetlinewidth{1.505625pt}%
\definecolor{currentstroke}{rgb}{1.000000,0.498039,0.054902}%
\pgfsetstrokecolor{currentstroke}%
\pgfsetdash{}{0pt}%
\pgfpathmoveto{\pgfqpoint{0.000000in}{0.000000in}}%
\pgfusepath{stroke}%
\end{pgfscope}%
\begin{pgfscope}%
\pgfpathrectangle{\pgfqpoint{4.720371in}{0.484568in}}{\pgfqpoint{1.530941in}{0.893210in}}%
\pgfusepath{clip}%
\pgfsetrectcap%
\pgfsetroundjoin%
\pgfsetlinewidth{1.505625pt}%
\definecolor{currentstroke}{rgb}{0.172549,0.627451,0.172549}%
\pgfsetstrokecolor{currentstroke}%
\pgfsetdash{}{0pt}%
\pgfpathmoveto{\pgfqpoint{4.789959in}{0.684386in}}%
\pgfpathlineto{\pgfqpoint{4.796953in}{0.684386in}}%
\pgfpathlineto{\pgfqpoint{4.810940in}{0.652542in}}%
\pgfpathlineto{\pgfqpoint{4.817934in}{0.652542in}}%
\pgfpathlineto{\pgfqpoint{4.831922in}{0.620699in}}%
\pgfpathlineto{\pgfqpoint{4.852903in}{0.620699in}}%
\pgfpathlineto{\pgfqpoint{4.859897in}{0.604777in}}%
\pgfpathlineto{\pgfqpoint{4.880878in}{0.604777in}}%
\pgfpathlineto{\pgfqpoint{4.887872in}{0.588855in}}%
\pgfpathlineto{\pgfqpoint{4.901860in}{0.588855in}}%
\pgfpathlineto{\pgfqpoint{4.908854in}{0.572934in}}%
\pgfpathlineto{\pgfqpoint{4.929835in}{0.572934in}}%
\pgfpathlineto{\pgfqpoint{4.936829in}{0.557012in}}%
\pgfpathlineto{\pgfqpoint{4.992779in}{0.557012in}}%
\pgfpathlineto{\pgfqpoint{4.999773in}{0.620699in}}%
\pgfpathlineto{\pgfqpoint{5.027748in}{0.811760in}}%
\pgfpathlineto{\pgfqpoint{5.062717in}{0.970977in}}%
\pgfpathlineto{\pgfqpoint{5.069711in}{0.986899in}}%
\pgfpathlineto{\pgfqpoint{5.076705in}{1.018743in}}%
\pgfpathlineto{\pgfqpoint{5.097686in}{1.066508in}}%
\pgfpathlineto{\pgfqpoint{5.104680in}{1.098351in}}%
\pgfpathlineto{\pgfqpoint{5.118667in}{1.130195in}}%
\pgfpathlineto{\pgfqpoint{5.125661in}{1.130195in}}%
\pgfpathlineto{\pgfqpoint{5.132655in}{1.162038in}}%
\pgfpathlineto{\pgfqpoint{5.139649in}{1.162038in}}%
\pgfpathlineto{\pgfqpoint{5.160630in}{1.209804in}}%
\pgfpathlineto{\pgfqpoint{5.167624in}{1.209804in}}%
\pgfpathlineto{\pgfqpoint{5.181611in}{1.241647in}}%
\pgfpathlineto{\pgfqpoint{5.195599in}{1.241647in}}%
\pgfpathlineto{\pgfqpoint{5.202593in}{1.257569in}}%
\pgfpathlineto{\pgfqpoint{5.244556in}{1.257569in}}%
\pgfpathlineto{\pgfqpoint{5.251549in}{1.289412in}}%
\pgfpathlineto{\pgfqpoint{5.258543in}{1.273491in}}%
\pgfpathlineto{\pgfqpoint{5.265537in}{1.305334in}}%
\pgfpathlineto{\pgfqpoint{5.272531in}{1.305334in}}%
\pgfpathlineto{\pgfqpoint{5.279525in}{1.289412in}}%
\pgfpathlineto{\pgfqpoint{5.286518in}{1.305334in}}%
\pgfpathlineto{\pgfqpoint{5.300506in}{1.305334in}}%
\pgfpathlineto{\pgfqpoint{5.307500in}{1.289412in}}%
\pgfpathlineto{\pgfqpoint{5.314494in}{1.321256in}}%
\pgfpathlineto{\pgfqpoint{5.321487in}{1.305334in}}%
\pgfpathlineto{\pgfqpoint{5.328481in}{1.321256in}}%
\pgfpathlineto{\pgfqpoint{5.335475in}{1.321256in}}%
\pgfpathlineto{\pgfqpoint{5.342469in}{1.305334in}}%
\pgfpathlineto{\pgfqpoint{5.349462in}{1.241647in}}%
\pgfpathlineto{\pgfqpoint{5.370444in}{1.098351in}}%
\pgfpathlineto{\pgfqpoint{5.377438in}{1.066508in}}%
\pgfpathlineto{\pgfqpoint{5.384431in}{1.002821in}}%
\pgfpathlineto{\pgfqpoint{5.391425in}{0.986899in}}%
\pgfpathlineto{\pgfqpoint{5.405413in}{0.923212in}}%
\pgfpathlineto{\pgfqpoint{5.412407in}{0.875447in}}%
\pgfpathlineto{\pgfqpoint{5.419400in}{0.875447in}}%
\pgfpathlineto{\pgfqpoint{5.426394in}{0.843603in}}%
\pgfpathlineto{\pgfqpoint{5.440382in}{0.811760in}}%
\pgfpathlineto{\pgfqpoint{5.454369in}{0.748073in}}%
\pgfpathlineto{\pgfqpoint{5.468357in}{0.748073in}}%
\pgfpathlineto{\pgfqpoint{5.475351in}{0.716229in}}%
\pgfpathlineto{\pgfqpoint{5.489338in}{0.684386in}}%
\pgfpathlineto{\pgfqpoint{5.496332in}{0.684386in}}%
\pgfpathlineto{\pgfqpoint{5.503326in}{0.668464in}}%
\pgfpathlineto{\pgfqpoint{5.510320in}{0.668464in}}%
\pgfpathlineto{\pgfqpoint{5.517313in}{0.652542in}}%
\pgfpathlineto{\pgfqpoint{5.524307in}{0.652542in}}%
\pgfpathlineto{\pgfqpoint{5.531301in}{0.620699in}}%
\pgfpathlineto{\pgfqpoint{5.552282in}{0.620699in}}%
\pgfpathlineto{\pgfqpoint{5.559276in}{0.604777in}}%
\pgfpathlineto{\pgfqpoint{5.580258in}{0.604777in}}%
\pgfpathlineto{\pgfqpoint{5.594245in}{0.572934in}}%
\pgfpathlineto{\pgfqpoint{5.601239in}{0.588855in}}%
\pgfpathlineto{\pgfqpoint{5.608233in}{0.588855in}}%
\pgfpathlineto{\pgfqpoint{5.615227in}{0.572934in}}%
\pgfpathlineto{\pgfqpoint{5.622220in}{0.572934in}}%
\pgfpathlineto{\pgfqpoint{5.629214in}{0.557012in}}%
\pgfpathlineto{\pgfqpoint{5.636208in}{0.572934in}}%
\pgfpathlineto{\pgfqpoint{5.643202in}{0.557012in}}%
\pgfpathlineto{\pgfqpoint{5.692158in}{0.557012in}}%
\pgfpathlineto{\pgfqpoint{5.699152in}{0.620699in}}%
\pgfpathlineto{\pgfqpoint{5.727127in}{0.811760in}}%
\pgfpathlineto{\pgfqpoint{5.769090in}{1.002821in}}%
\pgfpathlineto{\pgfqpoint{5.776084in}{1.002821in}}%
\pgfpathlineto{\pgfqpoint{5.790071in}{1.066508in}}%
\pgfpathlineto{\pgfqpoint{5.804059in}{1.098351in}}%
\pgfpathlineto{\pgfqpoint{5.811053in}{1.130195in}}%
\pgfpathlineto{\pgfqpoint{5.818047in}{1.130195in}}%
\pgfpathlineto{\pgfqpoint{5.846022in}{1.193882in}}%
\pgfpathlineto{\pgfqpoint{5.867003in}{1.193882in}}%
\pgfpathlineto{\pgfqpoint{5.894978in}{1.257569in}}%
\pgfpathlineto{\pgfqpoint{5.922953in}{1.257569in}}%
\pgfpathlineto{\pgfqpoint{5.929947in}{1.273491in}}%
\pgfpathlineto{\pgfqpoint{5.943935in}{1.273491in}}%
\pgfpathlineto{\pgfqpoint{5.950929in}{1.289412in}}%
\pgfpathlineto{\pgfqpoint{5.957922in}{1.273491in}}%
\pgfpathlineto{\pgfqpoint{5.964916in}{1.273491in}}%
\pgfpathlineto{\pgfqpoint{5.971910in}{1.305334in}}%
\pgfpathlineto{\pgfqpoint{5.978904in}{1.289412in}}%
\pgfpathlineto{\pgfqpoint{5.985898in}{1.289412in}}%
\pgfpathlineto{\pgfqpoint{5.992891in}{1.321256in}}%
\pgfpathlineto{\pgfqpoint{5.999885in}{1.289412in}}%
\pgfpathlineto{\pgfqpoint{6.006879in}{1.321256in}}%
\pgfpathlineto{\pgfqpoint{6.041848in}{1.321256in}}%
\pgfpathlineto{\pgfqpoint{6.048842in}{1.241647in}}%
\pgfpathlineto{\pgfqpoint{6.069823in}{1.098351in}}%
\pgfpathlineto{\pgfqpoint{6.076817in}{1.066508in}}%
\pgfpathlineto{\pgfqpoint{6.083811in}{1.002821in}}%
\pgfpathlineto{\pgfqpoint{6.090805in}{0.986899in}}%
\pgfpathlineto{\pgfqpoint{6.097798in}{0.939134in}}%
\pgfpathlineto{\pgfqpoint{6.104792in}{0.907290in}}%
\pgfpathlineto{\pgfqpoint{6.111786in}{0.891369in}}%
\pgfpathlineto{\pgfqpoint{6.118780in}{0.859525in}}%
\pgfpathlineto{\pgfqpoint{6.125773in}{0.843603in}}%
\pgfpathlineto{\pgfqpoint{6.132767in}{0.811760in}}%
\pgfpathlineto{\pgfqpoint{6.139761in}{0.811760in}}%
\pgfpathlineto{\pgfqpoint{6.153749in}{0.748073in}}%
\pgfpathlineto{\pgfqpoint{6.160742in}{0.748073in}}%
\pgfpathlineto{\pgfqpoint{6.167736in}{0.716229in}}%
\pgfpathlineto{\pgfqpoint{6.174730in}{0.716229in}}%
\pgfpathlineto{\pgfqpoint{6.181724in}{0.700308in}}%
\pgfpathlineto{\pgfqpoint{6.181724in}{0.700308in}}%
\pgfusepath{stroke}%
\end{pgfscope}%
\begin{pgfscope}%
\pgfpathrectangle{\pgfqpoint{4.720371in}{0.484568in}}{\pgfqpoint{1.530941in}{0.893210in}}%
\pgfusepath{clip}%
\pgfsetrectcap%
\pgfsetroundjoin%
\pgfsetlinewidth{1.505625pt}%
\definecolor{currentstroke}{rgb}{0.839216,0.152941,0.156863}%
\pgfsetstrokecolor{currentstroke}%
\pgfsetdash{}{0pt}%
\pgfpathmoveto{\pgfqpoint{0.000000in}{0.000000in}}%
\pgfusepath{stroke}%
\end{pgfscope}%
\begin{pgfscope}%
\pgfsetrectcap%
\pgfsetmiterjoin%
\pgfsetlinewidth{0.803000pt}%
\definecolor{currentstroke}{rgb}{0.000000,0.000000,0.000000}%
\pgfsetstrokecolor{currentstroke}%
\pgfsetdash{}{0pt}%
\pgfpathmoveto{\pgfqpoint{4.720371in}{0.484568in}}%
\pgfpathlineto{\pgfqpoint{4.720371in}{1.377778in}}%
\pgfusepath{stroke}%
\end{pgfscope}%
\begin{pgfscope}%
\pgfsetrectcap%
\pgfsetmiterjoin%
\pgfsetlinewidth{0.803000pt}%
\definecolor{currentstroke}{rgb}{0.000000,0.000000,0.000000}%
\pgfsetstrokecolor{currentstroke}%
\pgfsetdash{}{0pt}%
\pgfpathmoveto{\pgfqpoint{6.251312in}{0.484568in}}%
\pgfpathlineto{\pgfqpoint{6.251312in}{1.377778in}}%
\pgfusepath{stroke}%
\end{pgfscope}%
\begin{pgfscope}%
\pgfsetrectcap%
\pgfsetmiterjoin%
\pgfsetlinewidth{0.803000pt}%
\definecolor{currentstroke}{rgb}{0.000000,0.000000,0.000000}%
\pgfsetstrokecolor{currentstroke}%
\pgfsetdash{}{0pt}%
\pgfpathmoveto{\pgfqpoint{4.720371in}{0.484568in}}%
\pgfpathlineto{\pgfqpoint{6.251312in}{0.484568in}}%
\pgfusepath{stroke}%
\end{pgfscope}%
\begin{pgfscope}%
\pgfsetrectcap%
\pgfsetmiterjoin%
\pgfsetlinewidth{0.803000pt}%
\definecolor{currentstroke}{rgb}{0.000000,0.000000,0.000000}%
\pgfsetstrokecolor{currentstroke}%
\pgfsetdash{}{0pt}%
\pgfpathmoveto{\pgfqpoint{4.720371in}{1.377778in}}%
\pgfpathlineto{\pgfqpoint{6.251312in}{1.377778in}}%
\pgfusepath{stroke}%
\end{pgfscope}%
\begin{pgfscope}%
\definecolor{textcolor}{rgb}{0.000000,0.000000,0.000000}%
\pgfsetstrokecolor{textcolor}%
\pgfsetfillcolor{textcolor}%
\pgftext[x=5.485841in,y=1.461111in,,base]{\color{textcolor}\rmfamily\fontsize{12.000000}{14.400000}\selectfont 8}%
\end{pgfscope}%
\end{pgfpicture}%
\makeatother%
\endgroup%

        \end{center}
        \caption{Some imported matplotlib plot}\label{fig:figure}
    \end{figure}


\section{Some circuit}

\begin{circuitikz} \draw
    (2,0)to[battery, l_=$\epsilon$](4,0)
    (4,0)to[R=$R_1$](4,2)
    (2,0)to[R=$R_2$,*-*](2,4)
    (0,0)to[R=$R_3$](0,2)
    (0,2)--(2,4)
    (4,2)--(2,4)
    (0,0)--(2,0)
    ;
    \label{circuit}
\end{circuitikz}

\section{Hello nice code}
\begin{center}
    \begin{minted}[frame=single,]{cpp}
#include <iostream>
int main() { // Thats a main function
    std::cout << "Hello world!" << std::endl;    
    return 0;
}
\end{minted}
\end{center}

% Introduction and Overview
\section{Introduction and Overview}
Add your introduction and overview here.

% Example Subsection
\subsection{Subsection Title}
This is a subsection.

% Example Subsubsection
\subsubsection{Subsubsection Title}
This is a subsubsection.

%  Theoretical Background
\section{Theoretical Background}
Add your theoretical background here. Some example text: As we learned from our textbook \cite{kutz_2013}, Fourier introduced the concept of representing a given function $f(x)$ by a trigonometric series of sines and cosines:
\begin{equation}
    f(x) = \frac{a_0}{2} + \sum_{i=1}^\infty \left(a_n\cos{nx} + b_n\sin{nx}\right) \quad x \in (-\pi,\pi].
    \label{eqn:fourierseries}
\end{equation}
You can reference numbered equations, figures, tables, algorithms, and code like this: Equation~\ref{eqn:fourierseries}, etc.

% Algorithm Implementation and Development
\section{Algorithm Implementation and Development}
Add your algorithm implementation and development here. See Algorithm~\ref{alg:example} for how to include an algorithm in your document. This is how to make an \textit{ordered} list:
\begin{enumerate}
    \item Fluffy swallowed a marble.
    \item I took Fluffy to the vet.
    \item They took an ultrasound of Fluffy's intestines.
\end{enumerate}

% Computational Results
\section{Computational Results}
\begin{table}
    \centering
    \begin{tabular}{rll}
           & Name             & Years        \\
        \hline
        1  & Frosty           & 1922-1930    \\
        2  & Frosty II        & 1930-1936    \\
        3  & Wasky            & 1946         \\
        4  & Wasky II         & 1947         \\
        5  & Ski              & 1954         \\
        6  & Denali           & 1958         \\
        7  & King Chinook     & 1959-1968    \\
        8  & Regent Denali    & 1969         \\
        9  & Sundodger Denali & 1981-1992    \\
        10 & King Redoubt     & 1992-1998    \\
        11 & Prince Redoubt   & 1998         \\
        12 & Spirit           & 1999-2008    \\
        13 & Dubs I           & 2009-2018    \\
        14 & Dubs II          & 2018-Present
    \end{tabular}
    \caption{UW mascots as described in \cite{washington_huskies}.}
    \label{tab:mascots}
\end{table}


% Summary and Conclusions
\section{Summary and Conclusions}
Add your summary and conclusions here.

% Appendices
\begin{appendices}

    % MATLAB Functions
    \section{MATLAB Functions}
    Add your important MATLAB functions here with a brief implementation explanation. This is how to make an \textbf{unordered} list:
    \begin{itemize}
        \item \texttt{y = linspace(x1,x2,n)} returns a row vector of \texttt{n} evenly spaced points between \texttt{x1} and \texttt{x2}.
        \item \texttt{[X,Y] = meshgrid(x,y)} returns 2-D grid coordinates based on the coordinates contained in the vectors \texttt{x} and \texttt{y}. \text{X} is a matrix where each row is a copy of \texttt{x}, and \texttt{Y} is a matrix where each column is a copy of \texttt{y}. The grid represented by the coordinates \texttt{X} and \texttt{Y} has \texttt{length(y)} rows and \texttt{length(x)} columns.
    \end{itemize}


\end{appendices}

\end{document}
