\documentclass{article}
\usepackage[utf8]{inputenc}
\usepackage[margin=1in]{geometry}
\usepackage[titletoc,title]{appendix}
\usepackage{pgfplots}

% \usepackage[T1]{fontenc}
% \usepackage{lmodern}
% \usepackage{amsmath,amsfonts,amssymb,mathtools}
\usepackage{graphicx,float}
\pgfplotsset{compat=1.18}
% \usepackage{minted}
% \usepackage{circuitikz}
% \usemintedstyle{trac}
\graphicspath{plots}


\usepackage[backend=biber,
    style=numeric-comp,
    maxcitenames=1,
    maxbibnames=1,
    %backref=true
    ]{biblatex}
\addbibresource{references.bib}
% \addbibresource{references_2.bib}

\title{My text title}
\author{Piotr Drabik, Mateusz Kojro}
\date{\today}

\begin{document}

\maketitle

\begin{abstract}
In this paper we present the feasibility study of using deep learning techniques to cryptographically attack components of the AES encryption algorithm. We compare the relative performance of spiking and traditional neural networks in regression tasks at different levels of complexity. We show that due to encoding issues (conversion from numeric values to spiking time series) SNN cannot be trivially applied to this kinds of problems.
\end{abstract}


\section{Introduction}
Advanced Encryption standard (AES) is one of the most widely used encryption algorithms, it's resilience to direct attacks guarantees safe data encryption. Simple construction of cipher guarantees fast encryption that can be chained together, further complicating any efforts of direct attack, unfortunately the simplicity of algorithm in the future may lead to mathematical solution, that reverses the algorithm without need for knowledge of used key.

Spiking neural networks (SNN) provide new deep-learning solution more closely resembling neurons found in nature. New neuron model might enable faster learning rate crucial when working with big sets of data.

\section{Motivation}

Deep learning has proven over the years its ability for modeling and solving problems that were thought to be nearly impossible. Given this track record and emergence of new ideas in this space problems once though to be exceptionally hard need to be examine more closely with the context of recent advancements. In this paper we will investigate the feasibility of using deep learning (namely standard neural networks and spiking neural networks) in attacks on the Rijmen's mix columns algorithm - a major component of the ubiquitous AES encryption method.

\section{Theoretical background}

\subsection{Advanced Encryption Standard}

Introduced by Joan Daemen and Vincent Rijmen in year 1998, encryption standard is a widely used, fixed block size, symmetric encryption algorithm. Adopted by U.S. AES superseded DES(Data Encryption Standard) in year 2002 \cite{Cryptanalysis of Block Ciphers}.

\subsubsection{Known Attacks}
To this day AES cipher has not been broken, meaning there are no known algorithms that can decipher an encrypted message without knowledge of used key, in time shorted than needed to perform brute-force attack.
In year 2002 a theoretical attack, named the "XSL attack", was announced by Nicolas Courtois and Josef Pieprzyk, the soul existence of mathematical solutions to the algorithm behind AES, poses questions about it's security. Provided by Courtois and Pieprzyk mathematical equations used to break the algorithm require enormous amounts of computational power and time, this problem might be mitigated by usage of neural-networks \cite{Cryptanalysis of Block Ciphers}.

\subsubsection{Building blocks}
AES consists of 4 main building blocks, static sized key and Information block. 

\begin{itemize}
\item Encryption Key, based on chosen block size key is sized accordingly. The size of used Key specifies number of "rounds" of encoding, for 128-bit keys it is 10 rounds of encoding, 12 for 192-bit keys, 14 for 256-bit. 
\item Information block, encoded data is spited into chunks of sizes: 128, 192 or 256 bit, those are encoded separately.  
\end{itemize}
Building blocks of AES algorithm:
\begin{itemize}
\item Byte Sub
S-Box(substitution-box) stage, used to obscure relationship between key and cipher text, thus ensuring Shannon's property of confusion.
\item Shift Row, a transposition step where the rows of the state are shifted cyclically a certain number of steps.
\item Mix Column, a linear mixing operation which operates on the columns of the state, combining the four bytes in each column.
\item Add Round Key, each byte of data block is combined with key with xor operation 
\end{itemize}
 To every chunk of message those stages are applied in succession, this operation is repeated multiple times depending on key size

\subsection{Spiking neural networks}

Despite its successes in many areas across number of industries broad field of deep learning is constantly expanding and evolving. One of the results of this process is the introduction of neuron models more closely resembling those found in the human brain. One of the most popular examples of this trend are "leaky integrate and fire" neurons. They allow for an introduction of time dependent activation function and doing so are overstepping one of the big simplifications introduced by traditional deep learning models. Networks composed of this kinds of neurons are often called spiking neural networks (SNN).

\section{Methods}
\subsection{Models}

In order to compare performance of SNN and NN we prepared two simple models:

\subsubsection{SNN Model}
\subsubsection{NN Model}

\subsection{Dataset}

In order to test and compare the introduced models <we> prepared 2 datasets consisting of randomly generated <...>

\begin{itemize}
    \item The training dataset: N encoded massages with the <some> message size
    \item Evaluation dataset composed of N messages with the same sizes as the training dataset
\end{itemize}

\section{Training}

We trained both models using the same dataset for <x> epochs with <y> minibatches and evaluated the performance using a testing dataset 


\section{Results}



\section{Discussion}

As we have shown in spite of the many advantages they offer SNN are not really well fit for regression type tasks like approximating mathematical functions. Main reason for that is the difficulty of encoding and decoding the spikes because there is no well known and understood method for this operation. Research done by \cite{kahana_function_2022} shows that there is a possibility of using auto encoders like DeepONet \cite{lu_learning_2021} to achieve this but the work is still in the early stages (\cite{kahana_function_2022} shows SNN with DeepONet assistance learning functions like $x^2$ and $\sin{x}$). Additionally while using SNN without specialized hardware\cite{bouvier_spiking_2019} the training time increases dramatically \cite{kahana_function_2022}.





\begin{figure}
    %% Creator: Matplotlib, PGF backend
%%
%% To include the figure in your LaTeX document, write
%%   \input{<filename>.pgf}
%%
%% Make sure the required packages are loaded in your preamble
%%   \usepackage{pgf}
%%
%% Also ensure that all the required font packages are loaded; for instance,
%% the lmodern package is sometimes necessary when using math font.
%%   \usepackage{lmodern}
%%
%% Figures using additional raster images can only be included by \input if
%% they are in the same directory as the main LaTeX file. For loading figures
%% from other directories you can use the `import` package
%%   \usepackage{import}
%%
%% and then include the figures with
%%   \import{<path to file>}{<filename>.pgf}
%%
%% Matplotlib used the following preamble
%%   \usepackage{fontspec}
%%   \setmainfont{DejaVuSerif.ttf}[Path=\detokenize{/Users/mkojro/miniforge3/envs/nn-crypto/lib/python3.10/site-packages/matplotlib/mpl-data/fonts/ttf/}]
%%   \setsansfont{DejaVuSans.ttf}[Path=\detokenize{/Users/mkojro/miniforge3/envs/nn-crypto/lib/python3.10/site-packages/matplotlib/mpl-data/fonts/ttf/}]
%%   \setmonofont{DejaVuSansMono.ttf}[Path=\detokenize{/Users/mkojro/miniforge3/envs/nn-crypto/lib/python3.10/site-packages/matplotlib/mpl-data/fonts/ttf/}]
%%
\begingroup%
\makeatletter%
\begin{pgfpicture}%
\pgfpathrectangle{\pgfpointorigin}{\pgfqpoint{6.000000in}{4.000000in}}%
\pgfusepath{use as bounding box, clip}%
\begin{pgfscope}%
\pgfsetbuttcap%
\pgfsetmiterjoin%
\pgfsetlinewidth{0.000000pt}%
\definecolor{currentstroke}{rgb}{1.000000,1.000000,1.000000}%
\pgfsetstrokecolor{currentstroke}%
\pgfsetstrokeopacity{0.000000}%
\pgfsetdash{}{0pt}%
\pgfpathmoveto{\pgfqpoint{0.000000in}{0.000000in}}%
\pgfpathlineto{\pgfqpoint{6.000000in}{0.000000in}}%
\pgfpathlineto{\pgfqpoint{6.000000in}{4.000000in}}%
\pgfpathlineto{\pgfqpoint{0.000000in}{4.000000in}}%
\pgfpathlineto{\pgfqpoint{0.000000in}{0.000000in}}%
\pgfpathclose%
\pgfusepath{}%
\end{pgfscope}%
\begin{pgfscope}%
\pgfsetbuttcap%
\pgfsetmiterjoin%
\definecolor{currentfill}{rgb}{1.000000,1.000000,1.000000}%
\pgfsetfillcolor{currentfill}%
\pgfsetlinewidth{0.000000pt}%
\definecolor{currentstroke}{rgb}{0.000000,0.000000,0.000000}%
\pgfsetstrokecolor{currentstroke}%
\pgfsetstrokeopacity{0.000000}%
\pgfsetdash{}{0pt}%
\pgfpathmoveto{\pgfqpoint{0.750000in}{0.500000in}}%
\pgfpathlineto{\pgfqpoint{5.400000in}{0.500000in}}%
\pgfpathlineto{\pgfqpoint{5.400000in}{3.520000in}}%
\pgfpathlineto{\pgfqpoint{0.750000in}{3.520000in}}%
\pgfpathlineto{\pgfqpoint{0.750000in}{0.500000in}}%
\pgfpathclose%
\pgfusepath{fill}%
\end{pgfscope}%
\begin{pgfscope}%
\pgfpathrectangle{\pgfqpoint{0.750000in}{0.500000in}}{\pgfqpoint{4.650000in}{3.020000in}}%
\pgfusepath{clip}%
\pgfsetbuttcap%
\pgfsetmiterjoin%
\definecolor{currentfill}{rgb}{1.000000,0.000000,0.000000}%
\pgfsetfillcolor{currentfill}%
\pgfsetlinewidth{0.000000pt}%
\definecolor{currentstroke}{rgb}{0.000000,0.000000,0.000000}%
\pgfsetstrokecolor{currentstroke}%
\pgfsetstrokeopacity{0.000000}%
\pgfsetdash{}{0pt}%
\pgfpathmoveto{\pgfqpoint{0.961364in}{0.500000in}}%
\pgfpathlineto{\pgfqpoint{0.994389in}{0.500000in}}%
\pgfpathlineto{\pgfqpoint{0.994389in}{0.502926in}}%
\pgfpathlineto{\pgfqpoint{0.961364in}{0.502926in}}%
\pgfpathlineto{\pgfqpoint{0.961364in}{0.500000in}}%
\pgfpathclose%
\pgfusepath{fill}%
\end{pgfscope}%
\begin{pgfscope}%
\pgfpathrectangle{\pgfqpoint{0.750000in}{0.500000in}}{\pgfqpoint{4.650000in}{3.020000in}}%
\pgfusepath{clip}%
\pgfsetbuttcap%
\pgfsetmiterjoin%
\definecolor{currentfill}{rgb}{1.000000,0.000000,0.000000}%
\pgfsetfillcolor{currentfill}%
\pgfsetlinewidth{0.000000pt}%
\definecolor{currentstroke}{rgb}{0.000000,0.000000,0.000000}%
\pgfsetstrokecolor{currentstroke}%
\pgfsetstrokeopacity{0.000000}%
\pgfsetdash{}{0pt}%
\pgfpathmoveto{\pgfqpoint{0.994389in}{0.500000in}}%
\pgfpathlineto{\pgfqpoint{1.027415in}{0.500000in}}%
\pgfpathlineto{\pgfqpoint{1.027415in}{0.500000in}}%
\pgfpathlineto{\pgfqpoint{0.994389in}{0.500000in}}%
\pgfpathlineto{\pgfqpoint{0.994389in}{0.500000in}}%
\pgfpathclose%
\pgfusepath{fill}%
\end{pgfscope}%
\begin{pgfscope}%
\pgfpathrectangle{\pgfqpoint{0.750000in}{0.500000in}}{\pgfqpoint{4.650000in}{3.020000in}}%
\pgfusepath{clip}%
\pgfsetbuttcap%
\pgfsetmiterjoin%
\definecolor{currentfill}{rgb}{1.000000,0.000000,0.000000}%
\pgfsetfillcolor{currentfill}%
\pgfsetlinewidth{0.000000pt}%
\definecolor{currentstroke}{rgb}{0.000000,0.000000,0.000000}%
\pgfsetstrokecolor{currentstroke}%
\pgfsetstrokeopacity{0.000000}%
\pgfsetdash{}{0pt}%
\pgfpathmoveto{\pgfqpoint{1.027415in}{0.500000in}}%
\pgfpathlineto{\pgfqpoint{1.060440in}{0.500000in}}%
\pgfpathlineto{\pgfqpoint{1.060440in}{0.505852in}}%
\pgfpathlineto{\pgfqpoint{1.027415in}{0.505852in}}%
\pgfpathlineto{\pgfqpoint{1.027415in}{0.500000in}}%
\pgfpathclose%
\pgfusepath{fill}%
\end{pgfscope}%
\begin{pgfscope}%
\pgfpathrectangle{\pgfqpoint{0.750000in}{0.500000in}}{\pgfqpoint{4.650000in}{3.020000in}}%
\pgfusepath{clip}%
\pgfsetbuttcap%
\pgfsetmiterjoin%
\definecolor{currentfill}{rgb}{1.000000,0.000000,0.000000}%
\pgfsetfillcolor{currentfill}%
\pgfsetlinewidth{0.000000pt}%
\definecolor{currentstroke}{rgb}{0.000000,0.000000,0.000000}%
\pgfsetstrokecolor{currentstroke}%
\pgfsetstrokeopacity{0.000000}%
\pgfsetdash{}{0pt}%
\pgfpathmoveto{\pgfqpoint{1.060440in}{0.500000in}}%
\pgfpathlineto{\pgfqpoint{1.093466in}{0.500000in}}%
\pgfpathlineto{\pgfqpoint{1.093466in}{0.500000in}}%
\pgfpathlineto{\pgfqpoint{1.060440in}{0.500000in}}%
\pgfpathlineto{\pgfqpoint{1.060440in}{0.500000in}}%
\pgfpathclose%
\pgfusepath{fill}%
\end{pgfscope}%
\begin{pgfscope}%
\pgfpathrectangle{\pgfqpoint{0.750000in}{0.500000in}}{\pgfqpoint{4.650000in}{3.020000in}}%
\pgfusepath{clip}%
\pgfsetbuttcap%
\pgfsetmiterjoin%
\definecolor{currentfill}{rgb}{1.000000,0.000000,0.000000}%
\pgfsetfillcolor{currentfill}%
\pgfsetlinewidth{0.000000pt}%
\definecolor{currentstroke}{rgb}{0.000000,0.000000,0.000000}%
\pgfsetstrokecolor{currentstroke}%
\pgfsetstrokeopacity{0.000000}%
\pgfsetdash{}{0pt}%
\pgfpathmoveto{\pgfqpoint{1.093466in}{0.500000in}}%
\pgfpathlineto{\pgfqpoint{1.126491in}{0.500000in}}%
\pgfpathlineto{\pgfqpoint{1.126491in}{0.500000in}}%
\pgfpathlineto{\pgfqpoint{1.093466in}{0.500000in}}%
\pgfpathlineto{\pgfqpoint{1.093466in}{0.500000in}}%
\pgfpathclose%
\pgfusepath{fill}%
\end{pgfscope}%
\begin{pgfscope}%
\pgfpathrectangle{\pgfqpoint{0.750000in}{0.500000in}}{\pgfqpoint{4.650000in}{3.020000in}}%
\pgfusepath{clip}%
\pgfsetbuttcap%
\pgfsetmiterjoin%
\definecolor{currentfill}{rgb}{1.000000,0.000000,0.000000}%
\pgfsetfillcolor{currentfill}%
\pgfsetlinewidth{0.000000pt}%
\definecolor{currentstroke}{rgb}{0.000000,0.000000,0.000000}%
\pgfsetstrokecolor{currentstroke}%
\pgfsetstrokeopacity{0.000000}%
\pgfsetdash{}{0pt}%
\pgfpathmoveto{\pgfqpoint{1.126491in}{0.500000in}}%
\pgfpathlineto{\pgfqpoint{1.159517in}{0.500000in}}%
\pgfpathlineto{\pgfqpoint{1.159517in}{0.517556in}}%
\pgfpathlineto{\pgfqpoint{1.126491in}{0.517556in}}%
\pgfpathlineto{\pgfqpoint{1.126491in}{0.500000in}}%
\pgfpathclose%
\pgfusepath{fill}%
\end{pgfscope}%
\begin{pgfscope}%
\pgfpathrectangle{\pgfqpoint{0.750000in}{0.500000in}}{\pgfqpoint{4.650000in}{3.020000in}}%
\pgfusepath{clip}%
\pgfsetbuttcap%
\pgfsetmiterjoin%
\definecolor{currentfill}{rgb}{1.000000,0.000000,0.000000}%
\pgfsetfillcolor{currentfill}%
\pgfsetlinewidth{0.000000pt}%
\definecolor{currentstroke}{rgb}{0.000000,0.000000,0.000000}%
\pgfsetstrokecolor{currentstroke}%
\pgfsetstrokeopacity{0.000000}%
\pgfsetdash{}{0pt}%
\pgfpathmoveto{\pgfqpoint{1.159517in}{0.500000in}}%
\pgfpathlineto{\pgfqpoint{1.192543in}{0.500000in}}%
\pgfpathlineto{\pgfqpoint{1.192543in}{0.500000in}}%
\pgfpathlineto{\pgfqpoint{1.159517in}{0.500000in}}%
\pgfpathlineto{\pgfqpoint{1.159517in}{0.500000in}}%
\pgfpathclose%
\pgfusepath{fill}%
\end{pgfscope}%
\begin{pgfscope}%
\pgfpathrectangle{\pgfqpoint{0.750000in}{0.500000in}}{\pgfqpoint{4.650000in}{3.020000in}}%
\pgfusepath{clip}%
\pgfsetbuttcap%
\pgfsetmiterjoin%
\definecolor{currentfill}{rgb}{1.000000,0.000000,0.000000}%
\pgfsetfillcolor{currentfill}%
\pgfsetlinewidth{0.000000pt}%
\definecolor{currentstroke}{rgb}{0.000000,0.000000,0.000000}%
\pgfsetstrokecolor{currentstroke}%
\pgfsetstrokeopacity{0.000000}%
\pgfsetdash{}{0pt}%
\pgfpathmoveto{\pgfqpoint{1.192543in}{0.500000in}}%
\pgfpathlineto{\pgfqpoint{1.225568in}{0.500000in}}%
\pgfpathlineto{\pgfqpoint{1.225568in}{0.502926in}}%
\pgfpathlineto{\pgfqpoint{1.192543in}{0.502926in}}%
\pgfpathlineto{\pgfqpoint{1.192543in}{0.500000in}}%
\pgfpathclose%
\pgfusepath{fill}%
\end{pgfscope}%
\begin{pgfscope}%
\pgfpathrectangle{\pgfqpoint{0.750000in}{0.500000in}}{\pgfqpoint{4.650000in}{3.020000in}}%
\pgfusepath{clip}%
\pgfsetbuttcap%
\pgfsetmiterjoin%
\definecolor{currentfill}{rgb}{1.000000,0.000000,0.000000}%
\pgfsetfillcolor{currentfill}%
\pgfsetlinewidth{0.000000pt}%
\definecolor{currentstroke}{rgb}{0.000000,0.000000,0.000000}%
\pgfsetstrokecolor{currentstroke}%
\pgfsetstrokeopacity{0.000000}%
\pgfsetdash{}{0pt}%
\pgfpathmoveto{\pgfqpoint{1.225568in}{0.500000in}}%
\pgfpathlineto{\pgfqpoint{1.258594in}{0.500000in}}%
\pgfpathlineto{\pgfqpoint{1.258594in}{0.529259in}}%
\pgfpathlineto{\pgfqpoint{1.225568in}{0.529259in}}%
\pgfpathlineto{\pgfqpoint{1.225568in}{0.500000in}}%
\pgfpathclose%
\pgfusepath{fill}%
\end{pgfscope}%
\begin{pgfscope}%
\pgfpathrectangle{\pgfqpoint{0.750000in}{0.500000in}}{\pgfqpoint{4.650000in}{3.020000in}}%
\pgfusepath{clip}%
\pgfsetbuttcap%
\pgfsetmiterjoin%
\definecolor{currentfill}{rgb}{1.000000,0.000000,0.000000}%
\pgfsetfillcolor{currentfill}%
\pgfsetlinewidth{0.000000pt}%
\definecolor{currentstroke}{rgb}{0.000000,0.000000,0.000000}%
\pgfsetstrokecolor{currentstroke}%
\pgfsetstrokeopacity{0.000000}%
\pgfsetdash{}{0pt}%
\pgfpathmoveto{\pgfqpoint{1.258594in}{0.500000in}}%
\pgfpathlineto{\pgfqpoint{1.291619in}{0.500000in}}%
\pgfpathlineto{\pgfqpoint{1.291619in}{0.500000in}}%
\pgfpathlineto{\pgfqpoint{1.258594in}{0.500000in}}%
\pgfpathlineto{\pgfqpoint{1.258594in}{0.500000in}}%
\pgfpathclose%
\pgfusepath{fill}%
\end{pgfscope}%
\begin{pgfscope}%
\pgfpathrectangle{\pgfqpoint{0.750000in}{0.500000in}}{\pgfqpoint{4.650000in}{3.020000in}}%
\pgfusepath{clip}%
\pgfsetbuttcap%
\pgfsetmiterjoin%
\definecolor{currentfill}{rgb}{1.000000,0.000000,0.000000}%
\pgfsetfillcolor{currentfill}%
\pgfsetlinewidth{0.000000pt}%
\definecolor{currentstroke}{rgb}{0.000000,0.000000,0.000000}%
\pgfsetstrokecolor{currentstroke}%
\pgfsetstrokeopacity{0.000000}%
\pgfsetdash{}{0pt}%
\pgfpathmoveto{\pgfqpoint{1.291619in}{0.500000in}}%
\pgfpathlineto{\pgfqpoint{1.324645in}{0.500000in}}%
\pgfpathlineto{\pgfqpoint{1.324645in}{0.579000in}}%
\pgfpathlineto{\pgfqpoint{1.291619in}{0.579000in}}%
\pgfpathlineto{\pgfqpoint{1.291619in}{0.500000in}}%
\pgfpathclose%
\pgfusepath{fill}%
\end{pgfscope}%
\begin{pgfscope}%
\pgfpathrectangle{\pgfqpoint{0.750000in}{0.500000in}}{\pgfqpoint{4.650000in}{3.020000in}}%
\pgfusepath{clip}%
\pgfsetbuttcap%
\pgfsetmiterjoin%
\definecolor{currentfill}{rgb}{1.000000,0.000000,0.000000}%
\pgfsetfillcolor{currentfill}%
\pgfsetlinewidth{0.000000pt}%
\definecolor{currentstroke}{rgb}{0.000000,0.000000,0.000000}%
\pgfsetstrokecolor{currentstroke}%
\pgfsetstrokeopacity{0.000000}%
\pgfsetdash{}{0pt}%
\pgfpathmoveto{\pgfqpoint{1.324645in}{0.500000in}}%
\pgfpathlineto{\pgfqpoint{1.357670in}{0.500000in}}%
\pgfpathlineto{\pgfqpoint{1.357670in}{0.502926in}}%
\pgfpathlineto{\pgfqpoint{1.324645in}{0.502926in}}%
\pgfpathlineto{\pgfqpoint{1.324645in}{0.500000in}}%
\pgfpathclose%
\pgfusepath{fill}%
\end{pgfscope}%
\begin{pgfscope}%
\pgfpathrectangle{\pgfqpoint{0.750000in}{0.500000in}}{\pgfqpoint{4.650000in}{3.020000in}}%
\pgfusepath{clip}%
\pgfsetbuttcap%
\pgfsetmiterjoin%
\definecolor{currentfill}{rgb}{1.000000,0.000000,0.000000}%
\pgfsetfillcolor{currentfill}%
\pgfsetlinewidth{0.000000pt}%
\definecolor{currentstroke}{rgb}{0.000000,0.000000,0.000000}%
\pgfsetstrokecolor{currentstroke}%
\pgfsetstrokeopacity{0.000000}%
\pgfsetdash{}{0pt}%
\pgfpathmoveto{\pgfqpoint{1.357670in}{0.500000in}}%
\pgfpathlineto{\pgfqpoint{1.390696in}{0.500000in}}%
\pgfpathlineto{\pgfqpoint{1.390696in}{0.500000in}}%
\pgfpathlineto{\pgfqpoint{1.357670in}{0.500000in}}%
\pgfpathlineto{\pgfqpoint{1.357670in}{0.500000in}}%
\pgfpathclose%
\pgfusepath{fill}%
\end{pgfscope}%
\begin{pgfscope}%
\pgfpathrectangle{\pgfqpoint{0.750000in}{0.500000in}}{\pgfqpoint{4.650000in}{3.020000in}}%
\pgfusepath{clip}%
\pgfsetbuttcap%
\pgfsetmiterjoin%
\definecolor{currentfill}{rgb}{1.000000,0.000000,0.000000}%
\pgfsetfillcolor{currentfill}%
\pgfsetlinewidth{0.000000pt}%
\definecolor{currentstroke}{rgb}{0.000000,0.000000,0.000000}%
\pgfsetstrokecolor{currentstroke}%
\pgfsetstrokeopacity{0.000000}%
\pgfsetdash{}{0pt}%
\pgfpathmoveto{\pgfqpoint{1.390696in}{0.500000in}}%
\pgfpathlineto{\pgfqpoint{1.423722in}{0.500000in}}%
\pgfpathlineto{\pgfqpoint{1.423722in}{0.617037in}}%
\pgfpathlineto{\pgfqpoint{1.390696in}{0.617037in}}%
\pgfpathlineto{\pgfqpoint{1.390696in}{0.500000in}}%
\pgfpathclose%
\pgfusepath{fill}%
\end{pgfscope}%
\begin{pgfscope}%
\pgfpathrectangle{\pgfqpoint{0.750000in}{0.500000in}}{\pgfqpoint{4.650000in}{3.020000in}}%
\pgfusepath{clip}%
\pgfsetbuttcap%
\pgfsetmiterjoin%
\definecolor{currentfill}{rgb}{1.000000,0.000000,0.000000}%
\pgfsetfillcolor{currentfill}%
\pgfsetlinewidth{0.000000pt}%
\definecolor{currentstroke}{rgb}{0.000000,0.000000,0.000000}%
\pgfsetstrokecolor{currentstroke}%
\pgfsetstrokeopacity{0.000000}%
\pgfsetdash{}{0pt}%
\pgfpathmoveto{\pgfqpoint{1.423722in}{0.500000in}}%
\pgfpathlineto{\pgfqpoint{1.456747in}{0.500000in}}%
\pgfpathlineto{\pgfqpoint{1.456747in}{0.505852in}}%
\pgfpathlineto{\pgfqpoint{1.423722in}{0.505852in}}%
\pgfpathlineto{\pgfqpoint{1.423722in}{0.500000in}}%
\pgfpathclose%
\pgfusepath{fill}%
\end{pgfscope}%
\begin{pgfscope}%
\pgfpathrectangle{\pgfqpoint{0.750000in}{0.500000in}}{\pgfqpoint{4.650000in}{3.020000in}}%
\pgfusepath{clip}%
\pgfsetbuttcap%
\pgfsetmiterjoin%
\definecolor{currentfill}{rgb}{1.000000,0.000000,0.000000}%
\pgfsetfillcolor{currentfill}%
\pgfsetlinewidth{0.000000pt}%
\definecolor{currentstroke}{rgb}{0.000000,0.000000,0.000000}%
\pgfsetstrokecolor{currentstroke}%
\pgfsetstrokeopacity{0.000000}%
\pgfsetdash{}{0pt}%
\pgfpathmoveto{\pgfqpoint{1.456747in}{0.500000in}}%
\pgfpathlineto{\pgfqpoint{1.489773in}{0.500000in}}%
\pgfpathlineto{\pgfqpoint{1.489773in}{0.517556in}}%
\pgfpathlineto{\pgfqpoint{1.456747in}{0.517556in}}%
\pgfpathlineto{\pgfqpoint{1.456747in}{0.500000in}}%
\pgfpathclose%
\pgfusepath{fill}%
\end{pgfscope}%
\begin{pgfscope}%
\pgfpathrectangle{\pgfqpoint{0.750000in}{0.500000in}}{\pgfqpoint{4.650000in}{3.020000in}}%
\pgfusepath{clip}%
\pgfsetbuttcap%
\pgfsetmiterjoin%
\definecolor{currentfill}{rgb}{1.000000,0.000000,0.000000}%
\pgfsetfillcolor{currentfill}%
\pgfsetlinewidth{0.000000pt}%
\definecolor{currentstroke}{rgb}{0.000000,0.000000,0.000000}%
\pgfsetstrokecolor{currentstroke}%
\pgfsetstrokeopacity{0.000000}%
\pgfsetdash{}{0pt}%
\pgfpathmoveto{\pgfqpoint{1.489773in}{0.500000in}}%
\pgfpathlineto{\pgfqpoint{1.522798in}{0.500000in}}%
\pgfpathlineto{\pgfqpoint{1.522798in}{0.707741in}}%
\pgfpathlineto{\pgfqpoint{1.489773in}{0.707741in}}%
\pgfpathlineto{\pgfqpoint{1.489773in}{0.500000in}}%
\pgfpathclose%
\pgfusepath{fill}%
\end{pgfscope}%
\begin{pgfscope}%
\pgfpathrectangle{\pgfqpoint{0.750000in}{0.500000in}}{\pgfqpoint{4.650000in}{3.020000in}}%
\pgfusepath{clip}%
\pgfsetbuttcap%
\pgfsetmiterjoin%
\definecolor{currentfill}{rgb}{1.000000,0.000000,0.000000}%
\pgfsetfillcolor{currentfill}%
\pgfsetlinewidth{0.000000pt}%
\definecolor{currentstroke}{rgb}{0.000000,0.000000,0.000000}%
\pgfsetstrokecolor{currentstroke}%
\pgfsetstrokeopacity{0.000000}%
\pgfsetdash{}{0pt}%
\pgfpathmoveto{\pgfqpoint{1.522798in}{0.500000in}}%
\pgfpathlineto{\pgfqpoint{1.555824in}{0.500000in}}%
\pgfpathlineto{\pgfqpoint{1.555824in}{0.505852in}}%
\pgfpathlineto{\pgfqpoint{1.522798in}{0.505852in}}%
\pgfpathlineto{\pgfqpoint{1.522798in}{0.500000in}}%
\pgfpathclose%
\pgfusepath{fill}%
\end{pgfscope}%
\begin{pgfscope}%
\pgfpathrectangle{\pgfqpoint{0.750000in}{0.500000in}}{\pgfqpoint{4.650000in}{3.020000in}}%
\pgfusepath{clip}%
\pgfsetbuttcap%
\pgfsetmiterjoin%
\definecolor{currentfill}{rgb}{1.000000,0.000000,0.000000}%
\pgfsetfillcolor{currentfill}%
\pgfsetlinewidth{0.000000pt}%
\definecolor{currentstroke}{rgb}{0.000000,0.000000,0.000000}%
\pgfsetstrokecolor{currentstroke}%
\pgfsetstrokeopacity{0.000000}%
\pgfsetdash{}{0pt}%
\pgfpathmoveto{\pgfqpoint{1.555824in}{0.500000in}}%
\pgfpathlineto{\pgfqpoint{1.588849in}{0.500000in}}%
\pgfpathlineto{\pgfqpoint{1.588849in}{0.871593in}}%
\pgfpathlineto{\pgfqpoint{1.555824in}{0.871593in}}%
\pgfpathlineto{\pgfqpoint{1.555824in}{0.500000in}}%
\pgfpathclose%
\pgfusepath{fill}%
\end{pgfscope}%
\begin{pgfscope}%
\pgfpathrectangle{\pgfqpoint{0.750000in}{0.500000in}}{\pgfqpoint{4.650000in}{3.020000in}}%
\pgfusepath{clip}%
\pgfsetbuttcap%
\pgfsetmiterjoin%
\definecolor{currentfill}{rgb}{1.000000,0.000000,0.000000}%
\pgfsetfillcolor{currentfill}%
\pgfsetlinewidth{0.000000pt}%
\definecolor{currentstroke}{rgb}{0.000000,0.000000,0.000000}%
\pgfsetstrokecolor{currentstroke}%
\pgfsetstrokeopacity{0.000000}%
\pgfsetdash{}{0pt}%
\pgfpathmoveto{\pgfqpoint{1.588849in}{0.500000in}}%
\pgfpathlineto{\pgfqpoint{1.621875in}{0.500000in}}%
\pgfpathlineto{\pgfqpoint{1.621875in}{0.511704in}}%
\pgfpathlineto{\pgfqpoint{1.588849in}{0.511704in}}%
\pgfpathlineto{\pgfqpoint{1.588849in}{0.500000in}}%
\pgfpathclose%
\pgfusepath{fill}%
\end{pgfscope}%
\begin{pgfscope}%
\pgfpathrectangle{\pgfqpoint{0.750000in}{0.500000in}}{\pgfqpoint{4.650000in}{3.020000in}}%
\pgfusepath{clip}%
\pgfsetbuttcap%
\pgfsetmiterjoin%
\definecolor{currentfill}{rgb}{1.000000,0.000000,0.000000}%
\pgfsetfillcolor{currentfill}%
\pgfsetlinewidth{0.000000pt}%
\definecolor{currentstroke}{rgb}{0.000000,0.000000,0.000000}%
\pgfsetstrokecolor{currentstroke}%
\pgfsetstrokeopacity{0.000000}%
\pgfsetdash{}{0pt}%
\pgfpathmoveto{\pgfqpoint{1.621875in}{0.500000in}}%
\pgfpathlineto{\pgfqpoint{1.654901in}{0.500000in}}%
\pgfpathlineto{\pgfqpoint{1.654901in}{0.508778in}}%
\pgfpathlineto{\pgfqpoint{1.621875in}{0.508778in}}%
\pgfpathlineto{\pgfqpoint{1.621875in}{0.500000in}}%
\pgfpathclose%
\pgfusepath{fill}%
\end{pgfscope}%
\begin{pgfscope}%
\pgfpathrectangle{\pgfqpoint{0.750000in}{0.500000in}}{\pgfqpoint{4.650000in}{3.020000in}}%
\pgfusepath{clip}%
\pgfsetbuttcap%
\pgfsetmiterjoin%
\definecolor{currentfill}{rgb}{1.000000,0.000000,0.000000}%
\pgfsetfillcolor{currentfill}%
\pgfsetlinewidth{0.000000pt}%
\definecolor{currentstroke}{rgb}{0.000000,0.000000,0.000000}%
\pgfsetstrokecolor{currentstroke}%
\pgfsetstrokeopacity{0.000000}%
\pgfsetdash{}{0pt}%
\pgfpathmoveto{\pgfqpoint{1.654901in}{0.500000in}}%
\pgfpathlineto{\pgfqpoint{1.687926in}{0.500000in}}%
\pgfpathlineto{\pgfqpoint{1.687926in}{1.108594in}}%
\pgfpathlineto{\pgfqpoint{1.654901in}{1.108594in}}%
\pgfpathlineto{\pgfqpoint{1.654901in}{0.500000in}}%
\pgfpathclose%
\pgfusepath{fill}%
\end{pgfscope}%
\begin{pgfscope}%
\pgfpathrectangle{\pgfqpoint{0.750000in}{0.500000in}}{\pgfqpoint{4.650000in}{3.020000in}}%
\pgfusepath{clip}%
\pgfsetbuttcap%
\pgfsetmiterjoin%
\definecolor{currentfill}{rgb}{1.000000,0.000000,0.000000}%
\pgfsetfillcolor{currentfill}%
\pgfsetlinewidth{0.000000pt}%
\definecolor{currentstroke}{rgb}{0.000000,0.000000,0.000000}%
\pgfsetstrokecolor{currentstroke}%
\pgfsetstrokeopacity{0.000000}%
\pgfsetdash{}{0pt}%
\pgfpathmoveto{\pgfqpoint{1.687926in}{0.500000in}}%
\pgfpathlineto{\pgfqpoint{1.720952in}{0.500000in}}%
\pgfpathlineto{\pgfqpoint{1.720952in}{0.526333in}}%
\pgfpathlineto{\pgfqpoint{1.687926in}{0.526333in}}%
\pgfpathlineto{\pgfqpoint{1.687926in}{0.500000in}}%
\pgfpathclose%
\pgfusepath{fill}%
\end{pgfscope}%
\begin{pgfscope}%
\pgfpathrectangle{\pgfqpoint{0.750000in}{0.500000in}}{\pgfqpoint{4.650000in}{3.020000in}}%
\pgfusepath{clip}%
\pgfsetbuttcap%
\pgfsetmiterjoin%
\definecolor{currentfill}{rgb}{1.000000,0.000000,0.000000}%
\pgfsetfillcolor{currentfill}%
\pgfsetlinewidth{0.000000pt}%
\definecolor{currentstroke}{rgb}{0.000000,0.000000,0.000000}%
\pgfsetstrokecolor{currentstroke}%
\pgfsetstrokeopacity{0.000000}%
\pgfsetdash{}{0pt}%
\pgfpathmoveto{\pgfqpoint{1.720952in}{0.500000in}}%
\pgfpathlineto{\pgfqpoint{1.753977in}{0.500000in}}%
\pgfpathlineto{\pgfqpoint{1.753977in}{0.523407in}}%
\pgfpathlineto{\pgfqpoint{1.720952in}{0.523407in}}%
\pgfpathlineto{\pgfqpoint{1.720952in}{0.500000in}}%
\pgfpathclose%
\pgfusepath{fill}%
\end{pgfscope}%
\begin{pgfscope}%
\pgfpathrectangle{\pgfqpoint{0.750000in}{0.500000in}}{\pgfqpoint{4.650000in}{3.020000in}}%
\pgfusepath{clip}%
\pgfsetbuttcap%
\pgfsetmiterjoin%
\definecolor{currentfill}{rgb}{1.000000,0.000000,0.000000}%
\pgfsetfillcolor{currentfill}%
\pgfsetlinewidth{0.000000pt}%
\definecolor{currentstroke}{rgb}{0.000000,0.000000,0.000000}%
\pgfsetstrokecolor{currentstroke}%
\pgfsetstrokeopacity{0.000000}%
\pgfsetdash{}{0pt}%
\pgfpathmoveto{\pgfqpoint{1.753977in}{0.500000in}}%
\pgfpathlineto{\pgfqpoint{1.787003in}{0.500000in}}%
\pgfpathlineto{\pgfqpoint{1.787003in}{1.448002in}}%
\pgfpathlineto{\pgfqpoint{1.753977in}{1.448002in}}%
\pgfpathlineto{\pgfqpoint{1.753977in}{0.500000in}}%
\pgfpathclose%
\pgfusepath{fill}%
\end{pgfscope}%
\begin{pgfscope}%
\pgfpathrectangle{\pgfqpoint{0.750000in}{0.500000in}}{\pgfqpoint{4.650000in}{3.020000in}}%
\pgfusepath{clip}%
\pgfsetbuttcap%
\pgfsetmiterjoin%
\definecolor{currentfill}{rgb}{1.000000,0.000000,0.000000}%
\pgfsetfillcolor{currentfill}%
\pgfsetlinewidth{0.000000pt}%
\definecolor{currentstroke}{rgb}{0.000000,0.000000,0.000000}%
\pgfsetstrokecolor{currentstroke}%
\pgfsetstrokeopacity{0.000000}%
\pgfsetdash{}{0pt}%
\pgfpathmoveto{\pgfqpoint{1.787003in}{0.500000in}}%
\pgfpathlineto{\pgfqpoint{1.820028in}{0.500000in}}%
\pgfpathlineto{\pgfqpoint{1.820028in}{0.511704in}}%
\pgfpathlineto{\pgfqpoint{1.787003in}{0.511704in}}%
\pgfpathlineto{\pgfqpoint{1.787003in}{0.500000in}}%
\pgfpathclose%
\pgfusepath{fill}%
\end{pgfscope}%
\begin{pgfscope}%
\pgfpathrectangle{\pgfqpoint{0.750000in}{0.500000in}}{\pgfqpoint{4.650000in}{3.020000in}}%
\pgfusepath{clip}%
\pgfsetbuttcap%
\pgfsetmiterjoin%
\definecolor{currentfill}{rgb}{1.000000,0.000000,0.000000}%
\pgfsetfillcolor{currentfill}%
\pgfsetlinewidth{0.000000pt}%
\definecolor{currentstroke}{rgb}{0.000000,0.000000,0.000000}%
\pgfsetstrokecolor{currentstroke}%
\pgfsetstrokeopacity{0.000000}%
\pgfsetdash{}{0pt}%
\pgfpathmoveto{\pgfqpoint{1.820028in}{0.500000in}}%
\pgfpathlineto{\pgfqpoint{1.853054in}{0.500000in}}%
\pgfpathlineto{\pgfqpoint{1.853054in}{0.608259in}}%
\pgfpathlineto{\pgfqpoint{1.820028in}{0.608259in}}%
\pgfpathlineto{\pgfqpoint{1.820028in}{0.500000in}}%
\pgfpathclose%
\pgfusepath{fill}%
\end{pgfscope}%
\begin{pgfscope}%
\pgfpathrectangle{\pgfqpoint{0.750000in}{0.500000in}}{\pgfqpoint{4.650000in}{3.020000in}}%
\pgfusepath{clip}%
\pgfsetbuttcap%
\pgfsetmiterjoin%
\definecolor{currentfill}{rgb}{1.000000,0.000000,0.000000}%
\pgfsetfillcolor{currentfill}%
\pgfsetlinewidth{0.000000pt}%
\definecolor{currentstroke}{rgb}{0.000000,0.000000,0.000000}%
\pgfsetstrokecolor{currentstroke}%
\pgfsetstrokeopacity{0.000000}%
\pgfsetdash{}{0pt}%
\pgfpathmoveto{\pgfqpoint{1.853054in}{0.500000in}}%
\pgfpathlineto{\pgfqpoint{1.886080in}{0.500000in}}%
\pgfpathlineto{\pgfqpoint{1.886080in}{1.752299in}}%
\pgfpathlineto{\pgfqpoint{1.853054in}{1.752299in}}%
\pgfpathlineto{\pgfqpoint{1.853054in}{0.500000in}}%
\pgfpathclose%
\pgfusepath{fill}%
\end{pgfscope}%
\begin{pgfscope}%
\pgfpathrectangle{\pgfqpoint{0.750000in}{0.500000in}}{\pgfqpoint{4.650000in}{3.020000in}}%
\pgfusepath{clip}%
\pgfsetbuttcap%
\pgfsetmiterjoin%
\definecolor{currentfill}{rgb}{1.000000,0.000000,0.000000}%
\pgfsetfillcolor{currentfill}%
\pgfsetlinewidth{0.000000pt}%
\definecolor{currentstroke}{rgb}{0.000000,0.000000,0.000000}%
\pgfsetstrokecolor{currentstroke}%
\pgfsetstrokeopacity{0.000000}%
\pgfsetdash{}{0pt}%
\pgfpathmoveto{\pgfqpoint{1.886080in}{0.500000in}}%
\pgfpathlineto{\pgfqpoint{1.919105in}{0.500000in}}%
\pgfpathlineto{\pgfqpoint{1.919105in}{0.532185in}}%
\pgfpathlineto{\pgfqpoint{1.886080in}{0.532185in}}%
\pgfpathlineto{\pgfqpoint{1.886080in}{0.500000in}}%
\pgfpathclose%
\pgfusepath{fill}%
\end{pgfscope}%
\begin{pgfscope}%
\pgfpathrectangle{\pgfqpoint{0.750000in}{0.500000in}}{\pgfqpoint{4.650000in}{3.020000in}}%
\pgfusepath{clip}%
\pgfsetbuttcap%
\pgfsetmiterjoin%
\definecolor{currentfill}{rgb}{1.000000,0.000000,0.000000}%
\pgfsetfillcolor{currentfill}%
\pgfsetlinewidth{0.000000pt}%
\definecolor{currentstroke}{rgb}{0.000000,0.000000,0.000000}%
\pgfsetstrokecolor{currentstroke}%
\pgfsetstrokeopacity{0.000000}%
\pgfsetdash{}{0pt}%
\pgfpathmoveto{\pgfqpoint{1.919105in}{0.500000in}}%
\pgfpathlineto{\pgfqpoint{1.952131in}{0.500000in}}%
\pgfpathlineto{\pgfqpoint{1.952131in}{2.194114in}}%
\pgfpathlineto{\pgfqpoint{1.919105in}{2.194114in}}%
\pgfpathlineto{\pgfqpoint{1.919105in}{0.500000in}}%
\pgfpathclose%
\pgfusepath{fill}%
\end{pgfscope}%
\begin{pgfscope}%
\pgfpathrectangle{\pgfqpoint{0.750000in}{0.500000in}}{\pgfqpoint{4.650000in}{3.020000in}}%
\pgfusepath{clip}%
\pgfsetbuttcap%
\pgfsetmiterjoin%
\definecolor{currentfill}{rgb}{1.000000,0.000000,0.000000}%
\pgfsetfillcolor{currentfill}%
\pgfsetlinewidth{0.000000pt}%
\definecolor{currentstroke}{rgb}{0.000000,0.000000,0.000000}%
\pgfsetstrokecolor{currentstroke}%
\pgfsetstrokeopacity{0.000000}%
\pgfsetdash{}{0pt}%
\pgfpathmoveto{\pgfqpoint{1.952131in}{0.500000in}}%
\pgfpathlineto{\pgfqpoint{1.985156in}{0.500000in}}%
\pgfpathlineto{\pgfqpoint{1.985156in}{0.526333in}}%
\pgfpathlineto{\pgfqpoint{1.952131in}{0.526333in}}%
\pgfpathlineto{\pgfqpoint{1.952131in}{0.500000in}}%
\pgfpathclose%
\pgfusepath{fill}%
\end{pgfscope}%
\begin{pgfscope}%
\pgfpathrectangle{\pgfqpoint{0.750000in}{0.500000in}}{\pgfqpoint{4.650000in}{3.020000in}}%
\pgfusepath{clip}%
\pgfsetbuttcap%
\pgfsetmiterjoin%
\definecolor{currentfill}{rgb}{1.000000,0.000000,0.000000}%
\pgfsetfillcolor{currentfill}%
\pgfsetlinewidth{0.000000pt}%
\definecolor{currentstroke}{rgb}{0.000000,0.000000,0.000000}%
\pgfsetstrokecolor{currentstroke}%
\pgfsetstrokeopacity{0.000000}%
\pgfsetdash{}{0pt}%
\pgfpathmoveto{\pgfqpoint{1.985156in}{0.500000in}}%
\pgfpathlineto{\pgfqpoint{2.018182in}{0.500000in}}%
\pgfpathlineto{\pgfqpoint{2.018182in}{0.546815in}}%
\pgfpathlineto{\pgfqpoint{1.985156in}{0.546815in}}%
\pgfpathlineto{\pgfqpoint{1.985156in}{0.500000in}}%
\pgfpathclose%
\pgfusepath{fill}%
\end{pgfscope}%
\begin{pgfscope}%
\pgfpathrectangle{\pgfqpoint{0.750000in}{0.500000in}}{\pgfqpoint{4.650000in}{3.020000in}}%
\pgfusepath{clip}%
\pgfsetbuttcap%
\pgfsetmiterjoin%
\definecolor{currentfill}{rgb}{1.000000,0.000000,0.000000}%
\pgfsetfillcolor{currentfill}%
\pgfsetlinewidth{0.000000pt}%
\definecolor{currentstroke}{rgb}{0.000000,0.000000,0.000000}%
\pgfsetstrokecolor{currentstroke}%
\pgfsetstrokeopacity{0.000000}%
\pgfsetdash{}{0pt}%
\pgfpathmoveto{\pgfqpoint{2.018182in}{0.500000in}}%
\pgfpathlineto{\pgfqpoint{2.051207in}{0.500000in}}%
\pgfpathlineto{\pgfqpoint{2.051207in}{2.673967in}}%
\pgfpathlineto{\pgfqpoint{2.018182in}{2.673967in}}%
\pgfpathlineto{\pgfqpoint{2.018182in}{0.500000in}}%
\pgfpathclose%
\pgfusepath{fill}%
\end{pgfscope}%
\begin{pgfscope}%
\pgfpathrectangle{\pgfqpoint{0.750000in}{0.500000in}}{\pgfqpoint{4.650000in}{3.020000in}}%
\pgfusepath{clip}%
\pgfsetbuttcap%
\pgfsetmiterjoin%
\definecolor{currentfill}{rgb}{1.000000,0.000000,0.000000}%
\pgfsetfillcolor{currentfill}%
\pgfsetlinewidth{0.000000pt}%
\definecolor{currentstroke}{rgb}{0.000000,0.000000,0.000000}%
\pgfsetstrokecolor{currentstroke}%
\pgfsetstrokeopacity{0.000000}%
\pgfsetdash{}{0pt}%
\pgfpathmoveto{\pgfqpoint{2.051207in}{0.500000in}}%
\pgfpathlineto{\pgfqpoint{2.084233in}{0.500000in}}%
\pgfpathlineto{\pgfqpoint{2.084233in}{0.529259in}}%
\pgfpathlineto{\pgfqpoint{2.051207in}{0.529259in}}%
\pgfpathlineto{\pgfqpoint{2.051207in}{0.500000in}}%
\pgfpathclose%
\pgfusepath{fill}%
\end{pgfscope}%
\begin{pgfscope}%
\pgfpathrectangle{\pgfqpoint{0.750000in}{0.500000in}}{\pgfqpoint{4.650000in}{3.020000in}}%
\pgfusepath{clip}%
\pgfsetbuttcap%
\pgfsetmiterjoin%
\definecolor{currentfill}{rgb}{1.000000,0.000000,0.000000}%
\pgfsetfillcolor{currentfill}%
\pgfsetlinewidth{0.000000pt}%
\definecolor{currentstroke}{rgb}{0.000000,0.000000,0.000000}%
\pgfsetstrokecolor{currentstroke}%
\pgfsetstrokeopacity{0.000000}%
\pgfsetdash{}{0pt}%
\pgfpathmoveto{\pgfqpoint{2.084233in}{0.500000in}}%
\pgfpathlineto{\pgfqpoint{2.117259in}{0.500000in}}%
\pgfpathlineto{\pgfqpoint{2.117259in}{0.570222in}}%
\pgfpathlineto{\pgfqpoint{2.084233in}{0.570222in}}%
\pgfpathlineto{\pgfqpoint{2.084233in}{0.500000in}}%
\pgfpathclose%
\pgfusepath{fill}%
\end{pgfscope}%
\begin{pgfscope}%
\pgfpathrectangle{\pgfqpoint{0.750000in}{0.500000in}}{\pgfqpoint{4.650000in}{3.020000in}}%
\pgfusepath{clip}%
\pgfsetbuttcap%
\pgfsetmiterjoin%
\definecolor{currentfill}{rgb}{1.000000,0.000000,0.000000}%
\pgfsetfillcolor{currentfill}%
\pgfsetlinewidth{0.000000pt}%
\definecolor{currentstroke}{rgb}{0.000000,0.000000,0.000000}%
\pgfsetstrokecolor{currentstroke}%
\pgfsetstrokeopacity{0.000000}%
\pgfsetdash{}{0pt}%
\pgfpathmoveto{\pgfqpoint{2.117259in}{0.500000in}}%
\pgfpathlineto{\pgfqpoint{2.150284in}{0.500000in}}%
\pgfpathlineto{\pgfqpoint{2.150284in}{3.098227in}}%
\pgfpathlineto{\pgfqpoint{2.117259in}{3.098227in}}%
\pgfpathlineto{\pgfqpoint{2.117259in}{0.500000in}}%
\pgfpathclose%
\pgfusepath{fill}%
\end{pgfscope}%
\begin{pgfscope}%
\pgfpathrectangle{\pgfqpoint{0.750000in}{0.500000in}}{\pgfqpoint{4.650000in}{3.020000in}}%
\pgfusepath{clip}%
\pgfsetbuttcap%
\pgfsetmiterjoin%
\definecolor{currentfill}{rgb}{1.000000,0.000000,0.000000}%
\pgfsetfillcolor{currentfill}%
\pgfsetlinewidth{0.000000pt}%
\definecolor{currentstroke}{rgb}{0.000000,0.000000,0.000000}%
\pgfsetstrokecolor{currentstroke}%
\pgfsetstrokeopacity{0.000000}%
\pgfsetdash{}{0pt}%
\pgfpathmoveto{\pgfqpoint{2.150284in}{0.500000in}}%
\pgfpathlineto{\pgfqpoint{2.183310in}{0.500000in}}%
\pgfpathlineto{\pgfqpoint{2.183310in}{0.532185in}}%
\pgfpathlineto{\pgfqpoint{2.150284in}{0.532185in}}%
\pgfpathlineto{\pgfqpoint{2.150284in}{0.500000in}}%
\pgfpathclose%
\pgfusepath{fill}%
\end{pgfscope}%
\begin{pgfscope}%
\pgfpathrectangle{\pgfqpoint{0.750000in}{0.500000in}}{\pgfqpoint{4.650000in}{3.020000in}}%
\pgfusepath{clip}%
\pgfsetbuttcap%
\pgfsetmiterjoin%
\definecolor{currentfill}{rgb}{1.000000,0.000000,0.000000}%
\pgfsetfillcolor{currentfill}%
\pgfsetlinewidth{0.000000pt}%
\definecolor{currentstroke}{rgb}{0.000000,0.000000,0.000000}%
\pgfsetstrokecolor{currentstroke}%
\pgfsetstrokeopacity{0.000000}%
\pgfsetdash{}{0pt}%
\pgfpathmoveto{\pgfqpoint{2.183310in}{0.500000in}}%
\pgfpathlineto{\pgfqpoint{2.216335in}{0.500000in}}%
\pgfpathlineto{\pgfqpoint{2.216335in}{3.139190in}}%
\pgfpathlineto{\pgfqpoint{2.183310in}{3.139190in}}%
\pgfpathlineto{\pgfqpoint{2.183310in}{0.500000in}}%
\pgfpathclose%
\pgfusepath{fill}%
\end{pgfscope}%
\begin{pgfscope}%
\pgfpathrectangle{\pgfqpoint{0.750000in}{0.500000in}}{\pgfqpoint{4.650000in}{3.020000in}}%
\pgfusepath{clip}%
\pgfsetbuttcap%
\pgfsetmiterjoin%
\definecolor{currentfill}{rgb}{1.000000,0.000000,0.000000}%
\pgfsetfillcolor{currentfill}%
\pgfsetlinewidth{0.000000pt}%
\definecolor{currentstroke}{rgb}{0.000000,0.000000,0.000000}%
\pgfsetstrokecolor{currentstroke}%
\pgfsetstrokeopacity{0.000000}%
\pgfsetdash{}{0pt}%
\pgfpathmoveto{\pgfqpoint{2.216335in}{0.500000in}}%
\pgfpathlineto{\pgfqpoint{2.249361in}{0.500000in}}%
\pgfpathlineto{\pgfqpoint{2.249361in}{0.570222in}}%
\pgfpathlineto{\pgfqpoint{2.216335in}{0.570222in}}%
\pgfpathlineto{\pgfqpoint{2.216335in}{0.500000in}}%
\pgfpathclose%
\pgfusepath{fill}%
\end{pgfscope}%
\begin{pgfscope}%
\pgfpathrectangle{\pgfqpoint{0.750000in}{0.500000in}}{\pgfqpoint{4.650000in}{3.020000in}}%
\pgfusepath{clip}%
\pgfsetbuttcap%
\pgfsetmiterjoin%
\definecolor{currentfill}{rgb}{1.000000,0.000000,0.000000}%
\pgfsetfillcolor{currentfill}%
\pgfsetlinewidth{0.000000pt}%
\definecolor{currentstroke}{rgb}{0.000000,0.000000,0.000000}%
\pgfsetstrokecolor{currentstroke}%
\pgfsetstrokeopacity{0.000000}%
\pgfsetdash{}{0pt}%
\pgfpathmoveto{\pgfqpoint{2.249361in}{0.500000in}}%
\pgfpathlineto{\pgfqpoint{2.282386in}{0.500000in}}%
\pgfpathlineto{\pgfqpoint{2.282386in}{0.546815in}}%
\pgfpathlineto{\pgfqpoint{2.249361in}{0.546815in}}%
\pgfpathlineto{\pgfqpoint{2.249361in}{0.500000in}}%
\pgfpathclose%
\pgfusepath{fill}%
\end{pgfscope}%
\begin{pgfscope}%
\pgfpathrectangle{\pgfqpoint{0.750000in}{0.500000in}}{\pgfqpoint{4.650000in}{3.020000in}}%
\pgfusepath{clip}%
\pgfsetbuttcap%
\pgfsetmiterjoin%
\definecolor{currentfill}{rgb}{1.000000,0.000000,0.000000}%
\pgfsetfillcolor{currentfill}%
\pgfsetlinewidth{0.000000pt}%
\definecolor{currentstroke}{rgb}{0.000000,0.000000,0.000000}%
\pgfsetstrokecolor{currentstroke}%
\pgfsetstrokeopacity{0.000000}%
\pgfsetdash{}{0pt}%
\pgfpathmoveto{\pgfqpoint{2.282386in}{0.500000in}}%
\pgfpathlineto{\pgfqpoint{2.315412in}{0.500000in}}%
\pgfpathlineto{\pgfqpoint{2.315412in}{3.376190in}}%
\pgfpathlineto{\pgfqpoint{2.282386in}{3.376190in}}%
\pgfpathlineto{\pgfqpoint{2.282386in}{0.500000in}}%
\pgfpathclose%
\pgfusepath{fill}%
\end{pgfscope}%
\begin{pgfscope}%
\pgfpathrectangle{\pgfqpoint{0.750000in}{0.500000in}}{\pgfqpoint{4.650000in}{3.020000in}}%
\pgfusepath{clip}%
\pgfsetbuttcap%
\pgfsetmiterjoin%
\definecolor{currentfill}{rgb}{1.000000,0.000000,0.000000}%
\pgfsetfillcolor{currentfill}%
\pgfsetlinewidth{0.000000pt}%
\definecolor{currentstroke}{rgb}{0.000000,0.000000,0.000000}%
\pgfsetstrokecolor{currentstroke}%
\pgfsetstrokeopacity{0.000000}%
\pgfsetdash{}{0pt}%
\pgfpathmoveto{\pgfqpoint{2.315412in}{0.500000in}}%
\pgfpathlineto{\pgfqpoint{2.348437in}{0.500000in}}%
\pgfpathlineto{\pgfqpoint{2.348437in}{0.549741in}}%
\pgfpathlineto{\pgfqpoint{2.315412in}{0.549741in}}%
\pgfpathlineto{\pgfqpoint{2.315412in}{0.500000in}}%
\pgfpathclose%
\pgfusepath{fill}%
\end{pgfscope}%
\begin{pgfscope}%
\pgfpathrectangle{\pgfqpoint{0.750000in}{0.500000in}}{\pgfqpoint{4.650000in}{3.020000in}}%
\pgfusepath{clip}%
\pgfsetbuttcap%
\pgfsetmiterjoin%
\definecolor{currentfill}{rgb}{1.000000,0.000000,0.000000}%
\pgfsetfillcolor{currentfill}%
\pgfsetlinewidth{0.000000pt}%
\definecolor{currentstroke}{rgb}{0.000000,0.000000,0.000000}%
\pgfsetstrokecolor{currentstroke}%
\pgfsetstrokeopacity{0.000000}%
\pgfsetdash{}{0pt}%
\pgfpathmoveto{\pgfqpoint{2.348438in}{0.500000in}}%
\pgfpathlineto{\pgfqpoint{2.381463in}{0.500000in}}%
\pgfpathlineto{\pgfqpoint{2.381463in}{0.552667in}}%
\pgfpathlineto{\pgfqpoint{2.348438in}{0.552667in}}%
\pgfpathlineto{\pgfqpoint{2.348438in}{0.500000in}}%
\pgfpathclose%
\pgfusepath{fill}%
\end{pgfscope}%
\begin{pgfscope}%
\pgfpathrectangle{\pgfqpoint{0.750000in}{0.500000in}}{\pgfqpoint{4.650000in}{3.020000in}}%
\pgfusepath{clip}%
\pgfsetbuttcap%
\pgfsetmiterjoin%
\definecolor{currentfill}{rgb}{1.000000,0.000000,0.000000}%
\pgfsetfillcolor{currentfill}%
\pgfsetlinewidth{0.000000pt}%
\definecolor{currentstroke}{rgb}{0.000000,0.000000,0.000000}%
\pgfsetstrokecolor{currentstroke}%
\pgfsetstrokeopacity{0.000000}%
\pgfsetdash{}{0pt}%
\pgfpathmoveto{\pgfqpoint{2.381463in}{0.500000in}}%
\pgfpathlineto{\pgfqpoint{2.414489in}{0.500000in}}%
\pgfpathlineto{\pgfqpoint{2.414489in}{3.174301in}}%
\pgfpathlineto{\pgfqpoint{2.381463in}{3.174301in}}%
\pgfpathlineto{\pgfqpoint{2.381463in}{0.500000in}}%
\pgfpathclose%
\pgfusepath{fill}%
\end{pgfscope}%
\begin{pgfscope}%
\pgfpathrectangle{\pgfqpoint{0.750000in}{0.500000in}}{\pgfqpoint{4.650000in}{3.020000in}}%
\pgfusepath{clip}%
\pgfsetbuttcap%
\pgfsetmiterjoin%
\definecolor{currentfill}{rgb}{1.000000,0.000000,0.000000}%
\pgfsetfillcolor{currentfill}%
\pgfsetlinewidth{0.000000pt}%
\definecolor{currentstroke}{rgb}{0.000000,0.000000,0.000000}%
\pgfsetstrokecolor{currentstroke}%
\pgfsetstrokeopacity{0.000000}%
\pgfsetdash{}{0pt}%
\pgfpathmoveto{\pgfqpoint{2.414489in}{0.500000in}}%
\pgfpathlineto{\pgfqpoint{2.447514in}{0.500000in}}%
\pgfpathlineto{\pgfqpoint{2.447514in}{0.555593in}}%
\pgfpathlineto{\pgfqpoint{2.414489in}{0.555593in}}%
\pgfpathlineto{\pgfqpoint{2.414489in}{0.500000in}}%
\pgfpathclose%
\pgfusepath{fill}%
\end{pgfscope}%
\begin{pgfscope}%
\pgfpathrectangle{\pgfqpoint{0.750000in}{0.500000in}}{\pgfqpoint{4.650000in}{3.020000in}}%
\pgfusepath{clip}%
\pgfsetbuttcap%
\pgfsetmiterjoin%
\definecolor{currentfill}{rgb}{1.000000,0.000000,0.000000}%
\pgfsetfillcolor{currentfill}%
\pgfsetlinewidth{0.000000pt}%
\definecolor{currentstroke}{rgb}{0.000000,0.000000,0.000000}%
\pgfsetstrokecolor{currentstroke}%
\pgfsetstrokeopacity{0.000000}%
\pgfsetdash{}{0pt}%
\pgfpathmoveto{\pgfqpoint{2.447514in}{0.500000in}}%
\pgfpathlineto{\pgfqpoint{2.480540in}{0.500000in}}%
\pgfpathlineto{\pgfqpoint{2.480540in}{2.896338in}}%
\pgfpathlineto{\pgfqpoint{2.447514in}{2.896338in}}%
\pgfpathlineto{\pgfqpoint{2.447514in}{0.500000in}}%
\pgfpathclose%
\pgfusepath{fill}%
\end{pgfscope}%
\begin{pgfscope}%
\pgfpathrectangle{\pgfqpoint{0.750000in}{0.500000in}}{\pgfqpoint{4.650000in}{3.020000in}}%
\pgfusepath{clip}%
\pgfsetbuttcap%
\pgfsetmiterjoin%
\definecolor{currentfill}{rgb}{1.000000,0.000000,0.000000}%
\pgfsetfillcolor{currentfill}%
\pgfsetlinewidth{0.000000pt}%
\definecolor{currentstroke}{rgb}{0.000000,0.000000,0.000000}%
\pgfsetstrokecolor{currentstroke}%
\pgfsetstrokeopacity{0.000000}%
\pgfsetdash{}{0pt}%
\pgfpathmoveto{\pgfqpoint{2.480540in}{0.500000in}}%
\pgfpathlineto{\pgfqpoint{2.513565in}{0.500000in}}%
\pgfpathlineto{\pgfqpoint{2.513565in}{0.570222in}}%
\pgfpathlineto{\pgfqpoint{2.480540in}{0.570222in}}%
\pgfpathlineto{\pgfqpoint{2.480540in}{0.500000in}}%
\pgfpathclose%
\pgfusepath{fill}%
\end{pgfscope}%
\begin{pgfscope}%
\pgfpathrectangle{\pgfqpoint{0.750000in}{0.500000in}}{\pgfqpoint{4.650000in}{3.020000in}}%
\pgfusepath{clip}%
\pgfsetbuttcap%
\pgfsetmiterjoin%
\definecolor{currentfill}{rgb}{1.000000,0.000000,0.000000}%
\pgfsetfillcolor{currentfill}%
\pgfsetlinewidth{0.000000pt}%
\definecolor{currentstroke}{rgb}{0.000000,0.000000,0.000000}%
\pgfsetstrokecolor{currentstroke}%
\pgfsetstrokeopacity{0.000000}%
\pgfsetdash{}{0pt}%
\pgfpathmoveto{\pgfqpoint{2.513565in}{0.500000in}}%
\pgfpathlineto{\pgfqpoint{2.546591in}{0.500000in}}%
\pgfpathlineto{\pgfqpoint{2.546591in}{0.552667in}}%
\pgfpathlineto{\pgfqpoint{2.513565in}{0.552667in}}%
\pgfpathlineto{\pgfqpoint{2.513565in}{0.500000in}}%
\pgfpathclose%
\pgfusepath{fill}%
\end{pgfscope}%
\begin{pgfscope}%
\pgfpathrectangle{\pgfqpoint{0.750000in}{0.500000in}}{\pgfqpoint{4.650000in}{3.020000in}}%
\pgfusepath{clip}%
\pgfsetbuttcap%
\pgfsetmiterjoin%
\definecolor{currentfill}{rgb}{1.000000,0.000000,0.000000}%
\pgfsetfillcolor{currentfill}%
\pgfsetlinewidth{0.000000pt}%
\definecolor{currentstroke}{rgb}{0.000000,0.000000,0.000000}%
\pgfsetstrokecolor{currentstroke}%
\pgfsetstrokeopacity{0.000000}%
\pgfsetdash{}{0pt}%
\pgfpathmoveto{\pgfqpoint{2.546591in}{0.500000in}}%
\pgfpathlineto{\pgfqpoint{2.579616in}{0.500000in}}%
\pgfpathlineto{\pgfqpoint{2.579616in}{2.621300in}}%
\pgfpathlineto{\pgfqpoint{2.546591in}{2.621300in}}%
\pgfpathlineto{\pgfqpoint{2.546591in}{0.500000in}}%
\pgfpathclose%
\pgfusepath{fill}%
\end{pgfscope}%
\begin{pgfscope}%
\pgfpathrectangle{\pgfqpoint{0.750000in}{0.500000in}}{\pgfqpoint{4.650000in}{3.020000in}}%
\pgfusepath{clip}%
\pgfsetbuttcap%
\pgfsetmiterjoin%
\definecolor{currentfill}{rgb}{1.000000,0.000000,0.000000}%
\pgfsetfillcolor{currentfill}%
\pgfsetlinewidth{0.000000pt}%
\definecolor{currentstroke}{rgb}{0.000000,0.000000,0.000000}%
\pgfsetstrokecolor{currentstroke}%
\pgfsetstrokeopacity{0.000000}%
\pgfsetdash{}{0pt}%
\pgfpathmoveto{\pgfqpoint{2.579616in}{0.500000in}}%
\pgfpathlineto{\pgfqpoint{2.612642in}{0.500000in}}%
\pgfpathlineto{\pgfqpoint{2.612642in}{0.529259in}}%
\pgfpathlineto{\pgfqpoint{2.579616in}{0.529259in}}%
\pgfpathlineto{\pgfqpoint{2.579616in}{0.500000in}}%
\pgfpathclose%
\pgfusepath{fill}%
\end{pgfscope}%
\begin{pgfscope}%
\pgfpathrectangle{\pgfqpoint{0.750000in}{0.500000in}}{\pgfqpoint{4.650000in}{3.020000in}}%
\pgfusepath{clip}%
\pgfsetbuttcap%
\pgfsetmiterjoin%
\definecolor{currentfill}{rgb}{1.000000,0.000000,0.000000}%
\pgfsetfillcolor{currentfill}%
\pgfsetlinewidth{0.000000pt}%
\definecolor{currentstroke}{rgb}{0.000000,0.000000,0.000000}%
\pgfsetstrokecolor{currentstroke}%
\pgfsetstrokeopacity{0.000000}%
\pgfsetdash{}{0pt}%
\pgfpathmoveto{\pgfqpoint{2.612642in}{0.500000in}}%
\pgfpathlineto{\pgfqpoint{2.645668in}{0.500000in}}%
\pgfpathlineto{\pgfqpoint{2.645668in}{0.552667in}}%
\pgfpathlineto{\pgfqpoint{2.612642in}{0.552667in}}%
\pgfpathlineto{\pgfqpoint{2.612642in}{0.500000in}}%
\pgfpathclose%
\pgfusepath{fill}%
\end{pgfscope}%
\begin{pgfscope}%
\pgfpathrectangle{\pgfqpoint{0.750000in}{0.500000in}}{\pgfqpoint{4.650000in}{3.020000in}}%
\pgfusepath{clip}%
\pgfsetbuttcap%
\pgfsetmiterjoin%
\definecolor{currentfill}{rgb}{1.000000,0.000000,0.000000}%
\pgfsetfillcolor{currentfill}%
\pgfsetlinewidth{0.000000pt}%
\definecolor{currentstroke}{rgb}{0.000000,0.000000,0.000000}%
\pgfsetstrokecolor{currentstroke}%
\pgfsetstrokeopacity{0.000000}%
\pgfsetdash{}{0pt}%
\pgfpathmoveto{\pgfqpoint{2.645668in}{0.500000in}}%
\pgfpathlineto{\pgfqpoint{2.678693in}{0.500000in}}%
\pgfpathlineto{\pgfqpoint{2.678693in}{2.246781in}}%
\pgfpathlineto{\pgfqpoint{2.645668in}{2.246781in}}%
\pgfpathlineto{\pgfqpoint{2.645668in}{0.500000in}}%
\pgfpathclose%
\pgfusepath{fill}%
\end{pgfscope}%
\begin{pgfscope}%
\pgfpathrectangle{\pgfqpoint{0.750000in}{0.500000in}}{\pgfqpoint{4.650000in}{3.020000in}}%
\pgfusepath{clip}%
\pgfsetbuttcap%
\pgfsetmiterjoin%
\definecolor{currentfill}{rgb}{1.000000,0.000000,0.000000}%
\pgfsetfillcolor{currentfill}%
\pgfsetlinewidth{0.000000pt}%
\definecolor{currentstroke}{rgb}{0.000000,0.000000,0.000000}%
\pgfsetstrokecolor{currentstroke}%
\pgfsetstrokeopacity{0.000000}%
\pgfsetdash{}{0pt}%
\pgfpathmoveto{\pgfqpoint{2.678693in}{0.500000in}}%
\pgfpathlineto{\pgfqpoint{2.711719in}{0.500000in}}%
\pgfpathlineto{\pgfqpoint{2.711719in}{0.535111in}}%
\pgfpathlineto{\pgfqpoint{2.678693in}{0.535111in}}%
\pgfpathlineto{\pgfqpoint{2.678693in}{0.500000in}}%
\pgfpathclose%
\pgfusepath{fill}%
\end{pgfscope}%
\begin{pgfscope}%
\pgfpathrectangle{\pgfqpoint{0.750000in}{0.500000in}}{\pgfqpoint{4.650000in}{3.020000in}}%
\pgfusepath{clip}%
\pgfsetbuttcap%
\pgfsetmiterjoin%
\definecolor{currentfill}{rgb}{1.000000,0.000000,0.000000}%
\pgfsetfillcolor{currentfill}%
\pgfsetlinewidth{0.000000pt}%
\definecolor{currentstroke}{rgb}{0.000000,0.000000,0.000000}%
\pgfsetstrokecolor{currentstroke}%
\pgfsetstrokeopacity{0.000000}%
\pgfsetdash{}{0pt}%
\pgfpathmoveto{\pgfqpoint{2.711719in}{0.500000in}}%
\pgfpathlineto{\pgfqpoint{2.744744in}{0.500000in}}%
\pgfpathlineto{\pgfqpoint{2.744744in}{0.643371in}}%
\pgfpathlineto{\pgfqpoint{2.711719in}{0.643371in}}%
\pgfpathlineto{\pgfqpoint{2.711719in}{0.500000in}}%
\pgfpathclose%
\pgfusepath{fill}%
\end{pgfscope}%
\begin{pgfscope}%
\pgfpathrectangle{\pgfqpoint{0.750000in}{0.500000in}}{\pgfqpoint{4.650000in}{3.020000in}}%
\pgfusepath{clip}%
\pgfsetbuttcap%
\pgfsetmiterjoin%
\definecolor{currentfill}{rgb}{1.000000,0.000000,0.000000}%
\pgfsetfillcolor{currentfill}%
\pgfsetlinewidth{0.000000pt}%
\definecolor{currentstroke}{rgb}{0.000000,0.000000,0.000000}%
\pgfsetstrokecolor{currentstroke}%
\pgfsetstrokeopacity{0.000000}%
\pgfsetdash{}{0pt}%
\pgfpathmoveto{\pgfqpoint{2.744744in}{0.500000in}}%
\pgfpathlineto{\pgfqpoint{2.777770in}{0.500000in}}%
\pgfpathlineto{\pgfqpoint{2.777770in}{1.611854in}}%
\pgfpathlineto{\pgfqpoint{2.744744in}{1.611854in}}%
\pgfpathlineto{\pgfqpoint{2.744744in}{0.500000in}}%
\pgfpathclose%
\pgfusepath{fill}%
\end{pgfscope}%
\begin{pgfscope}%
\pgfpathrectangle{\pgfqpoint{0.750000in}{0.500000in}}{\pgfqpoint{4.650000in}{3.020000in}}%
\pgfusepath{clip}%
\pgfsetbuttcap%
\pgfsetmiterjoin%
\definecolor{currentfill}{rgb}{1.000000,0.000000,0.000000}%
\pgfsetfillcolor{currentfill}%
\pgfsetlinewidth{0.000000pt}%
\definecolor{currentstroke}{rgb}{0.000000,0.000000,0.000000}%
\pgfsetstrokecolor{currentstroke}%
\pgfsetstrokeopacity{0.000000}%
\pgfsetdash{}{0pt}%
\pgfpathmoveto{\pgfqpoint{2.777770in}{0.500000in}}%
\pgfpathlineto{\pgfqpoint{2.810795in}{0.500000in}}%
\pgfpathlineto{\pgfqpoint{2.810795in}{0.532185in}}%
\pgfpathlineto{\pgfqpoint{2.777770in}{0.532185in}}%
\pgfpathlineto{\pgfqpoint{2.777770in}{0.500000in}}%
\pgfpathclose%
\pgfusepath{fill}%
\end{pgfscope}%
\begin{pgfscope}%
\pgfpathrectangle{\pgfqpoint{0.750000in}{0.500000in}}{\pgfqpoint{4.650000in}{3.020000in}}%
\pgfusepath{clip}%
\pgfsetbuttcap%
\pgfsetmiterjoin%
\definecolor{currentfill}{rgb}{1.000000,0.000000,0.000000}%
\pgfsetfillcolor{currentfill}%
\pgfsetlinewidth{0.000000pt}%
\definecolor{currentstroke}{rgb}{0.000000,0.000000,0.000000}%
\pgfsetstrokecolor{currentstroke}%
\pgfsetstrokeopacity{0.000000}%
\pgfsetdash{}{0pt}%
\pgfpathmoveto{\pgfqpoint{2.810795in}{0.500000in}}%
\pgfpathlineto{\pgfqpoint{2.843821in}{0.500000in}}%
\pgfpathlineto{\pgfqpoint{2.843821in}{1.407039in}}%
\pgfpathlineto{\pgfqpoint{2.810795in}{1.407039in}}%
\pgfpathlineto{\pgfqpoint{2.810795in}{0.500000in}}%
\pgfpathclose%
\pgfusepath{fill}%
\end{pgfscope}%
\begin{pgfscope}%
\pgfpathrectangle{\pgfqpoint{0.750000in}{0.500000in}}{\pgfqpoint{4.650000in}{3.020000in}}%
\pgfusepath{clip}%
\pgfsetbuttcap%
\pgfsetmiterjoin%
\definecolor{currentfill}{rgb}{1.000000,0.000000,0.000000}%
\pgfsetfillcolor{currentfill}%
\pgfsetlinewidth{0.000000pt}%
\definecolor{currentstroke}{rgb}{0.000000,0.000000,0.000000}%
\pgfsetstrokecolor{currentstroke}%
\pgfsetstrokeopacity{0.000000}%
\pgfsetdash{}{0pt}%
\pgfpathmoveto{\pgfqpoint{2.843821in}{0.500000in}}%
\pgfpathlineto{\pgfqpoint{2.876847in}{0.500000in}}%
\pgfpathlineto{\pgfqpoint{2.876847in}{0.520482in}}%
\pgfpathlineto{\pgfqpoint{2.843821in}{0.520482in}}%
\pgfpathlineto{\pgfqpoint{2.843821in}{0.500000in}}%
\pgfpathclose%
\pgfusepath{fill}%
\end{pgfscope}%
\begin{pgfscope}%
\pgfpathrectangle{\pgfqpoint{0.750000in}{0.500000in}}{\pgfqpoint{4.650000in}{3.020000in}}%
\pgfusepath{clip}%
\pgfsetbuttcap%
\pgfsetmiterjoin%
\definecolor{currentfill}{rgb}{1.000000,0.000000,0.000000}%
\pgfsetfillcolor{currentfill}%
\pgfsetlinewidth{0.000000pt}%
\definecolor{currentstroke}{rgb}{0.000000,0.000000,0.000000}%
\pgfsetstrokecolor{currentstroke}%
\pgfsetstrokeopacity{0.000000}%
\pgfsetdash{}{0pt}%
\pgfpathmoveto{\pgfqpoint{2.876847in}{0.500000in}}%
\pgfpathlineto{\pgfqpoint{2.909872in}{0.500000in}}%
\pgfpathlineto{\pgfqpoint{2.909872in}{0.508778in}}%
\pgfpathlineto{\pgfqpoint{2.876847in}{0.508778in}}%
\pgfpathlineto{\pgfqpoint{2.876847in}{0.500000in}}%
\pgfpathclose%
\pgfusepath{fill}%
\end{pgfscope}%
\begin{pgfscope}%
\pgfpathrectangle{\pgfqpoint{0.750000in}{0.500000in}}{\pgfqpoint{4.650000in}{3.020000in}}%
\pgfusepath{clip}%
\pgfsetbuttcap%
\pgfsetmiterjoin%
\definecolor{currentfill}{rgb}{1.000000,0.000000,0.000000}%
\pgfsetfillcolor{currentfill}%
\pgfsetlinewidth{0.000000pt}%
\definecolor{currentstroke}{rgb}{0.000000,0.000000,0.000000}%
\pgfsetstrokecolor{currentstroke}%
\pgfsetstrokeopacity{0.000000}%
\pgfsetdash{}{0pt}%
\pgfpathmoveto{\pgfqpoint{2.909872in}{0.500000in}}%
\pgfpathlineto{\pgfqpoint{2.942898in}{0.500000in}}%
\pgfpathlineto{\pgfqpoint{2.942898in}{1.079334in}}%
\pgfpathlineto{\pgfqpoint{2.909872in}{1.079334in}}%
\pgfpathlineto{\pgfqpoint{2.909872in}{0.500000in}}%
\pgfpathclose%
\pgfusepath{fill}%
\end{pgfscope}%
\begin{pgfscope}%
\pgfpathrectangle{\pgfqpoint{0.750000in}{0.500000in}}{\pgfqpoint{4.650000in}{3.020000in}}%
\pgfusepath{clip}%
\pgfsetbuttcap%
\pgfsetmiterjoin%
\definecolor{currentfill}{rgb}{1.000000,0.000000,0.000000}%
\pgfsetfillcolor{currentfill}%
\pgfsetlinewidth{0.000000pt}%
\definecolor{currentstroke}{rgb}{0.000000,0.000000,0.000000}%
\pgfsetstrokecolor{currentstroke}%
\pgfsetstrokeopacity{0.000000}%
\pgfsetdash{}{0pt}%
\pgfpathmoveto{\pgfqpoint{2.942898in}{0.500000in}}%
\pgfpathlineto{\pgfqpoint{2.975923in}{0.500000in}}%
\pgfpathlineto{\pgfqpoint{2.975923in}{0.502926in}}%
\pgfpathlineto{\pgfqpoint{2.942898in}{0.502926in}}%
\pgfpathlineto{\pgfqpoint{2.942898in}{0.500000in}}%
\pgfpathclose%
\pgfusepath{fill}%
\end{pgfscope}%
\begin{pgfscope}%
\pgfpathrectangle{\pgfqpoint{0.750000in}{0.500000in}}{\pgfqpoint{4.650000in}{3.020000in}}%
\pgfusepath{clip}%
\pgfsetbuttcap%
\pgfsetmiterjoin%
\definecolor{currentfill}{rgb}{1.000000,0.000000,0.000000}%
\pgfsetfillcolor{currentfill}%
\pgfsetlinewidth{0.000000pt}%
\definecolor{currentstroke}{rgb}{0.000000,0.000000,0.000000}%
\pgfsetstrokecolor{currentstroke}%
\pgfsetstrokeopacity{0.000000}%
\pgfsetdash{}{0pt}%
\pgfpathmoveto{\pgfqpoint{2.975923in}{0.500000in}}%
\pgfpathlineto{\pgfqpoint{3.008949in}{0.500000in}}%
\pgfpathlineto{\pgfqpoint{3.008949in}{0.517556in}}%
\pgfpathlineto{\pgfqpoint{2.975923in}{0.517556in}}%
\pgfpathlineto{\pgfqpoint{2.975923in}{0.500000in}}%
\pgfpathclose%
\pgfusepath{fill}%
\end{pgfscope}%
\begin{pgfscope}%
\pgfpathrectangle{\pgfqpoint{0.750000in}{0.500000in}}{\pgfqpoint{4.650000in}{3.020000in}}%
\pgfusepath{clip}%
\pgfsetbuttcap%
\pgfsetmiterjoin%
\definecolor{currentfill}{rgb}{1.000000,0.000000,0.000000}%
\pgfsetfillcolor{currentfill}%
\pgfsetlinewidth{0.000000pt}%
\definecolor{currentstroke}{rgb}{0.000000,0.000000,0.000000}%
\pgfsetstrokecolor{currentstroke}%
\pgfsetstrokeopacity{0.000000}%
\pgfsetdash{}{0pt}%
\pgfpathmoveto{\pgfqpoint{3.008949in}{0.500000in}}%
\pgfpathlineto{\pgfqpoint{3.041974in}{0.500000in}}%
\pgfpathlineto{\pgfqpoint{3.041974in}{0.880371in}}%
\pgfpathlineto{\pgfqpoint{3.008949in}{0.880371in}}%
\pgfpathlineto{\pgfqpoint{3.008949in}{0.500000in}}%
\pgfpathclose%
\pgfusepath{fill}%
\end{pgfscope}%
\begin{pgfscope}%
\pgfpathrectangle{\pgfqpoint{0.750000in}{0.500000in}}{\pgfqpoint{4.650000in}{3.020000in}}%
\pgfusepath{clip}%
\pgfsetbuttcap%
\pgfsetmiterjoin%
\definecolor{currentfill}{rgb}{1.000000,0.000000,0.000000}%
\pgfsetfillcolor{currentfill}%
\pgfsetlinewidth{0.000000pt}%
\definecolor{currentstroke}{rgb}{0.000000,0.000000,0.000000}%
\pgfsetstrokecolor{currentstroke}%
\pgfsetstrokeopacity{0.000000}%
\pgfsetdash{}{0pt}%
\pgfpathmoveto{\pgfqpoint{3.041974in}{0.500000in}}%
\pgfpathlineto{\pgfqpoint{3.075000in}{0.500000in}}%
\pgfpathlineto{\pgfqpoint{3.075000in}{0.502926in}}%
\pgfpathlineto{\pgfqpoint{3.041974in}{0.502926in}}%
\pgfpathlineto{\pgfqpoint{3.041974in}{0.500000in}}%
\pgfpathclose%
\pgfusepath{fill}%
\end{pgfscope}%
\begin{pgfscope}%
\pgfpathrectangle{\pgfqpoint{0.750000in}{0.500000in}}{\pgfqpoint{4.650000in}{3.020000in}}%
\pgfusepath{clip}%
\pgfsetbuttcap%
\pgfsetmiterjoin%
\definecolor{currentfill}{rgb}{1.000000,0.000000,0.000000}%
\pgfsetfillcolor{currentfill}%
\pgfsetlinewidth{0.000000pt}%
\definecolor{currentstroke}{rgb}{0.000000,0.000000,0.000000}%
\pgfsetstrokecolor{currentstroke}%
\pgfsetstrokeopacity{0.000000}%
\pgfsetdash{}{0pt}%
\pgfpathmoveto{\pgfqpoint{3.075000in}{0.500000in}}%
\pgfpathlineto{\pgfqpoint{3.108026in}{0.500000in}}%
\pgfpathlineto{\pgfqpoint{3.108026in}{0.734075in}}%
\pgfpathlineto{\pgfqpoint{3.075000in}{0.734075in}}%
\pgfpathlineto{\pgfqpoint{3.075000in}{0.500000in}}%
\pgfpathclose%
\pgfusepath{fill}%
\end{pgfscope}%
\begin{pgfscope}%
\pgfpathrectangle{\pgfqpoint{0.750000in}{0.500000in}}{\pgfqpoint{4.650000in}{3.020000in}}%
\pgfusepath{clip}%
\pgfsetbuttcap%
\pgfsetmiterjoin%
\definecolor{currentfill}{rgb}{1.000000,0.000000,0.000000}%
\pgfsetfillcolor{currentfill}%
\pgfsetlinewidth{0.000000pt}%
\definecolor{currentstroke}{rgb}{0.000000,0.000000,0.000000}%
\pgfsetstrokecolor{currentstroke}%
\pgfsetstrokeopacity{0.000000}%
\pgfsetdash{}{0pt}%
\pgfpathmoveto{\pgfqpoint{3.108026in}{0.500000in}}%
\pgfpathlineto{\pgfqpoint{3.141051in}{0.500000in}}%
\pgfpathlineto{\pgfqpoint{3.141051in}{0.511704in}}%
\pgfpathlineto{\pgfqpoint{3.108026in}{0.511704in}}%
\pgfpathlineto{\pgfqpoint{3.108026in}{0.500000in}}%
\pgfpathclose%
\pgfusepath{fill}%
\end{pgfscope}%
\begin{pgfscope}%
\pgfpathrectangle{\pgfqpoint{0.750000in}{0.500000in}}{\pgfqpoint{4.650000in}{3.020000in}}%
\pgfusepath{clip}%
\pgfsetbuttcap%
\pgfsetmiterjoin%
\definecolor{currentfill}{rgb}{1.000000,0.000000,0.000000}%
\pgfsetfillcolor{currentfill}%
\pgfsetlinewidth{0.000000pt}%
\definecolor{currentstroke}{rgb}{0.000000,0.000000,0.000000}%
\pgfsetstrokecolor{currentstroke}%
\pgfsetstrokeopacity{0.000000}%
\pgfsetdash{}{0pt}%
\pgfpathmoveto{\pgfqpoint{3.141051in}{0.500000in}}%
\pgfpathlineto{\pgfqpoint{3.174077in}{0.500000in}}%
\pgfpathlineto{\pgfqpoint{3.174077in}{0.505852in}}%
\pgfpathlineto{\pgfqpoint{3.141051in}{0.505852in}}%
\pgfpathlineto{\pgfqpoint{3.141051in}{0.500000in}}%
\pgfpathclose%
\pgfusepath{fill}%
\end{pgfscope}%
\begin{pgfscope}%
\pgfpathrectangle{\pgfqpoint{0.750000in}{0.500000in}}{\pgfqpoint{4.650000in}{3.020000in}}%
\pgfusepath{clip}%
\pgfsetbuttcap%
\pgfsetmiterjoin%
\definecolor{currentfill}{rgb}{1.000000,0.000000,0.000000}%
\pgfsetfillcolor{currentfill}%
\pgfsetlinewidth{0.000000pt}%
\definecolor{currentstroke}{rgb}{0.000000,0.000000,0.000000}%
\pgfsetstrokecolor{currentstroke}%
\pgfsetstrokeopacity{0.000000}%
\pgfsetdash{}{0pt}%
\pgfpathmoveto{\pgfqpoint{3.174077in}{0.500000in}}%
\pgfpathlineto{\pgfqpoint{3.207102in}{0.500000in}}%
\pgfpathlineto{\pgfqpoint{3.207102in}{0.593630in}}%
\pgfpathlineto{\pgfqpoint{3.174077in}{0.593630in}}%
\pgfpathlineto{\pgfqpoint{3.174077in}{0.500000in}}%
\pgfpathclose%
\pgfusepath{fill}%
\end{pgfscope}%
\begin{pgfscope}%
\pgfpathrectangle{\pgfqpoint{0.750000in}{0.500000in}}{\pgfqpoint{4.650000in}{3.020000in}}%
\pgfusepath{clip}%
\pgfsetbuttcap%
\pgfsetmiterjoin%
\definecolor{currentfill}{rgb}{1.000000,0.000000,0.000000}%
\pgfsetfillcolor{currentfill}%
\pgfsetlinewidth{0.000000pt}%
\definecolor{currentstroke}{rgb}{0.000000,0.000000,0.000000}%
\pgfsetstrokecolor{currentstroke}%
\pgfsetstrokeopacity{0.000000}%
\pgfsetdash{}{0pt}%
\pgfpathmoveto{\pgfqpoint{3.207102in}{0.500000in}}%
\pgfpathlineto{\pgfqpoint{3.240128in}{0.500000in}}%
\pgfpathlineto{\pgfqpoint{3.240128in}{0.502926in}}%
\pgfpathlineto{\pgfqpoint{3.207102in}{0.502926in}}%
\pgfpathlineto{\pgfqpoint{3.207102in}{0.500000in}}%
\pgfpathclose%
\pgfusepath{fill}%
\end{pgfscope}%
\begin{pgfscope}%
\pgfpathrectangle{\pgfqpoint{0.750000in}{0.500000in}}{\pgfqpoint{4.650000in}{3.020000in}}%
\pgfusepath{clip}%
\pgfsetbuttcap%
\pgfsetmiterjoin%
\definecolor{currentfill}{rgb}{1.000000,0.000000,0.000000}%
\pgfsetfillcolor{currentfill}%
\pgfsetlinewidth{0.000000pt}%
\definecolor{currentstroke}{rgb}{0.000000,0.000000,0.000000}%
\pgfsetstrokecolor{currentstroke}%
\pgfsetstrokeopacity{0.000000}%
\pgfsetdash{}{0pt}%
\pgfpathmoveto{\pgfqpoint{3.240128in}{0.500000in}}%
\pgfpathlineto{\pgfqpoint{3.273153in}{0.500000in}}%
\pgfpathlineto{\pgfqpoint{3.273153in}{0.508778in}}%
\pgfpathlineto{\pgfqpoint{3.240128in}{0.508778in}}%
\pgfpathlineto{\pgfqpoint{3.240128in}{0.500000in}}%
\pgfpathclose%
\pgfusepath{fill}%
\end{pgfscope}%
\begin{pgfscope}%
\pgfpathrectangle{\pgfqpoint{0.750000in}{0.500000in}}{\pgfqpoint{4.650000in}{3.020000in}}%
\pgfusepath{clip}%
\pgfsetbuttcap%
\pgfsetmiterjoin%
\definecolor{currentfill}{rgb}{1.000000,0.000000,0.000000}%
\pgfsetfillcolor{currentfill}%
\pgfsetlinewidth{0.000000pt}%
\definecolor{currentstroke}{rgb}{0.000000,0.000000,0.000000}%
\pgfsetstrokecolor{currentstroke}%
\pgfsetstrokeopacity{0.000000}%
\pgfsetdash{}{0pt}%
\pgfpathmoveto{\pgfqpoint{3.273153in}{0.500000in}}%
\pgfpathlineto{\pgfqpoint{3.306179in}{0.500000in}}%
\pgfpathlineto{\pgfqpoint{3.306179in}{0.576074in}}%
\pgfpathlineto{\pgfqpoint{3.273153in}{0.576074in}}%
\pgfpathlineto{\pgfqpoint{3.273153in}{0.500000in}}%
\pgfpathclose%
\pgfusepath{fill}%
\end{pgfscope}%
\begin{pgfscope}%
\pgfpathrectangle{\pgfqpoint{0.750000in}{0.500000in}}{\pgfqpoint{4.650000in}{3.020000in}}%
\pgfusepath{clip}%
\pgfsetbuttcap%
\pgfsetmiterjoin%
\definecolor{currentfill}{rgb}{1.000000,0.000000,0.000000}%
\pgfsetfillcolor{currentfill}%
\pgfsetlinewidth{0.000000pt}%
\definecolor{currentstroke}{rgb}{0.000000,0.000000,0.000000}%
\pgfsetstrokecolor{currentstroke}%
\pgfsetstrokeopacity{0.000000}%
\pgfsetdash{}{0pt}%
\pgfpathmoveto{\pgfqpoint{3.306179in}{0.500000in}}%
\pgfpathlineto{\pgfqpoint{3.339205in}{0.500000in}}%
\pgfpathlineto{\pgfqpoint{3.339205in}{0.500000in}}%
\pgfpathlineto{\pgfqpoint{3.306179in}{0.500000in}}%
\pgfpathlineto{\pgfqpoint{3.306179in}{0.500000in}}%
\pgfpathclose%
\pgfusepath{fill}%
\end{pgfscope}%
\begin{pgfscope}%
\pgfpathrectangle{\pgfqpoint{0.750000in}{0.500000in}}{\pgfqpoint{4.650000in}{3.020000in}}%
\pgfusepath{clip}%
\pgfsetbuttcap%
\pgfsetmiterjoin%
\definecolor{currentfill}{rgb}{1.000000,0.000000,0.000000}%
\pgfsetfillcolor{currentfill}%
\pgfsetlinewidth{0.000000pt}%
\definecolor{currentstroke}{rgb}{0.000000,0.000000,0.000000}%
\pgfsetstrokecolor{currentstroke}%
\pgfsetstrokeopacity{0.000000}%
\pgfsetdash{}{0pt}%
\pgfpathmoveto{\pgfqpoint{3.339205in}{0.500000in}}%
\pgfpathlineto{\pgfqpoint{3.372230in}{0.500000in}}%
\pgfpathlineto{\pgfqpoint{3.372230in}{0.538037in}}%
\pgfpathlineto{\pgfqpoint{3.339205in}{0.538037in}}%
\pgfpathlineto{\pgfqpoint{3.339205in}{0.500000in}}%
\pgfpathclose%
\pgfusepath{fill}%
\end{pgfscope}%
\begin{pgfscope}%
\pgfpathrectangle{\pgfqpoint{0.750000in}{0.500000in}}{\pgfqpoint{4.650000in}{3.020000in}}%
\pgfusepath{clip}%
\pgfsetbuttcap%
\pgfsetmiterjoin%
\definecolor{currentfill}{rgb}{1.000000,0.000000,0.000000}%
\pgfsetfillcolor{currentfill}%
\pgfsetlinewidth{0.000000pt}%
\definecolor{currentstroke}{rgb}{0.000000,0.000000,0.000000}%
\pgfsetstrokecolor{currentstroke}%
\pgfsetstrokeopacity{0.000000}%
\pgfsetdash{}{0pt}%
\pgfpathmoveto{\pgfqpoint{3.372230in}{0.500000in}}%
\pgfpathlineto{\pgfqpoint{3.405256in}{0.500000in}}%
\pgfpathlineto{\pgfqpoint{3.405256in}{0.500000in}}%
\pgfpathlineto{\pgfqpoint{3.372230in}{0.500000in}}%
\pgfpathlineto{\pgfqpoint{3.372230in}{0.500000in}}%
\pgfpathclose%
\pgfusepath{fill}%
\end{pgfscope}%
\begin{pgfscope}%
\pgfpathrectangle{\pgfqpoint{0.750000in}{0.500000in}}{\pgfqpoint{4.650000in}{3.020000in}}%
\pgfusepath{clip}%
\pgfsetbuttcap%
\pgfsetmiterjoin%
\definecolor{currentfill}{rgb}{1.000000,0.000000,0.000000}%
\pgfsetfillcolor{currentfill}%
\pgfsetlinewidth{0.000000pt}%
\definecolor{currentstroke}{rgb}{0.000000,0.000000,0.000000}%
\pgfsetstrokecolor{currentstroke}%
\pgfsetstrokeopacity{0.000000}%
\pgfsetdash{}{0pt}%
\pgfpathmoveto{\pgfqpoint{3.405256in}{0.500000in}}%
\pgfpathlineto{\pgfqpoint{3.438281in}{0.500000in}}%
\pgfpathlineto{\pgfqpoint{3.438281in}{0.500000in}}%
\pgfpathlineto{\pgfqpoint{3.405256in}{0.500000in}}%
\pgfpathlineto{\pgfqpoint{3.405256in}{0.500000in}}%
\pgfpathclose%
\pgfusepath{fill}%
\end{pgfscope}%
\begin{pgfscope}%
\pgfpathrectangle{\pgfqpoint{0.750000in}{0.500000in}}{\pgfqpoint{4.650000in}{3.020000in}}%
\pgfusepath{clip}%
\pgfsetbuttcap%
\pgfsetmiterjoin%
\definecolor{currentfill}{rgb}{1.000000,0.000000,0.000000}%
\pgfsetfillcolor{currentfill}%
\pgfsetlinewidth{0.000000pt}%
\definecolor{currentstroke}{rgb}{0.000000,0.000000,0.000000}%
\pgfsetstrokecolor{currentstroke}%
\pgfsetstrokeopacity{0.000000}%
\pgfsetdash{}{0pt}%
\pgfpathmoveto{\pgfqpoint{3.438281in}{0.500000in}}%
\pgfpathlineto{\pgfqpoint{3.471307in}{0.500000in}}%
\pgfpathlineto{\pgfqpoint{3.471307in}{0.505852in}}%
\pgfpathlineto{\pgfqpoint{3.438281in}{0.505852in}}%
\pgfpathlineto{\pgfqpoint{3.438281in}{0.500000in}}%
\pgfpathclose%
\pgfusepath{fill}%
\end{pgfscope}%
\begin{pgfscope}%
\pgfpathrectangle{\pgfqpoint{0.750000in}{0.500000in}}{\pgfqpoint{4.650000in}{3.020000in}}%
\pgfusepath{clip}%
\pgfsetbuttcap%
\pgfsetmiterjoin%
\definecolor{currentfill}{rgb}{1.000000,0.000000,0.000000}%
\pgfsetfillcolor{currentfill}%
\pgfsetlinewidth{0.000000pt}%
\definecolor{currentstroke}{rgb}{0.000000,0.000000,0.000000}%
\pgfsetstrokecolor{currentstroke}%
\pgfsetstrokeopacity{0.000000}%
\pgfsetdash{}{0pt}%
\pgfpathmoveto{\pgfqpoint{3.471307in}{0.500000in}}%
\pgfpathlineto{\pgfqpoint{3.504332in}{0.500000in}}%
\pgfpathlineto{\pgfqpoint{3.504332in}{0.500000in}}%
\pgfpathlineto{\pgfqpoint{3.471307in}{0.500000in}}%
\pgfpathlineto{\pgfqpoint{3.471307in}{0.500000in}}%
\pgfpathclose%
\pgfusepath{fill}%
\end{pgfscope}%
\begin{pgfscope}%
\pgfpathrectangle{\pgfqpoint{0.750000in}{0.500000in}}{\pgfqpoint{4.650000in}{3.020000in}}%
\pgfusepath{clip}%
\pgfsetbuttcap%
\pgfsetmiterjoin%
\definecolor{currentfill}{rgb}{1.000000,0.000000,0.000000}%
\pgfsetfillcolor{currentfill}%
\pgfsetlinewidth{0.000000pt}%
\definecolor{currentstroke}{rgb}{0.000000,0.000000,0.000000}%
\pgfsetstrokecolor{currentstroke}%
\pgfsetstrokeopacity{0.000000}%
\pgfsetdash{}{0pt}%
\pgfpathmoveto{\pgfqpoint{3.504332in}{0.500000in}}%
\pgfpathlineto{\pgfqpoint{3.537358in}{0.500000in}}%
\pgfpathlineto{\pgfqpoint{3.537358in}{0.502926in}}%
\pgfpathlineto{\pgfqpoint{3.504332in}{0.502926in}}%
\pgfpathlineto{\pgfqpoint{3.504332in}{0.500000in}}%
\pgfpathclose%
\pgfusepath{fill}%
\end{pgfscope}%
\begin{pgfscope}%
\pgfpathrectangle{\pgfqpoint{0.750000in}{0.500000in}}{\pgfqpoint{4.650000in}{3.020000in}}%
\pgfusepath{clip}%
\pgfsetbuttcap%
\pgfsetmiterjoin%
\definecolor{currentfill}{rgb}{1.000000,0.000000,0.000000}%
\pgfsetfillcolor{currentfill}%
\pgfsetlinewidth{0.000000pt}%
\definecolor{currentstroke}{rgb}{0.000000,0.000000,0.000000}%
\pgfsetstrokecolor{currentstroke}%
\pgfsetstrokeopacity{0.000000}%
\pgfsetdash{}{0pt}%
\pgfpathmoveto{\pgfqpoint{3.537358in}{0.500000in}}%
\pgfpathlineto{\pgfqpoint{3.570384in}{0.500000in}}%
\pgfpathlineto{\pgfqpoint{3.570384in}{0.505852in}}%
\pgfpathlineto{\pgfqpoint{3.537358in}{0.505852in}}%
\pgfpathlineto{\pgfqpoint{3.537358in}{0.500000in}}%
\pgfpathclose%
\pgfusepath{fill}%
\end{pgfscope}%
\begin{pgfscope}%
\pgfpathrectangle{\pgfqpoint{0.750000in}{0.500000in}}{\pgfqpoint{4.650000in}{3.020000in}}%
\pgfusepath{clip}%
\pgfsetbuttcap%
\pgfsetmiterjoin%
\definecolor{currentfill}{rgb}{1.000000,0.000000,0.000000}%
\pgfsetfillcolor{currentfill}%
\pgfsetlinewidth{0.000000pt}%
\definecolor{currentstroke}{rgb}{0.000000,0.000000,0.000000}%
\pgfsetstrokecolor{currentstroke}%
\pgfsetstrokeopacity{0.000000}%
\pgfsetdash{}{0pt}%
\pgfpathmoveto{\pgfqpoint{3.570384in}{0.500000in}}%
\pgfpathlineto{\pgfqpoint{3.603409in}{0.500000in}}%
\pgfpathlineto{\pgfqpoint{3.603409in}{0.500000in}}%
\pgfpathlineto{\pgfqpoint{3.570384in}{0.500000in}}%
\pgfpathlineto{\pgfqpoint{3.570384in}{0.500000in}}%
\pgfpathclose%
\pgfusepath{fill}%
\end{pgfscope}%
\begin{pgfscope}%
\pgfpathrectangle{\pgfqpoint{0.750000in}{0.500000in}}{\pgfqpoint{4.650000in}{3.020000in}}%
\pgfusepath{clip}%
\pgfsetbuttcap%
\pgfsetmiterjoin%
\definecolor{currentfill}{rgb}{1.000000,0.000000,0.000000}%
\pgfsetfillcolor{currentfill}%
\pgfsetlinewidth{0.000000pt}%
\definecolor{currentstroke}{rgb}{0.000000,0.000000,0.000000}%
\pgfsetstrokecolor{currentstroke}%
\pgfsetstrokeopacity{0.000000}%
\pgfsetdash{}{0pt}%
\pgfpathmoveto{\pgfqpoint{3.603409in}{0.500000in}}%
\pgfpathlineto{\pgfqpoint{3.636435in}{0.500000in}}%
\pgfpathlineto{\pgfqpoint{3.636435in}{0.500000in}}%
\pgfpathlineto{\pgfqpoint{3.603409in}{0.500000in}}%
\pgfpathlineto{\pgfqpoint{3.603409in}{0.500000in}}%
\pgfpathclose%
\pgfusepath{fill}%
\end{pgfscope}%
\begin{pgfscope}%
\pgfpathrectangle{\pgfqpoint{0.750000in}{0.500000in}}{\pgfqpoint{4.650000in}{3.020000in}}%
\pgfusepath{clip}%
\pgfsetbuttcap%
\pgfsetmiterjoin%
\definecolor{currentfill}{rgb}{1.000000,0.000000,0.000000}%
\pgfsetfillcolor{currentfill}%
\pgfsetlinewidth{0.000000pt}%
\definecolor{currentstroke}{rgb}{0.000000,0.000000,0.000000}%
\pgfsetstrokecolor{currentstroke}%
\pgfsetstrokeopacity{0.000000}%
\pgfsetdash{}{0pt}%
\pgfpathmoveto{\pgfqpoint{3.636435in}{0.500000in}}%
\pgfpathlineto{\pgfqpoint{3.669460in}{0.500000in}}%
\pgfpathlineto{\pgfqpoint{3.669460in}{0.502926in}}%
\pgfpathlineto{\pgfqpoint{3.636435in}{0.502926in}}%
\pgfpathlineto{\pgfqpoint{3.636435in}{0.500000in}}%
\pgfpathclose%
\pgfusepath{fill}%
\end{pgfscope}%
\begin{pgfscope}%
\pgfpathrectangle{\pgfqpoint{0.750000in}{0.500000in}}{\pgfqpoint{4.650000in}{3.020000in}}%
\pgfusepath{clip}%
\pgfsetbuttcap%
\pgfsetmiterjoin%
\definecolor{currentfill}{rgb}{1.000000,0.000000,0.000000}%
\pgfsetfillcolor{currentfill}%
\pgfsetlinewidth{0.000000pt}%
\definecolor{currentstroke}{rgb}{0.000000,0.000000,0.000000}%
\pgfsetstrokecolor{currentstroke}%
\pgfsetstrokeopacity{0.000000}%
\pgfsetdash{}{0pt}%
\pgfpathmoveto{\pgfqpoint{3.669460in}{0.500000in}}%
\pgfpathlineto{\pgfqpoint{3.702486in}{0.500000in}}%
\pgfpathlineto{\pgfqpoint{3.702486in}{0.500000in}}%
\pgfpathlineto{\pgfqpoint{3.669460in}{0.500000in}}%
\pgfpathlineto{\pgfqpoint{3.669460in}{0.500000in}}%
\pgfpathclose%
\pgfusepath{fill}%
\end{pgfscope}%
\begin{pgfscope}%
\pgfpathrectangle{\pgfqpoint{0.750000in}{0.500000in}}{\pgfqpoint{4.650000in}{3.020000in}}%
\pgfusepath{clip}%
\pgfsetbuttcap%
\pgfsetmiterjoin%
\definecolor{currentfill}{rgb}{1.000000,0.000000,0.000000}%
\pgfsetfillcolor{currentfill}%
\pgfsetlinewidth{0.000000pt}%
\definecolor{currentstroke}{rgb}{0.000000,0.000000,0.000000}%
\pgfsetstrokecolor{currentstroke}%
\pgfsetstrokeopacity{0.000000}%
\pgfsetdash{}{0pt}%
\pgfpathmoveto{\pgfqpoint{3.702486in}{0.500000in}}%
\pgfpathlineto{\pgfqpoint{3.735511in}{0.500000in}}%
\pgfpathlineto{\pgfqpoint{3.735511in}{0.502926in}}%
\pgfpathlineto{\pgfqpoint{3.702486in}{0.502926in}}%
\pgfpathlineto{\pgfqpoint{3.702486in}{0.500000in}}%
\pgfpathclose%
\pgfusepath{fill}%
\end{pgfscope}%
\begin{pgfscope}%
\pgfpathrectangle{\pgfqpoint{0.750000in}{0.500000in}}{\pgfqpoint{4.650000in}{3.020000in}}%
\pgfusepath{clip}%
\pgfsetbuttcap%
\pgfsetmiterjoin%
\definecolor{currentfill}{rgb}{1.000000,0.000000,0.000000}%
\pgfsetfillcolor{currentfill}%
\pgfsetlinewidth{0.000000pt}%
\definecolor{currentstroke}{rgb}{0.000000,0.000000,0.000000}%
\pgfsetstrokecolor{currentstroke}%
\pgfsetstrokeopacity{0.000000}%
\pgfsetdash{}{0pt}%
\pgfpathmoveto{\pgfqpoint{3.735511in}{0.500000in}}%
\pgfpathlineto{\pgfqpoint{3.768537in}{0.500000in}}%
\pgfpathlineto{\pgfqpoint{3.768537in}{0.500000in}}%
\pgfpathlineto{\pgfqpoint{3.735511in}{0.500000in}}%
\pgfpathlineto{\pgfqpoint{3.735511in}{0.500000in}}%
\pgfpathclose%
\pgfusepath{fill}%
\end{pgfscope}%
\begin{pgfscope}%
\pgfpathrectangle{\pgfqpoint{0.750000in}{0.500000in}}{\pgfqpoint{4.650000in}{3.020000in}}%
\pgfusepath{clip}%
\pgfsetbuttcap%
\pgfsetmiterjoin%
\definecolor{currentfill}{rgb}{1.000000,0.000000,0.000000}%
\pgfsetfillcolor{currentfill}%
\pgfsetlinewidth{0.000000pt}%
\definecolor{currentstroke}{rgb}{0.000000,0.000000,0.000000}%
\pgfsetstrokecolor{currentstroke}%
\pgfsetstrokeopacity{0.000000}%
\pgfsetdash{}{0pt}%
\pgfpathmoveto{\pgfqpoint{3.768537in}{0.500000in}}%
\pgfpathlineto{\pgfqpoint{3.801563in}{0.500000in}}%
\pgfpathlineto{\pgfqpoint{3.801563in}{0.500000in}}%
\pgfpathlineto{\pgfqpoint{3.768537in}{0.500000in}}%
\pgfpathlineto{\pgfqpoint{3.768537in}{0.500000in}}%
\pgfpathclose%
\pgfusepath{fill}%
\end{pgfscope}%
\begin{pgfscope}%
\pgfpathrectangle{\pgfqpoint{0.750000in}{0.500000in}}{\pgfqpoint{4.650000in}{3.020000in}}%
\pgfusepath{clip}%
\pgfsetbuttcap%
\pgfsetmiterjoin%
\definecolor{currentfill}{rgb}{1.000000,0.000000,0.000000}%
\pgfsetfillcolor{currentfill}%
\pgfsetlinewidth{0.000000pt}%
\definecolor{currentstroke}{rgb}{0.000000,0.000000,0.000000}%
\pgfsetstrokecolor{currentstroke}%
\pgfsetstrokeopacity{0.000000}%
\pgfsetdash{}{0pt}%
\pgfpathmoveto{\pgfqpoint{3.801563in}{0.500000in}}%
\pgfpathlineto{\pgfqpoint{3.834588in}{0.500000in}}%
\pgfpathlineto{\pgfqpoint{3.834588in}{0.500000in}}%
\pgfpathlineto{\pgfqpoint{3.801563in}{0.500000in}}%
\pgfpathlineto{\pgfqpoint{3.801563in}{0.500000in}}%
\pgfpathclose%
\pgfusepath{fill}%
\end{pgfscope}%
\begin{pgfscope}%
\pgfpathrectangle{\pgfqpoint{0.750000in}{0.500000in}}{\pgfqpoint{4.650000in}{3.020000in}}%
\pgfusepath{clip}%
\pgfsetbuttcap%
\pgfsetmiterjoin%
\definecolor{currentfill}{rgb}{1.000000,0.000000,0.000000}%
\pgfsetfillcolor{currentfill}%
\pgfsetlinewidth{0.000000pt}%
\definecolor{currentstroke}{rgb}{0.000000,0.000000,0.000000}%
\pgfsetstrokecolor{currentstroke}%
\pgfsetstrokeopacity{0.000000}%
\pgfsetdash{}{0pt}%
\pgfpathmoveto{\pgfqpoint{3.834588in}{0.500000in}}%
\pgfpathlineto{\pgfqpoint{3.867614in}{0.500000in}}%
\pgfpathlineto{\pgfqpoint{3.867614in}{0.500000in}}%
\pgfpathlineto{\pgfqpoint{3.834588in}{0.500000in}}%
\pgfpathlineto{\pgfqpoint{3.834588in}{0.500000in}}%
\pgfpathclose%
\pgfusepath{fill}%
\end{pgfscope}%
\begin{pgfscope}%
\pgfpathrectangle{\pgfqpoint{0.750000in}{0.500000in}}{\pgfqpoint{4.650000in}{3.020000in}}%
\pgfusepath{clip}%
\pgfsetbuttcap%
\pgfsetmiterjoin%
\definecolor{currentfill}{rgb}{1.000000,0.000000,0.000000}%
\pgfsetfillcolor{currentfill}%
\pgfsetlinewidth{0.000000pt}%
\definecolor{currentstroke}{rgb}{0.000000,0.000000,0.000000}%
\pgfsetstrokecolor{currentstroke}%
\pgfsetstrokeopacity{0.000000}%
\pgfsetdash{}{0pt}%
\pgfpathmoveto{\pgfqpoint{3.867614in}{0.500000in}}%
\pgfpathlineto{\pgfqpoint{3.900639in}{0.500000in}}%
\pgfpathlineto{\pgfqpoint{3.900639in}{0.500000in}}%
\pgfpathlineto{\pgfqpoint{3.867614in}{0.500000in}}%
\pgfpathlineto{\pgfqpoint{3.867614in}{0.500000in}}%
\pgfpathclose%
\pgfusepath{fill}%
\end{pgfscope}%
\begin{pgfscope}%
\pgfpathrectangle{\pgfqpoint{0.750000in}{0.500000in}}{\pgfqpoint{4.650000in}{3.020000in}}%
\pgfusepath{clip}%
\pgfsetbuttcap%
\pgfsetmiterjoin%
\definecolor{currentfill}{rgb}{1.000000,0.000000,0.000000}%
\pgfsetfillcolor{currentfill}%
\pgfsetlinewidth{0.000000pt}%
\definecolor{currentstroke}{rgb}{0.000000,0.000000,0.000000}%
\pgfsetstrokecolor{currentstroke}%
\pgfsetstrokeopacity{0.000000}%
\pgfsetdash{}{0pt}%
\pgfpathmoveto{\pgfqpoint{3.900639in}{0.500000in}}%
\pgfpathlineto{\pgfqpoint{3.933665in}{0.500000in}}%
\pgfpathlineto{\pgfqpoint{3.933665in}{0.500000in}}%
\pgfpathlineto{\pgfqpoint{3.900639in}{0.500000in}}%
\pgfpathlineto{\pgfqpoint{3.900639in}{0.500000in}}%
\pgfpathclose%
\pgfusepath{fill}%
\end{pgfscope}%
\begin{pgfscope}%
\pgfpathrectangle{\pgfqpoint{0.750000in}{0.500000in}}{\pgfqpoint{4.650000in}{3.020000in}}%
\pgfusepath{clip}%
\pgfsetbuttcap%
\pgfsetmiterjoin%
\definecolor{currentfill}{rgb}{1.000000,0.000000,0.000000}%
\pgfsetfillcolor{currentfill}%
\pgfsetlinewidth{0.000000pt}%
\definecolor{currentstroke}{rgb}{0.000000,0.000000,0.000000}%
\pgfsetstrokecolor{currentstroke}%
\pgfsetstrokeopacity{0.000000}%
\pgfsetdash{}{0pt}%
\pgfpathmoveto{\pgfqpoint{3.933665in}{0.500000in}}%
\pgfpathlineto{\pgfqpoint{3.966690in}{0.500000in}}%
\pgfpathlineto{\pgfqpoint{3.966690in}{0.500000in}}%
\pgfpathlineto{\pgfqpoint{3.933665in}{0.500000in}}%
\pgfpathlineto{\pgfqpoint{3.933665in}{0.500000in}}%
\pgfpathclose%
\pgfusepath{fill}%
\end{pgfscope}%
\begin{pgfscope}%
\pgfpathrectangle{\pgfqpoint{0.750000in}{0.500000in}}{\pgfqpoint{4.650000in}{3.020000in}}%
\pgfusepath{clip}%
\pgfsetbuttcap%
\pgfsetmiterjoin%
\definecolor{currentfill}{rgb}{1.000000,0.000000,0.000000}%
\pgfsetfillcolor{currentfill}%
\pgfsetlinewidth{0.000000pt}%
\definecolor{currentstroke}{rgb}{0.000000,0.000000,0.000000}%
\pgfsetstrokecolor{currentstroke}%
\pgfsetstrokeopacity{0.000000}%
\pgfsetdash{}{0pt}%
\pgfpathmoveto{\pgfqpoint{3.966690in}{0.500000in}}%
\pgfpathlineto{\pgfqpoint{3.999716in}{0.500000in}}%
\pgfpathlineto{\pgfqpoint{3.999716in}{0.500000in}}%
\pgfpathlineto{\pgfqpoint{3.966690in}{0.500000in}}%
\pgfpathlineto{\pgfqpoint{3.966690in}{0.500000in}}%
\pgfpathclose%
\pgfusepath{fill}%
\end{pgfscope}%
\begin{pgfscope}%
\pgfpathrectangle{\pgfqpoint{0.750000in}{0.500000in}}{\pgfqpoint{4.650000in}{3.020000in}}%
\pgfusepath{clip}%
\pgfsetbuttcap%
\pgfsetmiterjoin%
\definecolor{currentfill}{rgb}{1.000000,0.000000,0.000000}%
\pgfsetfillcolor{currentfill}%
\pgfsetlinewidth{0.000000pt}%
\definecolor{currentstroke}{rgb}{0.000000,0.000000,0.000000}%
\pgfsetstrokecolor{currentstroke}%
\pgfsetstrokeopacity{0.000000}%
\pgfsetdash{}{0pt}%
\pgfpathmoveto{\pgfqpoint{3.999716in}{0.500000in}}%
\pgfpathlineto{\pgfqpoint{4.032741in}{0.500000in}}%
\pgfpathlineto{\pgfqpoint{4.032741in}{0.500000in}}%
\pgfpathlineto{\pgfqpoint{3.999716in}{0.500000in}}%
\pgfpathlineto{\pgfqpoint{3.999716in}{0.500000in}}%
\pgfpathclose%
\pgfusepath{fill}%
\end{pgfscope}%
\begin{pgfscope}%
\pgfpathrectangle{\pgfqpoint{0.750000in}{0.500000in}}{\pgfqpoint{4.650000in}{3.020000in}}%
\pgfusepath{clip}%
\pgfsetbuttcap%
\pgfsetmiterjoin%
\definecolor{currentfill}{rgb}{1.000000,0.000000,0.000000}%
\pgfsetfillcolor{currentfill}%
\pgfsetlinewidth{0.000000pt}%
\definecolor{currentstroke}{rgb}{0.000000,0.000000,0.000000}%
\pgfsetstrokecolor{currentstroke}%
\pgfsetstrokeopacity{0.000000}%
\pgfsetdash{}{0pt}%
\pgfpathmoveto{\pgfqpoint{4.032741in}{0.500000in}}%
\pgfpathlineto{\pgfqpoint{4.065767in}{0.500000in}}%
\pgfpathlineto{\pgfqpoint{4.065767in}{0.500000in}}%
\pgfpathlineto{\pgfqpoint{4.032741in}{0.500000in}}%
\pgfpathlineto{\pgfqpoint{4.032741in}{0.500000in}}%
\pgfpathclose%
\pgfusepath{fill}%
\end{pgfscope}%
\begin{pgfscope}%
\pgfpathrectangle{\pgfqpoint{0.750000in}{0.500000in}}{\pgfqpoint{4.650000in}{3.020000in}}%
\pgfusepath{clip}%
\pgfsetbuttcap%
\pgfsetmiterjoin%
\definecolor{currentfill}{rgb}{1.000000,0.000000,0.000000}%
\pgfsetfillcolor{currentfill}%
\pgfsetlinewidth{0.000000pt}%
\definecolor{currentstroke}{rgb}{0.000000,0.000000,0.000000}%
\pgfsetstrokecolor{currentstroke}%
\pgfsetstrokeopacity{0.000000}%
\pgfsetdash{}{0pt}%
\pgfpathmoveto{\pgfqpoint{4.065767in}{0.500000in}}%
\pgfpathlineto{\pgfqpoint{4.098793in}{0.500000in}}%
\pgfpathlineto{\pgfqpoint{4.098793in}{0.500000in}}%
\pgfpathlineto{\pgfqpoint{4.065767in}{0.500000in}}%
\pgfpathlineto{\pgfqpoint{4.065767in}{0.500000in}}%
\pgfpathclose%
\pgfusepath{fill}%
\end{pgfscope}%
\begin{pgfscope}%
\pgfpathrectangle{\pgfqpoint{0.750000in}{0.500000in}}{\pgfqpoint{4.650000in}{3.020000in}}%
\pgfusepath{clip}%
\pgfsetbuttcap%
\pgfsetmiterjoin%
\definecolor{currentfill}{rgb}{1.000000,0.000000,0.000000}%
\pgfsetfillcolor{currentfill}%
\pgfsetlinewidth{0.000000pt}%
\definecolor{currentstroke}{rgb}{0.000000,0.000000,0.000000}%
\pgfsetstrokecolor{currentstroke}%
\pgfsetstrokeopacity{0.000000}%
\pgfsetdash{}{0pt}%
\pgfpathmoveto{\pgfqpoint{4.098793in}{0.500000in}}%
\pgfpathlineto{\pgfqpoint{4.131818in}{0.500000in}}%
\pgfpathlineto{\pgfqpoint{4.131818in}{0.500000in}}%
\pgfpathlineto{\pgfqpoint{4.098793in}{0.500000in}}%
\pgfpathlineto{\pgfqpoint{4.098793in}{0.500000in}}%
\pgfpathclose%
\pgfusepath{fill}%
\end{pgfscope}%
\begin{pgfscope}%
\pgfpathrectangle{\pgfqpoint{0.750000in}{0.500000in}}{\pgfqpoint{4.650000in}{3.020000in}}%
\pgfusepath{clip}%
\pgfsetbuttcap%
\pgfsetmiterjoin%
\definecolor{currentfill}{rgb}{1.000000,0.000000,0.000000}%
\pgfsetfillcolor{currentfill}%
\pgfsetlinewidth{0.000000pt}%
\definecolor{currentstroke}{rgb}{0.000000,0.000000,0.000000}%
\pgfsetstrokecolor{currentstroke}%
\pgfsetstrokeopacity{0.000000}%
\pgfsetdash{}{0pt}%
\pgfpathmoveto{\pgfqpoint{4.131818in}{0.500000in}}%
\pgfpathlineto{\pgfqpoint{4.164844in}{0.500000in}}%
\pgfpathlineto{\pgfqpoint{4.164844in}{0.500000in}}%
\pgfpathlineto{\pgfqpoint{4.131818in}{0.500000in}}%
\pgfpathlineto{\pgfqpoint{4.131818in}{0.500000in}}%
\pgfpathclose%
\pgfusepath{fill}%
\end{pgfscope}%
\begin{pgfscope}%
\pgfpathrectangle{\pgfqpoint{0.750000in}{0.500000in}}{\pgfqpoint{4.650000in}{3.020000in}}%
\pgfusepath{clip}%
\pgfsetbuttcap%
\pgfsetmiterjoin%
\definecolor{currentfill}{rgb}{1.000000,0.000000,0.000000}%
\pgfsetfillcolor{currentfill}%
\pgfsetlinewidth{0.000000pt}%
\definecolor{currentstroke}{rgb}{0.000000,0.000000,0.000000}%
\pgfsetstrokecolor{currentstroke}%
\pgfsetstrokeopacity{0.000000}%
\pgfsetdash{}{0pt}%
\pgfpathmoveto{\pgfqpoint{4.164844in}{0.500000in}}%
\pgfpathlineto{\pgfqpoint{4.197869in}{0.500000in}}%
\pgfpathlineto{\pgfqpoint{4.197869in}{0.500000in}}%
\pgfpathlineto{\pgfqpoint{4.164844in}{0.500000in}}%
\pgfpathlineto{\pgfqpoint{4.164844in}{0.500000in}}%
\pgfpathclose%
\pgfusepath{fill}%
\end{pgfscope}%
\begin{pgfscope}%
\pgfpathrectangle{\pgfqpoint{0.750000in}{0.500000in}}{\pgfqpoint{4.650000in}{3.020000in}}%
\pgfusepath{clip}%
\pgfsetbuttcap%
\pgfsetmiterjoin%
\definecolor{currentfill}{rgb}{1.000000,0.000000,0.000000}%
\pgfsetfillcolor{currentfill}%
\pgfsetlinewidth{0.000000pt}%
\definecolor{currentstroke}{rgb}{0.000000,0.000000,0.000000}%
\pgfsetstrokecolor{currentstroke}%
\pgfsetstrokeopacity{0.000000}%
\pgfsetdash{}{0pt}%
\pgfpathmoveto{\pgfqpoint{4.197869in}{0.500000in}}%
\pgfpathlineto{\pgfqpoint{4.230895in}{0.500000in}}%
\pgfpathlineto{\pgfqpoint{4.230895in}{0.500000in}}%
\pgfpathlineto{\pgfqpoint{4.197869in}{0.500000in}}%
\pgfpathlineto{\pgfqpoint{4.197869in}{0.500000in}}%
\pgfpathclose%
\pgfusepath{fill}%
\end{pgfscope}%
\begin{pgfscope}%
\pgfpathrectangle{\pgfqpoint{0.750000in}{0.500000in}}{\pgfqpoint{4.650000in}{3.020000in}}%
\pgfusepath{clip}%
\pgfsetbuttcap%
\pgfsetmiterjoin%
\definecolor{currentfill}{rgb}{1.000000,0.000000,0.000000}%
\pgfsetfillcolor{currentfill}%
\pgfsetlinewidth{0.000000pt}%
\definecolor{currentstroke}{rgb}{0.000000,0.000000,0.000000}%
\pgfsetstrokecolor{currentstroke}%
\pgfsetstrokeopacity{0.000000}%
\pgfsetdash{}{0pt}%
\pgfpathmoveto{\pgfqpoint{4.230895in}{0.500000in}}%
\pgfpathlineto{\pgfqpoint{4.263920in}{0.500000in}}%
\pgfpathlineto{\pgfqpoint{4.263920in}{0.502926in}}%
\pgfpathlineto{\pgfqpoint{4.230895in}{0.502926in}}%
\pgfpathlineto{\pgfqpoint{4.230895in}{0.500000in}}%
\pgfpathclose%
\pgfusepath{fill}%
\end{pgfscope}%
\begin{pgfscope}%
\pgfpathrectangle{\pgfqpoint{0.750000in}{0.500000in}}{\pgfqpoint{4.650000in}{3.020000in}}%
\pgfusepath{clip}%
\pgfsetbuttcap%
\pgfsetmiterjoin%
\definecolor{currentfill}{rgb}{1.000000,0.000000,0.000000}%
\pgfsetfillcolor{currentfill}%
\pgfsetlinewidth{0.000000pt}%
\definecolor{currentstroke}{rgb}{0.000000,0.000000,0.000000}%
\pgfsetstrokecolor{currentstroke}%
\pgfsetstrokeopacity{0.000000}%
\pgfsetdash{}{0pt}%
\pgfpathmoveto{\pgfqpoint{4.263920in}{0.500000in}}%
\pgfpathlineto{\pgfqpoint{4.296946in}{0.500000in}}%
\pgfpathlineto{\pgfqpoint{4.296946in}{0.500000in}}%
\pgfpathlineto{\pgfqpoint{4.263920in}{0.500000in}}%
\pgfpathlineto{\pgfqpoint{4.263920in}{0.500000in}}%
\pgfpathclose%
\pgfusepath{fill}%
\end{pgfscope}%
\begin{pgfscope}%
\pgfpathrectangle{\pgfqpoint{0.750000in}{0.500000in}}{\pgfqpoint{4.650000in}{3.020000in}}%
\pgfusepath{clip}%
\pgfsetbuttcap%
\pgfsetmiterjoin%
\definecolor{currentfill}{rgb}{1.000000,0.000000,0.000000}%
\pgfsetfillcolor{currentfill}%
\pgfsetlinewidth{0.000000pt}%
\definecolor{currentstroke}{rgb}{0.000000,0.000000,0.000000}%
\pgfsetstrokecolor{currentstroke}%
\pgfsetstrokeopacity{0.000000}%
\pgfsetdash{}{0pt}%
\pgfpathmoveto{\pgfqpoint{4.296946in}{0.500000in}}%
\pgfpathlineto{\pgfqpoint{4.329972in}{0.500000in}}%
\pgfpathlineto{\pgfqpoint{4.329972in}{0.500000in}}%
\pgfpathlineto{\pgfqpoint{4.296946in}{0.500000in}}%
\pgfpathlineto{\pgfqpoint{4.296946in}{0.500000in}}%
\pgfpathclose%
\pgfusepath{fill}%
\end{pgfscope}%
\begin{pgfscope}%
\pgfpathrectangle{\pgfqpoint{0.750000in}{0.500000in}}{\pgfqpoint{4.650000in}{3.020000in}}%
\pgfusepath{clip}%
\pgfsetbuttcap%
\pgfsetmiterjoin%
\definecolor{currentfill}{rgb}{1.000000,0.000000,0.000000}%
\pgfsetfillcolor{currentfill}%
\pgfsetlinewidth{0.000000pt}%
\definecolor{currentstroke}{rgb}{0.000000,0.000000,0.000000}%
\pgfsetstrokecolor{currentstroke}%
\pgfsetstrokeopacity{0.000000}%
\pgfsetdash{}{0pt}%
\pgfpathmoveto{\pgfqpoint{4.329972in}{0.500000in}}%
\pgfpathlineto{\pgfqpoint{4.362997in}{0.500000in}}%
\pgfpathlineto{\pgfqpoint{4.362997in}{0.500000in}}%
\pgfpathlineto{\pgfqpoint{4.329972in}{0.500000in}}%
\pgfpathlineto{\pgfqpoint{4.329972in}{0.500000in}}%
\pgfpathclose%
\pgfusepath{fill}%
\end{pgfscope}%
\begin{pgfscope}%
\pgfpathrectangle{\pgfqpoint{0.750000in}{0.500000in}}{\pgfqpoint{4.650000in}{3.020000in}}%
\pgfusepath{clip}%
\pgfsetbuttcap%
\pgfsetmiterjoin%
\definecolor{currentfill}{rgb}{1.000000,0.000000,0.000000}%
\pgfsetfillcolor{currentfill}%
\pgfsetlinewidth{0.000000pt}%
\definecolor{currentstroke}{rgb}{0.000000,0.000000,0.000000}%
\pgfsetstrokecolor{currentstroke}%
\pgfsetstrokeopacity{0.000000}%
\pgfsetdash{}{0pt}%
\pgfpathmoveto{\pgfqpoint{4.362997in}{0.500000in}}%
\pgfpathlineto{\pgfqpoint{4.396023in}{0.500000in}}%
\pgfpathlineto{\pgfqpoint{4.396023in}{0.500000in}}%
\pgfpathlineto{\pgfqpoint{4.362997in}{0.500000in}}%
\pgfpathlineto{\pgfqpoint{4.362997in}{0.500000in}}%
\pgfpathclose%
\pgfusepath{fill}%
\end{pgfscope}%
\begin{pgfscope}%
\pgfpathrectangle{\pgfqpoint{0.750000in}{0.500000in}}{\pgfqpoint{4.650000in}{3.020000in}}%
\pgfusepath{clip}%
\pgfsetbuttcap%
\pgfsetmiterjoin%
\definecolor{currentfill}{rgb}{1.000000,0.000000,0.000000}%
\pgfsetfillcolor{currentfill}%
\pgfsetlinewidth{0.000000pt}%
\definecolor{currentstroke}{rgb}{0.000000,0.000000,0.000000}%
\pgfsetstrokecolor{currentstroke}%
\pgfsetstrokeopacity{0.000000}%
\pgfsetdash{}{0pt}%
\pgfpathmoveto{\pgfqpoint{4.396023in}{0.500000in}}%
\pgfpathlineto{\pgfqpoint{4.429048in}{0.500000in}}%
\pgfpathlineto{\pgfqpoint{4.429048in}{0.500000in}}%
\pgfpathlineto{\pgfqpoint{4.396023in}{0.500000in}}%
\pgfpathlineto{\pgfqpoint{4.396023in}{0.500000in}}%
\pgfpathclose%
\pgfusepath{fill}%
\end{pgfscope}%
\begin{pgfscope}%
\pgfpathrectangle{\pgfqpoint{0.750000in}{0.500000in}}{\pgfqpoint{4.650000in}{3.020000in}}%
\pgfusepath{clip}%
\pgfsetbuttcap%
\pgfsetmiterjoin%
\definecolor{currentfill}{rgb}{1.000000,0.000000,0.000000}%
\pgfsetfillcolor{currentfill}%
\pgfsetlinewidth{0.000000pt}%
\definecolor{currentstroke}{rgb}{0.000000,0.000000,0.000000}%
\pgfsetstrokecolor{currentstroke}%
\pgfsetstrokeopacity{0.000000}%
\pgfsetdash{}{0pt}%
\pgfpathmoveto{\pgfqpoint{4.429048in}{0.500000in}}%
\pgfpathlineto{\pgfqpoint{4.462074in}{0.500000in}}%
\pgfpathlineto{\pgfqpoint{4.462074in}{0.500000in}}%
\pgfpathlineto{\pgfqpoint{4.429048in}{0.500000in}}%
\pgfpathlineto{\pgfqpoint{4.429048in}{0.500000in}}%
\pgfpathclose%
\pgfusepath{fill}%
\end{pgfscope}%
\begin{pgfscope}%
\pgfpathrectangle{\pgfqpoint{0.750000in}{0.500000in}}{\pgfqpoint{4.650000in}{3.020000in}}%
\pgfusepath{clip}%
\pgfsetbuttcap%
\pgfsetmiterjoin%
\definecolor{currentfill}{rgb}{1.000000,0.000000,0.000000}%
\pgfsetfillcolor{currentfill}%
\pgfsetlinewidth{0.000000pt}%
\definecolor{currentstroke}{rgb}{0.000000,0.000000,0.000000}%
\pgfsetstrokecolor{currentstroke}%
\pgfsetstrokeopacity{0.000000}%
\pgfsetdash{}{0pt}%
\pgfpathmoveto{\pgfqpoint{4.462074in}{0.500000in}}%
\pgfpathlineto{\pgfqpoint{4.495099in}{0.500000in}}%
\pgfpathlineto{\pgfqpoint{4.495099in}{0.500000in}}%
\pgfpathlineto{\pgfqpoint{4.462074in}{0.500000in}}%
\pgfpathlineto{\pgfqpoint{4.462074in}{0.500000in}}%
\pgfpathclose%
\pgfusepath{fill}%
\end{pgfscope}%
\begin{pgfscope}%
\pgfpathrectangle{\pgfqpoint{0.750000in}{0.500000in}}{\pgfqpoint{4.650000in}{3.020000in}}%
\pgfusepath{clip}%
\pgfsetbuttcap%
\pgfsetmiterjoin%
\definecolor{currentfill}{rgb}{1.000000,0.000000,0.000000}%
\pgfsetfillcolor{currentfill}%
\pgfsetlinewidth{0.000000pt}%
\definecolor{currentstroke}{rgb}{0.000000,0.000000,0.000000}%
\pgfsetstrokecolor{currentstroke}%
\pgfsetstrokeopacity{0.000000}%
\pgfsetdash{}{0pt}%
\pgfpathmoveto{\pgfqpoint{4.495099in}{0.500000in}}%
\pgfpathlineto{\pgfqpoint{4.528125in}{0.500000in}}%
\pgfpathlineto{\pgfqpoint{4.528125in}{0.502926in}}%
\pgfpathlineto{\pgfqpoint{4.495099in}{0.502926in}}%
\pgfpathlineto{\pgfqpoint{4.495099in}{0.500000in}}%
\pgfpathclose%
\pgfusepath{fill}%
\end{pgfscope}%
\begin{pgfscope}%
\pgfpathrectangle{\pgfqpoint{0.750000in}{0.500000in}}{\pgfqpoint{4.650000in}{3.020000in}}%
\pgfusepath{clip}%
\pgfsetbuttcap%
\pgfsetmiterjoin%
\definecolor{currentfill}{rgb}{1.000000,0.000000,0.000000}%
\pgfsetfillcolor{currentfill}%
\pgfsetlinewidth{0.000000pt}%
\definecolor{currentstroke}{rgb}{0.000000,0.000000,0.000000}%
\pgfsetstrokecolor{currentstroke}%
\pgfsetstrokeopacity{0.000000}%
\pgfsetdash{}{0pt}%
\pgfpathmoveto{\pgfqpoint{4.528125in}{0.500000in}}%
\pgfpathlineto{\pgfqpoint{4.561151in}{0.500000in}}%
\pgfpathlineto{\pgfqpoint{4.561151in}{0.502926in}}%
\pgfpathlineto{\pgfqpoint{4.528125in}{0.502926in}}%
\pgfpathlineto{\pgfqpoint{4.528125in}{0.500000in}}%
\pgfpathclose%
\pgfusepath{fill}%
\end{pgfscope}%
\begin{pgfscope}%
\pgfpathrectangle{\pgfqpoint{0.750000in}{0.500000in}}{\pgfqpoint{4.650000in}{3.020000in}}%
\pgfusepath{clip}%
\pgfsetbuttcap%
\pgfsetmiterjoin%
\definecolor{currentfill}{rgb}{1.000000,0.000000,0.000000}%
\pgfsetfillcolor{currentfill}%
\pgfsetlinewidth{0.000000pt}%
\definecolor{currentstroke}{rgb}{0.000000,0.000000,0.000000}%
\pgfsetstrokecolor{currentstroke}%
\pgfsetstrokeopacity{0.000000}%
\pgfsetdash{}{0pt}%
\pgfpathmoveto{\pgfqpoint{4.561151in}{0.500000in}}%
\pgfpathlineto{\pgfqpoint{4.594176in}{0.500000in}}%
\pgfpathlineto{\pgfqpoint{4.594176in}{0.502926in}}%
\pgfpathlineto{\pgfqpoint{4.561151in}{0.502926in}}%
\pgfpathlineto{\pgfqpoint{4.561151in}{0.500000in}}%
\pgfpathclose%
\pgfusepath{fill}%
\end{pgfscope}%
\begin{pgfscope}%
\pgfpathrectangle{\pgfqpoint{0.750000in}{0.500000in}}{\pgfqpoint{4.650000in}{3.020000in}}%
\pgfusepath{clip}%
\pgfsetbuttcap%
\pgfsetmiterjoin%
\definecolor{currentfill}{rgb}{1.000000,0.000000,0.000000}%
\pgfsetfillcolor{currentfill}%
\pgfsetlinewidth{0.000000pt}%
\definecolor{currentstroke}{rgb}{0.000000,0.000000,0.000000}%
\pgfsetstrokecolor{currentstroke}%
\pgfsetstrokeopacity{0.000000}%
\pgfsetdash{}{0pt}%
\pgfpathmoveto{\pgfqpoint{4.594176in}{0.500000in}}%
\pgfpathlineto{\pgfqpoint{4.627202in}{0.500000in}}%
\pgfpathlineto{\pgfqpoint{4.627202in}{0.500000in}}%
\pgfpathlineto{\pgfqpoint{4.594176in}{0.500000in}}%
\pgfpathlineto{\pgfqpoint{4.594176in}{0.500000in}}%
\pgfpathclose%
\pgfusepath{fill}%
\end{pgfscope}%
\begin{pgfscope}%
\pgfpathrectangle{\pgfqpoint{0.750000in}{0.500000in}}{\pgfqpoint{4.650000in}{3.020000in}}%
\pgfusepath{clip}%
\pgfsetbuttcap%
\pgfsetmiterjoin%
\definecolor{currentfill}{rgb}{1.000000,0.000000,0.000000}%
\pgfsetfillcolor{currentfill}%
\pgfsetlinewidth{0.000000pt}%
\definecolor{currentstroke}{rgb}{0.000000,0.000000,0.000000}%
\pgfsetstrokecolor{currentstroke}%
\pgfsetstrokeopacity{0.000000}%
\pgfsetdash{}{0pt}%
\pgfpathmoveto{\pgfqpoint{4.627202in}{0.500000in}}%
\pgfpathlineto{\pgfqpoint{4.660227in}{0.500000in}}%
\pgfpathlineto{\pgfqpoint{4.660227in}{0.500000in}}%
\pgfpathlineto{\pgfqpoint{4.627202in}{0.500000in}}%
\pgfpathlineto{\pgfqpoint{4.627202in}{0.500000in}}%
\pgfpathclose%
\pgfusepath{fill}%
\end{pgfscope}%
\begin{pgfscope}%
\pgfpathrectangle{\pgfqpoint{0.750000in}{0.500000in}}{\pgfqpoint{4.650000in}{3.020000in}}%
\pgfusepath{clip}%
\pgfsetbuttcap%
\pgfsetmiterjoin%
\definecolor{currentfill}{rgb}{1.000000,0.000000,0.000000}%
\pgfsetfillcolor{currentfill}%
\pgfsetlinewidth{0.000000pt}%
\definecolor{currentstroke}{rgb}{0.000000,0.000000,0.000000}%
\pgfsetstrokecolor{currentstroke}%
\pgfsetstrokeopacity{0.000000}%
\pgfsetdash{}{0pt}%
\pgfpathmoveto{\pgfqpoint{4.660227in}{0.500000in}}%
\pgfpathlineto{\pgfqpoint{4.693253in}{0.500000in}}%
\pgfpathlineto{\pgfqpoint{4.693253in}{0.500000in}}%
\pgfpathlineto{\pgfqpoint{4.660227in}{0.500000in}}%
\pgfpathlineto{\pgfqpoint{4.660227in}{0.500000in}}%
\pgfpathclose%
\pgfusepath{fill}%
\end{pgfscope}%
\begin{pgfscope}%
\pgfpathrectangle{\pgfqpoint{0.750000in}{0.500000in}}{\pgfqpoint{4.650000in}{3.020000in}}%
\pgfusepath{clip}%
\pgfsetbuttcap%
\pgfsetmiterjoin%
\definecolor{currentfill}{rgb}{1.000000,0.000000,0.000000}%
\pgfsetfillcolor{currentfill}%
\pgfsetlinewidth{0.000000pt}%
\definecolor{currentstroke}{rgb}{0.000000,0.000000,0.000000}%
\pgfsetstrokecolor{currentstroke}%
\pgfsetstrokeopacity{0.000000}%
\pgfsetdash{}{0pt}%
\pgfpathmoveto{\pgfqpoint{4.693253in}{0.500000in}}%
\pgfpathlineto{\pgfqpoint{4.726278in}{0.500000in}}%
\pgfpathlineto{\pgfqpoint{4.726278in}{0.500000in}}%
\pgfpathlineto{\pgfqpoint{4.693253in}{0.500000in}}%
\pgfpathlineto{\pgfqpoint{4.693253in}{0.500000in}}%
\pgfpathclose%
\pgfusepath{fill}%
\end{pgfscope}%
\begin{pgfscope}%
\pgfpathrectangle{\pgfqpoint{0.750000in}{0.500000in}}{\pgfqpoint{4.650000in}{3.020000in}}%
\pgfusepath{clip}%
\pgfsetbuttcap%
\pgfsetmiterjoin%
\definecolor{currentfill}{rgb}{1.000000,0.000000,0.000000}%
\pgfsetfillcolor{currentfill}%
\pgfsetlinewidth{0.000000pt}%
\definecolor{currentstroke}{rgb}{0.000000,0.000000,0.000000}%
\pgfsetstrokecolor{currentstroke}%
\pgfsetstrokeopacity{0.000000}%
\pgfsetdash{}{0pt}%
\pgfpathmoveto{\pgfqpoint{4.726278in}{0.500000in}}%
\pgfpathlineto{\pgfqpoint{4.759304in}{0.500000in}}%
\pgfpathlineto{\pgfqpoint{4.759304in}{0.502926in}}%
\pgfpathlineto{\pgfqpoint{4.726278in}{0.502926in}}%
\pgfpathlineto{\pgfqpoint{4.726278in}{0.500000in}}%
\pgfpathclose%
\pgfusepath{fill}%
\end{pgfscope}%
\begin{pgfscope}%
\pgfpathrectangle{\pgfqpoint{0.750000in}{0.500000in}}{\pgfqpoint{4.650000in}{3.020000in}}%
\pgfusepath{clip}%
\pgfsetbuttcap%
\pgfsetmiterjoin%
\definecolor{currentfill}{rgb}{1.000000,0.000000,0.000000}%
\pgfsetfillcolor{currentfill}%
\pgfsetlinewidth{0.000000pt}%
\definecolor{currentstroke}{rgb}{0.000000,0.000000,0.000000}%
\pgfsetstrokecolor{currentstroke}%
\pgfsetstrokeopacity{0.000000}%
\pgfsetdash{}{0pt}%
\pgfpathmoveto{\pgfqpoint{4.759304in}{0.500000in}}%
\pgfpathlineto{\pgfqpoint{4.792330in}{0.500000in}}%
\pgfpathlineto{\pgfqpoint{4.792330in}{0.500000in}}%
\pgfpathlineto{\pgfqpoint{4.759304in}{0.500000in}}%
\pgfpathlineto{\pgfqpoint{4.759304in}{0.500000in}}%
\pgfpathclose%
\pgfusepath{fill}%
\end{pgfscope}%
\begin{pgfscope}%
\pgfpathrectangle{\pgfqpoint{0.750000in}{0.500000in}}{\pgfqpoint{4.650000in}{3.020000in}}%
\pgfusepath{clip}%
\pgfsetbuttcap%
\pgfsetmiterjoin%
\definecolor{currentfill}{rgb}{1.000000,0.000000,0.000000}%
\pgfsetfillcolor{currentfill}%
\pgfsetlinewidth{0.000000pt}%
\definecolor{currentstroke}{rgb}{0.000000,0.000000,0.000000}%
\pgfsetstrokecolor{currentstroke}%
\pgfsetstrokeopacity{0.000000}%
\pgfsetdash{}{0pt}%
\pgfpathmoveto{\pgfqpoint{4.792330in}{0.500000in}}%
\pgfpathlineto{\pgfqpoint{4.825355in}{0.500000in}}%
\pgfpathlineto{\pgfqpoint{4.825355in}{0.500000in}}%
\pgfpathlineto{\pgfqpoint{4.792330in}{0.500000in}}%
\pgfpathlineto{\pgfqpoint{4.792330in}{0.500000in}}%
\pgfpathclose%
\pgfusepath{fill}%
\end{pgfscope}%
\begin{pgfscope}%
\pgfpathrectangle{\pgfqpoint{0.750000in}{0.500000in}}{\pgfqpoint{4.650000in}{3.020000in}}%
\pgfusepath{clip}%
\pgfsetbuttcap%
\pgfsetmiterjoin%
\definecolor{currentfill}{rgb}{1.000000,0.000000,0.000000}%
\pgfsetfillcolor{currentfill}%
\pgfsetlinewidth{0.000000pt}%
\definecolor{currentstroke}{rgb}{0.000000,0.000000,0.000000}%
\pgfsetstrokecolor{currentstroke}%
\pgfsetstrokeopacity{0.000000}%
\pgfsetdash{}{0pt}%
\pgfpathmoveto{\pgfqpoint{4.825355in}{0.500000in}}%
\pgfpathlineto{\pgfqpoint{4.858381in}{0.500000in}}%
\pgfpathlineto{\pgfqpoint{4.858381in}{0.502926in}}%
\pgfpathlineto{\pgfqpoint{4.825355in}{0.502926in}}%
\pgfpathlineto{\pgfqpoint{4.825355in}{0.500000in}}%
\pgfpathclose%
\pgfusepath{fill}%
\end{pgfscope}%
\begin{pgfscope}%
\pgfpathrectangle{\pgfqpoint{0.750000in}{0.500000in}}{\pgfqpoint{4.650000in}{3.020000in}}%
\pgfusepath{clip}%
\pgfsetbuttcap%
\pgfsetmiterjoin%
\definecolor{currentfill}{rgb}{1.000000,0.000000,0.000000}%
\pgfsetfillcolor{currentfill}%
\pgfsetlinewidth{0.000000pt}%
\definecolor{currentstroke}{rgb}{0.000000,0.000000,0.000000}%
\pgfsetstrokecolor{currentstroke}%
\pgfsetstrokeopacity{0.000000}%
\pgfsetdash{}{0pt}%
\pgfpathmoveto{\pgfqpoint{4.858381in}{0.500000in}}%
\pgfpathlineto{\pgfqpoint{4.891406in}{0.500000in}}%
\pgfpathlineto{\pgfqpoint{4.891406in}{0.500000in}}%
\pgfpathlineto{\pgfqpoint{4.858381in}{0.500000in}}%
\pgfpathlineto{\pgfqpoint{4.858381in}{0.500000in}}%
\pgfpathclose%
\pgfusepath{fill}%
\end{pgfscope}%
\begin{pgfscope}%
\pgfpathrectangle{\pgfqpoint{0.750000in}{0.500000in}}{\pgfqpoint{4.650000in}{3.020000in}}%
\pgfusepath{clip}%
\pgfsetbuttcap%
\pgfsetmiterjoin%
\definecolor{currentfill}{rgb}{1.000000,0.000000,0.000000}%
\pgfsetfillcolor{currentfill}%
\pgfsetlinewidth{0.000000pt}%
\definecolor{currentstroke}{rgb}{0.000000,0.000000,0.000000}%
\pgfsetstrokecolor{currentstroke}%
\pgfsetstrokeopacity{0.000000}%
\pgfsetdash{}{0pt}%
\pgfpathmoveto{\pgfqpoint{4.891406in}{0.500000in}}%
\pgfpathlineto{\pgfqpoint{4.924432in}{0.500000in}}%
\pgfpathlineto{\pgfqpoint{4.924432in}{0.500000in}}%
\pgfpathlineto{\pgfqpoint{4.891406in}{0.500000in}}%
\pgfpathlineto{\pgfqpoint{4.891406in}{0.500000in}}%
\pgfpathclose%
\pgfusepath{fill}%
\end{pgfscope}%
\begin{pgfscope}%
\pgfpathrectangle{\pgfqpoint{0.750000in}{0.500000in}}{\pgfqpoint{4.650000in}{3.020000in}}%
\pgfusepath{clip}%
\pgfsetbuttcap%
\pgfsetmiterjoin%
\definecolor{currentfill}{rgb}{1.000000,0.000000,0.000000}%
\pgfsetfillcolor{currentfill}%
\pgfsetlinewidth{0.000000pt}%
\definecolor{currentstroke}{rgb}{0.000000,0.000000,0.000000}%
\pgfsetstrokecolor{currentstroke}%
\pgfsetstrokeopacity{0.000000}%
\pgfsetdash{}{0pt}%
\pgfpathmoveto{\pgfqpoint{4.924432in}{0.500000in}}%
\pgfpathlineto{\pgfqpoint{4.957457in}{0.500000in}}%
\pgfpathlineto{\pgfqpoint{4.957457in}{0.500000in}}%
\pgfpathlineto{\pgfqpoint{4.924432in}{0.500000in}}%
\pgfpathlineto{\pgfqpoint{4.924432in}{0.500000in}}%
\pgfpathclose%
\pgfusepath{fill}%
\end{pgfscope}%
\begin{pgfscope}%
\pgfpathrectangle{\pgfqpoint{0.750000in}{0.500000in}}{\pgfqpoint{4.650000in}{3.020000in}}%
\pgfusepath{clip}%
\pgfsetbuttcap%
\pgfsetmiterjoin%
\definecolor{currentfill}{rgb}{1.000000,0.000000,0.000000}%
\pgfsetfillcolor{currentfill}%
\pgfsetlinewidth{0.000000pt}%
\definecolor{currentstroke}{rgb}{0.000000,0.000000,0.000000}%
\pgfsetstrokecolor{currentstroke}%
\pgfsetstrokeopacity{0.000000}%
\pgfsetdash{}{0pt}%
\pgfpathmoveto{\pgfqpoint{4.957457in}{0.500000in}}%
\pgfpathlineto{\pgfqpoint{4.990483in}{0.500000in}}%
\pgfpathlineto{\pgfqpoint{4.990483in}{0.500000in}}%
\pgfpathlineto{\pgfqpoint{4.957457in}{0.500000in}}%
\pgfpathlineto{\pgfqpoint{4.957457in}{0.500000in}}%
\pgfpathclose%
\pgfusepath{fill}%
\end{pgfscope}%
\begin{pgfscope}%
\pgfpathrectangle{\pgfqpoint{0.750000in}{0.500000in}}{\pgfqpoint{4.650000in}{3.020000in}}%
\pgfusepath{clip}%
\pgfsetbuttcap%
\pgfsetmiterjoin%
\definecolor{currentfill}{rgb}{1.000000,0.000000,0.000000}%
\pgfsetfillcolor{currentfill}%
\pgfsetlinewidth{0.000000pt}%
\definecolor{currentstroke}{rgb}{0.000000,0.000000,0.000000}%
\pgfsetstrokecolor{currentstroke}%
\pgfsetstrokeopacity{0.000000}%
\pgfsetdash{}{0pt}%
\pgfpathmoveto{\pgfqpoint{4.990483in}{0.500000in}}%
\pgfpathlineto{\pgfqpoint{5.023509in}{0.500000in}}%
\pgfpathlineto{\pgfqpoint{5.023509in}{0.500000in}}%
\pgfpathlineto{\pgfqpoint{4.990483in}{0.500000in}}%
\pgfpathlineto{\pgfqpoint{4.990483in}{0.500000in}}%
\pgfpathclose%
\pgfusepath{fill}%
\end{pgfscope}%
\begin{pgfscope}%
\pgfpathrectangle{\pgfqpoint{0.750000in}{0.500000in}}{\pgfqpoint{4.650000in}{3.020000in}}%
\pgfusepath{clip}%
\pgfsetbuttcap%
\pgfsetmiterjoin%
\definecolor{currentfill}{rgb}{1.000000,0.000000,0.000000}%
\pgfsetfillcolor{currentfill}%
\pgfsetlinewidth{0.000000pt}%
\definecolor{currentstroke}{rgb}{0.000000,0.000000,0.000000}%
\pgfsetstrokecolor{currentstroke}%
\pgfsetstrokeopacity{0.000000}%
\pgfsetdash{}{0pt}%
\pgfpathmoveto{\pgfqpoint{5.023509in}{0.500000in}}%
\pgfpathlineto{\pgfqpoint{5.056534in}{0.500000in}}%
\pgfpathlineto{\pgfqpoint{5.056534in}{0.500000in}}%
\pgfpathlineto{\pgfqpoint{5.023509in}{0.500000in}}%
\pgfpathlineto{\pgfqpoint{5.023509in}{0.500000in}}%
\pgfpathclose%
\pgfusepath{fill}%
\end{pgfscope}%
\begin{pgfscope}%
\pgfpathrectangle{\pgfqpoint{0.750000in}{0.500000in}}{\pgfqpoint{4.650000in}{3.020000in}}%
\pgfusepath{clip}%
\pgfsetbuttcap%
\pgfsetmiterjoin%
\definecolor{currentfill}{rgb}{1.000000,0.000000,0.000000}%
\pgfsetfillcolor{currentfill}%
\pgfsetlinewidth{0.000000pt}%
\definecolor{currentstroke}{rgb}{0.000000,0.000000,0.000000}%
\pgfsetstrokecolor{currentstroke}%
\pgfsetstrokeopacity{0.000000}%
\pgfsetdash{}{0pt}%
\pgfpathmoveto{\pgfqpoint{5.056534in}{0.500000in}}%
\pgfpathlineto{\pgfqpoint{5.089560in}{0.500000in}}%
\pgfpathlineto{\pgfqpoint{5.089560in}{0.500000in}}%
\pgfpathlineto{\pgfqpoint{5.056534in}{0.500000in}}%
\pgfpathlineto{\pgfqpoint{5.056534in}{0.500000in}}%
\pgfpathclose%
\pgfusepath{fill}%
\end{pgfscope}%
\begin{pgfscope}%
\pgfpathrectangle{\pgfqpoint{0.750000in}{0.500000in}}{\pgfqpoint{4.650000in}{3.020000in}}%
\pgfusepath{clip}%
\pgfsetbuttcap%
\pgfsetmiterjoin%
\definecolor{currentfill}{rgb}{1.000000,0.000000,0.000000}%
\pgfsetfillcolor{currentfill}%
\pgfsetlinewidth{0.000000pt}%
\definecolor{currentstroke}{rgb}{0.000000,0.000000,0.000000}%
\pgfsetstrokecolor{currentstroke}%
\pgfsetstrokeopacity{0.000000}%
\pgfsetdash{}{0pt}%
\pgfpathmoveto{\pgfqpoint{5.089560in}{0.500000in}}%
\pgfpathlineto{\pgfqpoint{5.122585in}{0.500000in}}%
\pgfpathlineto{\pgfqpoint{5.122585in}{0.500000in}}%
\pgfpathlineto{\pgfqpoint{5.089560in}{0.500000in}}%
\pgfpathlineto{\pgfqpoint{5.089560in}{0.500000in}}%
\pgfpathclose%
\pgfusepath{fill}%
\end{pgfscope}%
\begin{pgfscope}%
\pgfpathrectangle{\pgfqpoint{0.750000in}{0.500000in}}{\pgfqpoint{4.650000in}{3.020000in}}%
\pgfusepath{clip}%
\pgfsetbuttcap%
\pgfsetmiterjoin%
\definecolor{currentfill}{rgb}{1.000000,0.000000,0.000000}%
\pgfsetfillcolor{currentfill}%
\pgfsetlinewidth{0.000000pt}%
\definecolor{currentstroke}{rgb}{0.000000,0.000000,0.000000}%
\pgfsetstrokecolor{currentstroke}%
\pgfsetstrokeopacity{0.000000}%
\pgfsetdash{}{0pt}%
\pgfpathmoveto{\pgfqpoint{5.122585in}{0.500000in}}%
\pgfpathlineto{\pgfqpoint{5.155611in}{0.500000in}}%
\pgfpathlineto{\pgfqpoint{5.155611in}{0.500000in}}%
\pgfpathlineto{\pgfqpoint{5.122585in}{0.500000in}}%
\pgfpathlineto{\pgfqpoint{5.122585in}{0.500000in}}%
\pgfpathclose%
\pgfusepath{fill}%
\end{pgfscope}%
\begin{pgfscope}%
\pgfpathrectangle{\pgfqpoint{0.750000in}{0.500000in}}{\pgfqpoint{4.650000in}{3.020000in}}%
\pgfusepath{clip}%
\pgfsetbuttcap%
\pgfsetmiterjoin%
\definecolor{currentfill}{rgb}{1.000000,0.000000,0.000000}%
\pgfsetfillcolor{currentfill}%
\pgfsetlinewidth{0.000000pt}%
\definecolor{currentstroke}{rgb}{0.000000,0.000000,0.000000}%
\pgfsetstrokecolor{currentstroke}%
\pgfsetstrokeopacity{0.000000}%
\pgfsetdash{}{0pt}%
\pgfpathmoveto{\pgfqpoint{5.155611in}{0.500000in}}%
\pgfpathlineto{\pgfqpoint{5.188636in}{0.500000in}}%
\pgfpathlineto{\pgfqpoint{5.188636in}{0.505852in}}%
\pgfpathlineto{\pgfqpoint{5.155611in}{0.505852in}}%
\pgfpathlineto{\pgfqpoint{5.155611in}{0.500000in}}%
\pgfpathclose%
\pgfusepath{fill}%
\end{pgfscope}%
\begin{pgfscope}%
\pgfsetbuttcap%
\pgfsetroundjoin%
\definecolor{currentfill}{rgb}{0.000000,0.000000,0.000000}%
\pgfsetfillcolor{currentfill}%
\pgfsetlinewidth{0.803000pt}%
\definecolor{currentstroke}{rgb}{0.000000,0.000000,0.000000}%
\pgfsetstrokecolor{currentstroke}%
\pgfsetdash{}{0pt}%
\pgfsys@defobject{currentmarker}{\pgfqpoint{0.000000in}{-0.048611in}}{\pgfqpoint{0.000000in}{0.000000in}}{%
\pgfpathmoveto{\pgfqpoint{0.000000in}{0.000000in}}%
\pgfpathlineto{\pgfqpoint{0.000000in}{-0.048611in}}%
\pgfusepath{stroke,fill}%
}%
\begin{pgfscope}%
\pgfsys@transformshift{1.157551in}{0.500000in}%
\pgfsys@useobject{currentmarker}{}%
\end{pgfscope}%
\end{pgfscope}%
\begin{pgfscope}%
\definecolor{textcolor}{rgb}{0.000000,0.000000,0.000000}%
\pgfsetstrokecolor{textcolor}%
\pgfsetfillcolor{textcolor}%
\pgftext[x=1.157551in,y=0.402778in,,top]{\color{textcolor}\sffamily\fontsize{10.000000}{12.000000}\selectfont 0.15}%
\end{pgfscope}%
\begin{pgfscope}%
\pgfsetbuttcap%
\pgfsetroundjoin%
\definecolor{currentfill}{rgb}{0.000000,0.000000,0.000000}%
\pgfsetfillcolor{currentfill}%
\pgfsetlinewidth{0.803000pt}%
\definecolor{currentstroke}{rgb}{0.000000,0.000000,0.000000}%
\pgfsetstrokecolor{currentstroke}%
\pgfsetdash{}{0pt}%
\pgfsys@defobject{currentmarker}{\pgfqpoint{0.000000in}{-0.048611in}}{\pgfqpoint{0.000000in}{0.000000in}}{%
\pgfpathmoveto{\pgfqpoint{0.000000in}{0.000000in}}%
\pgfpathlineto{\pgfqpoint{0.000000in}{-0.048611in}}%
\pgfusepath{stroke,fill}%
}%
\begin{pgfscope}%
\pgfsys@transformshift{1.728278in}{0.500000in}%
\pgfsys@useobject{currentmarker}{}%
\end{pgfscope}%
\end{pgfscope}%
\begin{pgfscope}%
\definecolor{textcolor}{rgb}{0.000000,0.000000,0.000000}%
\pgfsetstrokecolor{textcolor}%
\pgfsetfillcolor{textcolor}%
\pgftext[x=1.728278in,y=0.402778in,,top]{\color{textcolor}\sffamily\fontsize{10.000000}{12.000000}\selectfont 0.20}%
\end{pgfscope}%
\begin{pgfscope}%
\pgfsetbuttcap%
\pgfsetroundjoin%
\definecolor{currentfill}{rgb}{0.000000,0.000000,0.000000}%
\pgfsetfillcolor{currentfill}%
\pgfsetlinewidth{0.803000pt}%
\definecolor{currentstroke}{rgb}{0.000000,0.000000,0.000000}%
\pgfsetstrokecolor{currentstroke}%
\pgfsetdash{}{0pt}%
\pgfsys@defobject{currentmarker}{\pgfqpoint{0.000000in}{-0.048611in}}{\pgfqpoint{0.000000in}{0.000000in}}{%
\pgfpathmoveto{\pgfqpoint{0.000000in}{0.000000in}}%
\pgfpathlineto{\pgfqpoint{0.000000in}{-0.048611in}}%
\pgfusepath{stroke,fill}%
}%
\begin{pgfscope}%
\pgfsys@transformshift{2.299006in}{0.500000in}%
\pgfsys@useobject{currentmarker}{}%
\end{pgfscope}%
\end{pgfscope}%
\begin{pgfscope}%
\definecolor{textcolor}{rgb}{0.000000,0.000000,0.000000}%
\pgfsetstrokecolor{textcolor}%
\pgfsetfillcolor{textcolor}%
\pgftext[x=2.299006in,y=0.402778in,,top]{\color{textcolor}\sffamily\fontsize{10.000000}{12.000000}\selectfont 0.25}%
\end{pgfscope}%
\begin{pgfscope}%
\pgfsetbuttcap%
\pgfsetroundjoin%
\definecolor{currentfill}{rgb}{0.000000,0.000000,0.000000}%
\pgfsetfillcolor{currentfill}%
\pgfsetlinewidth{0.803000pt}%
\definecolor{currentstroke}{rgb}{0.000000,0.000000,0.000000}%
\pgfsetstrokecolor{currentstroke}%
\pgfsetdash{}{0pt}%
\pgfsys@defobject{currentmarker}{\pgfqpoint{0.000000in}{-0.048611in}}{\pgfqpoint{0.000000in}{0.000000in}}{%
\pgfpathmoveto{\pgfqpoint{0.000000in}{0.000000in}}%
\pgfpathlineto{\pgfqpoint{0.000000in}{-0.048611in}}%
\pgfusepath{stroke,fill}%
}%
\begin{pgfscope}%
\pgfsys@transformshift{2.869733in}{0.500000in}%
\pgfsys@useobject{currentmarker}{}%
\end{pgfscope}%
\end{pgfscope}%
\begin{pgfscope}%
\definecolor{textcolor}{rgb}{0.000000,0.000000,0.000000}%
\pgfsetstrokecolor{textcolor}%
\pgfsetfillcolor{textcolor}%
\pgftext[x=2.869733in,y=0.402778in,,top]{\color{textcolor}\sffamily\fontsize{10.000000}{12.000000}\selectfont 0.30}%
\end{pgfscope}%
\begin{pgfscope}%
\pgfsetbuttcap%
\pgfsetroundjoin%
\definecolor{currentfill}{rgb}{0.000000,0.000000,0.000000}%
\pgfsetfillcolor{currentfill}%
\pgfsetlinewidth{0.803000pt}%
\definecolor{currentstroke}{rgb}{0.000000,0.000000,0.000000}%
\pgfsetstrokecolor{currentstroke}%
\pgfsetdash{}{0pt}%
\pgfsys@defobject{currentmarker}{\pgfqpoint{0.000000in}{-0.048611in}}{\pgfqpoint{0.000000in}{0.000000in}}{%
\pgfpathmoveto{\pgfqpoint{0.000000in}{0.000000in}}%
\pgfpathlineto{\pgfqpoint{0.000000in}{-0.048611in}}%
\pgfusepath{stroke,fill}%
}%
\begin{pgfscope}%
\pgfsys@transformshift{3.440460in}{0.500000in}%
\pgfsys@useobject{currentmarker}{}%
\end{pgfscope}%
\end{pgfscope}%
\begin{pgfscope}%
\definecolor{textcolor}{rgb}{0.000000,0.000000,0.000000}%
\pgfsetstrokecolor{textcolor}%
\pgfsetfillcolor{textcolor}%
\pgftext[x=3.440460in,y=0.402778in,,top]{\color{textcolor}\sffamily\fontsize{10.000000}{12.000000}\selectfont 0.35}%
\end{pgfscope}%
\begin{pgfscope}%
\pgfsetbuttcap%
\pgfsetroundjoin%
\definecolor{currentfill}{rgb}{0.000000,0.000000,0.000000}%
\pgfsetfillcolor{currentfill}%
\pgfsetlinewidth{0.803000pt}%
\definecolor{currentstroke}{rgb}{0.000000,0.000000,0.000000}%
\pgfsetstrokecolor{currentstroke}%
\pgfsetdash{}{0pt}%
\pgfsys@defobject{currentmarker}{\pgfqpoint{0.000000in}{-0.048611in}}{\pgfqpoint{0.000000in}{0.000000in}}{%
\pgfpathmoveto{\pgfqpoint{0.000000in}{0.000000in}}%
\pgfpathlineto{\pgfqpoint{0.000000in}{-0.048611in}}%
\pgfusepath{stroke,fill}%
}%
\begin{pgfscope}%
\pgfsys@transformshift{4.011187in}{0.500000in}%
\pgfsys@useobject{currentmarker}{}%
\end{pgfscope}%
\end{pgfscope}%
\begin{pgfscope}%
\definecolor{textcolor}{rgb}{0.000000,0.000000,0.000000}%
\pgfsetstrokecolor{textcolor}%
\pgfsetfillcolor{textcolor}%
\pgftext[x=4.011187in,y=0.402778in,,top]{\color{textcolor}\sffamily\fontsize{10.000000}{12.000000}\selectfont 0.40}%
\end{pgfscope}%
\begin{pgfscope}%
\pgfsetbuttcap%
\pgfsetroundjoin%
\definecolor{currentfill}{rgb}{0.000000,0.000000,0.000000}%
\pgfsetfillcolor{currentfill}%
\pgfsetlinewidth{0.803000pt}%
\definecolor{currentstroke}{rgb}{0.000000,0.000000,0.000000}%
\pgfsetstrokecolor{currentstroke}%
\pgfsetdash{}{0pt}%
\pgfsys@defobject{currentmarker}{\pgfqpoint{0.000000in}{-0.048611in}}{\pgfqpoint{0.000000in}{0.000000in}}{%
\pgfpathmoveto{\pgfqpoint{0.000000in}{0.000000in}}%
\pgfpathlineto{\pgfqpoint{0.000000in}{-0.048611in}}%
\pgfusepath{stroke,fill}%
}%
\begin{pgfscope}%
\pgfsys@transformshift{4.581915in}{0.500000in}%
\pgfsys@useobject{currentmarker}{}%
\end{pgfscope}%
\end{pgfscope}%
\begin{pgfscope}%
\definecolor{textcolor}{rgb}{0.000000,0.000000,0.000000}%
\pgfsetstrokecolor{textcolor}%
\pgfsetfillcolor{textcolor}%
\pgftext[x=4.581915in,y=0.402778in,,top]{\color{textcolor}\sffamily\fontsize{10.000000}{12.000000}\selectfont 0.45}%
\end{pgfscope}%
\begin{pgfscope}%
\pgfsetbuttcap%
\pgfsetroundjoin%
\definecolor{currentfill}{rgb}{0.000000,0.000000,0.000000}%
\pgfsetfillcolor{currentfill}%
\pgfsetlinewidth{0.803000pt}%
\definecolor{currentstroke}{rgb}{0.000000,0.000000,0.000000}%
\pgfsetstrokecolor{currentstroke}%
\pgfsetdash{}{0pt}%
\pgfsys@defobject{currentmarker}{\pgfqpoint{0.000000in}{-0.048611in}}{\pgfqpoint{0.000000in}{0.000000in}}{%
\pgfpathmoveto{\pgfqpoint{0.000000in}{0.000000in}}%
\pgfpathlineto{\pgfqpoint{0.000000in}{-0.048611in}}%
\pgfusepath{stroke,fill}%
}%
\begin{pgfscope}%
\pgfsys@transformshift{5.152642in}{0.500000in}%
\pgfsys@useobject{currentmarker}{}%
\end{pgfscope}%
\end{pgfscope}%
\begin{pgfscope}%
\definecolor{textcolor}{rgb}{0.000000,0.000000,0.000000}%
\pgfsetstrokecolor{textcolor}%
\pgfsetfillcolor{textcolor}%
\pgftext[x=5.152642in,y=0.402778in,,top]{\color{textcolor}\sffamily\fontsize{10.000000}{12.000000}\selectfont 0.50}%
\end{pgfscope}%
\begin{pgfscope}%
\definecolor{textcolor}{rgb}{0.000000,0.000000,0.000000}%
\pgfsetstrokecolor{textcolor}%
\pgfsetfillcolor{textcolor}%
\pgftext[x=3.075000in,y=0.212809in,,top]{\color{textcolor}\sffamily\fontsize{10.000000}{12.000000}\selectfont Loss}%
\end{pgfscope}%
\begin{pgfscope}%
\pgfsetbuttcap%
\pgfsetroundjoin%
\definecolor{currentfill}{rgb}{0.000000,0.000000,0.000000}%
\pgfsetfillcolor{currentfill}%
\pgfsetlinewidth{0.803000pt}%
\definecolor{currentstroke}{rgb}{0.000000,0.000000,0.000000}%
\pgfsetstrokecolor{currentstroke}%
\pgfsetdash{}{0pt}%
\pgfsys@defobject{currentmarker}{\pgfqpoint{-0.048611in}{0.000000in}}{\pgfqpoint{-0.000000in}{0.000000in}}{%
\pgfpathmoveto{\pgfqpoint{-0.000000in}{0.000000in}}%
\pgfpathlineto{\pgfqpoint{-0.048611in}{0.000000in}}%
\pgfusepath{stroke,fill}%
}%
\begin{pgfscope}%
\pgfsys@transformshift{0.750000in}{0.500000in}%
\pgfsys@useobject{currentmarker}{}%
\end{pgfscope}%
\end{pgfscope}%
\begin{pgfscope}%
\definecolor{textcolor}{rgb}{0.000000,0.000000,0.000000}%
\pgfsetstrokecolor{textcolor}%
\pgfsetfillcolor{textcolor}%
\pgftext[x=0.564412in, y=0.447238in, left, base]{\color{textcolor}\sffamily\fontsize{10.000000}{12.000000}\selectfont 0}%
\end{pgfscope}%
\begin{pgfscope}%
\pgfsetbuttcap%
\pgfsetroundjoin%
\definecolor{currentfill}{rgb}{0.000000,0.000000,0.000000}%
\pgfsetfillcolor{currentfill}%
\pgfsetlinewidth{0.803000pt}%
\definecolor{currentstroke}{rgb}{0.000000,0.000000,0.000000}%
\pgfsetstrokecolor{currentstroke}%
\pgfsetdash{}{0pt}%
\pgfsys@defobject{currentmarker}{\pgfqpoint{-0.048611in}{0.000000in}}{\pgfqpoint{-0.000000in}{0.000000in}}{%
\pgfpathmoveto{\pgfqpoint{-0.000000in}{0.000000in}}%
\pgfpathlineto{\pgfqpoint{-0.048611in}{0.000000in}}%
\pgfusepath{stroke,fill}%
}%
\begin{pgfscope}%
\pgfsys@transformshift{0.750000in}{1.085186in}%
\pgfsys@useobject{currentmarker}{}%
\end{pgfscope}%
\end{pgfscope}%
\begin{pgfscope}%
\definecolor{textcolor}{rgb}{0.000000,0.000000,0.000000}%
\pgfsetstrokecolor{textcolor}%
\pgfsetfillcolor{textcolor}%
\pgftext[x=0.387682in, y=1.032425in, left, base]{\color{textcolor}\sffamily\fontsize{10.000000}{12.000000}\selectfont 200}%
\end{pgfscope}%
\begin{pgfscope}%
\pgfsetbuttcap%
\pgfsetroundjoin%
\definecolor{currentfill}{rgb}{0.000000,0.000000,0.000000}%
\pgfsetfillcolor{currentfill}%
\pgfsetlinewidth{0.803000pt}%
\definecolor{currentstroke}{rgb}{0.000000,0.000000,0.000000}%
\pgfsetstrokecolor{currentstroke}%
\pgfsetdash{}{0pt}%
\pgfsys@defobject{currentmarker}{\pgfqpoint{-0.048611in}{0.000000in}}{\pgfqpoint{-0.000000in}{0.000000in}}{%
\pgfpathmoveto{\pgfqpoint{-0.000000in}{0.000000in}}%
\pgfpathlineto{\pgfqpoint{-0.048611in}{0.000000in}}%
\pgfusepath{stroke,fill}%
}%
\begin{pgfscope}%
\pgfsys@transformshift{0.750000in}{1.670373in}%
\pgfsys@useobject{currentmarker}{}%
\end{pgfscope}%
\end{pgfscope}%
\begin{pgfscope}%
\definecolor{textcolor}{rgb}{0.000000,0.000000,0.000000}%
\pgfsetstrokecolor{textcolor}%
\pgfsetfillcolor{textcolor}%
\pgftext[x=0.387682in, y=1.617611in, left, base]{\color{textcolor}\sffamily\fontsize{10.000000}{12.000000}\selectfont 400}%
\end{pgfscope}%
\begin{pgfscope}%
\pgfsetbuttcap%
\pgfsetroundjoin%
\definecolor{currentfill}{rgb}{0.000000,0.000000,0.000000}%
\pgfsetfillcolor{currentfill}%
\pgfsetlinewidth{0.803000pt}%
\definecolor{currentstroke}{rgb}{0.000000,0.000000,0.000000}%
\pgfsetstrokecolor{currentstroke}%
\pgfsetdash{}{0pt}%
\pgfsys@defobject{currentmarker}{\pgfqpoint{-0.048611in}{0.000000in}}{\pgfqpoint{-0.000000in}{0.000000in}}{%
\pgfpathmoveto{\pgfqpoint{-0.000000in}{0.000000in}}%
\pgfpathlineto{\pgfqpoint{-0.048611in}{0.000000in}}%
\pgfusepath{stroke,fill}%
}%
\begin{pgfscope}%
\pgfsys@transformshift{0.750000in}{2.255559in}%
\pgfsys@useobject{currentmarker}{}%
\end{pgfscope}%
\end{pgfscope}%
\begin{pgfscope}%
\definecolor{textcolor}{rgb}{0.000000,0.000000,0.000000}%
\pgfsetstrokecolor{textcolor}%
\pgfsetfillcolor{textcolor}%
\pgftext[x=0.387682in, y=2.202797in, left, base]{\color{textcolor}\sffamily\fontsize{10.000000}{12.000000}\selectfont 600}%
\end{pgfscope}%
\begin{pgfscope}%
\pgfsetbuttcap%
\pgfsetroundjoin%
\definecolor{currentfill}{rgb}{0.000000,0.000000,0.000000}%
\pgfsetfillcolor{currentfill}%
\pgfsetlinewidth{0.803000pt}%
\definecolor{currentstroke}{rgb}{0.000000,0.000000,0.000000}%
\pgfsetstrokecolor{currentstroke}%
\pgfsetdash{}{0pt}%
\pgfsys@defobject{currentmarker}{\pgfqpoint{-0.048611in}{0.000000in}}{\pgfqpoint{-0.000000in}{0.000000in}}{%
\pgfpathmoveto{\pgfqpoint{-0.000000in}{0.000000in}}%
\pgfpathlineto{\pgfqpoint{-0.048611in}{0.000000in}}%
\pgfusepath{stroke,fill}%
}%
\begin{pgfscope}%
\pgfsys@transformshift{0.750000in}{2.840745in}%
\pgfsys@useobject{currentmarker}{}%
\end{pgfscope}%
\end{pgfscope}%
\begin{pgfscope}%
\definecolor{textcolor}{rgb}{0.000000,0.000000,0.000000}%
\pgfsetstrokecolor{textcolor}%
\pgfsetfillcolor{textcolor}%
\pgftext[x=0.387682in, y=2.787984in, left, base]{\color{textcolor}\sffamily\fontsize{10.000000}{12.000000}\selectfont 800}%
\end{pgfscope}%
\begin{pgfscope}%
\pgfsetbuttcap%
\pgfsetroundjoin%
\definecolor{currentfill}{rgb}{0.000000,0.000000,0.000000}%
\pgfsetfillcolor{currentfill}%
\pgfsetlinewidth{0.803000pt}%
\definecolor{currentstroke}{rgb}{0.000000,0.000000,0.000000}%
\pgfsetstrokecolor{currentstroke}%
\pgfsetdash{}{0pt}%
\pgfsys@defobject{currentmarker}{\pgfqpoint{-0.048611in}{0.000000in}}{\pgfqpoint{-0.000000in}{0.000000in}}{%
\pgfpathmoveto{\pgfqpoint{-0.000000in}{0.000000in}}%
\pgfpathlineto{\pgfqpoint{-0.048611in}{0.000000in}}%
\pgfusepath{stroke,fill}%
}%
\begin{pgfscope}%
\pgfsys@transformshift{0.750000in}{3.425931in}%
\pgfsys@useobject{currentmarker}{}%
\end{pgfscope}%
\end{pgfscope}%
\begin{pgfscope}%
\definecolor{textcolor}{rgb}{0.000000,0.000000,0.000000}%
\pgfsetstrokecolor{textcolor}%
\pgfsetfillcolor{textcolor}%
\pgftext[x=0.299316in, y=3.373170in, left, base]{\color{textcolor}\sffamily\fontsize{10.000000}{12.000000}\selectfont 1000}%
\end{pgfscope}%
\begin{pgfscope}%
\definecolor{textcolor}{rgb}{0.000000,0.000000,0.000000}%
\pgfsetstrokecolor{textcolor}%
\pgfsetfillcolor{textcolor}%
\pgftext[x=0.243761in,y=2.010000in,,bottom,rotate=90.000000]{\color{textcolor}\sffamily\fontsize{10.000000}{12.000000}\selectfont Count}%
\end{pgfscope}%
\begin{pgfscope}%
\pgfsetrectcap%
\pgfsetmiterjoin%
\pgfsetlinewidth{0.803000pt}%
\definecolor{currentstroke}{rgb}{0.000000,0.000000,0.000000}%
\pgfsetstrokecolor{currentstroke}%
\pgfsetdash{}{0pt}%
\pgfpathmoveto{\pgfqpoint{0.750000in}{0.500000in}}%
\pgfpathlineto{\pgfqpoint{0.750000in}{3.520000in}}%
\pgfusepath{stroke}%
\end{pgfscope}%
\begin{pgfscope}%
\pgfsetrectcap%
\pgfsetmiterjoin%
\pgfsetlinewidth{0.803000pt}%
\definecolor{currentstroke}{rgb}{0.000000,0.000000,0.000000}%
\pgfsetstrokecolor{currentstroke}%
\pgfsetdash{}{0pt}%
\pgfpathmoveto{\pgfqpoint{5.400000in}{0.500000in}}%
\pgfpathlineto{\pgfqpoint{5.400000in}{3.520000in}}%
\pgfusepath{stroke}%
\end{pgfscope}%
\begin{pgfscope}%
\pgfsetrectcap%
\pgfsetmiterjoin%
\pgfsetlinewidth{0.803000pt}%
\definecolor{currentstroke}{rgb}{0.000000,0.000000,0.000000}%
\pgfsetstrokecolor{currentstroke}%
\pgfsetdash{}{0pt}%
\pgfpathmoveto{\pgfqpoint{0.750000in}{0.500000in}}%
\pgfpathlineto{\pgfqpoint{5.400000in}{0.500000in}}%
\pgfusepath{stroke}%
\end{pgfscope}%
\begin{pgfscope}%
\pgfsetrectcap%
\pgfsetmiterjoin%
\pgfsetlinewidth{0.803000pt}%
\definecolor{currentstroke}{rgb}{0.000000,0.000000,0.000000}%
\pgfsetstrokecolor{currentstroke}%
\pgfsetdash{}{0pt}%
\pgfpathmoveto{\pgfqpoint{0.750000in}{3.520000in}}%
\pgfpathlineto{\pgfqpoint{5.400000in}{3.520000in}}%
\pgfusepath{stroke}%
\end{pgfscope}%
\begin{pgfscope}%
\definecolor{textcolor}{rgb}{0.000000,0.000000,0.000000}%
\pgfsetstrokecolor{textcolor}%
\pgfsetfillcolor{textcolor}%
\pgftext[x=3.075000in,y=3.603333in,,base]{\color{textcolor}\sffamily\fontsize{12.000000}{14.400000}\selectfont Loss Histogram for \(\displaystyle f(x)=x^2\)}%
\end{pgfscope}%
\begin{pgfscope}%
\pgfsetbuttcap%
\pgfsetmiterjoin%
\definecolor{currentfill}{rgb}{1.000000,1.000000,1.000000}%
\pgfsetfillcolor{currentfill}%
\pgfsetfillopacity{0.800000}%
\pgfsetlinewidth{1.003750pt}%
\definecolor{currentstroke}{rgb}{0.800000,0.800000,0.800000}%
\pgfsetstrokecolor{currentstroke}%
\pgfsetstrokeopacity{0.800000}%
\pgfsetdash{}{0pt}%
\pgfpathmoveto{\pgfqpoint{4.562381in}{3.205032in}}%
\pgfpathlineto{\pgfqpoint{5.302778in}{3.205032in}}%
\pgfpathquadraticcurveto{\pgfqpoint{5.330556in}{3.205032in}}{\pgfqpoint{5.330556in}{3.232809in}}%
\pgfpathlineto{\pgfqpoint{5.330556in}{3.422778in}}%
\pgfpathquadraticcurveto{\pgfqpoint{5.330556in}{3.450556in}}{\pgfqpoint{5.302778in}{3.450556in}}%
\pgfpathlineto{\pgfqpoint{4.562381in}{3.450556in}}%
\pgfpathquadraticcurveto{\pgfqpoint{4.534603in}{3.450556in}}{\pgfqpoint{4.534603in}{3.422778in}}%
\pgfpathlineto{\pgfqpoint{4.534603in}{3.232809in}}%
\pgfpathquadraticcurveto{\pgfqpoint{4.534603in}{3.205032in}}{\pgfqpoint{4.562381in}{3.205032in}}%
\pgfpathlineto{\pgfqpoint{4.562381in}{3.205032in}}%
\pgfpathclose%
\pgfusepath{stroke,fill}%
\end{pgfscope}%
\begin{pgfscope}%
\pgfsetbuttcap%
\pgfsetmiterjoin%
\definecolor{currentfill}{rgb}{1.000000,0.000000,0.000000}%
\pgfsetfillcolor{currentfill}%
\pgfsetlinewidth{0.000000pt}%
\definecolor{currentstroke}{rgb}{0.000000,0.000000,0.000000}%
\pgfsetstrokecolor{currentstroke}%
\pgfsetstrokeopacity{0.000000}%
\pgfsetdash{}{0pt}%
\pgfpathmoveto{\pgfqpoint{4.590158in}{3.289477in}}%
\pgfpathlineto{\pgfqpoint{4.867936in}{3.289477in}}%
\pgfpathlineto{\pgfqpoint{4.867936in}{3.386699in}}%
\pgfpathlineto{\pgfqpoint{4.590158in}{3.386699in}}%
\pgfpathlineto{\pgfqpoint{4.590158in}{3.289477in}}%
\pgfpathclose%
\pgfusepath{fill}%
\end{pgfscope}%
\begin{pgfscope}%
\definecolor{textcolor}{rgb}{0.000000,0.000000,0.000000}%
\pgfsetstrokecolor{textcolor}%
\pgfsetfillcolor{textcolor}%
\pgftext[x=4.979047in,y=3.289477in,left,base]{\color{textcolor}\sffamily\fontsize{10.000000}{12.000000}\selectfont SNN}%
\end{pgfscope}%
\end{pgfpicture}%
\makeatother%
\endgroup%

    \caption{Caption}
    \label{fig:my_label}
\end{figure}

\begin{figure}
%% Creator: Matplotlib, PGF backend
%%
%% To include the figure in your LaTeX document, write
%%   \input{<filename>.pgf}
%%
%% Make sure the required packages are loaded in your preamble
%%   \usepackage{pgf}
%%
%% Also ensure that all the required font packages are loaded; for instance,
%% the lmodern package is sometimes necessary when using math font.
%%   \usepackage{lmodern}
%%
%% Figures using additional raster images can only be included by \input if
%% they are in the same directory as the main LaTeX file. For loading figures
%% from other directories you can use the `import` package
%%   \usepackage{import}
%%
%% and then include the figures with
%%   \import{<path to file>}{<filename>.pgf}
%%
%% Matplotlib used the following preamble
%%   \usepackage{fontspec}
%%   \setmainfont{DejaVuSerif.ttf}[Path=\detokenize{/Users/mkojro/miniforge3/envs/nn-crypto/lib/python3.10/site-packages/matplotlib/mpl-data/fonts/ttf/}]
%%   \setsansfont{DejaVuSans.ttf}[Path=\detokenize{/Users/mkojro/miniforge3/envs/nn-crypto/lib/python3.10/site-packages/matplotlib/mpl-data/fonts/ttf/}]
%%   \setmonofont{DejaVuSansMono.ttf}[Path=\detokenize{/Users/mkojro/miniforge3/envs/nn-crypto/lib/python3.10/site-packages/matplotlib/mpl-data/fonts/ttf/}]
%%
\begingroup%
\makeatletter%
\begin{pgfpicture}%
\pgfpathrectangle{\pgfpointorigin}{\pgfqpoint{6.000000in}{4.000000in}}%
\pgfusepath{use as bounding box, clip}%
\begin{pgfscope}%
\pgfsetbuttcap%
\pgfsetmiterjoin%
\pgfsetlinewidth{0.000000pt}%
\definecolor{currentstroke}{rgb}{1.000000,1.000000,1.000000}%
\pgfsetstrokecolor{currentstroke}%
\pgfsetstrokeopacity{0.000000}%
\pgfsetdash{}{0pt}%
\pgfpathmoveto{\pgfqpoint{0.000000in}{0.000000in}}%
\pgfpathlineto{\pgfqpoint{6.000000in}{0.000000in}}%
\pgfpathlineto{\pgfqpoint{6.000000in}{4.000000in}}%
\pgfpathlineto{\pgfqpoint{0.000000in}{4.000000in}}%
\pgfpathlineto{\pgfqpoint{0.000000in}{0.000000in}}%
\pgfpathclose%
\pgfusepath{}%
\end{pgfscope}%
\begin{pgfscope}%
\pgfsetbuttcap%
\pgfsetmiterjoin%
\definecolor{currentfill}{rgb}{1.000000,1.000000,1.000000}%
\pgfsetfillcolor{currentfill}%
\pgfsetlinewidth{0.000000pt}%
\definecolor{currentstroke}{rgb}{0.000000,0.000000,0.000000}%
\pgfsetstrokecolor{currentstroke}%
\pgfsetstrokeopacity{0.000000}%
\pgfsetdash{}{0pt}%
\pgfpathmoveto{\pgfqpoint{0.750000in}{0.500000in}}%
\pgfpathlineto{\pgfqpoint{5.400000in}{0.500000in}}%
\pgfpathlineto{\pgfqpoint{5.400000in}{3.520000in}}%
\pgfpathlineto{\pgfqpoint{0.750000in}{3.520000in}}%
\pgfpathlineto{\pgfqpoint{0.750000in}{0.500000in}}%
\pgfpathclose%
\pgfusepath{fill}%
\end{pgfscope}%
\begin{pgfscope}%
\pgfpathrectangle{\pgfqpoint{0.750000in}{0.500000in}}{\pgfqpoint{4.650000in}{3.020000in}}%
\pgfusepath{clip}%
\pgfsetbuttcap%
\pgfsetmiterjoin%
\definecolor{currentfill}{rgb}{0.000000,0.500000,0.000000}%
\pgfsetfillcolor{currentfill}%
\pgfsetlinewidth{0.000000pt}%
\definecolor{currentstroke}{rgb}{0.000000,0.000000,0.000000}%
\pgfsetstrokecolor{currentstroke}%
\pgfsetstrokeopacity{0.000000}%
\pgfsetdash{}{0pt}%
\pgfpathmoveto{\pgfqpoint{0.961364in}{0.500000in}}%
\pgfpathlineto{\pgfqpoint{0.994389in}{0.500000in}}%
\pgfpathlineto{\pgfqpoint{0.994389in}{0.505552in}}%
\pgfpathlineto{\pgfqpoint{0.961364in}{0.505552in}}%
\pgfpathlineto{\pgfqpoint{0.961364in}{0.500000in}}%
\pgfpathclose%
\pgfusepath{fill}%
\end{pgfscope}%
\begin{pgfscope}%
\pgfpathrectangle{\pgfqpoint{0.750000in}{0.500000in}}{\pgfqpoint{4.650000in}{3.020000in}}%
\pgfusepath{clip}%
\pgfsetbuttcap%
\pgfsetmiterjoin%
\definecolor{currentfill}{rgb}{0.000000,0.500000,0.000000}%
\pgfsetfillcolor{currentfill}%
\pgfsetlinewidth{0.000000pt}%
\definecolor{currentstroke}{rgb}{0.000000,0.000000,0.000000}%
\pgfsetstrokecolor{currentstroke}%
\pgfsetstrokeopacity{0.000000}%
\pgfsetdash{}{0pt}%
\pgfpathmoveto{\pgfqpoint{0.994389in}{0.500000in}}%
\pgfpathlineto{\pgfqpoint{1.027415in}{0.500000in}}%
\pgfpathlineto{\pgfqpoint{1.027415in}{0.500000in}}%
\pgfpathlineto{\pgfqpoint{0.994389in}{0.500000in}}%
\pgfpathlineto{\pgfqpoint{0.994389in}{0.500000in}}%
\pgfpathclose%
\pgfusepath{fill}%
\end{pgfscope}%
\begin{pgfscope}%
\pgfpathrectangle{\pgfqpoint{0.750000in}{0.500000in}}{\pgfqpoint{4.650000in}{3.020000in}}%
\pgfusepath{clip}%
\pgfsetbuttcap%
\pgfsetmiterjoin%
\definecolor{currentfill}{rgb}{0.000000,0.500000,0.000000}%
\pgfsetfillcolor{currentfill}%
\pgfsetlinewidth{0.000000pt}%
\definecolor{currentstroke}{rgb}{0.000000,0.000000,0.000000}%
\pgfsetstrokecolor{currentstroke}%
\pgfsetstrokeopacity{0.000000}%
\pgfsetdash{}{0pt}%
\pgfpathmoveto{\pgfqpoint{1.027415in}{0.500000in}}%
\pgfpathlineto{\pgfqpoint{1.060440in}{0.500000in}}%
\pgfpathlineto{\pgfqpoint{1.060440in}{0.500000in}}%
\pgfpathlineto{\pgfqpoint{1.027415in}{0.500000in}}%
\pgfpathlineto{\pgfqpoint{1.027415in}{0.500000in}}%
\pgfpathclose%
\pgfusepath{fill}%
\end{pgfscope}%
\begin{pgfscope}%
\pgfpathrectangle{\pgfqpoint{0.750000in}{0.500000in}}{\pgfqpoint{4.650000in}{3.020000in}}%
\pgfusepath{clip}%
\pgfsetbuttcap%
\pgfsetmiterjoin%
\definecolor{currentfill}{rgb}{0.000000,0.500000,0.000000}%
\pgfsetfillcolor{currentfill}%
\pgfsetlinewidth{0.000000pt}%
\definecolor{currentstroke}{rgb}{0.000000,0.000000,0.000000}%
\pgfsetstrokecolor{currentstroke}%
\pgfsetstrokeopacity{0.000000}%
\pgfsetdash{}{0pt}%
\pgfpathmoveto{\pgfqpoint{1.060440in}{0.500000in}}%
\pgfpathlineto{\pgfqpoint{1.093466in}{0.500000in}}%
\pgfpathlineto{\pgfqpoint{1.093466in}{0.505552in}}%
\pgfpathlineto{\pgfqpoint{1.060440in}{0.505552in}}%
\pgfpathlineto{\pgfqpoint{1.060440in}{0.500000in}}%
\pgfpathclose%
\pgfusepath{fill}%
\end{pgfscope}%
\begin{pgfscope}%
\pgfpathrectangle{\pgfqpoint{0.750000in}{0.500000in}}{\pgfqpoint{4.650000in}{3.020000in}}%
\pgfusepath{clip}%
\pgfsetbuttcap%
\pgfsetmiterjoin%
\definecolor{currentfill}{rgb}{0.000000,0.500000,0.000000}%
\pgfsetfillcolor{currentfill}%
\pgfsetlinewidth{0.000000pt}%
\definecolor{currentstroke}{rgb}{0.000000,0.000000,0.000000}%
\pgfsetstrokecolor{currentstroke}%
\pgfsetstrokeopacity{0.000000}%
\pgfsetdash{}{0pt}%
\pgfpathmoveto{\pgfqpoint{1.093466in}{0.500000in}}%
\pgfpathlineto{\pgfqpoint{1.126491in}{0.500000in}}%
\pgfpathlineto{\pgfqpoint{1.126491in}{0.511105in}}%
\pgfpathlineto{\pgfqpoint{1.093466in}{0.511105in}}%
\pgfpathlineto{\pgfqpoint{1.093466in}{0.500000in}}%
\pgfpathclose%
\pgfusepath{fill}%
\end{pgfscope}%
\begin{pgfscope}%
\pgfpathrectangle{\pgfqpoint{0.750000in}{0.500000in}}{\pgfqpoint{4.650000in}{3.020000in}}%
\pgfusepath{clip}%
\pgfsetbuttcap%
\pgfsetmiterjoin%
\definecolor{currentfill}{rgb}{0.000000,0.500000,0.000000}%
\pgfsetfillcolor{currentfill}%
\pgfsetlinewidth{0.000000pt}%
\definecolor{currentstroke}{rgb}{0.000000,0.000000,0.000000}%
\pgfsetstrokecolor{currentstroke}%
\pgfsetstrokeopacity{0.000000}%
\pgfsetdash{}{0pt}%
\pgfpathmoveto{\pgfqpoint{1.126491in}{0.500000in}}%
\pgfpathlineto{\pgfqpoint{1.159517in}{0.500000in}}%
\pgfpathlineto{\pgfqpoint{1.159517in}{0.505552in}}%
\pgfpathlineto{\pgfqpoint{1.126491in}{0.505552in}}%
\pgfpathlineto{\pgfqpoint{1.126491in}{0.500000in}}%
\pgfpathclose%
\pgfusepath{fill}%
\end{pgfscope}%
\begin{pgfscope}%
\pgfpathrectangle{\pgfqpoint{0.750000in}{0.500000in}}{\pgfqpoint{4.650000in}{3.020000in}}%
\pgfusepath{clip}%
\pgfsetbuttcap%
\pgfsetmiterjoin%
\definecolor{currentfill}{rgb}{0.000000,0.500000,0.000000}%
\pgfsetfillcolor{currentfill}%
\pgfsetlinewidth{0.000000pt}%
\definecolor{currentstroke}{rgb}{0.000000,0.000000,0.000000}%
\pgfsetstrokecolor{currentstroke}%
\pgfsetstrokeopacity{0.000000}%
\pgfsetdash{}{0pt}%
\pgfpathmoveto{\pgfqpoint{1.159517in}{0.500000in}}%
\pgfpathlineto{\pgfqpoint{1.192543in}{0.500000in}}%
\pgfpathlineto{\pgfqpoint{1.192543in}{0.511105in}}%
\pgfpathlineto{\pgfqpoint{1.159517in}{0.511105in}}%
\pgfpathlineto{\pgfqpoint{1.159517in}{0.500000in}}%
\pgfpathclose%
\pgfusepath{fill}%
\end{pgfscope}%
\begin{pgfscope}%
\pgfpathrectangle{\pgfqpoint{0.750000in}{0.500000in}}{\pgfqpoint{4.650000in}{3.020000in}}%
\pgfusepath{clip}%
\pgfsetbuttcap%
\pgfsetmiterjoin%
\definecolor{currentfill}{rgb}{0.000000,0.500000,0.000000}%
\pgfsetfillcolor{currentfill}%
\pgfsetlinewidth{0.000000pt}%
\definecolor{currentstroke}{rgb}{0.000000,0.000000,0.000000}%
\pgfsetstrokecolor{currentstroke}%
\pgfsetstrokeopacity{0.000000}%
\pgfsetdash{}{0pt}%
\pgfpathmoveto{\pgfqpoint{1.192543in}{0.500000in}}%
\pgfpathlineto{\pgfqpoint{1.225568in}{0.500000in}}%
\pgfpathlineto{\pgfqpoint{1.225568in}{0.516657in}}%
\pgfpathlineto{\pgfqpoint{1.192543in}{0.516657in}}%
\pgfpathlineto{\pgfqpoint{1.192543in}{0.500000in}}%
\pgfpathclose%
\pgfusepath{fill}%
\end{pgfscope}%
\begin{pgfscope}%
\pgfpathrectangle{\pgfqpoint{0.750000in}{0.500000in}}{\pgfqpoint{4.650000in}{3.020000in}}%
\pgfusepath{clip}%
\pgfsetbuttcap%
\pgfsetmiterjoin%
\definecolor{currentfill}{rgb}{0.000000,0.500000,0.000000}%
\pgfsetfillcolor{currentfill}%
\pgfsetlinewidth{0.000000pt}%
\definecolor{currentstroke}{rgb}{0.000000,0.000000,0.000000}%
\pgfsetstrokecolor{currentstroke}%
\pgfsetstrokeopacity{0.000000}%
\pgfsetdash{}{0pt}%
\pgfpathmoveto{\pgfqpoint{1.225568in}{0.500000in}}%
\pgfpathlineto{\pgfqpoint{1.258594in}{0.500000in}}%
\pgfpathlineto{\pgfqpoint{1.258594in}{0.522210in}}%
\pgfpathlineto{\pgfqpoint{1.225568in}{0.522210in}}%
\pgfpathlineto{\pgfqpoint{1.225568in}{0.500000in}}%
\pgfpathclose%
\pgfusepath{fill}%
\end{pgfscope}%
\begin{pgfscope}%
\pgfpathrectangle{\pgfqpoint{0.750000in}{0.500000in}}{\pgfqpoint{4.650000in}{3.020000in}}%
\pgfusepath{clip}%
\pgfsetbuttcap%
\pgfsetmiterjoin%
\definecolor{currentfill}{rgb}{0.000000,0.500000,0.000000}%
\pgfsetfillcolor{currentfill}%
\pgfsetlinewidth{0.000000pt}%
\definecolor{currentstroke}{rgb}{0.000000,0.000000,0.000000}%
\pgfsetstrokecolor{currentstroke}%
\pgfsetstrokeopacity{0.000000}%
\pgfsetdash{}{0pt}%
\pgfpathmoveto{\pgfqpoint{1.258594in}{0.500000in}}%
\pgfpathlineto{\pgfqpoint{1.291619in}{0.500000in}}%
\pgfpathlineto{\pgfqpoint{1.291619in}{0.538867in}}%
\pgfpathlineto{\pgfqpoint{1.258594in}{0.538867in}}%
\pgfpathlineto{\pgfqpoint{1.258594in}{0.500000in}}%
\pgfpathclose%
\pgfusepath{fill}%
\end{pgfscope}%
\begin{pgfscope}%
\pgfpathrectangle{\pgfqpoint{0.750000in}{0.500000in}}{\pgfqpoint{4.650000in}{3.020000in}}%
\pgfusepath{clip}%
\pgfsetbuttcap%
\pgfsetmiterjoin%
\definecolor{currentfill}{rgb}{0.000000,0.500000,0.000000}%
\pgfsetfillcolor{currentfill}%
\pgfsetlinewidth{0.000000pt}%
\definecolor{currentstroke}{rgb}{0.000000,0.000000,0.000000}%
\pgfsetstrokecolor{currentstroke}%
\pgfsetstrokeopacity{0.000000}%
\pgfsetdash{}{0pt}%
\pgfpathmoveto{\pgfqpoint{1.291619in}{0.500000in}}%
\pgfpathlineto{\pgfqpoint{1.324645in}{0.500000in}}%
\pgfpathlineto{\pgfqpoint{1.324645in}{0.577735in}}%
\pgfpathlineto{\pgfqpoint{1.291619in}{0.577735in}}%
\pgfpathlineto{\pgfqpoint{1.291619in}{0.500000in}}%
\pgfpathclose%
\pgfusepath{fill}%
\end{pgfscope}%
\begin{pgfscope}%
\pgfpathrectangle{\pgfqpoint{0.750000in}{0.500000in}}{\pgfqpoint{4.650000in}{3.020000in}}%
\pgfusepath{clip}%
\pgfsetbuttcap%
\pgfsetmiterjoin%
\definecolor{currentfill}{rgb}{0.000000,0.500000,0.000000}%
\pgfsetfillcolor{currentfill}%
\pgfsetlinewidth{0.000000pt}%
\definecolor{currentstroke}{rgb}{0.000000,0.000000,0.000000}%
\pgfsetstrokecolor{currentstroke}%
\pgfsetstrokeopacity{0.000000}%
\pgfsetdash{}{0pt}%
\pgfpathmoveto{\pgfqpoint{1.324645in}{0.500000in}}%
\pgfpathlineto{\pgfqpoint{1.357670in}{0.500000in}}%
\pgfpathlineto{\pgfqpoint{1.357670in}{0.572182in}}%
\pgfpathlineto{\pgfqpoint{1.324645in}{0.572182in}}%
\pgfpathlineto{\pgfqpoint{1.324645in}{0.500000in}}%
\pgfpathclose%
\pgfusepath{fill}%
\end{pgfscope}%
\begin{pgfscope}%
\pgfpathrectangle{\pgfqpoint{0.750000in}{0.500000in}}{\pgfqpoint{4.650000in}{3.020000in}}%
\pgfusepath{clip}%
\pgfsetbuttcap%
\pgfsetmiterjoin%
\definecolor{currentfill}{rgb}{0.000000,0.500000,0.000000}%
\pgfsetfillcolor{currentfill}%
\pgfsetlinewidth{0.000000pt}%
\definecolor{currentstroke}{rgb}{0.000000,0.000000,0.000000}%
\pgfsetstrokecolor{currentstroke}%
\pgfsetstrokeopacity{0.000000}%
\pgfsetdash{}{0pt}%
\pgfpathmoveto{\pgfqpoint{1.357670in}{0.500000in}}%
\pgfpathlineto{\pgfqpoint{1.390696in}{0.500000in}}%
\pgfpathlineto{\pgfqpoint{1.390696in}{0.516657in}}%
\pgfpathlineto{\pgfqpoint{1.357670in}{0.516657in}}%
\pgfpathlineto{\pgfqpoint{1.357670in}{0.500000in}}%
\pgfpathclose%
\pgfusepath{fill}%
\end{pgfscope}%
\begin{pgfscope}%
\pgfpathrectangle{\pgfqpoint{0.750000in}{0.500000in}}{\pgfqpoint{4.650000in}{3.020000in}}%
\pgfusepath{clip}%
\pgfsetbuttcap%
\pgfsetmiterjoin%
\definecolor{currentfill}{rgb}{0.000000,0.500000,0.000000}%
\pgfsetfillcolor{currentfill}%
\pgfsetlinewidth{0.000000pt}%
\definecolor{currentstroke}{rgb}{0.000000,0.000000,0.000000}%
\pgfsetstrokecolor{currentstroke}%
\pgfsetstrokeopacity{0.000000}%
\pgfsetdash{}{0pt}%
\pgfpathmoveto{\pgfqpoint{1.390696in}{0.500000in}}%
\pgfpathlineto{\pgfqpoint{1.423722in}{0.500000in}}%
\pgfpathlineto{\pgfqpoint{1.423722in}{0.616602in}}%
\pgfpathlineto{\pgfqpoint{1.390696in}{0.616602in}}%
\pgfpathlineto{\pgfqpoint{1.390696in}{0.500000in}}%
\pgfpathclose%
\pgfusepath{fill}%
\end{pgfscope}%
\begin{pgfscope}%
\pgfpathrectangle{\pgfqpoint{0.750000in}{0.500000in}}{\pgfqpoint{4.650000in}{3.020000in}}%
\pgfusepath{clip}%
\pgfsetbuttcap%
\pgfsetmiterjoin%
\definecolor{currentfill}{rgb}{0.000000,0.500000,0.000000}%
\pgfsetfillcolor{currentfill}%
\pgfsetlinewidth{0.000000pt}%
\definecolor{currentstroke}{rgb}{0.000000,0.000000,0.000000}%
\pgfsetstrokecolor{currentstroke}%
\pgfsetstrokeopacity{0.000000}%
\pgfsetdash{}{0pt}%
\pgfpathmoveto{\pgfqpoint{1.423722in}{0.500000in}}%
\pgfpathlineto{\pgfqpoint{1.456747in}{0.500000in}}%
\pgfpathlineto{\pgfqpoint{1.456747in}{0.583287in}}%
\pgfpathlineto{\pgfqpoint{1.423722in}{0.583287in}}%
\pgfpathlineto{\pgfqpoint{1.423722in}{0.500000in}}%
\pgfpathclose%
\pgfusepath{fill}%
\end{pgfscope}%
\begin{pgfscope}%
\pgfpathrectangle{\pgfqpoint{0.750000in}{0.500000in}}{\pgfqpoint{4.650000in}{3.020000in}}%
\pgfusepath{clip}%
\pgfsetbuttcap%
\pgfsetmiterjoin%
\definecolor{currentfill}{rgb}{0.000000,0.500000,0.000000}%
\pgfsetfillcolor{currentfill}%
\pgfsetlinewidth{0.000000pt}%
\definecolor{currentstroke}{rgb}{0.000000,0.000000,0.000000}%
\pgfsetstrokecolor{currentstroke}%
\pgfsetstrokeopacity{0.000000}%
\pgfsetdash{}{0pt}%
\pgfpathmoveto{\pgfqpoint{1.456747in}{0.500000in}}%
\pgfpathlineto{\pgfqpoint{1.489773in}{0.500000in}}%
\pgfpathlineto{\pgfqpoint{1.489773in}{0.599945in}}%
\pgfpathlineto{\pgfqpoint{1.456747in}{0.599945in}}%
\pgfpathlineto{\pgfqpoint{1.456747in}{0.500000in}}%
\pgfpathclose%
\pgfusepath{fill}%
\end{pgfscope}%
\begin{pgfscope}%
\pgfpathrectangle{\pgfqpoint{0.750000in}{0.500000in}}{\pgfqpoint{4.650000in}{3.020000in}}%
\pgfusepath{clip}%
\pgfsetbuttcap%
\pgfsetmiterjoin%
\definecolor{currentfill}{rgb}{0.000000,0.500000,0.000000}%
\pgfsetfillcolor{currentfill}%
\pgfsetlinewidth{0.000000pt}%
\definecolor{currentstroke}{rgb}{0.000000,0.000000,0.000000}%
\pgfsetstrokecolor{currentstroke}%
\pgfsetstrokeopacity{0.000000}%
\pgfsetdash{}{0pt}%
\pgfpathmoveto{\pgfqpoint{1.489773in}{0.500000in}}%
\pgfpathlineto{\pgfqpoint{1.522798in}{0.500000in}}%
\pgfpathlineto{\pgfqpoint{1.522798in}{0.777625in}}%
\pgfpathlineto{\pgfqpoint{1.489773in}{0.777625in}}%
\pgfpathlineto{\pgfqpoint{1.489773in}{0.500000in}}%
\pgfpathclose%
\pgfusepath{fill}%
\end{pgfscope}%
\begin{pgfscope}%
\pgfpathrectangle{\pgfqpoint{0.750000in}{0.500000in}}{\pgfqpoint{4.650000in}{3.020000in}}%
\pgfusepath{clip}%
\pgfsetbuttcap%
\pgfsetmiterjoin%
\definecolor{currentfill}{rgb}{0.000000,0.500000,0.000000}%
\pgfsetfillcolor{currentfill}%
\pgfsetlinewidth{0.000000pt}%
\definecolor{currentstroke}{rgb}{0.000000,0.000000,0.000000}%
\pgfsetstrokecolor{currentstroke}%
\pgfsetstrokeopacity{0.000000}%
\pgfsetdash{}{0pt}%
\pgfpathmoveto{\pgfqpoint{1.522798in}{0.500000in}}%
\pgfpathlineto{\pgfqpoint{1.555824in}{0.500000in}}%
\pgfpathlineto{\pgfqpoint{1.555824in}{0.638812in}}%
\pgfpathlineto{\pgfqpoint{1.522798in}{0.638812in}}%
\pgfpathlineto{\pgfqpoint{1.522798in}{0.500000in}}%
\pgfpathclose%
\pgfusepath{fill}%
\end{pgfscope}%
\begin{pgfscope}%
\pgfpathrectangle{\pgfqpoint{0.750000in}{0.500000in}}{\pgfqpoint{4.650000in}{3.020000in}}%
\pgfusepath{clip}%
\pgfsetbuttcap%
\pgfsetmiterjoin%
\definecolor{currentfill}{rgb}{0.000000,0.500000,0.000000}%
\pgfsetfillcolor{currentfill}%
\pgfsetlinewidth{0.000000pt}%
\definecolor{currentstroke}{rgb}{0.000000,0.000000,0.000000}%
\pgfsetstrokecolor{currentstroke}%
\pgfsetstrokeopacity{0.000000}%
\pgfsetdash{}{0pt}%
\pgfpathmoveto{\pgfqpoint{1.555824in}{0.500000in}}%
\pgfpathlineto{\pgfqpoint{1.588849in}{0.500000in}}%
\pgfpathlineto{\pgfqpoint{1.588849in}{0.855359in}}%
\pgfpathlineto{\pgfqpoint{1.555824in}{0.855359in}}%
\pgfpathlineto{\pgfqpoint{1.555824in}{0.500000in}}%
\pgfpathclose%
\pgfusepath{fill}%
\end{pgfscope}%
\begin{pgfscope}%
\pgfpathrectangle{\pgfqpoint{0.750000in}{0.500000in}}{\pgfqpoint{4.650000in}{3.020000in}}%
\pgfusepath{clip}%
\pgfsetbuttcap%
\pgfsetmiterjoin%
\definecolor{currentfill}{rgb}{0.000000,0.500000,0.000000}%
\pgfsetfillcolor{currentfill}%
\pgfsetlinewidth{0.000000pt}%
\definecolor{currentstroke}{rgb}{0.000000,0.000000,0.000000}%
\pgfsetstrokecolor{currentstroke}%
\pgfsetstrokeopacity{0.000000}%
\pgfsetdash{}{0pt}%
\pgfpathmoveto{\pgfqpoint{1.588849in}{0.500000in}}%
\pgfpathlineto{\pgfqpoint{1.621875in}{0.500000in}}%
\pgfpathlineto{\pgfqpoint{1.621875in}{0.833149in}}%
\pgfpathlineto{\pgfqpoint{1.588849in}{0.833149in}}%
\pgfpathlineto{\pgfqpoint{1.588849in}{0.500000in}}%
\pgfpathclose%
\pgfusepath{fill}%
\end{pgfscope}%
\begin{pgfscope}%
\pgfpathrectangle{\pgfqpoint{0.750000in}{0.500000in}}{\pgfqpoint{4.650000in}{3.020000in}}%
\pgfusepath{clip}%
\pgfsetbuttcap%
\pgfsetmiterjoin%
\definecolor{currentfill}{rgb}{0.000000,0.500000,0.000000}%
\pgfsetfillcolor{currentfill}%
\pgfsetlinewidth{0.000000pt}%
\definecolor{currentstroke}{rgb}{0.000000,0.000000,0.000000}%
\pgfsetstrokecolor{currentstroke}%
\pgfsetstrokeopacity{0.000000}%
\pgfsetdash{}{0pt}%
\pgfpathmoveto{\pgfqpoint{1.621875in}{0.500000in}}%
\pgfpathlineto{\pgfqpoint{1.654901in}{0.500000in}}%
\pgfpathlineto{\pgfqpoint{1.654901in}{0.794282in}}%
\pgfpathlineto{\pgfqpoint{1.621875in}{0.794282in}}%
\pgfpathlineto{\pgfqpoint{1.621875in}{0.500000in}}%
\pgfpathclose%
\pgfusepath{fill}%
\end{pgfscope}%
\begin{pgfscope}%
\pgfpathrectangle{\pgfqpoint{0.750000in}{0.500000in}}{\pgfqpoint{4.650000in}{3.020000in}}%
\pgfusepath{clip}%
\pgfsetbuttcap%
\pgfsetmiterjoin%
\definecolor{currentfill}{rgb}{0.000000,0.500000,0.000000}%
\pgfsetfillcolor{currentfill}%
\pgfsetlinewidth{0.000000pt}%
\definecolor{currentstroke}{rgb}{0.000000,0.000000,0.000000}%
\pgfsetstrokecolor{currentstroke}%
\pgfsetstrokeopacity{0.000000}%
\pgfsetdash{}{0pt}%
\pgfpathmoveto{\pgfqpoint{1.654901in}{0.500000in}}%
\pgfpathlineto{\pgfqpoint{1.687926in}{0.500000in}}%
\pgfpathlineto{\pgfqpoint{1.687926in}{1.227376in}}%
\pgfpathlineto{\pgfqpoint{1.654901in}{1.227376in}}%
\pgfpathlineto{\pgfqpoint{1.654901in}{0.500000in}}%
\pgfpathclose%
\pgfusepath{fill}%
\end{pgfscope}%
\begin{pgfscope}%
\pgfpathrectangle{\pgfqpoint{0.750000in}{0.500000in}}{\pgfqpoint{4.650000in}{3.020000in}}%
\pgfusepath{clip}%
\pgfsetbuttcap%
\pgfsetmiterjoin%
\definecolor{currentfill}{rgb}{0.000000,0.500000,0.000000}%
\pgfsetfillcolor{currentfill}%
\pgfsetlinewidth{0.000000pt}%
\definecolor{currentstroke}{rgb}{0.000000,0.000000,0.000000}%
\pgfsetstrokecolor{currentstroke}%
\pgfsetstrokeopacity{0.000000}%
\pgfsetdash{}{0pt}%
\pgfpathmoveto{\pgfqpoint{1.687926in}{0.500000in}}%
\pgfpathlineto{\pgfqpoint{1.720952in}{0.500000in}}%
\pgfpathlineto{\pgfqpoint{1.720952in}{0.827597in}}%
\pgfpathlineto{\pgfqpoint{1.687926in}{0.827597in}}%
\pgfpathlineto{\pgfqpoint{1.687926in}{0.500000in}}%
\pgfpathclose%
\pgfusepath{fill}%
\end{pgfscope}%
\begin{pgfscope}%
\pgfpathrectangle{\pgfqpoint{0.750000in}{0.500000in}}{\pgfqpoint{4.650000in}{3.020000in}}%
\pgfusepath{clip}%
\pgfsetbuttcap%
\pgfsetmiterjoin%
\definecolor{currentfill}{rgb}{0.000000,0.500000,0.000000}%
\pgfsetfillcolor{currentfill}%
\pgfsetlinewidth{0.000000pt}%
\definecolor{currentstroke}{rgb}{0.000000,0.000000,0.000000}%
\pgfsetstrokecolor{currentstroke}%
\pgfsetstrokeopacity{0.000000}%
\pgfsetdash{}{0pt}%
\pgfpathmoveto{\pgfqpoint{1.720952in}{0.500000in}}%
\pgfpathlineto{\pgfqpoint{1.753977in}{0.500000in}}%
\pgfpathlineto{\pgfqpoint{1.753977in}{0.999724in}}%
\pgfpathlineto{\pgfqpoint{1.720952in}{0.999724in}}%
\pgfpathlineto{\pgfqpoint{1.720952in}{0.500000in}}%
\pgfpathclose%
\pgfusepath{fill}%
\end{pgfscope}%
\begin{pgfscope}%
\pgfpathrectangle{\pgfqpoint{0.750000in}{0.500000in}}{\pgfqpoint{4.650000in}{3.020000in}}%
\pgfusepath{clip}%
\pgfsetbuttcap%
\pgfsetmiterjoin%
\definecolor{currentfill}{rgb}{0.000000,0.500000,0.000000}%
\pgfsetfillcolor{currentfill}%
\pgfsetlinewidth{0.000000pt}%
\definecolor{currentstroke}{rgb}{0.000000,0.000000,0.000000}%
\pgfsetstrokecolor{currentstroke}%
\pgfsetstrokeopacity{0.000000}%
\pgfsetdash{}{0pt}%
\pgfpathmoveto{\pgfqpoint{1.753977in}{0.500000in}}%
\pgfpathlineto{\pgfqpoint{1.787003in}{0.500000in}}%
\pgfpathlineto{\pgfqpoint{1.787003in}{1.593841in}}%
\pgfpathlineto{\pgfqpoint{1.753977in}{1.593841in}}%
\pgfpathlineto{\pgfqpoint{1.753977in}{0.500000in}}%
\pgfpathclose%
\pgfusepath{fill}%
\end{pgfscope}%
\begin{pgfscope}%
\pgfpathrectangle{\pgfqpoint{0.750000in}{0.500000in}}{\pgfqpoint{4.650000in}{3.020000in}}%
\pgfusepath{clip}%
\pgfsetbuttcap%
\pgfsetmiterjoin%
\definecolor{currentfill}{rgb}{0.000000,0.500000,0.000000}%
\pgfsetfillcolor{currentfill}%
\pgfsetlinewidth{0.000000pt}%
\definecolor{currentstroke}{rgb}{0.000000,0.000000,0.000000}%
\pgfsetstrokecolor{currentstroke}%
\pgfsetstrokeopacity{0.000000}%
\pgfsetdash{}{0pt}%
\pgfpathmoveto{\pgfqpoint{1.787003in}{0.500000in}}%
\pgfpathlineto{\pgfqpoint{1.820028in}{0.500000in}}%
\pgfpathlineto{\pgfqpoint{1.820028in}{1.182956in}}%
\pgfpathlineto{\pgfqpoint{1.787003in}{1.182956in}}%
\pgfpathlineto{\pgfqpoint{1.787003in}{0.500000in}}%
\pgfpathclose%
\pgfusepath{fill}%
\end{pgfscope}%
\begin{pgfscope}%
\pgfpathrectangle{\pgfqpoint{0.750000in}{0.500000in}}{\pgfqpoint{4.650000in}{3.020000in}}%
\pgfusepath{clip}%
\pgfsetbuttcap%
\pgfsetmiterjoin%
\definecolor{currentfill}{rgb}{0.000000,0.500000,0.000000}%
\pgfsetfillcolor{currentfill}%
\pgfsetlinewidth{0.000000pt}%
\definecolor{currentstroke}{rgb}{0.000000,0.000000,0.000000}%
\pgfsetstrokecolor{currentstroke}%
\pgfsetstrokeopacity{0.000000}%
\pgfsetdash{}{0pt}%
\pgfpathmoveto{\pgfqpoint{1.820028in}{0.500000in}}%
\pgfpathlineto{\pgfqpoint{1.853054in}{0.500000in}}%
\pgfpathlineto{\pgfqpoint{1.853054in}{1.666023in}}%
\pgfpathlineto{\pgfqpoint{1.820028in}{1.666023in}}%
\pgfpathlineto{\pgfqpoint{1.820028in}{0.500000in}}%
\pgfpathclose%
\pgfusepath{fill}%
\end{pgfscope}%
\begin{pgfscope}%
\pgfpathrectangle{\pgfqpoint{0.750000in}{0.500000in}}{\pgfqpoint{4.650000in}{3.020000in}}%
\pgfusepath{clip}%
\pgfsetbuttcap%
\pgfsetmiterjoin%
\definecolor{currentfill}{rgb}{0.000000,0.500000,0.000000}%
\pgfsetfillcolor{currentfill}%
\pgfsetlinewidth{0.000000pt}%
\definecolor{currentstroke}{rgb}{0.000000,0.000000,0.000000}%
\pgfsetstrokecolor{currentstroke}%
\pgfsetstrokeopacity{0.000000}%
\pgfsetdash{}{0pt}%
\pgfpathmoveto{\pgfqpoint{1.853054in}{0.500000in}}%
\pgfpathlineto{\pgfqpoint{1.886080in}{0.500000in}}%
\pgfpathlineto{\pgfqpoint{1.886080in}{1.532763in}}%
\pgfpathlineto{\pgfqpoint{1.853054in}{1.532763in}}%
\pgfpathlineto{\pgfqpoint{1.853054in}{0.500000in}}%
\pgfpathclose%
\pgfusepath{fill}%
\end{pgfscope}%
\begin{pgfscope}%
\pgfpathrectangle{\pgfqpoint{0.750000in}{0.500000in}}{\pgfqpoint{4.650000in}{3.020000in}}%
\pgfusepath{clip}%
\pgfsetbuttcap%
\pgfsetmiterjoin%
\definecolor{currentfill}{rgb}{0.000000,0.500000,0.000000}%
\pgfsetfillcolor{currentfill}%
\pgfsetlinewidth{0.000000pt}%
\definecolor{currentstroke}{rgb}{0.000000,0.000000,0.000000}%
\pgfsetstrokecolor{currentstroke}%
\pgfsetstrokeopacity{0.000000}%
\pgfsetdash{}{0pt}%
\pgfpathmoveto{\pgfqpoint{1.886080in}{0.500000in}}%
\pgfpathlineto{\pgfqpoint{1.919105in}{0.500000in}}%
\pgfpathlineto{\pgfqpoint{1.919105in}{1.321769in}}%
\pgfpathlineto{\pgfqpoint{1.886080in}{1.321769in}}%
\pgfpathlineto{\pgfqpoint{1.886080in}{0.500000in}}%
\pgfpathclose%
\pgfusepath{fill}%
\end{pgfscope}%
\begin{pgfscope}%
\pgfpathrectangle{\pgfqpoint{0.750000in}{0.500000in}}{\pgfqpoint{4.650000in}{3.020000in}}%
\pgfusepath{clip}%
\pgfsetbuttcap%
\pgfsetmiterjoin%
\definecolor{currentfill}{rgb}{0.000000,0.500000,0.000000}%
\pgfsetfillcolor{currentfill}%
\pgfsetlinewidth{0.000000pt}%
\definecolor{currentstroke}{rgb}{0.000000,0.000000,0.000000}%
\pgfsetstrokecolor{currentstroke}%
\pgfsetstrokeopacity{0.000000}%
\pgfsetdash{}{0pt}%
\pgfpathmoveto{\pgfqpoint{1.919105in}{0.500000in}}%
\pgfpathlineto{\pgfqpoint{1.952131in}{0.500000in}}%
\pgfpathlineto{\pgfqpoint{1.952131in}{2.460029in}}%
\pgfpathlineto{\pgfqpoint{1.919105in}{2.460029in}}%
\pgfpathlineto{\pgfqpoint{1.919105in}{0.500000in}}%
\pgfpathclose%
\pgfusepath{fill}%
\end{pgfscope}%
\begin{pgfscope}%
\pgfpathrectangle{\pgfqpoint{0.750000in}{0.500000in}}{\pgfqpoint{4.650000in}{3.020000in}}%
\pgfusepath{clip}%
\pgfsetbuttcap%
\pgfsetmiterjoin%
\definecolor{currentfill}{rgb}{0.000000,0.500000,0.000000}%
\pgfsetfillcolor{currentfill}%
\pgfsetlinewidth{0.000000pt}%
\definecolor{currentstroke}{rgb}{0.000000,0.000000,0.000000}%
\pgfsetstrokecolor{currentstroke}%
\pgfsetstrokeopacity{0.000000}%
\pgfsetdash{}{0pt}%
\pgfpathmoveto{\pgfqpoint{1.952131in}{0.500000in}}%
\pgfpathlineto{\pgfqpoint{1.985156in}{0.500000in}}%
\pgfpathlineto{\pgfqpoint{1.985156in}{1.721548in}}%
\pgfpathlineto{\pgfqpoint{1.952131in}{1.721548in}}%
\pgfpathlineto{\pgfqpoint{1.952131in}{0.500000in}}%
\pgfpathclose%
\pgfusepath{fill}%
\end{pgfscope}%
\begin{pgfscope}%
\pgfpathrectangle{\pgfqpoint{0.750000in}{0.500000in}}{\pgfqpoint{4.650000in}{3.020000in}}%
\pgfusepath{clip}%
\pgfsetbuttcap%
\pgfsetmiterjoin%
\definecolor{currentfill}{rgb}{0.000000,0.500000,0.000000}%
\pgfsetfillcolor{currentfill}%
\pgfsetlinewidth{0.000000pt}%
\definecolor{currentstroke}{rgb}{0.000000,0.000000,0.000000}%
\pgfsetstrokecolor{currentstroke}%
\pgfsetstrokeopacity{0.000000}%
\pgfsetdash{}{0pt}%
\pgfpathmoveto{\pgfqpoint{1.985156in}{0.500000in}}%
\pgfpathlineto{\pgfqpoint{2.018182in}{0.500000in}}%
\pgfpathlineto{\pgfqpoint{2.018182in}{1.849255in}}%
\pgfpathlineto{\pgfqpoint{1.985156in}{1.849255in}}%
\pgfpathlineto{\pgfqpoint{1.985156in}{0.500000in}}%
\pgfpathclose%
\pgfusepath{fill}%
\end{pgfscope}%
\begin{pgfscope}%
\pgfpathrectangle{\pgfqpoint{0.750000in}{0.500000in}}{\pgfqpoint{4.650000in}{3.020000in}}%
\pgfusepath{clip}%
\pgfsetbuttcap%
\pgfsetmiterjoin%
\definecolor{currentfill}{rgb}{0.000000,0.500000,0.000000}%
\pgfsetfillcolor{currentfill}%
\pgfsetlinewidth{0.000000pt}%
\definecolor{currentstroke}{rgb}{0.000000,0.000000,0.000000}%
\pgfsetstrokecolor{currentstroke}%
\pgfsetstrokeopacity{0.000000}%
\pgfsetdash{}{0pt}%
\pgfpathmoveto{\pgfqpoint{2.018182in}{0.500000in}}%
\pgfpathlineto{\pgfqpoint{2.051207in}{0.500000in}}%
\pgfpathlineto{\pgfqpoint{2.051207in}{2.865361in}}%
\pgfpathlineto{\pgfqpoint{2.018182in}{2.865361in}}%
\pgfpathlineto{\pgfqpoint{2.018182in}{0.500000in}}%
\pgfpathclose%
\pgfusepath{fill}%
\end{pgfscope}%
\begin{pgfscope}%
\pgfpathrectangle{\pgfqpoint{0.750000in}{0.500000in}}{\pgfqpoint{4.650000in}{3.020000in}}%
\pgfusepath{clip}%
\pgfsetbuttcap%
\pgfsetmiterjoin%
\definecolor{currentfill}{rgb}{0.000000,0.500000,0.000000}%
\pgfsetfillcolor{currentfill}%
\pgfsetlinewidth{0.000000pt}%
\definecolor{currentstroke}{rgb}{0.000000,0.000000,0.000000}%
\pgfsetstrokecolor{currentstroke}%
\pgfsetstrokeopacity{0.000000}%
\pgfsetdash{}{0pt}%
\pgfpathmoveto{\pgfqpoint{2.051207in}{0.500000in}}%
\pgfpathlineto{\pgfqpoint{2.084233in}{0.500000in}}%
\pgfpathlineto{\pgfqpoint{2.084233in}{1.799283in}}%
\pgfpathlineto{\pgfqpoint{2.051207in}{1.799283in}}%
\pgfpathlineto{\pgfqpoint{2.051207in}{0.500000in}}%
\pgfpathclose%
\pgfusepath{fill}%
\end{pgfscope}%
\begin{pgfscope}%
\pgfpathrectangle{\pgfqpoint{0.750000in}{0.500000in}}{\pgfqpoint{4.650000in}{3.020000in}}%
\pgfusepath{clip}%
\pgfsetbuttcap%
\pgfsetmiterjoin%
\definecolor{currentfill}{rgb}{0.000000,0.500000,0.000000}%
\pgfsetfillcolor{currentfill}%
\pgfsetlinewidth{0.000000pt}%
\definecolor{currentstroke}{rgb}{0.000000,0.000000,0.000000}%
\pgfsetstrokecolor{currentstroke}%
\pgfsetstrokeopacity{0.000000}%
\pgfsetdash{}{0pt}%
\pgfpathmoveto{\pgfqpoint{2.084233in}{0.500000in}}%
\pgfpathlineto{\pgfqpoint{2.117259in}{0.500000in}}%
\pgfpathlineto{\pgfqpoint{2.117259in}{2.920886in}}%
\pgfpathlineto{\pgfqpoint{2.084233in}{2.920886in}}%
\pgfpathlineto{\pgfqpoint{2.084233in}{0.500000in}}%
\pgfpathclose%
\pgfusepath{fill}%
\end{pgfscope}%
\begin{pgfscope}%
\pgfpathrectangle{\pgfqpoint{0.750000in}{0.500000in}}{\pgfqpoint{4.650000in}{3.020000in}}%
\pgfusepath{clip}%
\pgfsetbuttcap%
\pgfsetmiterjoin%
\definecolor{currentfill}{rgb}{0.000000,0.500000,0.000000}%
\pgfsetfillcolor{currentfill}%
\pgfsetlinewidth{0.000000pt}%
\definecolor{currentstroke}{rgb}{0.000000,0.000000,0.000000}%
\pgfsetstrokecolor{currentstroke}%
\pgfsetstrokeopacity{0.000000}%
\pgfsetdash{}{0pt}%
\pgfpathmoveto{\pgfqpoint{2.117259in}{0.500000in}}%
\pgfpathlineto{\pgfqpoint{2.150284in}{0.500000in}}%
\pgfpathlineto{\pgfqpoint{2.150284in}{2.249035in}}%
\pgfpathlineto{\pgfqpoint{2.117259in}{2.249035in}}%
\pgfpathlineto{\pgfqpoint{2.117259in}{0.500000in}}%
\pgfpathclose%
\pgfusepath{fill}%
\end{pgfscope}%
\begin{pgfscope}%
\pgfpathrectangle{\pgfqpoint{0.750000in}{0.500000in}}{\pgfqpoint{4.650000in}{3.020000in}}%
\pgfusepath{clip}%
\pgfsetbuttcap%
\pgfsetmiterjoin%
\definecolor{currentfill}{rgb}{0.000000,0.500000,0.000000}%
\pgfsetfillcolor{currentfill}%
\pgfsetlinewidth{0.000000pt}%
\definecolor{currentstroke}{rgb}{0.000000,0.000000,0.000000}%
\pgfsetstrokecolor{currentstroke}%
\pgfsetstrokeopacity{0.000000}%
\pgfsetdash{}{0pt}%
\pgfpathmoveto{\pgfqpoint{2.150284in}{0.500000in}}%
\pgfpathlineto{\pgfqpoint{2.183310in}{0.500000in}}%
\pgfpathlineto{\pgfqpoint{2.183310in}{1.888123in}}%
\pgfpathlineto{\pgfqpoint{2.150284in}{1.888123in}}%
\pgfpathlineto{\pgfqpoint{2.150284in}{0.500000in}}%
\pgfpathclose%
\pgfusepath{fill}%
\end{pgfscope}%
\begin{pgfscope}%
\pgfpathrectangle{\pgfqpoint{0.750000in}{0.500000in}}{\pgfqpoint{4.650000in}{3.020000in}}%
\pgfusepath{clip}%
\pgfsetbuttcap%
\pgfsetmiterjoin%
\definecolor{currentfill}{rgb}{0.000000,0.500000,0.000000}%
\pgfsetfillcolor{currentfill}%
\pgfsetlinewidth{0.000000pt}%
\definecolor{currentstroke}{rgb}{0.000000,0.000000,0.000000}%
\pgfsetstrokecolor{currentstroke}%
\pgfsetstrokeopacity{0.000000}%
\pgfsetdash{}{0pt}%
\pgfpathmoveto{\pgfqpoint{2.183310in}{0.500000in}}%
\pgfpathlineto{\pgfqpoint{2.216335in}{0.500000in}}%
\pgfpathlineto{\pgfqpoint{2.216335in}{3.376190in}}%
\pgfpathlineto{\pgfqpoint{2.183310in}{3.376190in}}%
\pgfpathlineto{\pgfqpoint{2.183310in}{0.500000in}}%
\pgfpathclose%
\pgfusepath{fill}%
\end{pgfscope}%
\begin{pgfscope}%
\pgfpathrectangle{\pgfqpoint{0.750000in}{0.500000in}}{\pgfqpoint{4.650000in}{3.020000in}}%
\pgfusepath{clip}%
\pgfsetbuttcap%
\pgfsetmiterjoin%
\definecolor{currentfill}{rgb}{0.000000,0.500000,0.000000}%
\pgfsetfillcolor{currentfill}%
\pgfsetlinewidth{0.000000pt}%
\definecolor{currentstroke}{rgb}{0.000000,0.000000,0.000000}%
\pgfsetstrokecolor{currentstroke}%
\pgfsetstrokeopacity{0.000000}%
\pgfsetdash{}{0pt}%
\pgfpathmoveto{\pgfqpoint{2.216335in}{0.500000in}}%
\pgfpathlineto{\pgfqpoint{2.249361in}{0.500000in}}%
\pgfpathlineto{\pgfqpoint{2.249361in}{2.026935in}}%
\pgfpathlineto{\pgfqpoint{2.216335in}{2.026935in}}%
\pgfpathlineto{\pgfqpoint{2.216335in}{0.500000in}}%
\pgfpathclose%
\pgfusepath{fill}%
\end{pgfscope}%
\begin{pgfscope}%
\pgfpathrectangle{\pgfqpoint{0.750000in}{0.500000in}}{\pgfqpoint{4.650000in}{3.020000in}}%
\pgfusepath{clip}%
\pgfsetbuttcap%
\pgfsetmiterjoin%
\definecolor{currentfill}{rgb}{0.000000,0.500000,0.000000}%
\pgfsetfillcolor{currentfill}%
\pgfsetlinewidth{0.000000pt}%
\definecolor{currentstroke}{rgb}{0.000000,0.000000,0.000000}%
\pgfsetstrokecolor{currentstroke}%
\pgfsetstrokeopacity{0.000000}%
\pgfsetdash{}{0pt}%
\pgfpathmoveto{\pgfqpoint{2.249361in}{0.500000in}}%
\pgfpathlineto{\pgfqpoint{2.282386in}{0.500000in}}%
\pgfpathlineto{\pgfqpoint{2.282386in}{2.243482in}}%
\pgfpathlineto{\pgfqpoint{2.249361in}{2.243482in}}%
\pgfpathlineto{\pgfqpoint{2.249361in}{0.500000in}}%
\pgfpathclose%
\pgfusepath{fill}%
\end{pgfscope}%
\begin{pgfscope}%
\pgfpathrectangle{\pgfqpoint{0.750000in}{0.500000in}}{\pgfqpoint{4.650000in}{3.020000in}}%
\pgfusepath{clip}%
\pgfsetbuttcap%
\pgfsetmiterjoin%
\definecolor{currentfill}{rgb}{0.000000,0.500000,0.000000}%
\pgfsetfillcolor{currentfill}%
\pgfsetlinewidth{0.000000pt}%
\definecolor{currentstroke}{rgb}{0.000000,0.000000,0.000000}%
\pgfsetstrokecolor{currentstroke}%
\pgfsetstrokeopacity{0.000000}%
\pgfsetdash{}{0pt}%
\pgfpathmoveto{\pgfqpoint{2.282386in}{0.500000in}}%
\pgfpathlineto{\pgfqpoint{2.315412in}{0.500000in}}%
\pgfpathlineto{\pgfqpoint{2.315412in}{3.165196in}}%
\pgfpathlineto{\pgfqpoint{2.282386in}{3.165196in}}%
\pgfpathlineto{\pgfqpoint{2.282386in}{0.500000in}}%
\pgfpathclose%
\pgfusepath{fill}%
\end{pgfscope}%
\begin{pgfscope}%
\pgfpathrectangle{\pgfqpoint{0.750000in}{0.500000in}}{\pgfqpoint{4.650000in}{3.020000in}}%
\pgfusepath{clip}%
\pgfsetbuttcap%
\pgfsetmiterjoin%
\definecolor{currentfill}{rgb}{0.000000,0.500000,0.000000}%
\pgfsetfillcolor{currentfill}%
\pgfsetlinewidth{0.000000pt}%
\definecolor{currentstroke}{rgb}{0.000000,0.000000,0.000000}%
\pgfsetstrokecolor{currentstroke}%
\pgfsetstrokeopacity{0.000000}%
\pgfsetdash{}{0pt}%
\pgfpathmoveto{\pgfqpoint{2.315412in}{0.500000in}}%
\pgfpathlineto{\pgfqpoint{2.348437in}{0.500000in}}%
\pgfpathlineto{\pgfqpoint{2.348437in}{2.088013in}}%
\pgfpathlineto{\pgfqpoint{2.315412in}{2.088013in}}%
\pgfpathlineto{\pgfqpoint{2.315412in}{0.500000in}}%
\pgfpathclose%
\pgfusepath{fill}%
\end{pgfscope}%
\begin{pgfscope}%
\pgfpathrectangle{\pgfqpoint{0.750000in}{0.500000in}}{\pgfqpoint{4.650000in}{3.020000in}}%
\pgfusepath{clip}%
\pgfsetbuttcap%
\pgfsetmiterjoin%
\definecolor{currentfill}{rgb}{0.000000,0.500000,0.000000}%
\pgfsetfillcolor{currentfill}%
\pgfsetlinewidth{0.000000pt}%
\definecolor{currentstroke}{rgb}{0.000000,0.000000,0.000000}%
\pgfsetstrokecolor{currentstroke}%
\pgfsetstrokeopacity{0.000000}%
\pgfsetdash{}{0pt}%
\pgfpathmoveto{\pgfqpoint{2.348437in}{0.500000in}}%
\pgfpathlineto{\pgfqpoint{2.381463in}{0.500000in}}%
\pgfpathlineto{\pgfqpoint{2.381463in}{3.159643in}}%
\pgfpathlineto{\pgfqpoint{2.348437in}{3.159643in}}%
\pgfpathlineto{\pgfqpoint{2.348437in}{0.500000in}}%
\pgfpathclose%
\pgfusepath{fill}%
\end{pgfscope}%
\begin{pgfscope}%
\pgfpathrectangle{\pgfqpoint{0.750000in}{0.500000in}}{\pgfqpoint{4.650000in}{3.020000in}}%
\pgfusepath{clip}%
\pgfsetbuttcap%
\pgfsetmiterjoin%
\definecolor{currentfill}{rgb}{0.000000,0.500000,0.000000}%
\pgfsetfillcolor{currentfill}%
\pgfsetlinewidth{0.000000pt}%
\definecolor{currentstroke}{rgb}{0.000000,0.000000,0.000000}%
\pgfsetstrokecolor{currentstroke}%
\pgfsetstrokeopacity{0.000000}%
\pgfsetdash{}{0pt}%
\pgfpathmoveto{\pgfqpoint{2.381463in}{0.500000in}}%
\pgfpathlineto{\pgfqpoint{2.414489in}{0.500000in}}%
\pgfpathlineto{\pgfqpoint{2.414489in}{2.343427in}}%
\pgfpathlineto{\pgfqpoint{2.381463in}{2.343427in}}%
\pgfpathlineto{\pgfqpoint{2.381463in}{0.500000in}}%
\pgfpathclose%
\pgfusepath{fill}%
\end{pgfscope}%
\begin{pgfscope}%
\pgfpathrectangle{\pgfqpoint{0.750000in}{0.500000in}}{\pgfqpoint{4.650000in}{3.020000in}}%
\pgfusepath{clip}%
\pgfsetbuttcap%
\pgfsetmiterjoin%
\definecolor{currentfill}{rgb}{0.000000,0.500000,0.000000}%
\pgfsetfillcolor{currentfill}%
\pgfsetlinewidth{0.000000pt}%
\definecolor{currentstroke}{rgb}{0.000000,0.000000,0.000000}%
\pgfsetstrokecolor{currentstroke}%
\pgfsetstrokeopacity{0.000000}%
\pgfsetdash{}{0pt}%
\pgfpathmoveto{\pgfqpoint{2.414489in}{0.500000in}}%
\pgfpathlineto{\pgfqpoint{2.447514in}{0.500000in}}%
\pgfpathlineto{\pgfqpoint{2.447514in}{1.926990in}}%
\pgfpathlineto{\pgfqpoint{2.414489in}{1.926990in}}%
\pgfpathlineto{\pgfqpoint{2.414489in}{0.500000in}}%
\pgfpathclose%
\pgfusepath{fill}%
\end{pgfscope}%
\begin{pgfscope}%
\pgfpathrectangle{\pgfqpoint{0.750000in}{0.500000in}}{\pgfqpoint{4.650000in}{3.020000in}}%
\pgfusepath{clip}%
\pgfsetbuttcap%
\pgfsetmiterjoin%
\definecolor{currentfill}{rgb}{0.000000,0.500000,0.000000}%
\pgfsetfillcolor{currentfill}%
\pgfsetlinewidth{0.000000pt}%
\definecolor{currentstroke}{rgb}{0.000000,0.000000,0.000000}%
\pgfsetstrokecolor{currentstroke}%
\pgfsetstrokeopacity{0.000000}%
\pgfsetdash{}{0pt}%
\pgfpathmoveto{\pgfqpoint{2.447514in}{0.500000in}}%
\pgfpathlineto{\pgfqpoint{2.480540in}{0.500000in}}%
\pgfpathlineto{\pgfqpoint{2.480540in}{2.820941in}}%
\pgfpathlineto{\pgfqpoint{2.447514in}{2.820941in}}%
\pgfpathlineto{\pgfqpoint{2.447514in}{0.500000in}}%
\pgfpathclose%
\pgfusepath{fill}%
\end{pgfscope}%
\begin{pgfscope}%
\pgfpathrectangle{\pgfqpoint{0.750000in}{0.500000in}}{\pgfqpoint{4.650000in}{3.020000in}}%
\pgfusepath{clip}%
\pgfsetbuttcap%
\pgfsetmiterjoin%
\definecolor{currentfill}{rgb}{0.000000,0.500000,0.000000}%
\pgfsetfillcolor{currentfill}%
\pgfsetlinewidth{0.000000pt}%
\definecolor{currentstroke}{rgb}{0.000000,0.000000,0.000000}%
\pgfsetstrokecolor{currentstroke}%
\pgfsetstrokeopacity{0.000000}%
\pgfsetdash{}{0pt}%
\pgfpathmoveto{\pgfqpoint{2.480540in}{0.500000in}}%
\pgfpathlineto{\pgfqpoint{2.513565in}{0.500000in}}%
\pgfpathlineto{\pgfqpoint{2.513565in}{1.882570in}}%
\pgfpathlineto{\pgfqpoint{2.480540in}{1.882570in}}%
\pgfpathlineto{\pgfqpoint{2.480540in}{0.500000in}}%
\pgfpathclose%
\pgfusepath{fill}%
\end{pgfscope}%
\begin{pgfscope}%
\pgfpathrectangle{\pgfqpoint{0.750000in}{0.500000in}}{\pgfqpoint{4.650000in}{3.020000in}}%
\pgfusepath{clip}%
\pgfsetbuttcap%
\pgfsetmiterjoin%
\definecolor{currentfill}{rgb}{0.000000,0.500000,0.000000}%
\pgfsetfillcolor{currentfill}%
\pgfsetlinewidth{0.000000pt}%
\definecolor{currentstroke}{rgb}{0.000000,0.000000,0.000000}%
\pgfsetstrokecolor{currentstroke}%
\pgfsetstrokeopacity{0.000000}%
\pgfsetdash{}{0pt}%
\pgfpathmoveto{\pgfqpoint{2.513565in}{0.500000in}}%
\pgfpathlineto{\pgfqpoint{2.546591in}{0.500000in}}%
\pgfpathlineto{\pgfqpoint{2.546591in}{1.810388in}}%
\pgfpathlineto{\pgfqpoint{2.513565in}{1.810388in}}%
\pgfpathlineto{\pgfqpoint{2.513565in}{0.500000in}}%
\pgfpathclose%
\pgfusepath{fill}%
\end{pgfscope}%
\begin{pgfscope}%
\pgfpathrectangle{\pgfqpoint{0.750000in}{0.500000in}}{\pgfqpoint{4.650000in}{3.020000in}}%
\pgfusepath{clip}%
\pgfsetbuttcap%
\pgfsetmiterjoin%
\definecolor{currentfill}{rgb}{0.000000,0.500000,0.000000}%
\pgfsetfillcolor{currentfill}%
\pgfsetlinewidth{0.000000pt}%
\definecolor{currentstroke}{rgb}{0.000000,0.000000,0.000000}%
\pgfsetstrokecolor{currentstroke}%
\pgfsetstrokeopacity{0.000000}%
\pgfsetdash{}{0pt}%
\pgfpathmoveto{\pgfqpoint{2.546591in}{0.500000in}}%
\pgfpathlineto{\pgfqpoint{2.579616in}{0.500000in}}%
\pgfpathlineto{\pgfqpoint{2.579616in}{2.415609in}}%
\pgfpathlineto{\pgfqpoint{2.546591in}{2.415609in}}%
\pgfpathlineto{\pgfqpoint{2.546591in}{0.500000in}}%
\pgfpathclose%
\pgfusepath{fill}%
\end{pgfscope}%
\begin{pgfscope}%
\pgfpathrectangle{\pgfqpoint{0.750000in}{0.500000in}}{\pgfqpoint{4.650000in}{3.020000in}}%
\pgfusepath{clip}%
\pgfsetbuttcap%
\pgfsetmiterjoin%
\definecolor{currentfill}{rgb}{0.000000,0.500000,0.000000}%
\pgfsetfillcolor{currentfill}%
\pgfsetlinewidth{0.000000pt}%
\definecolor{currentstroke}{rgb}{0.000000,0.000000,0.000000}%
\pgfsetstrokecolor{currentstroke}%
\pgfsetstrokeopacity{0.000000}%
\pgfsetdash{}{0pt}%
\pgfpathmoveto{\pgfqpoint{2.579616in}{0.500000in}}%
\pgfpathlineto{\pgfqpoint{2.612642in}{0.500000in}}%
\pgfpathlineto{\pgfqpoint{2.612642in}{1.538316in}}%
\pgfpathlineto{\pgfqpoint{2.579616in}{1.538316in}}%
\pgfpathlineto{\pgfqpoint{2.579616in}{0.500000in}}%
\pgfpathclose%
\pgfusepath{fill}%
\end{pgfscope}%
\begin{pgfscope}%
\pgfpathrectangle{\pgfqpoint{0.750000in}{0.500000in}}{\pgfqpoint{4.650000in}{3.020000in}}%
\pgfusepath{clip}%
\pgfsetbuttcap%
\pgfsetmiterjoin%
\definecolor{currentfill}{rgb}{0.000000,0.500000,0.000000}%
\pgfsetfillcolor{currentfill}%
\pgfsetlinewidth{0.000000pt}%
\definecolor{currentstroke}{rgb}{0.000000,0.000000,0.000000}%
\pgfsetstrokecolor{currentstroke}%
\pgfsetstrokeopacity{0.000000}%
\pgfsetdash{}{0pt}%
\pgfpathmoveto{\pgfqpoint{2.612642in}{0.500000in}}%
\pgfpathlineto{\pgfqpoint{2.645668in}{0.500000in}}%
\pgfpathlineto{\pgfqpoint{2.645668in}{1.893675in}}%
\pgfpathlineto{\pgfqpoint{2.612642in}{1.893675in}}%
\pgfpathlineto{\pgfqpoint{2.612642in}{0.500000in}}%
\pgfpathclose%
\pgfusepath{fill}%
\end{pgfscope}%
\begin{pgfscope}%
\pgfpathrectangle{\pgfqpoint{0.750000in}{0.500000in}}{\pgfqpoint{4.650000in}{3.020000in}}%
\pgfusepath{clip}%
\pgfsetbuttcap%
\pgfsetmiterjoin%
\definecolor{currentfill}{rgb}{0.000000,0.500000,0.000000}%
\pgfsetfillcolor{currentfill}%
\pgfsetlinewidth{0.000000pt}%
\definecolor{currentstroke}{rgb}{0.000000,0.000000,0.000000}%
\pgfsetstrokecolor{currentstroke}%
\pgfsetstrokeopacity{0.000000}%
\pgfsetdash{}{0pt}%
\pgfpathmoveto{\pgfqpoint{2.645668in}{0.500000in}}%
\pgfpathlineto{\pgfqpoint{2.678693in}{0.500000in}}%
\pgfpathlineto{\pgfqpoint{2.678693in}{1.549421in}}%
\pgfpathlineto{\pgfqpoint{2.645668in}{1.549421in}}%
\pgfpathlineto{\pgfqpoint{2.645668in}{0.500000in}}%
\pgfpathclose%
\pgfusepath{fill}%
\end{pgfscope}%
\begin{pgfscope}%
\pgfpathrectangle{\pgfqpoint{0.750000in}{0.500000in}}{\pgfqpoint{4.650000in}{3.020000in}}%
\pgfusepath{clip}%
\pgfsetbuttcap%
\pgfsetmiterjoin%
\definecolor{currentfill}{rgb}{0.000000,0.500000,0.000000}%
\pgfsetfillcolor{currentfill}%
\pgfsetlinewidth{0.000000pt}%
\definecolor{currentstroke}{rgb}{0.000000,0.000000,0.000000}%
\pgfsetstrokecolor{currentstroke}%
\pgfsetstrokeopacity{0.000000}%
\pgfsetdash{}{0pt}%
\pgfpathmoveto{\pgfqpoint{2.678693in}{0.500000in}}%
\pgfpathlineto{\pgfqpoint{2.711719in}{0.500000in}}%
\pgfpathlineto{\pgfqpoint{2.711719in}{1.144089in}}%
\pgfpathlineto{\pgfqpoint{2.678693in}{1.144089in}}%
\pgfpathlineto{\pgfqpoint{2.678693in}{0.500000in}}%
\pgfpathclose%
\pgfusepath{fill}%
\end{pgfscope}%
\begin{pgfscope}%
\pgfpathrectangle{\pgfqpoint{0.750000in}{0.500000in}}{\pgfqpoint{4.650000in}{3.020000in}}%
\pgfusepath{clip}%
\pgfsetbuttcap%
\pgfsetmiterjoin%
\definecolor{currentfill}{rgb}{0.000000,0.500000,0.000000}%
\pgfsetfillcolor{currentfill}%
\pgfsetlinewidth{0.000000pt}%
\definecolor{currentstroke}{rgb}{0.000000,0.000000,0.000000}%
\pgfsetstrokecolor{currentstroke}%
\pgfsetstrokeopacity{0.000000}%
\pgfsetdash{}{0pt}%
\pgfpathmoveto{\pgfqpoint{2.711719in}{0.500000in}}%
\pgfpathlineto{\pgfqpoint{2.744744in}{0.500000in}}%
\pgfpathlineto{\pgfqpoint{2.744744in}{1.721548in}}%
\pgfpathlineto{\pgfqpoint{2.711719in}{1.721548in}}%
\pgfpathlineto{\pgfqpoint{2.711719in}{0.500000in}}%
\pgfpathclose%
\pgfusepath{fill}%
\end{pgfscope}%
\begin{pgfscope}%
\pgfpathrectangle{\pgfqpoint{0.750000in}{0.500000in}}{\pgfqpoint{4.650000in}{3.020000in}}%
\pgfusepath{clip}%
\pgfsetbuttcap%
\pgfsetmiterjoin%
\definecolor{currentfill}{rgb}{0.000000,0.500000,0.000000}%
\pgfsetfillcolor{currentfill}%
\pgfsetlinewidth{0.000000pt}%
\definecolor{currentstroke}{rgb}{0.000000,0.000000,0.000000}%
\pgfsetstrokecolor{currentstroke}%
\pgfsetstrokeopacity{0.000000}%
\pgfsetdash{}{0pt}%
\pgfpathmoveto{\pgfqpoint{2.744744in}{0.500000in}}%
\pgfpathlineto{\pgfqpoint{2.777770in}{0.500000in}}%
\pgfpathlineto{\pgfqpoint{2.777770in}{1.110774in}}%
\pgfpathlineto{\pgfqpoint{2.744744in}{1.110774in}}%
\pgfpathlineto{\pgfqpoint{2.744744in}{0.500000in}}%
\pgfpathclose%
\pgfusepath{fill}%
\end{pgfscope}%
\begin{pgfscope}%
\pgfpathrectangle{\pgfqpoint{0.750000in}{0.500000in}}{\pgfqpoint{4.650000in}{3.020000in}}%
\pgfusepath{clip}%
\pgfsetbuttcap%
\pgfsetmiterjoin%
\definecolor{currentfill}{rgb}{0.000000,0.500000,0.000000}%
\pgfsetfillcolor{currentfill}%
\pgfsetlinewidth{0.000000pt}%
\definecolor{currentstroke}{rgb}{0.000000,0.000000,0.000000}%
\pgfsetstrokecolor{currentstroke}%
\pgfsetstrokeopacity{0.000000}%
\pgfsetdash{}{0pt}%
\pgfpathmoveto{\pgfqpoint{2.777770in}{0.500000in}}%
\pgfpathlineto{\pgfqpoint{2.810795in}{0.500000in}}%
\pgfpathlineto{\pgfqpoint{2.810795in}{1.188509in}}%
\pgfpathlineto{\pgfqpoint{2.777770in}{1.188509in}}%
\pgfpathlineto{\pgfqpoint{2.777770in}{0.500000in}}%
\pgfpathclose%
\pgfusepath{fill}%
\end{pgfscope}%
\begin{pgfscope}%
\pgfpathrectangle{\pgfqpoint{0.750000in}{0.500000in}}{\pgfqpoint{4.650000in}{3.020000in}}%
\pgfusepath{clip}%
\pgfsetbuttcap%
\pgfsetmiterjoin%
\definecolor{currentfill}{rgb}{0.000000,0.500000,0.000000}%
\pgfsetfillcolor{currentfill}%
\pgfsetlinewidth{0.000000pt}%
\definecolor{currentstroke}{rgb}{0.000000,0.000000,0.000000}%
\pgfsetstrokecolor{currentstroke}%
\pgfsetstrokeopacity{0.000000}%
\pgfsetdash{}{0pt}%
\pgfpathmoveto{\pgfqpoint{2.810795in}{0.500000in}}%
\pgfpathlineto{\pgfqpoint{2.843821in}{0.500000in}}%
\pgfpathlineto{\pgfqpoint{2.843821in}{1.260691in}}%
\pgfpathlineto{\pgfqpoint{2.810795in}{1.260691in}}%
\pgfpathlineto{\pgfqpoint{2.810795in}{0.500000in}}%
\pgfpathclose%
\pgfusepath{fill}%
\end{pgfscope}%
\begin{pgfscope}%
\pgfpathrectangle{\pgfqpoint{0.750000in}{0.500000in}}{\pgfqpoint{4.650000in}{3.020000in}}%
\pgfusepath{clip}%
\pgfsetbuttcap%
\pgfsetmiterjoin%
\definecolor{currentfill}{rgb}{0.000000,0.500000,0.000000}%
\pgfsetfillcolor{currentfill}%
\pgfsetlinewidth{0.000000pt}%
\definecolor{currentstroke}{rgb}{0.000000,0.000000,0.000000}%
\pgfsetstrokecolor{currentstroke}%
\pgfsetstrokeopacity{0.000000}%
\pgfsetdash{}{0pt}%
\pgfpathmoveto{\pgfqpoint{2.843821in}{0.500000in}}%
\pgfpathlineto{\pgfqpoint{2.876847in}{0.500000in}}%
\pgfpathlineto{\pgfqpoint{2.876847in}{0.933094in}}%
\pgfpathlineto{\pgfqpoint{2.843821in}{0.933094in}}%
\pgfpathlineto{\pgfqpoint{2.843821in}{0.500000in}}%
\pgfpathclose%
\pgfusepath{fill}%
\end{pgfscope}%
\begin{pgfscope}%
\pgfpathrectangle{\pgfqpoint{0.750000in}{0.500000in}}{\pgfqpoint{4.650000in}{3.020000in}}%
\pgfusepath{clip}%
\pgfsetbuttcap%
\pgfsetmiterjoin%
\definecolor{currentfill}{rgb}{0.000000,0.500000,0.000000}%
\pgfsetfillcolor{currentfill}%
\pgfsetlinewidth{0.000000pt}%
\definecolor{currentstroke}{rgb}{0.000000,0.000000,0.000000}%
\pgfsetstrokecolor{currentstroke}%
\pgfsetstrokeopacity{0.000000}%
\pgfsetdash{}{0pt}%
\pgfpathmoveto{\pgfqpoint{2.876847in}{0.500000in}}%
\pgfpathlineto{\pgfqpoint{2.909872in}{0.500000in}}%
\pgfpathlineto{\pgfqpoint{2.909872in}{0.999724in}}%
\pgfpathlineto{\pgfqpoint{2.876847in}{0.999724in}}%
\pgfpathlineto{\pgfqpoint{2.876847in}{0.500000in}}%
\pgfpathclose%
\pgfusepath{fill}%
\end{pgfscope}%
\begin{pgfscope}%
\pgfpathrectangle{\pgfqpoint{0.750000in}{0.500000in}}{\pgfqpoint{4.650000in}{3.020000in}}%
\pgfusepath{clip}%
\pgfsetbuttcap%
\pgfsetmiterjoin%
\definecolor{currentfill}{rgb}{0.000000,0.500000,0.000000}%
\pgfsetfillcolor{currentfill}%
\pgfsetlinewidth{0.000000pt}%
\definecolor{currentstroke}{rgb}{0.000000,0.000000,0.000000}%
\pgfsetstrokecolor{currentstroke}%
\pgfsetstrokeopacity{0.000000}%
\pgfsetdash{}{0pt}%
\pgfpathmoveto{\pgfqpoint{2.909872in}{0.500000in}}%
\pgfpathlineto{\pgfqpoint{2.942898in}{0.500000in}}%
\pgfpathlineto{\pgfqpoint{2.942898in}{0.783177in}}%
\pgfpathlineto{\pgfqpoint{2.909872in}{0.783177in}}%
\pgfpathlineto{\pgfqpoint{2.909872in}{0.500000in}}%
\pgfpathclose%
\pgfusepath{fill}%
\end{pgfscope}%
\begin{pgfscope}%
\pgfpathrectangle{\pgfqpoint{0.750000in}{0.500000in}}{\pgfqpoint{4.650000in}{3.020000in}}%
\pgfusepath{clip}%
\pgfsetbuttcap%
\pgfsetmiterjoin%
\definecolor{currentfill}{rgb}{0.000000,0.500000,0.000000}%
\pgfsetfillcolor{currentfill}%
\pgfsetlinewidth{0.000000pt}%
\definecolor{currentstroke}{rgb}{0.000000,0.000000,0.000000}%
\pgfsetstrokecolor{currentstroke}%
\pgfsetstrokeopacity{0.000000}%
\pgfsetdash{}{0pt}%
\pgfpathmoveto{\pgfqpoint{2.942898in}{0.500000in}}%
\pgfpathlineto{\pgfqpoint{2.975923in}{0.500000in}}%
\pgfpathlineto{\pgfqpoint{2.975923in}{0.672127in}}%
\pgfpathlineto{\pgfqpoint{2.942898in}{0.672127in}}%
\pgfpathlineto{\pgfqpoint{2.942898in}{0.500000in}}%
\pgfpathclose%
\pgfusepath{fill}%
\end{pgfscope}%
\begin{pgfscope}%
\pgfpathrectangle{\pgfqpoint{0.750000in}{0.500000in}}{\pgfqpoint{4.650000in}{3.020000in}}%
\pgfusepath{clip}%
\pgfsetbuttcap%
\pgfsetmiterjoin%
\definecolor{currentfill}{rgb}{0.000000,0.500000,0.000000}%
\pgfsetfillcolor{currentfill}%
\pgfsetlinewidth{0.000000pt}%
\definecolor{currentstroke}{rgb}{0.000000,0.000000,0.000000}%
\pgfsetstrokecolor{currentstroke}%
\pgfsetstrokeopacity{0.000000}%
\pgfsetdash{}{0pt}%
\pgfpathmoveto{\pgfqpoint{2.975923in}{0.500000in}}%
\pgfpathlineto{\pgfqpoint{3.008949in}{0.500000in}}%
\pgfpathlineto{\pgfqpoint{3.008949in}{0.816492in}}%
\pgfpathlineto{\pgfqpoint{2.975923in}{0.816492in}}%
\pgfpathlineto{\pgfqpoint{2.975923in}{0.500000in}}%
\pgfpathclose%
\pgfusepath{fill}%
\end{pgfscope}%
\begin{pgfscope}%
\pgfpathrectangle{\pgfqpoint{0.750000in}{0.500000in}}{\pgfqpoint{4.650000in}{3.020000in}}%
\pgfusepath{clip}%
\pgfsetbuttcap%
\pgfsetmiterjoin%
\definecolor{currentfill}{rgb}{0.000000,0.500000,0.000000}%
\pgfsetfillcolor{currentfill}%
\pgfsetlinewidth{0.000000pt}%
\definecolor{currentstroke}{rgb}{0.000000,0.000000,0.000000}%
\pgfsetstrokecolor{currentstroke}%
\pgfsetstrokeopacity{0.000000}%
\pgfsetdash{}{0pt}%
\pgfpathmoveto{\pgfqpoint{3.008949in}{0.500000in}}%
\pgfpathlineto{\pgfqpoint{3.041974in}{0.500000in}}%
\pgfpathlineto{\pgfqpoint{3.041974in}{0.655470in}}%
\pgfpathlineto{\pgfqpoint{3.008949in}{0.655470in}}%
\pgfpathlineto{\pgfqpoint{3.008949in}{0.500000in}}%
\pgfpathclose%
\pgfusepath{fill}%
\end{pgfscope}%
\begin{pgfscope}%
\pgfpathrectangle{\pgfqpoint{0.750000in}{0.500000in}}{\pgfqpoint{4.650000in}{3.020000in}}%
\pgfusepath{clip}%
\pgfsetbuttcap%
\pgfsetmiterjoin%
\definecolor{currentfill}{rgb}{0.000000,0.500000,0.000000}%
\pgfsetfillcolor{currentfill}%
\pgfsetlinewidth{0.000000pt}%
\definecolor{currentstroke}{rgb}{0.000000,0.000000,0.000000}%
\pgfsetstrokecolor{currentstroke}%
\pgfsetstrokeopacity{0.000000}%
\pgfsetdash{}{0pt}%
\pgfpathmoveto{\pgfqpoint{3.041974in}{0.500000in}}%
\pgfpathlineto{\pgfqpoint{3.075000in}{0.500000in}}%
\pgfpathlineto{\pgfqpoint{3.075000in}{0.677680in}}%
\pgfpathlineto{\pgfqpoint{3.041974in}{0.677680in}}%
\pgfpathlineto{\pgfqpoint{3.041974in}{0.500000in}}%
\pgfpathclose%
\pgfusepath{fill}%
\end{pgfscope}%
\begin{pgfscope}%
\pgfpathrectangle{\pgfqpoint{0.750000in}{0.500000in}}{\pgfqpoint{4.650000in}{3.020000in}}%
\pgfusepath{clip}%
\pgfsetbuttcap%
\pgfsetmiterjoin%
\definecolor{currentfill}{rgb}{0.000000,0.500000,0.000000}%
\pgfsetfillcolor{currentfill}%
\pgfsetlinewidth{0.000000pt}%
\definecolor{currentstroke}{rgb}{0.000000,0.000000,0.000000}%
\pgfsetstrokecolor{currentstroke}%
\pgfsetstrokeopacity{0.000000}%
\pgfsetdash{}{0pt}%
\pgfpathmoveto{\pgfqpoint{3.075000in}{0.500000in}}%
\pgfpathlineto{\pgfqpoint{3.108026in}{0.500000in}}%
\pgfpathlineto{\pgfqpoint{3.108026in}{0.655470in}}%
\pgfpathlineto{\pgfqpoint{3.075000in}{0.655470in}}%
\pgfpathlineto{\pgfqpoint{3.075000in}{0.500000in}}%
\pgfpathclose%
\pgfusepath{fill}%
\end{pgfscope}%
\begin{pgfscope}%
\pgfpathrectangle{\pgfqpoint{0.750000in}{0.500000in}}{\pgfqpoint{4.650000in}{3.020000in}}%
\pgfusepath{clip}%
\pgfsetbuttcap%
\pgfsetmiterjoin%
\definecolor{currentfill}{rgb}{0.000000,0.500000,0.000000}%
\pgfsetfillcolor{currentfill}%
\pgfsetlinewidth{0.000000pt}%
\definecolor{currentstroke}{rgb}{0.000000,0.000000,0.000000}%
\pgfsetstrokecolor{currentstroke}%
\pgfsetstrokeopacity{0.000000}%
\pgfsetdash{}{0pt}%
\pgfpathmoveto{\pgfqpoint{3.108026in}{0.500000in}}%
\pgfpathlineto{\pgfqpoint{3.141051in}{0.500000in}}%
\pgfpathlineto{\pgfqpoint{3.141051in}{0.577735in}}%
\pgfpathlineto{\pgfqpoint{3.108026in}{0.577735in}}%
\pgfpathlineto{\pgfqpoint{3.108026in}{0.500000in}}%
\pgfpathclose%
\pgfusepath{fill}%
\end{pgfscope}%
\begin{pgfscope}%
\pgfpathrectangle{\pgfqpoint{0.750000in}{0.500000in}}{\pgfqpoint{4.650000in}{3.020000in}}%
\pgfusepath{clip}%
\pgfsetbuttcap%
\pgfsetmiterjoin%
\definecolor{currentfill}{rgb}{0.000000,0.500000,0.000000}%
\pgfsetfillcolor{currentfill}%
\pgfsetlinewidth{0.000000pt}%
\definecolor{currentstroke}{rgb}{0.000000,0.000000,0.000000}%
\pgfsetstrokecolor{currentstroke}%
\pgfsetstrokeopacity{0.000000}%
\pgfsetdash{}{0pt}%
\pgfpathmoveto{\pgfqpoint{3.141051in}{0.500000in}}%
\pgfpathlineto{\pgfqpoint{3.174077in}{0.500000in}}%
\pgfpathlineto{\pgfqpoint{3.174077in}{0.616602in}}%
\pgfpathlineto{\pgfqpoint{3.141051in}{0.616602in}}%
\pgfpathlineto{\pgfqpoint{3.141051in}{0.500000in}}%
\pgfpathclose%
\pgfusepath{fill}%
\end{pgfscope}%
\begin{pgfscope}%
\pgfpathrectangle{\pgfqpoint{0.750000in}{0.500000in}}{\pgfqpoint{4.650000in}{3.020000in}}%
\pgfusepath{clip}%
\pgfsetbuttcap%
\pgfsetmiterjoin%
\definecolor{currentfill}{rgb}{0.000000,0.500000,0.000000}%
\pgfsetfillcolor{currentfill}%
\pgfsetlinewidth{0.000000pt}%
\definecolor{currentstroke}{rgb}{0.000000,0.000000,0.000000}%
\pgfsetstrokecolor{currentstroke}%
\pgfsetstrokeopacity{0.000000}%
\pgfsetdash{}{0pt}%
\pgfpathmoveto{\pgfqpoint{3.174077in}{0.500000in}}%
\pgfpathlineto{\pgfqpoint{3.207102in}{0.500000in}}%
\pgfpathlineto{\pgfqpoint{3.207102in}{0.561077in}}%
\pgfpathlineto{\pgfqpoint{3.174077in}{0.561077in}}%
\pgfpathlineto{\pgfqpoint{3.174077in}{0.500000in}}%
\pgfpathclose%
\pgfusepath{fill}%
\end{pgfscope}%
\begin{pgfscope}%
\pgfpathrectangle{\pgfqpoint{0.750000in}{0.500000in}}{\pgfqpoint{4.650000in}{3.020000in}}%
\pgfusepath{clip}%
\pgfsetbuttcap%
\pgfsetmiterjoin%
\definecolor{currentfill}{rgb}{0.000000,0.500000,0.000000}%
\pgfsetfillcolor{currentfill}%
\pgfsetlinewidth{0.000000pt}%
\definecolor{currentstroke}{rgb}{0.000000,0.000000,0.000000}%
\pgfsetstrokecolor{currentstroke}%
\pgfsetstrokeopacity{0.000000}%
\pgfsetdash{}{0pt}%
\pgfpathmoveto{\pgfqpoint{3.207102in}{0.500000in}}%
\pgfpathlineto{\pgfqpoint{3.240128in}{0.500000in}}%
\pgfpathlineto{\pgfqpoint{3.240128in}{0.533315in}}%
\pgfpathlineto{\pgfqpoint{3.207102in}{0.533315in}}%
\pgfpathlineto{\pgfqpoint{3.207102in}{0.500000in}}%
\pgfpathclose%
\pgfusepath{fill}%
\end{pgfscope}%
\begin{pgfscope}%
\pgfpathrectangle{\pgfqpoint{0.750000in}{0.500000in}}{\pgfqpoint{4.650000in}{3.020000in}}%
\pgfusepath{clip}%
\pgfsetbuttcap%
\pgfsetmiterjoin%
\definecolor{currentfill}{rgb}{0.000000,0.500000,0.000000}%
\pgfsetfillcolor{currentfill}%
\pgfsetlinewidth{0.000000pt}%
\definecolor{currentstroke}{rgb}{0.000000,0.000000,0.000000}%
\pgfsetstrokecolor{currentstroke}%
\pgfsetstrokeopacity{0.000000}%
\pgfsetdash{}{0pt}%
\pgfpathmoveto{\pgfqpoint{3.240128in}{0.500000in}}%
\pgfpathlineto{\pgfqpoint{3.273153in}{0.500000in}}%
\pgfpathlineto{\pgfqpoint{3.273153in}{0.549972in}}%
\pgfpathlineto{\pgfqpoint{3.240128in}{0.549972in}}%
\pgfpathlineto{\pgfqpoint{3.240128in}{0.500000in}}%
\pgfpathclose%
\pgfusepath{fill}%
\end{pgfscope}%
\begin{pgfscope}%
\pgfpathrectangle{\pgfqpoint{0.750000in}{0.500000in}}{\pgfqpoint{4.650000in}{3.020000in}}%
\pgfusepath{clip}%
\pgfsetbuttcap%
\pgfsetmiterjoin%
\definecolor{currentfill}{rgb}{0.000000,0.500000,0.000000}%
\pgfsetfillcolor{currentfill}%
\pgfsetlinewidth{0.000000pt}%
\definecolor{currentstroke}{rgb}{0.000000,0.000000,0.000000}%
\pgfsetstrokecolor{currentstroke}%
\pgfsetstrokeopacity{0.000000}%
\pgfsetdash{}{0pt}%
\pgfpathmoveto{\pgfqpoint{3.273153in}{0.500000in}}%
\pgfpathlineto{\pgfqpoint{3.306179in}{0.500000in}}%
\pgfpathlineto{\pgfqpoint{3.306179in}{0.505552in}}%
\pgfpathlineto{\pgfqpoint{3.273153in}{0.505552in}}%
\pgfpathlineto{\pgfqpoint{3.273153in}{0.500000in}}%
\pgfpathclose%
\pgfusepath{fill}%
\end{pgfscope}%
\begin{pgfscope}%
\pgfpathrectangle{\pgfqpoint{0.750000in}{0.500000in}}{\pgfqpoint{4.650000in}{3.020000in}}%
\pgfusepath{clip}%
\pgfsetbuttcap%
\pgfsetmiterjoin%
\definecolor{currentfill}{rgb}{0.000000,0.500000,0.000000}%
\pgfsetfillcolor{currentfill}%
\pgfsetlinewidth{0.000000pt}%
\definecolor{currentstroke}{rgb}{0.000000,0.000000,0.000000}%
\pgfsetstrokecolor{currentstroke}%
\pgfsetstrokeopacity{0.000000}%
\pgfsetdash{}{0pt}%
\pgfpathmoveto{\pgfqpoint{3.306179in}{0.500000in}}%
\pgfpathlineto{\pgfqpoint{3.339205in}{0.500000in}}%
\pgfpathlineto{\pgfqpoint{3.339205in}{0.527762in}}%
\pgfpathlineto{\pgfqpoint{3.306179in}{0.527762in}}%
\pgfpathlineto{\pgfqpoint{3.306179in}{0.500000in}}%
\pgfpathclose%
\pgfusepath{fill}%
\end{pgfscope}%
\begin{pgfscope}%
\pgfpathrectangle{\pgfqpoint{0.750000in}{0.500000in}}{\pgfqpoint{4.650000in}{3.020000in}}%
\pgfusepath{clip}%
\pgfsetbuttcap%
\pgfsetmiterjoin%
\definecolor{currentfill}{rgb}{0.000000,0.500000,0.000000}%
\pgfsetfillcolor{currentfill}%
\pgfsetlinewidth{0.000000pt}%
\definecolor{currentstroke}{rgb}{0.000000,0.000000,0.000000}%
\pgfsetstrokecolor{currentstroke}%
\pgfsetstrokeopacity{0.000000}%
\pgfsetdash{}{0pt}%
\pgfpathmoveto{\pgfqpoint{3.339205in}{0.500000in}}%
\pgfpathlineto{\pgfqpoint{3.372230in}{0.500000in}}%
\pgfpathlineto{\pgfqpoint{3.372230in}{0.522210in}}%
\pgfpathlineto{\pgfqpoint{3.339205in}{0.522210in}}%
\pgfpathlineto{\pgfqpoint{3.339205in}{0.500000in}}%
\pgfpathclose%
\pgfusepath{fill}%
\end{pgfscope}%
\begin{pgfscope}%
\pgfpathrectangle{\pgfqpoint{0.750000in}{0.500000in}}{\pgfqpoint{4.650000in}{3.020000in}}%
\pgfusepath{clip}%
\pgfsetbuttcap%
\pgfsetmiterjoin%
\definecolor{currentfill}{rgb}{0.000000,0.500000,0.000000}%
\pgfsetfillcolor{currentfill}%
\pgfsetlinewidth{0.000000pt}%
\definecolor{currentstroke}{rgb}{0.000000,0.000000,0.000000}%
\pgfsetstrokecolor{currentstroke}%
\pgfsetstrokeopacity{0.000000}%
\pgfsetdash{}{0pt}%
\pgfpathmoveto{\pgfqpoint{3.372230in}{0.500000in}}%
\pgfpathlineto{\pgfqpoint{3.405256in}{0.500000in}}%
\pgfpathlineto{\pgfqpoint{3.405256in}{0.505552in}}%
\pgfpathlineto{\pgfqpoint{3.372230in}{0.505552in}}%
\pgfpathlineto{\pgfqpoint{3.372230in}{0.500000in}}%
\pgfpathclose%
\pgfusepath{fill}%
\end{pgfscope}%
\begin{pgfscope}%
\pgfpathrectangle{\pgfqpoint{0.750000in}{0.500000in}}{\pgfqpoint{4.650000in}{3.020000in}}%
\pgfusepath{clip}%
\pgfsetbuttcap%
\pgfsetmiterjoin%
\definecolor{currentfill}{rgb}{0.000000,0.500000,0.000000}%
\pgfsetfillcolor{currentfill}%
\pgfsetlinewidth{0.000000pt}%
\definecolor{currentstroke}{rgb}{0.000000,0.000000,0.000000}%
\pgfsetstrokecolor{currentstroke}%
\pgfsetstrokeopacity{0.000000}%
\pgfsetdash{}{0pt}%
\pgfpathmoveto{\pgfqpoint{3.405256in}{0.500000in}}%
\pgfpathlineto{\pgfqpoint{3.438281in}{0.500000in}}%
\pgfpathlineto{\pgfqpoint{3.438281in}{0.505552in}}%
\pgfpathlineto{\pgfqpoint{3.405256in}{0.505552in}}%
\pgfpathlineto{\pgfqpoint{3.405256in}{0.500000in}}%
\pgfpathclose%
\pgfusepath{fill}%
\end{pgfscope}%
\begin{pgfscope}%
\pgfpathrectangle{\pgfqpoint{0.750000in}{0.500000in}}{\pgfqpoint{4.650000in}{3.020000in}}%
\pgfusepath{clip}%
\pgfsetbuttcap%
\pgfsetmiterjoin%
\definecolor{currentfill}{rgb}{0.000000,0.500000,0.000000}%
\pgfsetfillcolor{currentfill}%
\pgfsetlinewidth{0.000000pt}%
\definecolor{currentstroke}{rgb}{0.000000,0.000000,0.000000}%
\pgfsetstrokecolor{currentstroke}%
\pgfsetstrokeopacity{0.000000}%
\pgfsetdash{}{0pt}%
\pgfpathmoveto{\pgfqpoint{3.438281in}{0.500000in}}%
\pgfpathlineto{\pgfqpoint{3.471307in}{0.500000in}}%
\pgfpathlineto{\pgfqpoint{3.471307in}{0.511105in}}%
\pgfpathlineto{\pgfqpoint{3.438281in}{0.511105in}}%
\pgfpathlineto{\pgfqpoint{3.438281in}{0.500000in}}%
\pgfpathclose%
\pgfusepath{fill}%
\end{pgfscope}%
\begin{pgfscope}%
\pgfpathrectangle{\pgfqpoint{0.750000in}{0.500000in}}{\pgfqpoint{4.650000in}{3.020000in}}%
\pgfusepath{clip}%
\pgfsetbuttcap%
\pgfsetmiterjoin%
\definecolor{currentfill}{rgb}{0.000000,0.500000,0.000000}%
\pgfsetfillcolor{currentfill}%
\pgfsetlinewidth{0.000000pt}%
\definecolor{currentstroke}{rgb}{0.000000,0.000000,0.000000}%
\pgfsetstrokecolor{currentstroke}%
\pgfsetstrokeopacity{0.000000}%
\pgfsetdash{}{0pt}%
\pgfpathmoveto{\pgfqpoint{3.471307in}{0.500000in}}%
\pgfpathlineto{\pgfqpoint{3.504332in}{0.500000in}}%
\pgfpathlineto{\pgfqpoint{3.504332in}{0.505552in}}%
\pgfpathlineto{\pgfqpoint{3.471307in}{0.505552in}}%
\pgfpathlineto{\pgfqpoint{3.471307in}{0.500000in}}%
\pgfpathclose%
\pgfusepath{fill}%
\end{pgfscope}%
\begin{pgfscope}%
\pgfpathrectangle{\pgfqpoint{0.750000in}{0.500000in}}{\pgfqpoint{4.650000in}{3.020000in}}%
\pgfusepath{clip}%
\pgfsetbuttcap%
\pgfsetmiterjoin%
\definecolor{currentfill}{rgb}{0.000000,0.500000,0.000000}%
\pgfsetfillcolor{currentfill}%
\pgfsetlinewidth{0.000000pt}%
\definecolor{currentstroke}{rgb}{0.000000,0.000000,0.000000}%
\pgfsetstrokecolor{currentstroke}%
\pgfsetstrokeopacity{0.000000}%
\pgfsetdash{}{0pt}%
\pgfpathmoveto{\pgfqpoint{3.504332in}{0.500000in}}%
\pgfpathlineto{\pgfqpoint{3.537358in}{0.500000in}}%
\pgfpathlineto{\pgfqpoint{3.537358in}{0.505552in}}%
\pgfpathlineto{\pgfqpoint{3.504332in}{0.505552in}}%
\pgfpathlineto{\pgfqpoint{3.504332in}{0.500000in}}%
\pgfpathclose%
\pgfusepath{fill}%
\end{pgfscope}%
\begin{pgfscope}%
\pgfpathrectangle{\pgfqpoint{0.750000in}{0.500000in}}{\pgfqpoint{4.650000in}{3.020000in}}%
\pgfusepath{clip}%
\pgfsetbuttcap%
\pgfsetmiterjoin%
\definecolor{currentfill}{rgb}{0.000000,0.500000,0.000000}%
\pgfsetfillcolor{currentfill}%
\pgfsetlinewidth{0.000000pt}%
\definecolor{currentstroke}{rgb}{0.000000,0.000000,0.000000}%
\pgfsetstrokecolor{currentstroke}%
\pgfsetstrokeopacity{0.000000}%
\pgfsetdash{}{0pt}%
\pgfpathmoveto{\pgfqpoint{3.537358in}{0.500000in}}%
\pgfpathlineto{\pgfqpoint{3.570384in}{0.500000in}}%
\pgfpathlineto{\pgfqpoint{3.570384in}{0.511105in}}%
\pgfpathlineto{\pgfqpoint{3.537358in}{0.511105in}}%
\pgfpathlineto{\pgfqpoint{3.537358in}{0.500000in}}%
\pgfpathclose%
\pgfusepath{fill}%
\end{pgfscope}%
\begin{pgfscope}%
\pgfpathrectangle{\pgfqpoint{0.750000in}{0.500000in}}{\pgfqpoint{4.650000in}{3.020000in}}%
\pgfusepath{clip}%
\pgfsetbuttcap%
\pgfsetmiterjoin%
\definecolor{currentfill}{rgb}{0.000000,0.500000,0.000000}%
\pgfsetfillcolor{currentfill}%
\pgfsetlinewidth{0.000000pt}%
\definecolor{currentstroke}{rgb}{0.000000,0.000000,0.000000}%
\pgfsetstrokecolor{currentstroke}%
\pgfsetstrokeopacity{0.000000}%
\pgfsetdash{}{0pt}%
\pgfpathmoveto{\pgfqpoint{3.570384in}{0.500000in}}%
\pgfpathlineto{\pgfqpoint{3.603409in}{0.500000in}}%
\pgfpathlineto{\pgfqpoint{3.603409in}{0.505552in}}%
\pgfpathlineto{\pgfqpoint{3.570384in}{0.505552in}}%
\pgfpathlineto{\pgfqpoint{3.570384in}{0.500000in}}%
\pgfpathclose%
\pgfusepath{fill}%
\end{pgfscope}%
\begin{pgfscope}%
\pgfpathrectangle{\pgfqpoint{0.750000in}{0.500000in}}{\pgfqpoint{4.650000in}{3.020000in}}%
\pgfusepath{clip}%
\pgfsetbuttcap%
\pgfsetmiterjoin%
\definecolor{currentfill}{rgb}{0.000000,0.500000,0.000000}%
\pgfsetfillcolor{currentfill}%
\pgfsetlinewidth{0.000000pt}%
\definecolor{currentstroke}{rgb}{0.000000,0.000000,0.000000}%
\pgfsetstrokecolor{currentstroke}%
\pgfsetstrokeopacity{0.000000}%
\pgfsetdash{}{0pt}%
\pgfpathmoveto{\pgfqpoint{3.603409in}{0.500000in}}%
\pgfpathlineto{\pgfqpoint{3.636435in}{0.500000in}}%
\pgfpathlineto{\pgfqpoint{3.636435in}{0.500000in}}%
\pgfpathlineto{\pgfqpoint{3.603409in}{0.500000in}}%
\pgfpathlineto{\pgfqpoint{3.603409in}{0.500000in}}%
\pgfpathclose%
\pgfusepath{fill}%
\end{pgfscope}%
\begin{pgfscope}%
\pgfpathrectangle{\pgfqpoint{0.750000in}{0.500000in}}{\pgfqpoint{4.650000in}{3.020000in}}%
\pgfusepath{clip}%
\pgfsetbuttcap%
\pgfsetmiterjoin%
\definecolor{currentfill}{rgb}{0.000000,0.500000,0.000000}%
\pgfsetfillcolor{currentfill}%
\pgfsetlinewidth{0.000000pt}%
\definecolor{currentstroke}{rgb}{0.000000,0.000000,0.000000}%
\pgfsetstrokecolor{currentstroke}%
\pgfsetstrokeopacity{0.000000}%
\pgfsetdash{}{0pt}%
\pgfpathmoveto{\pgfqpoint{3.636435in}{0.500000in}}%
\pgfpathlineto{\pgfqpoint{3.669460in}{0.500000in}}%
\pgfpathlineto{\pgfqpoint{3.669460in}{0.500000in}}%
\pgfpathlineto{\pgfqpoint{3.636435in}{0.500000in}}%
\pgfpathlineto{\pgfqpoint{3.636435in}{0.500000in}}%
\pgfpathclose%
\pgfusepath{fill}%
\end{pgfscope}%
\begin{pgfscope}%
\pgfpathrectangle{\pgfqpoint{0.750000in}{0.500000in}}{\pgfqpoint{4.650000in}{3.020000in}}%
\pgfusepath{clip}%
\pgfsetbuttcap%
\pgfsetmiterjoin%
\definecolor{currentfill}{rgb}{0.000000,0.500000,0.000000}%
\pgfsetfillcolor{currentfill}%
\pgfsetlinewidth{0.000000pt}%
\definecolor{currentstroke}{rgb}{0.000000,0.000000,0.000000}%
\pgfsetstrokecolor{currentstroke}%
\pgfsetstrokeopacity{0.000000}%
\pgfsetdash{}{0pt}%
\pgfpathmoveto{\pgfqpoint{3.669460in}{0.500000in}}%
\pgfpathlineto{\pgfqpoint{3.702486in}{0.500000in}}%
\pgfpathlineto{\pgfqpoint{3.702486in}{0.500000in}}%
\pgfpathlineto{\pgfqpoint{3.669460in}{0.500000in}}%
\pgfpathlineto{\pgfqpoint{3.669460in}{0.500000in}}%
\pgfpathclose%
\pgfusepath{fill}%
\end{pgfscope}%
\begin{pgfscope}%
\pgfpathrectangle{\pgfqpoint{0.750000in}{0.500000in}}{\pgfqpoint{4.650000in}{3.020000in}}%
\pgfusepath{clip}%
\pgfsetbuttcap%
\pgfsetmiterjoin%
\definecolor{currentfill}{rgb}{0.000000,0.500000,0.000000}%
\pgfsetfillcolor{currentfill}%
\pgfsetlinewidth{0.000000pt}%
\definecolor{currentstroke}{rgb}{0.000000,0.000000,0.000000}%
\pgfsetstrokecolor{currentstroke}%
\pgfsetstrokeopacity{0.000000}%
\pgfsetdash{}{0pt}%
\pgfpathmoveto{\pgfqpoint{3.702486in}{0.500000in}}%
\pgfpathlineto{\pgfqpoint{3.735511in}{0.500000in}}%
\pgfpathlineto{\pgfqpoint{3.735511in}{0.500000in}}%
\pgfpathlineto{\pgfqpoint{3.702486in}{0.500000in}}%
\pgfpathlineto{\pgfqpoint{3.702486in}{0.500000in}}%
\pgfpathclose%
\pgfusepath{fill}%
\end{pgfscope}%
\begin{pgfscope}%
\pgfpathrectangle{\pgfqpoint{0.750000in}{0.500000in}}{\pgfqpoint{4.650000in}{3.020000in}}%
\pgfusepath{clip}%
\pgfsetbuttcap%
\pgfsetmiterjoin%
\definecolor{currentfill}{rgb}{0.000000,0.500000,0.000000}%
\pgfsetfillcolor{currentfill}%
\pgfsetlinewidth{0.000000pt}%
\definecolor{currentstroke}{rgb}{0.000000,0.000000,0.000000}%
\pgfsetstrokecolor{currentstroke}%
\pgfsetstrokeopacity{0.000000}%
\pgfsetdash{}{0pt}%
\pgfpathmoveto{\pgfqpoint{3.735511in}{0.500000in}}%
\pgfpathlineto{\pgfqpoint{3.768537in}{0.500000in}}%
\pgfpathlineto{\pgfqpoint{3.768537in}{0.500000in}}%
\pgfpathlineto{\pgfqpoint{3.735511in}{0.500000in}}%
\pgfpathlineto{\pgfqpoint{3.735511in}{0.500000in}}%
\pgfpathclose%
\pgfusepath{fill}%
\end{pgfscope}%
\begin{pgfscope}%
\pgfpathrectangle{\pgfqpoint{0.750000in}{0.500000in}}{\pgfqpoint{4.650000in}{3.020000in}}%
\pgfusepath{clip}%
\pgfsetbuttcap%
\pgfsetmiterjoin%
\definecolor{currentfill}{rgb}{0.000000,0.500000,0.000000}%
\pgfsetfillcolor{currentfill}%
\pgfsetlinewidth{0.000000pt}%
\definecolor{currentstroke}{rgb}{0.000000,0.000000,0.000000}%
\pgfsetstrokecolor{currentstroke}%
\pgfsetstrokeopacity{0.000000}%
\pgfsetdash{}{0pt}%
\pgfpathmoveto{\pgfqpoint{3.768537in}{0.500000in}}%
\pgfpathlineto{\pgfqpoint{3.801563in}{0.500000in}}%
\pgfpathlineto{\pgfqpoint{3.801563in}{0.505552in}}%
\pgfpathlineto{\pgfqpoint{3.768537in}{0.505552in}}%
\pgfpathlineto{\pgfqpoint{3.768537in}{0.500000in}}%
\pgfpathclose%
\pgfusepath{fill}%
\end{pgfscope}%
\begin{pgfscope}%
\pgfpathrectangle{\pgfqpoint{0.750000in}{0.500000in}}{\pgfqpoint{4.650000in}{3.020000in}}%
\pgfusepath{clip}%
\pgfsetbuttcap%
\pgfsetmiterjoin%
\definecolor{currentfill}{rgb}{0.000000,0.500000,0.000000}%
\pgfsetfillcolor{currentfill}%
\pgfsetlinewidth{0.000000pt}%
\definecolor{currentstroke}{rgb}{0.000000,0.000000,0.000000}%
\pgfsetstrokecolor{currentstroke}%
\pgfsetstrokeopacity{0.000000}%
\pgfsetdash{}{0pt}%
\pgfpathmoveto{\pgfqpoint{3.801562in}{0.500000in}}%
\pgfpathlineto{\pgfqpoint{3.834588in}{0.500000in}}%
\pgfpathlineto{\pgfqpoint{3.834588in}{0.500000in}}%
\pgfpathlineto{\pgfqpoint{3.801562in}{0.500000in}}%
\pgfpathlineto{\pgfqpoint{3.801562in}{0.500000in}}%
\pgfpathclose%
\pgfusepath{fill}%
\end{pgfscope}%
\begin{pgfscope}%
\pgfpathrectangle{\pgfqpoint{0.750000in}{0.500000in}}{\pgfqpoint{4.650000in}{3.020000in}}%
\pgfusepath{clip}%
\pgfsetbuttcap%
\pgfsetmiterjoin%
\definecolor{currentfill}{rgb}{0.000000,0.500000,0.000000}%
\pgfsetfillcolor{currentfill}%
\pgfsetlinewidth{0.000000pt}%
\definecolor{currentstroke}{rgb}{0.000000,0.000000,0.000000}%
\pgfsetstrokecolor{currentstroke}%
\pgfsetstrokeopacity{0.000000}%
\pgfsetdash{}{0pt}%
\pgfpathmoveto{\pgfqpoint{3.834588in}{0.500000in}}%
\pgfpathlineto{\pgfqpoint{3.867614in}{0.500000in}}%
\pgfpathlineto{\pgfqpoint{3.867614in}{0.500000in}}%
\pgfpathlineto{\pgfqpoint{3.834588in}{0.500000in}}%
\pgfpathlineto{\pgfqpoint{3.834588in}{0.500000in}}%
\pgfpathclose%
\pgfusepath{fill}%
\end{pgfscope}%
\begin{pgfscope}%
\pgfpathrectangle{\pgfqpoint{0.750000in}{0.500000in}}{\pgfqpoint{4.650000in}{3.020000in}}%
\pgfusepath{clip}%
\pgfsetbuttcap%
\pgfsetmiterjoin%
\definecolor{currentfill}{rgb}{0.000000,0.500000,0.000000}%
\pgfsetfillcolor{currentfill}%
\pgfsetlinewidth{0.000000pt}%
\definecolor{currentstroke}{rgb}{0.000000,0.000000,0.000000}%
\pgfsetstrokecolor{currentstroke}%
\pgfsetstrokeopacity{0.000000}%
\pgfsetdash{}{0pt}%
\pgfpathmoveto{\pgfqpoint{3.867614in}{0.500000in}}%
\pgfpathlineto{\pgfqpoint{3.900639in}{0.500000in}}%
\pgfpathlineto{\pgfqpoint{3.900639in}{0.500000in}}%
\pgfpathlineto{\pgfqpoint{3.867614in}{0.500000in}}%
\pgfpathlineto{\pgfqpoint{3.867614in}{0.500000in}}%
\pgfpathclose%
\pgfusepath{fill}%
\end{pgfscope}%
\begin{pgfscope}%
\pgfpathrectangle{\pgfqpoint{0.750000in}{0.500000in}}{\pgfqpoint{4.650000in}{3.020000in}}%
\pgfusepath{clip}%
\pgfsetbuttcap%
\pgfsetmiterjoin%
\definecolor{currentfill}{rgb}{0.000000,0.500000,0.000000}%
\pgfsetfillcolor{currentfill}%
\pgfsetlinewidth{0.000000pt}%
\definecolor{currentstroke}{rgb}{0.000000,0.000000,0.000000}%
\pgfsetstrokecolor{currentstroke}%
\pgfsetstrokeopacity{0.000000}%
\pgfsetdash{}{0pt}%
\pgfpathmoveto{\pgfqpoint{3.900639in}{0.500000in}}%
\pgfpathlineto{\pgfqpoint{3.933665in}{0.500000in}}%
\pgfpathlineto{\pgfqpoint{3.933665in}{0.500000in}}%
\pgfpathlineto{\pgfqpoint{3.900639in}{0.500000in}}%
\pgfpathlineto{\pgfqpoint{3.900639in}{0.500000in}}%
\pgfpathclose%
\pgfusepath{fill}%
\end{pgfscope}%
\begin{pgfscope}%
\pgfpathrectangle{\pgfqpoint{0.750000in}{0.500000in}}{\pgfqpoint{4.650000in}{3.020000in}}%
\pgfusepath{clip}%
\pgfsetbuttcap%
\pgfsetmiterjoin%
\definecolor{currentfill}{rgb}{0.000000,0.500000,0.000000}%
\pgfsetfillcolor{currentfill}%
\pgfsetlinewidth{0.000000pt}%
\definecolor{currentstroke}{rgb}{0.000000,0.000000,0.000000}%
\pgfsetstrokecolor{currentstroke}%
\pgfsetstrokeopacity{0.000000}%
\pgfsetdash{}{0pt}%
\pgfpathmoveto{\pgfqpoint{3.933665in}{0.500000in}}%
\pgfpathlineto{\pgfqpoint{3.966690in}{0.500000in}}%
\pgfpathlineto{\pgfqpoint{3.966690in}{0.500000in}}%
\pgfpathlineto{\pgfqpoint{3.933665in}{0.500000in}}%
\pgfpathlineto{\pgfqpoint{3.933665in}{0.500000in}}%
\pgfpathclose%
\pgfusepath{fill}%
\end{pgfscope}%
\begin{pgfscope}%
\pgfpathrectangle{\pgfqpoint{0.750000in}{0.500000in}}{\pgfqpoint{4.650000in}{3.020000in}}%
\pgfusepath{clip}%
\pgfsetbuttcap%
\pgfsetmiterjoin%
\definecolor{currentfill}{rgb}{0.000000,0.500000,0.000000}%
\pgfsetfillcolor{currentfill}%
\pgfsetlinewidth{0.000000pt}%
\definecolor{currentstroke}{rgb}{0.000000,0.000000,0.000000}%
\pgfsetstrokecolor{currentstroke}%
\pgfsetstrokeopacity{0.000000}%
\pgfsetdash{}{0pt}%
\pgfpathmoveto{\pgfqpoint{3.966690in}{0.500000in}}%
\pgfpathlineto{\pgfqpoint{3.999716in}{0.500000in}}%
\pgfpathlineto{\pgfqpoint{3.999716in}{0.500000in}}%
\pgfpathlineto{\pgfqpoint{3.966690in}{0.500000in}}%
\pgfpathlineto{\pgfqpoint{3.966690in}{0.500000in}}%
\pgfpathclose%
\pgfusepath{fill}%
\end{pgfscope}%
\begin{pgfscope}%
\pgfpathrectangle{\pgfqpoint{0.750000in}{0.500000in}}{\pgfqpoint{4.650000in}{3.020000in}}%
\pgfusepath{clip}%
\pgfsetbuttcap%
\pgfsetmiterjoin%
\definecolor{currentfill}{rgb}{0.000000,0.500000,0.000000}%
\pgfsetfillcolor{currentfill}%
\pgfsetlinewidth{0.000000pt}%
\definecolor{currentstroke}{rgb}{0.000000,0.000000,0.000000}%
\pgfsetstrokecolor{currentstroke}%
\pgfsetstrokeopacity{0.000000}%
\pgfsetdash{}{0pt}%
\pgfpathmoveto{\pgfqpoint{3.999716in}{0.500000in}}%
\pgfpathlineto{\pgfqpoint{4.032741in}{0.500000in}}%
\pgfpathlineto{\pgfqpoint{4.032741in}{0.500000in}}%
\pgfpathlineto{\pgfqpoint{3.999716in}{0.500000in}}%
\pgfpathlineto{\pgfqpoint{3.999716in}{0.500000in}}%
\pgfpathclose%
\pgfusepath{fill}%
\end{pgfscope}%
\begin{pgfscope}%
\pgfpathrectangle{\pgfqpoint{0.750000in}{0.500000in}}{\pgfqpoint{4.650000in}{3.020000in}}%
\pgfusepath{clip}%
\pgfsetbuttcap%
\pgfsetmiterjoin%
\definecolor{currentfill}{rgb}{0.000000,0.500000,0.000000}%
\pgfsetfillcolor{currentfill}%
\pgfsetlinewidth{0.000000pt}%
\definecolor{currentstroke}{rgb}{0.000000,0.000000,0.000000}%
\pgfsetstrokecolor{currentstroke}%
\pgfsetstrokeopacity{0.000000}%
\pgfsetdash{}{0pt}%
\pgfpathmoveto{\pgfqpoint{4.032741in}{0.500000in}}%
\pgfpathlineto{\pgfqpoint{4.065767in}{0.500000in}}%
\pgfpathlineto{\pgfqpoint{4.065767in}{0.500000in}}%
\pgfpathlineto{\pgfqpoint{4.032741in}{0.500000in}}%
\pgfpathlineto{\pgfqpoint{4.032741in}{0.500000in}}%
\pgfpathclose%
\pgfusepath{fill}%
\end{pgfscope}%
\begin{pgfscope}%
\pgfpathrectangle{\pgfqpoint{0.750000in}{0.500000in}}{\pgfqpoint{4.650000in}{3.020000in}}%
\pgfusepath{clip}%
\pgfsetbuttcap%
\pgfsetmiterjoin%
\definecolor{currentfill}{rgb}{0.000000,0.500000,0.000000}%
\pgfsetfillcolor{currentfill}%
\pgfsetlinewidth{0.000000pt}%
\definecolor{currentstroke}{rgb}{0.000000,0.000000,0.000000}%
\pgfsetstrokecolor{currentstroke}%
\pgfsetstrokeopacity{0.000000}%
\pgfsetdash{}{0pt}%
\pgfpathmoveto{\pgfqpoint{4.065767in}{0.500000in}}%
\pgfpathlineto{\pgfqpoint{4.098793in}{0.500000in}}%
\pgfpathlineto{\pgfqpoint{4.098793in}{0.500000in}}%
\pgfpathlineto{\pgfqpoint{4.065767in}{0.500000in}}%
\pgfpathlineto{\pgfqpoint{4.065767in}{0.500000in}}%
\pgfpathclose%
\pgfusepath{fill}%
\end{pgfscope}%
\begin{pgfscope}%
\pgfpathrectangle{\pgfqpoint{0.750000in}{0.500000in}}{\pgfqpoint{4.650000in}{3.020000in}}%
\pgfusepath{clip}%
\pgfsetbuttcap%
\pgfsetmiterjoin%
\definecolor{currentfill}{rgb}{0.000000,0.500000,0.000000}%
\pgfsetfillcolor{currentfill}%
\pgfsetlinewidth{0.000000pt}%
\definecolor{currentstroke}{rgb}{0.000000,0.000000,0.000000}%
\pgfsetstrokecolor{currentstroke}%
\pgfsetstrokeopacity{0.000000}%
\pgfsetdash{}{0pt}%
\pgfpathmoveto{\pgfqpoint{4.098793in}{0.500000in}}%
\pgfpathlineto{\pgfqpoint{4.131818in}{0.500000in}}%
\pgfpathlineto{\pgfqpoint{4.131818in}{0.500000in}}%
\pgfpathlineto{\pgfqpoint{4.098793in}{0.500000in}}%
\pgfpathlineto{\pgfqpoint{4.098793in}{0.500000in}}%
\pgfpathclose%
\pgfusepath{fill}%
\end{pgfscope}%
\begin{pgfscope}%
\pgfpathrectangle{\pgfqpoint{0.750000in}{0.500000in}}{\pgfqpoint{4.650000in}{3.020000in}}%
\pgfusepath{clip}%
\pgfsetbuttcap%
\pgfsetmiterjoin%
\definecolor{currentfill}{rgb}{0.000000,0.500000,0.000000}%
\pgfsetfillcolor{currentfill}%
\pgfsetlinewidth{0.000000pt}%
\definecolor{currentstroke}{rgb}{0.000000,0.000000,0.000000}%
\pgfsetstrokecolor{currentstroke}%
\pgfsetstrokeopacity{0.000000}%
\pgfsetdash{}{0pt}%
\pgfpathmoveto{\pgfqpoint{4.131818in}{0.500000in}}%
\pgfpathlineto{\pgfqpoint{4.164844in}{0.500000in}}%
\pgfpathlineto{\pgfqpoint{4.164844in}{0.500000in}}%
\pgfpathlineto{\pgfqpoint{4.131818in}{0.500000in}}%
\pgfpathlineto{\pgfqpoint{4.131818in}{0.500000in}}%
\pgfpathclose%
\pgfusepath{fill}%
\end{pgfscope}%
\begin{pgfscope}%
\pgfpathrectangle{\pgfqpoint{0.750000in}{0.500000in}}{\pgfqpoint{4.650000in}{3.020000in}}%
\pgfusepath{clip}%
\pgfsetbuttcap%
\pgfsetmiterjoin%
\definecolor{currentfill}{rgb}{0.000000,0.500000,0.000000}%
\pgfsetfillcolor{currentfill}%
\pgfsetlinewidth{0.000000pt}%
\definecolor{currentstroke}{rgb}{0.000000,0.000000,0.000000}%
\pgfsetstrokecolor{currentstroke}%
\pgfsetstrokeopacity{0.000000}%
\pgfsetdash{}{0pt}%
\pgfpathmoveto{\pgfqpoint{4.164844in}{0.500000in}}%
\pgfpathlineto{\pgfqpoint{4.197869in}{0.500000in}}%
\pgfpathlineto{\pgfqpoint{4.197869in}{0.500000in}}%
\pgfpathlineto{\pgfqpoint{4.164844in}{0.500000in}}%
\pgfpathlineto{\pgfqpoint{4.164844in}{0.500000in}}%
\pgfpathclose%
\pgfusepath{fill}%
\end{pgfscope}%
\begin{pgfscope}%
\pgfpathrectangle{\pgfqpoint{0.750000in}{0.500000in}}{\pgfqpoint{4.650000in}{3.020000in}}%
\pgfusepath{clip}%
\pgfsetbuttcap%
\pgfsetmiterjoin%
\definecolor{currentfill}{rgb}{0.000000,0.500000,0.000000}%
\pgfsetfillcolor{currentfill}%
\pgfsetlinewidth{0.000000pt}%
\definecolor{currentstroke}{rgb}{0.000000,0.000000,0.000000}%
\pgfsetstrokecolor{currentstroke}%
\pgfsetstrokeopacity{0.000000}%
\pgfsetdash{}{0pt}%
\pgfpathmoveto{\pgfqpoint{4.197869in}{0.500000in}}%
\pgfpathlineto{\pgfqpoint{4.230895in}{0.500000in}}%
\pgfpathlineto{\pgfqpoint{4.230895in}{0.500000in}}%
\pgfpathlineto{\pgfqpoint{4.197869in}{0.500000in}}%
\pgfpathlineto{\pgfqpoint{4.197869in}{0.500000in}}%
\pgfpathclose%
\pgfusepath{fill}%
\end{pgfscope}%
\begin{pgfscope}%
\pgfpathrectangle{\pgfqpoint{0.750000in}{0.500000in}}{\pgfqpoint{4.650000in}{3.020000in}}%
\pgfusepath{clip}%
\pgfsetbuttcap%
\pgfsetmiterjoin%
\definecolor{currentfill}{rgb}{0.000000,0.500000,0.000000}%
\pgfsetfillcolor{currentfill}%
\pgfsetlinewidth{0.000000pt}%
\definecolor{currentstroke}{rgb}{0.000000,0.000000,0.000000}%
\pgfsetstrokecolor{currentstroke}%
\pgfsetstrokeopacity{0.000000}%
\pgfsetdash{}{0pt}%
\pgfpathmoveto{\pgfqpoint{4.230895in}{0.500000in}}%
\pgfpathlineto{\pgfqpoint{4.263920in}{0.500000in}}%
\pgfpathlineto{\pgfqpoint{4.263920in}{0.505552in}}%
\pgfpathlineto{\pgfqpoint{4.230895in}{0.505552in}}%
\pgfpathlineto{\pgfqpoint{4.230895in}{0.500000in}}%
\pgfpathclose%
\pgfusepath{fill}%
\end{pgfscope}%
\begin{pgfscope}%
\pgfpathrectangle{\pgfqpoint{0.750000in}{0.500000in}}{\pgfqpoint{4.650000in}{3.020000in}}%
\pgfusepath{clip}%
\pgfsetbuttcap%
\pgfsetmiterjoin%
\definecolor{currentfill}{rgb}{0.000000,0.500000,0.000000}%
\pgfsetfillcolor{currentfill}%
\pgfsetlinewidth{0.000000pt}%
\definecolor{currentstroke}{rgb}{0.000000,0.000000,0.000000}%
\pgfsetstrokecolor{currentstroke}%
\pgfsetstrokeopacity{0.000000}%
\pgfsetdash{}{0pt}%
\pgfpathmoveto{\pgfqpoint{4.263920in}{0.500000in}}%
\pgfpathlineto{\pgfqpoint{4.296946in}{0.500000in}}%
\pgfpathlineto{\pgfqpoint{4.296946in}{0.505552in}}%
\pgfpathlineto{\pgfqpoint{4.263920in}{0.505552in}}%
\pgfpathlineto{\pgfqpoint{4.263920in}{0.500000in}}%
\pgfpathclose%
\pgfusepath{fill}%
\end{pgfscope}%
\begin{pgfscope}%
\pgfpathrectangle{\pgfqpoint{0.750000in}{0.500000in}}{\pgfqpoint{4.650000in}{3.020000in}}%
\pgfusepath{clip}%
\pgfsetbuttcap%
\pgfsetmiterjoin%
\definecolor{currentfill}{rgb}{0.000000,0.500000,0.000000}%
\pgfsetfillcolor{currentfill}%
\pgfsetlinewidth{0.000000pt}%
\definecolor{currentstroke}{rgb}{0.000000,0.000000,0.000000}%
\pgfsetstrokecolor{currentstroke}%
\pgfsetstrokeopacity{0.000000}%
\pgfsetdash{}{0pt}%
\pgfpathmoveto{\pgfqpoint{4.296946in}{0.500000in}}%
\pgfpathlineto{\pgfqpoint{4.329972in}{0.500000in}}%
\pgfpathlineto{\pgfqpoint{4.329972in}{0.505552in}}%
\pgfpathlineto{\pgfqpoint{4.296946in}{0.505552in}}%
\pgfpathlineto{\pgfqpoint{4.296946in}{0.500000in}}%
\pgfpathclose%
\pgfusepath{fill}%
\end{pgfscope}%
\begin{pgfscope}%
\pgfpathrectangle{\pgfqpoint{0.750000in}{0.500000in}}{\pgfqpoint{4.650000in}{3.020000in}}%
\pgfusepath{clip}%
\pgfsetbuttcap%
\pgfsetmiterjoin%
\definecolor{currentfill}{rgb}{0.000000,0.500000,0.000000}%
\pgfsetfillcolor{currentfill}%
\pgfsetlinewidth{0.000000pt}%
\definecolor{currentstroke}{rgb}{0.000000,0.000000,0.000000}%
\pgfsetstrokecolor{currentstroke}%
\pgfsetstrokeopacity{0.000000}%
\pgfsetdash{}{0pt}%
\pgfpathmoveto{\pgfqpoint{4.329972in}{0.500000in}}%
\pgfpathlineto{\pgfqpoint{4.362997in}{0.500000in}}%
\pgfpathlineto{\pgfqpoint{4.362997in}{0.505552in}}%
\pgfpathlineto{\pgfqpoint{4.329972in}{0.505552in}}%
\pgfpathlineto{\pgfqpoint{4.329972in}{0.500000in}}%
\pgfpathclose%
\pgfusepath{fill}%
\end{pgfscope}%
\begin{pgfscope}%
\pgfpathrectangle{\pgfqpoint{0.750000in}{0.500000in}}{\pgfqpoint{4.650000in}{3.020000in}}%
\pgfusepath{clip}%
\pgfsetbuttcap%
\pgfsetmiterjoin%
\definecolor{currentfill}{rgb}{0.000000,0.500000,0.000000}%
\pgfsetfillcolor{currentfill}%
\pgfsetlinewidth{0.000000pt}%
\definecolor{currentstroke}{rgb}{0.000000,0.000000,0.000000}%
\pgfsetstrokecolor{currentstroke}%
\pgfsetstrokeopacity{0.000000}%
\pgfsetdash{}{0pt}%
\pgfpathmoveto{\pgfqpoint{4.362997in}{0.500000in}}%
\pgfpathlineto{\pgfqpoint{4.396023in}{0.500000in}}%
\pgfpathlineto{\pgfqpoint{4.396023in}{0.500000in}}%
\pgfpathlineto{\pgfqpoint{4.362997in}{0.500000in}}%
\pgfpathlineto{\pgfqpoint{4.362997in}{0.500000in}}%
\pgfpathclose%
\pgfusepath{fill}%
\end{pgfscope}%
\begin{pgfscope}%
\pgfpathrectangle{\pgfqpoint{0.750000in}{0.500000in}}{\pgfqpoint{4.650000in}{3.020000in}}%
\pgfusepath{clip}%
\pgfsetbuttcap%
\pgfsetmiterjoin%
\definecolor{currentfill}{rgb}{0.000000,0.500000,0.000000}%
\pgfsetfillcolor{currentfill}%
\pgfsetlinewidth{0.000000pt}%
\definecolor{currentstroke}{rgb}{0.000000,0.000000,0.000000}%
\pgfsetstrokecolor{currentstroke}%
\pgfsetstrokeopacity{0.000000}%
\pgfsetdash{}{0pt}%
\pgfpathmoveto{\pgfqpoint{4.396023in}{0.500000in}}%
\pgfpathlineto{\pgfqpoint{4.429048in}{0.500000in}}%
\pgfpathlineto{\pgfqpoint{4.429048in}{0.505552in}}%
\pgfpathlineto{\pgfqpoint{4.396023in}{0.505552in}}%
\pgfpathlineto{\pgfqpoint{4.396023in}{0.500000in}}%
\pgfpathclose%
\pgfusepath{fill}%
\end{pgfscope}%
\begin{pgfscope}%
\pgfpathrectangle{\pgfqpoint{0.750000in}{0.500000in}}{\pgfqpoint{4.650000in}{3.020000in}}%
\pgfusepath{clip}%
\pgfsetbuttcap%
\pgfsetmiterjoin%
\definecolor{currentfill}{rgb}{0.000000,0.500000,0.000000}%
\pgfsetfillcolor{currentfill}%
\pgfsetlinewidth{0.000000pt}%
\definecolor{currentstroke}{rgb}{0.000000,0.000000,0.000000}%
\pgfsetstrokecolor{currentstroke}%
\pgfsetstrokeopacity{0.000000}%
\pgfsetdash{}{0pt}%
\pgfpathmoveto{\pgfqpoint{4.429048in}{0.500000in}}%
\pgfpathlineto{\pgfqpoint{4.462074in}{0.500000in}}%
\pgfpathlineto{\pgfqpoint{4.462074in}{0.500000in}}%
\pgfpathlineto{\pgfqpoint{4.429048in}{0.500000in}}%
\pgfpathlineto{\pgfqpoint{4.429048in}{0.500000in}}%
\pgfpathclose%
\pgfusepath{fill}%
\end{pgfscope}%
\begin{pgfscope}%
\pgfpathrectangle{\pgfqpoint{0.750000in}{0.500000in}}{\pgfqpoint{4.650000in}{3.020000in}}%
\pgfusepath{clip}%
\pgfsetbuttcap%
\pgfsetmiterjoin%
\definecolor{currentfill}{rgb}{0.000000,0.500000,0.000000}%
\pgfsetfillcolor{currentfill}%
\pgfsetlinewidth{0.000000pt}%
\definecolor{currentstroke}{rgb}{0.000000,0.000000,0.000000}%
\pgfsetstrokecolor{currentstroke}%
\pgfsetstrokeopacity{0.000000}%
\pgfsetdash{}{0pt}%
\pgfpathmoveto{\pgfqpoint{4.462074in}{0.500000in}}%
\pgfpathlineto{\pgfqpoint{4.495099in}{0.500000in}}%
\pgfpathlineto{\pgfqpoint{4.495099in}{0.500000in}}%
\pgfpathlineto{\pgfqpoint{4.462074in}{0.500000in}}%
\pgfpathlineto{\pgfqpoint{4.462074in}{0.500000in}}%
\pgfpathclose%
\pgfusepath{fill}%
\end{pgfscope}%
\begin{pgfscope}%
\pgfpathrectangle{\pgfqpoint{0.750000in}{0.500000in}}{\pgfqpoint{4.650000in}{3.020000in}}%
\pgfusepath{clip}%
\pgfsetbuttcap%
\pgfsetmiterjoin%
\definecolor{currentfill}{rgb}{0.000000,0.500000,0.000000}%
\pgfsetfillcolor{currentfill}%
\pgfsetlinewidth{0.000000pt}%
\definecolor{currentstroke}{rgb}{0.000000,0.000000,0.000000}%
\pgfsetstrokecolor{currentstroke}%
\pgfsetstrokeopacity{0.000000}%
\pgfsetdash{}{0pt}%
\pgfpathmoveto{\pgfqpoint{4.495099in}{0.500000in}}%
\pgfpathlineto{\pgfqpoint{4.528125in}{0.500000in}}%
\pgfpathlineto{\pgfqpoint{4.528125in}{0.500000in}}%
\pgfpathlineto{\pgfqpoint{4.495099in}{0.500000in}}%
\pgfpathlineto{\pgfqpoint{4.495099in}{0.500000in}}%
\pgfpathclose%
\pgfusepath{fill}%
\end{pgfscope}%
\begin{pgfscope}%
\pgfpathrectangle{\pgfqpoint{0.750000in}{0.500000in}}{\pgfqpoint{4.650000in}{3.020000in}}%
\pgfusepath{clip}%
\pgfsetbuttcap%
\pgfsetmiterjoin%
\definecolor{currentfill}{rgb}{0.000000,0.500000,0.000000}%
\pgfsetfillcolor{currentfill}%
\pgfsetlinewidth{0.000000pt}%
\definecolor{currentstroke}{rgb}{0.000000,0.000000,0.000000}%
\pgfsetstrokecolor{currentstroke}%
\pgfsetstrokeopacity{0.000000}%
\pgfsetdash{}{0pt}%
\pgfpathmoveto{\pgfqpoint{4.528125in}{0.500000in}}%
\pgfpathlineto{\pgfqpoint{4.561151in}{0.500000in}}%
\pgfpathlineto{\pgfqpoint{4.561151in}{0.500000in}}%
\pgfpathlineto{\pgfqpoint{4.528125in}{0.500000in}}%
\pgfpathlineto{\pgfqpoint{4.528125in}{0.500000in}}%
\pgfpathclose%
\pgfusepath{fill}%
\end{pgfscope}%
\begin{pgfscope}%
\pgfpathrectangle{\pgfqpoint{0.750000in}{0.500000in}}{\pgfqpoint{4.650000in}{3.020000in}}%
\pgfusepath{clip}%
\pgfsetbuttcap%
\pgfsetmiterjoin%
\definecolor{currentfill}{rgb}{0.000000,0.500000,0.000000}%
\pgfsetfillcolor{currentfill}%
\pgfsetlinewidth{0.000000pt}%
\definecolor{currentstroke}{rgb}{0.000000,0.000000,0.000000}%
\pgfsetstrokecolor{currentstroke}%
\pgfsetstrokeopacity{0.000000}%
\pgfsetdash{}{0pt}%
\pgfpathmoveto{\pgfqpoint{4.561151in}{0.500000in}}%
\pgfpathlineto{\pgfqpoint{4.594176in}{0.500000in}}%
\pgfpathlineto{\pgfqpoint{4.594176in}{0.500000in}}%
\pgfpathlineto{\pgfqpoint{4.561151in}{0.500000in}}%
\pgfpathlineto{\pgfqpoint{4.561151in}{0.500000in}}%
\pgfpathclose%
\pgfusepath{fill}%
\end{pgfscope}%
\begin{pgfscope}%
\pgfpathrectangle{\pgfqpoint{0.750000in}{0.500000in}}{\pgfqpoint{4.650000in}{3.020000in}}%
\pgfusepath{clip}%
\pgfsetbuttcap%
\pgfsetmiterjoin%
\definecolor{currentfill}{rgb}{0.000000,0.500000,0.000000}%
\pgfsetfillcolor{currentfill}%
\pgfsetlinewidth{0.000000pt}%
\definecolor{currentstroke}{rgb}{0.000000,0.000000,0.000000}%
\pgfsetstrokecolor{currentstroke}%
\pgfsetstrokeopacity{0.000000}%
\pgfsetdash{}{0pt}%
\pgfpathmoveto{\pgfqpoint{4.594176in}{0.500000in}}%
\pgfpathlineto{\pgfqpoint{4.627202in}{0.500000in}}%
\pgfpathlineto{\pgfqpoint{4.627202in}{0.511105in}}%
\pgfpathlineto{\pgfqpoint{4.594176in}{0.511105in}}%
\pgfpathlineto{\pgfqpoint{4.594176in}{0.500000in}}%
\pgfpathclose%
\pgfusepath{fill}%
\end{pgfscope}%
\begin{pgfscope}%
\pgfpathrectangle{\pgfqpoint{0.750000in}{0.500000in}}{\pgfqpoint{4.650000in}{3.020000in}}%
\pgfusepath{clip}%
\pgfsetbuttcap%
\pgfsetmiterjoin%
\definecolor{currentfill}{rgb}{0.000000,0.500000,0.000000}%
\pgfsetfillcolor{currentfill}%
\pgfsetlinewidth{0.000000pt}%
\definecolor{currentstroke}{rgb}{0.000000,0.000000,0.000000}%
\pgfsetstrokecolor{currentstroke}%
\pgfsetstrokeopacity{0.000000}%
\pgfsetdash{}{0pt}%
\pgfpathmoveto{\pgfqpoint{4.627202in}{0.500000in}}%
\pgfpathlineto{\pgfqpoint{4.660227in}{0.500000in}}%
\pgfpathlineto{\pgfqpoint{4.660227in}{0.500000in}}%
\pgfpathlineto{\pgfqpoint{4.627202in}{0.500000in}}%
\pgfpathlineto{\pgfqpoint{4.627202in}{0.500000in}}%
\pgfpathclose%
\pgfusepath{fill}%
\end{pgfscope}%
\begin{pgfscope}%
\pgfpathrectangle{\pgfqpoint{0.750000in}{0.500000in}}{\pgfqpoint{4.650000in}{3.020000in}}%
\pgfusepath{clip}%
\pgfsetbuttcap%
\pgfsetmiterjoin%
\definecolor{currentfill}{rgb}{0.000000,0.500000,0.000000}%
\pgfsetfillcolor{currentfill}%
\pgfsetlinewidth{0.000000pt}%
\definecolor{currentstroke}{rgb}{0.000000,0.000000,0.000000}%
\pgfsetstrokecolor{currentstroke}%
\pgfsetstrokeopacity{0.000000}%
\pgfsetdash{}{0pt}%
\pgfpathmoveto{\pgfqpoint{4.660227in}{0.500000in}}%
\pgfpathlineto{\pgfqpoint{4.693253in}{0.500000in}}%
\pgfpathlineto{\pgfqpoint{4.693253in}{0.500000in}}%
\pgfpathlineto{\pgfqpoint{4.660227in}{0.500000in}}%
\pgfpathlineto{\pgfqpoint{4.660227in}{0.500000in}}%
\pgfpathclose%
\pgfusepath{fill}%
\end{pgfscope}%
\begin{pgfscope}%
\pgfpathrectangle{\pgfqpoint{0.750000in}{0.500000in}}{\pgfqpoint{4.650000in}{3.020000in}}%
\pgfusepath{clip}%
\pgfsetbuttcap%
\pgfsetmiterjoin%
\definecolor{currentfill}{rgb}{0.000000,0.500000,0.000000}%
\pgfsetfillcolor{currentfill}%
\pgfsetlinewidth{0.000000pt}%
\definecolor{currentstroke}{rgb}{0.000000,0.000000,0.000000}%
\pgfsetstrokecolor{currentstroke}%
\pgfsetstrokeopacity{0.000000}%
\pgfsetdash{}{0pt}%
\pgfpathmoveto{\pgfqpoint{4.693253in}{0.500000in}}%
\pgfpathlineto{\pgfqpoint{4.726278in}{0.500000in}}%
\pgfpathlineto{\pgfqpoint{4.726278in}{0.500000in}}%
\pgfpathlineto{\pgfqpoint{4.693253in}{0.500000in}}%
\pgfpathlineto{\pgfqpoint{4.693253in}{0.500000in}}%
\pgfpathclose%
\pgfusepath{fill}%
\end{pgfscope}%
\begin{pgfscope}%
\pgfpathrectangle{\pgfqpoint{0.750000in}{0.500000in}}{\pgfqpoint{4.650000in}{3.020000in}}%
\pgfusepath{clip}%
\pgfsetbuttcap%
\pgfsetmiterjoin%
\definecolor{currentfill}{rgb}{0.000000,0.500000,0.000000}%
\pgfsetfillcolor{currentfill}%
\pgfsetlinewidth{0.000000pt}%
\definecolor{currentstroke}{rgb}{0.000000,0.000000,0.000000}%
\pgfsetstrokecolor{currentstroke}%
\pgfsetstrokeopacity{0.000000}%
\pgfsetdash{}{0pt}%
\pgfpathmoveto{\pgfqpoint{4.726278in}{0.500000in}}%
\pgfpathlineto{\pgfqpoint{4.759304in}{0.500000in}}%
\pgfpathlineto{\pgfqpoint{4.759304in}{0.505552in}}%
\pgfpathlineto{\pgfqpoint{4.726278in}{0.505552in}}%
\pgfpathlineto{\pgfqpoint{4.726278in}{0.500000in}}%
\pgfpathclose%
\pgfusepath{fill}%
\end{pgfscope}%
\begin{pgfscope}%
\pgfpathrectangle{\pgfqpoint{0.750000in}{0.500000in}}{\pgfqpoint{4.650000in}{3.020000in}}%
\pgfusepath{clip}%
\pgfsetbuttcap%
\pgfsetmiterjoin%
\definecolor{currentfill}{rgb}{0.000000,0.500000,0.000000}%
\pgfsetfillcolor{currentfill}%
\pgfsetlinewidth{0.000000pt}%
\definecolor{currentstroke}{rgb}{0.000000,0.000000,0.000000}%
\pgfsetstrokecolor{currentstroke}%
\pgfsetstrokeopacity{0.000000}%
\pgfsetdash{}{0pt}%
\pgfpathmoveto{\pgfqpoint{4.759304in}{0.500000in}}%
\pgfpathlineto{\pgfqpoint{4.792330in}{0.500000in}}%
\pgfpathlineto{\pgfqpoint{4.792330in}{0.500000in}}%
\pgfpathlineto{\pgfqpoint{4.759304in}{0.500000in}}%
\pgfpathlineto{\pgfqpoint{4.759304in}{0.500000in}}%
\pgfpathclose%
\pgfusepath{fill}%
\end{pgfscope}%
\begin{pgfscope}%
\pgfpathrectangle{\pgfqpoint{0.750000in}{0.500000in}}{\pgfqpoint{4.650000in}{3.020000in}}%
\pgfusepath{clip}%
\pgfsetbuttcap%
\pgfsetmiterjoin%
\definecolor{currentfill}{rgb}{0.000000,0.500000,0.000000}%
\pgfsetfillcolor{currentfill}%
\pgfsetlinewidth{0.000000pt}%
\definecolor{currentstroke}{rgb}{0.000000,0.000000,0.000000}%
\pgfsetstrokecolor{currentstroke}%
\pgfsetstrokeopacity{0.000000}%
\pgfsetdash{}{0pt}%
\pgfpathmoveto{\pgfqpoint{4.792330in}{0.500000in}}%
\pgfpathlineto{\pgfqpoint{4.825355in}{0.500000in}}%
\pgfpathlineto{\pgfqpoint{4.825355in}{0.511105in}}%
\pgfpathlineto{\pgfqpoint{4.792330in}{0.511105in}}%
\pgfpathlineto{\pgfqpoint{4.792330in}{0.500000in}}%
\pgfpathclose%
\pgfusepath{fill}%
\end{pgfscope}%
\begin{pgfscope}%
\pgfpathrectangle{\pgfqpoint{0.750000in}{0.500000in}}{\pgfqpoint{4.650000in}{3.020000in}}%
\pgfusepath{clip}%
\pgfsetbuttcap%
\pgfsetmiterjoin%
\definecolor{currentfill}{rgb}{0.000000,0.500000,0.000000}%
\pgfsetfillcolor{currentfill}%
\pgfsetlinewidth{0.000000pt}%
\definecolor{currentstroke}{rgb}{0.000000,0.000000,0.000000}%
\pgfsetstrokecolor{currentstroke}%
\pgfsetstrokeopacity{0.000000}%
\pgfsetdash{}{0pt}%
\pgfpathmoveto{\pgfqpoint{4.825355in}{0.500000in}}%
\pgfpathlineto{\pgfqpoint{4.858381in}{0.500000in}}%
\pgfpathlineto{\pgfqpoint{4.858381in}{0.500000in}}%
\pgfpathlineto{\pgfqpoint{4.825355in}{0.500000in}}%
\pgfpathlineto{\pgfqpoint{4.825355in}{0.500000in}}%
\pgfpathclose%
\pgfusepath{fill}%
\end{pgfscope}%
\begin{pgfscope}%
\pgfpathrectangle{\pgfqpoint{0.750000in}{0.500000in}}{\pgfqpoint{4.650000in}{3.020000in}}%
\pgfusepath{clip}%
\pgfsetbuttcap%
\pgfsetmiterjoin%
\definecolor{currentfill}{rgb}{0.000000,0.500000,0.000000}%
\pgfsetfillcolor{currentfill}%
\pgfsetlinewidth{0.000000pt}%
\definecolor{currentstroke}{rgb}{0.000000,0.000000,0.000000}%
\pgfsetstrokecolor{currentstroke}%
\pgfsetstrokeopacity{0.000000}%
\pgfsetdash{}{0pt}%
\pgfpathmoveto{\pgfqpoint{4.858381in}{0.500000in}}%
\pgfpathlineto{\pgfqpoint{4.891406in}{0.500000in}}%
\pgfpathlineto{\pgfqpoint{4.891406in}{0.505552in}}%
\pgfpathlineto{\pgfqpoint{4.858381in}{0.505552in}}%
\pgfpathlineto{\pgfqpoint{4.858381in}{0.500000in}}%
\pgfpathclose%
\pgfusepath{fill}%
\end{pgfscope}%
\begin{pgfscope}%
\pgfpathrectangle{\pgfqpoint{0.750000in}{0.500000in}}{\pgfqpoint{4.650000in}{3.020000in}}%
\pgfusepath{clip}%
\pgfsetbuttcap%
\pgfsetmiterjoin%
\definecolor{currentfill}{rgb}{0.000000,0.500000,0.000000}%
\pgfsetfillcolor{currentfill}%
\pgfsetlinewidth{0.000000pt}%
\definecolor{currentstroke}{rgb}{0.000000,0.000000,0.000000}%
\pgfsetstrokecolor{currentstroke}%
\pgfsetstrokeopacity{0.000000}%
\pgfsetdash{}{0pt}%
\pgfpathmoveto{\pgfqpoint{4.891406in}{0.500000in}}%
\pgfpathlineto{\pgfqpoint{4.924432in}{0.500000in}}%
\pgfpathlineto{\pgfqpoint{4.924432in}{0.500000in}}%
\pgfpathlineto{\pgfqpoint{4.891406in}{0.500000in}}%
\pgfpathlineto{\pgfqpoint{4.891406in}{0.500000in}}%
\pgfpathclose%
\pgfusepath{fill}%
\end{pgfscope}%
\begin{pgfscope}%
\pgfpathrectangle{\pgfqpoint{0.750000in}{0.500000in}}{\pgfqpoint{4.650000in}{3.020000in}}%
\pgfusepath{clip}%
\pgfsetbuttcap%
\pgfsetmiterjoin%
\definecolor{currentfill}{rgb}{0.000000,0.500000,0.000000}%
\pgfsetfillcolor{currentfill}%
\pgfsetlinewidth{0.000000pt}%
\definecolor{currentstroke}{rgb}{0.000000,0.000000,0.000000}%
\pgfsetstrokecolor{currentstroke}%
\pgfsetstrokeopacity{0.000000}%
\pgfsetdash{}{0pt}%
\pgfpathmoveto{\pgfqpoint{4.924432in}{0.500000in}}%
\pgfpathlineto{\pgfqpoint{4.957457in}{0.500000in}}%
\pgfpathlineto{\pgfqpoint{4.957457in}{0.505552in}}%
\pgfpathlineto{\pgfqpoint{4.924432in}{0.505552in}}%
\pgfpathlineto{\pgfqpoint{4.924432in}{0.500000in}}%
\pgfpathclose%
\pgfusepath{fill}%
\end{pgfscope}%
\begin{pgfscope}%
\pgfpathrectangle{\pgfqpoint{0.750000in}{0.500000in}}{\pgfqpoint{4.650000in}{3.020000in}}%
\pgfusepath{clip}%
\pgfsetbuttcap%
\pgfsetmiterjoin%
\definecolor{currentfill}{rgb}{0.000000,0.500000,0.000000}%
\pgfsetfillcolor{currentfill}%
\pgfsetlinewidth{0.000000pt}%
\definecolor{currentstroke}{rgb}{0.000000,0.000000,0.000000}%
\pgfsetstrokecolor{currentstroke}%
\pgfsetstrokeopacity{0.000000}%
\pgfsetdash{}{0pt}%
\pgfpathmoveto{\pgfqpoint{4.957457in}{0.500000in}}%
\pgfpathlineto{\pgfqpoint{4.990483in}{0.500000in}}%
\pgfpathlineto{\pgfqpoint{4.990483in}{0.500000in}}%
\pgfpathlineto{\pgfqpoint{4.957457in}{0.500000in}}%
\pgfpathlineto{\pgfqpoint{4.957457in}{0.500000in}}%
\pgfpathclose%
\pgfusepath{fill}%
\end{pgfscope}%
\begin{pgfscope}%
\pgfpathrectangle{\pgfqpoint{0.750000in}{0.500000in}}{\pgfqpoint{4.650000in}{3.020000in}}%
\pgfusepath{clip}%
\pgfsetbuttcap%
\pgfsetmiterjoin%
\definecolor{currentfill}{rgb}{0.000000,0.500000,0.000000}%
\pgfsetfillcolor{currentfill}%
\pgfsetlinewidth{0.000000pt}%
\definecolor{currentstroke}{rgb}{0.000000,0.000000,0.000000}%
\pgfsetstrokecolor{currentstroke}%
\pgfsetstrokeopacity{0.000000}%
\pgfsetdash{}{0pt}%
\pgfpathmoveto{\pgfqpoint{4.990483in}{0.500000in}}%
\pgfpathlineto{\pgfqpoint{5.023509in}{0.500000in}}%
\pgfpathlineto{\pgfqpoint{5.023509in}{0.516657in}}%
\pgfpathlineto{\pgfqpoint{4.990483in}{0.516657in}}%
\pgfpathlineto{\pgfqpoint{4.990483in}{0.500000in}}%
\pgfpathclose%
\pgfusepath{fill}%
\end{pgfscope}%
\begin{pgfscope}%
\pgfpathrectangle{\pgfqpoint{0.750000in}{0.500000in}}{\pgfqpoint{4.650000in}{3.020000in}}%
\pgfusepath{clip}%
\pgfsetbuttcap%
\pgfsetmiterjoin%
\definecolor{currentfill}{rgb}{0.000000,0.500000,0.000000}%
\pgfsetfillcolor{currentfill}%
\pgfsetlinewidth{0.000000pt}%
\definecolor{currentstroke}{rgb}{0.000000,0.000000,0.000000}%
\pgfsetstrokecolor{currentstroke}%
\pgfsetstrokeopacity{0.000000}%
\pgfsetdash{}{0pt}%
\pgfpathmoveto{\pgfqpoint{5.023509in}{0.500000in}}%
\pgfpathlineto{\pgfqpoint{5.056534in}{0.500000in}}%
\pgfpathlineto{\pgfqpoint{5.056534in}{0.505552in}}%
\pgfpathlineto{\pgfqpoint{5.023509in}{0.505552in}}%
\pgfpathlineto{\pgfqpoint{5.023509in}{0.500000in}}%
\pgfpathclose%
\pgfusepath{fill}%
\end{pgfscope}%
\begin{pgfscope}%
\pgfpathrectangle{\pgfqpoint{0.750000in}{0.500000in}}{\pgfqpoint{4.650000in}{3.020000in}}%
\pgfusepath{clip}%
\pgfsetbuttcap%
\pgfsetmiterjoin%
\definecolor{currentfill}{rgb}{0.000000,0.500000,0.000000}%
\pgfsetfillcolor{currentfill}%
\pgfsetlinewidth{0.000000pt}%
\definecolor{currentstroke}{rgb}{0.000000,0.000000,0.000000}%
\pgfsetstrokecolor{currentstroke}%
\pgfsetstrokeopacity{0.000000}%
\pgfsetdash{}{0pt}%
\pgfpathmoveto{\pgfqpoint{5.056534in}{0.500000in}}%
\pgfpathlineto{\pgfqpoint{5.089560in}{0.500000in}}%
\pgfpathlineto{\pgfqpoint{5.089560in}{0.505552in}}%
\pgfpathlineto{\pgfqpoint{5.056534in}{0.505552in}}%
\pgfpathlineto{\pgfqpoint{5.056534in}{0.500000in}}%
\pgfpathclose%
\pgfusepath{fill}%
\end{pgfscope}%
\begin{pgfscope}%
\pgfpathrectangle{\pgfqpoint{0.750000in}{0.500000in}}{\pgfqpoint{4.650000in}{3.020000in}}%
\pgfusepath{clip}%
\pgfsetbuttcap%
\pgfsetmiterjoin%
\definecolor{currentfill}{rgb}{0.000000,0.500000,0.000000}%
\pgfsetfillcolor{currentfill}%
\pgfsetlinewidth{0.000000pt}%
\definecolor{currentstroke}{rgb}{0.000000,0.000000,0.000000}%
\pgfsetstrokecolor{currentstroke}%
\pgfsetstrokeopacity{0.000000}%
\pgfsetdash{}{0pt}%
\pgfpathmoveto{\pgfqpoint{5.089560in}{0.500000in}}%
\pgfpathlineto{\pgfqpoint{5.122585in}{0.500000in}}%
\pgfpathlineto{\pgfqpoint{5.122585in}{0.511105in}}%
\pgfpathlineto{\pgfqpoint{5.089560in}{0.511105in}}%
\pgfpathlineto{\pgfqpoint{5.089560in}{0.500000in}}%
\pgfpathclose%
\pgfusepath{fill}%
\end{pgfscope}%
\begin{pgfscope}%
\pgfpathrectangle{\pgfqpoint{0.750000in}{0.500000in}}{\pgfqpoint{4.650000in}{3.020000in}}%
\pgfusepath{clip}%
\pgfsetbuttcap%
\pgfsetmiterjoin%
\definecolor{currentfill}{rgb}{0.000000,0.500000,0.000000}%
\pgfsetfillcolor{currentfill}%
\pgfsetlinewidth{0.000000pt}%
\definecolor{currentstroke}{rgb}{0.000000,0.000000,0.000000}%
\pgfsetstrokecolor{currentstroke}%
\pgfsetstrokeopacity{0.000000}%
\pgfsetdash{}{0pt}%
\pgfpathmoveto{\pgfqpoint{5.122585in}{0.500000in}}%
\pgfpathlineto{\pgfqpoint{5.155611in}{0.500000in}}%
\pgfpathlineto{\pgfqpoint{5.155611in}{0.505552in}}%
\pgfpathlineto{\pgfqpoint{5.122585in}{0.505552in}}%
\pgfpathlineto{\pgfqpoint{5.122585in}{0.500000in}}%
\pgfpathclose%
\pgfusepath{fill}%
\end{pgfscope}%
\begin{pgfscope}%
\pgfpathrectangle{\pgfqpoint{0.750000in}{0.500000in}}{\pgfqpoint{4.650000in}{3.020000in}}%
\pgfusepath{clip}%
\pgfsetbuttcap%
\pgfsetmiterjoin%
\definecolor{currentfill}{rgb}{0.000000,0.500000,0.000000}%
\pgfsetfillcolor{currentfill}%
\pgfsetlinewidth{0.000000pt}%
\definecolor{currentstroke}{rgb}{0.000000,0.000000,0.000000}%
\pgfsetstrokecolor{currentstroke}%
\pgfsetstrokeopacity{0.000000}%
\pgfsetdash{}{0pt}%
\pgfpathmoveto{\pgfqpoint{5.155611in}{0.500000in}}%
\pgfpathlineto{\pgfqpoint{5.188636in}{0.500000in}}%
\pgfpathlineto{\pgfqpoint{5.188636in}{0.516657in}}%
\pgfpathlineto{\pgfqpoint{5.155611in}{0.516657in}}%
\pgfpathlineto{\pgfqpoint{5.155611in}{0.500000in}}%
\pgfpathclose%
\pgfusepath{fill}%
\end{pgfscope}%
\begin{pgfscope}%
\pgfsetbuttcap%
\pgfsetroundjoin%
\definecolor{currentfill}{rgb}{0.000000,0.000000,0.000000}%
\pgfsetfillcolor{currentfill}%
\pgfsetlinewidth{0.803000pt}%
\definecolor{currentstroke}{rgb}{0.000000,0.000000,0.000000}%
\pgfsetstrokecolor{currentstroke}%
\pgfsetdash{}{0pt}%
\pgfsys@defobject{currentmarker}{\pgfqpoint{0.000000in}{-0.048611in}}{\pgfqpoint{0.000000in}{0.000000in}}{%
\pgfpathmoveto{\pgfqpoint{0.000000in}{0.000000in}}%
\pgfpathlineto{\pgfqpoint{0.000000in}{-0.048611in}}%
\pgfusepath{stroke,fill}%
}%
\begin{pgfscope}%
\pgfsys@transformshift{1.243355in}{0.500000in}%
\pgfsys@useobject{currentmarker}{}%
\end{pgfscope}%
\end{pgfscope}%
\begin{pgfscope}%
\definecolor{textcolor}{rgb}{0.000000,0.000000,0.000000}%
\pgfsetstrokecolor{textcolor}%
\pgfsetfillcolor{textcolor}%
\pgftext[x=1.243355in,y=0.402778in,,top]{\color{textcolor}\sffamily\fontsize{10.000000}{12.000000}\selectfont 0.15}%
\end{pgfscope}%
\begin{pgfscope}%
\pgfsetbuttcap%
\pgfsetroundjoin%
\definecolor{currentfill}{rgb}{0.000000,0.000000,0.000000}%
\pgfsetfillcolor{currentfill}%
\pgfsetlinewidth{0.803000pt}%
\definecolor{currentstroke}{rgb}{0.000000,0.000000,0.000000}%
\pgfsetstrokecolor{currentstroke}%
\pgfsetdash{}{0pt}%
\pgfsys@defobject{currentmarker}{\pgfqpoint{0.000000in}{-0.048611in}}{\pgfqpoint{0.000000in}{0.000000in}}{%
\pgfpathmoveto{\pgfqpoint{0.000000in}{0.000000in}}%
\pgfpathlineto{\pgfqpoint{0.000000in}{-0.048611in}}%
\pgfusepath{stroke,fill}%
}%
\begin{pgfscope}%
\pgfsys@transformshift{1.809368in}{0.500000in}%
\pgfsys@useobject{currentmarker}{}%
\end{pgfscope}%
\end{pgfscope}%
\begin{pgfscope}%
\definecolor{textcolor}{rgb}{0.000000,0.000000,0.000000}%
\pgfsetstrokecolor{textcolor}%
\pgfsetfillcolor{textcolor}%
\pgftext[x=1.809368in,y=0.402778in,,top]{\color{textcolor}\sffamily\fontsize{10.000000}{12.000000}\selectfont 0.20}%
\end{pgfscope}%
\begin{pgfscope}%
\pgfsetbuttcap%
\pgfsetroundjoin%
\definecolor{currentfill}{rgb}{0.000000,0.000000,0.000000}%
\pgfsetfillcolor{currentfill}%
\pgfsetlinewidth{0.803000pt}%
\definecolor{currentstroke}{rgb}{0.000000,0.000000,0.000000}%
\pgfsetstrokecolor{currentstroke}%
\pgfsetdash{}{0pt}%
\pgfsys@defobject{currentmarker}{\pgfqpoint{0.000000in}{-0.048611in}}{\pgfqpoint{0.000000in}{0.000000in}}{%
\pgfpathmoveto{\pgfqpoint{0.000000in}{0.000000in}}%
\pgfpathlineto{\pgfqpoint{0.000000in}{-0.048611in}}%
\pgfusepath{stroke,fill}%
}%
\begin{pgfscope}%
\pgfsys@transformshift{2.375381in}{0.500000in}%
\pgfsys@useobject{currentmarker}{}%
\end{pgfscope}%
\end{pgfscope}%
\begin{pgfscope}%
\definecolor{textcolor}{rgb}{0.000000,0.000000,0.000000}%
\pgfsetstrokecolor{textcolor}%
\pgfsetfillcolor{textcolor}%
\pgftext[x=2.375381in,y=0.402778in,,top]{\color{textcolor}\sffamily\fontsize{10.000000}{12.000000}\selectfont 0.25}%
\end{pgfscope}%
\begin{pgfscope}%
\pgfsetbuttcap%
\pgfsetroundjoin%
\definecolor{currentfill}{rgb}{0.000000,0.000000,0.000000}%
\pgfsetfillcolor{currentfill}%
\pgfsetlinewidth{0.803000pt}%
\definecolor{currentstroke}{rgb}{0.000000,0.000000,0.000000}%
\pgfsetstrokecolor{currentstroke}%
\pgfsetdash{}{0pt}%
\pgfsys@defobject{currentmarker}{\pgfqpoint{0.000000in}{-0.048611in}}{\pgfqpoint{0.000000in}{0.000000in}}{%
\pgfpathmoveto{\pgfqpoint{0.000000in}{0.000000in}}%
\pgfpathlineto{\pgfqpoint{0.000000in}{-0.048611in}}%
\pgfusepath{stroke,fill}%
}%
\begin{pgfscope}%
\pgfsys@transformshift{2.941394in}{0.500000in}%
\pgfsys@useobject{currentmarker}{}%
\end{pgfscope}%
\end{pgfscope}%
\begin{pgfscope}%
\definecolor{textcolor}{rgb}{0.000000,0.000000,0.000000}%
\pgfsetstrokecolor{textcolor}%
\pgfsetfillcolor{textcolor}%
\pgftext[x=2.941394in,y=0.402778in,,top]{\color{textcolor}\sffamily\fontsize{10.000000}{12.000000}\selectfont 0.30}%
\end{pgfscope}%
\begin{pgfscope}%
\pgfsetbuttcap%
\pgfsetroundjoin%
\definecolor{currentfill}{rgb}{0.000000,0.000000,0.000000}%
\pgfsetfillcolor{currentfill}%
\pgfsetlinewidth{0.803000pt}%
\definecolor{currentstroke}{rgb}{0.000000,0.000000,0.000000}%
\pgfsetstrokecolor{currentstroke}%
\pgfsetdash{}{0pt}%
\pgfsys@defobject{currentmarker}{\pgfqpoint{0.000000in}{-0.048611in}}{\pgfqpoint{0.000000in}{0.000000in}}{%
\pgfpathmoveto{\pgfqpoint{0.000000in}{0.000000in}}%
\pgfpathlineto{\pgfqpoint{0.000000in}{-0.048611in}}%
\pgfusepath{stroke,fill}%
}%
\begin{pgfscope}%
\pgfsys@transformshift{3.507407in}{0.500000in}%
\pgfsys@useobject{currentmarker}{}%
\end{pgfscope}%
\end{pgfscope}%
\begin{pgfscope}%
\definecolor{textcolor}{rgb}{0.000000,0.000000,0.000000}%
\pgfsetstrokecolor{textcolor}%
\pgfsetfillcolor{textcolor}%
\pgftext[x=3.507407in,y=0.402778in,,top]{\color{textcolor}\sffamily\fontsize{10.000000}{12.000000}\selectfont 0.35}%
\end{pgfscope}%
\begin{pgfscope}%
\pgfsetbuttcap%
\pgfsetroundjoin%
\definecolor{currentfill}{rgb}{0.000000,0.000000,0.000000}%
\pgfsetfillcolor{currentfill}%
\pgfsetlinewidth{0.803000pt}%
\definecolor{currentstroke}{rgb}{0.000000,0.000000,0.000000}%
\pgfsetstrokecolor{currentstroke}%
\pgfsetdash{}{0pt}%
\pgfsys@defobject{currentmarker}{\pgfqpoint{0.000000in}{-0.048611in}}{\pgfqpoint{0.000000in}{0.000000in}}{%
\pgfpathmoveto{\pgfqpoint{0.000000in}{0.000000in}}%
\pgfpathlineto{\pgfqpoint{0.000000in}{-0.048611in}}%
\pgfusepath{stroke,fill}%
}%
\begin{pgfscope}%
\pgfsys@transformshift{4.073420in}{0.500000in}%
\pgfsys@useobject{currentmarker}{}%
\end{pgfscope}%
\end{pgfscope}%
\begin{pgfscope}%
\definecolor{textcolor}{rgb}{0.000000,0.000000,0.000000}%
\pgfsetstrokecolor{textcolor}%
\pgfsetfillcolor{textcolor}%
\pgftext[x=4.073420in,y=0.402778in,,top]{\color{textcolor}\sffamily\fontsize{10.000000}{12.000000}\selectfont 0.40}%
\end{pgfscope}%
\begin{pgfscope}%
\pgfsetbuttcap%
\pgfsetroundjoin%
\definecolor{currentfill}{rgb}{0.000000,0.000000,0.000000}%
\pgfsetfillcolor{currentfill}%
\pgfsetlinewidth{0.803000pt}%
\definecolor{currentstroke}{rgb}{0.000000,0.000000,0.000000}%
\pgfsetstrokecolor{currentstroke}%
\pgfsetdash{}{0pt}%
\pgfsys@defobject{currentmarker}{\pgfqpoint{0.000000in}{-0.048611in}}{\pgfqpoint{0.000000in}{0.000000in}}{%
\pgfpathmoveto{\pgfqpoint{0.000000in}{0.000000in}}%
\pgfpathlineto{\pgfqpoint{0.000000in}{-0.048611in}}%
\pgfusepath{stroke,fill}%
}%
\begin{pgfscope}%
\pgfsys@transformshift{4.639433in}{0.500000in}%
\pgfsys@useobject{currentmarker}{}%
\end{pgfscope}%
\end{pgfscope}%
\begin{pgfscope}%
\definecolor{textcolor}{rgb}{0.000000,0.000000,0.000000}%
\pgfsetstrokecolor{textcolor}%
\pgfsetfillcolor{textcolor}%
\pgftext[x=4.639433in,y=0.402778in,,top]{\color{textcolor}\sffamily\fontsize{10.000000}{12.000000}\selectfont 0.45}%
\end{pgfscope}%
\begin{pgfscope}%
\pgfsetbuttcap%
\pgfsetroundjoin%
\definecolor{currentfill}{rgb}{0.000000,0.000000,0.000000}%
\pgfsetfillcolor{currentfill}%
\pgfsetlinewidth{0.803000pt}%
\definecolor{currentstroke}{rgb}{0.000000,0.000000,0.000000}%
\pgfsetstrokecolor{currentstroke}%
\pgfsetdash{}{0pt}%
\pgfsys@defobject{currentmarker}{\pgfqpoint{0.000000in}{-0.048611in}}{\pgfqpoint{0.000000in}{0.000000in}}{%
\pgfpathmoveto{\pgfqpoint{0.000000in}{0.000000in}}%
\pgfpathlineto{\pgfqpoint{0.000000in}{-0.048611in}}%
\pgfusepath{stroke,fill}%
}%
\begin{pgfscope}%
\pgfsys@transformshift{5.205445in}{0.500000in}%
\pgfsys@useobject{currentmarker}{}%
\end{pgfscope}%
\end{pgfscope}%
\begin{pgfscope}%
\definecolor{textcolor}{rgb}{0.000000,0.000000,0.000000}%
\pgfsetstrokecolor{textcolor}%
\pgfsetfillcolor{textcolor}%
\pgftext[x=5.205445in,y=0.402778in,,top]{\color{textcolor}\sffamily\fontsize{10.000000}{12.000000}\selectfont 0.50}%
\end{pgfscope}%
\begin{pgfscope}%
\definecolor{textcolor}{rgb}{0.000000,0.000000,0.000000}%
\pgfsetstrokecolor{textcolor}%
\pgfsetfillcolor{textcolor}%
\pgftext[x=3.075000in,y=0.212809in,,top]{\color{textcolor}\sffamily\fontsize{10.000000}{12.000000}\selectfont Loss}%
\end{pgfscope}%
\begin{pgfscope}%
\pgfsetbuttcap%
\pgfsetroundjoin%
\definecolor{currentfill}{rgb}{0.000000,0.000000,0.000000}%
\pgfsetfillcolor{currentfill}%
\pgfsetlinewidth{0.803000pt}%
\definecolor{currentstroke}{rgb}{0.000000,0.000000,0.000000}%
\pgfsetstrokecolor{currentstroke}%
\pgfsetdash{}{0pt}%
\pgfsys@defobject{currentmarker}{\pgfqpoint{-0.048611in}{0.000000in}}{\pgfqpoint{-0.000000in}{0.000000in}}{%
\pgfpathmoveto{\pgfqpoint{-0.000000in}{0.000000in}}%
\pgfpathlineto{\pgfqpoint{-0.048611in}{0.000000in}}%
\pgfusepath{stroke,fill}%
}%
\begin{pgfscope}%
\pgfsys@transformshift{0.750000in}{0.500000in}%
\pgfsys@useobject{currentmarker}{}%
\end{pgfscope}%
\end{pgfscope}%
\begin{pgfscope}%
\definecolor{textcolor}{rgb}{0.000000,0.000000,0.000000}%
\pgfsetstrokecolor{textcolor}%
\pgfsetfillcolor{textcolor}%
\pgftext[x=0.564412in, y=0.447238in, left, base]{\color{textcolor}\sffamily\fontsize{10.000000}{12.000000}\selectfont 0}%
\end{pgfscope}%
\begin{pgfscope}%
\pgfsetbuttcap%
\pgfsetroundjoin%
\definecolor{currentfill}{rgb}{0.000000,0.000000,0.000000}%
\pgfsetfillcolor{currentfill}%
\pgfsetlinewidth{0.803000pt}%
\definecolor{currentstroke}{rgb}{0.000000,0.000000,0.000000}%
\pgfsetstrokecolor{currentstroke}%
\pgfsetdash{}{0pt}%
\pgfsys@defobject{currentmarker}{\pgfqpoint{-0.048611in}{0.000000in}}{\pgfqpoint{-0.000000in}{0.000000in}}{%
\pgfpathmoveto{\pgfqpoint{-0.000000in}{0.000000in}}%
\pgfpathlineto{\pgfqpoint{-0.048611in}{0.000000in}}%
\pgfusepath{stroke,fill}%
}%
\begin{pgfscope}%
\pgfsys@transformshift{0.750000in}{1.055249in}%
\pgfsys@useobject{currentmarker}{}%
\end{pgfscope}%
\end{pgfscope}%
\begin{pgfscope}%
\definecolor{textcolor}{rgb}{0.000000,0.000000,0.000000}%
\pgfsetstrokecolor{textcolor}%
\pgfsetfillcolor{textcolor}%
\pgftext[x=0.387682in, y=1.002488in, left, base]{\color{textcolor}\sffamily\fontsize{10.000000}{12.000000}\selectfont 100}%
\end{pgfscope}%
\begin{pgfscope}%
\pgfsetbuttcap%
\pgfsetroundjoin%
\definecolor{currentfill}{rgb}{0.000000,0.000000,0.000000}%
\pgfsetfillcolor{currentfill}%
\pgfsetlinewidth{0.803000pt}%
\definecolor{currentstroke}{rgb}{0.000000,0.000000,0.000000}%
\pgfsetstrokecolor{currentstroke}%
\pgfsetdash{}{0pt}%
\pgfsys@defobject{currentmarker}{\pgfqpoint{-0.048611in}{0.000000in}}{\pgfqpoint{-0.000000in}{0.000000in}}{%
\pgfpathmoveto{\pgfqpoint{-0.000000in}{0.000000in}}%
\pgfpathlineto{\pgfqpoint{-0.048611in}{0.000000in}}%
\pgfusepath{stroke,fill}%
}%
\begin{pgfscope}%
\pgfsys@transformshift{0.750000in}{1.610498in}%
\pgfsys@useobject{currentmarker}{}%
\end{pgfscope}%
\end{pgfscope}%
\begin{pgfscope}%
\definecolor{textcolor}{rgb}{0.000000,0.000000,0.000000}%
\pgfsetstrokecolor{textcolor}%
\pgfsetfillcolor{textcolor}%
\pgftext[x=0.387682in, y=1.557737in, left, base]{\color{textcolor}\sffamily\fontsize{10.000000}{12.000000}\selectfont 200}%
\end{pgfscope}%
\begin{pgfscope}%
\pgfsetbuttcap%
\pgfsetroundjoin%
\definecolor{currentfill}{rgb}{0.000000,0.000000,0.000000}%
\pgfsetfillcolor{currentfill}%
\pgfsetlinewidth{0.803000pt}%
\definecolor{currentstroke}{rgb}{0.000000,0.000000,0.000000}%
\pgfsetstrokecolor{currentstroke}%
\pgfsetdash{}{0pt}%
\pgfsys@defobject{currentmarker}{\pgfqpoint{-0.048611in}{0.000000in}}{\pgfqpoint{-0.000000in}{0.000000in}}{%
\pgfpathmoveto{\pgfqpoint{-0.000000in}{0.000000in}}%
\pgfpathlineto{\pgfqpoint{-0.048611in}{0.000000in}}%
\pgfusepath{stroke,fill}%
}%
\begin{pgfscope}%
\pgfsys@transformshift{0.750000in}{2.165747in}%
\pgfsys@useobject{currentmarker}{}%
\end{pgfscope}%
\end{pgfscope}%
\begin{pgfscope}%
\definecolor{textcolor}{rgb}{0.000000,0.000000,0.000000}%
\pgfsetstrokecolor{textcolor}%
\pgfsetfillcolor{textcolor}%
\pgftext[x=0.387682in, y=2.112986in, left, base]{\color{textcolor}\sffamily\fontsize{10.000000}{12.000000}\selectfont 300}%
\end{pgfscope}%
\begin{pgfscope}%
\pgfsetbuttcap%
\pgfsetroundjoin%
\definecolor{currentfill}{rgb}{0.000000,0.000000,0.000000}%
\pgfsetfillcolor{currentfill}%
\pgfsetlinewidth{0.803000pt}%
\definecolor{currentstroke}{rgb}{0.000000,0.000000,0.000000}%
\pgfsetstrokecolor{currentstroke}%
\pgfsetdash{}{0pt}%
\pgfsys@defobject{currentmarker}{\pgfqpoint{-0.048611in}{0.000000in}}{\pgfqpoint{-0.000000in}{0.000000in}}{%
\pgfpathmoveto{\pgfqpoint{-0.000000in}{0.000000in}}%
\pgfpathlineto{\pgfqpoint{-0.048611in}{0.000000in}}%
\pgfusepath{stroke,fill}%
}%
\begin{pgfscope}%
\pgfsys@transformshift{0.750000in}{2.720997in}%
\pgfsys@useobject{currentmarker}{}%
\end{pgfscope}%
\end{pgfscope}%
\begin{pgfscope}%
\definecolor{textcolor}{rgb}{0.000000,0.000000,0.000000}%
\pgfsetstrokecolor{textcolor}%
\pgfsetfillcolor{textcolor}%
\pgftext[x=0.387682in, y=2.668235in, left, base]{\color{textcolor}\sffamily\fontsize{10.000000}{12.000000}\selectfont 400}%
\end{pgfscope}%
\begin{pgfscope}%
\pgfsetbuttcap%
\pgfsetroundjoin%
\definecolor{currentfill}{rgb}{0.000000,0.000000,0.000000}%
\pgfsetfillcolor{currentfill}%
\pgfsetlinewidth{0.803000pt}%
\definecolor{currentstroke}{rgb}{0.000000,0.000000,0.000000}%
\pgfsetstrokecolor{currentstroke}%
\pgfsetdash{}{0pt}%
\pgfsys@defobject{currentmarker}{\pgfqpoint{-0.048611in}{0.000000in}}{\pgfqpoint{-0.000000in}{0.000000in}}{%
\pgfpathmoveto{\pgfqpoint{-0.000000in}{0.000000in}}%
\pgfpathlineto{\pgfqpoint{-0.048611in}{0.000000in}}%
\pgfusepath{stroke,fill}%
}%
\begin{pgfscope}%
\pgfsys@transformshift{0.750000in}{3.276246in}%
\pgfsys@useobject{currentmarker}{}%
\end{pgfscope}%
\end{pgfscope}%
\begin{pgfscope}%
\definecolor{textcolor}{rgb}{0.000000,0.000000,0.000000}%
\pgfsetstrokecolor{textcolor}%
\pgfsetfillcolor{textcolor}%
\pgftext[x=0.387682in, y=3.223484in, left, base]{\color{textcolor}\sffamily\fontsize{10.000000}{12.000000}\selectfont 500}%
\end{pgfscope}%
\begin{pgfscope}%
\definecolor{textcolor}{rgb}{0.000000,0.000000,0.000000}%
\pgfsetstrokecolor{textcolor}%
\pgfsetfillcolor{textcolor}%
\pgftext[x=0.332126in,y=2.010000in,,bottom,rotate=90.000000]{\color{textcolor}\sffamily\fontsize{10.000000}{12.000000}\selectfont Count}%
\end{pgfscope}%
\begin{pgfscope}%
\pgfsetrectcap%
\pgfsetmiterjoin%
\pgfsetlinewidth{0.803000pt}%
\definecolor{currentstroke}{rgb}{0.000000,0.000000,0.000000}%
\pgfsetstrokecolor{currentstroke}%
\pgfsetdash{}{0pt}%
\pgfpathmoveto{\pgfqpoint{0.750000in}{0.500000in}}%
\pgfpathlineto{\pgfqpoint{0.750000in}{3.520000in}}%
\pgfusepath{stroke}%
\end{pgfscope}%
\begin{pgfscope}%
\pgfsetrectcap%
\pgfsetmiterjoin%
\pgfsetlinewidth{0.803000pt}%
\definecolor{currentstroke}{rgb}{0.000000,0.000000,0.000000}%
\pgfsetstrokecolor{currentstroke}%
\pgfsetdash{}{0pt}%
\pgfpathmoveto{\pgfqpoint{5.400000in}{0.500000in}}%
\pgfpathlineto{\pgfqpoint{5.400000in}{3.520000in}}%
\pgfusepath{stroke}%
\end{pgfscope}%
\begin{pgfscope}%
\pgfsetrectcap%
\pgfsetmiterjoin%
\pgfsetlinewidth{0.803000pt}%
\definecolor{currentstroke}{rgb}{0.000000,0.000000,0.000000}%
\pgfsetstrokecolor{currentstroke}%
\pgfsetdash{}{0pt}%
\pgfpathmoveto{\pgfqpoint{0.750000in}{0.500000in}}%
\pgfpathlineto{\pgfqpoint{5.400000in}{0.500000in}}%
\pgfusepath{stroke}%
\end{pgfscope}%
\begin{pgfscope}%
\pgfsetrectcap%
\pgfsetmiterjoin%
\pgfsetlinewidth{0.803000pt}%
\definecolor{currentstroke}{rgb}{0.000000,0.000000,0.000000}%
\pgfsetstrokecolor{currentstroke}%
\pgfsetdash{}{0pt}%
\pgfpathmoveto{\pgfqpoint{0.750000in}{3.520000in}}%
\pgfpathlineto{\pgfqpoint{5.400000in}{3.520000in}}%
\pgfusepath{stroke}%
\end{pgfscope}%
\begin{pgfscope}%
\definecolor{textcolor}{rgb}{0.000000,0.000000,0.000000}%
\pgfsetstrokecolor{textcolor}%
\pgfsetfillcolor{textcolor}%
\pgftext[x=3.075000in,y=3.603333in,,base]{\color{textcolor}\sffamily\fontsize{12.000000}{14.400000}\selectfont Loss Histogram for \(\displaystyle f(x)=x^2\)}%
\end{pgfscope}%
\begin{pgfscope}%
\pgfsetbuttcap%
\pgfsetmiterjoin%
\definecolor{currentfill}{rgb}{1.000000,1.000000,1.000000}%
\pgfsetfillcolor{currentfill}%
\pgfsetfillopacity{0.800000}%
\pgfsetlinewidth{1.003750pt}%
\definecolor{currentstroke}{rgb}{0.800000,0.800000,0.800000}%
\pgfsetstrokecolor{currentstroke}%
\pgfsetstrokeopacity{0.800000}%
\pgfsetdash{}{0pt}%
\pgfpathmoveto{\pgfqpoint{4.650543in}{3.205032in}}%
\pgfpathlineto{\pgfqpoint{5.302778in}{3.205032in}}%
\pgfpathquadraticcurveto{\pgfqpoint{5.330556in}{3.205032in}}{\pgfqpoint{5.330556in}{3.232809in}}%
\pgfpathlineto{\pgfqpoint{5.330556in}{3.422778in}}%
\pgfpathquadraticcurveto{\pgfqpoint{5.330556in}{3.450556in}}{\pgfqpoint{5.302778in}{3.450556in}}%
\pgfpathlineto{\pgfqpoint{4.650543in}{3.450556in}}%
\pgfpathquadraticcurveto{\pgfqpoint{4.622765in}{3.450556in}}{\pgfqpoint{4.622765in}{3.422778in}}%
\pgfpathlineto{\pgfqpoint{4.622765in}{3.232809in}}%
\pgfpathquadraticcurveto{\pgfqpoint{4.622765in}{3.205032in}}{\pgfqpoint{4.650543in}{3.205032in}}%
\pgfpathlineto{\pgfqpoint{4.650543in}{3.205032in}}%
\pgfpathclose%
\pgfusepath{stroke,fill}%
\end{pgfscope}%
\begin{pgfscope}%
\pgfsetbuttcap%
\pgfsetmiterjoin%
\definecolor{currentfill}{rgb}{0.000000,0.500000,0.000000}%
\pgfsetfillcolor{currentfill}%
\pgfsetlinewidth{0.000000pt}%
\definecolor{currentstroke}{rgb}{0.000000,0.000000,0.000000}%
\pgfsetstrokecolor{currentstroke}%
\pgfsetstrokeopacity{0.000000}%
\pgfsetdash{}{0pt}%
\pgfpathmoveto{\pgfqpoint{4.678320in}{3.289477in}}%
\pgfpathlineto{\pgfqpoint{4.956098in}{3.289477in}}%
\pgfpathlineto{\pgfqpoint{4.956098in}{3.386699in}}%
\pgfpathlineto{\pgfqpoint{4.678320in}{3.386699in}}%
\pgfpathlineto{\pgfqpoint{4.678320in}{3.289477in}}%
\pgfpathclose%
\pgfusepath{fill}%
\end{pgfscope}%
\begin{pgfscope}%
\definecolor{textcolor}{rgb}{0.000000,0.000000,0.000000}%
\pgfsetstrokecolor{textcolor}%
\pgfsetfillcolor{textcolor}%
\pgftext[x=5.067209in,y=3.289477in,left,base]{\color{textcolor}\sffamily\fontsize{10.000000}{12.000000}\selectfont NN}%
\end{pgfscope}%
\end{pgfpicture}%
\makeatother%
\endgroup%

    \caption{Caption}
    \label{fig:my_label}
\end{figure}

\begin{figure}
%% Creator: Matplotlib, PGF backend
%%
%% To include the figure in your LaTeX document, write
%%   \input{<filename>.pgf}
%%
%% Make sure the required packages are loaded in your preamble
%%   \usepackage{pgf}
%%
%% Also ensure that all the required font packages are loaded; for instance,
%% the lmodern package is sometimes necessary when using math font.
%%   \usepackage{lmodern}
%%
%% Figures using additional raster images can only be included by \input if
%% they are in the same directory as the main LaTeX file. For loading figures
%% from other directories you can use the `import` package
%%   \usepackage{import}
%%
%% and then include the figures with
%%   \import{<path to file>}{<filename>.pgf}
%%
%% Matplotlib used the following preamble
%%   \usepackage{fontspec}
%%   \setmainfont{DejaVuSerif.ttf}[Path=\detokenize{/Users/mkojro/miniforge3/envs/nn-crypto/lib/python3.10/site-packages/matplotlib/mpl-data/fonts/ttf/}]
%%   \setsansfont{DejaVuSans.ttf}[Path=\detokenize{/Users/mkojro/miniforge3/envs/nn-crypto/lib/python3.10/site-packages/matplotlib/mpl-data/fonts/ttf/}]
%%   \setmonofont{DejaVuSansMono.ttf}[Path=\detokenize{/Users/mkojro/miniforge3/envs/nn-crypto/lib/python3.10/site-packages/matplotlib/mpl-data/fonts/ttf/}]
%%
\begingroup%
\makeatletter%
\begin{pgfpicture}%
\pgfpathrectangle{\pgfpointorigin}{\pgfqpoint{6.000000in}{4.000000in}}%
\pgfusepath{use as bounding box, clip}%
\begin{pgfscope}%
\pgfsetbuttcap%
\pgfsetmiterjoin%
\pgfsetlinewidth{0.000000pt}%
\definecolor{currentstroke}{rgb}{1.000000,1.000000,1.000000}%
\pgfsetstrokecolor{currentstroke}%
\pgfsetstrokeopacity{0.000000}%
\pgfsetdash{}{0pt}%
\pgfpathmoveto{\pgfqpoint{0.000000in}{0.000000in}}%
\pgfpathlineto{\pgfqpoint{6.000000in}{0.000000in}}%
\pgfpathlineto{\pgfqpoint{6.000000in}{4.000000in}}%
\pgfpathlineto{\pgfqpoint{0.000000in}{4.000000in}}%
\pgfpathlineto{\pgfqpoint{0.000000in}{0.000000in}}%
\pgfpathclose%
\pgfusepath{}%
\end{pgfscope}%
\begin{pgfscope}%
\pgfsetbuttcap%
\pgfsetmiterjoin%
\definecolor{currentfill}{rgb}{1.000000,1.000000,1.000000}%
\pgfsetfillcolor{currentfill}%
\pgfsetlinewidth{0.000000pt}%
\definecolor{currentstroke}{rgb}{0.000000,0.000000,0.000000}%
\pgfsetstrokecolor{currentstroke}%
\pgfsetstrokeopacity{0.000000}%
\pgfsetdash{}{0pt}%
\pgfpathmoveto{\pgfqpoint{0.750000in}{0.500000in}}%
\pgfpathlineto{\pgfqpoint{5.400000in}{0.500000in}}%
\pgfpathlineto{\pgfqpoint{5.400000in}{3.520000in}}%
\pgfpathlineto{\pgfqpoint{0.750000in}{3.520000in}}%
\pgfpathlineto{\pgfqpoint{0.750000in}{0.500000in}}%
\pgfpathclose%
\pgfusepath{fill}%
\end{pgfscope}%
\begin{pgfscope}%
\pgfpathrectangle{\pgfqpoint{0.750000in}{0.500000in}}{\pgfqpoint{4.650000in}{3.020000in}}%
\pgfusepath{clip}%
\pgfsetbuttcap%
\pgfsetmiterjoin%
\definecolor{currentfill}{rgb}{1.000000,0.000000,0.000000}%
\pgfsetfillcolor{currentfill}%
\pgfsetlinewidth{0.000000pt}%
\definecolor{currentstroke}{rgb}{0.000000,0.000000,0.000000}%
\pgfsetstrokecolor{currentstroke}%
\pgfsetstrokeopacity{0.000000}%
\pgfsetdash{}{0pt}%
\pgfpathmoveto{\pgfqpoint{1.047716in}{0.500000in}}%
\pgfpathlineto{\pgfqpoint{1.080067in}{0.500000in}}%
\pgfpathlineto{\pgfqpoint{1.080067in}{0.502926in}}%
\pgfpathlineto{\pgfqpoint{1.047716in}{0.502926in}}%
\pgfpathlineto{\pgfqpoint{1.047716in}{0.500000in}}%
\pgfpathclose%
\pgfusepath{fill}%
\end{pgfscope}%
\begin{pgfscope}%
\pgfpathrectangle{\pgfqpoint{0.750000in}{0.500000in}}{\pgfqpoint{4.650000in}{3.020000in}}%
\pgfusepath{clip}%
\pgfsetbuttcap%
\pgfsetmiterjoin%
\definecolor{currentfill}{rgb}{1.000000,0.000000,0.000000}%
\pgfsetfillcolor{currentfill}%
\pgfsetlinewidth{0.000000pt}%
\definecolor{currentstroke}{rgb}{0.000000,0.000000,0.000000}%
\pgfsetstrokecolor{currentstroke}%
\pgfsetstrokeopacity{0.000000}%
\pgfsetdash{}{0pt}%
\pgfpathmoveto{\pgfqpoint{1.080067in}{0.500000in}}%
\pgfpathlineto{\pgfqpoint{1.112418in}{0.500000in}}%
\pgfpathlineto{\pgfqpoint{1.112418in}{0.500000in}}%
\pgfpathlineto{\pgfqpoint{1.080067in}{0.500000in}}%
\pgfpathlineto{\pgfqpoint{1.080067in}{0.500000in}}%
\pgfpathclose%
\pgfusepath{fill}%
\end{pgfscope}%
\begin{pgfscope}%
\pgfpathrectangle{\pgfqpoint{0.750000in}{0.500000in}}{\pgfqpoint{4.650000in}{3.020000in}}%
\pgfusepath{clip}%
\pgfsetbuttcap%
\pgfsetmiterjoin%
\definecolor{currentfill}{rgb}{1.000000,0.000000,0.000000}%
\pgfsetfillcolor{currentfill}%
\pgfsetlinewidth{0.000000pt}%
\definecolor{currentstroke}{rgb}{0.000000,0.000000,0.000000}%
\pgfsetstrokecolor{currentstroke}%
\pgfsetstrokeopacity{0.000000}%
\pgfsetdash{}{0pt}%
\pgfpathmoveto{\pgfqpoint{1.112418in}{0.500000in}}%
\pgfpathlineto{\pgfqpoint{1.144769in}{0.500000in}}%
\pgfpathlineto{\pgfqpoint{1.144769in}{0.505852in}}%
\pgfpathlineto{\pgfqpoint{1.112418in}{0.505852in}}%
\pgfpathlineto{\pgfqpoint{1.112418in}{0.500000in}}%
\pgfpathclose%
\pgfusepath{fill}%
\end{pgfscope}%
\begin{pgfscope}%
\pgfpathrectangle{\pgfqpoint{0.750000in}{0.500000in}}{\pgfqpoint{4.650000in}{3.020000in}}%
\pgfusepath{clip}%
\pgfsetbuttcap%
\pgfsetmiterjoin%
\definecolor{currentfill}{rgb}{1.000000,0.000000,0.000000}%
\pgfsetfillcolor{currentfill}%
\pgfsetlinewidth{0.000000pt}%
\definecolor{currentstroke}{rgb}{0.000000,0.000000,0.000000}%
\pgfsetstrokecolor{currentstroke}%
\pgfsetstrokeopacity{0.000000}%
\pgfsetdash{}{0pt}%
\pgfpathmoveto{\pgfqpoint{1.144769in}{0.500000in}}%
\pgfpathlineto{\pgfqpoint{1.177120in}{0.500000in}}%
\pgfpathlineto{\pgfqpoint{1.177120in}{0.500000in}}%
\pgfpathlineto{\pgfqpoint{1.144769in}{0.500000in}}%
\pgfpathlineto{\pgfqpoint{1.144769in}{0.500000in}}%
\pgfpathclose%
\pgfusepath{fill}%
\end{pgfscope}%
\begin{pgfscope}%
\pgfpathrectangle{\pgfqpoint{0.750000in}{0.500000in}}{\pgfqpoint{4.650000in}{3.020000in}}%
\pgfusepath{clip}%
\pgfsetbuttcap%
\pgfsetmiterjoin%
\definecolor{currentfill}{rgb}{1.000000,0.000000,0.000000}%
\pgfsetfillcolor{currentfill}%
\pgfsetlinewidth{0.000000pt}%
\definecolor{currentstroke}{rgb}{0.000000,0.000000,0.000000}%
\pgfsetstrokecolor{currentstroke}%
\pgfsetstrokeopacity{0.000000}%
\pgfsetdash{}{0pt}%
\pgfpathmoveto{\pgfqpoint{1.177120in}{0.500000in}}%
\pgfpathlineto{\pgfqpoint{1.209470in}{0.500000in}}%
\pgfpathlineto{\pgfqpoint{1.209470in}{0.500000in}}%
\pgfpathlineto{\pgfqpoint{1.177120in}{0.500000in}}%
\pgfpathlineto{\pgfqpoint{1.177120in}{0.500000in}}%
\pgfpathclose%
\pgfusepath{fill}%
\end{pgfscope}%
\begin{pgfscope}%
\pgfpathrectangle{\pgfqpoint{0.750000in}{0.500000in}}{\pgfqpoint{4.650000in}{3.020000in}}%
\pgfusepath{clip}%
\pgfsetbuttcap%
\pgfsetmiterjoin%
\definecolor{currentfill}{rgb}{1.000000,0.000000,0.000000}%
\pgfsetfillcolor{currentfill}%
\pgfsetlinewidth{0.000000pt}%
\definecolor{currentstroke}{rgb}{0.000000,0.000000,0.000000}%
\pgfsetstrokecolor{currentstroke}%
\pgfsetstrokeopacity{0.000000}%
\pgfsetdash{}{0pt}%
\pgfpathmoveto{\pgfqpoint{1.209470in}{0.500000in}}%
\pgfpathlineto{\pgfqpoint{1.241821in}{0.500000in}}%
\pgfpathlineto{\pgfqpoint{1.241821in}{0.517556in}}%
\pgfpathlineto{\pgfqpoint{1.209470in}{0.517556in}}%
\pgfpathlineto{\pgfqpoint{1.209470in}{0.500000in}}%
\pgfpathclose%
\pgfusepath{fill}%
\end{pgfscope}%
\begin{pgfscope}%
\pgfpathrectangle{\pgfqpoint{0.750000in}{0.500000in}}{\pgfqpoint{4.650000in}{3.020000in}}%
\pgfusepath{clip}%
\pgfsetbuttcap%
\pgfsetmiterjoin%
\definecolor{currentfill}{rgb}{1.000000,0.000000,0.000000}%
\pgfsetfillcolor{currentfill}%
\pgfsetlinewidth{0.000000pt}%
\definecolor{currentstroke}{rgb}{0.000000,0.000000,0.000000}%
\pgfsetstrokecolor{currentstroke}%
\pgfsetstrokeopacity{0.000000}%
\pgfsetdash{}{0pt}%
\pgfpathmoveto{\pgfqpoint{1.241821in}{0.500000in}}%
\pgfpathlineto{\pgfqpoint{1.274172in}{0.500000in}}%
\pgfpathlineto{\pgfqpoint{1.274172in}{0.500000in}}%
\pgfpathlineto{\pgfqpoint{1.241821in}{0.500000in}}%
\pgfpathlineto{\pgfqpoint{1.241821in}{0.500000in}}%
\pgfpathclose%
\pgfusepath{fill}%
\end{pgfscope}%
\begin{pgfscope}%
\pgfpathrectangle{\pgfqpoint{0.750000in}{0.500000in}}{\pgfqpoint{4.650000in}{3.020000in}}%
\pgfusepath{clip}%
\pgfsetbuttcap%
\pgfsetmiterjoin%
\definecolor{currentfill}{rgb}{1.000000,0.000000,0.000000}%
\pgfsetfillcolor{currentfill}%
\pgfsetlinewidth{0.000000pt}%
\definecolor{currentstroke}{rgb}{0.000000,0.000000,0.000000}%
\pgfsetstrokecolor{currentstroke}%
\pgfsetstrokeopacity{0.000000}%
\pgfsetdash{}{0pt}%
\pgfpathmoveto{\pgfqpoint{1.274172in}{0.500000in}}%
\pgfpathlineto{\pgfqpoint{1.306523in}{0.500000in}}%
\pgfpathlineto{\pgfqpoint{1.306523in}{0.502926in}}%
\pgfpathlineto{\pgfqpoint{1.274172in}{0.502926in}}%
\pgfpathlineto{\pgfqpoint{1.274172in}{0.500000in}}%
\pgfpathclose%
\pgfusepath{fill}%
\end{pgfscope}%
\begin{pgfscope}%
\pgfpathrectangle{\pgfqpoint{0.750000in}{0.500000in}}{\pgfqpoint{4.650000in}{3.020000in}}%
\pgfusepath{clip}%
\pgfsetbuttcap%
\pgfsetmiterjoin%
\definecolor{currentfill}{rgb}{1.000000,0.000000,0.000000}%
\pgfsetfillcolor{currentfill}%
\pgfsetlinewidth{0.000000pt}%
\definecolor{currentstroke}{rgb}{0.000000,0.000000,0.000000}%
\pgfsetstrokecolor{currentstroke}%
\pgfsetstrokeopacity{0.000000}%
\pgfsetdash{}{0pt}%
\pgfpathmoveto{\pgfqpoint{1.306523in}{0.500000in}}%
\pgfpathlineto{\pgfqpoint{1.338874in}{0.500000in}}%
\pgfpathlineto{\pgfqpoint{1.338874in}{0.529259in}}%
\pgfpathlineto{\pgfqpoint{1.306523in}{0.529259in}}%
\pgfpathlineto{\pgfqpoint{1.306523in}{0.500000in}}%
\pgfpathclose%
\pgfusepath{fill}%
\end{pgfscope}%
\begin{pgfscope}%
\pgfpathrectangle{\pgfqpoint{0.750000in}{0.500000in}}{\pgfqpoint{4.650000in}{3.020000in}}%
\pgfusepath{clip}%
\pgfsetbuttcap%
\pgfsetmiterjoin%
\definecolor{currentfill}{rgb}{1.000000,0.000000,0.000000}%
\pgfsetfillcolor{currentfill}%
\pgfsetlinewidth{0.000000pt}%
\definecolor{currentstroke}{rgb}{0.000000,0.000000,0.000000}%
\pgfsetstrokecolor{currentstroke}%
\pgfsetstrokeopacity{0.000000}%
\pgfsetdash{}{0pt}%
\pgfpathmoveto{\pgfqpoint{1.338874in}{0.500000in}}%
\pgfpathlineto{\pgfqpoint{1.371225in}{0.500000in}}%
\pgfpathlineto{\pgfqpoint{1.371225in}{0.500000in}}%
\pgfpathlineto{\pgfqpoint{1.338874in}{0.500000in}}%
\pgfpathlineto{\pgfqpoint{1.338874in}{0.500000in}}%
\pgfpathclose%
\pgfusepath{fill}%
\end{pgfscope}%
\begin{pgfscope}%
\pgfpathrectangle{\pgfqpoint{0.750000in}{0.500000in}}{\pgfqpoint{4.650000in}{3.020000in}}%
\pgfusepath{clip}%
\pgfsetbuttcap%
\pgfsetmiterjoin%
\definecolor{currentfill}{rgb}{1.000000,0.000000,0.000000}%
\pgfsetfillcolor{currentfill}%
\pgfsetlinewidth{0.000000pt}%
\definecolor{currentstroke}{rgb}{0.000000,0.000000,0.000000}%
\pgfsetstrokecolor{currentstroke}%
\pgfsetstrokeopacity{0.000000}%
\pgfsetdash{}{0pt}%
\pgfpathmoveto{\pgfqpoint{1.371225in}{0.500000in}}%
\pgfpathlineto{\pgfqpoint{1.403576in}{0.500000in}}%
\pgfpathlineto{\pgfqpoint{1.403576in}{0.579000in}}%
\pgfpathlineto{\pgfqpoint{1.371225in}{0.579000in}}%
\pgfpathlineto{\pgfqpoint{1.371225in}{0.500000in}}%
\pgfpathclose%
\pgfusepath{fill}%
\end{pgfscope}%
\begin{pgfscope}%
\pgfpathrectangle{\pgfqpoint{0.750000in}{0.500000in}}{\pgfqpoint{4.650000in}{3.020000in}}%
\pgfusepath{clip}%
\pgfsetbuttcap%
\pgfsetmiterjoin%
\definecolor{currentfill}{rgb}{1.000000,0.000000,0.000000}%
\pgfsetfillcolor{currentfill}%
\pgfsetlinewidth{0.000000pt}%
\definecolor{currentstroke}{rgb}{0.000000,0.000000,0.000000}%
\pgfsetstrokecolor{currentstroke}%
\pgfsetstrokeopacity{0.000000}%
\pgfsetdash{}{0pt}%
\pgfpathmoveto{\pgfqpoint{1.403576in}{0.500000in}}%
\pgfpathlineto{\pgfqpoint{1.435927in}{0.500000in}}%
\pgfpathlineto{\pgfqpoint{1.435927in}{0.502926in}}%
\pgfpathlineto{\pgfqpoint{1.403576in}{0.502926in}}%
\pgfpathlineto{\pgfqpoint{1.403576in}{0.500000in}}%
\pgfpathclose%
\pgfusepath{fill}%
\end{pgfscope}%
\begin{pgfscope}%
\pgfpathrectangle{\pgfqpoint{0.750000in}{0.500000in}}{\pgfqpoint{4.650000in}{3.020000in}}%
\pgfusepath{clip}%
\pgfsetbuttcap%
\pgfsetmiterjoin%
\definecolor{currentfill}{rgb}{1.000000,0.000000,0.000000}%
\pgfsetfillcolor{currentfill}%
\pgfsetlinewidth{0.000000pt}%
\definecolor{currentstroke}{rgb}{0.000000,0.000000,0.000000}%
\pgfsetstrokecolor{currentstroke}%
\pgfsetstrokeopacity{0.000000}%
\pgfsetdash{}{0pt}%
\pgfpathmoveto{\pgfqpoint{1.435927in}{0.500000in}}%
\pgfpathlineto{\pgfqpoint{1.468278in}{0.500000in}}%
\pgfpathlineto{\pgfqpoint{1.468278in}{0.500000in}}%
\pgfpathlineto{\pgfqpoint{1.435927in}{0.500000in}}%
\pgfpathlineto{\pgfqpoint{1.435927in}{0.500000in}}%
\pgfpathclose%
\pgfusepath{fill}%
\end{pgfscope}%
\begin{pgfscope}%
\pgfpathrectangle{\pgfqpoint{0.750000in}{0.500000in}}{\pgfqpoint{4.650000in}{3.020000in}}%
\pgfusepath{clip}%
\pgfsetbuttcap%
\pgfsetmiterjoin%
\definecolor{currentfill}{rgb}{1.000000,0.000000,0.000000}%
\pgfsetfillcolor{currentfill}%
\pgfsetlinewidth{0.000000pt}%
\definecolor{currentstroke}{rgb}{0.000000,0.000000,0.000000}%
\pgfsetstrokecolor{currentstroke}%
\pgfsetstrokeopacity{0.000000}%
\pgfsetdash{}{0pt}%
\pgfpathmoveto{\pgfqpoint{1.468278in}{0.500000in}}%
\pgfpathlineto{\pgfqpoint{1.500629in}{0.500000in}}%
\pgfpathlineto{\pgfqpoint{1.500629in}{0.617037in}}%
\pgfpathlineto{\pgfqpoint{1.468278in}{0.617037in}}%
\pgfpathlineto{\pgfqpoint{1.468278in}{0.500000in}}%
\pgfpathclose%
\pgfusepath{fill}%
\end{pgfscope}%
\begin{pgfscope}%
\pgfpathrectangle{\pgfqpoint{0.750000in}{0.500000in}}{\pgfqpoint{4.650000in}{3.020000in}}%
\pgfusepath{clip}%
\pgfsetbuttcap%
\pgfsetmiterjoin%
\definecolor{currentfill}{rgb}{1.000000,0.000000,0.000000}%
\pgfsetfillcolor{currentfill}%
\pgfsetlinewidth{0.000000pt}%
\definecolor{currentstroke}{rgb}{0.000000,0.000000,0.000000}%
\pgfsetstrokecolor{currentstroke}%
\pgfsetstrokeopacity{0.000000}%
\pgfsetdash{}{0pt}%
\pgfpathmoveto{\pgfqpoint{1.500629in}{0.500000in}}%
\pgfpathlineto{\pgfqpoint{1.532980in}{0.500000in}}%
\pgfpathlineto{\pgfqpoint{1.532980in}{0.505852in}}%
\pgfpathlineto{\pgfqpoint{1.500629in}{0.505852in}}%
\pgfpathlineto{\pgfqpoint{1.500629in}{0.500000in}}%
\pgfpathclose%
\pgfusepath{fill}%
\end{pgfscope}%
\begin{pgfscope}%
\pgfpathrectangle{\pgfqpoint{0.750000in}{0.500000in}}{\pgfqpoint{4.650000in}{3.020000in}}%
\pgfusepath{clip}%
\pgfsetbuttcap%
\pgfsetmiterjoin%
\definecolor{currentfill}{rgb}{1.000000,0.000000,0.000000}%
\pgfsetfillcolor{currentfill}%
\pgfsetlinewidth{0.000000pt}%
\definecolor{currentstroke}{rgb}{0.000000,0.000000,0.000000}%
\pgfsetstrokecolor{currentstroke}%
\pgfsetstrokeopacity{0.000000}%
\pgfsetdash{}{0pt}%
\pgfpathmoveto{\pgfqpoint{1.532980in}{0.500000in}}%
\pgfpathlineto{\pgfqpoint{1.565331in}{0.500000in}}%
\pgfpathlineto{\pgfqpoint{1.565331in}{0.517556in}}%
\pgfpathlineto{\pgfqpoint{1.532980in}{0.517556in}}%
\pgfpathlineto{\pgfqpoint{1.532980in}{0.500000in}}%
\pgfpathclose%
\pgfusepath{fill}%
\end{pgfscope}%
\begin{pgfscope}%
\pgfpathrectangle{\pgfqpoint{0.750000in}{0.500000in}}{\pgfqpoint{4.650000in}{3.020000in}}%
\pgfusepath{clip}%
\pgfsetbuttcap%
\pgfsetmiterjoin%
\definecolor{currentfill}{rgb}{1.000000,0.000000,0.000000}%
\pgfsetfillcolor{currentfill}%
\pgfsetlinewidth{0.000000pt}%
\definecolor{currentstroke}{rgb}{0.000000,0.000000,0.000000}%
\pgfsetstrokecolor{currentstroke}%
\pgfsetstrokeopacity{0.000000}%
\pgfsetdash{}{0pt}%
\pgfpathmoveto{\pgfqpoint{1.565331in}{0.500000in}}%
\pgfpathlineto{\pgfqpoint{1.597682in}{0.500000in}}%
\pgfpathlineto{\pgfqpoint{1.597682in}{0.707741in}}%
\pgfpathlineto{\pgfqpoint{1.565331in}{0.707741in}}%
\pgfpathlineto{\pgfqpoint{1.565331in}{0.500000in}}%
\pgfpathclose%
\pgfusepath{fill}%
\end{pgfscope}%
\begin{pgfscope}%
\pgfpathrectangle{\pgfqpoint{0.750000in}{0.500000in}}{\pgfqpoint{4.650000in}{3.020000in}}%
\pgfusepath{clip}%
\pgfsetbuttcap%
\pgfsetmiterjoin%
\definecolor{currentfill}{rgb}{1.000000,0.000000,0.000000}%
\pgfsetfillcolor{currentfill}%
\pgfsetlinewidth{0.000000pt}%
\definecolor{currentstroke}{rgb}{0.000000,0.000000,0.000000}%
\pgfsetstrokecolor{currentstroke}%
\pgfsetstrokeopacity{0.000000}%
\pgfsetdash{}{0pt}%
\pgfpathmoveto{\pgfqpoint{1.597682in}{0.500000in}}%
\pgfpathlineto{\pgfqpoint{1.630033in}{0.500000in}}%
\pgfpathlineto{\pgfqpoint{1.630033in}{0.505852in}}%
\pgfpathlineto{\pgfqpoint{1.597682in}{0.505852in}}%
\pgfpathlineto{\pgfqpoint{1.597682in}{0.500000in}}%
\pgfpathclose%
\pgfusepath{fill}%
\end{pgfscope}%
\begin{pgfscope}%
\pgfpathrectangle{\pgfqpoint{0.750000in}{0.500000in}}{\pgfqpoint{4.650000in}{3.020000in}}%
\pgfusepath{clip}%
\pgfsetbuttcap%
\pgfsetmiterjoin%
\definecolor{currentfill}{rgb}{1.000000,0.000000,0.000000}%
\pgfsetfillcolor{currentfill}%
\pgfsetlinewidth{0.000000pt}%
\definecolor{currentstroke}{rgb}{0.000000,0.000000,0.000000}%
\pgfsetstrokecolor{currentstroke}%
\pgfsetstrokeopacity{0.000000}%
\pgfsetdash{}{0pt}%
\pgfpathmoveto{\pgfqpoint{1.630033in}{0.500000in}}%
\pgfpathlineto{\pgfqpoint{1.662384in}{0.500000in}}%
\pgfpathlineto{\pgfqpoint{1.662384in}{0.871593in}}%
\pgfpathlineto{\pgfqpoint{1.630033in}{0.871593in}}%
\pgfpathlineto{\pgfqpoint{1.630033in}{0.500000in}}%
\pgfpathclose%
\pgfusepath{fill}%
\end{pgfscope}%
\begin{pgfscope}%
\pgfpathrectangle{\pgfqpoint{0.750000in}{0.500000in}}{\pgfqpoint{4.650000in}{3.020000in}}%
\pgfusepath{clip}%
\pgfsetbuttcap%
\pgfsetmiterjoin%
\definecolor{currentfill}{rgb}{1.000000,0.000000,0.000000}%
\pgfsetfillcolor{currentfill}%
\pgfsetlinewidth{0.000000pt}%
\definecolor{currentstroke}{rgb}{0.000000,0.000000,0.000000}%
\pgfsetstrokecolor{currentstroke}%
\pgfsetstrokeopacity{0.000000}%
\pgfsetdash{}{0pt}%
\pgfpathmoveto{\pgfqpoint{1.662384in}{0.500000in}}%
\pgfpathlineto{\pgfqpoint{1.694735in}{0.500000in}}%
\pgfpathlineto{\pgfqpoint{1.694735in}{0.511704in}}%
\pgfpathlineto{\pgfqpoint{1.662384in}{0.511704in}}%
\pgfpathlineto{\pgfqpoint{1.662384in}{0.500000in}}%
\pgfpathclose%
\pgfusepath{fill}%
\end{pgfscope}%
\begin{pgfscope}%
\pgfpathrectangle{\pgfqpoint{0.750000in}{0.500000in}}{\pgfqpoint{4.650000in}{3.020000in}}%
\pgfusepath{clip}%
\pgfsetbuttcap%
\pgfsetmiterjoin%
\definecolor{currentfill}{rgb}{1.000000,0.000000,0.000000}%
\pgfsetfillcolor{currentfill}%
\pgfsetlinewidth{0.000000pt}%
\definecolor{currentstroke}{rgb}{0.000000,0.000000,0.000000}%
\pgfsetstrokecolor{currentstroke}%
\pgfsetstrokeopacity{0.000000}%
\pgfsetdash{}{0pt}%
\pgfpathmoveto{\pgfqpoint{1.694735in}{0.500000in}}%
\pgfpathlineto{\pgfqpoint{1.727086in}{0.500000in}}%
\pgfpathlineto{\pgfqpoint{1.727086in}{0.508778in}}%
\pgfpathlineto{\pgfqpoint{1.694735in}{0.508778in}}%
\pgfpathlineto{\pgfqpoint{1.694735in}{0.500000in}}%
\pgfpathclose%
\pgfusepath{fill}%
\end{pgfscope}%
\begin{pgfscope}%
\pgfpathrectangle{\pgfqpoint{0.750000in}{0.500000in}}{\pgfqpoint{4.650000in}{3.020000in}}%
\pgfusepath{clip}%
\pgfsetbuttcap%
\pgfsetmiterjoin%
\definecolor{currentfill}{rgb}{1.000000,0.000000,0.000000}%
\pgfsetfillcolor{currentfill}%
\pgfsetlinewidth{0.000000pt}%
\definecolor{currentstroke}{rgb}{0.000000,0.000000,0.000000}%
\pgfsetstrokecolor{currentstroke}%
\pgfsetstrokeopacity{0.000000}%
\pgfsetdash{}{0pt}%
\pgfpathmoveto{\pgfqpoint{1.727086in}{0.500000in}}%
\pgfpathlineto{\pgfqpoint{1.759437in}{0.500000in}}%
\pgfpathlineto{\pgfqpoint{1.759437in}{1.108594in}}%
\pgfpathlineto{\pgfqpoint{1.727086in}{1.108594in}}%
\pgfpathlineto{\pgfqpoint{1.727086in}{0.500000in}}%
\pgfpathclose%
\pgfusepath{fill}%
\end{pgfscope}%
\begin{pgfscope}%
\pgfpathrectangle{\pgfqpoint{0.750000in}{0.500000in}}{\pgfqpoint{4.650000in}{3.020000in}}%
\pgfusepath{clip}%
\pgfsetbuttcap%
\pgfsetmiterjoin%
\definecolor{currentfill}{rgb}{1.000000,0.000000,0.000000}%
\pgfsetfillcolor{currentfill}%
\pgfsetlinewidth{0.000000pt}%
\definecolor{currentstroke}{rgb}{0.000000,0.000000,0.000000}%
\pgfsetstrokecolor{currentstroke}%
\pgfsetstrokeopacity{0.000000}%
\pgfsetdash{}{0pt}%
\pgfpathmoveto{\pgfqpoint{1.759437in}{0.500000in}}%
\pgfpathlineto{\pgfqpoint{1.791787in}{0.500000in}}%
\pgfpathlineto{\pgfqpoint{1.791787in}{0.526333in}}%
\pgfpathlineto{\pgfqpoint{1.759437in}{0.526333in}}%
\pgfpathlineto{\pgfqpoint{1.759437in}{0.500000in}}%
\pgfpathclose%
\pgfusepath{fill}%
\end{pgfscope}%
\begin{pgfscope}%
\pgfpathrectangle{\pgfqpoint{0.750000in}{0.500000in}}{\pgfqpoint{4.650000in}{3.020000in}}%
\pgfusepath{clip}%
\pgfsetbuttcap%
\pgfsetmiterjoin%
\definecolor{currentfill}{rgb}{1.000000,0.000000,0.000000}%
\pgfsetfillcolor{currentfill}%
\pgfsetlinewidth{0.000000pt}%
\definecolor{currentstroke}{rgb}{0.000000,0.000000,0.000000}%
\pgfsetstrokecolor{currentstroke}%
\pgfsetstrokeopacity{0.000000}%
\pgfsetdash{}{0pt}%
\pgfpathmoveto{\pgfqpoint{1.791787in}{0.500000in}}%
\pgfpathlineto{\pgfqpoint{1.824138in}{0.500000in}}%
\pgfpathlineto{\pgfqpoint{1.824138in}{0.523407in}}%
\pgfpathlineto{\pgfqpoint{1.791787in}{0.523407in}}%
\pgfpathlineto{\pgfqpoint{1.791787in}{0.500000in}}%
\pgfpathclose%
\pgfusepath{fill}%
\end{pgfscope}%
\begin{pgfscope}%
\pgfpathrectangle{\pgfqpoint{0.750000in}{0.500000in}}{\pgfqpoint{4.650000in}{3.020000in}}%
\pgfusepath{clip}%
\pgfsetbuttcap%
\pgfsetmiterjoin%
\definecolor{currentfill}{rgb}{1.000000,0.000000,0.000000}%
\pgfsetfillcolor{currentfill}%
\pgfsetlinewidth{0.000000pt}%
\definecolor{currentstroke}{rgb}{0.000000,0.000000,0.000000}%
\pgfsetstrokecolor{currentstroke}%
\pgfsetstrokeopacity{0.000000}%
\pgfsetdash{}{0pt}%
\pgfpathmoveto{\pgfqpoint{1.824138in}{0.500000in}}%
\pgfpathlineto{\pgfqpoint{1.856489in}{0.500000in}}%
\pgfpathlineto{\pgfqpoint{1.856489in}{1.448002in}}%
\pgfpathlineto{\pgfqpoint{1.824138in}{1.448002in}}%
\pgfpathlineto{\pgfqpoint{1.824138in}{0.500000in}}%
\pgfpathclose%
\pgfusepath{fill}%
\end{pgfscope}%
\begin{pgfscope}%
\pgfpathrectangle{\pgfqpoint{0.750000in}{0.500000in}}{\pgfqpoint{4.650000in}{3.020000in}}%
\pgfusepath{clip}%
\pgfsetbuttcap%
\pgfsetmiterjoin%
\definecolor{currentfill}{rgb}{1.000000,0.000000,0.000000}%
\pgfsetfillcolor{currentfill}%
\pgfsetlinewidth{0.000000pt}%
\definecolor{currentstroke}{rgb}{0.000000,0.000000,0.000000}%
\pgfsetstrokecolor{currentstroke}%
\pgfsetstrokeopacity{0.000000}%
\pgfsetdash{}{0pt}%
\pgfpathmoveto{\pgfqpoint{1.856489in}{0.500000in}}%
\pgfpathlineto{\pgfqpoint{1.888840in}{0.500000in}}%
\pgfpathlineto{\pgfqpoint{1.888840in}{0.511704in}}%
\pgfpathlineto{\pgfqpoint{1.856489in}{0.511704in}}%
\pgfpathlineto{\pgfqpoint{1.856489in}{0.500000in}}%
\pgfpathclose%
\pgfusepath{fill}%
\end{pgfscope}%
\begin{pgfscope}%
\pgfpathrectangle{\pgfqpoint{0.750000in}{0.500000in}}{\pgfqpoint{4.650000in}{3.020000in}}%
\pgfusepath{clip}%
\pgfsetbuttcap%
\pgfsetmiterjoin%
\definecolor{currentfill}{rgb}{1.000000,0.000000,0.000000}%
\pgfsetfillcolor{currentfill}%
\pgfsetlinewidth{0.000000pt}%
\definecolor{currentstroke}{rgb}{0.000000,0.000000,0.000000}%
\pgfsetstrokecolor{currentstroke}%
\pgfsetstrokeopacity{0.000000}%
\pgfsetdash{}{0pt}%
\pgfpathmoveto{\pgfqpoint{1.888840in}{0.500000in}}%
\pgfpathlineto{\pgfqpoint{1.921191in}{0.500000in}}%
\pgfpathlineto{\pgfqpoint{1.921191in}{0.608259in}}%
\pgfpathlineto{\pgfqpoint{1.888840in}{0.608259in}}%
\pgfpathlineto{\pgfqpoint{1.888840in}{0.500000in}}%
\pgfpathclose%
\pgfusepath{fill}%
\end{pgfscope}%
\begin{pgfscope}%
\pgfpathrectangle{\pgfqpoint{0.750000in}{0.500000in}}{\pgfqpoint{4.650000in}{3.020000in}}%
\pgfusepath{clip}%
\pgfsetbuttcap%
\pgfsetmiterjoin%
\definecolor{currentfill}{rgb}{1.000000,0.000000,0.000000}%
\pgfsetfillcolor{currentfill}%
\pgfsetlinewidth{0.000000pt}%
\definecolor{currentstroke}{rgb}{0.000000,0.000000,0.000000}%
\pgfsetstrokecolor{currentstroke}%
\pgfsetstrokeopacity{0.000000}%
\pgfsetdash{}{0pt}%
\pgfpathmoveto{\pgfqpoint{1.921191in}{0.500000in}}%
\pgfpathlineto{\pgfqpoint{1.953542in}{0.500000in}}%
\pgfpathlineto{\pgfqpoint{1.953542in}{1.752299in}}%
\pgfpathlineto{\pgfqpoint{1.921191in}{1.752299in}}%
\pgfpathlineto{\pgfqpoint{1.921191in}{0.500000in}}%
\pgfpathclose%
\pgfusepath{fill}%
\end{pgfscope}%
\begin{pgfscope}%
\pgfpathrectangle{\pgfqpoint{0.750000in}{0.500000in}}{\pgfqpoint{4.650000in}{3.020000in}}%
\pgfusepath{clip}%
\pgfsetbuttcap%
\pgfsetmiterjoin%
\definecolor{currentfill}{rgb}{1.000000,0.000000,0.000000}%
\pgfsetfillcolor{currentfill}%
\pgfsetlinewidth{0.000000pt}%
\definecolor{currentstroke}{rgb}{0.000000,0.000000,0.000000}%
\pgfsetstrokecolor{currentstroke}%
\pgfsetstrokeopacity{0.000000}%
\pgfsetdash{}{0pt}%
\pgfpathmoveto{\pgfqpoint{1.953542in}{0.500000in}}%
\pgfpathlineto{\pgfqpoint{1.985893in}{0.500000in}}%
\pgfpathlineto{\pgfqpoint{1.985893in}{0.532185in}}%
\pgfpathlineto{\pgfqpoint{1.953542in}{0.532185in}}%
\pgfpathlineto{\pgfqpoint{1.953542in}{0.500000in}}%
\pgfpathclose%
\pgfusepath{fill}%
\end{pgfscope}%
\begin{pgfscope}%
\pgfpathrectangle{\pgfqpoint{0.750000in}{0.500000in}}{\pgfqpoint{4.650000in}{3.020000in}}%
\pgfusepath{clip}%
\pgfsetbuttcap%
\pgfsetmiterjoin%
\definecolor{currentfill}{rgb}{1.000000,0.000000,0.000000}%
\pgfsetfillcolor{currentfill}%
\pgfsetlinewidth{0.000000pt}%
\definecolor{currentstroke}{rgb}{0.000000,0.000000,0.000000}%
\pgfsetstrokecolor{currentstroke}%
\pgfsetstrokeopacity{0.000000}%
\pgfsetdash{}{0pt}%
\pgfpathmoveto{\pgfqpoint{1.985893in}{0.500000in}}%
\pgfpathlineto{\pgfqpoint{2.018244in}{0.500000in}}%
\pgfpathlineto{\pgfqpoint{2.018244in}{2.194114in}}%
\pgfpathlineto{\pgfqpoint{1.985893in}{2.194114in}}%
\pgfpathlineto{\pgfqpoint{1.985893in}{0.500000in}}%
\pgfpathclose%
\pgfusepath{fill}%
\end{pgfscope}%
\begin{pgfscope}%
\pgfpathrectangle{\pgfqpoint{0.750000in}{0.500000in}}{\pgfqpoint{4.650000in}{3.020000in}}%
\pgfusepath{clip}%
\pgfsetbuttcap%
\pgfsetmiterjoin%
\definecolor{currentfill}{rgb}{1.000000,0.000000,0.000000}%
\pgfsetfillcolor{currentfill}%
\pgfsetlinewidth{0.000000pt}%
\definecolor{currentstroke}{rgb}{0.000000,0.000000,0.000000}%
\pgfsetstrokecolor{currentstroke}%
\pgfsetstrokeopacity{0.000000}%
\pgfsetdash{}{0pt}%
\pgfpathmoveto{\pgfqpoint{2.018244in}{0.500000in}}%
\pgfpathlineto{\pgfqpoint{2.050595in}{0.500000in}}%
\pgfpathlineto{\pgfqpoint{2.050595in}{0.526333in}}%
\pgfpathlineto{\pgfqpoint{2.018244in}{0.526333in}}%
\pgfpathlineto{\pgfqpoint{2.018244in}{0.500000in}}%
\pgfpathclose%
\pgfusepath{fill}%
\end{pgfscope}%
\begin{pgfscope}%
\pgfpathrectangle{\pgfqpoint{0.750000in}{0.500000in}}{\pgfqpoint{4.650000in}{3.020000in}}%
\pgfusepath{clip}%
\pgfsetbuttcap%
\pgfsetmiterjoin%
\definecolor{currentfill}{rgb}{1.000000,0.000000,0.000000}%
\pgfsetfillcolor{currentfill}%
\pgfsetlinewidth{0.000000pt}%
\definecolor{currentstroke}{rgb}{0.000000,0.000000,0.000000}%
\pgfsetstrokecolor{currentstroke}%
\pgfsetstrokeopacity{0.000000}%
\pgfsetdash{}{0pt}%
\pgfpathmoveto{\pgfqpoint{2.050595in}{0.500000in}}%
\pgfpathlineto{\pgfqpoint{2.082946in}{0.500000in}}%
\pgfpathlineto{\pgfqpoint{2.082946in}{0.546815in}}%
\pgfpathlineto{\pgfqpoint{2.050595in}{0.546815in}}%
\pgfpathlineto{\pgfqpoint{2.050595in}{0.500000in}}%
\pgfpathclose%
\pgfusepath{fill}%
\end{pgfscope}%
\begin{pgfscope}%
\pgfpathrectangle{\pgfqpoint{0.750000in}{0.500000in}}{\pgfqpoint{4.650000in}{3.020000in}}%
\pgfusepath{clip}%
\pgfsetbuttcap%
\pgfsetmiterjoin%
\definecolor{currentfill}{rgb}{1.000000,0.000000,0.000000}%
\pgfsetfillcolor{currentfill}%
\pgfsetlinewidth{0.000000pt}%
\definecolor{currentstroke}{rgb}{0.000000,0.000000,0.000000}%
\pgfsetstrokecolor{currentstroke}%
\pgfsetstrokeopacity{0.000000}%
\pgfsetdash{}{0pt}%
\pgfpathmoveto{\pgfqpoint{2.082946in}{0.500000in}}%
\pgfpathlineto{\pgfqpoint{2.115297in}{0.500000in}}%
\pgfpathlineto{\pgfqpoint{2.115297in}{2.673967in}}%
\pgfpathlineto{\pgfqpoint{2.082946in}{2.673967in}}%
\pgfpathlineto{\pgfqpoint{2.082946in}{0.500000in}}%
\pgfpathclose%
\pgfusepath{fill}%
\end{pgfscope}%
\begin{pgfscope}%
\pgfpathrectangle{\pgfqpoint{0.750000in}{0.500000in}}{\pgfqpoint{4.650000in}{3.020000in}}%
\pgfusepath{clip}%
\pgfsetbuttcap%
\pgfsetmiterjoin%
\definecolor{currentfill}{rgb}{1.000000,0.000000,0.000000}%
\pgfsetfillcolor{currentfill}%
\pgfsetlinewidth{0.000000pt}%
\definecolor{currentstroke}{rgb}{0.000000,0.000000,0.000000}%
\pgfsetstrokecolor{currentstroke}%
\pgfsetstrokeopacity{0.000000}%
\pgfsetdash{}{0pt}%
\pgfpathmoveto{\pgfqpoint{2.115297in}{0.500000in}}%
\pgfpathlineto{\pgfqpoint{2.147648in}{0.500000in}}%
\pgfpathlineto{\pgfqpoint{2.147648in}{0.529259in}}%
\pgfpathlineto{\pgfqpoint{2.115297in}{0.529259in}}%
\pgfpathlineto{\pgfqpoint{2.115297in}{0.500000in}}%
\pgfpathclose%
\pgfusepath{fill}%
\end{pgfscope}%
\begin{pgfscope}%
\pgfpathrectangle{\pgfqpoint{0.750000in}{0.500000in}}{\pgfqpoint{4.650000in}{3.020000in}}%
\pgfusepath{clip}%
\pgfsetbuttcap%
\pgfsetmiterjoin%
\definecolor{currentfill}{rgb}{1.000000,0.000000,0.000000}%
\pgfsetfillcolor{currentfill}%
\pgfsetlinewidth{0.000000pt}%
\definecolor{currentstroke}{rgb}{0.000000,0.000000,0.000000}%
\pgfsetstrokecolor{currentstroke}%
\pgfsetstrokeopacity{0.000000}%
\pgfsetdash{}{0pt}%
\pgfpathmoveto{\pgfqpoint{2.147648in}{0.500000in}}%
\pgfpathlineto{\pgfqpoint{2.179999in}{0.500000in}}%
\pgfpathlineto{\pgfqpoint{2.179999in}{0.570222in}}%
\pgfpathlineto{\pgfqpoint{2.147648in}{0.570222in}}%
\pgfpathlineto{\pgfqpoint{2.147648in}{0.500000in}}%
\pgfpathclose%
\pgfusepath{fill}%
\end{pgfscope}%
\begin{pgfscope}%
\pgfpathrectangle{\pgfqpoint{0.750000in}{0.500000in}}{\pgfqpoint{4.650000in}{3.020000in}}%
\pgfusepath{clip}%
\pgfsetbuttcap%
\pgfsetmiterjoin%
\definecolor{currentfill}{rgb}{1.000000,0.000000,0.000000}%
\pgfsetfillcolor{currentfill}%
\pgfsetlinewidth{0.000000pt}%
\definecolor{currentstroke}{rgb}{0.000000,0.000000,0.000000}%
\pgfsetstrokecolor{currentstroke}%
\pgfsetstrokeopacity{0.000000}%
\pgfsetdash{}{0pt}%
\pgfpathmoveto{\pgfqpoint{2.179999in}{0.500000in}}%
\pgfpathlineto{\pgfqpoint{2.212350in}{0.500000in}}%
\pgfpathlineto{\pgfqpoint{2.212350in}{3.098227in}}%
\pgfpathlineto{\pgfqpoint{2.179999in}{3.098227in}}%
\pgfpathlineto{\pgfqpoint{2.179999in}{0.500000in}}%
\pgfpathclose%
\pgfusepath{fill}%
\end{pgfscope}%
\begin{pgfscope}%
\pgfpathrectangle{\pgfqpoint{0.750000in}{0.500000in}}{\pgfqpoint{4.650000in}{3.020000in}}%
\pgfusepath{clip}%
\pgfsetbuttcap%
\pgfsetmiterjoin%
\definecolor{currentfill}{rgb}{1.000000,0.000000,0.000000}%
\pgfsetfillcolor{currentfill}%
\pgfsetlinewidth{0.000000pt}%
\definecolor{currentstroke}{rgb}{0.000000,0.000000,0.000000}%
\pgfsetstrokecolor{currentstroke}%
\pgfsetstrokeopacity{0.000000}%
\pgfsetdash{}{0pt}%
\pgfpathmoveto{\pgfqpoint{2.212350in}{0.500000in}}%
\pgfpathlineto{\pgfqpoint{2.244701in}{0.500000in}}%
\pgfpathlineto{\pgfqpoint{2.244701in}{0.532185in}}%
\pgfpathlineto{\pgfqpoint{2.212350in}{0.532185in}}%
\pgfpathlineto{\pgfqpoint{2.212350in}{0.500000in}}%
\pgfpathclose%
\pgfusepath{fill}%
\end{pgfscope}%
\begin{pgfscope}%
\pgfpathrectangle{\pgfqpoint{0.750000in}{0.500000in}}{\pgfqpoint{4.650000in}{3.020000in}}%
\pgfusepath{clip}%
\pgfsetbuttcap%
\pgfsetmiterjoin%
\definecolor{currentfill}{rgb}{1.000000,0.000000,0.000000}%
\pgfsetfillcolor{currentfill}%
\pgfsetlinewidth{0.000000pt}%
\definecolor{currentstroke}{rgb}{0.000000,0.000000,0.000000}%
\pgfsetstrokecolor{currentstroke}%
\pgfsetstrokeopacity{0.000000}%
\pgfsetdash{}{0pt}%
\pgfpathmoveto{\pgfqpoint{2.244701in}{0.500000in}}%
\pgfpathlineto{\pgfqpoint{2.277052in}{0.500000in}}%
\pgfpathlineto{\pgfqpoint{2.277052in}{3.139190in}}%
\pgfpathlineto{\pgfqpoint{2.244701in}{3.139190in}}%
\pgfpathlineto{\pgfqpoint{2.244701in}{0.500000in}}%
\pgfpathclose%
\pgfusepath{fill}%
\end{pgfscope}%
\begin{pgfscope}%
\pgfpathrectangle{\pgfqpoint{0.750000in}{0.500000in}}{\pgfqpoint{4.650000in}{3.020000in}}%
\pgfusepath{clip}%
\pgfsetbuttcap%
\pgfsetmiterjoin%
\definecolor{currentfill}{rgb}{1.000000,0.000000,0.000000}%
\pgfsetfillcolor{currentfill}%
\pgfsetlinewidth{0.000000pt}%
\definecolor{currentstroke}{rgb}{0.000000,0.000000,0.000000}%
\pgfsetstrokecolor{currentstroke}%
\pgfsetstrokeopacity{0.000000}%
\pgfsetdash{}{0pt}%
\pgfpathmoveto{\pgfqpoint{2.277052in}{0.500000in}}%
\pgfpathlineto{\pgfqpoint{2.309403in}{0.500000in}}%
\pgfpathlineto{\pgfqpoint{2.309403in}{0.570222in}}%
\pgfpathlineto{\pgfqpoint{2.277052in}{0.570222in}}%
\pgfpathlineto{\pgfqpoint{2.277052in}{0.500000in}}%
\pgfpathclose%
\pgfusepath{fill}%
\end{pgfscope}%
\begin{pgfscope}%
\pgfpathrectangle{\pgfqpoint{0.750000in}{0.500000in}}{\pgfqpoint{4.650000in}{3.020000in}}%
\pgfusepath{clip}%
\pgfsetbuttcap%
\pgfsetmiterjoin%
\definecolor{currentfill}{rgb}{1.000000,0.000000,0.000000}%
\pgfsetfillcolor{currentfill}%
\pgfsetlinewidth{0.000000pt}%
\definecolor{currentstroke}{rgb}{0.000000,0.000000,0.000000}%
\pgfsetstrokecolor{currentstroke}%
\pgfsetstrokeopacity{0.000000}%
\pgfsetdash{}{0pt}%
\pgfpathmoveto{\pgfqpoint{2.309403in}{0.500000in}}%
\pgfpathlineto{\pgfqpoint{2.341753in}{0.500000in}}%
\pgfpathlineto{\pgfqpoint{2.341753in}{0.546815in}}%
\pgfpathlineto{\pgfqpoint{2.309403in}{0.546815in}}%
\pgfpathlineto{\pgfqpoint{2.309403in}{0.500000in}}%
\pgfpathclose%
\pgfusepath{fill}%
\end{pgfscope}%
\begin{pgfscope}%
\pgfpathrectangle{\pgfqpoint{0.750000in}{0.500000in}}{\pgfqpoint{4.650000in}{3.020000in}}%
\pgfusepath{clip}%
\pgfsetbuttcap%
\pgfsetmiterjoin%
\definecolor{currentfill}{rgb}{1.000000,0.000000,0.000000}%
\pgfsetfillcolor{currentfill}%
\pgfsetlinewidth{0.000000pt}%
\definecolor{currentstroke}{rgb}{0.000000,0.000000,0.000000}%
\pgfsetstrokecolor{currentstroke}%
\pgfsetstrokeopacity{0.000000}%
\pgfsetdash{}{0pt}%
\pgfpathmoveto{\pgfqpoint{2.341753in}{0.500000in}}%
\pgfpathlineto{\pgfqpoint{2.374104in}{0.500000in}}%
\pgfpathlineto{\pgfqpoint{2.374104in}{3.376190in}}%
\pgfpathlineto{\pgfqpoint{2.341753in}{3.376190in}}%
\pgfpathlineto{\pgfqpoint{2.341753in}{0.500000in}}%
\pgfpathclose%
\pgfusepath{fill}%
\end{pgfscope}%
\begin{pgfscope}%
\pgfpathrectangle{\pgfqpoint{0.750000in}{0.500000in}}{\pgfqpoint{4.650000in}{3.020000in}}%
\pgfusepath{clip}%
\pgfsetbuttcap%
\pgfsetmiterjoin%
\definecolor{currentfill}{rgb}{1.000000,0.000000,0.000000}%
\pgfsetfillcolor{currentfill}%
\pgfsetlinewidth{0.000000pt}%
\definecolor{currentstroke}{rgb}{0.000000,0.000000,0.000000}%
\pgfsetstrokecolor{currentstroke}%
\pgfsetstrokeopacity{0.000000}%
\pgfsetdash{}{0pt}%
\pgfpathmoveto{\pgfqpoint{2.374104in}{0.500000in}}%
\pgfpathlineto{\pgfqpoint{2.406455in}{0.500000in}}%
\pgfpathlineto{\pgfqpoint{2.406455in}{0.549741in}}%
\pgfpathlineto{\pgfqpoint{2.374104in}{0.549741in}}%
\pgfpathlineto{\pgfqpoint{2.374104in}{0.500000in}}%
\pgfpathclose%
\pgfusepath{fill}%
\end{pgfscope}%
\begin{pgfscope}%
\pgfpathrectangle{\pgfqpoint{0.750000in}{0.500000in}}{\pgfqpoint{4.650000in}{3.020000in}}%
\pgfusepath{clip}%
\pgfsetbuttcap%
\pgfsetmiterjoin%
\definecolor{currentfill}{rgb}{1.000000,0.000000,0.000000}%
\pgfsetfillcolor{currentfill}%
\pgfsetlinewidth{0.000000pt}%
\definecolor{currentstroke}{rgb}{0.000000,0.000000,0.000000}%
\pgfsetstrokecolor{currentstroke}%
\pgfsetstrokeopacity{0.000000}%
\pgfsetdash{}{0pt}%
\pgfpathmoveto{\pgfqpoint{2.406455in}{0.500000in}}%
\pgfpathlineto{\pgfqpoint{2.438806in}{0.500000in}}%
\pgfpathlineto{\pgfqpoint{2.438806in}{0.552667in}}%
\pgfpathlineto{\pgfqpoint{2.406455in}{0.552667in}}%
\pgfpathlineto{\pgfqpoint{2.406455in}{0.500000in}}%
\pgfpathclose%
\pgfusepath{fill}%
\end{pgfscope}%
\begin{pgfscope}%
\pgfpathrectangle{\pgfqpoint{0.750000in}{0.500000in}}{\pgfqpoint{4.650000in}{3.020000in}}%
\pgfusepath{clip}%
\pgfsetbuttcap%
\pgfsetmiterjoin%
\definecolor{currentfill}{rgb}{1.000000,0.000000,0.000000}%
\pgfsetfillcolor{currentfill}%
\pgfsetlinewidth{0.000000pt}%
\definecolor{currentstroke}{rgb}{0.000000,0.000000,0.000000}%
\pgfsetstrokecolor{currentstroke}%
\pgfsetstrokeopacity{0.000000}%
\pgfsetdash{}{0pt}%
\pgfpathmoveto{\pgfqpoint{2.438806in}{0.500000in}}%
\pgfpathlineto{\pgfqpoint{2.471157in}{0.500000in}}%
\pgfpathlineto{\pgfqpoint{2.471157in}{3.174301in}}%
\pgfpathlineto{\pgfqpoint{2.438806in}{3.174301in}}%
\pgfpathlineto{\pgfqpoint{2.438806in}{0.500000in}}%
\pgfpathclose%
\pgfusepath{fill}%
\end{pgfscope}%
\begin{pgfscope}%
\pgfpathrectangle{\pgfqpoint{0.750000in}{0.500000in}}{\pgfqpoint{4.650000in}{3.020000in}}%
\pgfusepath{clip}%
\pgfsetbuttcap%
\pgfsetmiterjoin%
\definecolor{currentfill}{rgb}{1.000000,0.000000,0.000000}%
\pgfsetfillcolor{currentfill}%
\pgfsetlinewidth{0.000000pt}%
\definecolor{currentstroke}{rgb}{0.000000,0.000000,0.000000}%
\pgfsetstrokecolor{currentstroke}%
\pgfsetstrokeopacity{0.000000}%
\pgfsetdash{}{0pt}%
\pgfpathmoveto{\pgfqpoint{2.471157in}{0.500000in}}%
\pgfpathlineto{\pgfqpoint{2.503508in}{0.500000in}}%
\pgfpathlineto{\pgfqpoint{2.503508in}{0.555593in}}%
\pgfpathlineto{\pgfqpoint{2.471157in}{0.555593in}}%
\pgfpathlineto{\pgfqpoint{2.471157in}{0.500000in}}%
\pgfpathclose%
\pgfusepath{fill}%
\end{pgfscope}%
\begin{pgfscope}%
\pgfpathrectangle{\pgfqpoint{0.750000in}{0.500000in}}{\pgfqpoint{4.650000in}{3.020000in}}%
\pgfusepath{clip}%
\pgfsetbuttcap%
\pgfsetmiterjoin%
\definecolor{currentfill}{rgb}{1.000000,0.000000,0.000000}%
\pgfsetfillcolor{currentfill}%
\pgfsetlinewidth{0.000000pt}%
\definecolor{currentstroke}{rgb}{0.000000,0.000000,0.000000}%
\pgfsetstrokecolor{currentstroke}%
\pgfsetstrokeopacity{0.000000}%
\pgfsetdash{}{0pt}%
\pgfpathmoveto{\pgfqpoint{2.503508in}{0.500000in}}%
\pgfpathlineto{\pgfqpoint{2.535859in}{0.500000in}}%
\pgfpathlineto{\pgfqpoint{2.535859in}{2.896338in}}%
\pgfpathlineto{\pgfqpoint{2.503508in}{2.896338in}}%
\pgfpathlineto{\pgfqpoint{2.503508in}{0.500000in}}%
\pgfpathclose%
\pgfusepath{fill}%
\end{pgfscope}%
\begin{pgfscope}%
\pgfpathrectangle{\pgfqpoint{0.750000in}{0.500000in}}{\pgfqpoint{4.650000in}{3.020000in}}%
\pgfusepath{clip}%
\pgfsetbuttcap%
\pgfsetmiterjoin%
\definecolor{currentfill}{rgb}{1.000000,0.000000,0.000000}%
\pgfsetfillcolor{currentfill}%
\pgfsetlinewidth{0.000000pt}%
\definecolor{currentstroke}{rgb}{0.000000,0.000000,0.000000}%
\pgfsetstrokecolor{currentstroke}%
\pgfsetstrokeopacity{0.000000}%
\pgfsetdash{}{0pt}%
\pgfpathmoveto{\pgfqpoint{2.535859in}{0.500000in}}%
\pgfpathlineto{\pgfqpoint{2.568210in}{0.500000in}}%
\pgfpathlineto{\pgfqpoint{2.568210in}{0.570222in}}%
\pgfpathlineto{\pgfqpoint{2.535859in}{0.570222in}}%
\pgfpathlineto{\pgfqpoint{2.535859in}{0.500000in}}%
\pgfpathclose%
\pgfusepath{fill}%
\end{pgfscope}%
\begin{pgfscope}%
\pgfpathrectangle{\pgfqpoint{0.750000in}{0.500000in}}{\pgfqpoint{4.650000in}{3.020000in}}%
\pgfusepath{clip}%
\pgfsetbuttcap%
\pgfsetmiterjoin%
\definecolor{currentfill}{rgb}{1.000000,0.000000,0.000000}%
\pgfsetfillcolor{currentfill}%
\pgfsetlinewidth{0.000000pt}%
\definecolor{currentstroke}{rgb}{0.000000,0.000000,0.000000}%
\pgfsetstrokecolor{currentstroke}%
\pgfsetstrokeopacity{0.000000}%
\pgfsetdash{}{0pt}%
\pgfpathmoveto{\pgfqpoint{2.568210in}{0.500000in}}%
\pgfpathlineto{\pgfqpoint{2.600561in}{0.500000in}}%
\pgfpathlineto{\pgfqpoint{2.600561in}{0.552667in}}%
\pgfpathlineto{\pgfqpoint{2.568210in}{0.552667in}}%
\pgfpathlineto{\pgfqpoint{2.568210in}{0.500000in}}%
\pgfpathclose%
\pgfusepath{fill}%
\end{pgfscope}%
\begin{pgfscope}%
\pgfpathrectangle{\pgfqpoint{0.750000in}{0.500000in}}{\pgfqpoint{4.650000in}{3.020000in}}%
\pgfusepath{clip}%
\pgfsetbuttcap%
\pgfsetmiterjoin%
\definecolor{currentfill}{rgb}{1.000000,0.000000,0.000000}%
\pgfsetfillcolor{currentfill}%
\pgfsetlinewidth{0.000000pt}%
\definecolor{currentstroke}{rgb}{0.000000,0.000000,0.000000}%
\pgfsetstrokecolor{currentstroke}%
\pgfsetstrokeopacity{0.000000}%
\pgfsetdash{}{0pt}%
\pgfpathmoveto{\pgfqpoint{2.600561in}{0.500000in}}%
\pgfpathlineto{\pgfqpoint{2.632912in}{0.500000in}}%
\pgfpathlineto{\pgfqpoint{2.632912in}{2.621300in}}%
\pgfpathlineto{\pgfqpoint{2.600561in}{2.621300in}}%
\pgfpathlineto{\pgfqpoint{2.600561in}{0.500000in}}%
\pgfpathclose%
\pgfusepath{fill}%
\end{pgfscope}%
\begin{pgfscope}%
\pgfpathrectangle{\pgfqpoint{0.750000in}{0.500000in}}{\pgfqpoint{4.650000in}{3.020000in}}%
\pgfusepath{clip}%
\pgfsetbuttcap%
\pgfsetmiterjoin%
\definecolor{currentfill}{rgb}{1.000000,0.000000,0.000000}%
\pgfsetfillcolor{currentfill}%
\pgfsetlinewidth{0.000000pt}%
\definecolor{currentstroke}{rgb}{0.000000,0.000000,0.000000}%
\pgfsetstrokecolor{currentstroke}%
\pgfsetstrokeopacity{0.000000}%
\pgfsetdash{}{0pt}%
\pgfpathmoveto{\pgfqpoint{2.632912in}{0.500000in}}%
\pgfpathlineto{\pgfqpoint{2.665263in}{0.500000in}}%
\pgfpathlineto{\pgfqpoint{2.665263in}{0.529259in}}%
\pgfpathlineto{\pgfqpoint{2.632912in}{0.529259in}}%
\pgfpathlineto{\pgfqpoint{2.632912in}{0.500000in}}%
\pgfpathclose%
\pgfusepath{fill}%
\end{pgfscope}%
\begin{pgfscope}%
\pgfpathrectangle{\pgfqpoint{0.750000in}{0.500000in}}{\pgfqpoint{4.650000in}{3.020000in}}%
\pgfusepath{clip}%
\pgfsetbuttcap%
\pgfsetmiterjoin%
\definecolor{currentfill}{rgb}{1.000000,0.000000,0.000000}%
\pgfsetfillcolor{currentfill}%
\pgfsetlinewidth{0.000000pt}%
\definecolor{currentstroke}{rgb}{0.000000,0.000000,0.000000}%
\pgfsetstrokecolor{currentstroke}%
\pgfsetstrokeopacity{0.000000}%
\pgfsetdash{}{0pt}%
\pgfpathmoveto{\pgfqpoint{2.665263in}{0.500000in}}%
\pgfpathlineto{\pgfqpoint{2.697614in}{0.500000in}}%
\pgfpathlineto{\pgfqpoint{2.697614in}{0.552667in}}%
\pgfpathlineto{\pgfqpoint{2.665263in}{0.552667in}}%
\pgfpathlineto{\pgfqpoint{2.665263in}{0.500000in}}%
\pgfpathclose%
\pgfusepath{fill}%
\end{pgfscope}%
\begin{pgfscope}%
\pgfpathrectangle{\pgfqpoint{0.750000in}{0.500000in}}{\pgfqpoint{4.650000in}{3.020000in}}%
\pgfusepath{clip}%
\pgfsetbuttcap%
\pgfsetmiterjoin%
\definecolor{currentfill}{rgb}{1.000000,0.000000,0.000000}%
\pgfsetfillcolor{currentfill}%
\pgfsetlinewidth{0.000000pt}%
\definecolor{currentstroke}{rgb}{0.000000,0.000000,0.000000}%
\pgfsetstrokecolor{currentstroke}%
\pgfsetstrokeopacity{0.000000}%
\pgfsetdash{}{0pt}%
\pgfpathmoveto{\pgfqpoint{2.697614in}{0.500000in}}%
\pgfpathlineto{\pgfqpoint{2.729965in}{0.500000in}}%
\pgfpathlineto{\pgfqpoint{2.729965in}{2.246781in}}%
\pgfpathlineto{\pgfqpoint{2.697614in}{2.246781in}}%
\pgfpathlineto{\pgfqpoint{2.697614in}{0.500000in}}%
\pgfpathclose%
\pgfusepath{fill}%
\end{pgfscope}%
\begin{pgfscope}%
\pgfpathrectangle{\pgfqpoint{0.750000in}{0.500000in}}{\pgfqpoint{4.650000in}{3.020000in}}%
\pgfusepath{clip}%
\pgfsetbuttcap%
\pgfsetmiterjoin%
\definecolor{currentfill}{rgb}{1.000000,0.000000,0.000000}%
\pgfsetfillcolor{currentfill}%
\pgfsetlinewidth{0.000000pt}%
\definecolor{currentstroke}{rgb}{0.000000,0.000000,0.000000}%
\pgfsetstrokecolor{currentstroke}%
\pgfsetstrokeopacity{0.000000}%
\pgfsetdash{}{0pt}%
\pgfpathmoveto{\pgfqpoint{2.729965in}{0.500000in}}%
\pgfpathlineto{\pgfqpoint{2.762316in}{0.500000in}}%
\pgfpathlineto{\pgfqpoint{2.762316in}{0.535111in}}%
\pgfpathlineto{\pgfqpoint{2.729965in}{0.535111in}}%
\pgfpathlineto{\pgfqpoint{2.729965in}{0.500000in}}%
\pgfpathclose%
\pgfusepath{fill}%
\end{pgfscope}%
\begin{pgfscope}%
\pgfpathrectangle{\pgfqpoint{0.750000in}{0.500000in}}{\pgfqpoint{4.650000in}{3.020000in}}%
\pgfusepath{clip}%
\pgfsetbuttcap%
\pgfsetmiterjoin%
\definecolor{currentfill}{rgb}{1.000000,0.000000,0.000000}%
\pgfsetfillcolor{currentfill}%
\pgfsetlinewidth{0.000000pt}%
\definecolor{currentstroke}{rgb}{0.000000,0.000000,0.000000}%
\pgfsetstrokecolor{currentstroke}%
\pgfsetstrokeopacity{0.000000}%
\pgfsetdash{}{0pt}%
\pgfpathmoveto{\pgfqpoint{2.762316in}{0.500000in}}%
\pgfpathlineto{\pgfqpoint{2.794667in}{0.500000in}}%
\pgfpathlineto{\pgfqpoint{2.794667in}{0.643371in}}%
\pgfpathlineto{\pgfqpoint{2.762316in}{0.643371in}}%
\pgfpathlineto{\pgfqpoint{2.762316in}{0.500000in}}%
\pgfpathclose%
\pgfusepath{fill}%
\end{pgfscope}%
\begin{pgfscope}%
\pgfpathrectangle{\pgfqpoint{0.750000in}{0.500000in}}{\pgfqpoint{4.650000in}{3.020000in}}%
\pgfusepath{clip}%
\pgfsetbuttcap%
\pgfsetmiterjoin%
\definecolor{currentfill}{rgb}{1.000000,0.000000,0.000000}%
\pgfsetfillcolor{currentfill}%
\pgfsetlinewidth{0.000000pt}%
\definecolor{currentstroke}{rgb}{0.000000,0.000000,0.000000}%
\pgfsetstrokecolor{currentstroke}%
\pgfsetstrokeopacity{0.000000}%
\pgfsetdash{}{0pt}%
\pgfpathmoveto{\pgfqpoint{2.794667in}{0.500000in}}%
\pgfpathlineto{\pgfqpoint{2.827018in}{0.500000in}}%
\pgfpathlineto{\pgfqpoint{2.827018in}{1.611854in}}%
\pgfpathlineto{\pgfqpoint{2.794667in}{1.611854in}}%
\pgfpathlineto{\pgfqpoint{2.794667in}{0.500000in}}%
\pgfpathclose%
\pgfusepath{fill}%
\end{pgfscope}%
\begin{pgfscope}%
\pgfpathrectangle{\pgfqpoint{0.750000in}{0.500000in}}{\pgfqpoint{4.650000in}{3.020000in}}%
\pgfusepath{clip}%
\pgfsetbuttcap%
\pgfsetmiterjoin%
\definecolor{currentfill}{rgb}{1.000000,0.000000,0.000000}%
\pgfsetfillcolor{currentfill}%
\pgfsetlinewidth{0.000000pt}%
\definecolor{currentstroke}{rgb}{0.000000,0.000000,0.000000}%
\pgfsetstrokecolor{currentstroke}%
\pgfsetstrokeopacity{0.000000}%
\pgfsetdash{}{0pt}%
\pgfpathmoveto{\pgfqpoint{2.827018in}{0.500000in}}%
\pgfpathlineto{\pgfqpoint{2.859369in}{0.500000in}}%
\pgfpathlineto{\pgfqpoint{2.859369in}{0.532185in}}%
\pgfpathlineto{\pgfqpoint{2.827018in}{0.532185in}}%
\pgfpathlineto{\pgfqpoint{2.827018in}{0.500000in}}%
\pgfpathclose%
\pgfusepath{fill}%
\end{pgfscope}%
\begin{pgfscope}%
\pgfpathrectangle{\pgfqpoint{0.750000in}{0.500000in}}{\pgfqpoint{4.650000in}{3.020000in}}%
\pgfusepath{clip}%
\pgfsetbuttcap%
\pgfsetmiterjoin%
\definecolor{currentfill}{rgb}{1.000000,0.000000,0.000000}%
\pgfsetfillcolor{currentfill}%
\pgfsetlinewidth{0.000000pt}%
\definecolor{currentstroke}{rgb}{0.000000,0.000000,0.000000}%
\pgfsetstrokecolor{currentstroke}%
\pgfsetstrokeopacity{0.000000}%
\pgfsetdash{}{0pt}%
\pgfpathmoveto{\pgfqpoint{2.859369in}{0.500000in}}%
\pgfpathlineto{\pgfqpoint{2.891719in}{0.500000in}}%
\pgfpathlineto{\pgfqpoint{2.891719in}{1.407039in}}%
\pgfpathlineto{\pgfqpoint{2.859369in}{1.407039in}}%
\pgfpathlineto{\pgfqpoint{2.859369in}{0.500000in}}%
\pgfpathclose%
\pgfusepath{fill}%
\end{pgfscope}%
\begin{pgfscope}%
\pgfpathrectangle{\pgfqpoint{0.750000in}{0.500000in}}{\pgfqpoint{4.650000in}{3.020000in}}%
\pgfusepath{clip}%
\pgfsetbuttcap%
\pgfsetmiterjoin%
\definecolor{currentfill}{rgb}{1.000000,0.000000,0.000000}%
\pgfsetfillcolor{currentfill}%
\pgfsetlinewidth{0.000000pt}%
\definecolor{currentstroke}{rgb}{0.000000,0.000000,0.000000}%
\pgfsetstrokecolor{currentstroke}%
\pgfsetstrokeopacity{0.000000}%
\pgfsetdash{}{0pt}%
\pgfpathmoveto{\pgfqpoint{2.891719in}{0.500000in}}%
\pgfpathlineto{\pgfqpoint{2.924070in}{0.500000in}}%
\pgfpathlineto{\pgfqpoint{2.924070in}{0.520482in}}%
\pgfpathlineto{\pgfqpoint{2.891719in}{0.520482in}}%
\pgfpathlineto{\pgfqpoint{2.891719in}{0.500000in}}%
\pgfpathclose%
\pgfusepath{fill}%
\end{pgfscope}%
\begin{pgfscope}%
\pgfpathrectangle{\pgfqpoint{0.750000in}{0.500000in}}{\pgfqpoint{4.650000in}{3.020000in}}%
\pgfusepath{clip}%
\pgfsetbuttcap%
\pgfsetmiterjoin%
\definecolor{currentfill}{rgb}{1.000000,0.000000,0.000000}%
\pgfsetfillcolor{currentfill}%
\pgfsetlinewidth{0.000000pt}%
\definecolor{currentstroke}{rgb}{0.000000,0.000000,0.000000}%
\pgfsetstrokecolor{currentstroke}%
\pgfsetstrokeopacity{0.000000}%
\pgfsetdash{}{0pt}%
\pgfpathmoveto{\pgfqpoint{2.924070in}{0.500000in}}%
\pgfpathlineto{\pgfqpoint{2.956421in}{0.500000in}}%
\pgfpathlineto{\pgfqpoint{2.956421in}{0.508778in}}%
\pgfpathlineto{\pgfqpoint{2.924070in}{0.508778in}}%
\pgfpathlineto{\pgfqpoint{2.924070in}{0.500000in}}%
\pgfpathclose%
\pgfusepath{fill}%
\end{pgfscope}%
\begin{pgfscope}%
\pgfpathrectangle{\pgfqpoint{0.750000in}{0.500000in}}{\pgfqpoint{4.650000in}{3.020000in}}%
\pgfusepath{clip}%
\pgfsetbuttcap%
\pgfsetmiterjoin%
\definecolor{currentfill}{rgb}{1.000000,0.000000,0.000000}%
\pgfsetfillcolor{currentfill}%
\pgfsetlinewidth{0.000000pt}%
\definecolor{currentstroke}{rgb}{0.000000,0.000000,0.000000}%
\pgfsetstrokecolor{currentstroke}%
\pgfsetstrokeopacity{0.000000}%
\pgfsetdash{}{0pt}%
\pgfpathmoveto{\pgfqpoint{2.956421in}{0.500000in}}%
\pgfpathlineto{\pgfqpoint{2.988772in}{0.500000in}}%
\pgfpathlineto{\pgfqpoint{2.988772in}{1.079334in}}%
\pgfpathlineto{\pgfqpoint{2.956421in}{1.079334in}}%
\pgfpathlineto{\pgfqpoint{2.956421in}{0.500000in}}%
\pgfpathclose%
\pgfusepath{fill}%
\end{pgfscope}%
\begin{pgfscope}%
\pgfpathrectangle{\pgfqpoint{0.750000in}{0.500000in}}{\pgfqpoint{4.650000in}{3.020000in}}%
\pgfusepath{clip}%
\pgfsetbuttcap%
\pgfsetmiterjoin%
\definecolor{currentfill}{rgb}{1.000000,0.000000,0.000000}%
\pgfsetfillcolor{currentfill}%
\pgfsetlinewidth{0.000000pt}%
\definecolor{currentstroke}{rgb}{0.000000,0.000000,0.000000}%
\pgfsetstrokecolor{currentstroke}%
\pgfsetstrokeopacity{0.000000}%
\pgfsetdash{}{0pt}%
\pgfpathmoveto{\pgfqpoint{2.988772in}{0.500000in}}%
\pgfpathlineto{\pgfqpoint{3.021123in}{0.500000in}}%
\pgfpathlineto{\pgfqpoint{3.021123in}{0.502926in}}%
\pgfpathlineto{\pgfqpoint{2.988772in}{0.502926in}}%
\pgfpathlineto{\pgfqpoint{2.988772in}{0.500000in}}%
\pgfpathclose%
\pgfusepath{fill}%
\end{pgfscope}%
\begin{pgfscope}%
\pgfpathrectangle{\pgfqpoint{0.750000in}{0.500000in}}{\pgfqpoint{4.650000in}{3.020000in}}%
\pgfusepath{clip}%
\pgfsetbuttcap%
\pgfsetmiterjoin%
\definecolor{currentfill}{rgb}{1.000000,0.000000,0.000000}%
\pgfsetfillcolor{currentfill}%
\pgfsetlinewidth{0.000000pt}%
\definecolor{currentstroke}{rgb}{0.000000,0.000000,0.000000}%
\pgfsetstrokecolor{currentstroke}%
\pgfsetstrokeopacity{0.000000}%
\pgfsetdash{}{0pt}%
\pgfpathmoveto{\pgfqpoint{3.021123in}{0.500000in}}%
\pgfpathlineto{\pgfqpoint{3.053474in}{0.500000in}}%
\pgfpathlineto{\pgfqpoint{3.053474in}{0.517556in}}%
\pgfpathlineto{\pgfqpoint{3.021123in}{0.517556in}}%
\pgfpathlineto{\pgfqpoint{3.021123in}{0.500000in}}%
\pgfpathclose%
\pgfusepath{fill}%
\end{pgfscope}%
\begin{pgfscope}%
\pgfpathrectangle{\pgfqpoint{0.750000in}{0.500000in}}{\pgfqpoint{4.650000in}{3.020000in}}%
\pgfusepath{clip}%
\pgfsetbuttcap%
\pgfsetmiterjoin%
\definecolor{currentfill}{rgb}{1.000000,0.000000,0.000000}%
\pgfsetfillcolor{currentfill}%
\pgfsetlinewidth{0.000000pt}%
\definecolor{currentstroke}{rgb}{0.000000,0.000000,0.000000}%
\pgfsetstrokecolor{currentstroke}%
\pgfsetstrokeopacity{0.000000}%
\pgfsetdash{}{0pt}%
\pgfpathmoveto{\pgfqpoint{3.053474in}{0.500000in}}%
\pgfpathlineto{\pgfqpoint{3.085825in}{0.500000in}}%
\pgfpathlineto{\pgfqpoint{3.085825in}{0.880371in}}%
\pgfpathlineto{\pgfqpoint{3.053474in}{0.880371in}}%
\pgfpathlineto{\pgfqpoint{3.053474in}{0.500000in}}%
\pgfpathclose%
\pgfusepath{fill}%
\end{pgfscope}%
\begin{pgfscope}%
\pgfpathrectangle{\pgfqpoint{0.750000in}{0.500000in}}{\pgfqpoint{4.650000in}{3.020000in}}%
\pgfusepath{clip}%
\pgfsetbuttcap%
\pgfsetmiterjoin%
\definecolor{currentfill}{rgb}{1.000000,0.000000,0.000000}%
\pgfsetfillcolor{currentfill}%
\pgfsetlinewidth{0.000000pt}%
\definecolor{currentstroke}{rgb}{0.000000,0.000000,0.000000}%
\pgfsetstrokecolor{currentstroke}%
\pgfsetstrokeopacity{0.000000}%
\pgfsetdash{}{0pt}%
\pgfpathmoveto{\pgfqpoint{3.085825in}{0.500000in}}%
\pgfpathlineto{\pgfqpoint{3.118176in}{0.500000in}}%
\pgfpathlineto{\pgfqpoint{3.118176in}{0.502926in}}%
\pgfpathlineto{\pgfqpoint{3.085825in}{0.502926in}}%
\pgfpathlineto{\pgfqpoint{3.085825in}{0.500000in}}%
\pgfpathclose%
\pgfusepath{fill}%
\end{pgfscope}%
\begin{pgfscope}%
\pgfpathrectangle{\pgfqpoint{0.750000in}{0.500000in}}{\pgfqpoint{4.650000in}{3.020000in}}%
\pgfusepath{clip}%
\pgfsetbuttcap%
\pgfsetmiterjoin%
\definecolor{currentfill}{rgb}{1.000000,0.000000,0.000000}%
\pgfsetfillcolor{currentfill}%
\pgfsetlinewidth{0.000000pt}%
\definecolor{currentstroke}{rgb}{0.000000,0.000000,0.000000}%
\pgfsetstrokecolor{currentstroke}%
\pgfsetstrokeopacity{0.000000}%
\pgfsetdash{}{0pt}%
\pgfpathmoveto{\pgfqpoint{3.118176in}{0.500000in}}%
\pgfpathlineto{\pgfqpoint{3.150527in}{0.500000in}}%
\pgfpathlineto{\pgfqpoint{3.150527in}{0.734075in}}%
\pgfpathlineto{\pgfqpoint{3.118176in}{0.734075in}}%
\pgfpathlineto{\pgfqpoint{3.118176in}{0.500000in}}%
\pgfpathclose%
\pgfusepath{fill}%
\end{pgfscope}%
\begin{pgfscope}%
\pgfpathrectangle{\pgfqpoint{0.750000in}{0.500000in}}{\pgfqpoint{4.650000in}{3.020000in}}%
\pgfusepath{clip}%
\pgfsetbuttcap%
\pgfsetmiterjoin%
\definecolor{currentfill}{rgb}{1.000000,0.000000,0.000000}%
\pgfsetfillcolor{currentfill}%
\pgfsetlinewidth{0.000000pt}%
\definecolor{currentstroke}{rgb}{0.000000,0.000000,0.000000}%
\pgfsetstrokecolor{currentstroke}%
\pgfsetstrokeopacity{0.000000}%
\pgfsetdash{}{0pt}%
\pgfpathmoveto{\pgfqpoint{3.150527in}{0.500000in}}%
\pgfpathlineto{\pgfqpoint{3.182878in}{0.500000in}}%
\pgfpathlineto{\pgfqpoint{3.182878in}{0.511704in}}%
\pgfpathlineto{\pgfqpoint{3.150527in}{0.511704in}}%
\pgfpathlineto{\pgfqpoint{3.150527in}{0.500000in}}%
\pgfpathclose%
\pgfusepath{fill}%
\end{pgfscope}%
\begin{pgfscope}%
\pgfpathrectangle{\pgfqpoint{0.750000in}{0.500000in}}{\pgfqpoint{4.650000in}{3.020000in}}%
\pgfusepath{clip}%
\pgfsetbuttcap%
\pgfsetmiterjoin%
\definecolor{currentfill}{rgb}{1.000000,0.000000,0.000000}%
\pgfsetfillcolor{currentfill}%
\pgfsetlinewidth{0.000000pt}%
\definecolor{currentstroke}{rgb}{0.000000,0.000000,0.000000}%
\pgfsetstrokecolor{currentstroke}%
\pgfsetstrokeopacity{0.000000}%
\pgfsetdash{}{0pt}%
\pgfpathmoveto{\pgfqpoint{3.182878in}{0.500000in}}%
\pgfpathlineto{\pgfqpoint{3.215229in}{0.500000in}}%
\pgfpathlineto{\pgfqpoint{3.215229in}{0.505852in}}%
\pgfpathlineto{\pgfqpoint{3.182878in}{0.505852in}}%
\pgfpathlineto{\pgfqpoint{3.182878in}{0.500000in}}%
\pgfpathclose%
\pgfusepath{fill}%
\end{pgfscope}%
\begin{pgfscope}%
\pgfpathrectangle{\pgfqpoint{0.750000in}{0.500000in}}{\pgfqpoint{4.650000in}{3.020000in}}%
\pgfusepath{clip}%
\pgfsetbuttcap%
\pgfsetmiterjoin%
\definecolor{currentfill}{rgb}{1.000000,0.000000,0.000000}%
\pgfsetfillcolor{currentfill}%
\pgfsetlinewidth{0.000000pt}%
\definecolor{currentstroke}{rgb}{0.000000,0.000000,0.000000}%
\pgfsetstrokecolor{currentstroke}%
\pgfsetstrokeopacity{0.000000}%
\pgfsetdash{}{0pt}%
\pgfpathmoveto{\pgfqpoint{3.215229in}{0.500000in}}%
\pgfpathlineto{\pgfqpoint{3.247580in}{0.500000in}}%
\pgfpathlineto{\pgfqpoint{3.247580in}{0.593630in}}%
\pgfpathlineto{\pgfqpoint{3.215229in}{0.593630in}}%
\pgfpathlineto{\pgfqpoint{3.215229in}{0.500000in}}%
\pgfpathclose%
\pgfusepath{fill}%
\end{pgfscope}%
\begin{pgfscope}%
\pgfpathrectangle{\pgfqpoint{0.750000in}{0.500000in}}{\pgfqpoint{4.650000in}{3.020000in}}%
\pgfusepath{clip}%
\pgfsetbuttcap%
\pgfsetmiterjoin%
\definecolor{currentfill}{rgb}{1.000000,0.000000,0.000000}%
\pgfsetfillcolor{currentfill}%
\pgfsetlinewidth{0.000000pt}%
\definecolor{currentstroke}{rgb}{0.000000,0.000000,0.000000}%
\pgfsetstrokecolor{currentstroke}%
\pgfsetstrokeopacity{0.000000}%
\pgfsetdash{}{0pt}%
\pgfpathmoveto{\pgfqpoint{3.247580in}{0.500000in}}%
\pgfpathlineto{\pgfqpoint{3.279931in}{0.500000in}}%
\pgfpathlineto{\pgfqpoint{3.279931in}{0.502926in}}%
\pgfpathlineto{\pgfqpoint{3.247580in}{0.502926in}}%
\pgfpathlineto{\pgfqpoint{3.247580in}{0.500000in}}%
\pgfpathclose%
\pgfusepath{fill}%
\end{pgfscope}%
\begin{pgfscope}%
\pgfpathrectangle{\pgfqpoint{0.750000in}{0.500000in}}{\pgfqpoint{4.650000in}{3.020000in}}%
\pgfusepath{clip}%
\pgfsetbuttcap%
\pgfsetmiterjoin%
\definecolor{currentfill}{rgb}{1.000000,0.000000,0.000000}%
\pgfsetfillcolor{currentfill}%
\pgfsetlinewidth{0.000000pt}%
\definecolor{currentstroke}{rgb}{0.000000,0.000000,0.000000}%
\pgfsetstrokecolor{currentstroke}%
\pgfsetstrokeopacity{0.000000}%
\pgfsetdash{}{0pt}%
\pgfpathmoveto{\pgfqpoint{3.279931in}{0.500000in}}%
\pgfpathlineto{\pgfqpoint{3.312282in}{0.500000in}}%
\pgfpathlineto{\pgfqpoint{3.312282in}{0.508778in}}%
\pgfpathlineto{\pgfqpoint{3.279931in}{0.508778in}}%
\pgfpathlineto{\pgfqpoint{3.279931in}{0.500000in}}%
\pgfpathclose%
\pgfusepath{fill}%
\end{pgfscope}%
\begin{pgfscope}%
\pgfpathrectangle{\pgfqpoint{0.750000in}{0.500000in}}{\pgfqpoint{4.650000in}{3.020000in}}%
\pgfusepath{clip}%
\pgfsetbuttcap%
\pgfsetmiterjoin%
\definecolor{currentfill}{rgb}{1.000000,0.000000,0.000000}%
\pgfsetfillcolor{currentfill}%
\pgfsetlinewidth{0.000000pt}%
\definecolor{currentstroke}{rgb}{0.000000,0.000000,0.000000}%
\pgfsetstrokecolor{currentstroke}%
\pgfsetstrokeopacity{0.000000}%
\pgfsetdash{}{0pt}%
\pgfpathmoveto{\pgfqpoint{3.312282in}{0.500000in}}%
\pgfpathlineto{\pgfqpoint{3.344633in}{0.500000in}}%
\pgfpathlineto{\pgfqpoint{3.344633in}{0.576074in}}%
\pgfpathlineto{\pgfqpoint{3.312282in}{0.576074in}}%
\pgfpathlineto{\pgfqpoint{3.312282in}{0.500000in}}%
\pgfpathclose%
\pgfusepath{fill}%
\end{pgfscope}%
\begin{pgfscope}%
\pgfpathrectangle{\pgfqpoint{0.750000in}{0.500000in}}{\pgfqpoint{4.650000in}{3.020000in}}%
\pgfusepath{clip}%
\pgfsetbuttcap%
\pgfsetmiterjoin%
\definecolor{currentfill}{rgb}{1.000000,0.000000,0.000000}%
\pgfsetfillcolor{currentfill}%
\pgfsetlinewidth{0.000000pt}%
\definecolor{currentstroke}{rgb}{0.000000,0.000000,0.000000}%
\pgfsetstrokecolor{currentstroke}%
\pgfsetstrokeopacity{0.000000}%
\pgfsetdash{}{0pt}%
\pgfpathmoveto{\pgfqpoint{3.344633in}{0.500000in}}%
\pgfpathlineto{\pgfqpoint{3.376984in}{0.500000in}}%
\pgfpathlineto{\pgfqpoint{3.376984in}{0.500000in}}%
\pgfpathlineto{\pgfqpoint{3.344633in}{0.500000in}}%
\pgfpathlineto{\pgfqpoint{3.344633in}{0.500000in}}%
\pgfpathclose%
\pgfusepath{fill}%
\end{pgfscope}%
\begin{pgfscope}%
\pgfpathrectangle{\pgfqpoint{0.750000in}{0.500000in}}{\pgfqpoint{4.650000in}{3.020000in}}%
\pgfusepath{clip}%
\pgfsetbuttcap%
\pgfsetmiterjoin%
\definecolor{currentfill}{rgb}{1.000000,0.000000,0.000000}%
\pgfsetfillcolor{currentfill}%
\pgfsetlinewidth{0.000000pt}%
\definecolor{currentstroke}{rgb}{0.000000,0.000000,0.000000}%
\pgfsetstrokecolor{currentstroke}%
\pgfsetstrokeopacity{0.000000}%
\pgfsetdash{}{0pt}%
\pgfpathmoveto{\pgfqpoint{3.376984in}{0.500000in}}%
\pgfpathlineto{\pgfqpoint{3.409335in}{0.500000in}}%
\pgfpathlineto{\pgfqpoint{3.409335in}{0.538037in}}%
\pgfpathlineto{\pgfqpoint{3.376984in}{0.538037in}}%
\pgfpathlineto{\pgfqpoint{3.376984in}{0.500000in}}%
\pgfpathclose%
\pgfusepath{fill}%
\end{pgfscope}%
\begin{pgfscope}%
\pgfpathrectangle{\pgfqpoint{0.750000in}{0.500000in}}{\pgfqpoint{4.650000in}{3.020000in}}%
\pgfusepath{clip}%
\pgfsetbuttcap%
\pgfsetmiterjoin%
\definecolor{currentfill}{rgb}{1.000000,0.000000,0.000000}%
\pgfsetfillcolor{currentfill}%
\pgfsetlinewidth{0.000000pt}%
\definecolor{currentstroke}{rgb}{0.000000,0.000000,0.000000}%
\pgfsetstrokecolor{currentstroke}%
\pgfsetstrokeopacity{0.000000}%
\pgfsetdash{}{0pt}%
\pgfpathmoveto{\pgfqpoint{3.409335in}{0.500000in}}%
\pgfpathlineto{\pgfqpoint{3.441685in}{0.500000in}}%
\pgfpathlineto{\pgfqpoint{3.441685in}{0.500000in}}%
\pgfpathlineto{\pgfqpoint{3.409335in}{0.500000in}}%
\pgfpathlineto{\pgfqpoint{3.409335in}{0.500000in}}%
\pgfpathclose%
\pgfusepath{fill}%
\end{pgfscope}%
\begin{pgfscope}%
\pgfpathrectangle{\pgfqpoint{0.750000in}{0.500000in}}{\pgfqpoint{4.650000in}{3.020000in}}%
\pgfusepath{clip}%
\pgfsetbuttcap%
\pgfsetmiterjoin%
\definecolor{currentfill}{rgb}{1.000000,0.000000,0.000000}%
\pgfsetfillcolor{currentfill}%
\pgfsetlinewidth{0.000000pt}%
\definecolor{currentstroke}{rgb}{0.000000,0.000000,0.000000}%
\pgfsetstrokecolor{currentstroke}%
\pgfsetstrokeopacity{0.000000}%
\pgfsetdash{}{0pt}%
\pgfpathmoveto{\pgfqpoint{3.441685in}{0.500000in}}%
\pgfpathlineto{\pgfqpoint{3.474036in}{0.500000in}}%
\pgfpathlineto{\pgfqpoint{3.474036in}{0.500000in}}%
\pgfpathlineto{\pgfqpoint{3.441685in}{0.500000in}}%
\pgfpathlineto{\pgfqpoint{3.441685in}{0.500000in}}%
\pgfpathclose%
\pgfusepath{fill}%
\end{pgfscope}%
\begin{pgfscope}%
\pgfpathrectangle{\pgfqpoint{0.750000in}{0.500000in}}{\pgfqpoint{4.650000in}{3.020000in}}%
\pgfusepath{clip}%
\pgfsetbuttcap%
\pgfsetmiterjoin%
\definecolor{currentfill}{rgb}{1.000000,0.000000,0.000000}%
\pgfsetfillcolor{currentfill}%
\pgfsetlinewidth{0.000000pt}%
\definecolor{currentstroke}{rgb}{0.000000,0.000000,0.000000}%
\pgfsetstrokecolor{currentstroke}%
\pgfsetstrokeopacity{0.000000}%
\pgfsetdash{}{0pt}%
\pgfpathmoveto{\pgfqpoint{3.474036in}{0.500000in}}%
\pgfpathlineto{\pgfqpoint{3.506387in}{0.500000in}}%
\pgfpathlineto{\pgfqpoint{3.506387in}{0.505852in}}%
\pgfpathlineto{\pgfqpoint{3.474036in}{0.505852in}}%
\pgfpathlineto{\pgfqpoint{3.474036in}{0.500000in}}%
\pgfpathclose%
\pgfusepath{fill}%
\end{pgfscope}%
\begin{pgfscope}%
\pgfpathrectangle{\pgfqpoint{0.750000in}{0.500000in}}{\pgfqpoint{4.650000in}{3.020000in}}%
\pgfusepath{clip}%
\pgfsetbuttcap%
\pgfsetmiterjoin%
\definecolor{currentfill}{rgb}{1.000000,0.000000,0.000000}%
\pgfsetfillcolor{currentfill}%
\pgfsetlinewidth{0.000000pt}%
\definecolor{currentstroke}{rgb}{0.000000,0.000000,0.000000}%
\pgfsetstrokecolor{currentstroke}%
\pgfsetstrokeopacity{0.000000}%
\pgfsetdash{}{0pt}%
\pgfpathmoveto{\pgfqpoint{3.506387in}{0.500000in}}%
\pgfpathlineto{\pgfqpoint{3.538738in}{0.500000in}}%
\pgfpathlineto{\pgfqpoint{3.538738in}{0.500000in}}%
\pgfpathlineto{\pgfqpoint{3.506387in}{0.500000in}}%
\pgfpathlineto{\pgfqpoint{3.506387in}{0.500000in}}%
\pgfpathclose%
\pgfusepath{fill}%
\end{pgfscope}%
\begin{pgfscope}%
\pgfpathrectangle{\pgfqpoint{0.750000in}{0.500000in}}{\pgfqpoint{4.650000in}{3.020000in}}%
\pgfusepath{clip}%
\pgfsetbuttcap%
\pgfsetmiterjoin%
\definecolor{currentfill}{rgb}{1.000000,0.000000,0.000000}%
\pgfsetfillcolor{currentfill}%
\pgfsetlinewidth{0.000000pt}%
\definecolor{currentstroke}{rgb}{0.000000,0.000000,0.000000}%
\pgfsetstrokecolor{currentstroke}%
\pgfsetstrokeopacity{0.000000}%
\pgfsetdash{}{0pt}%
\pgfpathmoveto{\pgfqpoint{3.538738in}{0.500000in}}%
\pgfpathlineto{\pgfqpoint{3.571089in}{0.500000in}}%
\pgfpathlineto{\pgfqpoint{3.571089in}{0.502926in}}%
\pgfpathlineto{\pgfqpoint{3.538738in}{0.502926in}}%
\pgfpathlineto{\pgfqpoint{3.538738in}{0.500000in}}%
\pgfpathclose%
\pgfusepath{fill}%
\end{pgfscope}%
\begin{pgfscope}%
\pgfpathrectangle{\pgfqpoint{0.750000in}{0.500000in}}{\pgfqpoint{4.650000in}{3.020000in}}%
\pgfusepath{clip}%
\pgfsetbuttcap%
\pgfsetmiterjoin%
\definecolor{currentfill}{rgb}{1.000000,0.000000,0.000000}%
\pgfsetfillcolor{currentfill}%
\pgfsetlinewidth{0.000000pt}%
\definecolor{currentstroke}{rgb}{0.000000,0.000000,0.000000}%
\pgfsetstrokecolor{currentstroke}%
\pgfsetstrokeopacity{0.000000}%
\pgfsetdash{}{0pt}%
\pgfpathmoveto{\pgfqpoint{3.571089in}{0.500000in}}%
\pgfpathlineto{\pgfqpoint{3.603440in}{0.500000in}}%
\pgfpathlineto{\pgfqpoint{3.603440in}{0.505852in}}%
\pgfpathlineto{\pgfqpoint{3.571089in}{0.505852in}}%
\pgfpathlineto{\pgfqpoint{3.571089in}{0.500000in}}%
\pgfpathclose%
\pgfusepath{fill}%
\end{pgfscope}%
\begin{pgfscope}%
\pgfpathrectangle{\pgfqpoint{0.750000in}{0.500000in}}{\pgfqpoint{4.650000in}{3.020000in}}%
\pgfusepath{clip}%
\pgfsetbuttcap%
\pgfsetmiterjoin%
\definecolor{currentfill}{rgb}{1.000000,0.000000,0.000000}%
\pgfsetfillcolor{currentfill}%
\pgfsetlinewidth{0.000000pt}%
\definecolor{currentstroke}{rgb}{0.000000,0.000000,0.000000}%
\pgfsetstrokecolor{currentstroke}%
\pgfsetstrokeopacity{0.000000}%
\pgfsetdash{}{0pt}%
\pgfpathmoveto{\pgfqpoint{3.603440in}{0.500000in}}%
\pgfpathlineto{\pgfqpoint{3.635791in}{0.500000in}}%
\pgfpathlineto{\pgfqpoint{3.635791in}{0.500000in}}%
\pgfpathlineto{\pgfqpoint{3.603440in}{0.500000in}}%
\pgfpathlineto{\pgfqpoint{3.603440in}{0.500000in}}%
\pgfpathclose%
\pgfusepath{fill}%
\end{pgfscope}%
\begin{pgfscope}%
\pgfpathrectangle{\pgfqpoint{0.750000in}{0.500000in}}{\pgfqpoint{4.650000in}{3.020000in}}%
\pgfusepath{clip}%
\pgfsetbuttcap%
\pgfsetmiterjoin%
\definecolor{currentfill}{rgb}{1.000000,0.000000,0.000000}%
\pgfsetfillcolor{currentfill}%
\pgfsetlinewidth{0.000000pt}%
\definecolor{currentstroke}{rgb}{0.000000,0.000000,0.000000}%
\pgfsetstrokecolor{currentstroke}%
\pgfsetstrokeopacity{0.000000}%
\pgfsetdash{}{0pt}%
\pgfpathmoveto{\pgfqpoint{3.635791in}{0.500000in}}%
\pgfpathlineto{\pgfqpoint{3.668142in}{0.500000in}}%
\pgfpathlineto{\pgfqpoint{3.668142in}{0.500000in}}%
\pgfpathlineto{\pgfqpoint{3.635791in}{0.500000in}}%
\pgfpathlineto{\pgfqpoint{3.635791in}{0.500000in}}%
\pgfpathclose%
\pgfusepath{fill}%
\end{pgfscope}%
\begin{pgfscope}%
\pgfpathrectangle{\pgfqpoint{0.750000in}{0.500000in}}{\pgfqpoint{4.650000in}{3.020000in}}%
\pgfusepath{clip}%
\pgfsetbuttcap%
\pgfsetmiterjoin%
\definecolor{currentfill}{rgb}{1.000000,0.000000,0.000000}%
\pgfsetfillcolor{currentfill}%
\pgfsetlinewidth{0.000000pt}%
\definecolor{currentstroke}{rgb}{0.000000,0.000000,0.000000}%
\pgfsetstrokecolor{currentstroke}%
\pgfsetstrokeopacity{0.000000}%
\pgfsetdash{}{0pt}%
\pgfpathmoveto{\pgfqpoint{3.668142in}{0.500000in}}%
\pgfpathlineto{\pgfqpoint{3.700493in}{0.500000in}}%
\pgfpathlineto{\pgfqpoint{3.700493in}{0.502926in}}%
\pgfpathlineto{\pgfqpoint{3.668142in}{0.502926in}}%
\pgfpathlineto{\pgfqpoint{3.668142in}{0.500000in}}%
\pgfpathclose%
\pgfusepath{fill}%
\end{pgfscope}%
\begin{pgfscope}%
\pgfpathrectangle{\pgfqpoint{0.750000in}{0.500000in}}{\pgfqpoint{4.650000in}{3.020000in}}%
\pgfusepath{clip}%
\pgfsetbuttcap%
\pgfsetmiterjoin%
\definecolor{currentfill}{rgb}{1.000000,0.000000,0.000000}%
\pgfsetfillcolor{currentfill}%
\pgfsetlinewidth{0.000000pt}%
\definecolor{currentstroke}{rgb}{0.000000,0.000000,0.000000}%
\pgfsetstrokecolor{currentstroke}%
\pgfsetstrokeopacity{0.000000}%
\pgfsetdash{}{0pt}%
\pgfpathmoveto{\pgfqpoint{3.700493in}{0.500000in}}%
\pgfpathlineto{\pgfqpoint{3.732844in}{0.500000in}}%
\pgfpathlineto{\pgfqpoint{3.732844in}{0.500000in}}%
\pgfpathlineto{\pgfqpoint{3.700493in}{0.500000in}}%
\pgfpathlineto{\pgfqpoint{3.700493in}{0.500000in}}%
\pgfpathclose%
\pgfusepath{fill}%
\end{pgfscope}%
\begin{pgfscope}%
\pgfpathrectangle{\pgfqpoint{0.750000in}{0.500000in}}{\pgfqpoint{4.650000in}{3.020000in}}%
\pgfusepath{clip}%
\pgfsetbuttcap%
\pgfsetmiterjoin%
\definecolor{currentfill}{rgb}{1.000000,0.000000,0.000000}%
\pgfsetfillcolor{currentfill}%
\pgfsetlinewidth{0.000000pt}%
\definecolor{currentstroke}{rgb}{0.000000,0.000000,0.000000}%
\pgfsetstrokecolor{currentstroke}%
\pgfsetstrokeopacity{0.000000}%
\pgfsetdash{}{0pt}%
\pgfpathmoveto{\pgfqpoint{3.732844in}{0.500000in}}%
\pgfpathlineto{\pgfqpoint{3.765195in}{0.500000in}}%
\pgfpathlineto{\pgfqpoint{3.765195in}{0.502926in}}%
\pgfpathlineto{\pgfqpoint{3.732844in}{0.502926in}}%
\pgfpathlineto{\pgfqpoint{3.732844in}{0.500000in}}%
\pgfpathclose%
\pgfusepath{fill}%
\end{pgfscope}%
\begin{pgfscope}%
\pgfpathrectangle{\pgfqpoint{0.750000in}{0.500000in}}{\pgfqpoint{4.650000in}{3.020000in}}%
\pgfusepath{clip}%
\pgfsetbuttcap%
\pgfsetmiterjoin%
\definecolor{currentfill}{rgb}{1.000000,0.000000,0.000000}%
\pgfsetfillcolor{currentfill}%
\pgfsetlinewidth{0.000000pt}%
\definecolor{currentstroke}{rgb}{0.000000,0.000000,0.000000}%
\pgfsetstrokecolor{currentstroke}%
\pgfsetstrokeopacity{0.000000}%
\pgfsetdash{}{0pt}%
\pgfpathmoveto{\pgfqpoint{3.765195in}{0.500000in}}%
\pgfpathlineto{\pgfqpoint{3.797546in}{0.500000in}}%
\pgfpathlineto{\pgfqpoint{3.797546in}{0.500000in}}%
\pgfpathlineto{\pgfqpoint{3.765195in}{0.500000in}}%
\pgfpathlineto{\pgfqpoint{3.765195in}{0.500000in}}%
\pgfpathclose%
\pgfusepath{fill}%
\end{pgfscope}%
\begin{pgfscope}%
\pgfpathrectangle{\pgfqpoint{0.750000in}{0.500000in}}{\pgfqpoint{4.650000in}{3.020000in}}%
\pgfusepath{clip}%
\pgfsetbuttcap%
\pgfsetmiterjoin%
\definecolor{currentfill}{rgb}{1.000000,0.000000,0.000000}%
\pgfsetfillcolor{currentfill}%
\pgfsetlinewidth{0.000000pt}%
\definecolor{currentstroke}{rgb}{0.000000,0.000000,0.000000}%
\pgfsetstrokecolor{currentstroke}%
\pgfsetstrokeopacity{0.000000}%
\pgfsetdash{}{0pt}%
\pgfpathmoveto{\pgfqpoint{3.797546in}{0.500000in}}%
\pgfpathlineto{\pgfqpoint{3.829897in}{0.500000in}}%
\pgfpathlineto{\pgfqpoint{3.829897in}{0.500000in}}%
\pgfpathlineto{\pgfqpoint{3.797546in}{0.500000in}}%
\pgfpathlineto{\pgfqpoint{3.797546in}{0.500000in}}%
\pgfpathclose%
\pgfusepath{fill}%
\end{pgfscope}%
\begin{pgfscope}%
\pgfpathrectangle{\pgfqpoint{0.750000in}{0.500000in}}{\pgfqpoint{4.650000in}{3.020000in}}%
\pgfusepath{clip}%
\pgfsetbuttcap%
\pgfsetmiterjoin%
\definecolor{currentfill}{rgb}{1.000000,0.000000,0.000000}%
\pgfsetfillcolor{currentfill}%
\pgfsetlinewidth{0.000000pt}%
\definecolor{currentstroke}{rgb}{0.000000,0.000000,0.000000}%
\pgfsetstrokecolor{currentstroke}%
\pgfsetstrokeopacity{0.000000}%
\pgfsetdash{}{0pt}%
\pgfpathmoveto{\pgfqpoint{3.829897in}{0.500000in}}%
\pgfpathlineto{\pgfqpoint{3.862248in}{0.500000in}}%
\pgfpathlineto{\pgfqpoint{3.862248in}{0.500000in}}%
\pgfpathlineto{\pgfqpoint{3.829897in}{0.500000in}}%
\pgfpathlineto{\pgfqpoint{3.829897in}{0.500000in}}%
\pgfpathclose%
\pgfusepath{fill}%
\end{pgfscope}%
\begin{pgfscope}%
\pgfpathrectangle{\pgfqpoint{0.750000in}{0.500000in}}{\pgfqpoint{4.650000in}{3.020000in}}%
\pgfusepath{clip}%
\pgfsetbuttcap%
\pgfsetmiterjoin%
\definecolor{currentfill}{rgb}{1.000000,0.000000,0.000000}%
\pgfsetfillcolor{currentfill}%
\pgfsetlinewidth{0.000000pt}%
\definecolor{currentstroke}{rgb}{0.000000,0.000000,0.000000}%
\pgfsetstrokecolor{currentstroke}%
\pgfsetstrokeopacity{0.000000}%
\pgfsetdash{}{0pt}%
\pgfpathmoveto{\pgfqpoint{3.862248in}{0.500000in}}%
\pgfpathlineto{\pgfqpoint{3.894599in}{0.500000in}}%
\pgfpathlineto{\pgfqpoint{3.894599in}{0.500000in}}%
\pgfpathlineto{\pgfqpoint{3.862248in}{0.500000in}}%
\pgfpathlineto{\pgfqpoint{3.862248in}{0.500000in}}%
\pgfpathclose%
\pgfusepath{fill}%
\end{pgfscope}%
\begin{pgfscope}%
\pgfpathrectangle{\pgfqpoint{0.750000in}{0.500000in}}{\pgfqpoint{4.650000in}{3.020000in}}%
\pgfusepath{clip}%
\pgfsetbuttcap%
\pgfsetmiterjoin%
\definecolor{currentfill}{rgb}{1.000000,0.000000,0.000000}%
\pgfsetfillcolor{currentfill}%
\pgfsetlinewidth{0.000000pt}%
\definecolor{currentstroke}{rgb}{0.000000,0.000000,0.000000}%
\pgfsetstrokecolor{currentstroke}%
\pgfsetstrokeopacity{0.000000}%
\pgfsetdash{}{0pt}%
\pgfpathmoveto{\pgfqpoint{3.894599in}{0.500000in}}%
\pgfpathlineto{\pgfqpoint{3.926950in}{0.500000in}}%
\pgfpathlineto{\pgfqpoint{3.926950in}{0.500000in}}%
\pgfpathlineto{\pgfqpoint{3.894599in}{0.500000in}}%
\pgfpathlineto{\pgfqpoint{3.894599in}{0.500000in}}%
\pgfpathclose%
\pgfusepath{fill}%
\end{pgfscope}%
\begin{pgfscope}%
\pgfpathrectangle{\pgfqpoint{0.750000in}{0.500000in}}{\pgfqpoint{4.650000in}{3.020000in}}%
\pgfusepath{clip}%
\pgfsetbuttcap%
\pgfsetmiterjoin%
\definecolor{currentfill}{rgb}{1.000000,0.000000,0.000000}%
\pgfsetfillcolor{currentfill}%
\pgfsetlinewidth{0.000000pt}%
\definecolor{currentstroke}{rgb}{0.000000,0.000000,0.000000}%
\pgfsetstrokecolor{currentstroke}%
\pgfsetstrokeopacity{0.000000}%
\pgfsetdash{}{0pt}%
\pgfpathmoveto{\pgfqpoint{3.926950in}{0.500000in}}%
\pgfpathlineto{\pgfqpoint{3.959301in}{0.500000in}}%
\pgfpathlineto{\pgfqpoint{3.959301in}{0.500000in}}%
\pgfpathlineto{\pgfqpoint{3.926950in}{0.500000in}}%
\pgfpathlineto{\pgfqpoint{3.926950in}{0.500000in}}%
\pgfpathclose%
\pgfusepath{fill}%
\end{pgfscope}%
\begin{pgfscope}%
\pgfpathrectangle{\pgfqpoint{0.750000in}{0.500000in}}{\pgfqpoint{4.650000in}{3.020000in}}%
\pgfusepath{clip}%
\pgfsetbuttcap%
\pgfsetmiterjoin%
\definecolor{currentfill}{rgb}{1.000000,0.000000,0.000000}%
\pgfsetfillcolor{currentfill}%
\pgfsetlinewidth{0.000000pt}%
\definecolor{currentstroke}{rgb}{0.000000,0.000000,0.000000}%
\pgfsetstrokecolor{currentstroke}%
\pgfsetstrokeopacity{0.000000}%
\pgfsetdash{}{0pt}%
\pgfpathmoveto{\pgfqpoint{3.959301in}{0.500000in}}%
\pgfpathlineto{\pgfqpoint{3.991652in}{0.500000in}}%
\pgfpathlineto{\pgfqpoint{3.991652in}{0.500000in}}%
\pgfpathlineto{\pgfqpoint{3.959301in}{0.500000in}}%
\pgfpathlineto{\pgfqpoint{3.959301in}{0.500000in}}%
\pgfpathclose%
\pgfusepath{fill}%
\end{pgfscope}%
\begin{pgfscope}%
\pgfpathrectangle{\pgfqpoint{0.750000in}{0.500000in}}{\pgfqpoint{4.650000in}{3.020000in}}%
\pgfusepath{clip}%
\pgfsetbuttcap%
\pgfsetmiterjoin%
\definecolor{currentfill}{rgb}{1.000000,0.000000,0.000000}%
\pgfsetfillcolor{currentfill}%
\pgfsetlinewidth{0.000000pt}%
\definecolor{currentstroke}{rgb}{0.000000,0.000000,0.000000}%
\pgfsetstrokecolor{currentstroke}%
\pgfsetstrokeopacity{0.000000}%
\pgfsetdash{}{0pt}%
\pgfpathmoveto{\pgfqpoint{3.991652in}{0.500000in}}%
\pgfpathlineto{\pgfqpoint{4.024002in}{0.500000in}}%
\pgfpathlineto{\pgfqpoint{4.024002in}{0.500000in}}%
\pgfpathlineto{\pgfqpoint{3.991652in}{0.500000in}}%
\pgfpathlineto{\pgfqpoint{3.991652in}{0.500000in}}%
\pgfpathclose%
\pgfusepath{fill}%
\end{pgfscope}%
\begin{pgfscope}%
\pgfpathrectangle{\pgfqpoint{0.750000in}{0.500000in}}{\pgfqpoint{4.650000in}{3.020000in}}%
\pgfusepath{clip}%
\pgfsetbuttcap%
\pgfsetmiterjoin%
\definecolor{currentfill}{rgb}{1.000000,0.000000,0.000000}%
\pgfsetfillcolor{currentfill}%
\pgfsetlinewidth{0.000000pt}%
\definecolor{currentstroke}{rgb}{0.000000,0.000000,0.000000}%
\pgfsetstrokecolor{currentstroke}%
\pgfsetstrokeopacity{0.000000}%
\pgfsetdash{}{0pt}%
\pgfpathmoveto{\pgfqpoint{4.024002in}{0.500000in}}%
\pgfpathlineto{\pgfqpoint{4.056353in}{0.500000in}}%
\pgfpathlineto{\pgfqpoint{4.056353in}{0.500000in}}%
\pgfpathlineto{\pgfqpoint{4.024002in}{0.500000in}}%
\pgfpathlineto{\pgfqpoint{4.024002in}{0.500000in}}%
\pgfpathclose%
\pgfusepath{fill}%
\end{pgfscope}%
\begin{pgfscope}%
\pgfpathrectangle{\pgfqpoint{0.750000in}{0.500000in}}{\pgfqpoint{4.650000in}{3.020000in}}%
\pgfusepath{clip}%
\pgfsetbuttcap%
\pgfsetmiterjoin%
\definecolor{currentfill}{rgb}{1.000000,0.000000,0.000000}%
\pgfsetfillcolor{currentfill}%
\pgfsetlinewidth{0.000000pt}%
\definecolor{currentstroke}{rgb}{0.000000,0.000000,0.000000}%
\pgfsetstrokecolor{currentstroke}%
\pgfsetstrokeopacity{0.000000}%
\pgfsetdash{}{0pt}%
\pgfpathmoveto{\pgfqpoint{4.056353in}{0.500000in}}%
\pgfpathlineto{\pgfqpoint{4.088704in}{0.500000in}}%
\pgfpathlineto{\pgfqpoint{4.088704in}{0.500000in}}%
\pgfpathlineto{\pgfqpoint{4.056353in}{0.500000in}}%
\pgfpathlineto{\pgfqpoint{4.056353in}{0.500000in}}%
\pgfpathclose%
\pgfusepath{fill}%
\end{pgfscope}%
\begin{pgfscope}%
\pgfpathrectangle{\pgfqpoint{0.750000in}{0.500000in}}{\pgfqpoint{4.650000in}{3.020000in}}%
\pgfusepath{clip}%
\pgfsetbuttcap%
\pgfsetmiterjoin%
\definecolor{currentfill}{rgb}{1.000000,0.000000,0.000000}%
\pgfsetfillcolor{currentfill}%
\pgfsetlinewidth{0.000000pt}%
\definecolor{currentstroke}{rgb}{0.000000,0.000000,0.000000}%
\pgfsetstrokecolor{currentstroke}%
\pgfsetstrokeopacity{0.000000}%
\pgfsetdash{}{0pt}%
\pgfpathmoveto{\pgfqpoint{4.088704in}{0.500000in}}%
\pgfpathlineto{\pgfqpoint{4.121055in}{0.500000in}}%
\pgfpathlineto{\pgfqpoint{4.121055in}{0.500000in}}%
\pgfpathlineto{\pgfqpoint{4.088704in}{0.500000in}}%
\pgfpathlineto{\pgfqpoint{4.088704in}{0.500000in}}%
\pgfpathclose%
\pgfusepath{fill}%
\end{pgfscope}%
\begin{pgfscope}%
\pgfpathrectangle{\pgfqpoint{0.750000in}{0.500000in}}{\pgfqpoint{4.650000in}{3.020000in}}%
\pgfusepath{clip}%
\pgfsetbuttcap%
\pgfsetmiterjoin%
\definecolor{currentfill}{rgb}{1.000000,0.000000,0.000000}%
\pgfsetfillcolor{currentfill}%
\pgfsetlinewidth{0.000000pt}%
\definecolor{currentstroke}{rgb}{0.000000,0.000000,0.000000}%
\pgfsetstrokecolor{currentstroke}%
\pgfsetstrokeopacity{0.000000}%
\pgfsetdash{}{0pt}%
\pgfpathmoveto{\pgfqpoint{4.121055in}{0.500000in}}%
\pgfpathlineto{\pgfqpoint{4.153406in}{0.500000in}}%
\pgfpathlineto{\pgfqpoint{4.153406in}{0.500000in}}%
\pgfpathlineto{\pgfqpoint{4.121055in}{0.500000in}}%
\pgfpathlineto{\pgfqpoint{4.121055in}{0.500000in}}%
\pgfpathclose%
\pgfusepath{fill}%
\end{pgfscope}%
\begin{pgfscope}%
\pgfpathrectangle{\pgfqpoint{0.750000in}{0.500000in}}{\pgfqpoint{4.650000in}{3.020000in}}%
\pgfusepath{clip}%
\pgfsetbuttcap%
\pgfsetmiterjoin%
\definecolor{currentfill}{rgb}{1.000000,0.000000,0.000000}%
\pgfsetfillcolor{currentfill}%
\pgfsetlinewidth{0.000000pt}%
\definecolor{currentstroke}{rgb}{0.000000,0.000000,0.000000}%
\pgfsetstrokecolor{currentstroke}%
\pgfsetstrokeopacity{0.000000}%
\pgfsetdash{}{0pt}%
\pgfpathmoveto{\pgfqpoint{4.153406in}{0.500000in}}%
\pgfpathlineto{\pgfqpoint{4.185757in}{0.500000in}}%
\pgfpathlineto{\pgfqpoint{4.185757in}{0.500000in}}%
\pgfpathlineto{\pgfqpoint{4.153406in}{0.500000in}}%
\pgfpathlineto{\pgfqpoint{4.153406in}{0.500000in}}%
\pgfpathclose%
\pgfusepath{fill}%
\end{pgfscope}%
\begin{pgfscope}%
\pgfpathrectangle{\pgfqpoint{0.750000in}{0.500000in}}{\pgfqpoint{4.650000in}{3.020000in}}%
\pgfusepath{clip}%
\pgfsetbuttcap%
\pgfsetmiterjoin%
\definecolor{currentfill}{rgb}{1.000000,0.000000,0.000000}%
\pgfsetfillcolor{currentfill}%
\pgfsetlinewidth{0.000000pt}%
\definecolor{currentstroke}{rgb}{0.000000,0.000000,0.000000}%
\pgfsetstrokecolor{currentstroke}%
\pgfsetstrokeopacity{0.000000}%
\pgfsetdash{}{0pt}%
\pgfpathmoveto{\pgfqpoint{4.185757in}{0.500000in}}%
\pgfpathlineto{\pgfqpoint{4.218108in}{0.500000in}}%
\pgfpathlineto{\pgfqpoint{4.218108in}{0.500000in}}%
\pgfpathlineto{\pgfqpoint{4.185757in}{0.500000in}}%
\pgfpathlineto{\pgfqpoint{4.185757in}{0.500000in}}%
\pgfpathclose%
\pgfusepath{fill}%
\end{pgfscope}%
\begin{pgfscope}%
\pgfpathrectangle{\pgfqpoint{0.750000in}{0.500000in}}{\pgfqpoint{4.650000in}{3.020000in}}%
\pgfusepath{clip}%
\pgfsetbuttcap%
\pgfsetmiterjoin%
\definecolor{currentfill}{rgb}{1.000000,0.000000,0.000000}%
\pgfsetfillcolor{currentfill}%
\pgfsetlinewidth{0.000000pt}%
\definecolor{currentstroke}{rgb}{0.000000,0.000000,0.000000}%
\pgfsetstrokecolor{currentstroke}%
\pgfsetstrokeopacity{0.000000}%
\pgfsetdash{}{0pt}%
\pgfpathmoveto{\pgfqpoint{4.218108in}{0.500000in}}%
\pgfpathlineto{\pgfqpoint{4.250459in}{0.500000in}}%
\pgfpathlineto{\pgfqpoint{4.250459in}{0.500000in}}%
\pgfpathlineto{\pgfqpoint{4.218108in}{0.500000in}}%
\pgfpathlineto{\pgfqpoint{4.218108in}{0.500000in}}%
\pgfpathclose%
\pgfusepath{fill}%
\end{pgfscope}%
\begin{pgfscope}%
\pgfpathrectangle{\pgfqpoint{0.750000in}{0.500000in}}{\pgfqpoint{4.650000in}{3.020000in}}%
\pgfusepath{clip}%
\pgfsetbuttcap%
\pgfsetmiterjoin%
\definecolor{currentfill}{rgb}{1.000000,0.000000,0.000000}%
\pgfsetfillcolor{currentfill}%
\pgfsetlinewidth{0.000000pt}%
\definecolor{currentstroke}{rgb}{0.000000,0.000000,0.000000}%
\pgfsetstrokecolor{currentstroke}%
\pgfsetstrokeopacity{0.000000}%
\pgfsetdash{}{0pt}%
\pgfpathmoveto{\pgfqpoint{4.250459in}{0.500000in}}%
\pgfpathlineto{\pgfqpoint{4.282810in}{0.500000in}}%
\pgfpathlineto{\pgfqpoint{4.282810in}{0.502926in}}%
\pgfpathlineto{\pgfqpoint{4.250459in}{0.502926in}}%
\pgfpathlineto{\pgfqpoint{4.250459in}{0.500000in}}%
\pgfpathclose%
\pgfusepath{fill}%
\end{pgfscope}%
\begin{pgfscope}%
\pgfpathrectangle{\pgfqpoint{0.750000in}{0.500000in}}{\pgfqpoint{4.650000in}{3.020000in}}%
\pgfusepath{clip}%
\pgfsetbuttcap%
\pgfsetmiterjoin%
\definecolor{currentfill}{rgb}{1.000000,0.000000,0.000000}%
\pgfsetfillcolor{currentfill}%
\pgfsetlinewidth{0.000000pt}%
\definecolor{currentstroke}{rgb}{0.000000,0.000000,0.000000}%
\pgfsetstrokecolor{currentstroke}%
\pgfsetstrokeopacity{0.000000}%
\pgfsetdash{}{0pt}%
\pgfpathmoveto{\pgfqpoint{4.282810in}{0.500000in}}%
\pgfpathlineto{\pgfqpoint{4.315161in}{0.500000in}}%
\pgfpathlineto{\pgfqpoint{4.315161in}{0.500000in}}%
\pgfpathlineto{\pgfqpoint{4.282810in}{0.500000in}}%
\pgfpathlineto{\pgfqpoint{4.282810in}{0.500000in}}%
\pgfpathclose%
\pgfusepath{fill}%
\end{pgfscope}%
\begin{pgfscope}%
\pgfpathrectangle{\pgfqpoint{0.750000in}{0.500000in}}{\pgfqpoint{4.650000in}{3.020000in}}%
\pgfusepath{clip}%
\pgfsetbuttcap%
\pgfsetmiterjoin%
\definecolor{currentfill}{rgb}{1.000000,0.000000,0.000000}%
\pgfsetfillcolor{currentfill}%
\pgfsetlinewidth{0.000000pt}%
\definecolor{currentstroke}{rgb}{0.000000,0.000000,0.000000}%
\pgfsetstrokecolor{currentstroke}%
\pgfsetstrokeopacity{0.000000}%
\pgfsetdash{}{0pt}%
\pgfpathmoveto{\pgfqpoint{4.315161in}{0.500000in}}%
\pgfpathlineto{\pgfqpoint{4.347512in}{0.500000in}}%
\pgfpathlineto{\pgfqpoint{4.347512in}{0.500000in}}%
\pgfpathlineto{\pgfqpoint{4.315161in}{0.500000in}}%
\pgfpathlineto{\pgfqpoint{4.315161in}{0.500000in}}%
\pgfpathclose%
\pgfusepath{fill}%
\end{pgfscope}%
\begin{pgfscope}%
\pgfpathrectangle{\pgfqpoint{0.750000in}{0.500000in}}{\pgfqpoint{4.650000in}{3.020000in}}%
\pgfusepath{clip}%
\pgfsetbuttcap%
\pgfsetmiterjoin%
\definecolor{currentfill}{rgb}{1.000000,0.000000,0.000000}%
\pgfsetfillcolor{currentfill}%
\pgfsetlinewidth{0.000000pt}%
\definecolor{currentstroke}{rgb}{0.000000,0.000000,0.000000}%
\pgfsetstrokecolor{currentstroke}%
\pgfsetstrokeopacity{0.000000}%
\pgfsetdash{}{0pt}%
\pgfpathmoveto{\pgfqpoint{4.347512in}{0.500000in}}%
\pgfpathlineto{\pgfqpoint{4.379863in}{0.500000in}}%
\pgfpathlineto{\pgfqpoint{4.379863in}{0.500000in}}%
\pgfpathlineto{\pgfqpoint{4.347512in}{0.500000in}}%
\pgfpathlineto{\pgfqpoint{4.347512in}{0.500000in}}%
\pgfpathclose%
\pgfusepath{fill}%
\end{pgfscope}%
\begin{pgfscope}%
\pgfpathrectangle{\pgfqpoint{0.750000in}{0.500000in}}{\pgfqpoint{4.650000in}{3.020000in}}%
\pgfusepath{clip}%
\pgfsetbuttcap%
\pgfsetmiterjoin%
\definecolor{currentfill}{rgb}{1.000000,0.000000,0.000000}%
\pgfsetfillcolor{currentfill}%
\pgfsetlinewidth{0.000000pt}%
\definecolor{currentstroke}{rgb}{0.000000,0.000000,0.000000}%
\pgfsetstrokecolor{currentstroke}%
\pgfsetstrokeopacity{0.000000}%
\pgfsetdash{}{0pt}%
\pgfpathmoveto{\pgfqpoint{4.379863in}{0.500000in}}%
\pgfpathlineto{\pgfqpoint{4.412214in}{0.500000in}}%
\pgfpathlineto{\pgfqpoint{4.412214in}{0.500000in}}%
\pgfpathlineto{\pgfqpoint{4.379863in}{0.500000in}}%
\pgfpathlineto{\pgfqpoint{4.379863in}{0.500000in}}%
\pgfpathclose%
\pgfusepath{fill}%
\end{pgfscope}%
\begin{pgfscope}%
\pgfpathrectangle{\pgfqpoint{0.750000in}{0.500000in}}{\pgfqpoint{4.650000in}{3.020000in}}%
\pgfusepath{clip}%
\pgfsetbuttcap%
\pgfsetmiterjoin%
\definecolor{currentfill}{rgb}{1.000000,0.000000,0.000000}%
\pgfsetfillcolor{currentfill}%
\pgfsetlinewidth{0.000000pt}%
\definecolor{currentstroke}{rgb}{0.000000,0.000000,0.000000}%
\pgfsetstrokecolor{currentstroke}%
\pgfsetstrokeopacity{0.000000}%
\pgfsetdash{}{0pt}%
\pgfpathmoveto{\pgfqpoint{4.412214in}{0.500000in}}%
\pgfpathlineto{\pgfqpoint{4.444565in}{0.500000in}}%
\pgfpathlineto{\pgfqpoint{4.444565in}{0.500000in}}%
\pgfpathlineto{\pgfqpoint{4.412214in}{0.500000in}}%
\pgfpathlineto{\pgfqpoint{4.412214in}{0.500000in}}%
\pgfpathclose%
\pgfusepath{fill}%
\end{pgfscope}%
\begin{pgfscope}%
\pgfpathrectangle{\pgfqpoint{0.750000in}{0.500000in}}{\pgfqpoint{4.650000in}{3.020000in}}%
\pgfusepath{clip}%
\pgfsetbuttcap%
\pgfsetmiterjoin%
\definecolor{currentfill}{rgb}{1.000000,0.000000,0.000000}%
\pgfsetfillcolor{currentfill}%
\pgfsetlinewidth{0.000000pt}%
\definecolor{currentstroke}{rgb}{0.000000,0.000000,0.000000}%
\pgfsetstrokecolor{currentstroke}%
\pgfsetstrokeopacity{0.000000}%
\pgfsetdash{}{0pt}%
\pgfpathmoveto{\pgfqpoint{4.444565in}{0.500000in}}%
\pgfpathlineto{\pgfqpoint{4.476916in}{0.500000in}}%
\pgfpathlineto{\pgfqpoint{4.476916in}{0.500000in}}%
\pgfpathlineto{\pgfqpoint{4.444565in}{0.500000in}}%
\pgfpathlineto{\pgfqpoint{4.444565in}{0.500000in}}%
\pgfpathclose%
\pgfusepath{fill}%
\end{pgfscope}%
\begin{pgfscope}%
\pgfpathrectangle{\pgfqpoint{0.750000in}{0.500000in}}{\pgfqpoint{4.650000in}{3.020000in}}%
\pgfusepath{clip}%
\pgfsetbuttcap%
\pgfsetmiterjoin%
\definecolor{currentfill}{rgb}{1.000000,0.000000,0.000000}%
\pgfsetfillcolor{currentfill}%
\pgfsetlinewidth{0.000000pt}%
\definecolor{currentstroke}{rgb}{0.000000,0.000000,0.000000}%
\pgfsetstrokecolor{currentstroke}%
\pgfsetstrokeopacity{0.000000}%
\pgfsetdash{}{0pt}%
\pgfpathmoveto{\pgfqpoint{4.476916in}{0.500000in}}%
\pgfpathlineto{\pgfqpoint{4.509267in}{0.500000in}}%
\pgfpathlineto{\pgfqpoint{4.509267in}{0.500000in}}%
\pgfpathlineto{\pgfqpoint{4.476916in}{0.500000in}}%
\pgfpathlineto{\pgfqpoint{4.476916in}{0.500000in}}%
\pgfpathclose%
\pgfusepath{fill}%
\end{pgfscope}%
\begin{pgfscope}%
\pgfpathrectangle{\pgfqpoint{0.750000in}{0.500000in}}{\pgfqpoint{4.650000in}{3.020000in}}%
\pgfusepath{clip}%
\pgfsetbuttcap%
\pgfsetmiterjoin%
\definecolor{currentfill}{rgb}{1.000000,0.000000,0.000000}%
\pgfsetfillcolor{currentfill}%
\pgfsetlinewidth{0.000000pt}%
\definecolor{currentstroke}{rgb}{0.000000,0.000000,0.000000}%
\pgfsetstrokecolor{currentstroke}%
\pgfsetstrokeopacity{0.000000}%
\pgfsetdash{}{0pt}%
\pgfpathmoveto{\pgfqpoint{4.509267in}{0.500000in}}%
\pgfpathlineto{\pgfqpoint{4.541618in}{0.500000in}}%
\pgfpathlineto{\pgfqpoint{4.541618in}{0.502926in}}%
\pgfpathlineto{\pgfqpoint{4.509267in}{0.502926in}}%
\pgfpathlineto{\pgfqpoint{4.509267in}{0.500000in}}%
\pgfpathclose%
\pgfusepath{fill}%
\end{pgfscope}%
\begin{pgfscope}%
\pgfpathrectangle{\pgfqpoint{0.750000in}{0.500000in}}{\pgfqpoint{4.650000in}{3.020000in}}%
\pgfusepath{clip}%
\pgfsetbuttcap%
\pgfsetmiterjoin%
\definecolor{currentfill}{rgb}{1.000000,0.000000,0.000000}%
\pgfsetfillcolor{currentfill}%
\pgfsetlinewidth{0.000000pt}%
\definecolor{currentstroke}{rgb}{0.000000,0.000000,0.000000}%
\pgfsetstrokecolor{currentstroke}%
\pgfsetstrokeopacity{0.000000}%
\pgfsetdash{}{0pt}%
\pgfpathmoveto{\pgfqpoint{4.541618in}{0.500000in}}%
\pgfpathlineto{\pgfqpoint{4.573968in}{0.500000in}}%
\pgfpathlineto{\pgfqpoint{4.573968in}{0.502926in}}%
\pgfpathlineto{\pgfqpoint{4.541618in}{0.502926in}}%
\pgfpathlineto{\pgfqpoint{4.541618in}{0.500000in}}%
\pgfpathclose%
\pgfusepath{fill}%
\end{pgfscope}%
\begin{pgfscope}%
\pgfpathrectangle{\pgfqpoint{0.750000in}{0.500000in}}{\pgfqpoint{4.650000in}{3.020000in}}%
\pgfusepath{clip}%
\pgfsetbuttcap%
\pgfsetmiterjoin%
\definecolor{currentfill}{rgb}{1.000000,0.000000,0.000000}%
\pgfsetfillcolor{currentfill}%
\pgfsetlinewidth{0.000000pt}%
\definecolor{currentstroke}{rgb}{0.000000,0.000000,0.000000}%
\pgfsetstrokecolor{currentstroke}%
\pgfsetstrokeopacity{0.000000}%
\pgfsetdash{}{0pt}%
\pgfpathmoveto{\pgfqpoint{4.573968in}{0.500000in}}%
\pgfpathlineto{\pgfqpoint{4.606319in}{0.500000in}}%
\pgfpathlineto{\pgfqpoint{4.606319in}{0.502926in}}%
\pgfpathlineto{\pgfqpoint{4.573968in}{0.502926in}}%
\pgfpathlineto{\pgfqpoint{4.573968in}{0.500000in}}%
\pgfpathclose%
\pgfusepath{fill}%
\end{pgfscope}%
\begin{pgfscope}%
\pgfpathrectangle{\pgfqpoint{0.750000in}{0.500000in}}{\pgfqpoint{4.650000in}{3.020000in}}%
\pgfusepath{clip}%
\pgfsetbuttcap%
\pgfsetmiterjoin%
\definecolor{currentfill}{rgb}{1.000000,0.000000,0.000000}%
\pgfsetfillcolor{currentfill}%
\pgfsetlinewidth{0.000000pt}%
\definecolor{currentstroke}{rgb}{0.000000,0.000000,0.000000}%
\pgfsetstrokecolor{currentstroke}%
\pgfsetstrokeopacity{0.000000}%
\pgfsetdash{}{0pt}%
\pgfpathmoveto{\pgfqpoint{4.606319in}{0.500000in}}%
\pgfpathlineto{\pgfqpoint{4.638670in}{0.500000in}}%
\pgfpathlineto{\pgfqpoint{4.638670in}{0.500000in}}%
\pgfpathlineto{\pgfqpoint{4.606319in}{0.500000in}}%
\pgfpathlineto{\pgfqpoint{4.606319in}{0.500000in}}%
\pgfpathclose%
\pgfusepath{fill}%
\end{pgfscope}%
\begin{pgfscope}%
\pgfpathrectangle{\pgfqpoint{0.750000in}{0.500000in}}{\pgfqpoint{4.650000in}{3.020000in}}%
\pgfusepath{clip}%
\pgfsetbuttcap%
\pgfsetmiterjoin%
\definecolor{currentfill}{rgb}{1.000000,0.000000,0.000000}%
\pgfsetfillcolor{currentfill}%
\pgfsetlinewidth{0.000000pt}%
\definecolor{currentstroke}{rgb}{0.000000,0.000000,0.000000}%
\pgfsetstrokecolor{currentstroke}%
\pgfsetstrokeopacity{0.000000}%
\pgfsetdash{}{0pt}%
\pgfpathmoveto{\pgfqpoint{4.638670in}{0.500000in}}%
\pgfpathlineto{\pgfqpoint{4.671021in}{0.500000in}}%
\pgfpathlineto{\pgfqpoint{4.671021in}{0.500000in}}%
\pgfpathlineto{\pgfqpoint{4.638670in}{0.500000in}}%
\pgfpathlineto{\pgfqpoint{4.638670in}{0.500000in}}%
\pgfpathclose%
\pgfusepath{fill}%
\end{pgfscope}%
\begin{pgfscope}%
\pgfpathrectangle{\pgfqpoint{0.750000in}{0.500000in}}{\pgfqpoint{4.650000in}{3.020000in}}%
\pgfusepath{clip}%
\pgfsetbuttcap%
\pgfsetmiterjoin%
\definecolor{currentfill}{rgb}{1.000000,0.000000,0.000000}%
\pgfsetfillcolor{currentfill}%
\pgfsetlinewidth{0.000000pt}%
\definecolor{currentstroke}{rgb}{0.000000,0.000000,0.000000}%
\pgfsetstrokecolor{currentstroke}%
\pgfsetstrokeopacity{0.000000}%
\pgfsetdash{}{0pt}%
\pgfpathmoveto{\pgfqpoint{4.671021in}{0.500000in}}%
\pgfpathlineto{\pgfqpoint{4.703372in}{0.500000in}}%
\pgfpathlineto{\pgfqpoint{4.703372in}{0.500000in}}%
\pgfpathlineto{\pgfqpoint{4.671021in}{0.500000in}}%
\pgfpathlineto{\pgfqpoint{4.671021in}{0.500000in}}%
\pgfpathclose%
\pgfusepath{fill}%
\end{pgfscope}%
\begin{pgfscope}%
\pgfpathrectangle{\pgfqpoint{0.750000in}{0.500000in}}{\pgfqpoint{4.650000in}{3.020000in}}%
\pgfusepath{clip}%
\pgfsetbuttcap%
\pgfsetmiterjoin%
\definecolor{currentfill}{rgb}{1.000000,0.000000,0.000000}%
\pgfsetfillcolor{currentfill}%
\pgfsetlinewidth{0.000000pt}%
\definecolor{currentstroke}{rgb}{0.000000,0.000000,0.000000}%
\pgfsetstrokecolor{currentstroke}%
\pgfsetstrokeopacity{0.000000}%
\pgfsetdash{}{0pt}%
\pgfpathmoveto{\pgfqpoint{4.703372in}{0.500000in}}%
\pgfpathlineto{\pgfqpoint{4.735723in}{0.500000in}}%
\pgfpathlineto{\pgfqpoint{4.735723in}{0.500000in}}%
\pgfpathlineto{\pgfqpoint{4.703372in}{0.500000in}}%
\pgfpathlineto{\pgfqpoint{4.703372in}{0.500000in}}%
\pgfpathclose%
\pgfusepath{fill}%
\end{pgfscope}%
\begin{pgfscope}%
\pgfpathrectangle{\pgfqpoint{0.750000in}{0.500000in}}{\pgfqpoint{4.650000in}{3.020000in}}%
\pgfusepath{clip}%
\pgfsetbuttcap%
\pgfsetmiterjoin%
\definecolor{currentfill}{rgb}{1.000000,0.000000,0.000000}%
\pgfsetfillcolor{currentfill}%
\pgfsetlinewidth{0.000000pt}%
\definecolor{currentstroke}{rgb}{0.000000,0.000000,0.000000}%
\pgfsetstrokecolor{currentstroke}%
\pgfsetstrokeopacity{0.000000}%
\pgfsetdash{}{0pt}%
\pgfpathmoveto{\pgfqpoint{4.735723in}{0.500000in}}%
\pgfpathlineto{\pgfqpoint{4.768074in}{0.500000in}}%
\pgfpathlineto{\pgfqpoint{4.768074in}{0.502926in}}%
\pgfpathlineto{\pgfqpoint{4.735723in}{0.502926in}}%
\pgfpathlineto{\pgfqpoint{4.735723in}{0.500000in}}%
\pgfpathclose%
\pgfusepath{fill}%
\end{pgfscope}%
\begin{pgfscope}%
\pgfpathrectangle{\pgfqpoint{0.750000in}{0.500000in}}{\pgfqpoint{4.650000in}{3.020000in}}%
\pgfusepath{clip}%
\pgfsetbuttcap%
\pgfsetmiterjoin%
\definecolor{currentfill}{rgb}{1.000000,0.000000,0.000000}%
\pgfsetfillcolor{currentfill}%
\pgfsetlinewidth{0.000000pt}%
\definecolor{currentstroke}{rgb}{0.000000,0.000000,0.000000}%
\pgfsetstrokecolor{currentstroke}%
\pgfsetstrokeopacity{0.000000}%
\pgfsetdash{}{0pt}%
\pgfpathmoveto{\pgfqpoint{4.768074in}{0.500000in}}%
\pgfpathlineto{\pgfqpoint{4.800425in}{0.500000in}}%
\pgfpathlineto{\pgfqpoint{4.800425in}{0.500000in}}%
\pgfpathlineto{\pgfqpoint{4.768074in}{0.500000in}}%
\pgfpathlineto{\pgfqpoint{4.768074in}{0.500000in}}%
\pgfpathclose%
\pgfusepath{fill}%
\end{pgfscope}%
\begin{pgfscope}%
\pgfpathrectangle{\pgfqpoint{0.750000in}{0.500000in}}{\pgfqpoint{4.650000in}{3.020000in}}%
\pgfusepath{clip}%
\pgfsetbuttcap%
\pgfsetmiterjoin%
\definecolor{currentfill}{rgb}{1.000000,0.000000,0.000000}%
\pgfsetfillcolor{currentfill}%
\pgfsetlinewidth{0.000000pt}%
\definecolor{currentstroke}{rgb}{0.000000,0.000000,0.000000}%
\pgfsetstrokecolor{currentstroke}%
\pgfsetstrokeopacity{0.000000}%
\pgfsetdash{}{0pt}%
\pgfpathmoveto{\pgfqpoint{4.800425in}{0.500000in}}%
\pgfpathlineto{\pgfqpoint{4.832776in}{0.500000in}}%
\pgfpathlineto{\pgfqpoint{4.832776in}{0.500000in}}%
\pgfpathlineto{\pgfqpoint{4.800425in}{0.500000in}}%
\pgfpathlineto{\pgfqpoint{4.800425in}{0.500000in}}%
\pgfpathclose%
\pgfusepath{fill}%
\end{pgfscope}%
\begin{pgfscope}%
\pgfpathrectangle{\pgfqpoint{0.750000in}{0.500000in}}{\pgfqpoint{4.650000in}{3.020000in}}%
\pgfusepath{clip}%
\pgfsetbuttcap%
\pgfsetmiterjoin%
\definecolor{currentfill}{rgb}{1.000000,0.000000,0.000000}%
\pgfsetfillcolor{currentfill}%
\pgfsetlinewidth{0.000000pt}%
\definecolor{currentstroke}{rgb}{0.000000,0.000000,0.000000}%
\pgfsetstrokecolor{currentstroke}%
\pgfsetstrokeopacity{0.000000}%
\pgfsetdash{}{0pt}%
\pgfpathmoveto{\pgfqpoint{4.832776in}{0.500000in}}%
\pgfpathlineto{\pgfqpoint{4.865127in}{0.500000in}}%
\pgfpathlineto{\pgfqpoint{4.865127in}{0.502926in}}%
\pgfpathlineto{\pgfqpoint{4.832776in}{0.502926in}}%
\pgfpathlineto{\pgfqpoint{4.832776in}{0.500000in}}%
\pgfpathclose%
\pgfusepath{fill}%
\end{pgfscope}%
\begin{pgfscope}%
\pgfpathrectangle{\pgfqpoint{0.750000in}{0.500000in}}{\pgfqpoint{4.650000in}{3.020000in}}%
\pgfusepath{clip}%
\pgfsetbuttcap%
\pgfsetmiterjoin%
\definecolor{currentfill}{rgb}{1.000000,0.000000,0.000000}%
\pgfsetfillcolor{currentfill}%
\pgfsetlinewidth{0.000000pt}%
\definecolor{currentstroke}{rgb}{0.000000,0.000000,0.000000}%
\pgfsetstrokecolor{currentstroke}%
\pgfsetstrokeopacity{0.000000}%
\pgfsetdash{}{0pt}%
\pgfpathmoveto{\pgfqpoint{4.865127in}{0.500000in}}%
\pgfpathlineto{\pgfqpoint{4.897478in}{0.500000in}}%
\pgfpathlineto{\pgfqpoint{4.897478in}{0.500000in}}%
\pgfpathlineto{\pgfqpoint{4.865127in}{0.500000in}}%
\pgfpathlineto{\pgfqpoint{4.865127in}{0.500000in}}%
\pgfpathclose%
\pgfusepath{fill}%
\end{pgfscope}%
\begin{pgfscope}%
\pgfpathrectangle{\pgfqpoint{0.750000in}{0.500000in}}{\pgfqpoint{4.650000in}{3.020000in}}%
\pgfusepath{clip}%
\pgfsetbuttcap%
\pgfsetmiterjoin%
\definecolor{currentfill}{rgb}{1.000000,0.000000,0.000000}%
\pgfsetfillcolor{currentfill}%
\pgfsetlinewidth{0.000000pt}%
\definecolor{currentstroke}{rgb}{0.000000,0.000000,0.000000}%
\pgfsetstrokecolor{currentstroke}%
\pgfsetstrokeopacity{0.000000}%
\pgfsetdash{}{0pt}%
\pgfpathmoveto{\pgfqpoint{4.897478in}{0.500000in}}%
\pgfpathlineto{\pgfqpoint{4.929829in}{0.500000in}}%
\pgfpathlineto{\pgfqpoint{4.929829in}{0.500000in}}%
\pgfpathlineto{\pgfqpoint{4.897478in}{0.500000in}}%
\pgfpathlineto{\pgfqpoint{4.897478in}{0.500000in}}%
\pgfpathclose%
\pgfusepath{fill}%
\end{pgfscope}%
\begin{pgfscope}%
\pgfpathrectangle{\pgfqpoint{0.750000in}{0.500000in}}{\pgfqpoint{4.650000in}{3.020000in}}%
\pgfusepath{clip}%
\pgfsetbuttcap%
\pgfsetmiterjoin%
\definecolor{currentfill}{rgb}{1.000000,0.000000,0.000000}%
\pgfsetfillcolor{currentfill}%
\pgfsetlinewidth{0.000000pt}%
\definecolor{currentstroke}{rgb}{0.000000,0.000000,0.000000}%
\pgfsetstrokecolor{currentstroke}%
\pgfsetstrokeopacity{0.000000}%
\pgfsetdash{}{0pt}%
\pgfpathmoveto{\pgfqpoint{4.929829in}{0.500000in}}%
\pgfpathlineto{\pgfqpoint{4.962180in}{0.500000in}}%
\pgfpathlineto{\pgfqpoint{4.962180in}{0.500000in}}%
\pgfpathlineto{\pgfqpoint{4.929829in}{0.500000in}}%
\pgfpathlineto{\pgfqpoint{4.929829in}{0.500000in}}%
\pgfpathclose%
\pgfusepath{fill}%
\end{pgfscope}%
\begin{pgfscope}%
\pgfpathrectangle{\pgfqpoint{0.750000in}{0.500000in}}{\pgfqpoint{4.650000in}{3.020000in}}%
\pgfusepath{clip}%
\pgfsetbuttcap%
\pgfsetmiterjoin%
\definecolor{currentfill}{rgb}{1.000000,0.000000,0.000000}%
\pgfsetfillcolor{currentfill}%
\pgfsetlinewidth{0.000000pt}%
\definecolor{currentstroke}{rgb}{0.000000,0.000000,0.000000}%
\pgfsetstrokecolor{currentstroke}%
\pgfsetstrokeopacity{0.000000}%
\pgfsetdash{}{0pt}%
\pgfpathmoveto{\pgfqpoint{4.962180in}{0.500000in}}%
\pgfpathlineto{\pgfqpoint{4.994531in}{0.500000in}}%
\pgfpathlineto{\pgfqpoint{4.994531in}{0.500000in}}%
\pgfpathlineto{\pgfqpoint{4.962180in}{0.500000in}}%
\pgfpathlineto{\pgfqpoint{4.962180in}{0.500000in}}%
\pgfpathclose%
\pgfusepath{fill}%
\end{pgfscope}%
\begin{pgfscope}%
\pgfpathrectangle{\pgfqpoint{0.750000in}{0.500000in}}{\pgfqpoint{4.650000in}{3.020000in}}%
\pgfusepath{clip}%
\pgfsetbuttcap%
\pgfsetmiterjoin%
\definecolor{currentfill}{rgb}{1.000000,0.000000,0.000000}%
\pgfsetfillcolor{currentfill}%
\pgfsetlinewidth{0.000000pt}%
\definecolor{currentstroke}{rgb}{0.000000,0.000000,0.000000}%
\pgfsetstrokecolor{currentstroke}%
\pgfsetstrokeopacity{0.000000}%
\pgfsetdash{}{0pt}%
\pgfpathmoveto{\pgfqpoint{4.994531in}{0.500000in}}%
\pgfpathlineto{\pgfqpoint{5.026882in}{0.500000in}}%
\pgfpathlineto{\pgfqpoint{5.026882in}{0.500000in}}%
\pgfpathlineto{\pgfqpoint{4.994531in}{0.500000in}}%
\pgfpathlineto{\pgfqpoint{4.994531in}{0.500000in}}%
\pgfpathclose%
\pgfusepath{fill}%
\end{pgfscope}%
\begin{pgfscope}%
\pgfpathrectangle{\pgfqpoint{0.750000in}{0.500000in}}{\pgfqpoint{4.650000in}{3.020000in}}%
\pgfusepath{clip}%
\pgfsetbuttcap%
\pgfsetmiterjoin%
\definecolor{currentfill}{rgb}{1.000000,0.000000,0.000000}%
\pgfsetfillcolor{currentfill}%
\pgfsetlinewidth{0.000000pt}%
\definecolor{currentstroke}{rgb}{0.000000,0.000000,0.000000}%
\pgfsetstrokecolor{currentstroke}%
\pgfsetstrokeopacity{0.000000}%
\pgfsetdash{}{0pt}%
\pgfpathmoveto{\pgfqpoint{5.026882in}{0.500000in}}%
\pgfpathlineto{\pgfqpoint{5.059233in}{0.500000in}}%
\pgfpathlineto{\pgfqpoint{5.059233in}{0.500000in}}%
\pgfpathlineto{\pgfqpoint{5.026882in}{0.500000in}}%
\pgfpathlineto{\pgfqpoint{5.026882in}{0.500000in}}%
\pgfpathclose%
\pgfusepath{fill}%
\end{pgfscope}%
\begin{pgfscope}%
\pgfpathrectangle{\pgfqpoint{0.750000in}{0.500000in}}{\pgfqpoint{4.650000in}{3.020000in}}%
\pgfusepath{clip}%
\pgfsetbuttcap%
\pgfsetmiterjoin%
\definecolor{currentfill}{rgb}{1.000000,0.000000,0.000000}%
\pgfsetfillcolor{currentfill}%
\pgfsetlinewidth{0.000000pt}%
\definecolor{currentstroke}{rgb}{0.000000,0.000000,0.000000}%
\pgfsetstrokecolor{currentstroke}%
\pgfsetstrokeopacity{0.000000}%
\pgfsetdash{}{0pt}%
\pgfpathmoveto{\pgfqpoint{5.059233in}{0.500000in}}%
\pgfpathlineto{\pgfqpoint{5.091584in}{0.500000in}}%
\pgfpathlineto{\pgfqpoint{5.091584in}{0.500000in}}%
\pgfpathlineto{\pgfqpoint{5.059233in}{0.500000in}}%
\pgfpathlineto{\pgfqpoint{5.059233in}{0.500000in}}%
\pgfpathclose%
\pgfusepath{fill}%
\end{pgfscope}%
\begin{pgfscope}%
\pgfpathrectangle{\pgfqpoint{0.750000in}{0.500000in}}{\pgfqpoint{4.650000in}{3.020000in}}%
\pgfusepath{clip}%
\pgfsetbuttcap%
\pgfsetmiterjoin%
\definecolor{currentfill}{rgb}{1.000000,0.000000,0.000000}%
\pgfsetfillcolor{currentfill}%
\pgfsetlinewidth{0.000000pt}%
\definecolor{currentstroke}{rgb}{0.000000,0.000000,0.000000}%
\pgfsetstrokecolor{currentstroke}%
\pgfsetstrokeopacity{0.000000}%
\pgfsetdash{}{0pt}%
\pgfpathmoveto{\pgfqpoint{5.091584in}{0.500000in}}%
\pgfpathlineto{\pgfqpoint{5.123934in}{0.500000in}}%
\pgfpathlineto{\pgfqpoint{5.123934in}{0.500000in}}%
\pgfpathlineto{\pgfqpoint{5.091584in}{0.500000in}}%
\pgfpathlineto{\pgfqpoint{5.091584in}{0.500000in}}%
\pgfpathclose%
\pgfusepath{fill}%
\end{pgfscope}%
\begin{pgfscope}%
\pgfpathrectangle{\pgfqpoint{0.750000in}{0.500000in}}{\pgfqpoint{4.650000in}{3.020000in}}%
\pgfusepath{clip}%
\pgfsetbuttcap%
\pgfsetmiterjoin%
\definecolor{currentfill}{rgb}{1.000000,0.000000,0.000000}%
\pgfsetfillcolor{currentfill}%
\pgfsetlinewidth{0.000000pt}%
\definecolor{currentstroke}{rgb}{0.000000,0.000000,0.000000}%
\pgfsetstrokecolor{currentstroke}%
\pgfsetstrokeopacity{0.000000}%
\pgfsetdash{}{0pt}%
\pgfpathmoveto{\pgfqpoint{5.123934in}{0.500000in}}%
\pgfpathlineto{\pgfqpoint{5.156285in}{0.500000in}}%
\pgfpathlineto{\pgfqpoint{5.156285in}{0.500000in}}%
\pgfpathlineto{\pgfqpoint{5.123934in}{0.500000in}}%
\pgfpathlineto{\pgfqpoint{5.123934in}{0.500000in}}%
\pgfpathclose%
\pgfusepath{fill}%
\end{pgfscope}%
\begin{pgfscope}%
\pgfpathrectangle{\pgfqpoint{0.750000in}{0.500000in}}{\pgfqpoint{4.650000in}{3.020000in}}%
\pgfusepath{clip}%
\pgfsetbuttcap%
\pgfsetmiterjoin%
\definecolor{currentfill}{rgb}{1.000000,0.000000,0.000000}%
\pgfsetfillcolor{currentfill}%
\pgfsetlinewidth{0.000000pt}%
\definecolor{currentstroke}{rgb}{0.000000,0.000000,0.000000}%
\pgfsetstrokecolor{currentstroke}%
\pgfsetstrokeopacity{0.000000}%
\pgfsetdash{}{0pt}%
\pgfpathmoveto{\pgfqpoint{5.156285in}{0.500000in}}%
\pgfpathlineto{\pgfqpoint{5.188636in}{0.500000in}}%
\pgfpathlineto{\pgfqpoint{5.188636in}{0.505852in}}%
\pgfpathlineto{\pgfqpoint{5.156285in}{0.505852in}}%
\pgfpathlineto{\pgfqpoint{5.156285in}{0.500000in}}%
\pgfpathclose%
\pgfusepath{fill}%
\end{pgfscope}%
\begin{pgfscope}%
\pgfpathrectangle{\pgfqpoint{0.750000in}{0.500000in}}{\pgfqpoint{4.650000in}{3.020000in}}%
\pgfusepath{clip}%
\pgfsetbuttcap%
\pgfsetmiterjoin%
\definecolor{currentfill}{rgb}{0.000000,0.500000,0.000000}%
\pgfsetfillcolor{currentfill}%
\pgfsetlinewidth{0.000000pt}%
\definecolor{currentstroke}{rgb}{0.000000,0.000000,0.000000}%
\pgfsetstrokecolor{currentstroke}%
\pgfsetstrokeopacity{0.000000}%
\pgfsetdash{}{0pt}%
\pgfpathmoveto{\pgfqpoint{0.961364in}{0.500000in}}%
\pgfpathlineto{\pgfqpoint{0.993984in}{0.500000in}}%
\pgfpathlineto{\pgfqpoint{0.993984in}{0.502926in}}%
\pgfpathlineto{\pgfqpoint{0.961364in}{0.502926in}}%
\pgfpathlineto{\pgfqpoint{0.961364in}{0.500000in}}%
\pgfpathclose%
\pgfusepath{fill}%
\end{pgfscope}%
\begin{pgfscope}%
\pgfpathrectangle{\pgfqpoint{0.750000in}{0.500000in}}{\pgfqpoint{4.650000in}{3.020000in}}%
\pgfusepath{clip}%
\pgfsetbuttcap%
\pgfsetmiterjoin%
\definecolor{currentfill}{rgb}{0.000000,0.500000,0.000000}%
\pgfsetfillcolor{currentfill}%
\pgfsetlinewidth{0.000000pt}%
\definecolor{currentstroke}{rgb}{0.000000,0.000000,0.000000}%
\pgfsetstrokecolor{currentstroke}%
\pgfsetstrokeopacity{0.000000}%
\pgfsetdash{}{0pt}%
\pgfpathmoveto{\pgfqpoint{0.993984in}{0.500000in}}%
\pgfpathlineto{\pgfqpoint{1.026604in}{0.500000in}}%
\pgfpathlineto{\pgfqpoint{1.026604in}{0.500000in}}%
\pgfpathlineto{\pgfqpoint{0.993984in}{0.500000in}}%
\pgfpathlineto{\pgfqpoint{0.993984in}{0.500000in}}%
\pgfpathclose%
\pgfusepath{fill}%
\end{pgfscope}%
\begin{pgfscope}%
\pgfpathrectangle{\pgfqpoint{0.750000in}{0.500000in}}{\pgfqpoint{4.650000in}{3.020000in}}%
\pgfusepath{clip}%
\pgfsetbuttcap%
\pgfsetmiterjoin%
\definecolor{currentfill}{rgb}{0.000000,0.500000,0.000000}%
\pgfsetfillcolor{currentfill}%
\pgfsetlinewidth{0.000000pt}%
\definecolor{currentstroke}{rgb}{0.000000,0.000000,0.000000}%
\pgfsetstrokecolor{currentstroke}%
\pgfsetstrokeopacity{0.000000}%
\pgfsetdash{}{0pt}%
\pgfpathmoveto{\pgfqpoint{1.026604in}{0.500000in}}%
\pgfpathlineto{\pgfqpoint{1.059225in}{0.500000in}}%
\pgfpathlineto{\pgfqpoint{1.059225in}{0.500000in}}%
\pgfpathlineto{\pgfqpoint{1.026604in}{0.500000in}}%
\pgfpathlineto{\pgfqpoint{1.026604in}{0.500000in}}%
\pgfpathclose%
\pgfusepath{fill}%
\end{pgfscope}%
\begin{pgfscope}%
\pgfpathrectangle{\pgfqpoint{0.750000in}{0.500000in}}{\pgfqpoint{4.650000in}{3.020000in}}%
\pgfusepath{clip}%
\pgfsetbuttcap%
\pgfsetmiterjoin%
\definecolor{currentfill}{rgb}{0.000000,0.500000,0.000000}%
\pgfsetfillcolor{currentfill}%
\pgfsetlinewidth{0.000000pt}%
\definecolor{currentstroke}{rgb}{0.000000,0.000000,0.000000}%
\pgfsetstrokecolor{currentstroke}%
\pgfsetstrokeopacity{0.000000}%
\pgfsetdash{}{0pt}%
\pgfpathmoveto{\pgfqpoint{1.059225in}{0.500000in}}%
\pgfpathlineto{\pgfqpoint{1.091845in}{0.500000in}}%
\pgfpathlineto{\pgfqpoint{1.091845in}{0.502926in}}%
\pgfpathlineto{\pgfqpoint{1.059225in}{0.502926in}}%
\pgfpathlineto{\pgfqpoint{1.059225in}{0.500000in}}%
\pgfpathclose%
\pgfusepath{fill}%
\end{pgfscope}%
\begin{pgfscope}%
\pgfpathrectangle{\pgfqpoint{0.750000in}{0.500000in}}{\pgfqpoint{4.650000in}{3.020000in}}%
\pgfusepath{clip}%
\pgfsetbuttcap%
\pgfsetmiterjoin%
\definecolor{currentfill}{rgb}{0.000000,0.500000,0.000000}%
\pgfsetfillcolor{currentfill}%
\pgfsetlinewidth{0.000000pt}%
\definecolor{currentstroke}{rgb}{0.000000,0.000000,0.000000}%
\pgfsetstrokecolor{currentstroke}%
\pgfsetstrokeopacity{0.000000}%
\pgfsetdash{}{0pt}%
\pgfpathmoveto{\pgfqpoint{1.091845in}{0.500000in}}%
\pgfpathlineto{\pgfqpoint{1.124466in}{0.500000in}}%
\pgfpathlineto{\pgfqpoint{1.124466in}{0.505852in}}%
\pgfpathlineto{\pgfqpoint{1.091845in}{0.505852in}}%
\pgfpathlineto{\pgfqpoint{1.091845in}{0.500000in}}%
\pgfpathclose%
\pgfusepath{fill}%
\end{pgfscope}%
\begin{pgfscope}%
\pgfpathrectangle{\pgfqpoint{0.750000in}{0.500000in}}{\pgfqpoint{4.650000in}{3.020000in}}%
\pgfusepath{clip}%
\pgfsetbuttcap%
\pgfsetmiterjoin%
\definecolor{currentfill}{rgb}{0.000000,0.500000,0.000000}%
\pgfsetfillcolor{currentfill}%
\pgfsetlinewidth{0.000000pt}%
\definecolor{currentstroke}{rgb}{0.000000,0.000000,0.000000}%
\pgfsetstrokecolor{currentstroke}%
\pgfsetstrokeopacity{0.000000}%
\pgfsetdash{}{0pt}%
\pgfpathmoveto{\pgfqpoint{1.124466in}{0.500000in}}%
\pgfpathlineto{\pgfqpoint{1.157086in}{0.500000in}}%
\pgfpathlineto{\pgfqpoint{1.157086in}{0.502926in}}%
\pgfpathlineto{\pgfqpoint{1.124466in}{0.502926in}}%
\pgfpathlineto{\pgfqpoint{1.124466in}{0.500000in}}%
\pgfpathclose%
\pgfusepath{fill}%
\end{pgfscope}%
\begin{pgfscope}%
\pgfpathrectangle{\pgfqpoint{0.750000in}{0.500000in}}{\pgfqpoint{4.650000in}{3.020000in}}%
\pgfusepath{clip}%
\pgfsetbuttcap%
\pgfsetmiterjoin%
\definecolor{currentfill}{rgb}{0.000000,0.500000,0.000000}%
\pgfsetfillcolor{currentfill}%
\pgfsetlinewidth{0.000000pt}%
\definecolor{currentstroke}{rgb}{0.000000,0.000000,0.000000}%
\pgfsetstrokecolor{currentstroke}%
\pgfsetstrokeopacity{0.000000}%
\pgfsetdash{}{0pt}%
\pgfpathmoveto{\pgfqpoint{1.157086in}{0.500000in}}%
\pgfpathlineto{\pgfqpoint{1.189706in}{0.500000in}}%
\pgfpathlineto{\pgfqpoint{1.189706in}{0.505852in}}%
\pgfpathlineto{\pgfqpoint{1.157086in}{0.505852in}}%
\pgfpathlineto{\pgfqpoint{1.157086in}{0.500000in}}%
\pgfpathclose%
\pgfusepath{fill}%
\end{pgfscope}%
\begin{pgfscope}%
\pgfpathrectangle{\pgfqpoint{0.750000in}{0.500000in}}{\pgfqpoint{4.650000in}{3.020000in}}%
\pgfusepath{clip}%
\pgfsetbuttcap%
\pgfsetmiterjoin%
\definecolor{currentfill}{rgb}{0.000000,0.500000,0.000000}%
\pgfsetfillcolor{currentfill}%
\pgfsetlinewidth{0.000000pt}%
\definecolor{currentstroke}{rgb}{0.000000,0.000000,0.000000}%
\pgfsetstrokecolor{currentstroke}%
\pgfsetstrokeopacity{0.000000}%
\pgfsetdash{}{0pt}%
\pgfpathmoveto{\pgfqpoint{1.189706in}{0.500000in}}%
\pgfpathlineto{\pgfqpoint{1.222327in}{0.500000in}}%
\pgfpathlineto{\pgfqpoint{1.222327in}{0.508778in}}%
\pgfpathlineto{\pgfqpoint{1.189706in}{0.508778in}}%
\pgfpathlineto{\pgfqpoint{1.189706in}{0.500000in}}%
\pgfpathclose%
\pgfusepath{fill}%
\end{pgfscope}%
\begin{pgfscope}%
\pgfpathrectangle{\pgfqpoint{0.750000in}{0.500000in}}{\pgfqpoint{4.650000in}{3.020000in}}%
\pgfusepath{clip}%
\pgfsetbuttcap%
\pgfsetmiterjoin%
\definecolor{currentfill}{rgb}{0.000000,0.500000,0.000000}%
\pgfsetfillcolor{currentfill}%
\pgfsetlinewidth{0.000000pt}%
\definecolor{currentstroke}{rgb}{0.000000,0.000000,0.000000}%
\pgfsetstrokecolor{currentstroke}%
\pgfsetstrokeopacity{0.000000}%
\pgfsetdash{}{0pt}%
\pgfpathmoveto{\pgfqpoint{1.222327in}{0.500000in}}%
\pgfpathlineto{\pgfqpoint{1.254947in}{0.500000in}}%
\pgfpathlineto{\pgfqpoint{1.254947in}{0.511704in}}%
\pgfpathlineto{\pgfqpoint{1.222327in}{0.511704in}}%
\pgfpathlineto{\pgfqpoint{1.222327in}{0.500000in}}%
\pgfpathclose%
\pgfusepath{fill}%
\end{pgfscope}%
\begin{pgfscope}%
\pgfpathrectangle{\pgfqpoint{0.750000in}{0.500000in}}{\pgfqpoint{4.650000in}{3.020000in}}%
\pgfusepath{clip}%
\pgfsetbuttcap%
\pgfsetmiterjoin%
\definecolor{currentfill}{rgb}{0.000000,0.500000,0.000000}%
\pgfsetfillcolor{currentfill}%
\pgfsetlinewidth{0.000000pt}%
\definecolor{currentstroke}{rgb}{0.000000,0.000000,0.000000}%
\pgfsetstrokecolor{currentstroke}%
\pgfsetstrokeopacity{0.000000}%
\pgfsetdash{}{0pt}%
\pgfpathmoveto{\pgfqpoint{1.254947in}{0.500000in}}%
\pgfpathlineto{\pgfqpoint{1.287568in}{0.500000in}}%
\pgfpathlineto{\pgfqpoint{1.287568in}{0.520482in}}%
\pgfpathlineto{\pgfqpoint{1.254947in}{0.520482in}}%
\pgfpathlineto{\pgfqpoint{1.254947in}{0.500000in}}%
\pgfpathclose%
\pgfusepath{fill}%
\end{pgfscope}%
\begin{pgfscope}%
\pgfpathrectangle{\pgfqpoint{0.750000in}{0.500000in}}{\pgfqpoint{4.650000in}{3.020000in}}%
\pgfusepath{clip}%
\pgfsetbuttcap%
\pgfsetmiterjoin%
\definecolor{currentfill}{rgb}{0.000000,0.500000,0.000000}%
\pgfsetfillcolor{currentfill}%
\pgfsetlinewidth{0.000000pt}%
\definecolor{currentstroke}{rgb}{0.000000,0.000000,0.000000}%
\pgfsetstrokecolor{currentstroke}%
\pgfsetstrokeopacity{0.000000}%
\pgfsetdash{}{0pt}%
\pgfpathmoveto{\pgfqpoint{1.287568in}{0.500000in}}%
\pgfpathlineto{\pgfqpoint{1.320188in}{0.500000in}}%
\pgfpathlineto{\pgfqpoint{1.320188in}{0.540963in}}%
\pgfpathlineto{\pgfqpoint{1.287568in}{0.540963in}}%
\pgfpathlineto{\pgfqpoint{1.287568in}{0.500000in}}%
\pgfpathclose%
\pgfusepath{fill}%
\end{pgfscope}%
\begin{pgfscope}%
\pgfpathrectangle{\pgfqpoint{0.750000in}{0.500000in}}{\pgfqpoint{4.650000in}{3.020000in}}%
\pgfusepath{clip}%
\pgfsetbuttcap%
\pgfsetmiterjoin%
\definecolor{currentfill}{rgb}{0.000000,0.500000,0.000000}%
\pgfsetfillcolor{currentfill}%
\pgfsetlinewidth{0.000000pt}%
\definecolor{currentstroke}{rgb}{0.000000,0.000000,0.000000}%
\pgfsetstrokecolor{currentstroke}%
\pgfsetstrokeopacity{0.000000}%
\pgfsetdash{}{0pt}%
\pgfpathmoveto{\pgfqpoint{1.320188in}{0.500000in}}%
\pgfpathlineto{\pgfqpoint{1.352808in}{0.500000in}}%
\pgfpathlineto{\pgfqpoint{1.352808in}{0.538037in}}%
\pgfpathlineto{\pgfqpoint{1.320188in}{0.538037in}}%
\pgfpathlineto{\pgfqpoint{1.320188in}{0.500000in}}%
\pgfpathclose%
\pgfusepath{fill}%
\end{pgfscope}%
\begin{pgfscope}%
\pgfpathrectangle{\pgfqpoint{0.750000in}{0.500000in}}{\pgfqpoint{4.650000in}{3.020000in}}%
\pgfusepath{clip}%
\pgfsetbuttcap%
\pgfsetmiterjoin%
\definecolor{currentfill}{rgb}{0.000000,0.500000,0.000000}%
\pgfsetfillcolor{currentfill}%
\pgfsetlinewidth{0.000000pt}%
\definecolor{currentstroke}{rgb}{0.000000,0.000000,0.000000}%
\pgfsetstrokecolor{currentstroke}%
\pgfsetstrokeopacity{0.000000}%
\pgfsetdash{}{0pt}%
\pgfpathmoveto{\pgfqpoint{1.352808in}{0.500000in}}%
\pgfpathlineto{\pgfqpoint{1.385429in}{0.500000in}}%
\pgfpathlineto{\pgfqpoint{1.385429in}{0.508778in}}%
\pgfpathlineto{\pgfqpoint{1.352808in}{0.508778in}}%
\pgfpathlineto{\pgfqpoint{1.352808in}{0.500000in}}%
\pgfpathclose%
\pgfusepath{fill}%
\end{pgfscope}%
\begin{pgfscope}%
\pgfpathrectangle{\pgfqpoint{0.750000in}{0.500000in}}{\pgfqpoint{4.650000in}{3.020000in}}%
\pgfusepath{clip}%
\pgfsetbuttcap%
\pgfsetmiterjoin%
\definecolor{currentfill}{rgb}{0.000000,0.500000,0.000000}%
\pgfsetfillcolor{currentfill}%
\pgfsetlinewidth{0.000000pt}%
\definecolor{currentstroke}{rgb}{0.000000,0.000000,0.000000}%
\pgfsetstrokecolor{currentstroke}%
\pgfsetstrokeopacity{0.000000}%
\pgfsetdash{}{0pt}%
\pgfpathmoveto{\pgfqpoint{1.385429in}{0.500000in}}%
\pgfpathlineto{\pgfqpoint{1.418049in}{0.500000in}}%
\pgfpathlineto{\pgfqpoint{1.418049in}{0.561445in}}%
\pgfpathlineto{\pgfqpoint{1.385429in}{0.561445in}}%
\pgfpathlineto{\pgfqpoint{1.385429in}{0.500000in}}%
\pgfpathclose%
\pgfusepath{fill}%
\end{pgfscope}%
\begin{pgfscope}%
\pgfpathrectangle{\pgfqpoint{0.750000in}{0.500000in}}{\pgfqpoint{4.650000in}{3.020000in}}%
\pgfusepath{clip}%
\pgfsetbuttcap%
\pgfsetmiterjoin%
\definecolor{currentfill}{rgb}{0.000000,0.500000,0.000000}%
\pgfsetfillcolor{currentfill}%
\pgfsetlinewidth{0.000000pt}%
\definecolor{currentstroke}{rgb}{0.000000,0.000000,0.000000}%
\pgfsetstrokecolor{currentstroke}%
\pgfsetstrokeopacity{0.000000}%
\pgfsetdash{}{0pt}%
\pgfpathmoveto{\pgfqpoint{1.418049in}{0.500000in}}%
\pgfpathlineto{\pgfqpoint{1.450670in}{0.500000in}}%
\pgfpathlineto{\pgfqpoint{1.450670in}{0.543889in}}%
\pgfpathlineto{\pgfqpoint{1.418049in}{0.543889in}}%
\pgfpathlineto{\pgfqpoint{1.418049in}{0.500000in}}%
\pgfpathclose%
\pgfusepath{fill}%
\end{pgfscope}%
\begin{pgfscope}%
\pgfpathrectangle{\pgfqpoint{0.750000in}{0.500000in}}{\pgfqpoint{4.650000in}{3.020000in}}%
\pgfusepath{clip}%
\pgfsetbuttcap%
\pgfsetmiterjoin%
\definecolor{currentfill}{rgb}{0.000000,0.500000,0.000000}%
\pgfsetfillcolor{currentfill}%
\pgfsetlinewidth{0.000000pt}%
\definecolor{currentstroke}{rgb}{0.000000,0.000000,0.000000}%
\pgfsetstrokecolor{currentstroke}%
\pgfsetstrokeopacity{0.000000}%
\pgfsetdash{}{0pt}%
\pgfpathmoveto{\pgfqpoint{1.450670in}{0.500000in}}%
\pgfpathlineto{\pgfqpoint{1.483290in}{0.500000in}}%
\pgfpathlineto{\pgfqpoint{1.483290in}{0.552667in}}%
\pgfpathlineto{\pgfqpoint{1.450670in}{0.552667in}}%
\pgfpathlineto{\pgfqpoint{1.450670in}{0.500000in}}%
\pgfpathclose%
\pgfusepath{fill}%
\end{pgfscope}%
\begin{pgfscope}%
\pgfpathrectangle{\pgfqpoint{0.750000in}{0.500000in}}{\pgfqpoint{4.650000in}{3.020000in}}%
\pgfusepath{clip}%
\pgfsetbuttcap%
\pgfsetmiterjoin%
\definecolor{currentfill}{rgb}{0.000000,0.500000,0.000000}%
\pgfsetfillcolor{currentfill}%
\pgfsetlinewidth{0.000000pt}%
\definecolor{currentstroke}{rgb}{0.000000,0.000000,0.000000}%
\pgfsetstrokecolor{currentstroke}%
\pgfsetstrokeopacity{0.000000}%
\pgfsetdash{}{0pt}%
\pgfpathmoveto{\pgfqpoint{1.483290in}{0.500000in}}%
\pgfpathlineto{\pgfqpoint{1.515910in}{0.500000in}}%
\pgfpathlineto{\pgfqpoint{1.515910in}{0.646297in}}%
\pgfpathlineto{\pgfqpoint{1.483290in}{0.646297in}}%
\pgfpathlineto{\pgfqpoint{1.483290in}{0.500000in}}%
\pgfpathclose%
\pgfusepath{fill}%
\end{pgfscope}%
\begin{pgfscope}%
\pgfpathrectangle{\pgfqpoint{0.750000in}{0.500000in}}{\pgfqpoint{4.650000in}{3.020000in}}%
\pgfusepath{clip}%
\pgfsetbuttcap%
\pgfsetmiterjoin%
\definecolor{currentfill}{rgb}{0.000000,0.500000,0.000000}%
\pgfsetfillcolor{currentfill}%
\pgfsetlinewidth{0.000000pt}%
\definecolor{currentstroke}{rgb}{0.000000,0.000000,0.000000}%
\pgfsetstrokecolor{currentstroke}%
\pgfsetstrokeopacity{0.000000}%
\pgfsetdash{}{0pt}%
\pgfpathmoveto{\pgfqpoint{1.515910in}{0.500000in}}%
\pgfpathlineto{\pgfqpoint{1.548531in}{0.500000in}}%
\pgfpathlineto{\pgfqpoint{1.548531in}{0.573148in}}%
\pgfpathlineto{\pgfqpoint{1.515910in}{0.573148in}}%
\pgfpathlineto{\pgfqpoint{1.515910in}{0.500000in}}%
\pgfpathclose%
\pgfusepath{fill}%
\end{pgfscope}%
\begin{pgfscope}%
\pgfpathrectangle{\pgfqpoint{0.750000in}{0.500000in}}{\pgfqpoint{4.650000in}{3.020000in}}%
\pgfusepath{clip}%
\pgfsetbuttcap%
\pgfsetmiterjoin%
\definecolor{currentfill}{rgb}{0.000000,0.500000,0.000000}%
\pgfsetfillcolor{currentfill}%
\pgfsetlinewidth{0.000000pt}%
\definecolor{currentstroke}{rgb}{0.000000,0.000000,0.000000}%
\pgfsetstrokecolor{currentstroke}%
\pgfsetstrokeopacity{0.000000}%
\pgfsetdash{}{0pt}%
\pgfpathmoveto{\pgfqpoint{1.548531in}{0.500000in}}%
\pgfpathlineto{\pgfqpoint{1.581151in}{0.500000in}}%
\pgfpathlineto{\pgfqpoint{1.581151in}{0.687260in}}%
\pgfpathlineto{\pgfqpoint{1.548531in}{0.687260in}}%
\pgfpathlineto{\pgfqpoint{1.548531in}{0.500000in}}%
\pgfpathclose%
\pgfusepath{fill}%
\end{pgfscope}%
\begin{pgfscope}%
\pgfpathrectangle{\pgfqpoint{0.750000in}{0.500000in}}{\pgfqpoint{4.650000in}{3.020000in}}%
\pgfusepath{clip}%
\pgfsetbuttcap%
\pgfsetmiterjoin%
\definecolor{currentfill}{rgb}{0.000000,0.500000,0.000000}%
\pgfsetfillcolor{currentfill}%
\pgfsetlinewidth{0.000000pt}%
\definecolor{currentstroke}{rgb}{0.000000,0.000000,0.000000}%
\pgfsetstrokecolor{currentstroke}%
\pgfsetstrokeopacity{0.000000}%
\pgfsetdash{}{0pt}%
\pgfpathmoveto{\pgfqpoint{1.581151in}{0.500000in}}%
\pgfpathlineto{\pgfqpoint{1.613772in}{0.500000in}}%
\pgfpathlineto{\pgfqpoint{1.613772in}{0.675556in}}%
\pgfpathlineto{\pgfqpoint{1.581151in}{0.675556in}}%
\pgfpathlineto{\pgfqpoint{1.581151in}{0.500000in}}%
\pgfpathclose%
\pgfusepath{fill}%
\end{pgfscope}%
\begin{pgfscope}%
\pgfpathrectangle{\pgfqpoint{0.750000in}{0.500000in}}{\pgfqpoint{4.650000in}{3.020000in}}%
\pgfusepath{clip}%
\pgfsetbuttcap%
\pgfsetmiterjoin%
\definecolor{currentfill}{rgb}{0.000000,0.500000,0.000000}%
\pgfsetfillcolor{currentfill}%
\pgfsetlinewidth{0.000000pt}%
\definecolor{currentstroke}{rgb}{0.000000,0.000000,0.000000}%
\pgfsetstrokecolor{currentstroke}%
\pgfsetstrokeopacity{0.000000}%
\pgfsetdash{}{0pt}%
\pgfpathmoveto{\pgfqpoint{1.613772in}{0.500000in}}%
\pgfpathlineto{\pgfqpoint{1.646392in}{0.500000in}}%
\pgfpathlineto{\pgfqpoint{1.646392in}{0.655074in}}%
\pgfpathlineto{\pgfqpoint{1.613772in}{0.655074in}}%
\pgfpathlineto{\pgfqpoint{1.613772in}{0.500000in}}%
\pgfpathclose%
\pgfusepath{fill}%
\end{pgfscope}%
\begin{pgfscope}%
\pgfpathrectangle{\pgfqpoint{0.750000in}{0.500000in}}{\pgfqpoint{4.650000in}{3.020000in}}%
\pgfusepath{clip}%
\pgfsetbuttcap%
\pgfsetmiterjoin%
\definecolor{currentfill}{rgb}{0.000000,0.500000,0.000000}%
\pgfsetfillcolor{currentfill}%
\pgfsetlinewidth{0.000000pt}%
\definecolor{currentstroke}{rgb}{0.000000,0.000000,0.000000}%
\pgfsetstrokecolor{currentstroke}%
\pgfsetstrokeopacity{0.000000}%
\pgfsetdash{}{0pt}%
\pgfpathmoveto{\pgfqpoint{1.646392in}{0.500000in}}%
\pgfpathlineto{\pgfqpoint{1.679012in}{0.500000in}}%
\pgfpathlineto{\pgfqpoint{1.679012in}{0.883297in}}%
\pgfpathlineto{\pgfqpoint{1.646392in}{0.883297in}}%
\pgfpathlineto{\pgfqpoint{1.646392in}{0.500000in}}%
\pgfpathclose%
\pgfusepath{fill}%
\end{pgfscope}%
\begin{pgfscope}%
\pgfpathrectangle{\pgfqpoint{0.750000in}{0.500000in}}{\pgfqpoint{4.650000in}{3.020000in}}%
\pgfusepath{clip}%
\pgfsetbuttcap%
\pgfsetmiterjoin%
\definecolor{currentfill}{rgb}{0.000000,0.500000,0.000000}%
\pgfsetfillcolor{currentfill}%
\pgfsetlinewidth{0.000000pt}%
\definecolor{currentstroke}{rgb}{0.000000,0.000000,0.000000}%
\pgfsetstrokecolor{currentstroke}%
\pgfsetstrokeopacity{0.000000}%
\pgfsetdash{}{0pt}%
\pgfpathmoveto{\pgfqpoint{1.679012in}{0.500000in}}%
\pgfpathlineto{\pgfqpoint{1.711633in}{0.500000in}}%
\pgfpathlineto{\pgfqpoint{1.711633in}{0.672630in}}%
\pgfpathlineto{\pgfqpoint{1.679012in}{0.672630in}}%
\pgfpathlineto{\pgfqpoint{1.679012in}{0.500000in}}%
\pgfpathclose%
\pgfusepath{fill}%
\end{pgfscope}%
\begin{pgfscope}%
\pgfpathrectangle{\pgfqpoint{0.750000in}{0.500000in}}{\pgfqpoint{4.650000in}{3.020000in}}%
\pgfusepath{clip}%
\pgfsetbuttcap%
\pgfsetmiterjoin%
\definecolor{currentfill}{rgb}{0.000000,0.500000,0.000000}%
\pgfsetfillcolor{currentfill}%
\pgfsetlinewidth{0.000000pt}%
\definecolor{currentstroke}{rgb}{0.000000,0.000000,0.000000}%
\pgfsetstrokecolor{currentstroke}%
\pgfsetstrokeopacity{0.000000}%
\pgfsetdash{}{0pt}%
\pgfpathmoveto{\pgfqpoint{1.711633in}{0.500000in}}%
\pgfpathlineto{\pgfqpoint{1.744253in}{0.500000in}}%
\pgfpathlineto{\pgfqpoint{1.744253in}{0.763334in}}%
\pgfpathlineto{\pgfqpoint{1.711633in}{0.763334in}}%
\pgfpathlineto{\pgfqpoint{1.711633in}{0.500000in}}%
\pgfpathclose%
\pgfusepath{fill}%
\end{pgfscope}%
\begin{pgfscope}%
\pgfpathrectangle{\pgfqpoint{0.750000in}{0.500000in}}{\pgfqpoint{4.650000in}{3.020000in}}%
\pgfusepath{clip}%
\pgfsetbuttcap%
\pgfsetmiterjoin%
\definecolor{currentfill}{rgb}{0.000000,0.500000,0.000000}%
\pgfsetfillcolor{currentfill}%
\pgfsetlinewidth{0.000000pt}%
\definecolor{currentstroke}{rgb}{0.000000,0.000000,0.000000}%
\pgfsetstrokecolor{currentstroke}%
\pgfsetstrokeopacity{0.000000}%
\pgfsetdash{}{0pt}%
\pgfpathmoveto{\pgfqpoint{1.744253in}{0.500000in}}%
\pgfpathlineto{\pgfqpoint{1.776874in}{0.500000in}}%
\pgfpathlineto{\pgfqpoint{1.776874in}{1.076408in}}%
\pgfpathlineto{\pgfqpoint{1.744253in}{1.076408in}}%
\pgfpathlineto{\pgfqpoint{1.744253in}{0.500000in}}%
\pgfpathclose%
\pgfusepath{fill}%
\end{pgfscope}%
\begin{pgfscope}%
\pgfpathrectangle{\pgfqpoint{0.750000in}{0.500000in}}{\pgfqpoint{4.650000in}{3.020000in}}%
\pgfusepath{clip}%
\pgfsetbuttcap%
\pgfsetmiterjoin%
\definecolor{currentfill}{rgb}{0.000000,0.500000,0.000000}%
\pgfsetfillcolor{currentfill}%
\pgfsetlinewidth{0.000000pt}%
\definecolor{currentstroke}{rgb}{0.000000,0.000000,0.000000}%
\pgfsetstrokecolor{currentstroke}%
\pgfsetstrokeopacity{0.000000}%
\pgfsetdash{}{0pt}%
\pgfpathmoveto{\pgfqpoint{1.776874in}{0.500000in}}%
\pgfpathlineto{\pgfqpoint{1.809494in}{0.500000in}}%
\pgfpathlineto{\pgfqpoint{1.809494in}{0.859890in}}%
\pgfpathlineto{\pgfqpoint{1.776874in}{0.859890in}}%
\pgfpathlineto{\pgfqpoint{1.776874in}{0.500000in}}%
\pgfpathclose%
\pgfusepath{fill}%
\end{pgfscope}%
\begin{pgfscope}%
\pgfpathrectangle{\pgfqpoint{0.750000in}{0.500000in}}{\pgfqpoint{4.650000in}{3.020000in}}%
\pgfusepath{clip}%
\pgfsetbuttcap%
\pgfsetmiterjoin%
\definecolor{currentfill}{rgb}{0.000000,0.500000,0.000000}%
\pgfsetfillcolor{currentfill}%
\pgfsetlinewidth{0.000000pt}%
\definecolor{currentstroke}{rgb}{0.000000,0.000000,0.000000}%
\pgfsetstrokecolor{currentstroke}%
\pgfsetstrokeopacity{0.000000}%
\pgfsetdash{}{0pt}%
\pgfpathmoveto{\pgfqpoint{1.809494in}{0.500000in}}%
\pgfpathlineto{\pgfqpoint{1.842114in}{0.500000in}}%
\pgfpathlineto{\pgfqpoint{1.842114in}{1.114446in}}%
\pgfpathlineto{\pgfqpoint{1.809494in}{1.114446in}}%
\pgfpathlineto{\pgfqpoint{1.809494in}{0.500000in}}%
\pgfpathclose%
\pgfusepath{fill}%
\end{pgfscope}%
\begin{pgfscope}%
\pgfpathrectangle{\pgfqpoint{0.750000in}{0.500000in}}{\pgfqpoint{4.650000in}{3.020000in}}%
\pgfusepath{clip}%
\pgfsetbuttcap%
\pgfsetmiterjoin%
\definecolor{currentfill}{rgb}{0.000000,0.500000,0.000000}%
\pgfsetfillcolor{currentfill}%
\pgfsetlinewidth{0.000000pt}%
\definecolor{currentstroke}{rgb}{0.000000,0.000000,0.000000}%
\pgfsetstrokecolor{currentstroke}%
\pgfsetstrokeopacity{0.000000}%
\pgfsetdash{}{0pt}%
\pgfpathmoveto{\pgfqpoint{1.842114in}{0.500000in}}%
\pgfpathlineto{\pgfqpoint{1.874735in}{0.500000in}}%
\pgfpathlineto{\pgfqpoint{1.874735in}{1.044223in}}%
\pgfpathlineto{\pgfqpoint{1.842114in}{1.044223in}}%
\pgfpathlineto{\pgfqpoint{1.842114in}{0.500000in}}%
\pgfpathclose%
\pgfusepath{fill}%
\end{pgfscope}%
\begin{pgfscope}%
\pgfpathrectangle{\pgfqpoint{0.750000in}{0.500000in}}{\pgfqpoint{4.650000in}{3.020000in}}%
\pgfusepath{clip}%
\pgfsetbuttcap%
\pgfsetmiterjoin%
\definecolor{currentfill}{rgb}{0.000000,0.500000,0.000000}%
\pgfsetfillcolor{currentfill}%
\pgfsetlinewidth{0.000000pt}%
\definecolor{currentstroke}{rgb}{0.000000,0.000000,0.000000}%
\pgfsetstrokecolor{currentstroke}%
\pgfsetstrokeopacity{0.000000}%
\pgfsetdash{}{0pt}%
\pgfpathmoveto{\pgfqpoint{1.874735in}{0.500000in}}%
\pgfpathlineto{\pgfqpoint{1.907355in}{0.500000in}}%
\pgfpathlineto{\pgfqpoint{1.907355in}{0.933038in}}%
\pgfpathlineto{\pgfqpoint{1.874735in}{0.933038in}}%
\pgfpathlineto{\pgfqpoint{1.874735in}{0.500000in}}%
\pgfpathclose%
\pgfusepath{fill}%
\end{pgfscope}%
\begin{pgfscope}%
\pgfpathrectangle{\pgfqpoint{0.750000in}{0.500000in}}{\pgfqpoint{4.650000in}{3.020000in}}%
\pgfusepath{clip}%
\pgfsetbuttcap%
\pgfsetmiterjoin%
\definecolor{currentfill}{rgb}{0.000000,0.500000,0.000000}%
\pgfsetfillcolor{currentfill}%
\pgfsetlinewidth{0.000000pt}%
\definecolor{currentstroke}{rgb}{0.000000,0.000000,0.000000}%
\pgfsetstrokecolor{currentstroke}%
\pgfsetstrokeopacity{0.000000}%
\pgfsetdash{}{0pt}%
\pgfpathmoveto{\pgfqpoint{1.907355in}{0.500000in}}%
\pgfpathlineto{\pgfqpoint{1.939976in}{0.500000in}}%
\pgfpathlineto{\pgfqpoint{1.939976in}{1.532854in}}%
\pgfpathlineto{\pgfqpoint{1.907355in}{1.532854in}}%
\pgfpathlineto{\pgfqpoint{1.907355in}{0.500000in}}%
\pgfpathclose%
\pgfusepath{fill}%
\end{pgfscope}%
\begin{pgfscope}%
\pgfpathrectangle{\pgfqpoint{0.750000in}{0.500000in}}{\pgfqpoint{4.650000in}{3.020000in}}%
\pgfusepath{clip}%
\pgfsetbuttcap%
\pgfsetmiterjoin%
\definecolor{currentfill}{rgb}{0.000000,0.500000,0.000000}%
\pgfsetfillcolor{currentfill}%
\pgfsetlinewidth{0.000000pt}%
\definecolor{currentstroke}{rgb}{0.000000,0.000000,0.000000}%
\pgfsetstrokecolor{currentstroke}%
\pgfsetstrokeopacity{0.000000}%
\pgfsetdash{}{0pt}%
\pgfpathmoveto{\pgfqpoint{1.939976in}{0.500000in}}%
\pgfpathlineto{\pgfqpoint{1.972596in}{0.500000in}}%
\pgfpathlineto{\pgfqpoint{1.972596in}{1.143705in}}%
\pgfpathlineto{\pgfqpoint{1.939976in}{1.143705in}}%
\pgfpathlineto{\pgfqpoint{1.939976in}{0.500000in}}%
\pgfpathclose%
\pgfusepath{fill}%
\end{pgfscope}%
\begin{pgfscope}%
\pgfpathrectangle{\pgfqpoint{0.750000in}{0.500000in}}{\pgfqpoint{4.650000in}{3.020000in}}%
\pgfusepath{clip}%
\pgfsetbuttcap%
\pgfsetmiterjoin%
\definecolor{currentfill}{rgb}{0.000000,0.500000,0.000000}%
\pgfsetfillcolor{currentfill}%
\pgfsetlinewidth{0.000000pt}%
\definecolor{currentstroke}{rgb}{0.000000,0.000000,0.000000}%
\pgfsetstrokecolor{currentstroke}%
\pgfsetstrokeopacity{0.000000}%
\pgfsetdash{}{0pt}%
\pgfpathmoveto{\pgfqpoint{1.972596in}{0.500000in}}%
\pgfpathlineto{\pgfqpoint{2.005216in}{0.500000in}}%
\pgfpathlineto{\pgfqpoint{2.005216in}{1.211001in}}%
\pgfpathlineto{\pgfqpoint{1.972596in}{1.211001in}}%
\pgfpathlineto{\pgfqpoint{1.972596in}{0.500000in}}%
\pgfpathclose%
\pgfusepath{fill}%
\end{pgfscope}%
\begin{pgfscope}%
\pgfpathrectangle{\pgfqpoint{0.750000in}{0.500000in}}{\pgfqpoint{4.650000in}{3.020000in}}%
\pgfusepath{clip}%
\pgfsetbuttcap%
\pgfsetmiterjoin%
\definecolor{currentfill}{rgb}{0.000000,0.500000,0.000000}%
\pgfsetfillcolor{currentfill}%
\pgfsetlinewidth{0.000000pt}%
\definecolor{currentstroke}{rgb}{0.000000,0.000000,0.000000}%
\pgfsetstrokecolor{currentstroke}%
\pgfsetstrokeopacity{0.000000}%
\pgfsetdash{}{0pt}%
\pgfpathmoveto{\pgfqpoint{2.005216in}{0.500000in}}%
\pgfpathlineto{\pgfqpoint{2.037837in}{0.500000in}}%
\pgfpathlineto{\pgfqpoint{2.037837in}{1.746447in}}%
\pgfpathlineto{\pgfqpoint{2.005216in}{1.746447in}}%
\pgfpathlineto{\pgfqpoint{2.005216in}{0.500000in}}%
\pgfpathclose%
\pgfusepath{fill}%
\end{pgfscope}%
\begin{pgfscope}%
\pgfpathrectangle{\pgfqpoint{0.750000in}{0.500000in}}{\pgfqpoint{4.650000in}{3.020000in}}%
\pgfusepath{clip}%
\pgfsetbuttcap%
\pgfsetmiterjoin%
\definecolor{currentfill}{rgb}{0.000000,0.500000,0.000000}%
\pgfsetfillcolor{currentfill}%
\pgfsetlinewidth{0.000000pt}%
\definecolor{currentstroke}{rgb}{0.000000,0.000000,0.000000}%
\pgfsetstrokecolor{currentstroke}%
\pgfsetstrokeopacity{0.000000}%
\pgfsetdash{}{0pt}%
\pgfpathmoveto{\pgfqpoint{2.037837in}{0.500000in}}%
\pgfpathlineto{\pgfqpoint{2.070457in}{0.500000in}}%
\pgfpathlineto{\pgfqpoint{2.070457in}{1.184668in}}%
\pgfpathlineto{\pgfqpoint{2.037837in}{1.184668in}}%
\pgfpathlineto{\pgfqpoint{2.037837in}{0.500000in}}%
\pgfpathclose%
\pgfusepath{fill}%
\end{pgfscope}%
\begin{pgfscope}%
\pgfpathrectangle{\pgfqpoint{0.750000in}{0.500000in}}{\pgfqpoint{4.650000in}{3.020000in}}%
\pgfusepath{clip}%
\pgfsetbuttcap%
\pgfsetmiterjoin%
\definecolor{currentfill}{rgb}{0.000000,0.500000,0.000000}%
\pgfsetfillcolor{currentfill}%
\pgfsetlinewidth{0.000000pt}%
\definecolor{currentstroke}{rgb}{0.000000,0.000000,0.000000}%
\pgfsetstrokecolor{currentstroke}%
\pgfsetstrokeopacity{0.000000}%
\pgfsetdash{}{0pt}%
\pgfpathmoveto{\pgfqpoint{2.070457in}{0.500000in}}%
\pgfpathlineto{\pgfqpoint{2.103077in}{0.500000in}}%
\pgfpathlineto{\pgfqpoint{2.103077in}{1.775706in}}%
\pgfpathlineto{\pgfqpoint{2.070457in}{1.775706in}}%
\pgfpathlineto{\pgfqpoint{2.070457in}{0.500000in}}%
\pgfpathclose%
\pgfusepath{fill}%
\end{pgfscope}%
\begin{pgfscope}%
\pgfpathrectangle{\pgfqpoint{0.750000in}{0.500000in}}{\pgfqpoint{4.650000in}{3.020000in}}%
\pgfusepath{clip}%
\pgfsetbuttcap%
\pgfsetmiterjoin%
\definecolor{currentfill}{rgb}{0.000000,0.500000,0.000000}%
\pgfsetfillcolor{currentfill}%
\pgfsetlinewidth{0.000000pt}%
\definecolor{currentstroke}{rgb}{0.000000,0.000000,0.000000}%
\pgfsetstrokecolor{currentstroke}%
\pgfsetstrokeopacity{0.000000}%
\pgfsetdash{}{0pt}%
\pgfpathmoveto{\pgfqpoint{2.103077in}{0.500000in}}%
\pgfpathlineto{\pgfqpoint{2.135698in}{0.500000in}}%
\pgfpathlineto{\pgfqpoint{2.135698in}{1.421668in}}%
\pgfpathlineto{\pgfqpoint{2.103077in}{1.421668in}}%
\pgfpathlineto{\pgfqpoint{2.103077in}{0.500000in}}%
\pgfpathclose%
\pgfusepath{fill}%
\end{pgfscope}%
\begin{pgfscope}%
\pgfpathrectangle{\pgfqpoint{0.750000in}{0.500000in}}{\pgfqpoint{4.650000in}{3.020000in}}%
\pgfusepath{clip}%
\pgfsetbuttcap%
\pgfsetmiterjoin%
\definecolor{currentfill}{rgb}{0.000000,0.500000,0.000000}%
\pgfsetfillcolor{currentfill}%
\pgfsetlinewidth{0.000000pt}%
\definecolor{currentstroke}{rgb}{0.000000,0.000000,0.000000}%
\pgfsetstrokecolor{currentstroke}%
\pgfsetstrokeopacity{0.000000}%
\pgfsetdash{}{0pt}%
\pgfpathmoveto{\pgfqpoint{2.135698in}{0.500000in}}%
\pgfpathlineto{\pgfqpoint{2.168318in}{0.500000in}}%
\pgfpathlineto{\pgfqpoint{2.168318in}{1.231483in}}%
\pgfpathlineto{\pgfqpoint{2.135698in}{1.231483in}}%
\pgfpathlineto{\pgfqpoint{2.135698in}{0.500000in}}%
\pgfpathclose%
\pgfusepath{fill}%
\end{pgfscope}%
\begin{pgfscope}%
\pgfpathrectangle{\pgfqpoint{0.750000in}{0.500000in}}{\pgfqpoint{4.650000in}{3.020000in}}%
\pgfusepath{clip}%
\pgfsetbuttcap%
\pgfsetmiterjoin%
\definecolor{currentfill}{rgb}{0.000000,0.500000,0.000000}%
\pgfsetfillcolor{currentfill}%
\pgfsetlinewidth{0.000000pt}%
\definecolor{currentstroke}{rgb}{0.000000,0.000000,0.000000}%
\pgfsetstrokecolor{currentstroke}%
\pgfsetstrokeopacity{0.000000}%
\pgfsetdash{}{0pt}%
\pgfpathmoveto{\pgfqpoint{2.168318in}{0.500000in}}%
\pgfpathlineto{\pgfqpoint{2.200939in}{0.500000in}}%
\pgfpathlineto{\pgfqpoint{2.200939in}{2.015632in}}%
\pgfpathlineto{\pgfqpoint{2.168318in}{2.015632in}}%
\pgfpathlineto{\pgfqpoint{2.168318in}{0.500000in}}%
\pgfpathclose%
\pgfusepath{fill}%
\end{pgfscope}%
\begin{pgfscope}%
\pgfpathrectangle{\pgfqpoint{0.750000in}{0.500000in}}{\pgfqpoint{4.650000in}{3.020000in}}%
\pgfusepath{clip}%
\pgfsetbuttcap%
\pgfsetmiterjoin%
\definecolor{currentfill}{rgb}{0.000000,0.500000,0.000000}%
\pgfsetfillcolor{currentfill}%
\pgfsetlinewidth{0.000000pt}%
\definecolor{currentstroke}{rgb}{0.000000,0.000000,0.000000}%
\pgfsetstrokecolor{currentstroke}%
\pgfsetstrokeopacity{0.000000}%
\pgfsetdash{}{0pt}%
\pgfpathmoveto{\pgfqpoint{2.200939in}{0.500000in}}%
\pgfpathlineto{\pgfqpoint{2.233559in}{0.500000in}}%
\pgfpathlineto{\pgfqpoint{2.233559in}{1.304631in}}%
\pgfpathlineto{\pgfqpoint{2.200939in}{1.304631in}}%
\pgfpathlineto{\pgfqpoint{2.200939in}{0.500000in}}%
\pgfpathclose%
\pgfusepath{fill}%
\end{pgfscope}%
\begin{pgfscope}%
\pgfpathrectangle{\pgfqpoint{0.750000in}{0.500000in}}{\pgfqpoint{4.650000in}{3.020000in}}%
\pgfusepath{clip}%
\pgfsetbuttcap%
\pgfsetmiterjoin%
\definecolor{currentfill}{rgb}{0.000000,0.500000,0.000000}%
\pgfsetfillcolor{currentfill}%
\pgfsetlinewidth{0.000000pt}%
\definecolor{currentstroke}{rgb}{0.000000,0.000000,0.000000}%
\pgfsetstrokecolor{currentstroke}%
\pgfsetstrokeopacity{0.000000}%
\pgfsetdash{}{0pt}%
\pgfpathmoveto{\pgfqpoint{2.233559in}{0.500000in}}%
\pgfpathlineto{\pgfqpoint{2.266179in}{0.500000in}}%
\pgfpathlineto{\pgfqpoint{2.266179in}{1.418742in}}%
\pgfpathlineto{\pgfqpoint{2.233559in}{1.418742in}}%
\pgfpathlineto{\pgfqpoint{2.233559in}{0.500000in}}%
\pgfpathclose%
\pgfusepath{fill}%
\end{pgfscope}%
\begin{pgfscope}%
\pgfpathrectangle{\pgfqpoint{0.750000in}{0.500000in}}{\pgfqpoint{4.650000in}{3.020000in}}%
\pgfusepath{clip}%
\pgfsetbuttcap%
\pgfsetmiterjoin%
\definecolor{currentfill}{rgb}{0.000000,0.500000,0.000000}%
\pgfsetfillcolor{currentfill}%
\pgfsetlinewidth{0.000000pt}%
\definecolor{currentstroke}{rgb}{0.000000,0.000000,0.000000}%
\pgfsetstrokecolor{currentstroke}%
\pgfsetstrokeopacity{0.000000}%
\pgfsetdash{}{0pt}%
\pgfpathmoveto{\pgfqpoint{2.266179in}{0.500000in}}%
\pgfpathlineto{\pgfqpoint{2.298800in}{0.500000in}}%
\pgfpathlineto{\pgfqpoint{2.298800in}{1.904447in}}%
\pgfpathlineto{\pgfqpoint{2.266179in}{1.904447in}}%
\pgfpathlineto{\pgfqpoint{2.266179in}{0.500000in}}%
\pgfpathclose%
\pgfusepath{fill}%
\end{pgfscope}%
\begin{pgfscope}%
\pgfpathrectangle{\pgfqpoint{0.750000in}{0.500000in}}{\pgfqpoint{4.650000in}{3.020000in}}%
\pgfusepath{clip}%
\pgfsetbuttcap%
\pgfsetmiterjoin%
\definecolor{currentfill}{rgb}{0.000000,0.500000,0.000000}%
\pgfsetfillcolor{currentfill}%
\pgfsetlinewidth{0.000000pt}%
\definecolor{currentstroke}{rgb}{0.000000,0.000000,0.000000}%
\pgfsetstrokecolor{currentstroke}%
\pgfsetstrokeopacity{0.000000}%
\pgfsetdash{}{0pt}%
\pgfpathmoveto{\pgfqpoint{2.298800in}{0.500000in}}%
\pgfpathlineto{\pgfqpoint{2.331420in}{0.500000in}}%
\pgfpathlineto{\pgfqpoint{2.331420in}{1.336816in}}%
\pgfpathlineto{\pgfqpoint{2.298800in}{1.336816in}}%
\pgfpathlineto{\pgfqpoint{2.298800in}{0.500000in}}%
\pgfpathclose%
\pgfusepath{fill}%
\end{pgfscope}%
\begin{pgfscope}%
\pgfpathrectangle{\pgfqpoint{0.750000in}{0.500000in}}{\pgfqpoint{4.650000in}{3.020000in}}%
\pgfusepath{clip}%
\pgfsetbuttcap%
\pgfsetmiterjoin%
\definecolor{currentfill}{rgb}{0.000000,0.500000,0.000000}%
\pgfsetfillcolor{currentfill}%
\pgfsetlinewidth{0.000000pt}%
\definecolor{currentstroke}{rgb}{0.000000,0.000000,0.000000}%
\pgfsetstrokecolor{currentstroke}%
\pgfsetstrokeopacity{0.000000}%
\pgfsetdash{}{0pt}%
\pgfpathmoveto{\pgfqpoint{2.331420in}{0.500000in}}%
\pgfpathlineto{\pgfqpoint{2.364041in}{0.500000in}}%
\pgfpathlineto{\pgfqpoint{2.364041in}{1.901521in}}%
\pgfpathlineto{\pgfqpoint{2.331420in}{1.901521in}}%
\pgfpathlineto{\pgfqpoint{2.331420in}{0.500000in}}%
\pgfpathclose%
\pgfusepath{fill}%
\end{pgfscope}%
\begin{pgfscope}%
\pgfpathrectangle{\pgfqpoint{0.750000in}{0.500000in}}{\pgfqpoint{4.650000in}{3.020000in}}%
\pgfusepath{clip}%
\pgfsetbuttcap%
\pgfsetmiterjoin%
\definecolor{currentfill}{rgb}{0.000000,0.500000,0.000000}%
\pgfsetfillcolor{currentfill}%
\pgfsetlinewidth{0.000000pt}%
\definecolor{currentstroke}{rgb}{0.000000,0.000000,0.000000}%
\pgfsetstrokecolor{currentstroke}%
\pgfsetstrokeopacity{0.000000}%
\pgfsetdash{}{0pt}%
\pgfpathmoveto{\pgfqpoint{2.364041in}{0.500000in}}%
\pgfpathlineto{\pgfqpoint{2.396661in}{0.500000in}}%
\pgfpathlineto{\pgfqpoint{2.396661in}{1.471409in}}%
\pgfpathlineto{\pgfqpoint{2.364041in}{1.471409in}}%
\pgfpathlineto{\pgfqpoint{2.364041in}{0.500000in}}%
\pgfpathclose%
\pgfusepath{fill}%
\end{pgfscope}%
\begin{pgfscope}%
\pgfpathrectangle{\pgfqpoint{0.750000in}{0.500000in}}{\pgfqpoint{4.650000in}{3.020000in}}%
\pgfusepath{clip}%
\pgfsetbuttcap%
\pgfsetmiterjoin%
\definecolor{currentfill}{rgb}{0.000000,0.500000,0.000000}%
\pgfsetfillcolor{currentfill}%
\pgfsetlinewidth{0.000000pt}%
\definecolor{currentstroke}{rgb}{0.000000,0.000000,0.000000}%
\pgfsetstrokecolor{currentstroke}%
\pgfsetstrokeopacity{0.000000}%
\pgfsetdash{}{0pt}%
\pgfpathmoveto{\pgfqpoint{2.396661in}{0.500000in}}%
\pgfpathlineto{\pgfqpoint{2.429281in}{0.500000in}}%
\pgfpathlineto{\pgfqpoint{2.429281in}{1.251964in}}%
\pgfpathlineto{\pgfqpoint{2.396661in}{1.251964in}}%
\pgfpathlineto{\pgfqpoint{2.396661in}{0.500000in}}%
\pgfpathclose%
\pgfusepath{fill}%
\end{pgfscope}%
\begin{pgfscope}%
\pgfpathrectangle{\pgfqpoint{0.750000in}{0.500000in}}{\pgfqpoint{4.650000in}{3.020000in}}%
\pgfusepath{clip}%
\pgfsetbuttcap%
\pgfsetmiterjoin%
\definecolor{currentfill}{rgb}{0.000000,0.500000,0.000000}%
\pgfsetfillcolor{currentfill}%
\pgfsetlinewidth{0.000000pt}%
\definecolor{currentstroke}{rgb}{0.000000,0.000000,0.000000}%
\pgfsetstrokecolor{currentstroke}%
\pgfsetstrokeopacity{0.000000}%
\pgfsetdash{}{0pt}%
\pgfpathmoveto{\pgfqpoint{2.429281in}{0.500000in}}%
\pgfpathlineto{\pgfqpoint{2.461902in}{0.500000in}}%
\pgfpathlineto{\pgfqpoint{2.461902in}{1.723039in}}%
\pgfpathlineto{\pgfqpoint{2.429281in}{1.723039in}}%
\pgfpathlineto{\pgfqpoint{2.429281in}{0.500000in}}%
\pgfpathclose%
\pgfusepath{fill}%
\end{pgfscope}%
\begin{pgfscope}%
\pgfpathrectangle{\pgfqpoint{0.750000in}{0.500000in}}{\pgfqpoint{4.650000in}{3.020000in}}%
\pgfusepath{clip}%
\pgfsetbuttcap%
\pgfsetmiterjoin%
\definecolor{currentfill}{rgb}{0.000000,0.500000,0.000000}%
\pgfsetfillcolor{currentfill}%
\pgfsetlinewidth{0.000000pt}%
\definecolor{currentstroke}{rgb}{0.000000,0.000000,0.000000}%
\pgfsetstrokecolor{currentstroke}%
\pgfsetstrokeopacity{0.000000}%
\pgfsetdash{}{0pt}%
\pgfpathmoveto{\pgfqpoint{2.461902in}{0.500000in}}%
\pgfpathlineto{\pgfqpoint{2.494522in}{0.500000in}}%
\pgfpathlineto{\pgfqpoint{2.494522in}{1.228557in}}%
\pgfpathlineto{\pgfqpoint{2.461902in}{1.228557in}}%
\pgfpathlineto{\pgfqpoint{2.461902in}{0.500000in}}%
\pgfpathclose%
\pgfusepath{fill}%
\end{pgfscope}%
\begin{pgfscope}%
\pgfpathrectangle{\pgfqpoint{0.750000in}{0.500000in}}{\pgfqpoint{4.650000in}{3.020000in}}%
\pgfusepath{clip}%
\pgfsetbuttcap%
\pgfsetmiterjoin%
\definecolor{currentfill}{rgb}{0.000000,0.500000,0.000000}%
\pgfsetfillcolor{currentfill}%
\pgfsetlinewidth{0.000000pt}%
\definecolor{currentstroke}{rgb}{0.000000,0.000000,0.000000}%
\pgfsetstrokecolor{currentstroke}%
\pgfsetstrokeopacity{0.000000}%
\pgfsetdash{}{0pt}%
\pgfpathmoveto{\pgfqpoint{2.494522in}{0.500000in}}%
\pgfpathlineto{\pgfqpoint{2.527143in}{0.500000in}}%
\pgfpathlineto{\pgfqpoint{2.527143in}{1.190520in}}%
\pgfpathlineto{\pgfqpoint{2.494522in}{1.190520in}}%
\pgfpathlineto{\pgfqpoint{2.494522in}{0.500000in}}%
\pgfpathclose%
\pgfusepath{fill}%
\end{pgfscope}%
\begin{pgfscope}%
\pgfpathrectangle{\pgfqpoint{0.750000in}{0.500000in}}{\pgfqpoint{4.650000in}{3.020000in}}%
\pgfusepath{clip}%
\pgfsetbuttcap%
\pgfsetmiterjoin%
\definecolor{currentfill}{rgb}{0.000000,0.500000,0.000000}%
\pgfsetfillcolor{currentfill}%
\pgfsetlinewidth{0.000000pt}%
\definecolor{currentstroke}{rgb}{0.000000,0.000000,0.000000}%
\pgfsetstrokecolor{currentstroke}%
\pgfsetstrokeopacity{0.000000}%
\pgfsetdash{}{0pt}%
\pgfpathmoveto{\pgfqpoint{2.527143in}{0.500000in}}%
\pgfpathlineto{\pgfqpoint{2.559763in}{0.500000in}}%
\pgfpathlineto{\pgfqpoint{2.559763in}{1.509446in}}%
\pgfpathlineto{\pgfqpoint{2.527143in}{1.509446in}}%
\pgfpathlineto{\pgfqpoint{2.527143in}{0.500000in}}%
\pgfpathclose%
\pgfusepath{fill}%
\end{pgfscope}%
\begin{pgfscope}%
\pgfpathrectangle{\pgfqpoint{0.750000in}{0.500000in}}{\pgfqpoint{4.650000in}{3.020000in}}%
\pgfusepath{clip}%
\pgfsetbuttcap%
\pgfsetmiterjoin%
\definecolor{currentfill}{rgb}{0.000000,0.500000,0.000000}%
\pgfsetfillcolor{currentfill}%
\pgfsetlinewidth{0.000000pt}%
\definecolor{currentstroke}{rgb}{0.000000,0.000000,0.000000}%
\pgfsetstrokecolor{currentstroke}%
\pgfsetstrokeopacity{0.000000}%
\pgfsetdash{}{0pt}%
\pgfpathmoveto{\pgfqpoint{2.559763in}{0.500000in}}%
\pgfpathlineto{\pgfqpoint{2.592383in}{0.500000in}}%
\pgfpathlineto{\pgfqpoint{2.592383in}{1.047149in}}%
\pgfpathlineto{\pgfqpoint{2.559763in}{1.047149in}}%
\pgfpathlineto{\pgfqpoint{2.559763in}{0.500000in}}%
\pgfpathclose%
\pgfusepath{fill}%
\end{pgfscope}%
\begin{pgfscope}%
\pgfpathrectangle{\pgfqpoint{0.750000in}{0.500000in}}{\pgfqpoint{4.650000in}{3.020000in}}%
\pgfusepath{clip}%
\pgfsetbuttcap%
\pgfsetmiterjoin%
\definecolor{currentfill}{rgb}{0.000000,0.500000,0.000000}%
\pgfsetfillcolor{currentfill}%
\pgfsetlinewidth{0.000000pt}%
\definecolor{currentstroke}{rgb}{0.000000,0.000000,0.000000}%
\pgfsetstrokecolor{currentstroke}%
\pgfsetstrokeopacity{0.000000}%
\pgfsetdash{}{0pt}%
\pgfpathmoveto{\pgfqpoint{2.592383in}{0.500000in}}%
\pgfpathlineto{\pgfqpoint{2.625004in}{0.500000in}}%
\pgfpathlineto{\pgfqpoint{2.625004in}{1.234409in}}%
\pgfpathlineto{\pgfqpoint{2.592383in}{1.234409in}}%
\pgfpathlineto{\pgfqpoint{2.592383in}{0.500000in}}%
\pgfpathclose%
\pgfusepath{fill}%
\end{pgfscope}%
\begin{pgfscope}%
\pgfpathrectangle{\pgfqpoint{0.750000in}{0.500000in}}{\pgfqpoint{4.650000in}{3.020000in}}%
\pgfusepath{clip}%
\pgfsetbuttcap%
\pgfsetmiterjoin%
\definecolor{currentfill}{rgb}{0.000000,0.500000,0.000000}%
\pgfsetfillcolor{currentfill}%
\pgfsetlinewidth{0.000000pt}%
\definecolor{currentstroke}{rgb}{0.000000,0.000000,0.000000}%
\pgfsetstrokecolor{currentstroke}%
\pgfsetstrokeopacity{0.000000}%
\pgfsetdash{}{0pt}%
\pgfpathmoveto{\pgfqpoint{2.625004in}{0.500000in}}%
\pgfpathlineto{\pgfqpoint{2.657624in}{0.500000in}}%
\pgfpathlineto{\pgfqpoint{2.657624in}{1.053001in}}%
\pgfpathlineto{\pgfqpoint{2.625004in}{1.053001in}}%
\pgfpathlineto{\pgfqpoint{2.625004in}{0.500000in}}%
\pgfpathclose%
\pgfusepath{fill}%
\end{pgfscope}%
\begin{pgfscope}%
\pgfpathrectangle{\pgfqpoint{0.750000in}{0.500000in}}{\pgfqpoint{4.650000in}{3.020000in}}%
\pgfusepath{clip}%
\pgfsetbuttcap%
\pgfsetmiterjoin%
\definecolor{currentfill}{rgb}{0.000000,0.500000,0.000000}%
\pgfsetfillcolor{currentfill}%
\pgfsetlinewidth{0.000000pt}%
\definecolor{currentstroke}{rgb}{0.000000,0.000000,0.000000}%
\pgfsetstrokecolor{currentstroke}%
\pgfsetstrokeopacity{0.000000}%
\pgfsetdash{}{0pt}%
\pgfpathmoveto{\pgfqpoint{2.657624in}{0.500000in}}%
\pgfpathlineto{\pgfqpoint{2.690245in}{0.500000in}}%
\pgfpathlineto{\pgfqpoint{2.690245in}{0.839408in}}%
\pgfpathlineto{\pgfqpoint{2.657624in}{0.839408in}}%
\pgfpathlineto{\pgfqpoint{2.657624in}{0.500000in}}%
\pgfpathclose%
\pgfusepath{fill}%
\end{pgfscope}%
\begin{pgfscope}%
\pgfpathrectangle{\pgfqpoint{0.750000in}{0.500000in}}{\pgfqpoint{4.650000in}{3.020000in}}%
\pgfusepath{clip}%
\pgfsetbuttcap%
\pgfsetmiterjoin%
\definecolor{currentfill}{rgb}{0.000000,0.500000,0.000000}%
\pgfsetfillcolor{currentfill}%
\pgfsetlinewidth{0.000000pt}%
\definecolor{currentstroke}{rgb}{0.000000,0.000000,0.000000}%
\pgfsetstrokecolor{currentstroke}%
\pgfsetstrokeopacity{0.000000}%
\pgfsetdash{}{0pt}%
\pgfpathmoveto{\pgfqpoint{2.690245in}{0.500000in}}%
\pgfpathlineto{\pgfqpoint{2.722865in}{0.500000in}}%
\pgfpathlineto{\pgfqpoint{2.722865in}{1.143705in}}%
\pgfpathlineto{\pgfqpoint{2.690245in}{1.143705in}}%
\pgfpathlineto{\pgfqpoint{2.690245in}{0.500000in}}%
\pgfpathclose%
\pgfusepath{fill}%
\end{pgfscope}%
\begin{pgfscope}%
\pgfpathrectangle{\pgfqpoint{0.750000in}{0.500000in}}{\pgfqpoint{4.650000in}{3.020000in}}%
\pgfusepath{clip}%
\pgfsetbuttcap%
\pgfsetmiterjoin%
\definecolor{currentfill}{rgb}{0.000000,0.500000,0.000000}%
\pgfsetfillcolor{currentfill}%
\pgfsetlinewidth{0.000000pt}%
\definecolor{currentstroke}{rgb}{0.000000,0.000000,0.000000}%
\pgfsetstrokecolor{currentstroke}%
\pgfsetstrokeopacity{0.000000}%
\pgfsetdash{}{0pt}%
\pgfpathmoveto{\pgfqpoint{2.722865in}{0.500000in}}%
\pgfpathlineto{\pgfqpoint{2.755485in}{0.500000in}}%
\pgfpathlineto{\pgfqpoint{2.755485in}{0.821852in}}%
\pgfpathlineto{\pgfqpoint{2.722865in}{0.821852in}}%
\pgfpathlineto{\pgfqpoint{2.722865in}{0.500000in}}%
\pgfpathclose%
\pgfusepath{fill}%
\end{pgfscope}%
\begin{pgfscope}%
\pgfpathrectangle{\pgfqpoint{0.750000in}{0.500000in}}{\pgfqpoint{4.650000in}{3.020000in}}%
\pgfusepath{clip}%
\pgfsetbuttcap%
\pgfsetmiterjoin%
\definecolor{currentfill}{rgb}{0.000000,0.500000,0.000000}%
\pgfsetfillcolor{currentfill}%
\pgfsetlinewidth{0.000000pt}%
\definecolor{currentstroke}{rgb}{0.000000,0.000000,0.000000}%
\pgfsetstrokecolor{currentstroke}%
\pgfsetstrokeopacity{0.000000}%
\pgfsetdash{}{0pt}%
\pgfpathmoveto{\pgfqpoint{2.755485in}{0.500000in}}%
\pgfpathlineto{\pgfqpoint{2.788106in}{0.500000in}}%
\pgfpathlineto{\pgfqpoint{2.788106in}{0.862815in}}%
\pgfpathlineto{\pgfqpoint{2.755485in}{0.862815in}}%
\pgfpathlineto{\pgfqpoint{2.755485in}{0.500000in}}%
\pgfpathclose%
\pgfusepath{fill}%
\end{pgfscope}%
\begin{pgfscope}%
\pgfpathrectangle{\pgfqpoint{0.750000in}{0.500000in}}{\pgfqpoint{4.650000in}{3.020000in}}%
\pgfusepath{clip}%
\pgfsetbuttcap%
\pgfsetmiterjoin%
\definecolor{currentfill}{rgb}{0.000000,0.500000,0.000000}%
\pgfsetfillcolor{currentfill}%
\pgfsetlinewidth{0.000000pt}%
\definecolor{currentstroke}{rgb}{0.000000,0.000000,0.000000}%
\pgfsetstrokecolor{currentstroke}%
\pgfsetstrokeopacity{0.000000}%
\pgfsetdash{}{0pt}%
\pgfpathmoveto{\pgfqpoint{2.788106in}{0.500000in}}%
\pgfpathlineto{\pgfqpoint{2.820726in}{0.500000in}}%
\pgfpathlineto{\pgfqpoint{2.820726in}{0.900853in}}%
\pgfpathlineto{\pgfqpoint{2.788106in}{0.900853in}}%
\pgfpathlineto{\pgfqpoint{2.788106in}{0.500000in}}%
\pgfpathclose%
\pgfusepath{fill}%
\end{pgfscope}%
\begin{pgfscope}%
\pgfpathrectangle{\pgfqpoint{0.750000in}{0.500000in}}{\pgfqpoint{4.650000in}{3.020000in}}%
\pgfusepath{clip}%
\pgfsetbuttcap%
\pgfsetmiterjoin%
\definecolor{currentfill}{rgb}{0.000000,0.500000,0.000000}%
\pgfsetfillcolor{currentfill}%
\pgfsetlinewidth{0.000000pt}%
\definecolor{currentstroke}{rgb}{0.000000,0.000000,0.000000}%
\pgfsetstrokecolor{currentstroke}%
\pgfsetstrokeopacity{0.000000}%
\pgfsetdash{}{0pt}%
\pgfpathmoveto{\pgfqpoint{2.820726in}{0.500000in}}%
\pgfpathlineto{\pgfqpoint{2.853347in}{0.500000in}}%
\pgfpathlineto{\pgfqpoint{2.853347in}{0.728223in}}%
\pgfpathlineto{\pgfqpoint{2.820726in}{0.728223in}}%
\pgfpathlineto{\pgfqpoint{2.820726in}{0.500000in}}%
\pgfpathclose%
\pgfusepath{fill}%
\end{pgfscope}%
\begin{pgfscope}%
\pgfpathrectangle{\pgfqpoint{0.750000in}{0.500000in}}{\pgfqpoint{4.650000in}{3.020000in}}%
\pgfusepath{clip}%
\pgfsetbuttcap%
\pgfsetmiterjoin%
\definecolor{currentfill}{rgb}{0.000000,0.500000,0.000000}%
\pgfsetfillcolor{currentfill}%
\pgfsetlinewidth{0.000000pt}%
\definecolor{currentstroke}{rgb}{0.000000,0.000000,0.000000}%
\pgfsetstrokecolor{currentstroke}%
\pgfsetstrokeopacity{0.000000}%
\pgfsetdash{}{0pt}%
\pgfpathmoveto{\pgfqpoint{2.853347in}{0.500000in}}%
\pgfpathlineto{\pgfqpoint{2.885967in}{0.500000in}}%
\pgfpathlineto{\pgfqpoint{2.885967in}{0.763334in}}%
\pgfpathlineto{\pgfqpoint{2.853347in}{0.763334in}}%
\pgfpathlineto{\pgfqpoint{2.853347in}{0.500000in}}%
\pgfpathclose%
\pgfusepath{fill}%
\end{pgfscope}%
\begin{pgfscope}%
\pgfpathrectangle{\pgfqpoint{0.750000in}{0.500000in}}{\pgfqpoint{4.650000in}{3.020000in}}%
\pgfusepath{clip}%
\pgfsetbuttcap%
\pgfsetmiterjoin%
\definecolor{currentfill}{rgb}{0.000000,0.500000,0.000000}%
\pgfsetfillcolor{currentfill}%
\pgfsetlinewidth{0.000000pt}%
\definecolor{currentstroke}{rgb}{0.000000,0.000000,0.000000}%
\pgfsetstrokecolor{currentstroke}%
\pgfsetstrokeopacity{0.000000}%
\pgfsetdash{}{0pt}%
\pgfpathmoveto{\pgfqpoint{2.885967in}{0.500000in}}%
\pgfpathlineto{\pgfqpoint{2.918587in}{0.500000in}}%
\pgfpathlineto{\pgfqpoint{2.918587in}{0.649222in}}%
\pgfpathlineto{\pgfqpoint{2.885967in}{0.649222in}}%
\pgfpathlineto{\pgfqpoint{2.885967in}{0.500000in}}%
\pgfpathclose%
\pgfusepath{fill}%
\end{pgfscope}%
\begin{pgfscope}%
\pgfpathrectangle{\pgfqpoint{0.750000in}{0.500000in}}{\pgfqpoint{4.650000in}{3.020000in}}%
\pgfusepath{clip}%
\pgfsetbuttcap%
\pgfsetmiterjoin%
\definecolor{currentfill}{rgb}{0.000000,0.500000,0.000000}%
\pgfsetfillcolor{currentfill}%
\pgfsetlinewidth{0.000000pt}%
\definecolor{currentstroke}{rgb}{0.000000,0.000000,0.000000}%
\pgfsetstrokecolor{currentstroke}%
\pgfsetstrokeopacity{0.000000}%
\pgfsetdash{}{0pt}%
\pgfpathmoveto{\pgfqpoint{2.918587in}{0.500000in}}%
\pgfpathlineto{\pgfqpoint{2.951208in}{0.500000in}}%
\pgfpathlineto{\pgfqpoint{2.951208in}{0.590704in}}%
\pgfpathlineto{\pgfqpoint{2.918587in}{0.590704in}}%
\pgfpathlineto{\pgfqpoint{2.918587in}{0.500000in}}%
\pgfpathclose%
\pgfusepath{fill}%
\end{pgfscope}%
\begin{pgfscope}%
\pgfpathrectangle{\pgfqpoint{0.750000in}{0.500000in}}{\pgfqpoint{4.650000in}{3.020000in}}%
\pgfusepath{clip}%
\pgfsetbuttcap%
\pgfsetmiterjoin%
\definecolor{currentfill}{rgb}{0.000000,0.500000,0.000000}%
\pgfsetfillcolor{currentfill}%
\pgfsetlinewidth{0.000000pt}%
\definecolor{currentstroke}{rgb}{0.000000,0.000000,0.000000}%
\pgfsetstrokecolor{currentstroke}%
\pgfsetstrokeopacity{0.000000}%
\pgfsetdash{}{0pt}%
\pgfpathmoveto{\pgfqpoint{2.951208in}{0.500000in}}%
\pgfpathlineto{\pgfqpoint{2.983828in}{0.500000in}}%
\pgfpathlineto{\pgfqpoint{2.983828in}{0.666778in}}%
\pgfpathlineto{\pgfqpoint{2.951208in}{0.666778in}}%
\pgfpathlineto{\pgfqpoint{2.951208in}{0.500000in}}%
\pgfpathclose%
\pgfusepath{fill}%
\end{pgfscope}%
\begin{pgfscope}%
\pgfpathrectangle{\pgfqpoint{0.750000in}{0.500000in}}{\pgfqpoint{4.650000in}{3.020000in}}%
\pgfusepath{clip}%
\pgfsetbuttcap%
\pgfsetmiterjoin%
\definecolor{currentfill}{rgb}{0.000000,0.500000,0.000000}%
\pgfsetfillcolor{currentfill}%
\pgfsetlinewidth{0.000000pt}%
\definecolor{currentstroke}{rgb}{0.000000,0.000000,0.000000}%
\pgfsetstrokecolor{currentstroke}%
\pgfsetstrokeopacity{0.000000}%
\pgfsetdash{}{0pt}%
\pgfpathmoveto{\pgfqpoint{2.983828in}{0.500000in}}%
\pgfpathlineto{\pgfqpoint{3.016449in}{0.500000in}}%
\pgfpathlineto{\pgfqpoint{3.016449in}{0.581926in}}%
\pgfpathlineto{\pgfqpoint{2.983828in}{0.581926in}}%
\pgfpathlineto{\pgfqpoint{2.983828in}{0.500000in}}%
\pgfpathclose%
\pgfusepath{fill}%
\end{pgfscope}%
\begin{pgfscope}%
\pgfpathrectangle{\pgfqpoint{0.750000in}{0.500000in}}{\pgfqpoint{4.650000in}{3.020000in}}%
\pgfusepath{clip}%
\pgfsetbuttcap%
\pgfsetmiterjoin%
\definecolor{currentfill}{rgb}{0.000000,0.500000,0.000000}%
\pgfsetfillcolor{currentfill}%
\pgfsetlinewidth{0.000000pt}%
\definecolor{currentstroke}{rgb}{0.000000,0.000000,0.000000}%
\pgfsetstrokecolor{currentstroke}%
\pgfsetstrokeopacity{0.000000}%
\pgfsetdash{}{0pt}%
\pgfpathmoveto{\pgfqpoint{3.016449in}{0.500000in}}%
\pgfpathlineto{\pgfqpoint{3.049069in}{0.500000in}}%
\pgfpathlineto{\pgfqpoint{3.049069in}{0.593630in}}%
\pgfpathlineto{\pgfqpoint{3.016449in}{0.593630in}}%
\pgfpathlineto{\pgfqpoint{3.016449in}{0.500000in}}%
\pgfpathclose%
\pgfusepath{fill}%
\end{pgfscope}%
\begin{pgfscope}%
\pgfpathrectangle{\pgfqpoint{0.750000in}{0.500000in}}{\pgfqpoint{4.650000in}{3.020000in}}%
\pgfusepath{clip}%
\pgfsetbuttcap%
\pgfsetmiterjoin%
\definecolor{currentfill}{rgb}{0.000000,0.500000,0.000000}%
\pgfsetfillcolor{currentfill}%
\pgfsetlinewidth{0.000000pt}%
\definecolor{currentstroke}{rgb}{0.000000,0.000000,0.000000}%
\pgfsetstrokecolor{currentstroke}%
\pgfsetstrokeopacity{0.000000}%
\pgfsetdash{}{0pt}%
\pgfpathmoveto{\pgfqpoint{3.049069in}{0.500000in}}%
\pgfpathlineto{\pgfqpoint{3.081689in}{0.500000in}}%
\pgfpathlineto{\pgfqpoint{3.081689in}{0.581926in}}%
\pgfpathlineto{\pgfqpoint{3.049069in}{0.581926in}}%
\pgfpathlineto{\pgfqpoint{3.049069in}{0.500000in}}%
\pgfpathclose%
\pgfusepath{fill}%
\end{pgfscope}%
\begin{pgfscope}%
\pgfpathrectangle{\pgfqpoint{0.750000in}{0.500000in}}{\pgfqpoint{4.650000in}{3.020000in}}%
\pgfusepath{clip}%
\pgfsetbuttcap%
\pgfsetmiterjoin%
\definecolor{currentfill}{rgb}{0.000000,0.500000,0.000000}%
\pgfsetfillcolor{currentfill}%
\pgfsetlinewidth{0.000000pt}%
\definecolor{currentstroke}{rgb}{0.000000,0.000000,0.000000}%
\pgfsetstrokecolor{currentstroke}%
\pgfsetstrokeopacity{0.000000}%
\pgfsetdash{}{0pt}%
\pgfpathmoveto{\pgfqpoint{3.081689in}{0.500000in}}%
\pgfpathlineto{\pgfqpoint{3.114310in}{0.500000in}}%
\pgfpathlineto{\pgfqpoint{3.114310in}{0.540963in}}%
\pgfpathlineto{\pgfqpoint{3.081689in}{0.540963in}}%
\pgfpathlineto{\pgfqpoint{3.081689in}{0.500000in}}%
\pgfpathclose%
\pgfusepath{fill}%
\end{pgfscope}%
\begin{pgfscope}%
\pgfpathrectangle{\pgfqpoint{0.750000in}{0.500000in}}{\pgfqpoint{4.650000in}{3.020000in}}%
\pgfusepath{clip}%
\pgfsetbuttcap%
\pgfsetmiterjoin%
\definecolor{currentfill}{rgb}{0.000000,0.500000,0.000000}%
\pgfsetfillcolor{currentfill}%
\pgfsetlinewidth{0.000000pt}%
\definecolor{currentstroke}{rgb}{0.000000,0.000000,0.000000}%
\pgfsetstrokecolor{currentstroke}%
\pgfsetstrokeopacity{0.000000}%
\pgfsetdash{}{0pt}%
\pgfpathmoveto{\pgfqpoint{3.114310in}{0.500000in}}%
\pgfpathlineto{\pgfqpoint{3.146930in}{0.500000in}}%
\pgfpathlineto{\pgfqpoint{3.146930in}{0.561445in}}%
\pgfpathlineto{\pgfqpoint{3.114310in}{0.561445in}}%
\pgfpathlineto{\pgfqpoint{3.114310in}{0.500000in}}%
\pgfpathclose%
\pgfusepath{fill}%
\end{pgfscope}%
\begin{pgfscope}%
\pgfpathrectangle{\pgfqpoint{0.750000in}{0.500000in}}{\pgfqpoint{4.650000in}{3.020000in}}%
\pgfusepath{clip}%
\pgfsetbuttcap%
\pgfsetmiterjoin%
\definecolor{currentfill}{rgb}{0.000000,0.500000,0.000000}%
\pgfsetfillcolor{currentfill}%
\pgfsetlinewidth{0.000000pt}%
\definecolor{currentstroke}{rgb}{0.000000,0.000000,0.000000}%
\pgfsetstrokecolor{currentstroke}%
\pgfsetstrokeopacity{0.000000}%
\pgfsetdash{}{0pt}%
\pgfpathmoveto{\pgfqpoint{3.146930in}{0.500000in}}%
\pgfpathlineto{\pgfqpoint{3.179551in}{0.500000in}}%
\pgfpathlineto{\pgfqpoint{3.179551in}{0.532185in}}%
\pgfpathlineto{\pgfqpoint{3.146930in}{0.532185in}}%
\pgfpathlineto{\pgfqpoint{3.146930in}{0.500000in}}%
\pgfpathclose%
\pgfusepath{fill}%
\end{pgfscope}%
\begin{pgfscope}%
\pgfpathrectangle{\pgfqpoint{0.750000in}{0.500000in}}{\pgfqpoint{4.650000in}{3.020000in}}%
\pgfusepath{clip}%
\pgfsetbuttcap%
\pgfsetmiterjoin%
\definecolor{currentfill}{rgb}{0.000000,0.500000,0.000000}%
\pgfsetfillcolor{currentfill}%
\pgfsetlinewidth{0.000000pt}%
\definecolor{currentstroke}{rgb}{0.000000,0.000000,0.000000}%
\pgfsetstrokecolor{currentstroke}%
\pgfsetstrokeopacity{0.000000}%
\pgfsetdash{}{0pt}%
\pgfpathmoveto{\pgfqpoint{3.179551in}{0.500000in}}%
\pgfpathlineto{\pgfqpoint{3.212171in}{0.500000in}}%
\pgfpathlineto{\pgfqpoint{3.212171in}{0.517556in}}%
\pgfpathlineto{\pgfqpoint{3.179551in}{0.517556in}}%
\pgfpathlineto{\pgfqpoint{3.179551in}{0.500000in}}%
\pgfpathclose%
\pgfusepath{fill}%
\end{pgfscope}%
\begin{pgfscope}%
\pgfpathrectangle{\pgfqpoint{0.750000in}{0.500000in}}{\pgfqpoint{4.650000in}{3.020000in}}%
\pgfusepath{clip}%
\pgfsetbuttcap%
\pgfsetmiterjoin%
\definecolor{currentfill}{rgb}{0.000000,0.500000,0.000000}%
\pgfsetfillcolor{currentfill}%
\pgfsetlinewidth{0.000000pt}%
\definecolor{currentstroke}{rgb}{0.000000,0.000000,0.000000}%
\pgfsetstrokecolor{currentstroke}%
\pgfsetstrokeopacity{0.000000}%
\pgfsetdash{}{0pt}%
\pgfpathmoveto{\pgfqpoint{3.212171in}{0.500000in}}%
\pgfpathlineto{\pgfqpoint{3.244791in}{0.500000in}}%
\pgfpathlineto{\pgfqpoint{3.244791in}{0.526333in}}%
\pgfpathlineto{\pgfqpoint{3.212171in}{0.526333in}}%
\pgfpathlineto{\pgfqpoint{3.212171in}{0.500000in}}%
\pgfpathclose%
\pgfusepath{fill}%
\end{pgfscope}%
\begin{pgfscope}%
\pgfpathrectangle{\pgfqpoint{0.750000in}{0.500000in}}{\pgfqpoint{4.650000in}{3.020000in}}%
\pgfusepath{clip}%
\pgfsetbuttcap%
\pgfsetmiterjoin%
\definecolor{currentfill}{rgb}{0.000000,0.500000,0.000000}%
\pgfsetfillcolor{currentfill}%
\pgfsetlinewidth{0.000000pt}%
\definecolor{currentstroke}{rgb}{0.000000,0.000000,0.000000}%
\pgfsetstrokecolor{currentstroke}%
\pgfsetstrokeopacity{0.000000}%
\pgfsetdash{}{0pt}%
\pgfpathmoveto{\pgfqpoint{3.244791in}{0.500000in}}%
\pgfpathlineto{\pgfqpoint{3.277412in}{0.500000in}}%
\pgfpathlineto{\pgfqpoint{3.277412in}{0.502926in}}%
\pgfpathlineto{\pgfqpoint{3.244791in}{0.502926in}}%
\pgfpathlineto{\pgfqpoint{3.244791in}{0.500000in}}%
\pgfpathclose%
\pgfusepath{fill}%
\end{pgfscope}%
\begin{pgfscope}%
\pgfpathrectangle{\pgfqpoint{0.750000in}{0.500000in}}{\pgfqpoint{4.650000in}{3.020000in}}%
\pgfusepath{clip}%
\pgfsetbuttcap%
\pgfsetmiterjoin%
\definecolor{currentfill}{rgb}{0.000000,0.500000,0.000000}%
\pgfsetfillcolor{currentfill}%
\pgfsetlinewidth{0.000000pt}%
\definecolor{currentstroke}{rgb}{0.000000,0.000000,0.000000}%
\pgfsetstrokecolor{currentstroke}%
\pgfsetstrokeopacity{0.000000}%
\pgfsetdash{}{0pt}%
\pgfpathmoveto{\pgfqpoint{3.277412in}{0.500000in}}%
\pgfpathlineto{\pgfqpoint{3.310032in}{0.500000in}}%
\pgfpathlineto{\pgfqpoint{3.310032in}{0.514630in}}%
\pgfpathlineto{\pgfqpoint{3.277412in}{0.514630in}}%
\pgfpathlineto{\pgfqpoint{3.277412in}{0.500000in}}%
\pgfpathclose%
\pgfusepath{fill}%
\end{pgfscope}%
\begin{pgfscope}%
\pgfpathrectangle{\pgfqpoint{0.750000in}{0.500000in}}{\pgfqpoint{4.650000in}{3.020000in}}%
\pgfusepath{clip}%
\pgfsetbuttcap%
\pgfsetmiterjoin%
\definecolor{currentfill}{rgb}{0.000000,0.500000,0.000000}%
\pgfsetfillcolor{currentfill}%
\pgfsetlinewidth{0.000000pt}%
\definecolor{currentstroke}{rgb}{0.000000,0.000000,0.000000}%
\pgfsetstrokecolor{currentstroke}%
\pgfsetstrokeopacity{0.000000}%
\pgfsetdash{}{0pt}%
\pgfpathmoveto{\pgfqpoint{3.310032in}{0.500000in}}%
\pgfpathlineto{\pgfqpoint{3.342653in}{0.500000in}}%
\pgfpathlineto{\pgfqpoint{3.342653in}{0.511704in}}%
\pgfpathlineto{\pgfqpoint{3.310032in}{0.511704in}}%
\pgfpathlineto{\pgfqpoint{3.310032in}{0.500000in}}%
\pgfpathclose%
\pgfusepath{fill}%
\end{pgfscope}%
\begin{pgfscope}%
\pgfpathrectangle{\pgfqpoint{0.750000in}{0.500000in}}{\pgfqpoint{4.650000in}{3.020000in}}%
\pgfusepath{clip}%
\pgfsetbuttcap%
\pgfsetmiterjoin%
\definecolor{currentfill}{rgb}{0.000000,0.500000,0.000000}%
\pgfsetfillcolor{currentfill}%
\pgfsetlinewidth{0.000000pt}%
\definecolor{currentstroke}{rgb}{0.000000,0.000000,0.000000}%
\pgfsetstrokecolor{currentstroke}%
\pgfsetstrokeopacity{0.000000}%
\pgfsetdash{}{0pt}%
\pgfpathmoveto{\pgfqpoint{3.342653in}{0.500000in}}%
\pgfpathlineto{\pgfqpoint{3.375273in}{0.500000in}}%
\pgfpathlineto{\pgfqpoint{3.375273in}{0.502926in}}%
\pgfpathlineto{\pgfqpoint{3.342653in}{0.502926in}}%
\pgfpathlineto{\pgfqpoint{3.342653in}{0.500000in}}%
\pgfpathclose%
\pgfusepath{fill}%
\end{pgfscope}%
\begin{pgfscope}%
\pgfpathrectangle{\pgfqpoint{0.750000in}{0.500000in}}{\pgfqpoint{4.650000in}{3.020000in}}%
\pgfusepath{clip}%
\pgfsetbuttcap%
\pgfsetmiterjoin%
\definecolor{currentfill}{rgb}{0.000000,0.500000,0.000000}%
\pgfsetfillcolor{currentfill}%
\pgfsetlinewidth{0.000000pt}%
\definecolor{currentstroke}{rgb}{0.000000,0.000000,0.000000}%
\pgfsetstrokecolor{currentstroke}%
\pgfsetstrokeopacity{0.000000}%
\pgfsetdash{}{0pt}%
\pgfpathmoveto{\pgfqpoint{3.375273in}{0.500000in}}%
\pgfpathlineto{\pgfqpoint{3.407893in}{0.500000in}}%
\pgfpathlineto{\pgfqpoint{3.407893in}{0.502926in}}%
\pgfpathlineto{\pgfqpoint{3.375273in}{0.502926in}}%
\pgfpathlineto{\pgfqpoint{3.375273in}{0.500000in}}%
\pgfpathclose%
\pgfusepath{fill}%
\end{pgfscope}%
\begin{pgfscope}%
\pgfpathrectangle{\pgfqpoint{0.750000in}{0.500000in}}{\pgfqpoint{4.650000in}{3.020000in}}%
\pgfusepath{clip}%
\pgfsetbuttcap%
\pgfsetmiterjoin%
\definecolor{currentfill}{rgb}{0.000000,0.500000,0.000000}%
\pgfsetfillcolor{currentfill}%
\pgfsetlinewidth{0.000000pt}%
\definecolor{currentstroke}{rgb}{0.000000,0.000000,0.000000}%
\pgfsetstrokecolor{currentstroke}%
\pgfsetstrokeopacity{0.000000}%
\pgfsetdash{}{0pt}%
\pgfpathmoveto{\pgfqpoint{3.407893in}{0.500000in}}%
\pgfpathlineto{\pgfqpoint{3.440514in}{0.500000in}}%
\pgfpathlineto{\pgfqpoint{3.440514in}{0.505852in}}%
\pgfpathlineto{\pgfqpoint{3.407893in}{0.505852in}}%
\pgfpathlineto{\pgfqpoint{3.407893in}{0.500000in}}%
\pgfpathclose%
\pgfusepath{fill}%
\end{pgfscope}%
\begin{pgfscope}%
\pgfpathrectangle{\pgfqpoint{0.750000in}{0.500000in}}{\pgfqpoint{4.650000in}{3.020000in}}%
\pgfusepath{clip}%
\pgfsetbuttcap%
\pgfsetmiterjoin%
\definecolor{currentfill}{rgb}{0.000000,0.500000,0.000000}%
\pgfsetfillcolor{currentfill}%
\pgfsetlinewidth{0.000000pt}%
\definecolor{currentstroke}{rgb}{0.000000,0.000000,0.000000}%
\pgfsetstrokecolor{currentstroke}%
\pgfsetstrokeopacity{0.000000}%
\pgfsetdash{}{0pt}%
\pgfpathmoveto{\pgfqpoint{3.440514in}{0.500000in}}%
\pgfpathlineto{\pgfqpoint{3.473134in}{0.500000in}}%
\pgfpathlineto{\pgfqpoint{3.473134in}{0.502926in}}%
\pgfpathlineto{\pgfqpoint{3.440514in}{0.502926in}}%
\pgfpathlineto{\pgfqpoint{3.440514in}{0.500000in}}%
\pgfpathclose%
\pgfusepath{fill}%
\end{pgfscope}%
\begin{pgfscope}%
\pgfpathrectangle{\pgfqpoint{0.750000in}{0.500000in}}{\pgfqpoint{4.650000in}{3.020000in}}%
\pgfusepath{clip}%
\pgfsetbuttcap%
\pgfsetmiterjoin%
\definecolor{currentfill}{rgb}{0.000000,0.500000,0.000000}%
\pgfsetfillcolor{currentfill}%
\pgfsetlinewidth{0.000000pt}%
\definecolor{currentstroke}{rgb}{0.000000,0.000000,0.000000}%
\pgfsetstrokecolor{currentstroke}%
\pgfsetstrokeopacity{0.000000}%
\pgfsetdash{}{0pt}%
\pgfpathmoveto{\pgfqpoint{3.473134in}{0.500000in}}%
\pgfpathlineto{\pgfqpoint{3.505755in}{0.500000in}}%
\pgfpathlineto{\pgfqpoint{3.505755in}{0.502926in}}%
\pgfpathlineto{\pgfqpoint{3.473134in}{0.502926in}}%
\pgfpathlineto{\pgfqpoint{3.473134in}{0.500000in}}%
\pgfpathclose%
\pgfusepath{fill}%
\end{pgfscope}%
\begin{pgfscope}%
\pgfpathrectangle{\pgfqpoint{0.750000in}{0.500000in}}{\pgfqpoint{4.650000in}{3.020000in}}%
\pgfusepath{clip}%
\pgfsetbuttcap%
\pgfsetmiterjoin%
\definecolor{currentfill}{rgb}{0.000000,0.500000,0.000000}%
\pgfsetfillcolor{currentfill}%
\pgfsetlinewidth{0.000000pt}%
\definecolor{currentstroke}{rgb}{0.000000,0.000000,0.000000}%
\pgfsetstrokecolor{currentstroke}%
\pgfsetstrokeopacity{0.000000}%
\pgfsetdash{}{0pt}%
\pgfpathmoveto{\pgfqpoint{3.505755in}{0.500000in}}%
\pgfpathlineto{\pgfqpoint{3.538375in}{0.500000in}}%
\pgfpathlineto{\pgfqpoint{3.538375in}{0.505852in}}%
\pgfpathlineto{\pgfqpoint{3.505755in}{0.505852in}}%
\pgfpathlineto{\pgfqpoint{3.505755in}{0.500000in}}%
\pgfpathclose%
\pgfusepath{fill}%
\end{pgfscope}%
\begin{pgfscope}%
\pgfpathrectangle{\pgfqpoint{0.750000in}{0.500000in}}{\pgfqpoint{4.650000in}{3.020000in}}%
\pgfusepath{clip}%
\pgfsetbuttcap%
\pgfsetmiterjoin%
\definecolor{currentfill}{rgb}{0.000000,0.500000,0.000000}%
\pgfsetfillcolor{currentfill}%
\pgfsetlinewidth{0.000000pt}%
\definecolor{currentstroke}{rgb}{0.000000,0.000000,0.000000}%
\pgfsetstrokecolor{currentstroke}%
\pgfsetstrokeopacity{0.000000}%
\pgfsetdash{}{0pt}%
\pgfpathmoveto{\pgfqpoint{3.538375in}{0.500000in}}%
\pgfpathlineto{\pgfqpoint{3.570995in}{0.500000in}}%
\pgfpathlineto{\pgfqpoint{3.570995in}{0.502926in}}%
\pgfpathlineto{\pgfqpoint{3.538375in}{0.502926in}}%
\pgfpathlineto{\pgfqpoint{3.538375in}{0.500000in}}%
\pgfpathclose%
\pgfusepath{fill}%
\end{pgfscope}%
\begin{pgfscope}%
\pgfpathrectangle{\pgfqpoint{0.750000in}{0.500000in}}{\pgfqpoint{4.650000in}{3.020000in}}%
\pgfusepath{clip}%
\pgfsetbuttcap%
\pgfsetmiterjoin%
\definecolor{currentfill}{rgb}{0.000000,0.500000,0.000000}%
\pgfsetfillcolor{currentfill}%
\pgfsetlinewidth{0.000000pt}%
\definecolor{currentstroke}{rgb}{0.000000,0.000000,0.000000}%
\pgfsetstrokecolor{currentstroke}%
\pgfsetstrokeopacity{0.000000}%
\pgfsetdash{}{0pt}%
\pgfpathmoveto{\pgfqpoint{3.570995in}{0.500000in}}%
\pgfpathlineto{\pgfqpoint{3.603616in}{0.500000in}}%
\pgfpathlineto{\pgfqpoint{3.603616in}{0.500000in}}%
\pgfpathlineto{\pgfqpoint{3.570995in}{0.500000in}}%
\pgfpathlineto{\pgfqpoint{3.570995in}{0.500000in}}%
\pgfpathclose%
\pgfusepath{fill}%
\end{pgfscope}%
\begin{pgfscope}%
\pgfpathrectangle{\pgfqpoint{0.750000in}{0.500000in}}{\pgfqpoint{4.650000in}{3.020000in}}%
\pgfusepath{clip}%
\pgfsetbuttcap%
\pgfsetmiterjoin%
\definecolor{currentfill}{rgb}{0.000000,0.500000,0.000000}%
\pgfsetfillcolor{currentfill}%
\pgfsetlinewidth{0.000000pt}%
\definecolor{currentstroke}{rgb}{0.000000,0.000000,0.000000}%
\pgfsetstrokecolor{currentstroke}%
\pgfsetstrokeopacity{0.000000}%
\pgfsetdash{}{0pt}%
\pgfpathmoveto{\pgfqpoint{3.603616in}{0.500000in}}%
\pgfpathlineto{\pgfqpoint{3.636236in}{0.500000in}}%
\pgfpathlineto{\pgfqpoint{3.636236in}{0.500000in}}%
\pgfpathlineto{\pgfqpoint{3.603616in}{0.500000in}}%
\pgfpathlineto{\pgfqpoint{3.603616in}{0.500000in}}%
\pgfpathclose%
\pgfusepath{fill}%
\end{pgfscope}%
\begin{pgfscope}%
\pgfpathrectangle{\pgfqpoint{0.750000in}{0.500000in}}{\pgfqpoint{4.650000in}{3.020000in}}%
\pgfusepath{clip}%
\pgfsetbuttcap%
\pgfsetmiterjoin%
\definecolor{currentfill}{rgb}{0.000000,0.500000,0.000000}%
\pgfsetfillcolor{currentfill}%
\pgfsetlinewidth{0.000000pt}%
\definecolor{currentstroke}{rgb}{0.000000,0.000000,0.000000}%
\pgfsetstrokecolor{currentstroke}%
\pgfsetstrokeopacity{0.000000}%
\pgfsetdash{}{0pt}%
\pgfpathmoveto{\pgfqpoint{3.636236in}{0.500000in}}%
\pgfpathlineto{\pgfqpoint{3.668856in}{0.500000in}}%
\pgfpathlineto{\pgfqpoint{3.668856in}{0.500000in}}%
\pgfpathlineto{\pgfqpoint{3.636236in}{0.500000in}}%
\pgfpathlineto{\pgfqpoint{3.636236in}{0.500000in}}%
\pgfpathclose%
\pgfusepath{fill}%
\end{pgfscope}%
\begin{pgfscope}%
\pgfpathrectangle{\pgfqpoint{0.750000in}{0.500000in}}{\pgfqpoint{4.650000in}{3.020000in}}%
\pgfusepath{clip}%
\pgfsetbuttcap%
\pgfsetmiterjoin%
\definecolor{currentfill}{rgb}{0.000000,0.500000,0.000000}%
\pgfsetfillcolor{currentfill}%
\pgfsetlinewidth{0.000000pt}%
\definecolor{currentstroke}{rgb}{0.000000,0.000000,0.000000}%
\pgfsetstrokecolor{currentstroke}%
\pgfsetstrokeopacity{0.000000}%
\pgfsetdash{}{0pt}%
\pgfpathmoveto{\pgfqpoint{3.668856in}{0.500000in}}%
\pgfpathlineto{\pgfqpoint{3.701477in}{0.500000in}}%
\pgfpathlineto{\pgfqpoint{3.701477in}{0.500000in}}%
\pgfpathlineto{\pgfqpoint{3.668856in}{0.500000in}}%
\pgfpathlineto{\pgfqpoint{3.668856in}{0.500000in}}%
\pgfpathclose%
\pgfusepath{fill}%
\end{pgfscope}%
\begin{pgfscope}%
\pgfpathrectangle{\pgfqpoint{0.750000in}{0.500000in}}{\pgfqpoint{4.650000in}{3.020000in}}%
\pgfusepath{clip}%
\pgfsetbuttcap%
\pgfsetmiterjoin%
\definecolor{currentfill}{rgb}{0.000000,0.500000,0.000000}%
\pgfsetfillcolor{currentfill}%
\pgfsetlinewidth{0.000000pt}%
\definecolor{currentstroke}{rgb}{0.000000,0.000000,0.000000}%
\pgfsetstrokecolor{currentstroke}%
\pgfsetstrokeopacity{0.000000}%
\pgfsetdash{}{0pt}%
\pgfpathmoveto{\pgfqpoint{3.701477in}{0.500000in}}%
\pgfpathlineto{\pgfqpoint{3.734097in}{0.500000in}}%
\pgfpathlineto{\pgfqpoint{3.734097in}{0.500000in}}%
\pgfpathlineto{\pgfqpoint{3.701477in}{0.500000in}}%
\pgfpathlineto{\pgfqpoint{3.701477in}{0.500000in}}%
\pgfpathclose%
\pgfusepath{fill}%
\end{pgfscope}%
\begin{pgfscope}%
\pgfpathrectangle{\pgfqpoint{0.750000in}{0.500000in}}{\pgfqpoint{4.650000in}{3.020000in}}%
\pgfusepath{clip}%
\pgfsetbuttcap%
\pgfsetmiterjoin%
\definecolor{currentfill}{rgb}{0.000000,0.500000,0.000000}%
\pgfsetfillcolor{currentfill}%
\pgfsetlinewidth{0.000000pt}%
\definecolor{currentstroke}{rgb}{0.000000,0.000000,0.000000}%
\pgfsetstrokecolor{currentstroke}%
\pgfsetstrokeopacity{0.000000}%
\pgfsetdash{}{0pt}%
\pgfpathmoveto{\pgfqpoint{3.734097in}{0.500000in}}%
\pgfpathlineto{\pgfqpoint{3.766718in}{0.500000in}}%
\pgfpathlineto{\pgfqpoint{3.766718in}{0.502926in}}%
\pgfpathlineto{\pgfqpoint{3.734097in}{0.502926in}}%
\pgfpathlineto{\pgfqpoint{3.734097in}{0.500000in}}%
\pgfpathclose%
\pgfusepath{fill}%
\end{pgfscope}%
\begin{pgfscope}%
\pgfpathrectangle{\pgfqpoint{0.750000in}{0.500000in}}{\pgfqpoint{4.650000in}{3.020000in}}%
\pgfusepath{clip}%
\pgfsetbuttcap%
\pgfsetmiterjoin%
\definecolor{currentfill}{rgb}{0.000000,0.500000,0.000000}%
\pgfsetfillcolor{currentfill}%
\pgfsetlinewidth{0.000000pt}%
\definecolor{currentstroke}{rgb}{0.000000,0.000000,0.000000}%
\pgfsetstrokecolor{currentstroke}%
\pgfsetstrokeopacity{0.000000}%
\pgfsetdash{}{0pt}%
\pgfpathmoveto{\pgfqpoint{3.766718in}{0.500000in}}%
\pgfpathlineto{\pgfqpoint{3.799338in}{0.500000in}}%
\pgfpathlineto{\pgfqpoint{3.799338in}{0.500000in}}%
\pgfpathlineto{\pgfqpoint{3.766718in}{0.500000in}}%
\pgfpathlineto{\pgfqpoint{3.766718in}{0.500000in}}%
\pgfpathclose%
\pgfusepath{fill}%
\end{pgfscope}%
\begin{pgfscope}%
\pgfpathrectangle{\pgfqpoint{0.750000in}{0.500000in}}{\pgfqpoint{4.650000in}{3.020000in}}%
\pgfusepath{clip}%
\pgfsetbuttcap%
\pgfsetmiterjoin%
\definecolor{currentfill}{rgb}{0.000000,0.500000,0.000000}%
\pgfsetfillcolor{currentfill}%
\pgfsetlinewidth{0.000000pt}%
\definecolor{currentstroke}{rgb}{0.000000,0.000000,0.000000}%
\pgfsetstrokecolor{currentstroke}%
\pgfsetstrokeopacity{0.000000}%
\pgfsetdash{}{0pt}%
\pgfpathmoveto{\pgfqpoint{3.799338in}{0.500000in}}%
\pgfpathlineto{\pgfqpoint{3.831958in}{0.500000in}}%
\pgfpathlineto{\pgfqpoint{3.831958in}{0.500000in}}%
\pgfpathlineto{\pgfqpoint{3.799338in}{0.500000in}}%
\pgfpathlineto{\pgfqpoint{3.799338in}{0.500000in}}%
\pgfpathclose%
\pgfusepath{fill}%
\end{pgfscope}%
\begin{pgfscope}%
\pgfpathrectangle{\pgfqpoint{0.750000in}{0.500000in}}{\pgfqpoint{4.650000in}{3.020000in}}%
\pgfusepath{clip}%
\pgfsetbuttcap%
\pgfsetmiterjoin%
\definecolor{currentfill}{rgb}{0.000000,0.500000,0.000000}%
\pgfsetfillcolor{currentfill}%
\pgfsetlinewidth{0.000000pt}%
\definecolor{currentstroke}{rgb}{0.000000,0.000000,0.000000}%
\pgfsetstrokecolor{currentstroke}%
\pgfsetstrokeopacity{0.000000}%
\pgfsetdash{}{0pt}%
\pgfpathmoveto{\pgfqpoint{3.831958in}{0.500000in}}%
\pgfpathlineto{\pgfqpoint{3.864579in}{0.500000in}}%
\pgfpathlineto{\pgfqpoint{3.864579in}{0.500000in}}%
\pgfpathlineto{\pgfqpoint{3.831958in}{0.500000in}}%
\pgfpathlineto{\pgfqpoint{3.831958in}{0.500000in}}%
\pgfpathclose%
\pgfusepath{fill}%
\end{pgfscope}%
\begin{pgfscope}%
\pgfpathrectangle{\pgfqpoint{0.750000in}{0.500000in}}{\pgfqpoint{4.650000in}{3.020000in}}%
\pgfusepath{clip}%
\pgfsetbuttcap%
\pgfsetmiterjoin%
\definecolor{currentfill}{rgb}{0.000000,0.500000,0.000000}%
\pgfsetfillcolor{currentfill}%
\pgfsetlinewidth{0.000000pt}%
\definecolor{currentstroke}{rgb}{0.000000,0.000000,0.000000}%
\pgfsetstrokecolor{currentstroke}%
\pgfsetstrokeopacity{0.000000}%
\pgfsetdash{}{0pt}%
\pgfpathmoveto{\pgfqpoint{3.864579in}{0.500000in}}%
\pgfpathlineto{\pgfqpoint{3.897199in}{0.500000in}}%
\pgfpathlineto{\pgfqpoint{3.897199in}{0.500000in}}%
\pgfpathlineto{\pgfqpoint{3.864579in}{0.500000in}}%
\pgfpathlineto{\pgfqpoint{3.864579in}{0.500000in}}%
\pgfpathclose%
\pgfusepath{fill}%
\end{pgfscope}%
\begin{pgfscope}%
\pgfpathrectangle{\pgfqpoint{0.750000in}{0.500000in}}{\pgfqpoint{4.650000in}{3.020000in}}%
\pgfusepath{clip}%
\pgfsetbuttcap%
\pgfsetmiterjoin%
\definecolor{currentfill}{rgb}{0.000000,0.500000,0.000000}%
\pgfsetfillcolor{currentfill}%
\pgfsetlinewidth{0.000000pt}%
\definecolor{currentstroke}{rgb}{0.000000,0.000000,0.000000}%
\pgfsetstrokecolor{currentstroke}%
\pgfsetstrokeopacity{0.000000}%
\pgfsetdash{}{0pt}%
\pgfpathmoveto{\pgfqpoint{3.897199in}{0.500000in}}%
\pgfpathlineto{\pgfqpoint{3.929820in}{0.500000in}}%
\pgfpathlineto{\pgfqpoint{3.929820in}{0.500000in}}%
\pgfpathlineto{\pgfqpoint{3.897199in}{0.500000in}}%
\pgfpathlineto{\pgfqpoint{3.897199in}{0.500000in}}%
\pgfpathclose%
\pgfusepath{fill}%
\end{pgfscope}%
\begin{pgfscope}%
\pgfpathrectangle{\pgfqpoint{0.750000in}{0.500000in}}{\pgfqpoint{4.650000in}{3.020000in}}%
\pgfusepath{clip}%
\pgfsetbuttcap%
\pgfsetmiterjoin%
\definecolor{currentfill}{rgb}{0.000000,0.500000,0.000000}%
\pgfsetfillcolor{currentfill}%
\pgfsetlinewidth{0.000000pt}%
\definecolor{currentstroke}{rgb}{0.000000,0.000000,0.000000}%
\pgfsetstrokecolor{currentstroke}%
\pgfsetstrokeopacity{0.000000}%
\pgfsetdash{}{0pt}%
\pgfpathmoveto{\pgfqpoint{3.929820in}{0.500000in}}%
\pgfpathlineto{\pgfqpoint{3.962440in}{0.500000in}}%
\pgfpathlineto{\pgfqpoint{3.962440in}{0.500000in}}%
\pgfpathlineto{\pgfqpoint{3.929820in}{0.500000in}}%
\pgfpathlineto{\pgfqpoint{3.929820in}{0.500000in}}%
\pgfpathclose%
\pgfusepath{fill}%
\end{pgfscope}%
\begin{pgfscope}%
\pgfpathrectangle{\pgfqpoint{0.750000in}{0.500000in}}{\pgfqpoint{4.650000in}{3.020000in}}%
\pgfusepath{clip}%
\pgfsetbuttcap%
\pgfsetmiterjoin%
\definecolor{currentfill}{rgb}{0.000000,0.500000,0.000000}%
\pgfsetfillcolor{currentfill}%
\pgfsetlinewidth{0.000000pt}%
\definecolor{currentstroke}{rgb}{0.000000,0.000000,0.000000}%
\pgfsetstrokecolor{currentstroke}%
\pgfsetstrokeopacity{0.000000}%
\pgfsetdash{}{0pt}%
\pgfpathmoveto{\pgfqpoint{3.962440in}{0.500000in}}%
\pgfpathlineto{\pgfqpoint{3.995060in}{0.500000in}}%
\pgfpathlineto{\pgfqpoint{3.995060in}{0.500000in}}%
\pgfpathlineto{\pgfqpoint{3.962440in}{0.500000in}}%
\pgfpathlineto{\pgfqpoint{3.962440in}{0.500000in}}%
\pgfpathclose%
\pgfusepath{fill}%
\end{pgfscope}%
\begin{pgfscope}%
\pgfpathrectangle{\pgfqpoint{0.750000in}{0.500000in}}{\pgfqpoint{4.650000in}{3.020000in}}%
\pgfusepath{clip}%
\pgfsetbuttcap%
\pgfsetmiterjoin%
\definecolor{currentfill}{rgb}{0.000000,0.500000,0.000000}%
\pgfsetfillcolor{currentfill}%
\pgfsetlinewidth{0.000000pt}%
\definecolor{currentstroke}{rgb}{0.000000,0.000000,0.000000}%
\pgfsetstrokecolor{currentstroke}%
\pgfsetstrokeopacity{0.000000}%
\pgfsetdash{}{0pt}%
\pgfpathmoveto{\pgfqpoint{3.995060in}{0.500000in}}%
\pgfpathlineto{\pgfqpoint{4.027681in}{0.500000in}}%
\pgfpathlineto{\pgfqpoint{4.027681in}{0.500000in}}%
\pgfpathlineto{\pgfqpoint{3.995060in}{0.500000in}}%
\pgfpathlineto{\pgfqpoint{3.995060in}{0.500000in}}%
\pgfpathclose%
\pgfusepath{fill}%
\end{pgfscope}%
\begin{pgfscope}%
\pgfpathrectangle{\pgfqpoint{0.750000in}{0.500000in}}{\pgfqpoint{4.650000in}{3.020000in}}%
\pgfusepath{clip}%
\pgfsetbuttcap%
\pgfsetmiterjoin%
\definecolor{currentfill}{rgb}{0.000000,0.500000,0.000000}%
\pgfsetfillcolor{currentfill}%
\pgfsetlinewidth{0.000000pt}%
\definecolor{currentstroke}{rgb}{0.000000,0.000000,0.000000}%
\pgfsetstrokecolor{currentstroke}%
\pgfsetstrokeopacity{0.000000}%
\pgfsetdash{}{0pt}%
\pgfpathmoveto{\pgfqpoint{4.027681in}{0.500000in}}%
\pgfpathlineto{\pgfqpoint{4.060301in}{0.500000in}}%
\pgfpathlineto{\pgfqpoint{4.060301in}{0.500000in}}%
\pgfpathlineto{\pgfqpoint{4.027681in}{0.500000in}}%
\pgfpathlineto{\pgfqpoint{4.027681in}{0.500000in}}%
\pgfpathclose%
\pgfusepath{fill}%
\end{pgfscope}%
\begin{pgfscope}%
\pgfpathrectangle{\pgfqpoint{0.750000in}{0.500000in}}{\pgfqpoint{4.650000in}{3.020000in}}%
\pgfusepath{clip}%
\pgfsetbuttcap%
\pgfsetmiterjoin%
\definecolor{currentfill}{rgb}{0.000000,0.500000,0.000000}%
\pgfsetfillcolor{currentfill}%
\pgfsetlinewidth{0.000000pt}%
\definecolor{currentstroke}{rgb}{0.000000,0.000000,0.000000}%
\pgfsetstrokecolor{currentstroke}%
\pgfsetstrokeopacity{0.000000}%
\pgfsetdash{}{0pt}%
\pgfpathmoveto{\pgfqpoint{4.060301in}{0.500000in}}%
\pgfpathlineto{\pgfqpoint{4.092922in}{0.500000in}}%
\pgfpathlineto{\pgfqpoint{4.092922in}{0.500000in}}%
\pgfpathlineto{\pgfqpoint{4.060301in}{0.500000in}}%
\pgfpathlineto{\pgfqpoint{4.060301in}{0.500000in}}%
\pgfpathclose%
\pgfusepath{fill}%
\end{pgfscope}%
\begin{pgfscope}%
\pgfpathrectangle{\pgfqpoint{0.750000in}{0.500000in}}{\pgfqpoint{4.650000in}{3.020000in}}%
\pgfusepath{clip}%
\pgfsetbuttcap%
\pgfsetmiterjoin%
\definecolor{currentfill}{rgb}{0.000000,0.500000,0.000000}%
\pgfsetfillcolor{currentfill}%
\pgfsetlinewidth{0.000000pt}%
\definecolor{currentstroke}{rgb}{0.000000,0.000000,0.000000}%
\pgfsetstrokecolor{currentstroke}%
\pgfsetstrokeopacity{0.000000}%
\pgfsetdash{}{0pt}%
\pgfpathmoveto{\pgfqpoint{4.092922in}{0.500000in}}%
\pgfpathlineto{\pgfqpoint{4.125542in}{0.500000in}}%
\pgfpathlineto{\pgfqpoint{4.125542in}{0.500000in}}%
\pgfpathlineto{\pgfqpoint{4.092922in}{0.500000in}}%
\pgfpathlineto{\pgfqpoint{4.092922in}{0.500000in}}%
\pgfpathclose%
\pgfusepath{fill}%
\end{pgfscope}%
\begin{pgfscope}%
\pgfpathrectangle{\pgfqpoint{0.750000in}{0.500000in}}{\pgfqpoint{4.650000in}{3.020000in}}%
\pgfusepath{clip}%
\pgfsetbuttcap%
\pgfsetmiterjoin%
\definecolor{currentfill}{rgb}{0.000000,0.500000,0.000000}%
\pgfsetfillcolor{currentfill}%
\pgfsetlinewidth{0.000000pt}%
\definecolor{currentstroke}{rgb}{0.000000,0.000000,0.000000}%
\pgfsetstrokecolor{currentstroke}%
\pgfsetstrokeopacity{0.000000}%
\pgfsetdash{}{0pt}%
\pgfpathmoveto{\pgfqpoint{4.125542in}{0.500000in}}%
\pgfpathlineto{\pgfqpoint{4.158162in}{0.500000in}}%
\pgfpathlineto{\pgfqpoint{4.158162in}{0.500000in}}%
\pgfpathlineto{\pgfqpoint{4.125542in}{0.500000in}}%
\pgfpathlineto{\pgfqpoint{4.125542in}{0.500000in}}%
\pgfpathclose%
\pgfusepath{fill}%
\end{pgfscope}%
\begin{pgfscope}%
\pgfpathrectangle{\pgfqpoint{0.750000in}{0.500000in}}{\pgfqpoint{4.650000in}{3.020000in}}%
\pgfusepath{clip}%
\pgfsetbuttcap%
\pgfsetmiterjoin%
\definecolor{currentfill}{rgb}{0.000000,0.500000,0.000000}%
\pgfsetfillcolor{currentfill}%
\pgfsetlinewidth{0.000000pt}%
\definecolor{currentstroke}{rgb}{0.000000,0.000000,0.000000}%
\pgfsetstrokecolor{currentstroke}%
\pgfsetstrokeopacity{0.000000}%
\pgfsetdash{}{0pt}%
\pgfpathmoveto{\pgfqpoint{4.158162in}{0.500000in}}%
\pgfpathlineto{\pgfqpoint{4.190783in}{0.500000in}}%
\pgfpathlineto{\pgfqpoint{4.190783in}{0.500000in}}%
\pgfpathlineto{\pgfqpoint{4.158162in}{0.500000in}}%
\pgfpathlineto{\pgfqpoint{4.158162in}{0.500000in}}%
\pgfpathclose%
\pgfusepath{fill}%
\end{pgfscope}%
\begin{pgfscope}%
\pgfpathrectangle{\pgfqpoint{0.750000in}{0.500000in}}{\pgfqpoint{4.650000in}{3.020000in}}%
\pgfusepath{clip}%
\pgfsetbuttcap%
\pgfsetmiterjoin%
\definecolor{currentfill}{rgb}{0.000000,0.500000,0.000000}%
\pgfsetfillcolor{currentfill}%
\pgfsetlinewidth{0.000000pt}%
\definecolor{currentstroke}{rgb}{0.000000,0.000000,0.000000}%
\pgfsetstrokecolor{currentstroke}%
\pgfsetstrokeopacity{0.000000}%
\pgfsetdash{}{0pt}%
\pgfpathmoveto{\pgfqpoint{4.190783in}{0.500000in}}%
\pgfpathlineto{\pgfqpoint{4.223403in}{0.500000in}}%
\pgfpathlineto{\pgfqpoint{4.223403in}{0.502926in}}%
\pgfpathlineto{\pgfqpoint{4.190783in}{0.502926in}}%
\pgfpathlineto{\pgfqpoint{4.190783in}{0.500000in}}%
\pgfpathclose%
\pgfusepath{fill}%
\end{pgfscope}%
\begin{pgfscope}%
\pgfpathrectangle{\pgfqpoint{0.750000in}{0.500000in}}{\pgfqpoint{4.650000in}{3.020000in}}%
\pgfusepath{clip}%
\pgfsetbuttcap%
\pgfsetmiterjoin%
\definecolor{currentfill}{rgb}{0.000000,0.500000,0.000000}%
\pgfsetfillcolor{currentfill}%
\pgfsetlinewidth{0.000000pt}%
\definecolor{currentstroke}{rgb}{0.000000,0.000000,0.000000}%
\pgfsetstrokecolor{currentstroke}%
\pgfsetstrokeopacity{0.000000}%
\pgfsetdash{}{0pt}%
\pgfpathmoveto{\pgfqpoint{4.223403in}{0.500000in}}%
\pgfpathlineto{\pgfqpoint{4.256024in}{0.500000in}}%
\pgfpathlineto{\pgfqpoint{4.256024in}{0.502926in}}%
\pgfpathlineto{\pgfqpoint{4.223403in}{0.502926in}}%
\pgfpathlineto{\pgfqpoint{4.223403in}{0.500000in}}%
\pgfpathclose%
\pgfusepath{fill}%
\end{pgfscope}%
\begin{pgfscope}%
\pgfpathrectangle{\pgfqpoint{0.750000in}{0.500000in}}{\pgfqpoint{4.650000in}{3.020000in}}%
\pgfusepath{clip}%
\pgfsetbuttcap%
\pgfsetmiterjoin%
\definecolor{currentfill}{rgb}{0.000000,0.500000,0.000000}%
\pgfsetfillcolor{currentfill}%
\pgfsetlinewidth{0.000000pt}%
\definecolor{currentstroke}{rgb}{0.000000,0.000000,0.000000}%
\pgfsetstrokecolor{currentstroke}%
\pgfsetstrokeopacity{0.000000}%
\pgfsetdash{}{0pt}%
\pgfpathmoveto{\pgfqpoint{4.256024in}{0.500000in}}%
\pgfpathlineto{\pgfqpoint{4.288644in}{0.500000in}}%
\pgfpathlineto{\pgfqpoint{4.288644in}{0.502926in}}%
\pgfpathlineto{\pgfqpoint{4.256024in}{0.502926in}}%
\pgfpathlineto{\pgfqpoint{4.256024in}{0.500000in}}%
\pgfpathclose%
\pgfusepath{fill}%
\end{pgfscope}%
\begin{pgfscope}%
\pgfpathrectangle{\pgfqpoint{0.750000in}{0.500000in}}{\pgfqpoint{4.650000in}{3.020000in}}%
\pgfusepath{clip}%
\pgfsetbuttcap%
\pgfsetmiterjoin%
\definecolor{currentfill}{rgb}{0.000000,0.500000,0.000000}%
\pgfsetfillcolor{currentfill}%
\pgfsetlinewidth{0.000000pt}%
\definecolor{currentstroke}{rgb}{0.000000,0.000000,0.000000}%
\pgfsetstrokecolor{currentstroke}%
\pgfsetstrokeopacity{0.000000}%
\pgfsetdash{}{0pt}%
\pgfpathmoveto{\pgfqpoint{4.288644in}{0.500000in}}%
\pgfpathlineto{\pgfqpoint{4.321264in}{0.500000in}}%
\pgfpathlineto{\pgfqpoint{4.321264in}{0.502926in}}%
\pgfpathlineto{\pgfqpoint{4.288644in}{0.502926in}}%
\pgfpathlineto{\pgfqpoint{4.288644in}{0.500000in}}%
\pgfpathclose%
\pgfusepath{fill}%
\end{pgfscope}%
\begin{pgfscope}%
\pgfpathrectangle{\pgfqpoint{0.750000in}{0.500000in}}{\pgfqpoint{4.650000in}{3.020000in}}%
\pgfusepath{clip}%
\pgfsetbuttcap%
\pgfsetmiterjoin%
\definecolor{currentfill}{rgb}{0.000000,0.500000,0.000000}%
\pgfsetfillcolor{currentfill}%
\pgfsetlinewidth{0.000000pt}%
\definecolor{currentstroke}{rgb}{0.000000,0.000000,0.000000}%
\pgfsetstrokecolor{currentstroke}%
\pgfsetstrokeopacity{0.000000}%
\pgfsetdash{}{0pt}%
\pgfpathmoveto{\pgfqpoint{4.321264in}{0.500000in}}%
\pgfpathlineto{\pgfqpoint{4.353885in}{0.500000in}}%
\pgfpathlineto{\pgfqpoint{4.353885in}{0.500000in}}%
\pgfpathlineto{\pgfqpoint{4.321264in}{0.500000in}}%
\pgfpathlineto{\pgfqpoint{4.321264in}{0.500000in}}%
\pgfpathclose%
\pgfusepath{fill}%
\end{pgfscope}%
\begin{pgfscope}%
\pgfpathrectangle{\pgfqpoint{0.750000in}{0.500000in}}{\pgfqpoint{4.650000in}{3.020000in}}%
\pgfusepath{clip}%
\pgfsetbuttcap%
\pgfsetmiterjoin%
\definecolor{currentfill}{rgb}{0.000000,0.500000,0.000000}%
\pgfsetfillcolor{currentfill}%
\pgfsetlinewidth{0.000000pt}%
\definecolor{currentstroke}{rgb}{0.000000,0.000000,0.000000}%
\pgfsetstrokecolor{currentstroke}%
\pgfsetstrokeopacity{0.000000}%
\pgfsetdash{}{0pt}%
\pgfpathmoveto{\pgfqpoint{4.353885in}{0.500000in}}%
\pgfpathlineto{\pgfqpoint{4.386505in}{0.500000in}}%
\pgfpathlineto{\pgfqpoint{4.386505in}{0.502926in}}%
\pgfpathlineto{\pgfqpoint{4.353885in}{0.502926in}}%
\pgfpathlineto{\pgfqpoint{4.353885in}{0.500000in}}%
\pgfpathclose%
\pgfusepath{fill}%
\end{pgfscope}%
\begin{pgfscope}%
\pgfpathrectangle{\pgfqpoint{0.750000in}{0.500000in}}{\pgfqpoint{4.650000in}{3.020000in}}%
\pgfusepath{clip}%
\pgfsetbuttcap%
\pgfsetmiterjoin%
\definecolor{currentfill}{rgb}{0.000000,0.500000,0.000000}%
\pgfsetfillcolor{currentfill}%
\pgfsetlinewidth{0.000000pt}%
\definecolor{currentstroke}{rgb}{0.000000,0.000000,0.000000}%
\pgfsetstrokecolor{currentstroke}%
\pgfsetstrokeopacity{0.000000}%
\pgfsetdash{}{0pt}%
\pgfpathmoveto{\pgfqpoint{4.386505in}{0.500000in}}%
\pgfpathlineto{\pgfqpoint{4.419126in}{0.500000in}}%
\pgfpathlineto{\pgfqpoint{4.419126in}{0.500000in}}%
\pgfpathlineto{\pgfqpoint{4.386505in}{0.500000in}}%
\pgfpathlineto{\pgfqpoint{4.386505in}{0.500000in}}%
\pgfpathclose%
\pgfusepath{fill}%
\end{pgfscope}%
\begin{pgfscope}%
\pgfpathrectangle{\pgfqpoint{0.750000in}{0.500000in}}{\pgfqpoint{4.650000in}{3.020000in}}%
\pgfusepath{clip}%
\pgfsetbuttcap%
\pgfsetmiterjoin%
\definecolor{currentfill}{rgb}{0.000000,0.500000,0.000000}%
\pgfsetfillcolor{currentfill}%
\pgfsetlinewidth{0.000000pt}%
\definecolor{currentstroke}{rgb}{0.000000,0.000000,0.000000}%
\pgfsetstrokecolor{currentstroke}%
\pgfsetstrokeopacity{0.000000}%
\pgfsetdash{}{0pt}%
\pgfpathmoveto{\pgfqpoint{4.419126in}{0.500000in}}%
\pgfpathlineto{\pgfqpoint{4.451746in}{0.500000in}}%
\pgfpathlineto{\pgfqpoint{4.451746in}{0.500000in}}%
\pgfpathlineto{\pgfqpoint{4.419126in}{0.500000in}}%
\pgfpathlineto{\pgfqpoint{4.419126in}{0.500000in}}%
\pgfpathclose%
\pgfusepath{fill}%
\end{pgfscope}%
\begin{pgfscope}%
\pgfpathrectangle{\pgfqpoint{0.750000in}{0.500000in}}{\pgfqpoint{4.650000in}{3.020000in}}%
\pgfusepath{clip}%
\pgfsetbuttcap%
\pgfsetmiterjoin%
\definecolor{currentfill}{rgb}{0.000000,0.500000,0.000000}%
\pgfsetfillcolor{currentfill}%
\pgfsetlinewidth{0.000000pt}%
\definecolor{currentstroke}{rgb}{0.000000,0.000000,0.000000}%
\pgfsetstrokecolor{currentstroke}%
\pgfsetstrokeopacity{0.000000}%
\pgfsetdash{}{0pt}%
\pgfpathmoveto{\pgfqpoint{4.451746in}{0.500000in}}%
\pgfpathlineto{\pgfqpoint{4.484366in}{0.500000in}}%
\pgfpathlineto{\pgfqpoint{4.484366in}{0.500000in}}%
\pgfpathlineto{\pgfqpoint{4.451746in}{0.500000in}}%
\pgfpathlineto{\pgfqpoint{4.451746in}{0.500000in}}%
\pgfpathclose%
\pgfusepath{fill}%
\end{pgfscope}%
\begin{pgfscope}%
\pgfpathrectangle{\pgfqpoint{0.750000in}{0.500000in}}{\pgfqpoint{4.650000in}{3.020000in}}%
\pgfusepath{clip}%
\pgfsetbuttcap%
\pgfsetmiterjoin%
\definecolor{currentfill}{rgb}{0.000000,0.500000,0.000000}%
\pgfsetfillcolor{currentfill}%
\pgfsetlinewidth{0.000000pt}%
\definecolor{currentstroke}{rgb}{0.000000,0.000000,0.000000}%
\pgfsetstrokecolor{currentstroke}%
\pgfsetstrokeopacity{0.000000}%
\pgfsetdash{}{0pt}%
\pgfpathmoveto{\pgfqpoint{4.484366in}{0.500000in}}%
\pgfpathlineto{\pgfqpoint{4.516987in}{0.500000in}}%
\pgfpathlineto{\pgfqpoint{4.516987in}{0.500000in}}%
\pgfpathlineto{\pgfqpoint{4.484366in}{0.500000in}}%
\pgfpathlineto{\pgfqpoint{4.484366in}{0.500000in}}%
\pgfpathclose%
\pgfusepath{fill}%
\end{pgfscope}%
\begin{pgfscope}%
\pgfpathrectangle{\pgfqpoint{0.750000in}{0.500000in}}{\pgfqpoint{4.650000in}{3.020000in}}%
\pgfusepath{clip}%
\pgfsetbuttcap%
\pgfsetmiterjoin%
\definecolor{currentfill}{rgb}{0.000000,0.500000,0.000000}%
\pgfsetfillcolor{currentfill}%
\pgfsetlinewidth{0.000000pt}%
\definecolor{currentstroke}{rgb}{0.000000,0.000000,0.000000}%
\pgfsetstrokecolor{currentstroke}%
\pgfsetstrokeopacity{0.000000}%
\pgfsetdash{}{0pt}%
\pgfpathmoveto{\pgfqpoint{4.516987in}{0.500000in}}%
\pgfpathlineto{\pgfqpoint{4.549607in}{0.500000in}}%
\pgfpathlineto{\pgfqpoint{4.549607in}{0.500000in}}%
\pgfpathlineto{\pgfqpoint{4.516987in}{0.500000in}}%
\pgfpathlineto{\pgfqpoint{4.516987in}{0.500000in}}%
\pgfpathclose%
\pgfusepath{fill}%
\end{pgfscope}%
\begin{pgfscope}%
\pgfpathrectangle{\pgfqpoint{0.750000in}{0.500000in}}{\pgfqpoint{4.650000in}{3.020000in}}%
\pgfusepath{clip}%
\pgfsetbuttcap%
\pgfsetmiterjoin%
\definecolor{currentfill}{rgb}{0.000000,0.500000,0.000000}%
\pgfsetfillcolor{currentfill}%
\pgfsetlinewidth{0.000000pt}%
\definecolor{currentstroke}{rgb}{0.000000,0.000000,0.000000}%
\pgfsetstrokecolor{currentstroke}%
\pgfsetstrokeopacity{0.000000}%
\pgfsetdash{}{0pt}%
\pgfpathmoveto{\pgfqpoint{4.549607in}{0.500000in}}%
\pgfpathlineto{\pgfqpoint{4.582228in}{0.500000in}}%
\pgfpathlineto{\pgfqpoint{4.582228in}{0.505852in}}%
\pgfpathlineto{\pgfqpoint{4.549607in}{0.505852in}}%
\pgfpathlineto{\pgfqpoint{4.549607in}{0.500000in}}%
\pgfpathclose%
\pgfusepath{fill}%
\end{pgfscope}%
\begin{pgfscope}%
\pgfpathrectangle{\pgfqpoint{0.750000in}{0.500000in}}{\pgfqpoint{4.650000in}{3.020000in}}%
\pgfusepath{clip}%
\pgfsetbuttcap%
\pgfsetmiterjoin%
\definecolor{currentfill}{rgb}{0.000000,0.500000,0.000000}%
\pgfsetfillcolor{currentfill}%
\pgfsetlinewidth{0.000000pt}%
\definecolor{currentstroke}{rgb}{0.000000,0.000000,0.000000}%
\pgfsetstrokecolor{currentstroke}%
\pgfsetstrokeopacity{0.000000}%
\pgfsetdash{}{0pt}%
\pgfpathmoveto{\pgfqpoint{4.582228in}{0.500000in}}%
\pgfpathlineto{\pgfqpoint{4.614848in}{0.500000in}}%
\pgfpathlineto{\pgfqpoint{4.614848in}{0.500000in}}%
\pgfpathlineto{\pgfqpoint{4.582228in}{0.500000in}}%
\pgfpathlineto{\pgfqpoint{4.582228in}{0.500000in}}%
\pgfpathclose%
\pgfusepath{fill}%
\end{pgfscope}%
\begin{pgfscope}%
\pgfpathrectangle{\pgfqpoint{0.750000in}{0.500000in}}{\pgfqpoint{4.650000in}{3.020000in}}%
\pgfusepath{clip}%
\pgfsetbuttcap%
\pgfsetmiterjoin%
\definecolor{currentfill}{rgb}{0.000000,0.500000,0.000000}%
\pgfsetfillcolor{currentfill}%
\pgfsetlinewidth{0.000000pt}%
\definecolor{currentstroke}{rgb}{0.000000,0.000000,0.000000}%
\pgfsetstrokecolor{currentstroke}%
\pgfsetstrokeopacity{0.000000}%
\pgfsetdash{}{0pt}%
\pgfpathmoveto{\pgfqpoint{4.614848in}{0.500000in}}%
\pgfpathlineto{\pgfqpoint{4.647468in}{0.500000in}}%
\pgfpathlineto{\pgfqpoint{4.647468in}{0.500000in}}%
\pgfpathlineto{\pgfqpoint{4.614848in}{0.500000in}}%
\pgfpathlineto{\pgfqpoint{4.614848in}{0.500000in}}%
\pgfpathclose%
\pgfusepath{fill}%
\end{pgfscope}%
\begin{pgfscope}%
\pgfpathrectangle{\pgfqpoint{0.750000in}{0.500000in}}{\pgfqpoint{4.650000in}{3.020000in}}%
\pgfusepath{clip}%
\pgfsetbuttcap%
\pgfsetmiterjoin%
\definecolor{currentfill}{rgb}{0.000000,0.500000,0.000000}%
\pgfsetfillcolor{currentfill}%
\pgfsetlinewidth{0.000000pt}%
\definecolor{currentstroke}{rgb}{0.000000,0.000000,0.000000}%
\pgfsetstrokecolor{currentstroke}%
\pgfsetstrokeopacity{0.000000}%
\pgfsetdash{}{0pt}%
\pgfpathmoveto{\pgfqpoint{4.647468in}{0.500000in}}%
\pgfpathlineto{\pgfqpoint{4.680089in}{0.500000in}}%
\pgfpathlineto{\pgfqpoint{4.680089in}{0.500000in}}%
\pgfpathlineto{\pgfqpoint{4.647468in}{0.500000in}}%
\pgfpathlineto{\pgfqpoint{4.647468in}{0.500000in}}%
\pgfpathclose%
\pgfusepath{fill}%
\end{pgfscope}%
\begin{pgfscope}%
\pgfpathrectangle{\pgfqpoint{0.750000in}{0.500000in}}{\pgfqpoint{4.650000in}{3.020000in}}%
\pgfusepath{clip}%
\pgfsetbuttcap%
\pgfsetmiterjoin%
\definecolor{currentfill}{rgb}{0.000000,0.500000,0.000000}%
\pgfsetfillcolor{currentfill}%
\pgfsetlinewidth{0.000000pt}%
\definecolor{currentstroke}{rgb}{0.000000,0.000000,0.000000}%
\pgfsetstrokecolor{currentstroke}%
\pgfsetstrokeopacity{0.000000}%
\pgfsetdash{}{0pt}%
\pgfpathmoveto{\pgfqpoint{4.680089in}{0.500000in}}%
\pgfpathlineto{\pgfqpoint{4.712709in}{0.500000in}}%
\pgfpathlineto{\pgfqpoint{4.712709in}{0.502926in}}%
\pgfpathlineto{\pgfqpoint{4.680089in}{0.502926in}}%
\pgfpathlineto{\pgfqpoint{4.680089in}{0.500000in}}%
\pgfpathclose%
\pgfusepath{fill}%
\end{pgfscope}%
\begin{pgfscope}%
\pgfpathrectangle{\pgfqpoint{0.750000in}{0.500000in}}{\pgfqpoint{4.650000in}{3.020000in}}%
\pgfusepath{clip}%
\pgfsetbuttcap%
\pgfsetmiterjoin%
\definecolor{currentfill}{rgb}{0.000000,0.500000,0.000000}%
\pgfsetfillcolor{currentfill}%
\pgfsetlinewidth{0.000000pt}%
\definecolor{currentstroke}{rgb}{0.000000,0.000000,0.000000}%
\pgfsetstrokecolor{currentstroke}%
\pgfsetstrokeopacity{0.000000}%
\pgfsetdash{}{0pt}%
\pgfpathmoveto{\pgfqpoint{4.712709in}{0.500000in}}%
\pgfpathlineto{\pgfqpoint{4.745330in}{0.500000in}}%
\pgfpathlineto{\pgfqpoint{4.745330in}{0.500000in}}%
\pgfpathlineto{\pgfqpoint{4.712709in}{0.500000in}}%
\pgfpathlineto{\pgfqpoint{4.712709in}{0.500000in}}%
\pgfpathclose%
\pgfusepath{fill}%
\end{pgfscope}%
\begin{pgfscope}%
\pgfpathrectangle{\pgfqpoint{0.750000in}{0.500000in}}{\pgfqpoint{4.650000in}{3.020000in}}%
\pgfusepath{clip}%
\pgfsetbuttcap%
\pgfsetmiterjoin%
\definecolor{currentfill}{rgb}{0.000000,0.500000,0.000000}%
\pgfsetfillcolor{currentfill}%
\pgfsetlinewidth{0.000000pt}%
\definecolor{currentstroke}{rgb}{0.000000,0.000000,0.000000}%
\pgfsetstrokecolor{currentstroke}%
\pgfsetstrokeopacity{0.000000}%
\pgfsetdash{}{0pt}%
\pgfpathmoveto{\pgfqpoint{4.745330in}{0.500000in}}%
\pgfpathlineto{\pgfqpoint{4.777950in}{0.500000in}}%
\pgfpathlineto{\pgfqpoint{4.777950in}{0.505852in}}%
\pgfpathlineto{\pgfqpoint{4.745330in}{0.505852in}}%
\pgfpathlineto{\pgfqpoint{4.745330in}{0.500000in}}%
\pgfpathclose%
\pgfusepath{fill}%
\end{pgfscope}%
\begin{pgfscope}%
\pgfpathrectangle{\pgfqpoint{0.750000in}{0.500000in}}{\pgfqpoint{4.650000in}{3.020000in}}%
\pgfusepath{clip}%
\pgfsetbuttcap%
\pgfsetmiterjoin%
\definecolor{currentfill}{rgb}{0.000000,0.500000,0.000000}%
\pgfsetfillcolor{currentfill}%
\pgfsetlinewidth{0.000000pt}%
\definecolor{currentstroke}{rgb}{0.000000,0.000000,0.000000}%
\pgfsetstrokecolor{currentstroke}%
\pgfsetstrokeopacity{0.000000}%
\pgfsetdash{}{0pt}%
\pgfpathmoveto{\pgfqpoint{4.777950in}{0.500000in}}%
\pgfpathlineto{\pgfqpoint{4.810570in}{0.500000in}}%
\pgfpathlineto{\pgfqpoint{4.810570in}{0.500000in}}%
\pgfpathlineto{\pgfqpoint{4.777950in}{0.500000in}}%
\pgfpathlineto{\pgfqpoint{4.777950in}{0.500000in}}%
\pgfpathclose%
\pgfusepath{fill}%
\end{pgfscope}%
\begin{pgfscope}%
\pgfpathrectangle{\pgfqpoint{0.750000in}{0.500000in}}{\pgfqpoint{4.650000in}{3.020000in}}%
\pgfusepath{clip}%
\pgfsetbuttcap%
\pgfsetmiterjoin%
\definecolor{currentfill}{rgb}{0.000000,0.500000,0.000000}%
\pgfsetfillcolor{currentfill}%
\pgfsetlinewidth{0.000000pt}%
\definecolor{currentstroke}{rgb}{0.000000,0.000000,0.000000}%
\pgfsetstrokecolor{currentstroke}%
\pgfsetstrokeopacity{0.000000}%
\pgfsetdash{}{0pt}%
\pgfpathmoveto{\pgfqpoint{4.810570in}{0.500000in}}%
\pgfpathlineto{\pgfqpoint{4.843191in}{0.500000in}}%
\pgfpathlineto{\pgfqpoint{4.843191in}{0.502926in}}%
\pgfpathlineto{\pgfqpoint{4.810570in}{0.502926in}}%
\pgfpathlineto{\pgfqpoint{4.810570in}{0.500000in}}%
\pgfpathclose%
\pgfusepath{fill}%
\end{pgfscope}%
\begin{pgfscope}%
\pgfpathrectangle{\pgfqpoint{0.750000in}{0.500000in}}{\pgfqpoint{4.650000in}{3.020000in}}%
\pgfusepath{clip}%
\pgfsetbuttcap%
\pgfsetmiterjoin%
\definecolor{currentfill}{rgb}{0.000000,0.500000,0.000000}%
\pgfsetfillcolor{currentfill}%
\pgfsetlinewidth{0.000000pt}%
\definecolor{currentstroke}{rgb}{0.000000,0.000000,0.000000}%
\pgfsetstrokecolor{currentstroke}%
\pgfsetstrokeopacity{0.000000}%
\pgfsetdash{}{0pt}%
\pgfpathmoveto{\pgfqpoint{4.843191in}{0.500000in}}%
\pgfpathlineto{\pgfqpoint{4.875811in}{0.500000in}}%
\pgfpathlineto{\pgfqpoint{4.875811in}{0.500000in}}%
\pgfpathlineto{\pgfqpoint{4.843191in}{0.500000in}}%
\pgfpathlineto{\pgfqpoint{4.843191in}{0.500000in}}%
\pgfpathclose%
\pgfusepath{fill}%
\end{pgfscope}%
\begin{pgfscope}%
\pgfpathrectangle{\pgfqpoint{0.750000in}{0.500000in}}{\pgfqpoint{4.650000in}{3.020000in}}%
\pgfusepath{clip}%
\pgfsetbuttcap%
\pgfsetmiterjoin%
\definecolor{currentfill}{rgb}{0.000000,0.500000,0.000000}%
\pgfsetfillcolor{currentfill}%
\pgfsetlinewidth{0.000000pt}%
\definecolor{currentstroke}{rgb}{0.000000,0.000000,0.000000}%
\pgfsetstrokecolor{currentstroke}%
\pgfsetstrokeopacity{0.000000}%
\pgfsetdash{}{0pt}%
\pgfpathmoveto{\pgfqpoint{4.875811in}{0.500000in}}%
\pgfpathlineto{\pgfqpoint{4.908432in}{0.500000in}}%
\pgfpathlineto{\pgfqpoint{4.908432in}{0.502926in}}%
\pgfpathlineto{\pgfqpoint{4.875811in}{0.502926in}}%
\pgfpathlineto{\pgfqpoint{4.875811in}{0.500000in}}%
\pgfpathclose%
\pgfusepath{fill}%
\end{pgfscope}%
\begin{pgfscope}%
\pgfpathrectangle{\pgfqpoint{0.750000in}{0.500000in}}{\pgfqpoint{4.650000in}{3.020000in}}%
\pgfusepath{clip}%
\pgfsetbuttcap%
\pgfsetmiterjoin%
\definecolor{currentfill}{rgb}{0.000000,0.500000,0.000000}%
\pgfsetfillcolor{currentfill}%
\pgfsetlinewidth{0.000000pt}%
\definecolor{currentstroke}{rgb}{0.000000,0.000000,0.000000}%
\pgfsetstrokecolor{currentstroke}%
\pgfsetstrokeopacity{0.000000}%
\pgfsetdash{}{0pt}%
\pgfpathmoveto{\pgfqpoint{4.908432in}{0.500000in}}%
\pgfpathlineto{\pgfqpoint{4.941052in}{0.500000in}}%
\pgfpathlineto{\pgfqpoint{4.941052in}{0.500000in}}%
\pgfpathlineto{\pgfqpoint{4.908432in}{0.500000in}}%
\pgfpathlineto{\pgfqpoint{4.908432in}{0.500000in}}%
\pgfpathclose%
\pgfusepath{fill}%
\end{pgfscope}%
\begin{pgfscope}%
\pgfpathrectangle{\pgfqpoint{0.750000in}{0.500000in}}{\pgfqpoint{4.650000in}{3.020000in}}%
\pgfusepath{clip}%
\pgfsetbuttcap%
\pgfsetmiterjoin%
\definecolor{currentfill}{rgb}{0.000000,0.500000,0.000000}%
\pgfsetfillcolor{currentfill}%
\pgfsetlinewidth{0.000000pt}%
\definecolor{currentstroke}{rgb}{0.000000,0.000000,0.000000}%
\pgfsetstrokecolor{currentstroke}%
\pgfsetstrokeopacity{0.000000}%
\pgfsetdash{}{0pt}%
\pgfpathmoveto{\pgfqpoint{4.941052in}{0.500000in}}%
\pgfpathlineto{\pgfqpoint{4.973672in}{0.500000in}}%
\pgfpathlineto{\pgfqpoint{4.973672in}{0.508778in}}%
\pgfpathlineto{\pgfqpoint{4.941052in}{0.508778in}}%
\pgfpathlineto{\pgfqpoint{4.941052in}{0.500000in}}%
\pgfpathclose%
\pgfusepath{fill}%
\end{pgfscope}%
\begin{pgfscope}%
\pgfpathrectangle{\pgfqpoint{0.750000in}{0.500000in}}{\pgfqpoint{4.650000in}{3.020000in}}%
\pgfusepath{clip}%
\pgfsetbuttcap%
\pgfsetmiterjoin%
\definecolor{currentfill}{rgb}{0.000000,0.500000,0.000000}%
\pgfsetfillcolor{currentfill}%
\pgfsetlinewidth{0.000000pt}%
\definecolor{currentstroke}{rgb}{0.000000,0.000000,0.000000}%
\pgfsetstrokecolor{currentstroke}%
\pgfsetstrokeopacity{0.000000}%
\pgfsetdash{}{0pt}%
\pgfpathmoveto{\pgfqpoint{4.973672in}{0.500000in}}%
\pgfpathlineto{\pgfqpoint{5.006293in}{0.500000in}}%
\pgfpathlineto{\pgfqpoint{5.006293in}{0.502926in}}%
\pgfpathlineto{\pgfqpoint{4.973672in}{0.502926in}}%
\pgfpathlineto{\pgfqpoint{4.973672in}{0.500000in}}%
\pgfpathclose%
\pgfusepath{fill}%
\end{pgfscope}%
\begin{pgfscope}%
\pgfpathrectangle{\pgfqpoint{0.750000in}{0.500000in}}{\pgfqpoint{4.650000in}{3.020000in}}%
\pgfusepath{clip}%
\pgfsetbuttcap%
\pgfsetmiterjoin%
\definecolor{currentfill}{rgb}{0.000000,0.500000,0.000000}%
\pgfsetfillcolor{currentfill}%
\pgfsetlinewidth{0.000000pt}%
\definecolor{currentstroke}{rgb}{0.000000,0.000000,0.000000}%
\pgfsetstrokecolor{currentstroke}%
\pgfsetstrokeopacity{0.000000}%
\pgfsetdash{}{0pt}%
\pgfpathmoveto{\pgfqpoint{5.006293in}{0.500000in}}%
\pgfpathlineto{\pgfqpoint{5.038913in}{0.500000in}}%
\pgfpathlineto{\pgfqpoint{5.038913in}{0.502926in}}%
\pgfpathlineto{\pgfqpoint{5.006293in}{0.502926in}}%
\pgfpathlineto{\pgfqpoint{5.006293in}{0.500000in}}%
\pgfpathclose%
\pgfusepath{fill}%
\end{pgfscope}%
\begin{pgfscope}%
\pgfpathrectangle{\pgfqpoint{0.750000in}{0.500000in}}{\pgfqpoint{4.650000in}{3.020000in}}%
\pgfusepath{clip}%
\pgfsetbuttcap%
\pgfsetmiterjoin%
\definecolor{currentfill}{rgb}{0.000000,0.500000,0.000000}%
\pgfsetfillcolor{currentfill}%
\pgfsetlinewidth{0.000000pt}%
\definecolor{currentstroke}{rgb}{0.000000,0.000000,0.000000}%
\pgfsetstrokecolor{currentstroke}%
\pgfsetstrokeopacity{0.000000}%
\pgfsetdash{}{0pt}%
\pgfpathmoveto{\pgfqpoint{5.038913in}{0.500000in}}%
\pgfpathlineto{\pgfqpoint{5.071534in}{0.500000in}}%
\pgfpathlineto{\pgfqpoint{5.071534in}{0.505852in}}%
\pgfpathlineto{\pgfqpoint{5.038913in}{0.505852in}}%
\pgfpathlineto{\pgfqpoint{5.038913in}{0.500000in}}%
\pgfpathclose%
\pgfusepath{fill}%
\end{pgfscope}%
\begin{pgfscope}%
\pgfpathrectangle{\pgfqpoint{0.750000in}{0.500000in}}{\pgfqpoint{4.650000in}{3.020000in}}%
\pgfusepath{clip}%
\pgfsetbuttcap%
\pgfsetmiterjoin%
\definecolor{currentfill}{rgb}{0.000000,0.500000,0.000000}%
\pgfsetfillcolor{currentfill}%
\pgfsetlinewidth{0.000000pt}%
\definecolor{currentstroke}{rgb}{0.000000,0.000000,0.000000}%
\pgfsetstrokecolor{currentstroke}%
\pgfsetstrokeopacity{0.000000}%
\pgfsetdash{}{0pt}%
\pgfpathmoveto{\pgfqpoint{5.071534in}{0.500000in}}%
\pgfpathlineto{\pgfqpoint{5.104154in}{0.500000in}}%
\pgfpathlineto{\pgfqpoint{5.104154in}{0.502926in}}%
\pgfpathlineto{\pgfqpoint{5.071534in}{0.502926in}}%
\pgfpathlineto{\pgfqpoint{5.071534in}{0.500000in}}%
\pgfpathclose%
\pgfusepath{fill}%
\end{pgfscope}%
\begin{pgfscope}%
\pgfpathrectangle{\pgfqpoint{0.750000in}{0.500000in}}{\pgfqpoint{4.650000in}{3.020000in}}%
\pgfusepath{clip}%
\pgfsetbuttcap%
\pgfsetmiterjoin%
\definecolor{currentfill}{rgb}{0.000000,0.500000,0.000000}%
\pgfsetfillcolor{currentfill}%
\pgfsetlinewidth{0.000000pt}%
\definecolor{currentstroke}{rgb}{0.000000,0.000000,0.000000}%
\pgfsetstrokecolor{currentstroke}%
\pgfsetstrokeopacity{0.000000}%
\pgfsetdash{}{0pt}%
\pgfpathmoveto{\pgfqpoint{5.104154in}{0.500000in}}%
\pgfpathlineto{\pgfqpoint{5.136774in}{0.500000in}}%
\pgfpathlineto{\pgfqpoint{5.136774in}{0.508778in}}%
\pgfpathlineto{\pgfqpoint{5.104154in}{0.508778in}}%
\pgfpathlineto{\pgfqpoint{5.104154in}{0.500000in}}%
\pgfpathclose%
\pgfusepath{fill}%
\end{pgfscope}%
\begin{pgfscope}%
\pgfsetbuttcap%
\pgfsetroundjoin%
\definecolor{currentfill}{rgb}{0.000000,0.000000,0.000000}%
\pgfsetfillcolor{currentfill}%
\pgfsetlinewidth{0.803000pt}%
\definecolor{currentstroke}{rgb}{0.000000,0.000000,0.000000}%
\pgfsetstrokecolor{currentstroke}%
\pgfsetdash{}{0pt}%
\pgfsys@defobject{currentmarker}{\pgfqpoint{0.000000in}{-0.048611in}}{\pgfqpoint{0.000000in}{0.000000in}}{%
\pgfpathmoveto{\pgfqpoint{0.000000in}{0.000000in}}%
\pgfpathlineto{\pgfqpoint{0.000000in}{-0.048611in}}%
\pgfusepath{stroke,fill}%
}%
\begin{pgfscope}%
\pgfsys@transformshift{1.239896in}{0.500000in}%
\pgfsys@useobject{currentmarker}{}%
\end{pgfscope}%
\end{pgfscope}%
\begin{pgfscope}%
\definecolor{textcolor}{rgb}{0.000000,0.000000,0.000000}%
\pgfsetstrokecolor{textcolor}%
\pgfsetfillcolor{textcolor}%
\pgftext[x=1.239896in,y=0.402778in,,top]{\color{textcolor}\sffamily\fontsize{10.000000}{12.000000}\selectfont 0.15}%
\end{pgfscope}%
\begin{pgfscope}%
\pgfsetbuttcap%
\pgfsetroundjoin%
\definecolor{currentfill}{rgb}{0.000000,0.000000,0.000000}%
\pgfsetfillcolor{currentfill}%
\pgfsetlinewidth{0.803000pt}%
\definecolor{currentstroke}{rgb}{0.000000,0.000000,0.000000}%
\pgfsetstrokecolor{currentstroke}%
\pgfsetdash{}{0pt}%
\pgfsys@defobject{currentmarker}{\pgfqpoint{0.000000in}{-0.048611in}}{\pgfqpoint{0.000000in}{0.000000in}}{%
\pgfpathmoveto{\pgfqpoint{0.000000in}{0.000000in}}%
\pgfpathlineto{\pgfqpoint{0.000000in}{-0.048611in}}%
\pgfusepath{stroke,fill}%
}%
\begin{pgfscope}%
\pgfsys@transformshift{1.798964in}{0.500000in}%
\pgfsys@useobject{currentmarker}{}%
\end{pgfscope}%
\end{pgfscope}%
\begin{pgfscope}%
\definecolor{textcolor}{rgb}{0.000000,0.000000,0.000000}%
\pgfsetstrokecolor{textcolor}%
\pgfsetfillcolor{textcolor}%
\pgftext[x=1.798964in,y=0.402778in,,top]{\color{textcolor}\sffamily\fontsize{10.000000}{12.000000}\selectfont 0.20}%
\end{pgfscope}%
\begin{pgfscope}%
\pgfsetbuttcap%
\pgfsetroundjoin%
\definecolor{currentfill}{rgb}{0.000000,0.000000,0.000000}%
\pgfsetfillcolor{currentfill}%
\pgfsetlinewidth{0.803000pt}%
\definecolor{currentstroke}{rgb}{0.000000,0.000000,0.000000}%
\pgfsetstrokecolor{currentstroke}%
\pgfsetdash{}{0pt}%
\pgfsys@defobject{currentmarker}{\pgfqpoint{0.000000in}{-0.048611in}}{\pgfqpoint{0.000000in}{0.000000in}}{%
\pgfpathmoveto{\pgfqpoint{0.000000in}{0.000000in}}%
\pgfpathlineto{\pgfqpoint{0.000000in}{-0.048611in}}%
\pgfusepath{stroke,fill}%
}%
\begin{pgfscope}%
\pgfsys@transformshift{2.358033in}{0.500000in}%
\pgfsys@useobject{currentmarker}{}%
\end{pgfscope}%
\end{pgfscope}%
\begin{pgfscope}%
\definecolor{textcolor}{rgb}{0.000000,0.000000,0.000000}%
\pgfsetstrokecolor{textcolor}%
\pgfsetfillcolor{textcolor}%
\pgftext[x=2.358033in,y=0.402778in,,top]{\color{textcolor}\sffamily\fontsize{10.000000}{12.000000}\selectfont 0.25}%
\end{pgfscope}%
\begin{pgfscope}%
\pgfsetbuttcap%
\pgfsetroundjoin%
\definecolor{currentfill}{rgb}{0.000000,0.000000,0.000000}%
\pgfsetfillcolor{currentfill}%
\pgfsetlinewidth{0.803000pt}%
\definecolor{currentstroke}{rgb}{0.000000,0.000000,0.000000}%
\pgfsetstrokecolor{currentstroke}%
\pgfsetdash{}{0pt}%
\pgfsys@defobject{currentmarker}{\pgfqpoint{0.000000in}{-0.048611in}}{\pgfqpoint{0.000000in}{0.000000in}}{%
\pgfpathmoveto{\pgfqpoint{0.000000in}{0.000000in}}%
\pgfpathlineto{\pgfqpoint{0.000000in}{-0.048611in}}%
\pgfusepath{stroke,fill}%
}%
\begin{pgfscope}%
\pgfsys@transformshift{2.917102in}{0.500000in}%
\pgfsys@useobject{currentmarker}{}%
\end{pgfscope}%
\end{pgfscope}%
\begin{pgfscope}%
\definecolor{textcolor}{rgb}{0.000000,0.000000,0.000000}%
\pgfsetstrokecolor{textcolor}%
\pgfsetfillcolor{textcolor}%
\pgftext[x=2.917102in,y=0.402778in,,top]{\color{textcolor}\sffamily\fontsize{10.000000}{12.000000}\selectfont 0.30}%
\end{pgfscope}%
\begin{pgfscope}%
\pgfsetbuttcap%
\pgfsetroundjoin%
\definecolor{currentfill}{rgb}{0.000000,0.000000,0.000000}%
\pgfsetfillcolor{currentfill}%
\pgfsetlinewidth{0.803000pt}%
\definecolor{currentstroke}{rgb}{0.000000,0.000000,0.000000}%
\pgfsetstrokecolor{currentstroke}%
\pgfsetdash{}{0pt}%
\pgfsys@defobject{currentmarker}{\pgfqpoint{0.000000in}{-0.048611in}}{\pgfqpoint{0.000000in}{0.000000in}}{%
\pgfpathmoveto{\pgfqpoint{0.000000in}{0.000000in}}%
\pgfpathlineto{\pgfqpoint{0.000000in}{-0.048611in}}%
\pgfusepath{stroke,fill}%
}%
\begin{pgfscope}%
\pgfsys@transformshift{3.476171in}{0.500000in}%
\pgfsys@useobject{currentmarker}{}%
\end{pgfscope}%
\end{pgfscope}%
\begin{pgfscope}%
\definecolor{textcolor}{rgb}{0.000000,0.000000,0.000000}%
\pgfsetstrokecolor{textcolor}%
\pgfsetfillcolor{textcolor}%
\pgftext[x=3.476171in,y=0.402778in,,top]{\color{textcolor}\sffamily\fontsize{10.000000}{12.000000}\selectfont 0.35}%
\end{pgfscope}%
\begin{pgfscope}%
\pgfsetbuttcap%
\pgfsetroundjoin%
\definecolor{currentfill}{rgb}{0.000000,0.000000,0.000000}%
\pgfsetfillcolor{currentfill}%
\pgfsetlinewidth{0.803000pt}%
\definecolor{currentstroke}{rgb}{0.000000,0.000000,0.000000}%
\pgfsetstrokecolor{currentstroke}%
\pgfsetdash{}{0pt}%
\pgfsys@defobject{currentmarker}{\pgfqpoint{0.000000in}{-0.048611in}}{\pgfqpoint{0.000000in}{0.000000in}}{%
\pgfpathmoveto{\pgfqpoint{0.000000in}{0.000000in}}%
\pgfpathlineto{\pgfqpoint{0.000000in}{-0.048611in}}%
\pgfusepath{stroke,fill}%
}%
\begin{pgfscope}%
\pgfsys@transformshift{4.035240in}{0.500000in}%
\pgfsys@useobject{currentmarker}{}%
\end{pgfscope}%
\end{pgfscope}%
\begin{pgfscope}%
\definecolor{textcolor}{rgb}{0.000000,0.000000,0.000000}%
\pgfsetstrokecolor{textcolor}%
\pgfsetfillcolor{textcolor}%
\pgftext[x=4.035240in,y=0.402778in,,top]{\color{textcolor}\sffamily\fontsize{10.000000}{12.000000}\selectfont 0.40}%
\end{pgfscope}%
\begin{pgfscope}%
\pgfsetbuttcap%
\pgfsetroundjoin%
\definecolor{currentfill}{rgb}{0.000000,0.000000,0.000000}%
\pgfsetfillcolor{currentfill}%
\pgfsetlinewidth{0.803000pt}%
\definecolor{currentstroke}{rgb}{0.000000,0.000000,0.000000}%
\pgfsetstrokecolor{currentstroke}%
\pgfsetdash{}{0pt}%
\pgfsys@defobject{currentmarker}{\pgfqpoint{0.000000in}{-0.048611in}}{\pgfqpoint{0.000000in}{0.000000in}}{%
\pgfpathmoveto{\pgfqpoint{0.000000in}{0.000000in}}%
\pgfpathlineto{\pgfqpoint{0.000000in}{-0.048611in}}%
\pgfusepath{stroke,fill}%
}%
\begin{pgfscope}%
\pgfsys@transformshift{4.594308in}{0.500000in}%
\pgfsys@useobject{currentmarker}{}%
\end{pgfscope}%
\end{pgfscope}%
\begin{pgfscope}%
\definecolor{textcolor}{rgb}{0.000000,0.000000,0.000000}%
\pgfsetstrokecolor{textcolor}%
\pgfsetfillcolor{textcolor}%
\pgftext[x=4.594308in,y=0.402778in,,top]{\color{textcolor}\sffamily\fontsize{10.000000}{12.000000}\selectfont 0.45}%
\end{pgfscope}%
\begin{pgfscope}%
\pgfsetbuttcap%
\pgfsetroundjoin%
\definecolor{currentfill}{rgb}{0.000000,0.000000,0.000000}%
\pgfsetfillcolor{currentfill}%
\pgfsetlinewidth{0.803000pt}%
\definecolor{currentstroke}{rgb}{0.000000,0.000000,0.000000}%
\pgfsetstrokecolor{currentstroke}%
\pgfsetdash{}{0pt}%
\pgfsys@defobject{currentmarker}{\pgfqpoint{0.000000in}{-0.048611in}}{\pgfqpoint{0.000000in}{0.000000in}}{%
\pgfpathmoveto{\pgfqpoint{0.000000in}{0.000000in}}%
\pgfpathlineto{\pgfqpoint{0.000000in}{-0.048611in}}%
\pgfusepath{stroke,fill}%
}%
\begin{pgfscope}%
\pgfsys@transformshift{5.153377in}{0.500000in}%
\pgfsys@useobject{currentmarker}{}%
\end{pgfscope}%
\end{pgfscope}%
\begin{pgfscope}%
\definecolor{textcolor}{rgb}{0.000000,0.000000,0.000000}%
\pgfsetstrokecolor{textcolor}%
\pgfsetfillcolor{textcolor}%
\pgftext[x=5.153377in,y=0.402778in,,top]{\color{textcolor}\sffamily\fontsize{10.000000}{12.000000}\selectfont 0.50}%
\end{pgfscope}%
\begin{pgfscope}%
\definecolor{textcolor}{rgb}{0.000000,0.000000,0.000000}%
\pgfsetstrokecolor{textcolor}%
\pgfsetfillcolor{textcolor}%
\pgftext[x=3.075000in,y=0.212809in,,top]{\color{textcolor}\sffamily\fontsize{10.000000}{12.000000}\selectfont Loss}%
\end{pgfscope}%
\begin{pgfscope}%
\pgfsetbuttcap%
\pgfsetroundjoin%
\definecolor{currentfill}{rgb}{0.000000,0.000000,0.000000}%
\pgfsetfillcolor{currentfill}%
\pgfsetlinewidth{0.803000pt}%
\definecolor{currentstroke}{rgb}{0.000000,0.000000,0.000000}%
\pgfsetstrokecolor{currentstroke}%
\pgfsetdash{}{0pt}%
\pgfsys@defobject{currentmarker}{\pgfqpoint{-0.048611in}{0.000000in}}{\pgfqpoint{-0.000000in}{0.000000in}}{%
\pgfpathmoveto{\pgfqpoint{-0.000000in}{0.000000in}}%
\pgfpathlineto{\pgfqpoint{-0.048611in}{0.000000in}}%
\pgfusepath{stroke,fill}%
}%
\begin{pgfscope}%
\pgfsys@transformshift{0.750000in}{0.500000in}%
\pgfsys@useobject{currentmarker}{}%
\end{pgfscope}%
\end{pgfscope}%
\begin{pgfscope}%
\definecolor{textcolor}{rgb}{0.000000,0.000000,0.000000}%
\pgfsetstrokecolor{textcolor}%
\pgfsetfillcolor{textcolor}%
\pgftext[x=0.564412in, y=0.447238in, left, base]{\color{textcolor}\sffamily\fontsize{10.000000}{12.000000}\selectfont 0}%
\end{pgfscope}%
\begin{pgfscope}%
\pgfsetbuttcap%
\pgfsetroundjoin%
\definecolor{currentfill}{rgb}{0.000000,0.000000,0.000000}%
\pgfsetfillcolor{currentfill}%
\pgfsetlinewidth{0.803000pt}%
\definecolor{currentstroke}{rgb}{0.000000,0.000000,0.000000}%
\pgfsetstrokecolor{currentstroke}%
\pgfsetdash{}{0pt}%
\pgfsys@defobject{currentmarker}{\pgfqpoint{-0.048611in}{0.000000in}}{\pgfqpoint{-0.000000in}{0.000000in}}{%
\pgfpathmoveto{\pgfqpoint{-0.000000in}{0.000000in}}%
\pgfpathlineto{\pgfqpoint{-0.048611in}{0.000000in}}%
\pgfusepath{stroke,fill}%
}%
\begin{pgfscope}%
\pgfsys@transformshift{0.750000in}{1.085186in}%
\pgfsys@useobject{currentmarker}{}%
\end{pgfscope}%
\end{pgfscope}%
\begin{pgfscope}%
\definecolor{textcolor}{rgb}{0.000000,0.000000,0.000000}%
\pgfsetstrokecolor{textcolor}%
\pgfsetfillcolor{textcolor}%
\pgftext[x=0.387682in, y=1.032425in, left, base]{\color{textcolor}\sffamily\fontsize{10.000000}{12.000000}\selectfont 200}%
\end{pgfscope}%
\begin{pgfscope}%
\pgfsetbuttcap%
\pgfsetroundjoin%
\definecolor{currentfill}{rgb}{0.000000,0.000000,0.000000}%
\pgfsetfillcolor{currentfill}%
\pgfsetlinewidth{0.803000pt}%
\definecolor{currentstroke}{rgb}{0.000000,0.000000,0.000000}%
\pgfsetstrokecolor{currentstroke}%
\pgfsetdash{}{0pt}%
\pgfsys@defobject{currentmarker}{\pgfqpoint{-0.048611in}{0.000000in}}{\pgfqpoint{-0.000000in}{0.000000in}}{%
\pgfpathmoveto{\pgfqpoint{-0.000000in}{0.000000in}}%
\pgfpathlineto{\pgfqpoint{-0.048611in}{0.000000in}}%
\pgfusepath{stroke,fill}%
}%
\begin{pgfscope}%
\pgfsys@transformshift{0.750000in}{1.670373in}%
\pgfsys@useobject{currentmarker}{}%
\end{pgfscope}%
\end{pgfscope}%
\begin{pgfscope}%
\definecolor{textcolor}{rgb}{0.000000,0.000000,0.000000}%
\pgfsetstrokecolor{textcolor}%
\pgfsetfillcolor{textcolor}%
\pgftext[x=0.387682in, y=1.617611in, left, base]{\color{textcolor}\sffamily\fontsize{10.000000}{12.000000}\selectfont 400}%
\end{pgfscope}%
\begin{pgfscope}%
\pgfsetbuttcap%
\pgfsetroundjoin%
\definecolor{currentfill}{rgb}{0.000000,0.000000,0.000000}%
\pgfsetfillcolor{currentfill}%
\pgfsetlinewidth{0.803000pt}%
\definecolor{currentstroke}{rgb}{0.000000,0.000000,0.000000}%
\pgfsetstrokecolor{currentstroke}%
\pgfsetdash{}{0pt}%
\pgfsys@defobject{currentmarker}{\pgfqpoint{-0.048611in}{0.000000in}}{\pgfqpoint{-0.000000in}{0.000000in}}{%
\pgfpathmoveto{\pgfqpoint{-0.000000in}{0.000000in}}%
\pgfpathlineto{\pgfqpoint{-0.048611in}{0.000000in}}%
\pgfusepath{stroke,fill}%
}%
\begin{pgfscope}%
\pgfsys@transformshift{0.750000in}{2.255559in}%
\pgfsys@useobject{currentmarker}{}%
\end{pgfscope}%
\end{pgfscope}%
\begin{pgfscope}%
\definecolor{textcolor}{rgb}{0.000000,0.000000,0.000000}%
\pgfsetstrokecolor{textcolor}%
\pgfsetfillcolor{textcolor}%
\pgftext[x=0.387682in, y=2.202797in, left, base]{\color{textcolor}\sffamily\fontsize{10.000000}{12.000000}\selectfont 600}%
\end{pgfscope}%
\begin{pgfscope}%
\pgfsetbuttcap%
\pgfsetroundjoin%
\definecolor{currentfill}{rgb}{0.000000,0.000000,0.000000}%
\pgfsetfillcolor{currentfill}%
\pgfsetlinewidth{0.803000pt}%
\definecolor{currentstroke}{rgb}{0.000000,0.000000,0.000000}%
\pgfsetstrokecolor{currentstroke}%
\pgfsetdash{}{0pt}%
\pgfsys@defobject{currentmarker}{\pgfqpoint{-0.048611in}{0.000000in}}{\pgfqpoint{-0.000000in}{0.000000in}}{%
\pgfpathmoveto{\pgfqpoint{-0.000000in}{0.000000in}}%
\pgfpathlineto{\pgfqpoint{-0.048611in}{0.000000in}}%
\pgfusepath{stroke,fill}%
}%
\begin{pgfscope}%
\pgfsys@transformshift{0.750000in}{2.840745in}%
\pgfsys@useobject{currentmarker}{}%
\end{pgfscope}%
\end{pgfscope}%
\begin{pgfscope}%
\definecolor{textcolor}{rgb}{0.000000,0.000000,0.000000}%
\pgfsetstrokecolor{textcolor}%
\pgfsetfillcolor{textcolor}%
\pgftext[x=0.387682in, y=2.787984in, left, base]{\color{textcolor}\sffamily\fontsize{10.000000}{12.000000}\selectfont 800}%
\end{pgfscope}%
\begin{pgfscope}%
\pgfsetbuttcap%
\pgfsetroundjoin%
\definecolor{currentfill}{rgb}{0.000000,0.000000,0.000000}%
\pgfsetfillcolor{currentfill}%
\pgfsetlinewidth{0.803000pt}%
\definecolor{currentstroke}{rgb}{0.000000,0.000000,0.000000}%
\pgfsetstrokecolor{currentstroke}%
\pgfsetdash{}{0pt}%
\pgfsys@defobject{currentmarker}{\pgfqpoint{-0.048611in}{0.000000in}}{\pgfqpoint{-0.000000in}{0.000000in}}{%
\pgfpathmoveto{\pgfqpoint{-0.000000in}{0.000000in}}%
\pgfpathlineto{\pgfqpoint{-0.048611in}{0.000000in}}%
\pgfusepath{stroke,fill}%
}%
\begin{pgfscope}%
\pgfsys@transformshift{0.750000in}{3.425931in}%
\pgfsys@useobject{currentmarker}{}%
\end{pgfscope}%
\end{pgfscope}%
\begin{pgfscope}%
\definecolor{textcolor}{rgb}{0.000000,0.000000,0.000000}%
\pgfsetstrokecolor{textcolor}%
\pgfsetfillcolor{textcolor}%
\pgftext[x=0.299316in, y=3.373170in, left, base]{\color{textcolor}\sffamily\fontsize{10.000000}{12.000000}\selectfont 1000}%
\end{pgfscope}%
\begin{pgfscope}%
\definecolor{textcolor}{rgb}{0.000000,0.000000,0.000000}%
\pgfsetstrokecolor{textcolor}%
\pgfsetfillcolor{textcolor}%
\pgftext[x=0.243761in,y=2.010000in,,bottom,rotate=90.000000]{\color{textcolor}\sffamily\fontsize{10.000000}{12.000000}\selectfont Count}%
\end{pgfscope}%
\begin{pgfscope}%
\pgfsetrectcap%
\pgfsetmiterjoin%
\pgfsetlinewidth{0.803000pt}%
\definecolor{currentstroke}{rgb}{0.000000,0.000000,0.000000}%
\pgfsetstrokecolor{currentstroke}%
\pgfsetdash{}{0pt}%
\pgfpathmoveto{\pgfqpoint{0.750000in}{0.500000in}}%
\pgfpathlineto{\pgfqpoint{0.750000in}{3.520000in}}%
\pgfusepath{stroke}%
\end{pgfscope}%
\begin{pgfscope}%
\pgfsetrectcap%
\pgfsetmiterjoin%
\pgfsetlinewidth{0.803000pt}%
\definecolor{currentstroke}{rgb}{0.000000,0.000000,0.000000}%
\pgfsetstrokecolor{currentstroke}%
\pgfsetdash{}{0pt}%
\pgfpathmoveto{\pgfqpoint{5.400000in}{0.500000in}}%
\pgfpathlineto{\pgfqpoint{5.400000in}{3.520000in}}%
\pgfusepath{stroke}%
\end{pgfscope}%
\begin{pgfscope}%
\pgfsetrectcap%
\pgfsetmiterjoin%
\pgfsetlinewidth{0.803000pt}%
\definecolor{currentstroke}{rgb}{0.000000,0.000000,0.000000}%
\pgfsetstrokecolor{currentstroke}%
\pgfsetdash{}{0pt}%
\pgfpathmoveto{\pgfqpoint{0.750000in}{0.500000in}}%
\pgfpathlineto{\pgfqpoint{5.400000in}{0.500000in}}%
\pgfusepath{stroke}%
\end{pgfscope}%
\begin{pgfscope}%
\pgfsetrectcap%
\pgfsetmiterjoin%
\pgfsetlinewidth{0.803000pt}%
\definecolor{currentstroke}{rgb}{0.000000,0.000000,0.000000}%
\pgfsetstrokecolor{currentstroke}%
\pgfsetdash{}{0pt}%
\pgfpathmoveto{\pgfqpoint{0.750000in}{3.520000in}}%
\pgfpathlineto{\pgfqpoint{5.400000in}{3.520000in}}%
\pgfusepath{stroke}%
\end{pgfscope}%
\begin{pgfscope}%
\definecolor{textcolor}{rgb}{0.000000,0.000000,0.000000}%
\pgfsetstrokecolor{textcolor}%
\pgfsetfillcolor{textcolor}%
\pgftext[x=3.075000in,y=3.603333in,,base]{\color{textcolor}\sffamily\fontsize{12.000000}{14.400000}\selectfont Loss Histogram for \(\displaystyle f(x)=x^2\)}%
\end{pgfscope}%
\begin{pgfscope}%
\pgfsetbuttcap%
\pgfsetmiterjoin%
\definecolor{currentfill}{rgb}{1.000000,1.000000,1.000000}%
\pgfsetfillcolor{currentfill}%
\pgfsetfillopacity{0.800000}%
\pgfsetlinewidth{1.003750pt}%
\definecolor{currentstroke}{rgb}{0.800000,0.800000,0.800000}%
\pgfsetstrokecolor{currentstroke}%
\pgfsetstrokeopacity{0.800000}%
\pgfsetdash{}{0pt}%
\pgfpathmoveto{\pgfqpoint{4.562381in}{3.001174in}}%
\pgfpathlineto{\pgfqpoint{5.302778in}{3.001174in}}%
\pgfpathquadraticcurveto{\pgfqpoint{5.330556in}{3.001174in}}{\pgfqpoint{5.330556in}{3.028952in}}%
\pgfpathlineto{\pgfqpoint{5.330556in}{3.422778in}}%
\pgfpathquadraticcurveto{\pgfqpoint{5.330556in}{3.450556in}}{\pgfqpoint{5.302778in}{3.450556in}}%
\pgfpathlineto{\pgfqpoint{4.562381in}{3.450556in}}%
\pgfpathquadraticcurveto{\pgfqpoint{4.534603in}{3.450556in}}{\pgfqpoint{4.534603in}{3.422778in}}%
\pgfpathlineto{\pgfqpoint{4.534603in}{3.028952in}}%
\pgfpathquadraticcurveto{\pgfqpoint{4.534603in}{3.001174in}}{\pgfqpoint{4.562381in}{3.001174in}}%
\pgfpathlineto{\pgfqpoint{4.562381in}{3.001174in}}%
\pgfpathclose%
\pgfusepath{stroke,fill}%
\end{pgfscope}%
\begin{pgfscope}%
\pgfsetbuttcap%
\pgfsetmiterjoin%
\definecolor{currentfill}{rgb}{1.000000,0.000000,0.000000}%
\pgfsetfillcolor{currentfill}%
\pgfsetlinewidth{0.000000pt}%
\definecolor{currentstroke}{rgb}{0.000000,0.000000,0.000000}%
\pgfsetstrokecolor{currentstroke}%
\pgfsetstrokeopacity{0.000000}%
\pgfsetdash{}{0pt}%
\pgfpathmoveto{\pgfqpoint{4.590158in}{3.289477in}}%
\pgfpathlineto{\pgfqpoint{4.867936in}{3.289477in}}%
\pgfpathlineto{\pgfqpoint{4.867936in}{3.386699in}}%
\pgfpathlineto{\pgfqpoint{4.590158in}{3.386699in}}%
\pgfpathlineto{\pgfqpoint{4.590158in}{3.289477in}}%
\pgfpathclose%
\pgfusepath{fill}%
\end{pgfscope}%
\begin{pgfscope}%
\definecolor{textcolor}{rgb}{0.000000,0.000000,0.000000}%
\pgfsetstrokecolor{textcolor}%
\pgfsetfillcolor{textcolor}%
\pgftext[x=4.979047in,y=3.289477in,left,base]{\color{textcolor}\sffamily\fontsize{10.000000}{12.000000}\selectfont SNN}%
\end{pgfscope}%
\begin{pgfscope}%
\pgfsetbuttcap%
\pgfsetmiterjoin%
\definecolor{currentfill}{rgb}{0.000000,0.500000,0.000000}%
\pgfsetfillcolor{currentfill}%
\pgfsetlinewidth{0.000000pt}%
\definecolor{currentstroke}{rgb}{0.000000,0.000000,0.000000}%
\pgfsetstrokecolor{currentstroke}%
\pgfsetstrokeopacity{0.000000}%
\pgfsetdash{}{0pt}%
\pgfpathmoveto{\pgfqpoint{4.590158in}{3.085620in}}%
\pgfpathlineto{\pgfqpoint{4.867936in}{3.085620in}}%
\pgfpathlineto{\pgfqpoint{4.867936in}{3.182842in}}%
\pgfpathlineto{\pgfqpoint{4.590158in}{3.182842in}}%
\pgfpathlineto{\pgfqpoint{4.590158in}{3.085620in}}%
\pgfpathclose%
\pgfusepath{fill}%
\end{pgfscope}%
\begin{pgfscope}%
\definecolor{textcolor}{rgb}{0.000000,0.000000,0.000000}%
\pgfsetstrokecolor{textcolor}%
\pgfsetfillcolor{textcolor}%
\pgftext[x=4.979047in,y=3.085620in,left,base]{\color{textcolor}\sffamily\fontsize{10.000000}{12.000000}\selectfont NN}%
\end{pgfscope}%
\end{pgfpicture}%
\makeatother%
\endgroup%

    \caption{Caption}
    \label{fig:my_label}
\end{figure}

\begin{figure}
%% Creator: Matplotlib, PGF backend
%%
%% To include the figure in your LaTeX document, write
%%   \input{<filename>.pgf}
%%
%% Make sure the required packages are loaded in your preamble
%%   \usepackage{pgf}
%%
%% Also ensure that all the required font packages are loaded; for instance,
%% the lmodern package is sometimes necessary when using math font.
%%   \usepackage{lmodern}
%%
%% Figures using additional raster images can only be included by \input if
%% they are in the same directory as the main LaTeX file. For loading figures
%% from other directories you can use the `import` package
%%   \usepackage{import}
%%
%% and then include the figures with
%%   \import{<path to file>}{<filename>.pgf}
%%
%% Matplotlib used the following preamble
%%   \usepackage{fontspec}
%%   \setmainfont{DejaVuSerif.ttf}[Path=\detokenize{/Users/mkojro/miniforge3/envs/nn-crypto/lib/python3.10/site-packages/matplotlib/mpl-data/fonts/ttf/}]
%%   \setsansfont{DejaVuSans.ttf}[Path=\detokenize{/Users/mkojro/miniforge3/envs/nn-crypto/lib/python3.10/site-packages/matplotlib/mpl-data/fonts/ttf/}]
%%   \setmonofont{DejaVuSansMono.ttf}[Path=\detokenize{/Users/mkojro/miniforge3/envs/nn-crypto/lib/python3.10/site-packages/matplotlib/mpl-data/fonts/ttf/}]
%%
\begingroup%
\makeatletter%
\begin{pgfpicture}%
\pgfpathrectangle{\pgfpointorigin}{\pgfqpoint{6.000000in}{4.000000in}}%
\pgfusepath{use as bounding box, clip}%
\begin{pgfscope}%
\pgfsetbuttcap%
\pgfsetmiterjoin%
\pgfsetlinewidth{0.000000pt}%
\definecolor{currentstroke}{rgb}{1.000000,1.000000,1.000000}%
\pgfsetstrokecolor{currentstroke}%
\pgfsetstrokeopacity{0.000000}%
\pgfsetdash{}{0pt}%
\pgfpathmoveto{\pgfqpoint{0.000000in}{0.000000in}}%
\pgfpathlineto{\pgfqpoint{6.000000in}{0.000000in}}%
\pgfpathlineto{\pgfqpoint{6.000000in}{4.000000in}}%
\pgfpathlineto{\pgfqpoint{0.000000in}{4.000000in}}%
\pgfpathlineto{\pgfqpoint{0.000000in}{0.000000in}}%
\pgfpathclose%
\pgfusepath{}%
\end{pgfscope}%
\begin{pgfscope}%
\pgfsetbuttcap%
\pgfsetmiterjoin%
\definecolor{currentfill}{rgb}{1.000000,1.000000,1.000000}%
\pgfsetfillcolor{currentfill}%
\pgfsetlinewidth{0.000000pt}%
\definecolor{currentstroke}{rgb}{0.000000,0.000000,0.000000}%
\pgfsetstrokecolor{currentstroke}%
\pgfsetstrokeopacity{0.000000}%
\pgfsetdash{}{0pt}%
\pgfpathmoveto{\pgfqpoint{0.750000in}{0.500000in}}%
\pgfpathlineto{\pgfqpoint{5.400000in}{0.500000in}}%
\pgfpathlineto{\pgfqpoint{5.400000in}{3.520000in}}%
\pgfpathlineto{\pgfqpoint{0.750000in}{3.520000in}}%
\pgfpathlineto{\pgfqpoint{0.750000in}{0.500000in}}%
\pgfpathclose%
\pgfusepath{fill}%
\end{pgfscope}%
\begin{pgfscope}%
\pgfpathrectangle{\pgfqpoint{0.750000in}{0.500000in}}{\pgfqpoint{4.650000in}{3.020000in}}%
\pgfusepath{clip}%
\pgfsetbuttcap%
\pgfsetmiterjoin%
\definecolor{currentfill}{rgb}{0.121569,0.466667,0.705882}%
\pgfsetfillcolor{currentfill}%
\pgfsetlinewidth{0.000000pt}%
\definecolor{currentstroke}{rgb}{0.000000,0.000000,0.000000}%
\pgfsetstrokecolor{currentstroke}%
\pgfsetstrokeopacity{0.000000}%
\pgfsetdash{}{0pt}%
\pgfpathmoveto{\pgfqpoint{0.961364in}{0.500000in}}%
\pgfpathlineto{\pgfqpoint{0.987825in}{0.500000in}}%
\pgfpathlineto{\pgfqpoint{0.987825in}{0.500000in}}%
\pgfpathlineto{\pgfqpoint{0.961364in}{0.500000in}}%
\pgfpathlineto{\pgfqpoint{0.961364in}{0.500000in}}%
\pgfpathclose%
\pgfusepath{fill}%
\end{pgfscope}%
\begin{pgfscope}%
\pgfpathrectangle{\pgfqpoint{0.750000in}{0.500000in}}{\pgfqpoint{4.650000in}{3.020000in}}%
\pgfusepath{clip}%
\pgfsetbuttcap%
\pgfsetmiterjoin%
\definecolor{currentfill}{rgb}{0.121569,0.466667,0.705882}%
\pgfsetfillcolor{currentfill}%
\pgfsetlinewidth{0.000000pt}%
\definecolor{currentstroke}{rgb}{0.000000,0.000000,0.000000}%
\pgfsetstrokecolor{currentstroke}%
\pgfsetstrokeopacity{0.000000}%
\pgfsetdash{}{0pt}%
\pgfpathmoveto{\pgfqpoint{0.994441in}{0.500000in}}%
\pgfpathlineto{\pgfqpoint{1.020903in}{0.500000in}}%
\pgfpathlineto{\pgfqpoint{1.020903in}{0.500000in}}%
\pgfpathlineto{\pgfqpoint{0.994441in}{0.500000in}}%
\pgfpathlineto{\pgfqpoint{0.994441in}{0.500000in}}%
\pgfpathclose%
\pgfusepath{fill}%
\end{pgfscope}%
\begin{pgfscope}%
\pgfpathrectangle{\pgfqpoint{0.750000in}{0.500000in}}{\pgfqpoint{4.650000in}{3.020000in}}%
\pgfusepath{clip}%
\pgfsetbuttcap%
\pgfsetmiterjoin%
\definecolor{currentfill}{rgb}{0.121569,0.466667,0.705882}%
\pgfsetfillcolor{currentfill}%
\pgfsetlinewidth{0.000000pt}%
\definecolor{currentstroke}{rgb}{0.000000,0.000000,0.000000}%
\pgfsetstrokecolor{currentstroke}%
\pgfsetstrokeopacity{0.000000}%
\pgfsetdash{}{0pt}%
\pgfpathmoveto{\pgfqpoint{1.027518in}{0.500000in}}%
\pgfpathlineto{\pgfqpoint{1.053980in}{0.500000in}}%
\pgfpathlineto{\pgfqpoint{1.053980in}{0.500000in}}%
\pgfpathlineto{\pgfqpoint{1.027518in}{0.500000in}}%
\pgfpathlineto{\pgfqpoint{1.027518in}{0.500000in}}%
\pgfpathclose%
\pgfusepath{fill}%
\end{pgfscope}%
\begin{pgfscope}%
\pgfpathrectangle{\pgfqpoint{0.750000in}{0.500000in}}{\pgfqpoint{4.650000in}{3.020000in}}%
\pgfusepath{clip}%
\pgfsetbuttcap%
\pgfsetmiterjoin%
\definecolor{currentfill}{rgb}{0.121569,0.466667,0.705882}%
\pgfsetfillcolor{currentfill}%
\pgfsetlinewidth{0.000000pt}%
\definecolor{currentstroke}{rgb}{0.000000,0.000000,0.000000}%
\pgfsetstrokecolor{currentstroke}%
\pgfsetstrokeopacity{0.000000}%
\pgfsetdash{}{0pt}%
\pgfpathmoveto{\pgfqpoint{1.060595in}{0.500000in}}%
\pgfpathlineto{\pgfqpoint{1.087057in}{0.500000in}}%
\pgfpathlineto{\pgfqpoint{1.087057in}{0.500000in}}%
\pgfpathlineto{\pgfqpoint{1.060595in}{0.500000in}}%
\pgfpathlineto{\pgfqpoint{1.060595in}{0.500000in}}%
\pgfpathclose%
\pgfusepath{fill}%
\end{pgfscope}%
\begin{pgfscope}%
\pgfpathrectangle{\pgfqpoint{0.750000in}{0.500000in}}{\pgfqpoint{4.650000in}{3.020000in}}%
\pgfusepath{clip}%
\pgfsetbuttcap%
\pgfsetmiterjoin%
\definecolor{currentfill}{rgb}{0.121569,0.466667,0.705882}%
\pgfsetfillcolor{currentfill}%
\pgfsetlinewidth{0.000000pt}%
\definecolor{currentstroke}{rgb}{0.000000,0.000000,0.000000}%
\pgfsetstrokecolor{currentstroke}%
\pgfsetstrokeopacity{0.000000}%
\pgfsetdash{}{0pt}%
\pgfpathmoveto{\pgfqpoint{1.093673in}{0.500000in}}%
\pgfpathlineto{\pgfqpoint{1.120134in}{0.500000in}}%
\pgfpathlineto{\pgfqpoint{1.120134in}{0.500000in}}%
\pgfpathlineto{\pgfqpoint{1.093673in}{0.500000in}}%
\pgfpathlineto{\pgfqpoint{1.093673in}{0.500000in}}%
\pgfpathclose%
\pgfusepath{fill}%
\end{pgfscope}%
\begin{pgfscope}%
\pgfpathrectangle{\pgfqpoint{0.750000in}{0.500000in}}{\pgfqpoint{4.650000in}{3.020000in}}%
\pgfusepath{clip}%
\pgfsetbuttcap%
\pgfsetmiterjoin%
\definecolor{currentfill}{rgb}{0.121569,0.466667,0.705882}%
\pgfsetfillcolor{currentfill}%
\pgfsetlinewidth{0.000000pt}%
\definecolor{currentstroke}{rgb}{0.000000,0.000000,0.000000}%
\pgfsetstrokecolor{currentstroke}%
\pgfsetstrokeopacity{0.000000}%
\pgfsetdash{}{0pt}%
\pgfpathmoveto{\pgfqpoint{1.126750in}{0.500000in}}%
\pgfpathlineto{\pgfqpoint{1.153212in}{0.500000in}}%
\pgfpathlineto{\pgfqpoint{1.153212in}{0.500000in}}%
\pgfpathlineto{\pgfqpoint{1.126750in}{0.500000in}}%
\pgfpathlineto{\pgfqpoint{1.126750in}{0.500000in}}%
\pgfpathclose%
\pgfusepath{fill}%
\end{pgfscope}%
\begin{pgfscope}%
\pgfpathrectangle{\pgfqpoint{0.750000in}{0.500000in}}{\pgfqpoint{4.650000in}{3.020000in}}%
\pgfusepath{clip}%
\pgfsetbuttcap%
\pgfsetmiterjoin%
\definecolor{currentfill}{rgb}{0.121569,0.466667,0.705882}%
\pgfsetfillcolor{currentfill}%
\pgfsetlinewidth{0.000000pt}%
\definecolor{currentstroke}{rgb}{0.000000,0.000000,0.000000}%
\pgfsetstrokecolor{currentstroke}%
\pgfsetstrokeopacity{0.000000}%
\pgfsetdash{}{0pt}%
\pgfpathmoveto{\pgfqpoint{1.159827in}{0.500000in}}%
\pgfpathlineto{\pgfqpoint{1.186289in}{0.500000in}}%
\pgfpathlineto{\pgfqpoint{1.186289in}{0.500000in}}%
\pgfpathlineto{\pgfqpoint{1.159827in}{0.500000in}}%
\pgfpathlineto{\pgfqpoint{1.159827in}{0.500000in}}%
\pgfpathclose%
\pgfusepath{fill}%
\end{pgfscope}%
\begin{pgfscope}%
\pgfpathrectangle{\pgfqpoint{0.750000in}{0.500000in}}{\pgfqpoint{4.650000in}{3.020000in}}%
\pgfusepath{clip}%
\pgfsetbuttcap%
\pgfsetmiterjoin%
\definecolor{currentfill}{rgb}{0.121569,0.466667,0.705882}%
\pgfsetfillcolor{currentfill}%
\pgfsetlinewidth{0.000000pt}%
\definecolor{currentstroke}{rgb}{0.000000,0.000000,0.000000}%
\pgfsetstrokecolor{currentstroke}%
\pgfsetstrokeopacity{0.000000}%
\pgfsetdash{}{0pt}%
\pgfpathmoveto{\pgfqpoint{1.192904in}{0.500000in}}%
\pgfpathlineto{\pgfqpoint{1.219366in}{0.500000in}}%
\pgfpathlineto{\pgfqpoint{1.219366in}{0.500000in}}%
\pgfpathlineto{\pgfqpoint{1.192904in}{0.500000in}}%
\pgfpathlineto{\pgfqpoint{1.192904in}{0.500000in}}%
\pgfpathclose%
\pgfusepath{fill}%
\end{pgfscope}%
\begin{pgfscope}%
\pgfpathrectangle{\pgfqpoint{0.750000in}{0.500000in}}{\pgfqpoint{4.650000in}{3.020000in}}%
\pgfusepath{clip}%
\pgfsetbuttcap%
\pgfsetmiterjoin%
\definecolor{currentfill}{rgb}{0.121569,0.466667,0.705882}%
\pgfsetfillcolor{currentfill}%
\pgfsetlinewidth{0.000000pt}%
\definecolor{currentstroke}{rgb}{0.000000,0.000000,0.000000}%
\pgfsetstrokecolor{currentstroke}%
\pgfsetstrokeopacity{0.000000}%
\pgfsetdash{}{0pt}%
\pgfpathmoveto{\pgfqpoint{1.225982in}{0.500000in}}%
\pgfpathlineto{\pgfqpoint{1.252443in}{0.500000in}}%
\pgfpathlineto{\pgfqpoint{1.252443in}{0.500000in}}%
\pgfpathlineto{\pgfqpoint{1.225982in}{0.500000in}}%
\pgfpathlineto{\pgfqpoint{1.225982in}{0.500000in}}%
\pgfpathclose%
\pgfusepath{fill}%
\end{pgfscope}%
\begin{pgfscope}%
\pgfpathrectangle{\pgfqpoint{0.750000in}{0.500000in}}{\pgfqpoint{4.650000in}{3.020000in}}%
\pgfusepath{clip}%
\pgfsetbuttcap%
\pgfsetmiterjoin%
\definecolor{currentfill}{rgb}{0.121569,0.466667,0.705882}%
\pgfsetfillcolor{currentfill}%
\pgfsetlinewidth{0.000000pt}%
\definecolor{currentstroke}{rgb}{0.000000,0.000000,0.000000}%
\pgfsetstrokecolor{currentstroke}%
\pgfsetstrokeopacity{0.000000}%
\pgfsetdash{}{0pt}%
\pgfpathmoveto{\pgfqpoint{1.259059in}{0.500000in}}%
\pgfpathlineto{\pgfqpoint{1.285521in}{0.500000in}}%
\pgfpathlineto{\pgfqpoint{1.285521in}{0.500000in}}%
\pgfpathlineto{\pgfqpoint{1.259059in}{0.500000in}}%
\pgfpathlineto{\pgfqpoint{1.259059in}{0.500000in}}%
\pgfpathclose%
\pgfusepath{fill}%
\end{pgfscope}%
\begin{pgfscope}%
\pgfpathrectangle{\pgfqpoint{0.750000in}{0.500000in}}{\pgfqpoint{4.650000in}{3.020000in}}%
\pgfusepath{clip}%
\pgfsetbuttcap%
\pgfsetmiterjoin%
\definecolor{currentfill}{rgb}{0.121569,0.466667,0.705882}%
\pgfsetfillcolor{currentfill}%
\pgfsetlinewidth{0.000000pt}%
\definecolor{currentstroke}{rgb}{0.000000,0.000000,0.000000}%
\pgfsetstrokecolor{currentstroke}%
\pgfsetstrokeopacity{0.000000}%
\pgfsetdash{}{0pt}%
\pgfpathmoveto{\pgfqpoint{1.292136in}{0.500000in}}%
\pgfpathlineto{\pgfqpoint{1.318598in}{0.500000in}}%
\pgfpathlineto{\pgfqpoint{1.318598in}{0.500000in}}%
\pgfpathlineto{\pgfqpoint{1.292136in}{0.500000in}}%
\pgfpathlineto{\pgfqpoint{1.292136in}{0.500000in}}%
\pgfpathclose%
\pgfusepath{fill}%
\end{pgfscope}%
\begin{pgfscope}%
\pgfpathrectangle{\pgfqpoint{0.750000in}{0.500000in}}{\pgfqpoint{4.650000in}{3.020000in}}%
\pgfusepath{clip}%
\pgfsetbuttcap%
\pgfsetmiterjoin%
\definecolor{currentfill}{rgb}{0.121569,0.466667,0.705882}%
\pgfsetfillcolor{currentfill}%
\pgfsetlinewidth{0.000000pt}%
\definecolor{currentstroke}{rgb}{0.000000,0.000000,0.000000}%
\pgfsetstrokecolor{currentstroke}%
\pgfsetstrokeopacity{0.000000}%
\pgfsetdash{}{0pt}%
\pgfpathmoveto{\pgfqpoint{1.325213in}{0.500000in}}%
\pgfpathlineto{\pgfqpoint{1.351675in}{0.500000in}}%
\pgfpathlineto{\pgfqpoint{1.351675in}{0.500000in}}%
\pgfpathlineto{\pgfqpoint{1.325213in}{0.500000in}}%
\pgfpathlineto{\pgfqpoint{1.325213in}{0.500000in}}%
\pgfpathclose%
\pgfusepath{fill}%
\end{pgfscope}%
\begin{pgfscope}%
\pgfpathrectangle{\pgfqpoint{0.750000in}{0.500000in}}{\pgfqpoint{4.650000in}{3.020000in}}%
\pgfusepath{clip}%
\pgfsetbuttcap%
\pgfsetmiterjoin%
\definecolor{currentfill}{rgb}{0.121569,0.466667,0.705882}%
\pgfsetfillcolor{currentfill}%
\pgfsetlinewidth{0.000000pt}%
\definecolor{currentstroke}{rgb}{0.000000,0.000000,0.000000}%
\pgfsetstrokecolor{currentstroke}%
\pgfsetstrokeopacity{0.000000}%
\pgfsetdash{}{0pt}%
\pgfpathmoveto{\pgfqpoint{1.358291in}{0.500000in}}%
\pgfpathlineto{\pgfqpoint{1.384752in}{0.500000in}}%
\pgfpathlineto{\pgfqpoint{1.384752in}{0.500000in}}%
\pgfpathlineto{\pgfqpoint{1.358291in}{0.500000in}}%
\pgfpathlineto{\pgfqpoint{1.358291in}{0.500000in}}%
\pgfpathclose%
\pgfusepath{fill}%
\end{pgfscope}%
\begin{pgfscope}%
\pgfpathrectangle{\pgfqpoint{0.750000in}{0.500000in}}{\pgfqpoint{4.650000in}{3.020000in}}%
\pgfusepath{clip}%
\pgfsetbuttcap%
\pgfsetmiterjoin%
\definecolor{currentfill}{rgb}{0.121569,0.466667,0.705882}%
\pgfsetfillcolor{currentfill}%
\pgfsetlinewidth{0.000000pt}%
\definecolor{currentstroke}{rgb}{0.000000,0.000000,0.000000}%
\pgfsetstrokecolor{currentstroke}%
\pgfsetstrokeopacity{0.000000}%
\pgfsetdash{}{0pt}%
\pgfpathmoveto{\pgfqpoint{1.391368in}{0.500000in}}%
\pgfpathlineto{\pgfqpoint{1.417830in}{0.500000in}}%
\pgfpathlineto{\pgfqpoint{1.417830in}{0.500000in}}%
\pgfpathlineto{\pgfqpoint{1.391368in}{0.500000in}}%
\pgfpathlineto{\pgfqpoint{1.391368in}{0.500000in}}%
\pgfpathclose%
\pgfusepath{fill}%
\end{pgfscope}%
\begin{pgfscope}%
\pgfpathrectangle{\pgfqpoint{0.750000in}{0.500000in}}{\pgfqpoint{4.650000in}{3.020000in}}%
\pgfusepath{clip}%
\pgfsetbuttcap%
\pgfsetmiterjoin%
\definecolor{currentfill}{rgb}{0.121569,0.466667,0.705882}%
\pgfsetfillcolor{currentfill}%
\pgfsetlinewidth{0.000000pt}%
\definecolor{currentstroke}{rgb}{0.000000,0.000000,0.000000}%
\pgfsetstrokecolor{currentstroke}%
\pgfsetstrokeopacity{0.000000}%
\pgfsetdash{}{0pt}%
\pgfpathmoveto{\pgfqpoint{1.424445in}{0.500000in}}%
\pgfpathlineto{\pgfqpoint{1.450907in}{0.500000in}}%
\pgfpathlineto{\pgfqpoint{1.450907in}{0.500000in}}%
\pgfpathlineto{\pgfqpoint{1.424445in}{0.500000in}}%
\pgfpathlineto{\pgfqpoint{1.424445in}{0.500000in}}%
\pgfpathclose%
\pgfusepath{fill}%
\end{pgfscope}%
\begin{pgfscope}%
\pgfpathrectangle{\pgfqpoint{0.750000in}{0.500000in}}{\pgfqpoint{4.650000in}{3.020000in}}%
\pgfusepath{clip}%
\pgfsetbuttcap%
\pgfsetmiterjoin%
\definecolor{currentfill}{rgb}{0.121569,0.466667,0.705882}%
\pgfsetfillcolor{currentfill}%
\pgfsetlinewidth{0.000000pt}%
\definecolor{currentstroke}{rgb}{0.000000,0.000000,0.000000}%
\pgfsetstrokecolor{currentstroke}%
\pgfsetstrokeopacity{0.000000}%
\pgfsetdash{}{0pt}%
\pgfpathmoveto{\pgfqpoint{1.457522in}{0.500000in}}%
\pgfpathlineto{\pgfqpoint{1.483984in}{0.500000in}}%
\pgfpathlineto{\pgfqpoint{1.483984in}{0.500000in}}%
\pgfpathlineto{\pgfqpoint{1.457522in}{0.500000in}}%
\pgfpathlineto{\pgfqpoint{1.457522in}{0.500000in}}%
\pgfpathclose%
\pgfusepath{fill}%
\end{pgfscope}%
\begin{pgfscope}%
\pgfpathrectangle{\pgfqpoint{0.750000in}{0.500000in}}{\pgfqpoint{4.650000in}{3.020000in}}%
\pgfusepath{clip}%
\pgfsetbuttcap%
\pgfsetmiterjoin%
\definecolor{currentfill}{rgb}{0.121569,0.466667,0.705882}%
\pgfsetfillcolor{currentfill}%
\pgfsetlinewidth{0.000000pt}%
\definecolor{currentstroke}{rgb}{0.000000,0.000000,0.000000}%
\pgfsetstrokecolor{currentstroke}%
\pgfsetstrokeopacity{0.000000}%
\pgfsetdash{}{0pt}%
\pgfpathmoveto{\pgfqpoint{1.490600in}{0.500000in}}%
\pgfpathlineto{\pgfqpoint{1.517061in}{0.500000in}}%
\pgfpathlineto{\pgfqpoint{1.517061in}{0.500000in}}%
\pgfpathlineto{\pgfqpoint{1.490600in}{0.500000in}}%
\pgfpathlineto{\pgfqpoint{1.490600in}{0.500000in}}%
\pgfpathclose%
\pgfusepath{fill}%
\end{pgfscope}%
\begin{pgfscope}%
\pgfpathrectangle{\pgfqpoint{0.750000in}{0.500000in}}{\pgfqpoint{4.650000in}{3.020000in}}%
\pgfusepath{clip}%
\pgfsetbuttcap%
\pgfsetmiterjoin%
\definecolor{currentfill}{rgb}{0.121569,0.466667,0.705882}%
\pgfsetfillcolor{currentfill}%
\pgfsetlinewidth{0.000000pt}%
\definecolor{currentstroke}{rgb}{0.000000,0.000000,0.000000}%
\pgfsetstrokecolor{currentstroke}%
\pgfsetstrokeopacity{0.000000}%
\pgfsetdash{}{0pt}%
\pgfpathmoveto{\pgfqpoint{1.523677in}{0.500000in}}%
\pgfpathlineto{\pgfqpoint{1.550139in}{0.500000in}}%
\pgfpathlineto{\pgfqpoint{1.550139in}{0.500000in}}%
\pgfpathlineto{\pgfqpoint{1.523677in}{0.500000in}}%
\pgfpathlineto{\pgfqpoint{1.523677in}{0.500000in}}%
\pgfpathclose%
\pgfusepath{fill}%
\end{pgfscope}%
\begin{pgfscope}%
\pgfpathrectangle{\pgfqpoint{0.750000in}{0.500000in}}{\pgfqpoint{4.650000in}{3.020000in}}%
\pgfusepath{clip}%
\pgfsetbuttcap%
\pgfsetmiterjoin%
\definecolor{currentfill}{rgb}{0.121569,0.466667,0.705882}%
\pgfsetfillcolor{currentfill}%
\pgfsetlinewidth{0.000000pt}%
\definecolor{currentstroke}{rgb}{0.000000,0.000000,0.000000}%
\pgfsetstrokecolor{currentstroke}%
\pgfsetstrokeopacity{0.000000}%
\pgfsetdash{}{0pt}%
\pgfpathmoveto{\pgfqpoint{1.556754in}{0.500000in}}%
\pgfpathlineto{\pgfqpoint{1.583216in}{0.500000in}}%
\pgfpathlineto{\pgfqpoint{1.583216in}{0.500000in}}%
\pgfpathlineto{\pgfqpoint{1.556754in}{0.500000in}}%
\pgfpathlineto{\pgfqpoint{1.556754in}{0.500000in}}%
\pgfpathclose%
\pgfusepath{fill}%
\end{pgfscope}%
\begin{pgfscope}%
\pgfpathrectangle{\pgfqpoint{0.750000in}{0.500000in}}{\pgfqpoint{4.650000in}{3.020000in}}%
\pgfusepath{clip}%
\pgfsetbuttcap%
\pgfsetmiterjoin%
\definecolor{currentfill}{rgb}{0.121569,0.466667,0.705882}%
\pgfsetfillcolor{currentfill}%
\pgfsetlinewidth{0.000000pt}%
\definecolor{currentstroke}{rgb}{0.000000,0.000000,0.000000}%
\pgfsetstrokecolor{currentstroke}%
\pgfsetstrokeopacity{0.000000}%
\pgfsetdash{}{0pt}%
\pgfpathmoveto{\pgfqpoint{1.589831in}{0.500000in}}%
\pgfpathlineto{\pgfqpoint{1.616293in}{0.500000in}}%
\pgfpathlineto{\pgfqpoint{1.616293in}{0.500000in}}%
\pgfpathlineto{\pgfqpoint{1.589831in}{0.500000in}}%
\pgfpathlineto{\pgfqpoint{1.589831in}{0.500000in}}%
\pgfpathclose%
\pgfusepath{fill}%
\end{pgfscope}%
\begin{pgfscope}%
\pgfpathrectangle{\pgfqpoint{0.750000in}{0.500000in}}{\pgfqpoint{4.650000in}{3.020000in}}%
\pgfusepath{clip}%
\pgfsetbuttcap%
\pgfsetmiterjoin%
\definecolor{currentfill}{rgb}{0.121569,0.466667,0.705882}%
\pgfsetfillcolor{currentfill}%
\pgfsetlinewidth{0.000000pt}%
\definecolor{currentstroke}{rgb}{0.000000,0.000000,0.000000}%
\pgfsetstrokecolor{currentstroke}%
\pgfsetstrokeopacity{0.000000}%
\pgfsetdash{}{0pt}%
\pgfpathmoveto{\pgfqpoint{1.622909in}{0.500000in}}%
\pgfpathlineto{\pgfqpoint{1.649370in}{0.500000in}}%
\pgfpathlineto{\pgfqpoint{1.649370in}{0.500000in}}%
\pgfpathlineto{\pgfqpoint{1.622909in}{0.500000in}}%
\pgfpathlineto{\pgfqpoint{1.622909in}{0.500000in}}%
\pgfpathclose%
\pgfusepath{fill}%
\end{pgfscope}%
\begin{pgfscope}%
\pgfpathrectangle{\pgfqpoint{0.750000in}{0.500000in}}{\pgfqpoint{4.650000in}{3.020000in}}%
\pgfusepath{clip}%
\pgfsetbuttcap%
\pgfsetmiterjoin%
\definecolor{currentfill}{rgb}{0.121569,0.466667,0.705882}%
\pgfsetfillcolor{currentfill}%
\pgfsetlinewidth{0.000000pt}%
\definecolor{currentstroke}{rgb}{0.000000,0.000000,0.000000}%
\pgfsetstrokecolor{currentstroke}%
\pgfsetstrokeopacity{0.000000}%
\pgfsetdash{}{0pt}%
\pgfpathmoveto{\pgfqpoint{1.655986in}{0.500000in}}%
\pgfpathlineto{\pgfqpoint{1.682448in}{0.500000in}}%
\pgfpathlineto{\pgfqpoint{1.682448in}{0.500000in}}%
\pgfpathlineto{\pgfqpoint{1.655986in}{0.500000in}}%
\pgfpathlineto{\pgfqpoint{1.655986in}{0.500000in}}%
\pgfpathclose%
\pgfusepath{fill}%
\end{pgfscope}%
\begin{pgfscope}%
\pgfpathrectangle{\pgfqpoint{0.750000in}{0.500000in}}{\pgfqpoint{4.650000in}{3.020000in}}%
\pgfusepath{clip}%
\pgfsetbuttcap%
\pgfsetmiterjoin%
\definecolor{currentfill}{rgb}{0.121569,0.466667,0.705882}%
\pgfsetfillcolor{currentfill}%
\pgfsetlinewidth{0.000000pt}%
\definecolor{currentstroke}{rgb}{0.000000,0.000000,0.000000}%
\pgfsetstrokecolor{currentstroke}%
\pgfsetstrokeopacity{0.000000}%
\pgfsetdash{}{0pt}%
\pgfpathmoveto{\pgfqpoint{1.689063in}{0.500000in}}%
\pgfpathlineto{\pgfqpoint{1.715525in}{0.500000in}}%
\pgfpathlineto{\pgfqpoint{1.715525in}{0.500000in}}%
\pgfpathlineto{\pgfqpoint{1.689063in}{0.500000in}}%
\pgfpathlineto{\pgfqpoint{1.689063in}{0.500000in}}%
\pgfpathclose%
\pgfusepath{fill}%
\end{pgfscope}%
\begin{pgfscope}%
\pgfpathrectangle{\pgfqpoint{0.750000in}{0.500000in}}{\pgfqpoint{4.650000in}{3.020000in}}%
\pgfusepath{clip}%
\pgfsetbuttcap%
\pgfsetmiterjoin%
\definecolor{currentfill}{rgb}{0.121569,0.466667,0.705882}%
\pgfsetfillcolor{currentfill}%
\pgfsetlinewidth{0.000000pt}%
\definecolor{currentstroke}{rgb}{0.000000,0.000000,0.000000}%
\pgfsetstrokecolor{currentstroke}%
\pgfsetstrokeopacity{0.000000}%
\pgfsetdash{}{0pt}%
\pgfpathmoveto{\pgfqpoint{1.722140in}{0.500000in}}%
\pgfpathlineto{\pgfqpoint{1.748602in}{0.500000in}}%
\pgfpathlineto{\pgfqpoint{1.748602in}{0.500000in}}%
\pgfpathlineto{\pgfqpoint{1.722140in}{0.500000in}}%
\pgfpathlineto{\pgfqpoint{1.722140in}{0.500000in}}%
\pgfpathclose%
\pgfusepath{fill}%
\end{pgfscope}%
\begin{pgfscope}%
\pgfpathrectangle{\pgfqpoint{0.750000in}{0.500000in}}{\pgfqpoint{4.650000in}{3.020000in}}%
\pgfusepath{clip}%
\pgfsetbuttcap%
\pgfsetmiterjoin%
\definecolor{currentfill}{rgb}{0.121569,0.466667,0.705882}%
\pgfsetfillcolor{currentfill}%
\pgfsetlinewidth{0.000000pt}%
\definecolor{currentstroke}{rgb}{0.000000,0.000000,0.000000}%
\pgfsetstrokecolor{currentstroke}%
\pgfsetstrokeopacity{0.000000}%
\pgfsetdash{}{0pt}%
\pgfpathmoveto{\pgfqpoint{1.755218in}{0.500000in}}%
\pgfpathlineto{\pgfqpoint{1.781679in}{0.500000in}}%
\pgfpathlineto{\pgfqpoint{1.781679in}{0.500000in}}%
\pgfpathlineto{\pgfqpoint{1.755218in}{0.500000in}}%
\pgfpathlineto{\pgfqpoint{1.755218in}{0.500000in}}%
\pgfpathclose%
\pgfusepath{fill}%
\end{pgfscope}%
\begin{pgfscope}%
\pgfpathrectangle{\pgfqpoint{0.750000in}{0.500000in}}{\pgfqpoint{4.650000in}{3.020000in}}%
\pgfusepath{clip}%
\pgfsetbuttcap%
\pgfsetmiterjoin%
\definecolor{currentfill}{rgb}{0.121569,0.466667,0.705882}%
\pgfsetfillcolor{currentfill}%
\pgfsetlinewidth{0.000000pt}%
\definecolor{currentstroke}{rgb}{0.000000,0.000000,0.000000}%
\pgfsetstrokecolor{currentstroke}%
\pgfsetstrokeopacity{0.000000}%
\pgfsetdash{}{0pt}%
\pgfpathmoveto{\pgfqpoint{1.788295in}{0.500000in}}%
\pgfpathlineto{\pgfqpoint{1.814757in}{0.500000in}}%
\pgfpathlineto{\pgfqpoint{1.814757in}{0.500000in}}%
\pgfpathlineto{\pgfqpoint{1.788295in}{0.500000in}}%
\pgfpathlineto{\pgfqpoint{1.788295in}{0.500000in}}%
\pgfpathclose%
\pgfusepath{fill}%
\end{pgfscope}%
\begin{pgfscope}%
\pgfpathrectangle{\pgfqpoint{0.750000in}{0.500000in}}{\pgfqpoint{4.650000in}{3.020000in}}%
\pgfusepath{clip}%
\pgfsetbuttcap%
\pgfsetmiterjoin%
\definecolor{currentfill}{rgb}{0.121569,0.466667,0.705882}%
\pgfsetfillcolor{currentfill}%
\pgfsetlinewidth{0.000000pt}%
\definecolor{currentstroke}{rgb}{0.000000,0.000000,0.000000}%
\pgfsetstrokecolor{currentstroke}%
\pgfsetstrokeopacity{0.000000}%
\pgfsetdash{}{0pt}%
\pgfpathmoveto{\pgfqpoint{1.821372in}{0.500000in}}%
\pgfpathlineto{\pgfqpoint{1.847834in}{0.500000in}}%
\pgfpathlineto{\pgfqpoint{1.847834in}{0.500000in}}%
\pgfpathlineto{\pgfqpoint{1.821372in}{0.500000in}}%
\pgfpathlineto{\pgfqpoint{1.821372in}{0.500000in}}%
\pgfpathclose%
\pgfusepath{fill}%
\end{pgfscope}%
\begin{pgfscope}%
\pgfpathrectangle{\pgfqpoint{0.750000in}{0.500000in}}{\pgfqpoint{4.650000in}{3.020000in}}%
\pgfusepath{clip}%
\pgfsetbuttcap%
\pgfsetmiterjoin%
\definecolor{currentfill}{rgb}{0.121569,0.466667,0.705882}%
\pgfsetfillcolor{currentfill}%
\pgfsetlinewidth{0.000000pt}%
\definecolor{currentstroke}{rgb}{0.000000,0.000000,0.000000}%
\pgfsetstrokecolor{currentstroke}%
\pgfsetstrokeopacity{0.000000}%
\pgfsetdash{}{0pt}%
\pgfpathmoveto{\pgfqpoint{1.854449in}{0.500000in}}%
\pgfpathlineto{\pgfqpoint{1.880911in}{0.500000in}}%
\pgfpathlineto{\pgfqpoint{1.880911in}{0.500000in}}%
\pgfpathlineto{\pgfqpoint{1.854449in}{0.500000in}}%
\pgfpathlineto{\pgfqpoint{1.854449in}{0.500000in}}%
\pgfpathclose%
\pgfusepath{fill}%
\end{pgfscope}%
\begin{pgfscope}%
\pgfpathrectangle{\pgfqpoint{0.750000in}{0.500000in}}{\pgfqpoint{4.650000in}{3.020000in}}%
\pgfusepath{clip}%
\pgfsetbuttcap%
\pgfsetmiterjoin%
\definecolor{currentfill}{rgb}{0.121569,0.466667,0.705882}%
\pgfsetfillcolor{currentfill}%
\pgfsetlinewidth{0.000000pt}%
\definecolor{currentstroke}{rgb}{0.000000,0.000000,0.000000}%
\pgfsetstrokecolor{currentstroke}%
\pgfsetstrokeopacity{0.000000}%
\pgfsetdash{}{0pt}%
\pgfpathmoveto{\pgfqpoint{1.887527in}{0.500000in}}%
\pgfpathlineto{\pgfqpoint{1.913988in}{0.500000in}}%
\pgfpathlineto{\pgfqpoint{1.913988in}{0.500000in}}%
\pgfpathlineto{\pgfqpoint{1.887527in}{0.500000in}}%
\pgfpathlineto{\pgfqpoint{1.887527in}{0.500000in}}%
\pgfpathclose%
\pgfusepath{fill}%
\end{pgfscope}%
\begin{pgfscope}%
\pgfpathrectangle{\pgfqpoint{0.750000in}{0.500000in}}{\pgfqpoint{4.650000in}{3.020000in}}%
\pgfusepath{clip}%
\pgfsetbuttcap%
\pgfsetmiterjoin%
\definecolor{currentfill}{rgb}{0.121569,0.466667,0.705882}%
\pgfsetfillcolor{currentfill}%
\pgfsetlinewidth{0.000000pt}%
\definecolor{currentstroke}{rgb}{0.000000,0.000000,0.000000}%
\pgfsetstrokecolor{currentstroke}%
\pgfsetstrokeopacity{0.000000}%
\pgfsetdash{}{0pt}%
\pgfpathmoveto{\pgfqpoint{1.920604in}{0.500000in}}%
\pgfpathlineto{\pgfqpoint{1.947066in}{0.500000in}}%
\pgfpathlineto{\pgfqpoint{1.947066in}{0.500000in}}%
\pgfpathlineto{\pgfqpoint{1.920604in}{0.500000in}}%
\pgfpathlineto{\pgfqpoint{1.920604in}{0.500000in}}%
\pgfpathclose%
\pgfusepath{fill}%
\end{pgfscope}%
\begin{pgfscope}%
\pgfpathrectangle{\pgfqpoint{0.750000in}{0.500000in}}{\pgfqpoint{4.650000in}{3.020000in}}%
\pgfusepath{clip}%
\pgfsetbuttcap%
\pgfsetmiterjoin%
\definecolor{currentfill}{rgb}{0.121569,0.466667,0.705882}%
\pgfsetfillcolor{currentfill}%
\pgfsetlinewidth{0.000000pt}%
\definecolor{currentstroke}{rgb}{0.000000,0.000000,0.000000}%
\pgfsetstrokecolor{currentstroke}%
\pgfsetstrokeopacity{0.000000}%
\pgfsetdash{}{0pt}%
\pgfpathmoveto{\pgfqpoint{1.953681in}{0.500000in}}%
\pgfpathlineto{\pgfqpoint{1.980143in}{0.500000in}}%
\pgfpathlineto{\pgfqpoint{1.980143in}{0.500000in}}%
\pgfpathlineto{\pgfqpoint{1.953681in}{0.500000in}}%
\pgfpathlineto{\pgfqpoint{1.953681in}{0.500000in}}%
\pgfpathclose%
\pgfusepath{fill}%
\end{pgfscope}%
\begin{pgfscope}%
\pgfpathrectangle{\pgfqpoint{0.750000in}{0.500000in}}{\pgfqpoint{4.650000in}{3.020000in}}%
\pgfusepath{clip}%
\pgfsetbuttcap%
\pgfsetmiterjoin%
\definecolor{currentfill}{rgb}{0.121569,0.466667,0.705882}%
\pgfsetfillcolor{currentfill}%
\pgfsetlinewidth{0.000000pt}%
\definecolor{currentstroke}{rgb}{0.000000,0.000000,0.000000}%
\pgfsetstrokecolor{currentstroke}%
\pgfsetstrokeopacity{0.000000}%
\pgfsetdash{}{0pt}%
\pgfpathmoveto{\pgfqpoint{1.986758in}{0.500000in}}%
\pgfpathlineto{\pgfqpoint{2.013220in}{0.500000in}}%
\pgfpathlineto{\pgfqpoint{2.013220in}{0.500000in}}%
\pgfpathlineto{\pgfqpoint{1.986758in}{0.500000in}}%
\pgfpathlineto{\pgfqpoint{1.986758in}{0.500000in}}%
\pgfpathclose%
\pgfusepath{fill}%
\end{pgfscope}%
\begin{pgfscope}%
\pgfpathrectangle{\pgfqpoint{0.750000in}{0.500000in}}{\pgfqpoint{4.650000in}{3.020000in}}%
\pgfusepath{clip}%
\pgfsetbuttcap%
\pgfsetmiterjoin%
\definecolor{currentfill}{rgb}{0.121569,0.466667,0.705882}%
\pgfsetfillcolor{currentfill}%
\pgfsetlinewidth{0.000000pt}%
\definecolor{currentstroke}{rgb}{0.000000,0.000000,0.000000}%
\pgfsetstrokecolor{currentstroke}%
\pgfsetstrokeopacity{0.000000}%
\pgfsetdash{}{0pt}%
\pgfpathmoveto{\pgfqpoint{2.019836in}{0.500000in}}%
\pgfpathlineto{\pgfqpoint{2.046297in}{0.500000in}}%
\pgfpathlineto{\pgfqpoint{2.046297in}{0.500000in}}%
\pgfpathlineto{\pgfqpoint{2.019836in}{0.500000in}}%
\pgfpathlineto{\pgfqpoint{2.019836in}{0.500000in}}%
\pgfpathclose%
\pgfusepath{fill}%
\end{pgfscope}%
\begin{pgfscope}%
\pgfpathrectangle{\pgfqpoint{0.750000in}{0.500000in}}{\pgfqpoint{4.650000in}{3.020000in}}%
\pgfusepath{clip}%
\pgfsetbuttcap%
\pgfsetmiterjoin%
\definecolor{currentfill}{rgb}{0.121569,0.466667,0.705882}%
\pgfsetfillcolor{currentfill}%
\pgfsetlinewidth{0.000000pt}%
\definecolor{currentstroke}{rgb}{0.000000,0.000000,0.000000}%
\pgfsetstrokecolor{currentstroke}%
\pgfsetstrokeopacity{0.000000}%
\pgfsetdash{}{0pt}%
\pgfpathmoveto{\pgfqpoint{2.052913in}{0.500000in}}%
\pgfpathlineto{\pgfqpoint{2.079375in}{0.500000in}}%
\pgfpathlineto{\pgfqpoint{2.079375in}{0.500001in}}%
\pgfpathlineto{\pgfqpoint{2.052913in}{0.500001in}}%
\pgfpathlineto{\pgfqpoint{2.052913in}{0.500000in}}%
\pgfpathclose%
\pgfusepath{fill}%
\end{pgfscope}%
\begin{pgfscope}%
\pgfpathrectangle{\pgfqpoint{0.750000in}{0.500000in}}{\pgfqpoint{4.650000in}{3.020000in}}%
\pgfusepath{clip}%
\pgfsetbuttcap%
\pgfsetmiterjoin%
\definecolor{currentfill}{rgb}{0.121569,0.466667,0.705882}%
\pgfsetfillcolor{currentfill}%
\pgfsetlinewidth{0.000000pt}%
\definecolor{currentstroke}{rgb}{0.000000,0.000000,0.000000}%
\pgfsetstrokecolor{currentstroke}%
\pgfsetstrokeopacity{0.000000}%
\pgfsetdash{}{0pt}%
\pgfpathmoveto{\pgfqpoint{2.085990in}{0.500000in}}%
\pgfpathlineto{\pgfqpoint{2.112452in}{0.500000in}}%
\pgfpathlineto{\pgfqpoint{2.112452in}{0.500001in}}%
\pgfpathlineto{\pgfqpoint{2.085990in}{0.500001in}}%
\pgfpathlineto{\pgfqpoint{2.085990in}{0.500000in}}%
\pgfpathclose%
\pgfusepath{fill}%
\end{pgfscope}%
\begin{pgfscope}%
\pgfpathrectangle{\pgfqpoint{0.750000in}{0.500000in}}{\pgfqpoint{4.650000in}{3.020000in}}%
\pgfusepath{clip}%
\pgfsetbuttcap%
\pgfsetmiterjoin%
\definecolor{currentfill}{rgb}{0.121569,0.466667,0.705882}%
\pgfsetfillcolor{currentfill}%
\pgfsetlinewidth{0.000000pt}%
\definecolor{currentstroke}{rgb}{0.000000,0.000000,0.000000}%
\pgfsetstrokecolor{currentstroke}%
\pgfsetstrokeopacity{0.000000}%
\pgfsetdash{}{0pt}%
\pgfpathmoveto{\pgfqpoint{2.119067in}{0.500000in}}%
\pgfpathlineto{\pgfqpoint{2.145529in}{0.500000in}}%
\pgfpathlineto{\pgfqpoint{2.145529in}{0.500004in}}%
\pgfpathlineto{\pgfqpoint{2.119067in}{0.500004in}}%
\pgfpathlineto{\pgfqpoint{2.119067in}{0.500000in}}%
\pgfpathclose%
\pgfusepath{fill}%
\end{pgfscope}%
\begin{pgfscope}%
\pgfpathrectangle{\pgfqpoint{0.750000in}{0.500000in}}{\pgfqpoint{4.650000in}{3.020000in}}%
\pgfusepath{clip}%
\pgfsetbuttcap%
\pgfsetmiterjoin%
\definecolor{currentfill}{rgb}{0.121569,0.466667,0.705882}%
\pgfsetfillcolor{currentfill}%
\pgfsetlinewidth{0.000000pt}%
\definecolor{currentstroke}{rgb}{0.000000,0.000000,0.000000}%
\pgfsetstrokecolor{currentstroke}%
\pgfsetstrokeopacity{0.000000}%
\pgfsetdash{}{0pt}%
\pgfpathmoveto{\pgfqpoint{2.152145in}{0.500000in}}%
\pgfpathlineto{\pgfqpoint{2.178606in}{0.500000in}}%
\pgfpathlineto{\pgfqpoint{2.178606in}{0.500010in}}%
\pgfpathlineto{\pgfqpoint{2.152145in}{0.500010in}}%
\pgfpathlineto{\pgfqpoint{2.152145in}{0.500000in}}%
\pgfpathclose%
\pgfusepath{fill}%
\end{pgfscope}%
\begin{pgfscope}%
\pgfpathrectangle{\pgfqpoint{0.750000in}{0.500000in}}{\pgfqpoint{4.650000in}{3.020000in}}%
\pgfusepath{clip}%
\pgfsetbuttcap%
\pgfsetmiterjoin%
\definecolor{currentfill}{rgb}{0.121569,0.466667,0.705882}%
\pgfsetfillcolor{currentfill}%
\pgfsetlinewidth{0.000000pt}%
\definecolor{currentstroke}{rgb}{0.000000,0.000000,0.000000}%
\pgfsetstrokecolor{currentstroke}%
\pgfsetstrokeopacity{0.000000}%
\pgfsetdash{}{0pt}%
\pgfpathmoveto{\pgfqpoint{2.185222in}{0.500000in}}%
\pgfpathlineto{\pgfqpoint{2.211684in}{0.500000in}}%
\pgfpathlineto{\pgfqpoint{2.211684in}{0.500025in}}%
\pgfpathlineto{\pgfqpoint{2.185222in}{0.500025in}}%
\pgfpathlineto{\pgfqpoint{2.185222in}{0.500000in}}%
\pgfpathclose%
\pgfusepath{fill}%
\end{pgfscope}%
\begin{pgfscope}%
\pgfpathrectangle{\pgfqpoint{0.750000in}{0.500000in}}{\pgfqpoint{4.650000in}{3.020000in}}%
\pgfusepath{clip}%
\pgfsetbuttcap%
\pgfsetmiterjoin%
\definecolor{currentfill}{rgb}{0.121569,0.466667,0.705882}%
\pgfsetfillcolor{currentfill}%
\pgfsetlinewidth{0.000000pt}%
\definecolor{currentstroke}{rgb}{0.000000,0.000000,0.000000}%
\pgfsetstrokecolor{currentstroke}%
\pgfsetstrokeopacity{0.000000}%
\pgfsetdash{}{0pt}%
\pgfpathmoveto{\pgfqpoint{2.218299in}{0.500000in}}%
\pgfpathlineto{\pgfqpoint{2.244761in}{0.500000in}}%
\pgfpathlineto{\pgfqpoint{2.244761in}{0.500060in}}%
\pgfpathlineto{\pgfqpoint{2.218299in}{0.500060in}}%
\pgfpathlineto{\pgfqpoint{2.218299in}{0.500000in}}%
\pgfpathclose%
\pgfusepath{fill}%
\end{pgfscope}%
\begin{pgfscope}%
\pgfpathrectangle{\pgfqpoint{0.750000in}{0.500000in}}{\pgfqpoint{4.650000in}{3.020000in}}%
\pgfusepath{clip}%
\pgfsetbuttcap%
\pgfsetmiterjoin%
\definecolor{currentfill}{rgb}{0.121569,0.466667,0.705882}%
\pgfsetfillcolor{currentfill}%
\pgfsetlinewidth{0.000000pt}%
\definecolor{currentstroke}{rgb}{0.000000,0.000000,0.000000}%
\pgfsetstrokecolor{currentstroke}%
\pgfsetstrokeopacity{0.000000}%
\pgfsetdash{}{0pt}%
\pgfpathmoveto{\pgfqpoint{2.251376in}{0.500000in}}%
\pgfpathlineto{\pgfqpoint{2.277838in}{0.500000in}}%
\pgfpathlineto{\pgfqpoint{2.277838in}{0.500138in}}%
\pgfpathlineto{\pgfqpoint{2.251376in}{0.500138in}}%
\pgfpathlineto{\pgfqpoint{2.251376in}{0.500000in}}%
\pgfpathclose%
\pgfusepath{fill}%
\end{pgfscope}%
\begin{pgfscope}%
\pgfpathrectangle{\pgfqpoint{0.750000in}{0.500000in}}{\pgfqpoint{4.650000in}{3.020000in}}%
\pgfusepath{clip}%
\pgfsetbuttcap%
\pgfsetmiterjoin%
\definecolor{currentfill}{rgb}{0.121569,0.466667,0.705882}%
\pgfsetfillcolor{currentfill}%
\pgfsetlinewidth{0.000000pt}%
\definecolor{currentstroke}{rgb}{0.000000,0.000000,0.000000}%
\pgfsetstrokecolor{currentstroke}%
\pgfsetstrokeopacity{0.000000}%
\pgfsetdash{}{0pt}%
\pgfpathmoveto{\pgfqpoint{2.284454in}{0.500000in}}%
\pgfpathlineto{\pgfqpoint{2.310915in}{0.500000in}}%
\pgfpathlineto{\pgfqpoint{2.310915in}{0.500306in}}%
\pgfpathlineto{\pgfqpoint{2.284454in}{0.500306in}}%
\pgfpathlineto{\pgfqpoint{2.284454in}{0.500000in}}%
\pgfpathclose%
\pgfusepath{fill}%
\end{pgfscope}%
\begin{pgfscope}%
\pgfpathrectangle{\pgfqpoint{0.750000in}{0.500000in}}{\pgfqpoint{4.650000in}{3.020000in}}%
\pgfusepath{clip}%
\pgfsetbuttcap%
\pgfsetmiterjoin%
\definecolor{currentfill}{rgb}{0.121569,0.466667,0.705882}%
\pgfsetfillcolor{currentfill}%
\pgfsetlinewidth{0.000000pt}%
\definecolor{currentstroke}{rgb}{0.000000,0.000000,0.000000}%
\pgfsetstrokecolor{currentstroke}%
\pgfsetstrokeopacity{0.000000}%
\pgfsetdash{}{0pt}%
\pgfpathmoveto{\pgfqpoint{2.317531in}{0.500000in}}%
\pgfpathlineto{\pgfqpoint{2.343993in}{0.500000in}}%
\pgfpathlineto{\pgfqpoint{2.343993in}{0.500657in}}%
\pgfpathlineto{\pgfqpoint{2.317531in}{0.500657in}}%
\pgfpathlineto{\pgfqpoint{2.317531in}{0.500000in}}%
\pgfpathclose%
\pgfusepath{fill}%
\end{pgfscope}%
\begin{pgfscope}%
\pgfpathrectangle{\pgfqpoint{0.750000in}{0.500000in}}{\pgfqpoint{4.650000in}{3.020000in}}%
\pgfusepath{clip}%
\pgfsetbuttcap%
\pgfsetmiterjoin%
\definecolor{currentfill}{rgb}{0.121569,0.466667,0.705882}%
\pgfsetfillcolor{currentfill}%
\pgfsetlinewidth{0.000000pt}%
\definecolor{currentstroke}{rgb}{0.000000,0.000000,0.000000}%
\pgfsetstrokecolor{currentstroke}%
\pgfsetstrokeopacity{0.000000}%
\pgfsetdash{}{0pt}%
\pgfpathmoveto{\pgfqpoint{2.350608in}{0.500000in}}%
\pgfpathlineto{\pgfqpoint{2.377070in}{0.500000in}}%
\pgfpathlineto{\pgfqpoint{2.377070in}{0.501360in}}%
\pgfpathlineto{\pgfqpoint{2.350608in}{0.501360in}}%
\pgfpathlineto{\pgfqpoint{2.350608in}{0.500000in}}%
\pgfpathclose%
\pgfusepath{fill}%
\end{pgfscope}%
\begin{pgfscope}%
\pgfpathrectangle{\pgfqpoint{0.750000in}{0.500000in}}{\pgfqpoint{4.650000in}{3.020000in}}%
\pgfusepath{clip}%
\pgfsetbuttcap%
\pgfsetmiterjoin%
\definecolor{currentfill}{rgb}{0.121569,0.466667,0.705882}%
\pgfsetfillcolor{currentfill}%
\pgfsetlinewidth{0.000000pt}%
\definecolor{currentstroke}{rgb}{0.000000,0.000000,0.000000}%
\pgfsetstrokecolor{currentstroke}%
\pgfsetstrokeopacity{0.000000}%
\pgfsetdash{}{0pt}%
\pgfpathmoveto{\pgfqpoint{2.383685in}{0.500000in}}%
\pgfpathlineto{\pgfqpoint{2.410147in}{0.500000in}}%
\pgfpathlineto{\pgfqpoint{2.410147in}{0.502721in}}%
\pgfpathlineto{\pgfqpoint{2.383685in}{0.502721in}}%
\pgfpathlineto{\pgfqpoint{2.383685in}{0.500000in}}%
\pgfpathclose%
\pgfusepath{fill}%
\end{pgfscope}%
\begin{pgfscope}%
\pgfpathrectangle{\pgfqpoint{0.750000in}{0.500000in}}{\pgfqpoint{4.650000in}{3.020000in}}%
\pgfusepath{clip}%
\pgfsetbuttcap%
\pgfsetmiterjoin%
\definecolor{currentfill}{rgb}{0.121569,0.466667,0.705882}%
\pgfsetfillcolor{currentfill}%
\pgfsetlinewidth{0.000000pt}%
\definecolor{currentstroke}{rgb}{0.000000,0.000000,0.000000}%
\pgfsetstrokecolor{currentstroke}%
\pgfsetstrokeopacity{0.000000}%
\pgfsetdash{}{0pt}%
\pgfpathmoveto{\pgfqpoint{2.416763in}{0.500000in}}%
\pgfpathlineto{\pgfqpoint{2.443224in}{0.500000in}}%
\pgfpathlineto{\pgfqpoint{2.443224in}{0.505256in}}%
\pgfpathlineto{\pgfqpoint{2.416763in}{0.505256in}}%
\pgfpathlineto{\pgfqpoint{2.416763in}{0.500000in}}%
\pgfpathclose%
\pgfusepath{fill}%
\end{pgfscope}%
\begin{pgfscope}%
\pgfpathrectangle{\pgfqpoint{0.750000in}{0.500000in}}{\pgfqpoint{4.650000in}{3.020000in}}%
\pgfusepath{clip}%
\pgfsetbuttcap%
\pgfsetmiterjoin%
\definecolor{currentfill}{rgb}{0.121569,0.466667,0.705882}%
\pgfsetfillcolor{currentfill}%
\pgfsetlinewidth{0.000000pt}%
\definecolor{currentstroke}{rgb}{0.000000,0.000000,0.000000}%
\pgfsetstrokecolor{currentstroke}%
\pgfsetstrokeopacity{0.000000}%
\pgfsetdash{}{0pt}%
\pgfpathmoveto{\pgfqpoint{2.449840in}{0.500000in}}%
\pgfpathlineto{\pgfqpoint{2.476302in}{0.500000in}}%
\pgfpathlineto{\pgfqpoint{2.476302in}{0.509811in}}%
\pgfpathlineto{\pgfqpoint{2.449840in}{0.509811in}}%
\pgfpathlineto{\pgfqpoint{2.449840in}{0.500000in}}%
\pgfpathclose%
\pgfusepath{fill}%
\end{pgfscope}%
\begin{pgfscope}%
\pgfpathrectangle{\pgfqpoint{0.750000in}{0.500000in}}{\pgfqpoint{4.650000in}{3.020000in}}%
\pgfusepath{clip}%
\pgfsetbuttcap%
\pgfsetmiterjoin%
\definecolor{currentfill}{rgb}{0.121569,0.466667,0.705882}%
\pgfsetfillcolor{currentfill}%
\pgfsetlinewidth{0.000000pt}%
\definecolor{currentstroke}{rgb}{0.000000,0.000000,0.000000}%
\pgfsetstrokecolor{currentstroke}%
\pgfsetstrokeopacity{0.000000}%
\pgfsetdash{}{0pt}%
\pgfpathmoveto{\pgfqpoint{2.482917in}{0.500000in}}%
\pgfpathlineto{\pgfqpoint{2.509379in}{0.500000in}}%
\pgfpathlineto{\pgfqpoint{2.509379in}{0.517703in}}%
\pgfpathlineto{\pgfqpoint{2.482917in}{0.517703in}}%
\pgfpathlineto{\pgfqpoint{2.482917in}{0.500000in}}%
\pgfpathclose%
\pgfusepath{fill}%
\end{pgfscope}%
\begin{pgfscope}%
\pgfpathrectangle{\pgfqpoint{0.750000in}{0.500000in}}{\pgfqpoint{4.650000in}{3.020000in}}%
\pgfusepath{clip}%
\pgfsetbuttcap%
\pgfsetmiterjoin%
\definecolor{currentfill}{rgb}{0.121569,0.466667,0.705882}%
\pgfsetfillcolor{currentfill}%
\pgfsetlinewidth{0.000000pt}%
\definecolor{currentstroke}{rgb}{0.000000,0.000000,0.000000}%
\pgfsetstrokecolor{currentstroke}%
\pgfsetstrokeopacity{0.000000}%
\pgfsetdash{}{0pt}%
\pgfpathmoveto{\pgfqpoint{2.515994in}{0.500000in}}%
\pgfpathlineto{\pgfqpoint{2.542456in}{0.500000in}}%
\pgfpathlineto{\pgfqpoint{2.542456in}{0.530887in}}%
\pgfpathlineto{\pgfqpoint{2.515994in}{0.530887in}}%
\pgfpathlineto{\pgfqpoint{2.515994in}{0.500000in}}%
\pgfpathclose%
\pgfusepath{fill}%
\end{pgfscope}%
\begin{pgfscope}%
\pgfpathrectangle{\pgfqpoint{0.750000in}{0.500000in}}{\pgfqpoint{4.650000in}{3.020000in}}%
\pgfusepath{clip}%
\pgfsetbuttcap%
\pgfsetmiterjoin%
\definecolor{currentfill}{rgb}{0.121569,0.466667,0.705882}%
\pgfsetfillcolor{currentfill}%
\pgfsetlinewidth{0.000000pt}%
\definecolor{currentstroke}{rgb}{0.000000,0.000000,0.000000}%
\pgfsetstrokecolor{currentstroke}%
\pgfsetstrokeopacity{0.000000}%
\pgfsetdash{}{0pt}%
\pgfpathmoveto{\pgfqpoint{2.549072in}{0.500000in}}%
\pgfpathlineto{\pgfqpoint{2.575534in}{0.500000in}}%
\pgfpathlineto{\pgfqpoint{2.575534in}{0.552121in}}%
\pgfpathlineto{\pgfqpoint{2.549072in}{0.552121in}}%
\pgfpathlineto{\pgfqpoint{2.549072in}{0.500000in}}%
\pgfpathclose%
\pgfusepath{fill}%
\end{pgfscope}%
\begin{pgfscope}%
\pgfpathrectangle{\pgfqpoint{0.750000in}{0.500000in}}{\pgfqpoint{4.650000in}{3.020000in}}%
\pgfusepath{clip}%
\pgfsetbuttcap%
\pgfsetmiterjoin%
\definecolor{currentfill}{rgb}{0.121569,0.466667,0.705882}%
\pgfsetfillcolor{currentfill}%
\pgfsetlinewidth{0.000000pt}%
\definecolor{currentstroke}{rgb}{0.000000,0.000000,0.000000}%
\pgfsetstrokecolor{currentstroke}%
\pgfsetstrokeopacity{0.000000}%
\pgfsetdash{}{0pt}%
\pgfpathmoveto{\pgfqpoint{2.582149in}{0.500000in}}%
\pgfpathlineto{\pgfqpoint{2.608611in}{0.500000in}}%
\pgfpathlineto{\pgfqpoint{2.608611in}{0.585096in}}%
\pgfpathlineto{\pgfqpoint{2.582149in}{0.585096in}}%
\pgfpathlineto{\pgfqpoint{2.582149in}{0.500000in}}%
\pgfpathclose%
\pgfusepath{fill}%
\end{pgfscope}%
\begin{pgfscope}%
\pgfpathrectangle{\pgfqpoint{0.750000in}{0.500000in}}{\pgfqpoint{4.650000in}{3.020000in}}%
\pgfusepath{clip}%
\pgfsetbuttcap%
\pgfsetmiterjoin%
\definecolor{currentfill}{rgb}{0.121569,0.466667,0.705882}%
\pgfsetfillcolor{currentfill}%
\pgfsetlinewidth{0.000000pt}%
\definecolor{currentstroke}{rgb}{0.000000,0.000000,0.000000}%
\pgfsetstrokecolor{currentstroke}%
\pgfsetstrokeopacity{0.000000}%
\pgfsetdash{}{0pt}%
\pgfpathmoveto{\pgfqpoint{2.615226in}{0.500000in}}%
\pgfpathlineto{\pgfqpoint{2.641688in}{0.500000in}}%
\pgfpathlineto{\pgfqpoint{2.641688in}{0.634452in}}%
\pgfpathlineto{\pgfqpoint{2.615226in}{0.634452in}}%
\pgfpathlineto{\pgfqpoint{2.615226in}{0.500000in}}%
\pgfpathclose%
\pgfusepath{fill}%
\end{pgfscope}%
\begin{pgfscope}%
\pgfpathrectangle{\pgfqpoint{0.750000in}{0.500000in}}{\pgfqpoint{4.650000in}{3.020000in}}%
\pgfusepath{clip}%
\pgfsetbuttcap%
\pgfsetmiterjoin%
\definecolor{currentfill}{rgb}{0.121569,0.466667,0.705882}%
\pgfsetfillcolor{currentfill}%
\pgfsetlinewidth{0.000000pt}%
\definecolor{currentstroke}{rgb}{0.000000,0.000000,0.000000}%
\pgfsetstrokecolor{currentstroke}%
\pgfsetstrokeopacity{0.000000}%
\pgfsetdash{}{0pt}%
\pgfpathmoveto{\pgfqpoint{2.648303in}{0.500000in}}%
\pgfpathlineto{\pgfqpoint{2.674765in}{0.500000in}}%
\pgfpathlineto{\pgfqpoint{2.674765in}{0.705632in}}%
\pgfpathlineto{\pgfqpoint{2.648303in}{0.705632in}}%
\pgfpathlineto{\pgfqpoint{2.648303in}{0.500000in}}%
\pgfpathclose%
\pgfusepath{fill}%
\end{pgfscope}%
\begin{pgfscope}%
\pgfpathrectangle{\pgfqpoint{0.750000in}{0.500000in}}{\pgfqpoint{4.650000in}{3.020000in}}%
\pgfusepath{clip}%
\pgfsetbuttcap%
\pgfsetmiterjoin%
\definecolor{currentfill}{rgb}{0.121569,0.466667,0.705882}%
\pgfsetfillcolor{currentfill}%
\pgfsetlinewidth{0.000000pt}%
\definecolor{currentstroke}{rgb}{0.000000,0.000000,0.000000}%
\pgfsetstrokecolor{currentstroke}%
\pgfsetstrokeopacity{0.000000}%
\pgfsetdash{}{0pt}%
\pgfpathmoveto{\pgfqpoint{2.681381in}{0.500000in}}%
\pgfpathlineto{\pgfqpoint{2.707843in}{0.500000in}}%
\pgfpathlineto{\pgfqpoint{2.707843in}{0.804493in}}%
\pgfpathlineto{\pgfqpoint{2.681381in}{0.804493in}}%
\pgfpathlineto{\pgfqpoint{2.681381in}{0.500000in}}%
\pgfpathclose%
\pgfusepath{fill}%
\end{pgfscope}%
\begin{pgfscope}%
\pgfpathrectangle{\pgfqpoint{0.750000in}{0.500000in}}{\pgfqpoint{4.650000in}{3.020000in}}%
\pgfusepath{clip}%
\pgfsetbuttcap%
\pgfsetmiterjoin%
\definecolor{currentfill}{rgb}{0.121569,0.466667,0.705882}%
\pgfsetfillcolor{currentfill}%
\pgfsetlinewidth{0.000000pt}%
\definecolor{currentstroke}{rgb}{0.000000,0.000000,0.000000}%
\pgfsetstrokecolor{currentstroke}%
\pgfsetstrokeopacity{0.000000}%
\pgfsetdash{}{0pt}%
\pgfpathmoveto{\pgfqpoint{2.714458in}{0.500000in}}%
\pgfpathlineto{\pgfqpoint{2.740920in}{0.500000in}}%
\pgfpathlineto{\pgfqpoint{2.740920in}{0.936632in}}%
\pgfpathlineto{\pgfqpoint{2.714458in}{0.936632in}}%
\pgfpathlineto{\pgfqpoint{2.714458in}{0.500000in}}%
\pgfpathclose%
\pgfusepath{fill}%
\end{pgfscope}%
\begin{pgfscope}%
\pgfpathrectangle{\pgfqpoint{0.750000in}{0.500000in}}{\pgfqpoint{4.650000in}{3.020000in}}%
\pgfusepath{clip}%
\pgfsetbuttcap%
\pgfsetmiterjoin%
\definecolor{currentfill}{rgb}{0.121569,0.466667,0.705882}%
\pgfsetfillcolor{currentfill}%
\pgfsetlinewidth{0.000000pt}%
\definecolor{currentstroke}{rgb}{0.000000,0.000000,0.000000}%
\pgfsetstrokecolor{currentstroke}%
\pgfsetstrokeopacity{0.000000}%
\pgfsetdash{}{0pt}%
\pgfpathmoveto{\pgfqpoint{2.747535in}{0.500000in}}%
\pgfpathlineto{\pgfqpoint{2.773997in}{0.500000in}}%
\pgfpathlineto{\pgfqpoint{2.773997in}{1.106433in}}%
\pgfpathlineto{\pgfqpoint{2.747535in}{1.106433in}}%
\pgfpathlineto{\pgfqpoint{2.747535in}{0.500000in}}%
\pgfpathclose%
\pgfusepath{fill}%
\end{pgfscope}%
\begin{pgfscope}%
\pgfpathrectangle{\pgfqpoint{0.750000in}{0.500000in}}{\pgfqpoint{4.650000in}{3.020000in}}%
\pgfusepath{clip}%
\pgfsetbuttcap%
\pgfsetmiterjoin%
\definecolor{currentfill}{rgb}{0.121569,0.466667,0.705882}%
\pgfsetfillcolor{currentfill}%
\pgfsetlinewidth{0.000000pt}%
\definecolor{currentstroke}{rgb}{0.000000,0.000000,0.000000}%
\pgfsetstrokecolor{currentstroke}%
\pgfsetstrokeopacity{0.000000}%
\pgfsetdash{}{0pt}%
\pgfpathmoveto{\pgfqpoint{2.780612in}{0.500000in}}%
\pgfpathlineto{\pgfqpoint{2.807074in}{0.500000in}}%
\pgfpathlineto{\pgfqpoint{2.807074in}{1.315928in}}%
\pgfpathlineto{\pgfqpoint{2.780612in}{1.315928in}}%
\pgfpathlineto{\pgfqpoint{2.780612in}{0.500000in}}%
\pgfpathclose%
\pgfusepath{fill}%
\end{pgfscope}%
\begin{pgfscope}%
\pgfpathrectangle{\pgfqpoint{0.750000in}{0.500000in}}{\pgfqpoint{4.650000in}{3.020000in}}%
\pgfusepath{clip}%
\pgfsetbuttcap%
\pgfsetmiterjoin%
\definecolor{currentfill}{rgb}{0.121569,0.466667,0.705882}%
\pgfsetfillcolor{currentfill}%
\pgfsetlinewidth{0.000000pt}%
\definecolor{currentstroke}{rgb}{0.000000,0.000000,0.000000}%
\pgfsetstrokecolor{currentstroke}%
\pgfsetstrokeopacity{0.000000}%
\pgfsetdash{}{0pt}%
\pgfpathmoveto{\pgfqpoint{2.813690in}{0.500000in}}%
\pgfpathlineto{\pgfqpoint{2.840152in}{0.500000in}}%
\pgfpathlineto{\pgfqpoint{2.840152in}{1.563621in}}%
\pgfpathlineto{\pgfqpoint{2.813690in}{1.563621in}}%
\pgfpathlineto{\pgfqpoint{2.813690in}{0.500000in}}%
\pgfpathclose%
\pgfusepath{fill}%
\end{pgfscope}%
\begin{pgfscope}%
\pgfpathrectangle{\pgfqpoint{0.750000in}{0.500000in}}{\pgfqpoint{4.650000in}{3.020000in}}%
\pgfusepath{clip}%
\pgfsetbuttcap%
\pgfsetmiterjoin%
\definecolor{currentfill}{rgb}{0.121569,0.466667,0.705882}%
\pgfsetfillcolor{currentfill}%
\pgfsetlinewidth{0.000000pt}%
\definecolor{currentstroke}{rgb}{0.000000,0.000000,0.000000}%
\pgfsetstrokecolor{currentstroke}%
\pgfsetstrokeopacity{0.000000}%
\pgfsetdash{}{0pt}%
\pgfpathmoveto{\pgfqpoint{2.846767in}{0.500000in}}%
\pgfpathlineto{\pgfqpoint{2.873229in}{0.500000in}}%
\pgfpathlineto{\pgfqpoint{2.873229in}{1.843521in}}%
\pgfpathlineto{\pgfqpoint{2.846767in}{1.843521in}}%
\pgfpathlineto{\pgfqpoint{2.846767in}{0.500000in}}%
\pgfpathclose%
\pgfusepath{fill}%
\end{pgfscope}%
\begin{pgfscope}%
\pgfpathrectangle{\pgfqpoint{0.750000in}{0.500000in}}{\pgfqpoint{4.650000in}{3.020000in}}%
\pgfusepath{clip}%
\pgfsetbuttcap%
\pgfsetmiterjoin%
\definecolor{currentfill}{rgb}{0.121569,0.466667,0.705882}%
\pgfsetfillcolor{currentfill}%
\pgfsetlinewidth{0.000000pt}%
\definecolor{currentstroke}{rgb}{0.000000,0.000000,0.000000}%
\pgfsetstrokecolor{currentstroke}%
\pgfsetstrokeopacity{0.000000}%
\pgfsetdash{}{0pt}%
\pgfpathmoveto{\pgfqpoint{2.879844in}{0.500000in}}%
\pgfpathlineto{\pgfqpoint{2.906306in}{0.500000in}}%
\pgfpathlineto{\pgfqpoint{2.906306in}{2.144655in}}%
\pgfpathlineto{\pgfqpoint{2.879844in}{2.144655in}}%
\pgfpathlineto{\pgfqpoint{2.879844in}{0.500000in}}%
\pgfpathclose%
\pgfusepath{fill}%
\end{pgfscope}%
\begin{pgfscope}%
\pgfpathrectangle{\pgfqpoint{0.750000in}{0.500000in}}{\pgfqpoint{4.650000in}{3.020000in}}%
\pgfusepath{clip}%
\pgfsetbuttcap%
\pgfsetmiterjoin%
\definecolor{currentfill}{rgb}{0.121569,0.466667,0.705882}%
\pgfsetfillcolor{currentfill}%
\pgfsetlinewidth{0.000000pt}%
\definecolor{currentstroke}{rgb}{0.000000,0.000000,0.000000}%
\pgfsetstrokecolor{currentstroke}%
\pgfsetstrokeopacity{0.000000}%
\pgfsetdash{}{0pt}%
\pgfpathmoveto{\pgfqpoint{2.912921in}{0.500000in}}%
\pgfpathlineto{\pgfqpoint{2.939383in}{0.500000in}}%
\pgfpathlineto{\pgfqpoint{2.939383in}{2.451286in}}%
\pgfpathlineto{\pgfqpoint{2.912921in}{2.451286in}}%
\pgfpathlineto{\pgfqpoint{2.912921in}{0.500000in}}%
\pgfpathclose%
\pgfusepath{fill}%
\end{pgfscope}%
\begin{pgfscope}%
\pgfpathrectangle{\pgfqpoint{0.750000in}{0.500000in}}{\pgfqpoint{4.650000in}{3.020000in}}%
\pgfusepath{clip}%
\pgfsetbuttcap%
\pgfsetmiterjoin%
\definecolor{currentfill}{rgb}{0.121569,0.466667,0.705882}%
\pgfsetfillcolor{currentfill}%
\pgfsetlinewidth{0.000000pt}%
\definecolor{currentstroke}{rgb}{0.000000,0.000000,0.000000}%
\pgfsetstrokecolor{currentstroke}%
\pgfsetstrokeopacity{0.000000}%
\pgfsetdash{}{0pt}%
\pgfpathmoveto{\pgfqpoint{2.945999in}{0.500000in}}%
\pgfpathlineto{\pgfqpoint{2.972461in}{0.500000in}}%
\pgfpathlineto{\pgfqpoint{2.972461in}{2.743978in}}%
\pgfpathlineto{\pgfqpoint{2.945999in}{2.743978in}}%
\pgfpathlineto{\pgfqpoint{2.945999in}{0.500000in}}%
\pgfpathclose%
\pgfusepath{fill}%
\end{pgfscope}%
\begin{pgfscope}%
\pgfpathrectangle{\pgfqpoint{0.750000in}{0.500000in}}{\pgfqpoint{4.650000in}{3.020000in}}%
\pgfusepath{clip}%
\pgfsetbuttcap%
\pgfsetmiterjoin%
\definecolor{currentfill}{rgb}{0.121569,0.466667,0.705882}%
\pgfsetfillcolor{currentfill}%
\pgfsetlinewidth{0.000000pt}%
\definecolor{currentstroke}{rgb}{0.000000,0.000000,0.000000}%
\pgfsetstrokecolor{currentstroke}%
\pgfsetstrokeopacity{0.000000}%
\pgfsetdash{}{0pt}%
\pgfpathmoveto{\pgfqpoint{2.979076in}{0.500000in}}%
\pgfpathlineto{\pgfqpoint{3.005538in}{0.500000in}}%
\pgfpathlineto{\pgfqpoint{3.005538in}{3.001484in}}%
\pgfpathlineto{\pgfqpoint{2.979076in}{3.001484in}}%
\pgfpathlineto{\pgfqpoint{2.979076in}{0.500000in}}%
\pgfpathclose%
\pgfusepath{fill}%
\end{pgfscope}%
\begin{pgfscope}%
\pgfpathrectangle{\pgfqpoint{0.750000in}{0.500000in}}{\pgfqpoint{4.650000in}{3.020000in}}%
\pgfusepath{clip}%
\pgfsetbuttcap%
\pgfsetmiterjoin%
\definecolor{currentfill}{rgb}{0.121569,0.466667,0.705882}%
\pgfsetfillcolor{currentfill}%
\pgfsetlinewidth{0.000000pt}%
\definecolor{currentstroke}{rgb}{0.000000,0.000000,0.000000}%
\pgfsetstrokecolor{currentstroke}%
\pgfsetstrokeopacity{0.000000}%
\pgfsetdash{}{0pt}%
\pgfpathmoveto{\pgfqpoint{3.012153in}{0.500000in}}%
\pgfpathlineto{\pgfqpoint{3.038615in}{0.500000in}}%
\pgfpathlineto{\pgfqpoint{3.038615in}{3.203217in}}%
\pgfpathlineto{\pgfqpoint{3.012153in}{3.203217in}}%
\pgfpathlineto{\pgfqpoint{3.012153in}{0.500000in}}%
\pgfpathclose%
\pgfusepath{fill}%
\end{pgfscope}%
\begin{pgfscope}%
\pgfpathrectangle{\pgfqpoint{0.750000in}{0.500000in}}{\pgfqpoint{4.650000in}{3.020000in}}%
\pgfusepath{clip}%
\pgfsetbuttcap%
\pgfsetmiterjoin%
\definecolor{currentfill}{rgb}{0.121569,0.466667,0.705882}%
\pgfsetfillcolor{currentfill}%
\pgfsetlinewidth{0.000000pt}%
\definecolor{currentstroke}{rgb}{0.000000,0.000000,0.000000}%
\pgfsetstrokecolor{currentstroke}%
\pgfsetstrokeopacity{0.000000}%
\pgfsetdash{}{0pt}%
\pgfpathmoveto{\pgfqpoint{3.045230in}{0.500000in}}%
\pgfpathlineto{\pgfqpoint{3.071692in}{0.500000in}}%
\pgfpathlineto{\pgfqpoint{3.071692in}{3.331941in}}%
\pgfpathlineto{\pgfqpoint{3.045230in}{3.331941in}}%
\pgfpathlineto{\pgfqpoint{3.045230in}{0.500000in}}%
\pgfpathclose%
\pgfusepath{fill}%
\end{pgfscope}%
\begin{pgfscope}%
\pgfpathrectangle{\pgfqpoint{0.750000in}{0.500000in}}{\pgfqpoint{4.650000in}{3.020000in}}%
\pgfusepath{clip}%
\pgfsetbuttcap%
\pgfsetmiterjoin%
\definecolor{currentfill}{rgb}{0.121569,0.466667,0.705882}%
\pgfsetfillcolor{currentfill}%
\pgfsetlinewidth{0.000000pt}%
\definecolor{currentstroke}{rgb}{0.000000,0.000000,0.000000}%
\pgfsetstrokecolor{currentstroke}%
\pgfsetstrokeopacity{0.000000}%
\pgfsetdash{}{0pt}%
\pgfpathmoveto{\pgfqpoint{3.078308in}{0.500000in}}%
\pgfpathlineto{\pgfqpoint{3.104770in}{0.500000in}}%
\pgfpathlineto{\pgfqpoint{3.104770in}{3.376190in}}%
\pgfpathlineto{\pgfqpoint{3.078308in}{3.376190in}}%
\pgfpathlineto{\pgfqpoint{3.078308in}{0.500000in}}%
\pgfpathclose%
\pgfusepath{fill}%
\end{pgfscope}%
\begin{pgfscope}%
\pgfpathrectangle{\pgfqpoint{0.750000in}{0.500000in}}{\pgfqpoint{4.650000in}{3.020000in}}%
\pgfusepath{clip}%
\pgfsetbuttcap%
\pgfsetmiterjoin%
\definecolor{currentfill}{rgb}{0.121569,0.466667,0.705882}%
\pgfsetfillcolor{currentfill}%
\pgfsetlinewidth{0.000000pt}%
\definecolor{currentstroke}{rgb}{0.000000,0.000000,0.000000}%
\pgfsetstrokecolor{currentstroke}%
\pgfsetstrokeopacity{0.000000}%
\pgfsetdash{}{0pt}%
\pgfpathmoveto{\pgfqpoint{3.111385in}{0.500000in}}%
\pgfpathlineto{\pgfqpoint{3.137847in}{0.500000in}}%
\pgfpathlineto{\pgfqpoint{3.137847in}{3.331941in}}%
\pgfpathlineto{\pgfqpoint{3.111385in}{3.331941in}}%
\pgfpathlineto{\pgfqpoint{3.111385in}{0.500000in}}%
\pgfpathclose%
\pgfusepath{fill}%
\end{pgfscope}%
\begin{pgfscope}%
\pgfpathrectangle{\pgfqpoint{0.750000in}{0.500000in}}{\pgfqpoint{4.650000in}{3.020000in}}%
\pgfusepath{clip}%
\pgfsetbuttcap%
\pgfsetmiterjoin%
\definecolor{currentfill}{rgb}{0.121569,0.466667,0.705882}%
\pgfsetfillcolor{currentfill}%
\pgfsetlinewidth{0.000000pt}%
\definecolor{currentstroke}{rgb}{0.000000,0.000000,0.000000}%
\pgfsetstrokecolor{currentstroke}%
\pgfsetstrokeopacity{0.000000}%
\pgfsetdash{}{0pt}%
\pgfpathmoveto{\pgfqpoint{3.144462in}{0.500000in}}%
\pgfpathlineto{\pgfqpoint{3.170924in}{0.500000in}}%
\pgfpathlineto{\pgfqpoint{3.170924in}{3.203217in}}%
\pgfpathlineto{\pgfqpoint{3.144462in}{3.203217in}}%
\pgfpathlineto{\pgfqpoint{3.144462in}{0.500000in}}%
\pgfpathclose%
\pgfusepath{fill}%
\end{pgfscope}%
\begin{pgfscope}%
\pgfpathrectangle{\pgfqpoint{0.750000in}{0.500000in}}{\pgfqpoint{4.650000in}{3.020000in}}%
\pgfusepath{clip}%
\pgfsetbuttcap%
\pgfsetmiterjoin%
\definecolor{currentfill}{rgb}{0.121569,0.466667,0.705882}%
\pgfsetfillcolor{currentfill}%
\pgfsetlinewidth{0.000000pt}%
\definecolor{currentstroke}{rgb}{0.000000,0.000000,0.000000}%
\pgfsetstrokecolor{currentstroke}%
\pgfsetstrokeopacity{0.000000}%
\pgfsetdash{}{0pt}%
\pgfpathmoveto{\pgfqpoint{3.177539in}{0.500000in}}%
\pgfpathlineto{\pgfqpoint{3.204001in}{0.500000in}}%
\pgfpathlineto{\pgfqpoint{3.204001in}{3.001484in}}%
\pgfpathlineto{\pgfqpoint{3.177539in}{3.001484in}}%
\pgfpathlineto{\pgfqpoint{3.177539in}{0.500000in}}%
\pgfpathclose%
\pgfusepath{fill}%
\end{pgfscope}%
\begin{pgfscope}%
\pgfpathrectangle{\pgfqpoint{0.750000in}{0.500000in}}{\pgfqpoint{4.650000in}{3.020000in}}%
\pgfusepath{clip}%
\pgfsetbuttcap%
\pgfsetmiterjoin%
\definecolor{currentfill}{rgb}{0.121569,0.466667,0.705882}%
\pgfsetfillcolor{currentfill}%
\pgfsetlinewidth{0.000000pt}%
\definecolor{currentstroke}{rgb}{0.000000,0.000000,0.000000}%
\pgfsetstrokecolor{currentstroke}%
\pgfsetstrokeopacity{0.000000}%
\pgfsetdash{}{0pt}%
\pgfpathmoveto{\pgfqpoint{3.210617in}{0.500000in}}%
\pgfpathlineto{\pgfqpoint{3.237079in}{0.500000in}}%
\pgfpathlineto{\pgfqpoint{3.237079in}{2.743978in}}%
\pgfpathlineto{\pgfqpoint{3.210617in}{2.743978in}}%
\pgfpathlineto{\pgfqpoint{3.210617in}{0.500000in}}%
\pgfpathclose%
\pgfusepath{fill}%
\end{pgfscope}%
\begin{pgfscope}%
\pgfpathrectangle{\pgfqpoint{0.750000in}{0.500000in}}{\pgfqpoint{4.650000in}{3.020000in}}%
\pgfusepath{clip}%
\pgfsetbuttcap%
\pgfsetmiterjoin%
\definecolor{currentfill}{rgb}{0.121569,0.466667,0.705882}%
\pgfsetfillcolor{currentfill}%
\pgfsetlinewidth{0.000000pt}%
\definecolor{currentstroke}{rgb}{0.000000,0.000000,0.000000}%
\pgfsetstrokecolor{currentstroke}%
\pgfsetstrokeopacity{0.000000}%
\pgfsetdash{}{0pt}%
\pgfpathmoveto{\pgfqpoint{3.243694in}{0.500000in}}%
\pgfpathlineto{\pgfqpoint{3.270156in}{0.500000in}}%
\pgfpathlineto{\pgfqpoint{3.270156in}{2.451286in}}%
\pgfpathlineto{\pgfqpoint{3.243694in}{2.451286in}}%
\pgfpathlineto{\pgfqpoint{3.243694in}{0.500000in}}%
\pgfpathclose%
\pgfusepath{fill}%
\end{pgfscope}%
\begin{pgfscope}%
\pgfpathrectangle{\pgfqpoint{0.750000in}{0.500000in}}{\pgfqpoint{4.650000in}{3.020000in}}%
\pgfusepath{clip}%
\pgfsetbuttcap%
\pgfsetmiterjoin%
\definecolor{currentfill}{rgb}{0.121569,0.466667,0.705882}%
\pgfsetfillcolor{currentfill}%
\pgfsetlinewidth{0.000000pt}%
\definecolor{currentstroke}{rgb}{0.000000,0.000000,0.000000}%
\pgfsetstrokecolor{currentstroke}%
\pgfsetstrokeopacity{0.000000}%
\pgfsetdash{}{0pt}%
\pgfpathmoveto{\pgfqpoint{3.276771in}{0.500000in}}%
\pgfpathlineto{\pgfqpoint{3.303233in}{0.500000in}}%
\pgfpathlineto{\pgfqpoint{3.303233in}{2.144655in}}%
\pgfpathlineto{\pgfqpoint{3.276771in}{2.144655in}}%
\pgfpathlineto{\pgfqpoint{3.276771in}{0.500000in}}%
\pgfpathclose%
\pgfusepath{fill}%
\end{pgfscope}%
\begin{pgfscope}%
\pgfpathrectangle{\pgfqpoint{0.750000in}{0.500000in}}{\pgfqpoint{4.650000in}{3.020000in}}%
\pgfusepath{clip}%
\pgfsetbuttcap%
\pgfsetmiterjoin%
\definecolor{currentfill}{rgb}{0.121569,0.466667,0.705882}%
\pgfsetfillcolor{currentfill}%
\pgfsetlinewidth{0.000000pt}%
\definecolor{currentstroke}{rgb}{0.000000,0.000000,0.000000}%
\pgfsetstrokecolor{currentstroke}%
\pgfsetstrokeopacity{0.000000}%
\pgfsetdash{}{0pt}%
\pgfpathmoveto{\pgfqpoint{3.309848in}{0.500000in}}%
\pgfpathlineto{\pgfqpoint{3.336310in}{0.500000in}}%
\pgfpathlineto{\pgfqpoint{3.336310in}{1.843521in}}%
\pgfpathlineto{\pgfqpoint{3.309848in}{1.843521in}}%
\pgfpathlineto{\pgfqpoint{3.309848in}{0.500000in}}%
\pgfpathclose%
\pgfusepath{fill}%
\end{pgfscope}%
\begin{pgfscope}%
\pgfpathrectangle{\pgfqpoint{0.750000in}{0.500000in}}{\pgfqpoint{4.650000in}{3.020000in}}%
\pgfusepath{clip}%
\pgfsetbuttcap%
\pgfsetmiterjoin%
\definecolor{currentfill}{rgb}{0.121569,0.466667,0.705882}%
\pgfsetfillcolor{currentfill}%
\pgfsetlinewidth{0.000000pt}%
\definecolor{currentstroke}{rgb}{0.000000,0.000000,0.000000}%
\pgfsetstrokecolor{currentstroke}%
\pgfsetstrokeopacity{0.000000}%
\pgfsetdash{}{0pt}%
\pgfpathmoveto{\pgfqpoint{3.342926in}{0.500000in}}%
\pgfpathlineto{\pgfqpoint{3.369388in}{0.500000in}}%
\pgfpathlineto{\pgfqpoint{3.369388in}{1.563621in}}%
\pgfpathlineto{\pgfqpoint{3.342926in}{1.563621in}}%
\pgfpathlineto{\pgfqpoint{3.342926in}{0.500000in}}%
\pgfpathclose%
\pgfusepath{fill}%
\end{pgfscope}%
\begin{pgfscope}%
\pgfpathrectangle{\pgfqpoint{0.750000in}{0.500000in}}{\pgfqpoint{4.650000in}{3.020000in}}%
\pgfusepath{clip}%
\pgfsetbuttcap%
\pgfsetmiterjoin%
\definecolor{currentfill}{rgb}{0.121569,0.466667,0.705882}%
\pgfsetfillcolor{currentfill}%
\pgfsetlinewidth{0.000000pt}%
\definecolor{currentstroke}{rgb}{0.000000,0.000000,0.000000}%
\pgfsetstrokecolor{currentstroke}%
\pgfsetstrokeopacity{0.000000}%
\pgfsetdash{}{0pt}%
\pgfpathmoveto{\pgfqpoint{3.376003in}{0.500000in}}%
\pgfpathlineto{\pgfqpoint{3.402465in}{0.500000in}}%
\pgfpathlineto{\pgfqpoint{3.402465in}{1.315928in}}%
\pgfpathlineto{\pgfqpoint{3.376003in}{1.315928in}}%
\pgfpathlineto{\pgfqpoint{3.376003in}{0.500000in}}%
\pgfpathclose%
\pgfusepath{fill}%
\end{pgfscope}%
\begin{pgfscope}%
\pgfpathrectangle{\pgfqpoint{0.750000in}{0.500000in}}{\pgfqpoint{4.650000in}{3.020000in}}%
\pgfusepath{clip}%
\pgfsetbuttcap%
\pgfsetmiterjoin%
\definecolor{currentfill}{rgb}{0.121569,0.466667,0.705882}%
\pgfsetfillcolor{currentfill}%
\pgfsetlinewidth{0.000000pt}%
\definecolor{currentstroke}{rgb}{0.000000,0.000000,0.000000}%
\pgfsetstrokecolor{currentstroke}%
\pgfsetstrokeopacity{0.000000}%
\pgfsetdash{}{0pt}%
\pgfpathmoveto{\pgfqpoint{3.409080in}{0.500000in}}%
\pgfpathlineto{\pgfqpoint{3.435542in}{0.500000in}}%
\pgfpathlineto{\pgfqpoint{3.435542in}{1.106433in}}%
\pgfpathlineto{\pgfqpoint{3.409080in}{1.106433in}}%
\pgfpathlineto{\pgfqpoint{3.409080in}{0.500000in}}%
\pgfpathclose%
\pgfusepath{fill}%
\end{pgfscope}%
\begin{pgfscope}%
\pgfpathrectangle{\pgfqpoint{0.750000in}{0.500000in}}{\pgfqpoint{4.650000in}{3.020000in}}%
\pgfusepath{clip}%
\pgfsetbuttcap%
\pgfsetmiterjoin%
\definecolor{currentfill}{rgb}{0.121569,0.466667,0.705882}%
\pgfsetfillcolor{currentfill}%
\pgfsetlinewidth{0.000000pt}%
\definecolor{currentstroke}{rgb}{0.000000,0.000000,0.000000}%
\pgfsetstrokecolor{currentstroke}%
\pgfsetstrokeopacity{0.000000}%
\pgfsetdash{}{0pt}%
\pgfpathmoveto{\pgfqpoint{3.442157in}{0.500000in}}%
\pgfpathlineto{\pgfqpoint{3.468619in}{0.500000in}}%
\pgfpathlineto{\pgfqpoint{3.468619in}{0.936632in}}%
\pgfpathlineto{\pgfqpoint{3.442157in}{0.936632in}}%
\pgfpathlineto{\pgfqpoint{3.442157in}{0.500000in}}%
\pgfpathclose%
\pgfusepath{fill}%
\end{pgfscope}%
\begin{pgfscope}%
\pgfpathrectangle{\pgfqpoint{0.750000in}{0.500000in}}{\pgfqpoint{4.650000in}{3.020000in}}%
\pgfusepath{clip}%
\pgfsetbuttcap%
\pgfsetmiterjoin%
\definecolor{currentfill}{rgb}{0.121569,0.466667,0.705882}%
\pgfsetfillcolor{currentfill}%
\pgfsetlinewidth{0.000000pt}%
\definecolor{currentstroke}{rgb}{0.000000,0.000000,0.000000}%
\pgfsetstrokecolor{currentstroke}%
\pgfsetstrokeopacity{0.000000}%
\pgfsetdash{}{0pt}%
\pgfpathmoveto{\pgfqpoint{3.475235in}{0.500000in}}%
\pgfpathlineto{\pgfqpoint{3.501697in}{0.500000in}}%
\pgfpathlineto{\pgfqpoint{3.501697in}{0.804493in}}%
\pgfpathlineto{\pgfqpoint{3.475235in}{0.804493in}}%
\pgfpathlineto{\pgfqpoint{3.475235in}{0.500000in}}%
\pgfpathclose%
\pgfusepath{fill}%
\end{pgfscope}%
\begin{pgfscope}%
\pgfpathrectangle{\pgfqpoint{0.750000in}{0.500000in}}{\pgfqpoint{4.650000in}{3.020000in}}%
\pgfusepath{clip}%
\pgfsetbuttcap%
\pgfsetmiterjoin%
\definecolor{currentfill}{rgb}{0.121569,0.466667,0.705882}%
\pgfsetfillcolor{currentfill}%
\pgfsetlinewidth{0.000000pt}%
\definecolor{currentstroke}{rgb}{0.000000,0.000000,0.000000}%
\pgfsetstrokecolor{currentstroke}%
\pgfsetstrokeopacity{0.000000}%
\pgfsetdash{}{0pt}%
\pgfpathmoveto{\pgfqpoint{3.508312in}{0.500000in}}%
\pgfpathlineto{\pgfqpoint{3.534774in}{0.500000in}}%
\pgfpathlineto{\pgfqpoint{3.534774in}{0.705632in}}%
\pgfpathlineto{\pgfqpoint{3.508312in}{0.705632in}}%
\pgfpathlineto{\pgfqpoint{3.508312in}{0.500000in}}%
\pgfpathclose%
\pgfusepath{fill}%
\end{pgfscope}%
\begin{pgfscope}%
\pgfpathrectangle{\pgfqpoint{0.750000in}{0.500000in}}{\pgfqpoint{4.650000in}{3.020000in}}%
\pgfusepath{clip}%
\pgfsetbuttcap%
\pgfsetmiterjoin%
\definecolor{currentfill}{rgb}{0.121569,0.466667,0.705882}%
\pgfsetfillcolor{currentfill}%
\pgfsetlinewidth{0.000000pt}%
\definecolor{currentstroke}{rgb}{0.000000,0.000000,0.000000}%
\pgfsetstrokecolor{currentstroke}%
\pgfsetstrokeopacity{0.000000}%
\pgfsetdash{}{0pt}%
\pgfpathmoveto{\pgfqpoint{3.541389in}{0.500000in}}%
\pgfpathlineto{\pgfqpoint{3.567851in}{0.500000in}}%
\pgfpathlineto{\pgfqpoint{3.567851in}{0.634452in}}%
\pgfpathlineto{\pgfqpoint{3.541389in}{0.634452in}}%
\pgfpathlineto{\pgfqpoint{3.541389in}{0.500000in}}%
\pgfpathclose%
\pgfusepath{fill}%
\end{pgfscope}%
\begin{pgfscope}%
\pgfpathrectangle{\pgfqpoint{0.750000in}{0.500000in}}{\pgfqpoint{4.650000in}{3.020000in}}%
\pgfusepath{clip}%
\pgfsetbuttcap%
\pgfsetmiterjoin%
\definecolor{currentfill}{rgb}{0.121569,0.466667,0.705882}%
\pgfsetfillcolor{currentfill}%
\pgfsetlinewidth{0.000000pt}%
\definecolor{currentstroke}{rgb}{0.000000,0.000000,0.000000}%
\pgfsetstrokecolor{currentstroke}%
\pgfsetstrokeopacity{0.000000}%
\pgfsetdash{}{0pt}%
\pgfpathmoveto{\pgfqpoint{3.574466in}{0.500000in}}%
\pgfpathlineto{\pgfqpoint{3.600928in}{0.500000in}}%
\pgfpathlineto{\pgfqpoint{3.600928in}{0.585096in}}%
\pgfpathlineto{\pgfqpoint{3.574466in}{0.585096in}}%
\pgfpathlineto{\pgfqpoint{3.574466in}{0.500000in}}%
\pgfpathclose%
\pgfusepath{fill}%
\end{pgfscope}%
\begin{pgfscope}%
\pgfpathrectangle{\pgfqpoint{0.750000in}{0.500000in}}{\pgfqpoint{4.650000in}{3.020000in}}%
\pgfusepath{clip}%
\pgfsetbuttcap%
\pgfsetmiterjoin%
\definecolor{currentfill}{rgb}{0.121569,0.466667,0.705882}%
\pgfsetfillcolor{currentfill}%
\pgfsetlinewidth{0.000000pt}%
\definecolor{currentstroke}{rgb}{0.000000,0.000000,0.000000}%
\pgfsetstrokecolor{currentstroke}%
\pgfsetstrokeopacity{0.000000}%
\pgfsetdash{}{0pt}%
\pgfpathmoveto{\pgfqpoint{3.607544in}{0.500000in}}%
\pgfpathlineto{\pgfqpoint{3.634006in}{0.500000in}}%
\pgfpathlineto{\pgfqpoint{3.634006in}{0.552121in}}%
\pgfpathlineto{\pgfqpoint{3.607544in}{0.552121in}}%
\pgfpathlineto{\pgfqpoint{3.607544in}{0.500000in}}%
\pgfpathclose%
\pgfusepath{fill}%
\end{pgfscope}%
\begin{pgfscope}%
\pgfpathrectangle{\pgfqpoint{0.750000in}{0.500000in}}{\pgfqpoint{4.650000in}{3.020000in}}%
\pgfusepath{clip}%
\pgfsetbuttcap%
\pgfsetmiterjoin%
\definecolor{currentfill}{rgb}{0.121569,0.466667,0.705882}%
\pgfsetfillcolor{currentfill}%
\pgfsetlinewidth{0.000000pt}%
\definecolor{currentstroke}{rgb}{0.000000,0.000000,0.000000}%
\pgfsetstrokecolor{currentstroke}%
\pgfsetstrokeopacity{0.000000}%
\pgfsetdash{}{0pt}%
\pgfpathmoveto{\pgfqpoint{3.640621in}{0.500000in}}%
\pgfpathlineto{\pgfqpoint{3.667083in}{0.500000in}}%
\pgfpathlineto{\pgfqpoint{3.667083in}{0.530887in}}%
\pgfpathlineto{\pgfqpoint{3.640621in}{0.530887in}}%
\pgfpathlineto{\pgfqpoint{3.640621in}{0.500000in}}%
\pgfpathclose%
\pgfusepath{fill}%
\end{pgfscope}%
\begin{pgfscope}%
\pgfpathrectangle{\pgfqpoint{0.750000in}{0.500000in}}{\pgfqpoint{4.650000in}{3.020000in}}%
\pgfusepath{clip}%
\pgfsetbuttcap%
\pgfsetmiterjoin%
\definecolor{currentfill}{rgb}{0.121569,0.466667,0.705882}%
\pgfsetfillcolor{currentfill}%
\pgfsetlinewidth{0.000000pt}%
\definecolor{currentstroke}{rgb}{0.000000,0.000000,0.000000}%
\pgfsetstrokecolor{currentstroke}%
\pgfsetstrokeopacity{0.000000}%
\pgfsetdash{}{0pt}%
\pgfpathmoveto{\pgfqpoint{3.673698in}{0.500000in}}%
\pgfpathlineto{\pgfqpoint{3.700160in}{0.500000in}}%
\pgfpathlineto{\pgfqpoint{3.700160in}{0.517703in}}%
\pgfpathlineto{\pgfqpoint{3.673698in}{0.517703in}}%
\pgfpathlineto{\pgfqpoint{3.673698in}{0.500000in}}%
\pgfpathclose%
\pgfusepath{fill}%
\end{pgfscope}%
\begin{pgfscope}%
\pgfpathrectangle{\pgfqpoint{0.750000in}{0.500000in}}{\pgfqpoint{4.650000in}{3.020000in}}%
\pgfusepath{clip}%
\pgfsetbuttcap%
\pgfsetmiterjoin%
\definecolor{currentfill}{rgb}{0.121569,0.466667,0.705882}%
\pgfsetfillcolor{currentfill}%
\pgfsetlinewidth{0.000000pt}%
\definecolor{currentstroke}{rgb}{0.000000,0.000000,0.000000}%
\pgfsetstrokecolor{currentstroke}%
\pgfsetstrokeopacity{0.000000}%
\pgfsetdash{}{0pt}%
\pgfpathmoveto{\pgfqpoint{3.706776in}{0.500000in}}%
\pgfpathlineto{\pgfqpoint{3.733237in}{0.500000in}}%
\pgfpathlineto{\pgfqpoint{3.733237in}{0.509811in}}%
\pgfpathlineto{\pgfqpoint{3.706776in}{0.509811in}}%
\pgfpathlineto{\pgfqpoint{3.706776in}{0.500000in}}%
\pgfpathclose%
\pgfusepath{fill}%
\end{pgfscope}%
\begin{pgfscope}%
\pgfpathrectangle{\pgfqpoint{0.750000in}{0.500000in}}{\pgfqpoint{4.650000in}{3.020000in}}%
\pgfusepath{clip}%
\pgfsetbuttcap%
\pgfsetmiterjoin%
\definecolor{currentfill}{rgb}{0.121569,0.466667,0.705882}%
\pgfsetfillcolor{currentfill}%
\pgfsetlinewidth{0.000000pt}%
\definecolor{currentstroke}{rgb}{0.000000,0.000000,0.000000}%
\pgfsetstrokecolor{currentstroke}%
\pgfsetstrokeopacity{0.000000}%
\pgfsetdash{}{0pt}%
\pgfpathmoveto{\pgfqpoint{3.739853in}{0.500000in}}%
\pgfpathlineto{\pgfqpoint{3.766315in}{0.500000in}}%
\pgfpathlineto{\pgfqpoint{3.766315in}{0.505256in}}%
\pgfpathlineto{\pgfqpoint{3.739853in}{0.505256in}}%
\pgfpathlineto{\pgfqpoint{3.739853in}{0.500000in}}%
\pgfpathclose%
\pgfusepath{fill}%
\end{pgfscope}%
\begin{pgfscope}%
\pgfpathrectangle{\pgfqpoint{0.750000in}{0.500000in}}{\pgfqpoint{4.650000in}{3.020000in}}%
\pgfusepath{clip}%
\pgfsetbuttcap%
\pgfsetmiterjoin%
\definecolor{currentfill}{rgb}{0.121569,0.466667,0.705882}%
\pgfsetfillcolor{currentfill}%
\pgfsetlinewidth{0.000000pt}%
\definecolor{currentstroke}{rgb}{0.000000,0.000000,0.000000}%
\pgfsetstrokecolor{currentstroke}%
\pgfsetstrokeopacity{0.000000}%
\pgfsetdash{}{0pt}%
\pgfpathmoveto{\pgfqpoint{3.772930in}{0.500000in}}%
\pgfpathlineto{\pgfqpoint{3.799392in}{0.500000in}}%
\pgfpathlineto{\pgfqpoint{3.799392in}{0.502721in}}%
\pgfpathlineto{\pgfqpoint{3.772930in}{0.502721in}}%
\pgfpathlineto{\pgfqpoint{3.772930in}{0.500000in}}%
\pgfpathclose%
\pgfusepath{fill}%
\end{pgfscope}%
\begin{pgfscope}%
\pgfpathrectangle{\pgfqpoint{0.750000in}{0.500000in}}{\pgfqpoint{4.650000in}{3.020000in}}%
\pgfusepath{clip}%
\pgfsetbuttcap%
\pgfsetmiterjoin%
\definecolor{currentfill}{rgb}{0.121569,0.466667,0.705882}%
\pgfsetfillcolor{currentfill}%
\pgfsetlinewidth{0.000000pt}%
\definecolor{currentstroke}{rgb}{0.000000,0.000000,0.000000}%
\pgfsetstrokecolor{currentstroke}%
\pgfsetstrokeopacity{0.000000}%
\pgfsetdash{}{0pt}%
\pgfpathmoveto{\pgfqpoint{3.806007in}{0.500000in}}%
\pgfpathlineto{\pgfqpoint{3.832469in}{0.500000in}}%
\pgfpathlineto{\pgfqpoint{3.832469in}{0.501360in}}%
\pgfpathlineto{\pgfqpoint{3.806007in}{0.501360in}}%
\pgfpathlineto{\pgfqpoint{3.806007in}{0.500000in}}%
\pgfpathclose%
\pgfusepath{fill}%
\end{pgfscope}%
\begin{pgfscope}%
\pgfpathrectangle{\pgfqpoint{0.750000in}{0.500000in}}{\pgfqpoint{4.650000in}{3.020000in}}%
\pgfusepath{clip}%
\pgfsetbuttcap%
\pgfsetmiterjoin%
\definecolor{currentfill}{rgb}{0.121569,0.466667,0.705882}%
\pgfsetfillcolor{currentfill}%
\pgfsetlinewidth{0.000000pt}%
\definecolor{currentstroke}{rgb}{0.000000,0.000000,0.000000}%
\pgfsetstrokecolor{currentstroke}%
\pgfsetstrokeopacity{0.000000}%
\pgfsetdash{}{0pt}%
\pgfpathmoveto{\pgfqpoint{3.839085in}{0.500000in}}%
\pgfpathlineto{\pgfqpoint{3.865546in}{0.500000in}}%
\pgfpathlineto{\pgfqpoint{3.865546in}{0.500657in}}%
\pgfpathlineto{\pgfqpoint{3.839085in}{0.500657in}}%
\pgfpathlineto{\pgfqpoint{3.839085in}{0.500000in}}%
\pgfpathclose%
\pgfusepath{fill}%
\end{pgfscope}%
\begin{pgfscope}%
\pgfpathrectangle{\pgfqpoint{0.750000in}{0.500000in}}{\pgfqpoint{4.650000in}{3.020000in}}%
\pgfusepath{clip}%
\pgfsetbuttcap%
\pgfsetmiterjoin%
\definecolor{currentfill}{rgb}{0.121569,0.466667,0.705882}%
\pgfsetfillcolor{currentfill}%
\pgfsetlinewidth{0.000000pt}%
\definecolor{currentstroke}{rgb}{0.000000,0.000000,0.000000}%
\pgfsetstrokecolor{currentstroke}%
\pgfsetstrokeopacity{0.000000}%
\pgfsetdash{}{0pt}%
\pgfpathmoveto{\pgfqpoint{3.872162in}{0.500000in}}%
\pgfpathlineto{\pgfqpoint{3.898624in}{0.500000in}}%
\pgfpathlineto{\pgfqpoint{3.898624in}{0.500306in}}%
\pgfpathlineto{\pgfqpoint{3.872162in}{0.500306in}}%
\pgfpathlineto{\pgfqpoint{3.872162in}{0.500000in}}%
\pgfpathclose%
\pgfusepath{fill}%
\end{pgfscope}%
\begin{pgfscope}%
\pgfpathrectangle{\pgfqpoint{0.750000in}{0.500000in}}{\pgfqpoint{4.650000in}{3.020000in}}%
\pgfusepath{clip}%
\pgfsetbuttcap%
\pgfsetmiterjoin%
\definecolor{currentfill}{rgb}{0.121569,0.466667,0.705882}%
\pgfsetfillcolor{currentfill}%
\pgfsetlinewidth{0.000000pt}%
\definecolor{currentstroke}{rgb}{0.000000,0.000000,0.000000}%
\pgfsetstrokecolor{currentstroke}%
\pgfsetstrokeopacity{0.000000}%
\pgfsetdash{}{0pt}%
\pgfpathmoveto{\pgfqpoint{3.905239in}{0.500000in}}%
\pgfpathlineto{\pgfqpoint{3.931701in}{0.500000in}}%
\pgfpathlineto{\pgfqpoint{3.931701in}{0.500138in}}%
\pgfpathlineto{\pgfqpoint{3.905239in}{0.500138in}}%
\pgfpathlineto{\pgfqpoint{3.905239in}{0.500000in}}%
\pgfpathclose%
\pgfusepath{fill}%
\end{pgfscope}%
\begin{pgfscope}%
\pgfpathrectangle{\pgfqpoint{0.750000in}{0.500000in}}{\pgfqpoint{4.650000in}{3.020000in}}%
\pgfusepath{clip}%
\pgfsetbuttcap%
\pgfsetmiterjoin%
\definecolor{currentfill}{rgb}{0.121569,0.466667,0.705882}%
\pgfsetfillcolor{currentfill}%
\pgfsetlinewidth{0.000000pt}%
\definecolor{currentstroke}{rgb}{0.000000,0.000000,0.000000}%
\pgfsetstrokecolor{currentstroke}%
\pgfsetstrokeopacity{0.000000}%
\pgfsetdash{}{0pt}%
\pgfpathmoveto{\pgfqpoint{3.938316in}{0.500000in}}%
\pgfpathlineto{\pgfqpoint{3.964778in}{0.500000in}}%
\pgfpathlineto{\pgfqpoint{3.964778in}{0.500060in}}%
\pgfpathlineto{\pgfqpoint{3.938316in}{0.500060in}}%
\pgfpathlineto{\pgfqpoint{3.938316in}{0.500000in}}%
\pgfpathclose%
\pgfusepath{fill}%
\end{pgfscope}%
\begin{pgfscope}%
\pgfpathrectangle{\pgfqpoint{0.750000in}{0.500000in}}{\pgfqpoint{4.650000in}{3.020000in}}%
\pgfusepath{clip}%
\pgfsetbuttcap%
\pgfsetmiterjoin%
\definecolor{currentfill}{rgb}{0.121569,0.466667,0.705882}%
\pgfsetfillcolor{currentfill}%
\pgfsetlinewidth{0.000000pt}%
\definecolor{currentstroke}{rgb}{0.000000,0.000000,0.000000}%
\pgfsetstrokecolor{currentstroke}%
\pgfsetstrokeopacity{0.000000}%
\pgfsetdash{}{0pt}%
\pgfpathmoveto{\pgfqpoint{3.971394in}{0.500000in}}%
\pgfpathlineto{\pgfqpoint{3.997855in}{0.500000in}}%
\pgfpathlineto{\pgfqpoint{3.997855in}{0.500025in}}%
\pgfpathlineto{\pgfqpoint{3.971394in}{0.500025in}}%
\pgfpathlineto{\pgfqpoint{3.971394in}{0.500000in}}%
\pgfpathclose%
\pgfusepath{fill}%
\end{pgfscope}%
\begin{pgfscope}%
\pgfpathrectangle{\pgfqpoint{0.750000in}{0.500000in}}{\pgfqpoint{4.650000in}{3.020000in}}%
\pgfusepath{clip}%
\pgfsetbuttcap%
\pgfsetmiterjoin%
\definecolor{currentfill}{rgb}{0.121569,0.466667,0.705882}%
\pgfsetfillcolor{currentfill}%
\pgfsetlinewidth{0.000000pt}%
\definecolor{currentstroke}{rgb}{0.000000,0.000000,0.000000}%
\pgfsetstrokecolor{currentstroke}%
\pgfsetstrokeopacity{0.000000}%
\pgfsetdash{}{0pt}%
\pgfpathmoveto{\pgfqpoint{4.004471in}{0.500000in}}%
\pgfpathlineto{\pgfqpoint{4.030933in}{0.500000in}}%
\pgfpathlineto{\pgfqpoint{4.030933in}{0.500010in}}%
\pgfpathlineto{\pgfqpoint{4.004471in}{0.500010in}}%
\pgfpathlineto{\pgfqpoint{4.004471in}{0.500000in}}%
\pgfpathclose%
\pgfusepath{fill}%
\end{pgfscope}%
\begin{pgfscope}%
\pgfpathrectangle{\pgfqpoint{0.750000in}{0.500000in}}{\pgfqpoint{4.650000in}{3.020000in}}%
\pgfusepath{clip}%
\pgfsetbuttcap%
\pgfsetmiterjoin%
\definecolor{currentfill}{rgb}{0.121569,0.466667,0.705882}%
\pgfsetfillcolor{currentfill}%
\pgfsetlinewidth{0.000000pt}%
\definecolor{currentstroke}{rgb}{0.000000,0.000000,0.000000}%
\pgfsetstrokecolor{currentstroke}%
\pgfsetstrokeopacity{0.000000}%
\pgfsetdash{}{0pt}%
\pgfpathmoveto{\pgfqpoint{4.037548in}{0.500000in}}%
\pgfpathlineto{\pgfqpoint{4.064010in}{0.500000in}}%
\pgfpathlineto{\pgfqpoint{4.064010in}{0.500004in}}%
\pgfpathlineto{\pgfqpoint{4.037548in}{0.500004in}}%
\pgfpathlineto{\pgfqpoint{4.037548in}{0.500000in}}%
\pgfpathclose%
\pgfusepath{fill}%
\end{pgfscope}%
\begin{pgfscope}%
\pgfpathrectangle{\pgfqpoint{0.750000in}{0.500000in}}{\pgfqpoint{4.650000in}{3.020000in}}%
\pgfusepath{clip}%
\pgfsetbuttcap%
\pgfsetmiterjoin%
\definecolor{currentfill}{rgb}{0.121569,0.466667,0.705882}%
\pgfsetfillcolor{currentfill}%
\pgfsetlinewidth{0.000000pt}%
\definecolor{currentstroke}{rgb}{0.000000,0.000000,0.000000}%
\pgfsetstrokecolor{currentstroke}%
\pgfsetstrokeopacity{0.000000}%
\pgfsetdash{}{0pt}%
\pgfpathmoveto{\pgfqpoint{4.070625in}{0.500000in}}%
\pgfpathlineto{\pgfqpoint{4.097087in}{0.500000in}}%
\pgfpathlineto{\pgfqpoint{4.097087in}{0.500001in}}%
\pgfpathlineto{\pgfqpoint{4.070625in}{0.500001in}}%
\pgfpathlineto{\pgfqpoint{4.070625in}{0.500000in}}%
\pgfpathclose%
\pgfusepath{fill}%
\end{pgfscope}%
\begin{pgfscope}%
\pgfpathrectangle{\pgfqpoint{0.750000in}{0.500000in}}{\pgfqpoint{4.650000in}{3.020000in}}%
\pgfusepath{clip}%
\pgfsetbuttcap%
\pgfsetmiterjoin%
\definecolor{currentfill}{rgb}{0.121569,0.466667,0.705882}%
\pgfsetfillcolor{currentfill}%
\pgfsetlinewidth{0.000000pt}%
\definecolor{currentstroke}{rgb}{0.000000,0.000000,0.000000}%
\pgfsetstrokecolor{currentstroke}%
\pgfsetstrokeopacity{0.000000}%
\pgfsetdash{}{0pt}%
\pgfpathmoveto{\pgfqpoint{4.103703in}{0.500000in}}%
\pgfpathlineto{\pgfqpoint{4.130164in}{0.500000in}}%
\pgfpathlineto{\pgfqpoint{4.130164in}{0.500001in}}%
\pgfpathlineto{\pgfqpoint{4.103703in}{0.500001in}}%
\pgfpathlineto{\pgfqpoint{4.103703in}{0.500000in}}%
\pgfpathclose%
\pgfusepath{fill}%
\end{pgfscope}%
\begin{pgfscope}%
\pgfpathrectangle{\pgfqpoint{0.750000in}{0.500000in}}{\pgfqpoint{4.650000in}{3.020000in}}%
\pgfusepath{clip}%
\pgfsetbuttcap%
\pgfsetmiterjoin%
\definecolor{currentfill}{rgb}{0.121569,0.466667,0.705882}%
\pgfsetfillcolor{currentfill}%
\pgfsetlinewidth{0.000000pt}%
\definecolor{currentstroke}{rgb}{0.000000,0.000000,0.000000}%
\pgfsetstrokecolor{currentstroke}%
\pgfsetstrokeopacity{0.000000}%
\pgfsetdash{}{0pt}%
\pgfpathmoveto{\pgfqpoint{4.136780in}{0.500000in}}%
\pgfpathlineto{\pgfqpoint{4.163242in}{0.500000in}}%
\pgfpathlineto{\pgfqpoint{4.163242in}{0.500000in}}%
\pgfpathlineto{\pgfqpoint{4.136780in}{0.500000in}}%
\pgfpathlineto{\pgfqpoint{4.136780in}{0.500000in}}%
\pgfpathclose%
\pgfusepath{fill}%
\end{pgfscope}%
\begin{pgfscope}%
\pgfpathrectangle{\pgfqpoint{0.750000in}{0.500000in}}{\pgfqpoint{4.650000in}{3.020000in}}%
\pgfusepath{clip}%
\pgfsetbuttcap%
\pgfsetmiterjoin%
\definecolor{currentfill}{rgb}{0.121569,0.466667,0.705882}%
\pgfsetfillcolor{currentfill}%
\pgfsetlinewidth{0.000000pt}%
\definecolor{currentstroke}{rgb}{0.000000,0.000000,0.000000}%
\pgfsetstrokecolor{currentstroke}%
\pgfsetstrokeopacity{0.000000}%
\pgfsetdash{}{0pt}%
\pgfpathmoveto{\pgfqpoint{4.169857in}{0.500000in}}%
\pgfpathlineto{\pgfqpoint{4.196319in}{0.500000in}}%
\pgfpathlineto{\pgfqpoint{4.196319in}{0.500000in}}%
\pgfpathlineto{\pgfqpoint{4.169857in}{0.500000in}}%
\pgfpathlineto{\pgfqpoint{4.169857in}{0.500000in}}%
\pgfpathclose%
\pgfusepath{fill}%
\end{pgfscope}%
\begin{pgfscope}%
\pgfpathrectangle{\pgfqpoint{0.750000in}{0.500000in}}{\pgfqpoint{4.650000in}{3.020000in}}%
\pgfusepath{clip}%
\pgfsetbuttcap%
\pgfsetmiterjoin%
\definecolor{currentfill}{rgb}{0.121569,0.466667,0.705882}%
\pgfsetfillcolor{currentfill}%
\pgfsetlinewidth{0.000000pt}%
\definecolor{currentstroke}{rgb}{0.000000,0.000000,0.000000}%
\pgfsetstrokecolor{currentstroke}%
\pgfsetstrokeopacity{0.000000}%
\pgfsetdash{}{0pt}%
\pgfpathmoveto{\pgfqpoint{4.202934in}{0.500000in}}%
\pgfpathlineto{\pgfqpoint{4.229396in}{0.500000in}}%
\pgfpathlineto{\pgfqpoint{4.229396in}{0.500000in}}%
\pgfpathlineto{\pgfqpoint{4.202934in}{0.500000in}}%
\pgfpathlineto{\pgfqpoint{4.202934in}{0.500000in}}%
\pgfpathclose%
\pgfusepath{fill}%
\end{pgfscope}%
\begin{pgfscope}%
\pgfpathrectangle{\pgfqpoint{0.750000in}{0.500000in}}{\pgfqpoint{4.650000in}{3.020000in}}%
\pgfusepath{clip}%
\pgfsetbuttcap%
\pgfsetmiterjoin%
\definecolor{currentfill}{rgb}{0.121569,0.466667,0.705882}%
\pgfsetfillcolor{currentfill}%
\pgfsetlinewidth{0.000000pt}%
\definecolor{currentstroke}{rgb}{0.000000,0.000000,0.000000}%
\pgfsetstrokecolor{currentstroke}%
\pgfsetstrokeopacity{0.000000}%
\pgfsetdash{}{0pt}%
\pgfpathmoveto{\pgfqpoint{4.236012in}{0.500000in}}%
\pgfpathlineto{\pgfqpoint{4.262473in}{0.500000in}}%
\pgfpathlineto{\pgfqpoint{4.262473in}{0.500000in}}%
\pgfpathlineto{\pgfqpoint{4.236012in}{0.500000in}}%
\pgfpathlineto{\pgfqpoint{4.236012in}{0.500000in}}%
\pgfpathclose%
\pgfusepath{fill}%
\end{pgfscope}%
\begin{pgfscope}%
\pgfpathrectangle{\pgfqpoint{0.750000in}{0.500000in}}{\pgfqpoint{4.650000in}{3.020000in}}%
\pgfusepath{clip}%
\pgfsetbuttcap%
\pgfsetmiterjoin%
\definecolor{currentfill}{rgb}{0.121569,0.466667,0.705882}%
\pgfsetfillcolor{currentfill}%
\pgfsetlinewidth{0.000000pt}%
\definecolor{currentstroke}{rgb}{0.000000,0.000000,0.000000}%
\pgfsetstrokecolor{currentstroke}%
\pgfsetstrokeopacity{0.000000}%
\pgfsetdash{}{0pt}%
\pgfpathmoveto{\pgfqpoint{4.269089in}{0.500000in}}%
\pgfpathlineto{\pgfqpoint{4.295551in}{0.500000in}}%
\pgfpathlineto{\pgfqpoint{4.295551in}{0.500000in}}%
\pgfpathlineto{\pgfqpoint{4.269089in}{0.500000in}}%
\pgfpathlineto{\pgfqpoint{4.269089in}{0.500000in}}%
\pgfpathclose%
\pgfusepath{fill}%
\end{pgfscope}%
\begin{pgfscope}%
\pgfpathrectangle{\pgfqpoint{0.750000in}{0.500000in}}{\pgfqpoint{4.650000in}{3.020000in}}%
\pgfusepath{clip}%
\pgfsetbuttcap%
\pgfsetmiterjoin%
\definecolor{currentfill}{rgb}{0.121569,0.466667,0.705882}%
\pgfsetfillcolor{currentfill}%
\pgfsetlinewidth{0.000000pt}%
\definecolor{currentstroke}{rgb}{0.000000,0.000000,0.000000}%
\pgfsetstrokecolor{currentstroke}%
\pgfsetstrokeopacity{0.000000}%
\pgfsetdash{}{0pt}%
\pgfpathmoveto{\pgfqpoint{4.302166in}{0.500000in}}%
\pgfpathlineto{\pgfqpoint{4.328628in}{0.500000in}}%
\pgfpathlineto{\pgfqpoint{4.328628in}{0.500000in}}%
\pgfpathlineto{\pgfqpoint{4.302166in}{0.500000in}}%
\pgfpathlineto{\pgfqpoint{4.302166in}{0.500000in}}%
\pgfpathclose%
\pgfusepath{fill}%
\end{pgfscope}%
\begin{pgfscope}%
\pgfpathrectangle{\pgfqpoint{0.750000in}{0.500000in}}{\pgfqpoint{4.650000in}{3.020000in}}%
\pgfusepath{clip}%
\pgfsetbuttcap%
\pgfsetmiterjoin%
\definecolor{currentfill}{rgb}{0.121569,0.466667,0.705882}%
\pgfsetfillcolor{currentfill}%
\pgfsetlinewidth{0.000000pt}%
\definecolor{currentstroke}{rgb}{0.000000,0.000000,0.000000}%
\pgfsetstrokecolor{currentstroke}%
\pgfsetstrokeopacity{0.000000}%
\pgfsetdash{}{0pt}%
\pgfpathmoveto{\pgfqpoint{4.335243in}{0.500000in}}%
\pgfpathlineto{\pgfqpoint{4.361705in}{0.500000in}}%
\pgfpathlineto{\pgfqpoint{4.361705in}{0.500000in}}%
\pgfpathlineto{\pgfqpoint{4.335243in}{0.500000in}}%
\pgfpathlineto{\pgfqpoint{4.335243in}{0.500000in}}%
\pgfpathclose%
\pgfusepath{fill}%
\end{pgfscope}%
\begin{pgfscope}%
\pgfpathrectangle{\pgfqpoint{0.750000in}{0.500000in}}{\pgfqpoint{4.650000in}{3.020000in}}%
\pgfusepath{clip}%
\pgfsetbuttcap%
\pgfsetmiterjoin%
\definecolor{currentfill}{rgb}{0.121569,0.466667,0.705882}%
\pgfsetfillcolor{currentfill}%
\pgfsetlinewidth{0.000000pt}%
\definecolor{currentstroke}{rgb}{0.000000,0.000000,0.000000}%
\pgfsetstrokecolor{currentstroke}%
\pgfsetstrokeopacity{0.000000}%
\pgfsetdash{}{0pt}%
\pgfpathmoveto{\pgfqpoint{4.368321in}{0.500000in}}%
\pgfpathlineto{\pgfqpoint{4.394782in}{0.500000in}}%
\pgfpathlineto{\pgfqpoint{4.394782in}{0.500000in}}%
\pgfpathlineto{\pgfqpoint{4.368321in}{0.500000in}}%
\pgfpathlineto{\pgfqpoint{4.368321in}{0.500000in}}%
\pgfpathclose%
\pgfusepath{fill}%
\end{pgfscope}%
\begin{pgfscope}%
\pgfpathrectangle{\pgfqpoint{0.750000in}{0.500000in}}{\pgfqpoint{4.650000in}{3.020000in}}%
\pgfusepath{clip}%
\pgfsetbuttcap%
\pgfsetmiterjoin%
\definecolor{currentfill}{rgb}{0.121569,0.466667,0.705882}%
\pgfsetfillcolor{currentfill}%
\pgfsetlinewidth{0.000000pt}%
\definecolor{currentstroke}{rgb}{0.000000,0.000000,0.000000}%
\pgfsetstrokecolor{currentstroke}%
\pgfsetstrokeopacity{0.000000}%
\pgfsetdash{}{0pt}%
\pgfpathmoveto{\pgfqpoint{4.401398in}{0.500000in}}%
\pgfpathlineto{\pgfqpoint{4.427860in}{0.500000in}}%
\pgfpathlineto{\pgfqpoint{4.427860in}{0.500000in}}%
\pgfpathlineto{\pgfqpoint{4.401398in}{0.500000in}}%
\pgfpathlineto{\pgfqpoint{4.401398in}{0.500000in}}%
\pgfpathclose%
\pgfusepath{fill}%
\end{pgfscope}%
\begin{pgfscope}%
\pgfpathrectangle{\pgfqpoint{0.750000in}{0.500000in}}{\pgfqpoint{4.650000in}{3.020000in}}%
\pgfusepath{clip}%
\pgfsetbuttcap%
\pgfsetmiterjoin%
\definecolor{currentfill}{rgb}{0.121569,0.466667,0.705882}%
\pgfsetfillcolor{currentfill}%
\pgfsetlinewidth{0.000000pt}%
\definecolor{currentstroke}{rgb}{0.000000,0.000000,0.000000}%
\pgfsetstrokecolor{currentstroke}%
\pgfsetstrokeopacity{0.000000}%
\pgfsetdash{}{0pt}%
\pgfpathmoveto{\pgfqpoint{4.434475in}{0.500000in}}%
\pgfpathlineto{\pgfqpoint{4.460937in}{0.500000in}}%
\pgfpathlineto{\pgfqpoint{4.460937in}{0.500000in}}%
\pgfpathlineto{\pgfqpoint{4.434475in}{0.500000in}}%
\pgfpathlineto{\pgfqpoint{4.434475in}{0.500000in}}%
\pgfpathclose%
\pgfusepath{fill}%
\end{pgfscope}%
\begin{pgfscope}%
\pgfpathrectangle{\pgfqpoint{0.750000in}{0.500000in}}{\pgfqpoint{4.650000in}{3.020000in}}%
\pgfusepath{clip}%
\pgfsetbuttcap%
\pgfsetmiterjoin%
\definecolor{currentfill}{rgb}{0.121569,0.466667,0.705882}%
\pgfsetfillcolor{currentfill}%
\pgfsetlinewidth{0.000000pt}%
\definecolor{currentstroke}{rgb}{0.000000,0.000000,0.000000}%
\pgfsetstrokecolor{currentstroke}%
\pgfsetstrokeopacity{0.000000}%
\pgfsetdash{}{0pt}%
\pgfpathmoveto{\pgfqpoint{4.467552in}{0.500000in}}%
\pgfpathlineto{\pgfqpoint{4.494014in}{0.500000in}}%
\pgfpathlineto{\pgfqpoint{4.494014in}{0.500000in}}%
\pgfpathlineto{\pgfqpoint{4.467552in}{0.500000in}}%
\pgfpathlineto{\pgfqpoint{4.467552in}{0.500000in}}%
\pgfpathclose%
\pgfusepath{fill}%
\end{pgfscope}%
\begin{pgfscope}%
\pgfpathrectangle{\pgfqpoint{0.750000in}{0.500000in}}{\pgfqpoint{4.650000in}{3.020000in}}%
\pgfusepath{clip}%
\pgfsetbuttcap%
\pgfsetmiterjoin%
\definecolor{currentfill}{rgb}{0.121569,0.466667,0.705882}%
\pgfsetfillcolor{currentfill}%
\pgfsetlinewidth{0.000000pt}%
\definecolor{currentstroke}{rgb}{0.000000,0.000000,0.000000}%
\pgfsetstrokecolor{currentstroke}%
\pgfsetstrokeopacity{0.000000}%
\pgfsetdash{}{0pt}%
\pgfpathmoveto{\pgfqpoint{4.500630in}{0.500000in}}%
\pgfpathlineto{\pgfqpoint{4.527091in}{0.500000in}}%
\pgfpathlineto{\pgfqpoint{4.527091in}{0.500000in}}%
\pgfpathlineto{\pgfqpoint{4.500630in}{0.500000in}}%
\pgfpathlineto{\pgfqpoint{4.500630in}{0.500000in}}%
\pgfpathclose%
\pgfusepath{fill}%
\end{pgfscope}%
\begin{pgfscope}%
\pgfpathrectangle{\pgfqpoint{0.750000in}{0.500000in}}{\pgfqpoint{4.650000in}{3.020000in}}%
\pgfusepath{clip}%
\pgfsetbuttcap%
\pgfsetmiterjoin%
\definecolor{currentfill}{rgb}{0.121569,0.466667,0.705882}%
\pgfsetfillcolor{currentfill}%
\pgfsetlinewidth{0.000000pt}%
\definecolor{currentstroke}{rgb}{0.000000,0.000000,0.000000}%
\pgfsetstrokecolor{currentstroke}%
\pgfsetstrokeopacity{0.000000}%
\pgfsetdash{}{0pt}%
\pgfpathmoveto{\pgfqpoint{4.533707in}{0.500000in}}%
\pgfpathlineto{\pgfqpoint{4.560169in}{0.500000in}}%
\pgfpathlineto{\pgfqpoint{4.560169in}{0.500000in}}%
\pgfpathlineto{\pgfqpoint{4.533707in}{0.500000in}}%
\pgfpathlineto{\pgfqpoint{4.533707in}{0.500000in}}%
\pgfpathclose%
\pgfusepath{fill}%
\end{pgfscope}%
\begin{pgfscope}%
\pgfpathrectangle{\pgfqpoint{0.750000in}{0.500000in}}{\pgfqpoint{4.650000in}{3.020000in}}%
\pgfusepath{clip}%
\pgfsetbuttcap%
\pgfsetmiterjoin%
\definecolor{currentfill}{rgb}{0.121569,0.466667,0.705882}%
\pgfsetfillcolor{currentfill}%
\pgfsetlinewidth{0.000000pt}%
\definecolor{currentstroke}{rgb}{0.000000,0.000000,0.000000}%
\pgfsetstrokecolor{currentstroke}%
\pgfsetstrokeopacity{0.000000}%
\pgfsetdash{}{0pt}%
\pgfpathmoveto{\pgfqpoint{4.566784in}{0.500000in}}%
\pgfpathlineto{\pgfqpoint{4.593246in}{0.500000in}}%
\pgfpathlineto{\pgfqpoint{4.593246in}{0.500000in}}%
\pgfpathlineto{\pgfqpoint{4.566784in}{0.500000in}}%
\pgfpathlineto{\pgfqpoint{4.566784in}{0.500000in}}%
\pgfpathclose%
\pgfusepath{fill}%
\end{pgfscope}%
\begin{pgfscope}%
\pgfpathrectangle{\pgfqpoint{0.750000in}{0.500000in}}{\pgfqpoint{4.650000in}{3.020000in}}%
\pgfusepath{clip}%
\pgfsetbuttcap%
\pgfsetmiterjoin%
\definecolor{currentfill}{rgb}{0.121569,0.466667,0.705882}%
\pgfsetfillcolor{currentfill}%
\pgfsetlinewidth{0.000000pt}%
\definecolor{currentstroke}{rgb}{0.000000,0.000000,0.000000}%
\pgfsetstrokecolor{currentstroke}%
\pgfsetstrokeopacity{0.000000}%
\pgfsetdash{}{0pt}%
\pgfpathmoveto{\pgfqpoint{4.599861in}{0.500000in}}%
\pgfpathlineto{\pgfqpoint{4.626323in}{0.500000in}}%
\pgfpathlineto{\pgfqpoint{4.626323in}{0.500000in}}%
\pgfpathlineto{\pgfqpoint{4.599861in}{0.500000in}}%
\pgfpathlineto{\pgfqpoint{4.599861in}{0.500000in}}%
\pgfpathclose%
\pgfusepath{fill}%
\end{pgfscope}%
\begin{pgfscope}%
\pgfpathrectangle{\pgfqpoint{0.750000in}{0.500000in}}{\pgfqpoint{4.650000in}{3.020000in}}%
\pgfusepath{clip}%
\pgfsetbuttcap%
\pgfsetmiterjoin%
\definecolor{currentfill}{rgb}{0.121569,0.466667,0.705882}%
\pgfsetfillcolor{currentfill}%
\pgfsetlinewidth{0.000000pt}%
\definecolor{currentstroke}{rgb}{0.000000,0.000000,0.000000}%
\pgfsetstrokecolor{currentstroke}%
\pgfsetstrokeopacity{0.000000}%
\pgfsetdash{}{0pt}%
\pgfpathmoveto{\pgfqpoint{4.632939in}{0.500000in}}%
\pgfpathlineto{\pgfqpoint{4.659400in}{0.500000in}}%
\pgfpathlineto{\pgfqpoint{4.659400in}{0.500000in}}%
\pgfpathlineto{\pgfqpoint{4.632939in}{0.500000in}}%
\pgfpathlineto{\pgfqpoint{4.632939in}{0.500000in}}%
\pgfpathclose%
\pgfusepath{fill}%
\end{pgfscope}%
\begin{pgfscope}%
\pgfpathrectangle{\pgfqpoint{0.750000in}{0.500000in}}{\pgfqpoint{4.650000in}{3.020000in}}%
\pgfusepath{clip}%
\pgfsetbuttcap%
\pgfsetmiterjoin%
\definecolor{currentfill}{rgb}{0.121569,0.466667,0.705882}%
\pgfsetfillcolor{currentfill}%
\pgfsetlinewidth{0.000000pt}%
\definecolor{currentstroke}{rgb}{0.000000,0.000000,0.000000}%
\pgfsetstrokecolor{currentstroke}%
\pgfsetstrokeopacity{0.000000}%
\pgfsetdash{}{0pt}%
\pgfpathmoveto{\pgfqpoint{4.666016in}{0.500000in}}%
\pgfpathlineto{\pgfqpoint{4.692478in}{0.500000in}}%
\pgfpathlineto{\pgfqpoint{4.692478in}{0.500000in}}%
\pgfpathlineto{\pgfqpoint{4.666016in}{0.500000in}}%
\pgfpathlineto{\pgfqpoint{4.666016in}{0.500000in}}%
\pgfpathclose%
\pgfusepath{fill}%
\end{pgfscope}%
\begin{pgfscope}%
\pgfpathrectangle{\pgfqpoint{0.750000in}{0.500000in}}{\pgfqpoint{4.650000in}{3.020000in}}%
\pgfusepath{clip}%
\pgfsetbuttcap%
\pgfsetmiterjoin%
\definecolor{currentfill}{rgb}{0.121569,0.466667,0.705882}%
\pgfsetfillcolor{currentfill}%
\pgfsetlinewidth{0.000000pt}%
\definecolor{currentstroke}{rgb}{0.000000,0.000000,0.000000}%
\pgfsetstrokecolor{currentstroke}%
\pgfsetstrokeopacity{0.000000}%
\pgfsetdash{}{0pt}%
\pgfpathmoveto{\pgfqpoint{4.699093in}{0.500000in}}%
\pgfpathlineto{\pgfqpoint{4.725555in}{0.500000in}}%
\pgfpathlineto{\pgfqpoint{4.725555in}{0.500000in}}%
\pgfpathlineto{\pgfqpoint{4.699093in}{0.500000in}}%
\pgfpathlineto{\pgfqpoint{4.699093in}{0.500000in}}%
\pgfpathclose%
\pgfusepath{fill}%
\end{pgfscope}%
\begin{pgfscope}%
\pgfpathrectangle{\pgfqpoint{0.750000in}{0.500000in}}{\pgfqpoint{4.650000in}{3.020000in}}%
\pgfusepath{clip}%
\pgfsetbuttcap%
\pgfsetmiterjoin%
\definecolor{currentfill}{rgb}{0.121569,0.466667,0.705882}%
\pgfsetfillcolor{currentfill}%
\pgfsetlinewidth{0.000000pt}%
\definecolor{currentstroke}{rgb}{0.000000,0.000000,0.000000}%
\pgfsetstrokecolor{currentstroke}%
\pgfsetstrokeopacity{0.000000}%
\pgfsetdash{}{0pt}%
\pgfpathmoveto{\pgfqpoint{4.732170in}{0.500000in}}%
\pgfpathlineto{\pgfqpoint{4.758632in}{0.500000in}}%
\pgfpathlineto{\pgfqpoint{4.758632in}{0.500000in}}%
\pgfpathlineto{\pgfqpoint{4.732170in}{0.500000in}}%
\pgfpathlineto{\pgfqpoint{4.732170in}{0.500000in}}%
\pgfpathclose%
\pgfusepath{fill}%
\end{pgfscope}%
\begin{pgfscope}%
\pgfpathrectangle{\pgfqpoint{0.750000in}{0.500000in}}{\pgfqpoint{4.650000in}{3.020000in}}%
\pgfusepath{clip}%
\pgfsetbuttcap%
\pgfsetmiterjoin%
\definecolor{currentfill}{rgb}{0.121569,0.466667,0.705882}%
\pgfsetfillcolor{currentfill}%
\pgfsetlinewidth{0.000000pt}%
\definecolor{currentstroke}{rgb}{0.000000,0.000000,0.000000}%
\pgfsetstrokecolor{currentstroke}%
\pgfsetstrokeopacity{0.000000}%
\pgfsetdash{}{0pt}%
\pgfpathmoveto{\pgfqpoint{4.765248in}{0.500000in}}%
\pgfpathlineto{\pgfqpoint{4.791709in}{0.500000in}}%
\pgfpathlineto{\pgfqpoint{4.791709in}{0.500000in}}%
\pgfpathlineto{\pgfqpoint{4.765248in}{0.500000in}}%
\pgfpathlineto{\pgfqpoint{4.765248in}{0.500000in}}%
\pgfpathclose%
\pgfusepath{fill}%
\end{pgfscope}%
\begin{pgfscope}%
\pgfpathrectangle{\pgfqpoint{0.750000in}{0.500000in}}{\pgfqpoint{4.650000in}{3.020000in}}%
\pgfusepath{clip}%
\pgfsetbuttcap%
\pgfsetmiterjoin%
\definecolor{currentfill}{rgb}{0.121569,0.466667,0.705882}%
\pgfsetfillcolor{currentfill}%
\pgfsetlinewidth{0.000000pt}%
\definecolor{currentstroke}{rgb}{0.000000,0.000000,0.000000}%
\pgfsetstrokecolor{currentstroke}%
\pgfsetstrokeopacity{0.000000}%
\pgfsetdash{}{0pt}%
\pgfpathmoveto{\pgfqpoint{4.798325in}{0.500000in}}%
\pgfpathlineto{\pgfqpoint{4.824787in}{0.500000in}}%
\pgfpathlineto{\pgfqpoint{4.824787in}{0.500000in}}%
\pgfpathlineto{\pgfqpoint{4.798325in}{0.500000in}}%
\pgfpathlineto{\pgfqpoint{4.798325in}{0.500000in}}%
\pgfpathclose%
\pgfusepath{fill}%
\end{pgfscope}%
\begin{pgfscope}%
\pgfpathrectangle{\pgfqpoint{0.750000in}{0.500000in}}{\pgfqpoint{4.650000in}{3.020000in}}%
\pgfusepath{clip}%
\pgfsetbuttcap%
\pgfsetmiterjoin%
\definecolor{currentfill}{rgb}{0.121569,0.466667,0.705882}%
\pgfsetfillcolor{currentfill}%
\pgfsetlinewidth{0.000000pt}%
\definecolor{currentstroke}{rgb}{0.000000,0.000000,0.000000}%
\pgfsetstrokecolor{currentstroke}%
\pgfsetstrokeopacity{0.000000}%
\pgfsetdash{}{0pt}%
\pgfpathmoveto{\pgfqpoint{4.831402in}{0.500000in}}%
\pgfpathlineto{\pgfqpoint{4.857864in}{0.500000in}}%
\pgfpathlineto{\pgfqpoint{4.857864in}{0.500000in}}%
\pgfpathlineto{\pgfqpoint{4.831402in}{0.500000in}}%
\pgfpathlineto{\pgfqpoint{4.831402in}{0.500000in}}%
\pgfpathclose%
\pgfusepath{fill}%
\end{pgfscope}%
\begin{pgfscope}%
\pgfpathrectangle{\pgfqpoint{0.750000in}{0.500000in}}{\pgfqpoint{4.650000in}{3.020000in}}%
\pgfusepath{clip}%
\pgfsetbuttcap%
\pgfsetmiterjoin%
\definecolor{currentfill}{rgb}{0.121569,0.466667,0.705882}%
\pgfsetfillcolor{currentfill}%
\pgfsetlinewidth{0.000000pt}%
\definecolor{currentstroke}{rgb}{0.000000,0.000000,0.000000}%
\pgfsetstrokecolor{currentstroke}%
\pgfsetstrokeopacity{0.000000}%
\pgfsetdash{}{0pt}%
\pgfpathmoveto{\pgfqpoint{4.864479in}{0.500000in}}%
\pgfpathlineto{\pgfqpoint{4.890941in}{0.500000in}}%
\pgfpathlineto{\pgfqpoint{4.890941in}{0.500000in}}%
\pgfpathlineto{\pgfqpoint{4.864479in}{0.500000in}}%
\pgfpathlineto{\pgfqpoint{4.864479in}{0.500000in}}%
\pgfpathclose%
\pgfusepath{fill}%
\end{pgfscope}%
\begin{pgfscope}%
\pgfpathrectangle{\pgfqpoint{0.750000in}{0.500000in}}{\pgfqpoint{4.650000in}{3.020000in}}%
\pgfusepath{clip}%
\pgfsetbuttcap%
\pgfsetmiterjoin%
\definecolor{currentfill}{rgb}{0.121569,0.466667,0.705882}%
\pgfsetfillcolor{currentfill}%
\pgfsetlinewidth{0.000000pt}%
\definecolor{currentstroke}{rgb}{0.000000,0.000000,0.000000}%
\pgfsetstrokecolor{currentstroke}%
\pgfsetstrokeopacity{0.000000}%
\pgfsetdash{}{0pt}%
\pgfpathmoveto{\pgfqpoint{4.897557in}{0.500000in}}%
\pgfpathlineto{\pgfqpoint{4.924018in}{0.500000in}}%
\pgfpathlineto{\pgfqpoint{4.924018in}{0.500000in}}%
\pgfpathlineto{\pgfqpoint{4.897557in}{0.500000in}}%
\pgfpathlineto{\pgfqpoint{4.897557in}{0.500000in}}%
\pgfpathclose%
\pgfusepath{fill}%
\end{pgfscope}%
\begin{pgfscope}%
\pgfpathrectangle{\pgfqpoint{0.750000in}{0.500000in}}{\pgfqpoint{4.650000in}{3.020000in}}%
\pgfusepath{clip}%
\pgfsetbuttcap%
\pgfsetmiterjoin%
\definecolor{currentfill}{rgb}{0.121569,0.466667,0.705882}%
\pgfsetfillcolor{currentfill}%
\pgfsetlinewidth{0.000000pt}%
\definecolor{currentstroke}{rgb}{0.000000,0.000000,0.000000}%
\pgfsetstrokecolor{currentstroke}%
\pgfsetstrokeopacity{0.000000}%
\pgfsetdash{}{0pt}%
\pgfpathmoveto{\pgfqpoint{4.930634in}{0.500000in}}%
\pgfpathlineto{\pgfqpoint{4.957096in}{0.500000in}}%
\pgfpathlineto{\pgfqpoint{4.957096in}{0.500000in}}%
\pgfpathlineto{\pgfqpoint{4.930634in}{0.500000in}}%
\pgfpathlineto{\pgfqpoint{4.930634in}{0.500000in}}%
\pgfpathclose%
\pgfusepath{fill}%
\end{pgfscope}%
\begin{pgfscope}%
\pgfpathrectangle{\pgfqpoint{0.750000in}{0.500000in}}{\pgfqpoint{4.650000in}{3.020000in}}%
\pgfusepath{clip}%
\pgfsetbuttcap%
\pgfsetmiterjoin%
\definecolor{currentfill}{rgb}{0.121569,0.466667,0.705882}%
\pgfsetfillcolor{currentfill}%
\pgfsetlinewidth{0.000000pt}%
\definecolor{currentstroke}{rgb}{0.000000,0.000000,0.000000}%
\pgfsetstrokecolor{currentstroke}%
\pgfsetstrokeopacity{0.000000}%
\pgfsetdash{}{0pt}%
\pgfpathmoveto{\pgfqpoint{4.963711in}{0.500000in}}%
\pgfpathlineto{\pgfqpoint{4.990173in}{0.500000in}}%
\pgfpathlineto{\pgfqpoint{4.990173in}{0.500000in}}%
\pgfpathlineto{\pgfqpoint{4.963711in}{0.500000in}}%
\pgfpathlineto{\pgfqpoint{4.963711in}{0.500000in}}%
\pgfpathclose%
\pgfusepath{fill}%
\end{pgfscope}%
\begin{pgfscope}%
\pgfpathrectangle{\pgfqpoint{0.750000in}{0.500000in}}{\pgfqpoint{4.650000in}{3.020000in}}%
\pgfusepath{clip}%
\pgfsetbuttcap%
\pgfsetmiterjoin%
\definecolor{currentfill}{rgb}{0.121569,0.466667,0.705882}%
\pgfsetfillcolor{currentfill}%
\pgfsetlinewidth{0.000000pt}%
\definecolor{currentstroke}{rgb}{0.000000,0.000000,0.000000}%
\pgfsetstrokecolor{currentstroke}%
\pgfsetstrokeopacity{0.000000}%
\pgfsetdash{}{0pt}%
\pgfpathmoveto{\pgfqpoint{4.996788in}{0.500000in}}%
\pgfpathlineto{\pgfqpoint{5.023250in}{0.500000in}}%
\pgfpathlineto{\pgfqpoint{5.023250in}{0.500000in}}%
\pgfpathlineto{\pgfqpoint{4.996788in}{0.500000in}}%
\pgfpathlineto{\pgfqpoint{4.996788in}{0.500000in}}%
\pgfpathclose%
\pgfusepath{fill}%
\end{pgfscope}%
\begin{pgfscope}%
\pgfpathrectangle{\pgfqpoint{0.750000in}{0.500000in}}{\pgfqpoint{4.650000in}{3.020000in}}%
\pgfusepath{clip}%
\pgfsetbuttcap%
\pgfsetmiterjoin%
\definecolor{currentfill}{rgb}{0.121569,0.466667,0.705882}%
\pgfsetfillcolor{currentfill}%
\pgfsetlinewidth{0.000000pt}%
\definecolor{currentstroke}{rgb}{0.000000,0.000000,0.000000}%
\pgfsetstrokecolor{currentstroke}%
\pgfsetstrokeopacity{0.000000}%
\pgfsetdash{}{0pt}%
\pgfpathmoveto{\pgfqpoint{5.029866in}{0.500000in}}%
\pgfpathlineto{\pgfqpoint{5.056327in}{0.500000in}}%
\pgfpathlineto{\pgfqpoint{5.056327in}{0.500000in}}%
\pgfpathlineto{\pgfqpoint{5.029866in}{0.500000in}}%
\pgfpathlineto{\pgfqpoint{5.029866in}{0.500000in}}%
\pgfpathclose%
\pgfusepath{fill}%
\end{pgfscope}%
\begin{pgfscope}%
\pgfpathrectangle{\pgfqpoint{0.750000in}{0.500000in}}{\pgfqpoint{4.650000in}{3.020000in}}%
\pgfusepath{clip}%
\pgfsetbuttcap%
\pgfsetmiterjoin%
\definecolor{currentfill}{rgb}{0.121569,0.466667,0.705882}%
\pgfsetfillcolor{currentfill}%
\pgfsetlinewidth{0.000000pt}%
\definecolor{currentstroke}{rgb}{0.000000,0.000000,0.000000}%
\pgfsetstrokecolor{currentstroke}%
\pgfsetstrokeopacity{0.000000}%
\pgfsetdash{}{0pt}%
\pgfpathmoveto{\pgfqpoint{5.062943in}{0.500000in}}%
\pgfpathlineto{\pgfqpoint{5.089405in}{0.500000in}}%
\pgfpathlineto{\pgfqpoint{5.089405in}{0.500000in}}%
\pgfpathlineto{\pgfqpoint{5.062943in}{0.500000in}}%
\pgfpathlineto{\pgfqpoint{5.062943in}{0.500000in}}%
\pgfpathclose%
\pgfusepath{fill}%
\end{pgfscope}%
\begin{pgfscope}%
\pgfpathrectangle{\pgfqpoint{0.750000in}{0.500000in}}{\pgfqpoint{4.650000in}{3.020000in}}%
\pgfusepath{clip}%
\pgfsetbuttcap%
\pgfsetmiterjoin%
\definecolor{currentfill}{rgb}{0.121569,0.466667,0.705882}%
\pgfsetfillcolor{currentfill}%
\pgfsetlinewidth{0.000000pt}%
\definecolor{currentstroke}{rgb}{0.000000,0.000000,0.000000}%
\pgfsetstrokecolor{currentstroke}%
\pgfsetstrokeopacity{0.000000}%
\pgfsetdash{}{0pt}%
\pgfpathmoveto{\pgfqpoint{5.096020in}{0.500000in}}%
\pgfpathlineto{\pgfqpoint{5.122482in}{0.500000in}}%
\pgfpathlineto{\pgfqpoint{5.122482in}{0.500000in}}%
\pgfpathlineto{\pgfqpoint{5.096020in}{0.500000in}}%
\pgfpathlineto{\pgfqpoint{5.096020in}{0.500000in}}%
\pgfpathclose%
\pgfusepath{fill}%
\end{pgfscope}%
\begin{pgfscope}%
\pgfpathrectangle{\pgfqpoint{0.750000in}{0.500000in}}{\pgfqpoint{4.650000in}{3.020000in}}%
\pgfusepath{clip}%
\pgfsetbuttcap%
\pgfsetmiterjoin%
\definecolor{currentfill}{rgb}{0.121569,0.466667,0.705882}%
\pgfsetfillcolor{currentfill}%
\pgfsetlinewidth{0.000000pt}%
\definecolor{currentstroke}{rgb}{0.000000,0.000000,0.000000}%
\pgfsetstrokecolor{currentstroke}%
\pgfsetstrokeopacity{0.000000}%
\pgfsetdash{}{0pt}%
\pgfpathmoveto{\pgfqpoint{5.129097in}{0.500000in}}%
\pgfpathlineto{\pgfqpoint{5.155559in}{0.500000in}}%
\pgfpathlineto{\pgfqpoint{5.155559in}{0.500000in}}%
\pgfpathlineto{\pgfqpoint{5.129097in}{0.500000in}}%
\pgfpathlineto{\pgfqpoint{5.129097in}{0.500000in}}%
\pgfpathclose%
\pgfusepath{fill}%
\end{pgfscope}%
\begin{pgfscope}%
\pgfpathrectangle{\pgfqpoint{0.750000in}{0.500000in}}{\pgfqpoint{4.650000in}{3.020000in}}%
\pgfusepath{clip}%
\pgfsetbuttcap%
\pgfsetmiterjoin%
\definecolor{currentfill}{rgb}{0.121569,0.466667,0.705882}%
\pgfsetfillcolor{currentfill}%
\pgfsetlinewidth{0.000000pt}%
\definecolor{currentstroke}{rgb}{0.000000,0.000000,0.000000}%
\pgfsetstrokecolor{currentstroke}%
\pgfsetstrokeopacity{0.000000}%
\pgfsetdash{}{0pt}%
\pgfpathmoveto{\pgfqpoint{5.162175in}{0.500000in}}%
\pgfpathlineto{\pgfqpoint{5.188636in}{0.500000in}}%
\pgfpathlineto{\pgfqpoint{5.188636in}{0.500000in}}%
\pgfpathlineto{\pgfqpoint{5.162175in}{0.500000in}}%
\pgfpathlineto{\pgfqpoint{5.162175in}{0.500000in}}%
\pgfpathclose%
\pgfusepath{fill}%
\end{pgfscope}%
\begin{pgfscope}%
\pgfpathrectangle{\pgfqpoint{0.750000in}{0.500000in}}{\pgfqpoint{4.650000in}{3.020000in}}%
\pgfusepath{clip}%
\pgfsetbuttcap%
\pgfsetroundjoin%
\definecolor{currentfill}{rgb}{1.000000,0.000000,0.000000}%
\pgfsetfillcolor{currentfill}%
\pgfsetlinewidth{1.003750pt}%
\definecolor{currentstroke}{rgb}{1.000000,0.000000,0.000000}%
\pgfsetstrokecolor{currentstroke}%
\pgfsetdash{}{0pt}%
\pgfsys@defobject{currentmarker}{\pgfqpoint{-0.041667in}{-0.041667in}}{\pgfqpoint{0.041667in}{0.041667in}}{%
\pgfpathmoveto{\pgfqpoint{0.000000in}{-0.041667in}}%
\pgfpathcurveto{\pgfqpoint{0.011050in}{-0.041667in}}{\pgfqpoint{0.021649in}{-0.037276in}}{\pgfqpoint{0.029463in}{-0.029463in}}%
\pgfpathcurveto{\pgfqpoint{0.037276in}{-0.021649in}}{\pgfqpoint{0.041667in}{-0.011050in}}{\pgfqpoint{0.041667in}{0.000000in}}%
\pgfpathcurveto{\pgfqpoint{0.041667in}{0.011050in}}{\pgfqpoint{0.037276in}{0.021649in}}{\pgfqpoint{0.029463in}{0.029463in}}%
\pgfpathcurveto{\pgfqpoint{0.021649in}{0.037276in}}{\pgfqpoint{0.011050in}{0.041667in}}{\pgfqpoint{0.000000in}{0.041667in}}%
\pgfpathcurveto{\pgfqpoint{-0.011050in}{0.041667in}}{\pgfqpoint{-0.021649in}{0.037276in}}{\pgfqpoint{-0.029463in}{0.029463in}}%
\pgfpathcurveto{\pgfqpoint{-0.037276in}{0.021649in}}{\pgfqpoint{-0.041667in}{0.011050in}}{\pgfqpoint{-0.041667in}{0.000000in}}%
\pgfpathcurveto{\pgfqpoint{-0.041667in}{-0.011050in}}{\pgfqpoint{-0.037276in}{-0.021649in}}{\pgfqpoint{-0.029463in}{-0.029463in}}%
\pgfpathcurveto{\pgfqpoint{-0.021649in}{-0.037276in}}{\pgfqpoint{-0.011050in}{-0.041667in}}{\pgfqpoint{0.000000in}{-0.041667in}}%
\pgfpathlineto{\pgfqpoint{0.000000in}{-0.041667in}}%
\pgfpathclose%
\pgfusepath{stroke,fill}%
}%
\begin{pgfscope}%
\pgfsys@transformshift{4.150011in}{0.500000in}%
\pgfsys@useobject{currentmarker}{}%
\end{pgfscope}%
\end{pgfscope}%
\begin{pgfscope}%
\pgfpathrectangle{\pgfqpoint{0.750000in}{0.500000in}}{\pgfqpoint{4.650000in}{3.020000in}}%
\pgfusepath{clip}%
\pgfsetbuttcap%
\pgfsetroundjoin%
\definecolor{currentfill}{rgb}{0.000000,0.500000,0.000000}%
\pgfsetfillcolor{currentfill}%
\pgfsetlinewidth{1.003750pt}%
\definecolor{currentstroke}{rgb}{0.000000,0.500000,0.000000}%
\pgfsetstrokecolor{currentstroke}%
\pgfsetdash{}{0pt}%
\pgfsys@defobject{currentmarker}{\pgfqpoint{-0.041667in}{-0.041667in}}{\pgfqpoint{0.041667in}{0.041667in}}{%
\pgfpathmoveto{\pgfqpoint{0.000000in}{-0.041667in}}%
\pgfpathcurveto{\pgfqpoint{0.011050in}{-0.041667in}}{\pgfqpoint{0.021649in}{-0.037276in}}{\pgfqpoint{0.029463in}{-0.029463in}}%
\pgfpathcurveto{\pgfqpoint{0.037276in}{-0.021649in}}{\pgfqpoint{0.041667in}{-0.011050in}}{\pgfqpoint{0.041667in}{0.000000in}}%
\pgfpathcurveto{\pgfqpoint{0.041667in}{0.011050in}}{\pgfqpoint{0.037276in}{0.021649in}}{\pgfqpoint{0.029463in}{0.029463in}}%
\pgfpathcurveto{\pgfqpoint{0.021649in}{0.037276in}}{\pgfqpoint{0.011050in}{0.041667in}}{\pgfqpoint{0.000000in}{0.041667in}}%
\pgfpathcurveto{\pgfqpoint{-0.011050in}{0.041667in}}{\pgfqpoint{-0.021649in}{0.037276in}}{\pgfqpoint{-0.029463in}{0.029463in}}%
\pgfpathcurveto{\pgfqpoint{-0.037276in}{0.021649in}}{\pgfqpoint{-0.041667in}{0.011050in}}{\pgfqpoint{-0.041667in}{0.000000in}}%
\pgfpathcurveto{\pgfqpoint{-0.041667in}{-0.011050in}}{\pgfqpoint{-0.037276in}{-0.021649in}}{\pgfqpoint{-0.029463in}{-0.029463in}}%
\pgfpathcurveto{\pgfqpoint{-0.021649in}{-0.037276in}}{\pgfqpoint{-0.011050in}{-0.041667in}}{\pgfqpoint{0.000000in}{-0.041667in}}%
\pgfpathlineto{\pgfqpoint{0.000000in}{-0.041667in}}%
\pgfpathclose%
\pgfusepath{stroke,fill}%
}%
\begin{pgfscope}%
\pgfsys@transformshift{4.216165in}{0.500000in}%
\pgfsys@useobject{currentmarker}{}%
\end{pgfscope}%
\end{pgfscope}%
\begin{pgfscope}%
\pgfsetbuttcap%
\pgfsetroundjoin%
\definecolor{currentfill}{rgb}{0.000000,0.000000,0.000000}%
\pgfsetfillcolor{currentfill}%
\pgfsetlinewidth{0.803000pt}%
\definecolor{currentstroke}{rgb}{0.000000,0.000000,0.000000}%
\pgfsetstrokecolor{currentstroke}%
\pgfsetdash{}{0pt}%
\pgfsys@defobject{currentmarker}{\pgfqpoint{0.000000in}{-0.048611in}}{\pgfqpoint{0.000000in}{0.000000in}}{%
\pgfpathmoveto{\pgfqpoint{0.000000in}{0.000000in}}%
\pgfpathlineto{\pgfqpoint{0.000000in}{-0.048611in}}%
\pgfusepath{stroke,fill}%
}%
\begin{pgfscope}%
\pgfsys@transformshift{0.974595in}{0.500000in}%
\pgfsys@useobject{currentmarker}{}%
\end{pgfscope}%
\end{pgfscope}%
\begin{pgfscope}%
\definecolor{textcolor}{rgb}{0.000000,0.000000,0.000000}%
\pgfsetstrokecolor{textcolor}%
\pgfsetfillcolor{textcolor}%
\pgftext[x=0.974595in,y=0.402778in,,top]{\color{textcolor}\sffamily\fontsize{10.000000}{12.000000}\selectfont 0}%
\end{pgfscope}%
\begin{pgfscope}%
\pgfsetbuttcap%
\pgfsetroundjoin%
\definecolor{currentfill}{rgb}{0.000000,0.000000,0.000000}%
\pgfsetfillcolor{currentfill}%
\pgfsetlinewidth{0.803000pt}%
\definecolor{currentstroke}{rgb}{0.000000,0.000000,0.000000}%
\pgfsetstrokecolor{currentstroke}%
\pgfsetdash{}{0pt}%
\pgfsys@defobject{currentmarker}{\pgfqpoint{0.000000in}{-0.048611in}}{\pgfqpoint{0.000000in}{0.000000in}}{%
\pgfpathmoveto{\pgfqpoint{0.000000in}{0.000000in}}%
\pgfpathlineto{\pgfqpoint{0.000000in}{-0.048611in}}%
\pgfusepath{stroke,fill}%
}%
\begin{pgfscope}%
\pgfsys@transformshift{1.636140in}{0.500000in}%
\pgfsys@useobject{currentmarker}{}%
\end{pgfscope}%
\end{pgfscope}%
\begin{pgfscope}%
\definecolor{textcolor}{rgb}{0.000000,0.000000,0.000000}%
\pgfsetstrokecolor{textcolor}%
\pgfsetfillcolor{textcolor}%
\pgftext[x=1.636140in,y=0.402778in,,top]{\color{textcolor}\sffamily\fontsize{10.000000}{12.000000}\selectfont 20}%
\end{pgfscope}%
\begin{pgfscope}%
\pgfsetbuttcap%
\pgfsetroundjoin%
\definecolor{currentfill}{rgb}{0.000000,0.000000,0.000000}%
\pgfsetfillcolor{currentfill}%
\pgfsetlinewidth{0.803000pt}%
\definecolor{currentstroke}{rgb}{0.000000,0.000000,0.000000}%
\pgfsetstrokecolor{currentstroke}%
\pgfsetdash{}{0pt}%
\pgfsys@defobject{currentmarker}{\pgfqpoint{0.000000in}{-0.048611in}}{\pgfqpoint{0.000000in}{0.000000in}}{%
\pgfpathmoveto{\pgfqpoint{0.000000in}{0.000000in}}%
\pgfpathlineto{\pgfqpoint{0.000000in}{-0.048611in}}%
\pgfusepath{stroke,fill}%
}%
\begin{pgfscope}%
\pgfsys@transformshift{2.297685in}{0.500000in}%
\pgfsys@useobject{currentmarker}{}%
\end{pgfscope}%
\end{pgfscope}%
\begin{pgfscope}%
\definecolor{textcolor}{rgb}{0.000000,0.000000,0.000000}%
\pgfsetstrokecolor{textcolor}%
\pgfsetfillcolor{textcolor}%
\pgftext[x=2.297685in,y=0.402778in,,top]{\color{textcolor}\sffamily\fontsize{10.000000}{12.000000}\selectfont 40}%
\end{pgfscope}%
\begin{pgfscope}%
\pgfsetbuttcap%
\pgfsetroundjoin%
\definecolor{currentfill}{rgb}{0.000000,0.000000,0.000000}%
\pgfsetfillcolor{currentfill}%
\pgfsetlinewidth{0.803000pt}%
\definecolor{currentstroke}{rgb}{0.000000,0.000000,0.000000}%
\pgfsetstrokecolor{currentstroke}%
\pgfsetdash{}{0pt}%
\pgfsys@defobject{currentmarker}{\pgfqpoint{0.000000in}{-0.048611in}}{\pgfqpoint{0.000000in}{0.000000in}}{%
\pgfpathmoveto{\pgfqpoint{0.000000in}{0.000000in}}%
\pgfpathlineto{\pgfqpoint{0.000000in}{-0.048611in}}%
\pgfusepath{stroke,fill}%
}%
\begin{pgfscope}%
\pgfsys@transformshift{2.959230in}{0.500000in}%
\pgfsys@useobject{currentmarker}{}%
\end{pgfscope}%
\end{pgfscope}%
\begin{pgfscope}%
\definecolor{textcolor}{rgb}{0.000000,0.000000,0.000000}%
\pgfsetstrokecolor{textcolor}%
\pgfsetfillcolor{textcolor}%
\pgftext[x=2.959230in,y=0.402778in,,top]{\color{textcolor}\sffamily\fontsize{10.000000}{12.000000}\selectfont 60}%
\end{pgfscope}%
\begin{pgfscope}%
\pgfsetbuttcap%
\pgfsetroundjoin%
\definecolor{currentfill}{rgb}{0.000000,0.000000,0.000000}%
\pgfsetfillcolor{currentfill}%
\pgfsetlinewidth{0.803000pt}%
\definecolor{currentstroke}{rgb}{0.000000,0.000000,0.000000}%
\pgfsetstrokecolor{currentstroke}%
\pgfsetdash{}{0pt}%
\pgfsys@defobject{currentmarker}{\pgfqpoint{0.000000in}{-0.048611in}}{\pgfqpoint{0.000000in}{0.000000in}}{%
\pgfpathmoveto{\pgfqpoint{0.000000in}{0.000000in}}%
\pgfpathlineto{\pgfqpoint{0.000000in}{-0.048611in}}%
\pgfusepath{stroke,fill}%
}%
\begin{pgfscope}%
\pgfsys@transformshift{3.620775in}{0.500000in}%
\pgfsys@useobject{currentmarker}{}%
\end{pgfscope}%
\end{pgfscope}%
\begin{pgfscope}%
\definecolor{textcolor}{rgb}{0.000000,0.000000,0.000000}%
\pgfsetstrokecolor{textcolor}%
\pgfsetfillcolor{textcolor}%
\pgftext[x=3.620775in,y=0.402778in,,top]{\color{textcolor}\sffamily\fontsize{10.000000}{12.000000}\selectfont 80}%
\end{pgfscope}%
\begin{pgfscope}%
\pgfsetbuttcap%
\pgfsetroundjoin%
\definecolor{currentfill}{rgb}{0.000000,0.000000,0.000000}%
\pgfsetfillcolor{currentfill}%
\pgfsetlinewidth{0.803000pt}%
\definecolor{currentstroke}{rgb}{0.000000,0.000000,0.000000}%
\pgfsetstrokecolor{currentstroke}%
\pgfsetdash{}{0pt}%
\pgfsys@defobject{currentmarker}{\pgfqpoint{0.000000in}{-0.048611in}}{\pgfqpoint{0.000000in}{0.000000in}}{%
\pgfpathmoveto{\pgfqpoint{0.000000in}{0.000000in}}%
\pgfpathlineto{\pgfqpoint{0.000000in}{-0.048611in}}%
\pgfusepath{stroke,fill}%
}%
\begin{pgfscope}%
\pgfsys@transformshift{4.282320in}{0.500000in}%
\pgfsys@useobject{currentmarker}{}%
\end{pgfscope}%
\end{pgfscope}%
\begin{pgfscope}%
\definecolor{textcolor}{rgb}{0.000000,0.000000,0.000000}%
\pgfsetstrokecolor{textcolor}%
\pgfsetfillcolor{textcolor}%
\pgftext[x=4.282320in,y=0.402778in,,top]{\color{textcolor}\sffamily\fontsize{10.000000}{12.000000}\selectfont 100}%
\end{pgfscope}%
\begin{pgfscope}%
\pgfsetbuttcap%
\pgfsetroundjoin%
\definecolor{currentfill}{rgb}{0.000000,0.000000,0.000000}%
\pgfsetfillcolor{currentfill}%
\pgfsetlinewidth{0.803000pt}%
\definecolor{currentstroke}{rgb}{0.000000,0.000000,0.000000}%
\pgfsetstrokecolor{currentstroke}%
\pgfsetdash{}{0pt}%
\pgfsys@defobject{currentmarker}{\pgfqpoint{0.000000in}{-0.048611in}}{\pgfqpoint{0.000000in}{0.000000in}}{%
\pgfpathmoveto{\pgfqpoint{0.000000in}{0.000000in}}%
\pgfpathlineto{\pgfqpoint{0.000000in}{-0.048611in}}%
\pgfusepath{stroke,fill}%
}%
\begin{pgfscope}%
\pgfsys@transformshift{4.943865in}{0.500000in}%
\pgfsys@useobject{currentmarker}{}%
\end{pgfscope}%
\end{pgfscope}%
\begin{pgfscope}%
\definecolor{textcolor}{rgb}{0.000000,0.000000,0.000000}%
\pgfsetstrokecolor{textcolor}%
\pgfsetfillcolor{textcolor}%
\pgftext[x=4.943865in,y=0.402778in,,top]{\color{textcolor}\sffamily\fontsize{10.000000}{12.000000}\selectfont 120}%
\end{pgfscope}%
\begin{pgfscope}%
\definecolor{textcolor}{rgb}{0.000000,0.000000,0.000000}%
\pgfsetstrokecolor{textcolor}%
\pgfsetfillcolor{textcolor}%
\pgftext[x=3.075000in,y=0.212809in,,top]{\color{textcolor}\sffamily\fontsize{10.000000}{12.000000}\selectfont Number of correct bits}%
\end{pgfscope}%
\begin{pgfscope}%
\pgfsetbuttcap%
\pgfsetroundjoin%
\definecolor{currentfill}{rgb}{0.000000,0.000000,0.000000}%
\pgfsetfillcolor{currentfill}%
\pgfsetlinewidth{0.803000pt}%
\definecolor{currentstroke}{rgb}{0.000000,0.000000,0.000000}%
\pgfsetstrokecolor{currentstroke}%
\pgfsetdash{}{0pt}%
\pgfsys@defobject{currentmarker}{\pgfqpoint{-0.048611in}{0.000000in}}{\pgfqpoint{-0.000000in}{0.000000in}}{%
\pgfpathmoveto{\pgfqpoint{-0.000000in}{0.000000in}}%
\pgfpathlineto{\pgfqpoint{-0.048611in}{0.000000in}}%
\pgfusepath{stroke,fill}%
}%
\begin{pgfscope}%
\pgfsys@transformshift{0.750000in}{0.500000in}%
\pgfsys@useobject{currentmarker}{}%
\end{pgfscope}%
\end{pgfscope}%
\begin{pgfscope}%
\definecolor{textcolor}{rgb}{0.000000,0.000000,0.000000}%
\pgfsetstrokecolor{textcolor}%
\pgfsetfillcolor{textcolor}%
\pgftext[x=0.343533in, y=0.447238in, left, base]{\color{textcolor}\sffamily\fontsize{10.000000}{12.000000}\selectfont 0.00}%
\end{pgfscope}%
\begin{pgfscope}%
\pgfsetbuttcap%
\pgfsetroundjoin%
\definecolor{currentfill}{rgb}{0.000000,0.000000,0.000000}%
\pgfsetfillcolor{currentfill}%
\pgfsetlinewidth{0.803000pt}%
\definecolor{currentstroke}{rgb}{0.000000,0.000000,0.000000}%
\pgfsetstrokecolor{currentstroke}%
\pgfsetdash{}{0pt}%
\pgfsys@defobject{currentmarker}{\pgfqpoint{-0.048611in}{0.000000in}}{\pgfqpoint{-0.000000in}{0.000000in}}{%
\pgfpathmoveto{\pgfqpoint{-0.000000in}{0.000000in}}%
\pgfpathlineto{\pgfqpoint{-0.048611in}{0.000000in}}%
\pgfusepath{stroke,fill}%
}%
\begin{pgfscope}%
\pgfsys@transformshift{0.750000in}{0.908631in}%
\pgfsys@useobject{currentmarker}{}%
\end{pgfscope}%
\end{pgfscope}%
\begin{pgfscope}%
\definecolor{textcolor}{rgb}{0.000000,0.000000,0.000000}%
\pgfsetstrokecolor{textcolor}%
\pgfsetfillcolor{textcolor}%
\pgftext[x=0.343533in, y=0.855869in, left, base]{\color{textcolor}\sffamily\fontsize{10.000000}{12.000000}\selectfont 0.01}%
\end{pgfscope}%
\begin{pgfscope}%
\pgfsetbuttcap%
\pgfsetroundjoin%
\definecolor{currentfill}{rgb}{0.000000,0.000000,0.000000}%
\pgfsetfillcolor{currentfill}%
\pgfsetlinewidth{0.803000pt}%
\definecolor{currentstroke}{rgb}{0.000000,0.000000,0.000000}%
\pgfsetstrokecolor{currentstroke}%
\pgfsetdash{}{0pt}%
\pgfsys@defobject{currentmarker}{\pgfqpoint{-0.048611in}{0.000000in}}{\pgfqpoint{-0.000000in}{0.000000in}}{%
\pgfpathmoveto{\pgfqpoint{-0.000000in}{0.000000in}}%
\pgfpathlineto{\pgfqpoint{-0.048611in}{0.000000in}}%
\pgfusepath{stroke,fill}%
}%
\begin{pgfscope}%
\pgfsys@transformshift{0.750000in}{1.317261in}%
\pgfsys@useobject{currentmarker}{}%
\end{pgfscope}%
\end{pgfscope}%
\begin{pgfscope}%
\definecolor{textcolor}{rgb}{0.000000,0.000000,0.000000}%
\pgfsetstrokecolor{textcolor}%
\pgfsetfillcolor{textcolor}%
\pgftext[x=0.343533in, y=1.264499in, left, base]{\color{textcolor}\sffamily\fontsize{10.000000}{12.000000}\selectfont 0.02}%
\end{pgfscope}%
\begin{pgfscope}%
\pgfsetbuttcap%
\pgfsetroundjoin%
\definecolor{currentfill}{rgb}{0.000000,0.000000,0.000000}%
\pgfsetfillcolor{currentfill}%
\pgfsetlinewidth{0.803000pt}%
\definecolor{currentstroke}{rgb}{0.000000,0.000000,0.000000}%
\pgfsetstrokecolor{currentstroke}%
\pgfsetdash{}{0pt}%
\pgfsys@defobject{currentmarker}{\pgfqpoint{-0.048611in}{0.000000in}}{\pgfqpoint{-0.000000in}{0.000000in}}{%
\pgfpathmoveto{\pgfqpoint{-0.000000in}{0.000000in}}%
\pgfpathlineto{\pgfqpoint{-0.048611in}{0.000000in}}%
\pgfusepath{stroke,fill}%
}%
\begin{pgfscope}%
\pgfsys@transformshift{0.750000in}{1.725892in}%
\pgfsys@useobject{currentmarker}{}%
\end{pgfscope}%
\end{pgfscope}%
\begin{pgfscope}%
\definecolor{textcolor}{rgb}{0.000000,0.000000,0.000000}%
\pgfsetstrokecolor{textcolor}%
\pgfsetfillcolor{textcolor}%
\pgftext[x=0.343533in, y=1.673130in, left, base]{\color{textcolor}\sffamily\fontsize{10.000000}{12.000000}\selectfont 0.03}%
\end{pgfscope}%
\begin{pgfscope}%
\pgfsetbuttcap%
\pgfsetroundjoin%
\definecolor{currentfill}{rgb}{0.000000,0.000000,0.000000}%
\pgfsetfillcolor{currentfill}%
\pgfsetlinewidth{0.803000pt}%
\definecolor{currentstroke}{rgb}{0.000000,0.000000,0.000000}%
\pgfsetstrokecolor{currentstroke}%
\pgfsetdash{}{0pt}%
\pgfsys@defobject{currentmarker}{\pgfqpoint{-0.048611in}{0.000000in}}{\pgfqpoint{-0.000000in}{0.000000in}}{%
\pgfpathmoveto{\pgfqpoint{-0.000000in}{0.000000in}}%
\pgfpathlineto{\pgfqpoint{-0.048611in}{0.000000in}}%
\pgfusepath{stroke,fill}%
}%
\begin{pgfscope}%
\pgfsys@transformshift{0.750000in}{2.134522in}%
\pgfsys@useobject{currentmarker}{}%
\end{pgfscope}%
\end{pgfscope}%
\begin{pgfscope}%
\definecolor{textcolor}{rgb}{0.000000,0.000000,0.000000}%
\pgfsetstrokecolor{textcolor}%
\pgfsetfillcolor{textcolor}%
\pgftext[x=0.343533in, y=2.081761in, left, base]{\color{textcolor}\sffamily\fontsize{10.000000}{12.000000}\selectfont 0.04}%
\end{pgfscope}%
\begin{pgfscope}%
\pgfsetbuttcap%
\pgfsetroundjoin%
\definecolor{currentfill}{rgb}{0.000000,0.000000,0.000000}%
\pgfsetfillcolor{currentfill}%
\pgfsetlinewidth{0.803000pt}%
\definecolor{currentstroke}{rgb}{0.000000,0.000000,0.000000}%
\pgfsetstrokecolor{currentstroke}%
\pgfsetdash{}{0pt}%
\pgfsys@defobject{currentmarker}{\pgfqpoint{-0.048611in}{0.000000in}}{\pgfqpoint{-0.000000in}{0.000000in}}{%
\pgfpathmoveto{\pgfqpoint{-0.000000in}{0.000000in}}%
\pgfpathlineto{\pgfqpoint{-0.048611in}{0.000000in}}%
\pgfusepath{stroke,fill}%
}%
\begin{pgfscope}%
\pgfsys@transformshift{0.750000in}{2.543153in}%
\pgfsys@useobject{currentmarker}{}%
\end{pgfscope}%
\end{pgfscope}%
\begin{pgfscope}%
\definecolor{textcolor}{rgb}{0.000000,0.000000,0.000000}%
\pgfsetstrokecolor{textcolor}%
\pgfsetfillcolor{textcolor}%
\pgftext[x=0.343533in, y=2.490391in, left, base]{\color{textcolor}\sffamily\fontsize{10.000000}{12.000000}\selectfont 0.05}%
\end{pgfscope}%
\begin{pgfscope}%
\pgfsetbuttcap%
\pgfsetroundjoin%
\definecolor{currentfill}{rgb}{0.000000,0.000000,0.000000}%
\pgfsetfillcolor{currentfill}%
\pgfsetlinewidth{0.803000pt}%
\definecolor{currentstroke}{rgb}{0.000000,0.000000,0.000000}%
\pgfsetstrokecolor{currentstroke}%
\pgfsetdash{}{0pt}%
\pgfsys@defobject{currentmarker}{\pgfqpoint{-0.048611in}{0.000000in}}{\pgfqpoint{-0.000000in}{0.000000in}}{%
\pgfpathmoveto{\pgfqpoint{-0.000000in}{0.000000in}}%
\pgfpathlineto{\pgfqpoint{-0.048611in}{0.000000in}}%
\pgfusepath{stroke,fill}%
}%
\begin{pgfscope}%
\pgfsys@transformshift{0.750000in}{2.951783in}%
\pgfsys@useobject{currentmarker}{}%
\end{pgfscope}%
\end{pgfscope}%
\begin{pgfscope}%
\definecolor{textcolor}{rgb}{0.000000,0.000000,0.000000}%
\pgfsetstrokecolor{textcolor}%
\pgfsetfillcolor{textcolor}%
\pgftext[x=0.343533in, y=2.899022in, left, base]{\color{textcolor}\sffamily\fontsize{10.000000}{12.000000}\selectfont 0.06}%
\end{pgfscope}%
\begin{pgfscope}%
\pgfsetbuttcap%
\pgfsetroundjoin%
\definecolor{currentfill}{rgb}{0.000000,0.000000,0.000000}%
\pgfsetfillcolor{currentfill}%
\pgfsetlinewidth{0.803000pt}%
\definecolor{currentstroke}{rgb}{0.000000,0.000000,0.000000}%
\pgfsetstrokecolor{currentstroke}%
\pgfsetdash{}{0pt}%
\pgfsys@defobject{currentmarker}{\pgfqpoint{-0.048611in}{0.000000in}}{\pgfqpoint{-0.000000in}{0.000000in}}{%
\pgfpathmoveto{\pgfqpoint{-0.000000in}{0.000000in}}%
\pgfpathlineto{\pgfqpoint{-0.048611in}{0.000000in}}%
\pgfusepath{stroke,fill}%
}%
\begin{pgfscope}%
\pgfsys@transformshift{0.750000in}{3.360414in}%
\pgfsys@useobject{currentmarker}{}%
\end{pgfscope}%
\end{pgfscope}%
\begin{pgfscope}%
\definecolor{textcolor}{rgb}{0.000000,0.000000,0.000000}%
\pgfsetstrokecolor{textcolor}%
\pgfsetfillcolor{textcolor}%
\pgftext[x=0.343533in, y=3.307652in, left, base]{\color{textcolor}\sffamily\fontsize{10.000000}{12.000000}\selectfont 0.07}%
\end{pgfscope}%
\begin{pgfscope}%
\definecolor{textcolor}{rgb}{0.000000,0.000000,0.000000}%
\pgfsetstrokecolor{textcolor}%
\pgfsetfillcolor{textcolor}%
\pgftext[x=0.287977in,y=2.010000in,,bottom,rotate=90.000000]{\color{textcolor}\sffamily\fontsize{10.000000}{12.000000}\selectfont Probability}%
\end{pgfscope}%
\begin{pgfscope}%
\pgfsetrectcap%
\pgfsetmiterjoin%
\pgfsetlinewidth{0.803000pt}%
\definecolor{currentstroke}{rgb}{0.000000,0.000000,0.000000}%
\pgfsetstrokecolor{currentstroke}%
\pgfsetdash{}{0pt}%
\pgfpathmoveto{\pgfqpoint{0.750000in}{0.500000in}}%
\pgfpathlineto{\pgfqpoint{0.750000in}{3.520000in}}%
\pgfusepath{stroke}%
\end{pgfscope}%
\begin{pgfscope}%
\pgfsetrectcap%
\pgfsetmiterjoin%
\pgfsetlinewidth{0.803000pt}%
\definecolor{currentstroke}{rgb}{0.000000,0.000000,0.000000}%
\pgfsetstrokecolor{currentstroke}%
\pgfsetdash{}{0pt}%
\pgfpathmoveto{\pgfqpoint{5.400000in}{0.500000in}}%
\pgfpathlineto{\pgfqpoint{5.400000in}{3.520000in}}%
\pgfusepath{stroke}%
\end{pgfscope}%
\begin{pgfscope}%
\pgfsetrectcap%
\pgfsetmiterjoin%
\pgfsetlinewidth{0.803000pt}%
\definecolor{currentstroke}{rgb}{0.000000,0.000000,0.000000}%
\pgfsetstrokecolor{currentstroke}%
\pgfsetdash{}{0pt}%
\pgfpathmoveto{\pgfqpoint{0.750000in}{0.500000in}}%
\pgfpathlineto{\pgfqpoint{5.400000in}{0.500000in}}%
\pgfusepath{stroke}%
\end{pgfscope}%
\begin{pgfscope}%
\pgfsetrectcap%
\pgfsetmiterjoin%
\pgfsetlinewidth{0.803000pt}%
\definecolor{currentstroke}{rgb}{0.000000,0.000000,0.000000}%
\pgfsetstrokecolor{currentstroke}%
\pgfsetdash{}{0pt}%
\pgfpathmoveto{\pgfqpoint{0.750000in}{3.520000in}}%
\pgfpathlineto{\pgfqpoint{5.400000in}{3.520000in}}%
\pgfusepath{stroke}%
\end{pgfscope}%
\begin{pgfscope}%
\definecolor{textcolor}{rgb}{0.000000,0.000000,0.000000}%
\pgfsetstrokecolor{textcolor}%
\pgfsetfillcolor{textcolor}%
\pgftext[x=3.075000in,y=3.603333in,,base]{\color{textcolor}\sffamily\fontsize{12.000000}{14.400000}\selectfont PDF for random bits}%
\end{pgfscope}%
\begin{pgfscope}%
\pgfsetbuttcap%
\pgfsetmiterjoin%
\definecolor{currentfill}{rgb}{1.000000,1.000000,1.000000}%
\pgfsetfillcolor{currentfill}%
\pgfsetfillopacity{0.800000}%
\pgfsetlinewidth{1.003750pt}%
\definecolor{currentstroke}{rgb}{0.800000,0.800000,0.800000}%
\pgfsetstrokecolor{currentstroke}%
\pgfsetstrokeopacity{0.800000}%
\pgfsetdash{}{0pt}%
\pgfpathmoveto{\pgfqpoint{3.456831in}{2.797317in}}%
\pgfpathlineto{\pgfqpoint{5.302778in}{2.797317in}}%
\pgfpathquadraticcurveto{\pgfqpoint{5.330556in}{2.797317in}}{\pgfqpoint{5.330556in}{2.825095in}}%
\pgfpathlineto{\pgfqpoint{5.330556in}{3.422778in}}%
\pgfpathquadraticcurveto{\pgfqpoint{5.330556in}{3.450556in}}{\pgfqpoint{5.302778in}{3.450556in}}%
\pgfpathlineto{\pgfqpoint{3.456831in}{3.450556in}}%
\pgfpathquadraticcurveto{\pgfqpoint{3.429053in}{3.450556in}}{\pgfqpoint{3.429053in}{3.422778in}}%
\pgfpathlineto{\pgfqpoint{3.429053in}{2.825095in}}%
\pgfpathquadraticcurveto{\pgfqpoint{3.429053in}{2.797317in}}{\pgfqpoint{3.456831in}{2.797317in}}%
\pgfpathlineto{\pgfqpoint{3.456831in}{2.797317in}}%
\pgfpathclose%
\pgfusepath{stroke,fill}%
\end{pgfscope}%
\begin{pgfscope}%
\pgfsetbuttcap%
\pgfsetroundjoin%
\definecolor{currentfill}{rgb}{1.000000,0.000000,0.000000}%
\pgfsetfillcolor{currentfill}%
\pgfsetlinewidth{1.003750pt}%
\definecolor{currentstroke}{rgb}{1.000000,0.000000,0.000000}%
\pgfsetstrokecolor{currentstroke}%
\pgfsetdash{}{0pt}%
\pgfsys@defobject{currentmarker}{\pgfqpoint{-0.041667in}{-0.041667in}}{\pgfqpoint{0.041667in}{0.041667in}}{%
\pgfpathmoveto{\pgfqpoint{0.000000in}{-0.041667in}}%
\pgfpathcurveto{\pgfqpoint{0.011050in}{-0.041667in}}{\pgfqpoint{0.021649in}{-0.037276in}}{\pgfqpoint{0.029463in}{-0.029463in}}%
\pgfpathcurveto{\pgfqpoint{0.037276in}{-0.021649in}}{\pgfqpoint{0.041667in}{-0.011050in}}{\pgfqpoint{0.041667in}{0.000000in}}%
\pgfpathcurveto{\pgfqpoint{0.041667in}{0.011050in}}{\pgfqpoint{0.037276in}{0.021649in}}{\pgfqpoint{0.029463in}{0.029463in}}%
\pgfpathcurveto{\pgfqpoint{0.021649in}{0.037276in}}{\pgfqpoint{0.011050in}{0.041667in}}{\pgfqpoint{0.000000in}{0.041667in}}%
\pgfpathcurveto{\pgfqpoint{-0.011050in}{0.041667in}}{\pgfqpoint{-0.021649in}{0.037276in}}{\pgfqpoint{-0.029463in}{0.029463in}}%
\pgfpathcurveto{\pgfqpoint{-0.037276in}{0.021649in}}{\pgfqpoint{-0.041667in}{0.011050in}}{\pgfqpoint{-0.041667in}{0.000000in}}%
\pgfpathcurveto{\pgfqpoint{-0.041667in}{-0.011050in}}{\pgfqpoint{-0.037276in}{-0.021649in}}{\pgfqpoint{-0.029463in}{-0.029463in}}%
\pgfpathcurveto{\pgfqpoint{-0.021649in}{-0.037276in}}{\pgfqpoint{-0.011050in}{-0.041667in}}{\pgfqpoint{0.000000in}{-0.041667in}}%
\pgfpathlineto{\pgfqpoint{0.000000in}{-0.041667in}}%
\pgfpathclose%
\pgfusepath{stroke,fill}%
}%
\begin{pgfscope}%
\pgfsys@transformshift{3.623497in}{3.325935in}%
\pgfsys@useobject{currentmarker}{}%
\end{pgfscope}%
\end{pgfscope}%
\begin{pgfscope}%
\definecolor{textcolor}{rgb}{0.000000,0.000000,0.000000}%
\pgfsetstrokecolor{textcolor}%
\pgfsetfillcolor{textcolor}%
\pgftext[x=3.873497in,y=3.289477in,left,base]{\color{textcolor}\sffamily\fontsize{10.000000}{12.000000}\selectfont SNN}%
\end{pgfscope}%
\begin{pgfscope}%
\pgfsetbuttcap%
\pgfsetroundjoin%
\definecolor{currentfill}{rgb}{0.000000,0.500000,0.000000}%
\pgfsetfillcolor{currentfill}%
\pgfsetlinewidth{1.003750pt}%
\definecolor{currentstroke}{rgb}{0.000000,0.500000,0.000000}%
\pgfsetstrokecolor{currentstroke}%
\pgfsetdash{}{0pt}%
\pgfsys@defobject{currentmarker}{\pgfqpoint{-0.041667in}{-0.041667in}}{\pgfqpoint{0.041667in}{0.041667in}}{%
\pgfpathmoveto{\pgfqpoint{0.000000in}{-0.041667in}}%
\pgfpathcurveto{\pgfqpoint{0.011050in}{-0.041667in}}{\pgfqpoint{0.021649in}{-0.037276in}}{\pgfqpoint{0.029463in}{-0.029463in}}%
\pgfpathcurveto{\pgfqpoint{0.037276in}{-0.021649in}}{\pgfqpoint{0.041667in}{-0.011050in}}{\pgfqpoint{0.041667in}{0.000000in}}%
\pgfpathcurveto{\pgfqpoint{0.041667in}{0.011050in}}{\pgfqpoint{0.037276in}{0.021649in}}{\pgfqpoint{0.029463in}{0.029463in}}%
\pgfpathcurveto{\pgfqpoint{0.021649in}{0.037276in}}{\pgfqpoint{0.011050in}{0.041667in}}{\pgfqpoint{0.000000in}{0.041667in}}%
\pgfpathcurveto{\pgfqpoint{-0.011050in}{0.041667in}}{\pgfqpoint{-0.021649in}{0.037276in}}{\pgfqpoint{-0.029463in}{0.029463in}}%
\pgfpathcurveto{\pgfqpoint{-0.037276in}{0.021649in}}{\pgfqpoint{-0.041667in}{0.011050in}}{\pgfqpoint{-0.041667in}{0.000000in}}%
\pgfpathcurveto{\pgfqpoint{-0.041667in}{-0.011050in}}{\pgfqpoint{-0.037276in}{-0.021649in}}{\pgfqpoint{-0.029463in}{-0.029463in}}%
\pgfpathcurveto{\pgfqpoint{-0.021649in}{-0.037276in}}{\pgfqpoint{-0.011050in}{-0.041667in}}{\pgfqpoint{0.000000in}{-0.041667in}}%
\pgfpathlineto{\pgfqpoint{0.000000in}{-0.041667in}}%
\pgfpathclose%
\pgfusepath{stroke,fill}%
}%
\begin{pgfscope}%
\pgfsys@transformshift{3.623497in}{3.122078in}%
\pgfsys@useobject{currentmarker}{}%
\end{pgfscope}%
\end{pgfscope}%
\begin{pgfscope}%
\definecolor{textcolor}{rgb}{0.000000,0.000000,0.000000}%
\pgfsetstrokecolor{textcolor}%
\pgfsetfillcolor{textcolor}%
\pgftext[x=3.873497in,y=3.085620in,left,base]{\color{textcolor}\sffamily\fontsize{10.000000}{12.000000}\selectfont NN}%
\end{pgfscope}%
\begin{pgfscope}%
\pgfsetbuttcap%
\pgfsetmiterjoin%
\definecolor{currentfill}{rgb}{0.121569,0.466667,0.705882}%
\pgfsetfillcolor{currentfill}%
\pgfsetlinewidth{0.000000pt}%
\definecolor{currentstroke}{rgb}{0.000000,0.000000,0.000000}%
\pgfsetstrokecolor{currentstroke}%
\pgfsetstrokeopacity{0.000000}%
\pgfsetdash{}{0pt}%
\pgfpathmoveto{\pgfqpoint{3.484608in}{2.881762in}}%
\pgfpathlineto{\pgfqpoint{3.762386in}{2.881762in}}%
\pgfpathlineto{\pgfqpoint{3.762386in}{2.978985in}}%
\pgfpathlineto{\pgfqpoint{3.484608in}{2.978985in}}%
\pgfpathlineto{\pgfqpoint{3.484608in}{2.881762in}}%
\pgfpathclose%
\pgfusepath{fill}%
\end{pgfscope}%
\begin{pgfscope}%
\definecolor{textcolor}{rgb}{0.000000,0.000000,0.000000}%
\pgfsetstrokecolor{textcolor}%
\pgfsetfillcolor{textcolor}%
\pgftext[x=3.873497in,y=2.881762in,left,base]{\color{textcolor}\sffamily\fontsize{10.000000}{12.000000}\selectfont Binomial distributon}%
\end{pgfscope}%
\end{pgfpicture}%
\makeatother%
\endgroup%

    \caption{Caption}
    \label{fig:my_label}
\end{figure}

\begin{figure}
    \centering
    %% Creator: Matplotlib, PGF backend
%%
%% To include the figure in your LaTeX document, write
%%   \input{<filename>.pgf}
%%
%% Make sure the required packages are loaded in your preamble
%%   \usepackage{pgf}
%%
%% Also ensure that all the required font packages are loaded; for instance,
%% the lmodern package is sometimes necessary when using math font.
%%   \usepackage{lmodern}
%%
%% Figures using additional raster images can only be included by \input if
%% they are in the same directory as the main LaTeX file. For loading figures
%% from other directories you can use the `import` package
%%   \usepackage{import}
%%
%% and then include the figures with
%%   \import{<path to file>}{<filename>.pgf}
%%
%% Matplotlib used the following preamble
%%   \usepackage{fontspec}
%%   \setmainfont{DejaVuSerif.ttf}[Path=\detokenize{/Users/mkojro/miniforge3/envs/nn-crypto/lib/python3.10/site-packages/matplotlib/mpl-data/fonts/ttf/}]
%%   \setsansfont{DejaVuSans.ttf}[Path=\detokenize{/Users/mkojro/miniforge3/envs/nn-crypto/lib/python3.10/site-packages/matplotlib/mpl-data/fonts/ttf/}]
%%   \setmonofont{DejaVuSansMono.ttf}[Path=\detokenize{/Users/mkojro/miniforge3/envs/nn-crypto/lib/python3.10/site-packages/matplotlib/mpl-data/fonts/ttf/}]
%%
\begingroup%
\makeatletter%
\begin{pgfpicture}%
\pgfpathrectangle{\pgfpointorigin}{\pgfqpoint{6.000000in}{4.000000in}}%
\pgfusepath{use as bounding box, clip}%
\begin{pgfscope}%
\pgfsetbuttcap%
\pgfsetmiterjoin%
\pgfsetlinewidth{0.000000pt}%
\definecolor{currentstroke}{rgb}{1.000000,1.000000,1.000000}%
\pgfsetstrokecolor{currentstroke}%
\pgfsetstrokeopacity{0.000000}%
\pgfsetdash{}{0pt}%
\pgfpathmoveto{\pgfqpoint{0.000000in}{0.000000in}}%
\pgfpathlineto{\pgfqpoint{6.000000in}{0.000000in}}%
\pgfpathlineto{\pgfqpoint{6.000000in}{4.000000in}}%
\pgfpathlineto{\pgfqpoint{0.000000in}{4.000000in}}%
\pgfpathlineto{\pgfqpoint{0.000000in}{0.000000in}}%
\pgfpathclose%
\pgfusepath{}%
\end{pgfscope}%
\begin{pgfscope}%
\pgfsetbuttcap%
\pgfsetmiterjoin%
\definecolor{currentfill}{rgb}{1.000000,1.000000,1.000000}%
\pgfsetfillcolor{currentfill}%
\pgfsetlinewidth{0.000000pt}%
\definecolor{currentstroke}{rgb}{0.000000,0.000000,0.000000}%
\pgfsetstrokecolor{currentstroke}%
\pgfsetstrokeopacity{0.000000}%
\pgfsetdash{}{0pt}%
\pgfpathmoveto{\pgfqpoint{0.750000in}{0.500000in}}%
\pgfpathlineto{\pgfqpoint{5.400000in}{0.500000in}}%
\pgfpathlineto{\pgfqpoint{5.400000in}{3.520000in}}%
\pgfpathlineto{\pgfqpoint{0.750000in}{3.520000in}}%
\pgfpathlineto{\pgfqpoint{0.750000in}{0.500000in}}%
\pgfpathclose%
\pgfusepath{fill}%
\end{pgfscope}%
\begin{pgfscope}%
\pgfpathrectangle{\pgfqpoint{0.750000in}{0.500000in}}{\pgfqpoint{4.650000in}{3.020000in}}%
\pgfusepath{clip}%
\pgfsetbuttcap%
\pgfsetmiterjoin%
\definecolor{currentfill}{rgb}{0.121569,0.466667,0.705882}%
\pgfsetfillcolor{currentfill}%
\pgfsetlinewidth{0.000000pt}%
\definecolor{currentstroke}{rgb}{0.000000,0.000000,0.000000}%
\pgfsetstrokecolor{currentstroke}%
\pgfsetstrokeopacity{0.000000}%
\pgfsetdash{}{0pt}%
\pgfpathmoveto{\pgfqpoint{0.961364in}{0.500000in}}%
\pgfpathlineto{\pgfqpoint{0.987825in}{0.500000in}}%
\pgfpathlineto{\pgfqpoint{0.987825in}{0.500000in}}%
\pgfpathlineto{\pgfqpoint{0.961364in}{0.500000in}}%
\pgfpathlineto{\pgfqpoint{0.961364in}{0.500000in}}%
\pgfpathclose%
\pgfusepath{fill}%
\end{pgfscope}%
\begin{pgfscope}%
\pgfpathrectangle{\pgfqpoint{0.750000in}{0.500000in}}{\pgfqpoint{4.650000in}{3.020000in}}%
\pgfusepath{clip}%
\pgfsetbuttcap%
\pgfsetmiterjoin%
\definecolor{currentfill}{rgb}{0.121569,0.466667,0.705882}%
\pgfsetfillcolor{currentfill}%
\pgfsetlinewidth{0.000000pt}%
\definecolor{currentstroke}{rgb}{0.000000,0.000000,0.000000}%
\pgfsetstrokecolor{currentstroke}%
\pgfsetstrokeopacity{0.000000}%
\pgfsetdash{}{0pt}%
\pgfpathmoveto{\pgfqpoint{0.994441in}{0.500000in}}%
\pgfpathlineto{\pgfqpoint{1.020903in}{0.500000in}}%
\pgfpathlineto{\pgfqpoint{1.020903in}{0.500000in}}%
\pgfpathlineto{\pgfqpoint{0.994441in}{0.500000in}}%
\pgfpathlineto{\pgfqpoint{0.994441in}{0.500000in}}%
\pgfpathclose%
\pgfusepath{fill}%
\end{pgfscope}%
\begin{pgfscope}%
\pgfpathrectangle{\pgfqpoint{0.750000in}{0.500000in}}{\pgfqpoint{4.650000in}{3.020000in}}%
\pgfusepath{clip}%
\pgfsetbuttcap%
\pgfsetmiterjoin%
\definecolor{currentfill}{rgb}{0.121569,0.466667,0.705882}%
\pgfsetfillcolor{currentfill}%
\pgfsetlinewidth{0.000000pt}%
\definecolor{currentstroke}{rgb}{0.000000,0.000000,0.000000}%
\pgfsetstrokecolor{currentstroke}%
\pgfsetstrokeopacity{0.000000}%
\pgfsetdash{}{0pt}%
\pgfpathmoveto{\pgfqpoint{1.027518in}{0.500000in}}%
\pgfpathlineto{\pgfqpoint{1.053980in}{0.500000in}}%
\pgfpathlineto{\pgfqpoint{1.053980in}{0.500000in}}%
\pgfpathlineto{\pgfqpoint{1.027518in}{0.500000in}}%
\pgfpathlineto{\pgfqpoint{1.027518in}{0.500000in}}%
\pgfpathclose%
\pgfusepath{fill}%
\end{pgfscope}%
\begin{pgfscope}%
\pgfpathrectangle{\pgfqpoint{0.750000in}{0.500000in}}{\pgfqpoint{4.650000in}{3.020000in}}%
\pgfusepath{clip}%
\pgfsetbuttcap%
\pgfsetmiterjoin%
\definecolor{currentfill}{rgb}{0.121569,0.466667,0.705882}%
\pgfsetfillcolor{currentfill}%
\pgfsetlinewidth{0.000000pt}%
\definecolor{currentstroke}{rgb}{0.000000,0.000000,0.000000}%
\pgfsetstrokecolor{currentstroke}%
\pgfsetstrokeopacity{0.000000}%
\pgfsetdash{}{0pt}%
\pgfpathmoveto{\pgfqpoint{1.060595in}{0.500000in}}%
\pgfpathlineto{\pgfqpoint{1.087057in}{0.500000in}}%
\pgfpathlineto{\pgfqpoint{1.087057in}{0.500000in}}%
\pgfpathlineto{\pgfqpoint{1.060595in}{0.500000in}}%
\pgfpathlineto{\pgfqpoint{1.060595in}{0.500000in}}%
\pgfpathclose%
\pgfusepath{fill}%
\end{pgfscope}%
\begin{pgfscope}%
\pgfpathrectangle{\pgfqpoint{0.750000in}{0.500000in}}{\pgfqpoint{4.650000in}{3.020000in}}%
\pgfusepath{clip}%
\pgfsetbuttcap%
\pgfsetmiterjoin%
\definecolor{currentfill}{rgb}{0.121569,0.466667,0.705882}%
\pgfsetfillcolor{currentfill}%
\pgfsetlinewidth{0.000000pt}%
\definecolor{currentstroke}{rgb}{0.000000,0.000000,0.000000}%
\pgfsetstrokecolor{currentstroke}%
\pgfsetstrokeopacity{0.000000}%
\pgfsetdash{}{0pt}%
\pgfpathmoveto{\pgfqpoint{1.093673in}{0.500000in}}%
\pgfpathlineto{\pgfqpoint{1.120134in}{0.500000in}}%
\pgfpathlineto{\pgfqpoint{1.120134in}{0.500000in}}%
\pgfpathlineto{\pgfqpoint{1.093673in}{0.500000in}}%
\pgfpathlineto{\pgfqpoint{1.093673in}{0.500000in}}%
\pgfpathclose%
\pgfusepath{fill}%
\end{pgfscope}%
\begin{pgfscope}%
\pgfpathrectangle{\pgfqpoint{0.750000in}{0.500000in}}{\pgfqpoint{4.650000in}{3.020000in}}%
\pgfusepath{clip}%
\pgfsetbuttcap%
\pgfsetmiterjoin%
\definecolor{currentfill}{rgb}{0.121569,0.466667,0.705882}%
\pgfsetfillcolor{currentfill}%
\pgfsetlinewidth{0.000000pt}%
\definecolor{currentstroke}{rgb}{0.000000,0.000000,0.000000}%
\pgfsetstrokecolor{currentstroke}%
\pgfsetstrokeopacity{0.000000}%
\pgfsetdash{}{0pt}%
\pgfpathmoveto{\pgfqpoint{1.126750in}{0.500000in}}%
\pgfpathlineto{\pgfqpoint{1.153212in}{0.500000in}}%
\pgfpathlineto{\pgfqpoint{1.153212in}{0.500000in}}%
\pgfpathlineto{\pgfqpoint{1.126750in}{0.500000in}}%
\pgfpathlineto{\pgfqpoint{1.126750in}{0.500000in}}%
\pgfpathclose%
\pgfusepath{fill}%
\end{pgfscope}%
\begin{pgfscope}%
\pgfpathrectangle{\pgfqpoint{0.750000in}{0.500000in}}{\pgfqpoint{4.650000in}{3.020000in}}%
\pgfusepath{clip}%
\pgfsetbuttcap%
\pgfsetmiterjoin%
\definecolor{currentfill}{rgb}{0.121569,0.466667,0.705882}%
\pgfsetfillcolor{currentfill}%
\pgfsetlinewidth{0.000000pt}%
\definecolor{currentstroke}{rgb}{0.000000,0.000000,0.000000}%
\pgfsetstrokecolor{currentstroke}%
\pgfsetstrokeopacity{0.000000}%
\pgfsetdash{}{0pt}%
\pgfpathmoveto{\pgfqpoint{1.159827in}{0.500000in}}%
\pgfpathlineto{\pgfqpoint{1.186289in}{0.500000in}}%
\pgfpathlineto{\pgfqpoint{1.186289in}{0.500000in}}%
\pgfpathlineto{\pgfqpoint{1.159827in}{0.500000in}}%
\pgfpathlineto{\pgfqpoint{1.159827in}{0.500000in}}%
\pgfpathclose%
\pgfusepath{fill}%
\end{pgfscope}%
\begin{pgfscope}%
\pgfpathrectangle{\pgfqpoint{0.750000in}{0.500000in}}{\pgfqpoint{4.650000in}{3.020000in}}%
\pgfusepath{clip}%
\pgfsetbuttcap%
\pgfsetmiterjoin%
\definecolor{currentfill}{rgb}{0.121569,0.466667,0.705882}%
\pgfsetfillcolor{currentfill}%
\pgfsetlinewidth{0.000000pt}%
\definecolor{currentstroke}{rgb}{0.000000,0.000000,0.000000}%
\pgfsetstrokecolor{currentstroke}%
\pgfsetstrokeopacity{0.000000}%
\pgfsetdash{}{0pt}%
\pgfpathmoveto{\pgfqpoint{1.192904in}{0.500000in}}%
\pgfpathlineto{\pgfqpoint{1.219366in}{0.500000in}}%
\pgfpathlineto{\pgfqpoint{1.219366in}{0.500000in}}%
\pgfpathlineto{\pgfqpoint{1.192904in}{0.500000in}}%
\pgfpathlineto{\pgfqpoint{1.192904in}{0.500000in}}%
\pgfpathclose%
\pgfusepath{fill}%
\end{pgfscope}%
\begin{pgfscope}%
\pgfpathrectangle{\pgfqpoint{0.750000in}{0.500000in}}{\pgfqpoint{4.650000in}{3.020000in}}%
\pgfusepath{clip}%
\pgfsetbuttcap%
\pgfsetmiterjoin%
\definecolor{currentfill}{rgb}{0.121569,0.466667,0.705882}%
\pgfsetfillcolor{currentfill}%
\pgfsetlinewidth{0.000000pt}%
\definecolor{currentstroke}{rgb}{0.000000,0.000000,0.000000}%
\pgfsetstrokecolor{currentstroke}%
\pgfsetstrokeopacity{0.000000}%
\pgfsetdash{}{0pt}%
\pgfpathmoveto{\pgfqpoint{1.225982in}{0.500000in}}%
\pgfpathlineto{\pgfqpoint{1.252443in}{0.500000in}}%
\pgfpathlineto{\pgfqpoint{1.252443in}{0.500000in}}%
\pgfpathlineto{\pgfqpoint{1.225982in}{0.500000in}}%
\pgfpathlineto{\pgfqpoint{1.225982in}{0.500000in}}%
\pgfpathclose%
\pgfusepath{fill}%
\end{pgfscope}%
\begin{pgfscope}%
\pgfpathrectangle{\pgfqpoint{0.750000in}{0.500000in}}{\pgfqpoint{4.650000in}{3.020000in}}%
\pgfusepath{clip}%
\pgfsetbuttcap%
\pgfsetmiterjoin%
\definecolor{currentfill}{rgb}{0.121569,0.466667,0.705882}%
\pgfsetfillcolor{currentfill}%
\pgfsetlinewidth{0.000000pt}%
\definecolor{currentstroke}{rgb}{0.000000,0.000000,0.000000}%
\pgfsetstrokecolor{currentstroke}%
\pgfsetstrokeopacity{0.000000}%
\pgfsetdash{}{0pt}%
\pgfpathmoveto{\pgfqpoint{1.259059in}{0.500000in}}%
\pgfpathlineto{\pgfqpoint{1.285521in}{0.500000in}}%
\pgfpathlineto{\pgfqpoint{1.285521in}{0.500000in}}%
\pgfpathlineto{\pgfqpoint{1.259059in}{0.500000in}}%
\pgfpathlineto{\pgfqpoint{1.259059in}{0.500000in}}%
\pgfpathclose%
\pgfusepath{fill}%
\end{pgfscope}%
\begin{pgfscope}%
\pgfpathrectangle{\pgfqpoint{0.750000in}{0.500000in}}{\pgfqpoint{4.650000in}{3.020000in}}%
\pgfusepath{clip}%
\pgfsetbuttcap%
\pgfsetmiterjoin%
\definecolor{currentfill}{rgb}{0.121569,0.466667,0.705882}%
\pgfsetfillcolor{currentfill}%
\pgfsetlinewidth{0.000000pt}%
\definecolor{currentstroke}{rgb}{0.000000,0.000000,0.000000}%
\pgfsetstrokecolor{currentstroke}%
\pgfsetstrokeopacity{0.000000}%
\pgfsetdash{}{0pt}%
\pgfpathmoveto{\pgfqpoint{1.292136in}{0.500000in}}%
\pgfpathlineto{\pgfqpoint{1.318598in}{0.500000in}}%
\pgfpathlineto{\pgfqpoint{1.318598in}{0.500000in}}%
\pgfpathlineto{\pgfqpoint{1.292136in}{0.500000in}}%
\pgfpathlineto{\pgfqpoint{1.292136in}{0.500000in}}%
\pgfpathclose%
\pgfusepath{fill}%
\end{pgfscope}%
\begin{pgfscope}%
\pgfpathrectangle{\pgfqpoint{0.750000in}{0.500000in}}{\pgfqpoint{4.650000in}{3.020000in}}%
\pgfusepath{clip}%
\pgfsetbuttcap%
\pgfsetmiterjoin%
\definecolor{currentfill}{rgb}{0.121569,0.466667,0.705882}%
\pgfsetfillcolor{currentfill}%
\pgfsetlinewidth{0.000000pt}%
\definecolor{currentstroke}{rgb}{0.000000,0.000000,0.000000}%
\pgfsetstrokecolor{currentstroke}%
\pgfsetstrokeopacity{0.000000}%
\pgfsetdash{}{0pt}%
\pgfpathmoveto{\pgfqpoint{1.325213in}{0.500000in}}%
\pgfpathlineto{\pgfqpoint{1.351675in}{0.500000in}}%
\pgfpathlineto{\pgfqpoint{1.351675in}{0.500000in}}%
\pgfpathlineto{\pgfqpoint{1.325213in}{0.500000in}}%
\pgfpathlineto{\pgfqpoint{1.325213in}{0.500000in}}%
\pgfpathclose%
\pgfusepath{fill}%
\end{pgfscope}%
\begin{pgfscope}%
\pgfpathrectangle{\pgfqpoint{0.750000in}{0.500000in}}{\pgfqpoint{4.650000in}{3.020000in}}%
\pgfusepath{clip}%
\pgfsetbuttcap%
\pgfsetmiterjoin%
\definecolor{currentfill}{rgb}{0.121569,0.466667,0.705882}%
\pgfsetfillcolor{currentfill}%
\pgfsetlinewidth{0.000000pt}%
\definecolor{currentstroke}{rgb}{0.000000,0.000000,0.000000}%
\pgfsetstrokecolor{currentstroke}%
\pgfsetstrokeopacity{0.000000}%
\pgfsetdash{}{0pt}%
\pgfpathmoveto{\pgfqpoint{1.358291in}{0.500000in}}%
\pgfpathlineto{\pgfqpoint{1.384752in}{0.500000in}}%
\pgfpathlineto{\pgfqpoint{1.384752in}{0.500000in}}%
\pgfpathlineto{\pgfqpoint{1.358291in}{0.500000in}}%
\pgfpathlineto{\pgfqpoint{1.358291in}{0.500000in}}%
\pgfpathclose%
\pgfusepath{fill}%
\end{pgfscope}%
\begin{pgfscope}%
\pgfpathrectangle{\pgfqpoint{0.750000in}{0.500000in}}{\pgfqpoint{4.650000in}{3.020000in}}%
\pgfusepath{clip}%
\pgfsetbuttcap%
\pgfsetmiterjoin%
\definecolor{currentfill}{rgb}{0.121569,0.466667,0.705882}%
\pgfsetfillcolor{currentfill}%
\pgfsetlinewidth{0.000000pt}%
\definecolor{currentstroke}{rgb}{0.000000,0.000000,0.000000}%
\pgfsetstrokecolor{currentstroke}%
\pgfsetstrokeopacity{0.000000}%
\pgfsetdash{}{0pt}%
\pgfpathmoveto{\pgfqpoint{1.391368in}{0.500000in}}%
\pgfpathlineto{\pgfqpoint{1.417830in}{0.500000in}}%
\pgfpathlineto{\pgfqpoint{1.417830in}{0.500000in}}%
\pgfpathlineto{\pgfqpoint{1.391368in}{0.500000in}}%
\pgfpathlineto{\pgfqpoint{1.391368in}{0.500000in}}%
\pgfpathclose%
\pgfusepath{fill}%
\end{pgfscope}%
\begin{pgfscope}%
\pgfpathrectangle{\pgfqpoint{0.750000in}{0.500000in}}{\pgfqpoint{4.650000in}{3.020000in}}%
\pgfusepath{clip}%
\pgfsetbuttcap%
\pgfsetmiterjoin%
\definecolor{currentfill}{rgb}{0.121569,0.466667,0.705882}%
\pgfsetfillcolor{currentfill}%
\pgfsetlinewidth{0.000000pt}%
\definecolor{currentstroke}{rgb}{0.000000,0.000000,0.000000}%
\pgfsetstrokecolor{currentstroke}%
\pgfsetstrokeopacity{0.000000}%
\pgfsetdash{}{0pt}%
\pgfpathmoveto{\pgfqpoint{1.424445in}{0.500000in}}%
\pgfpathlineto{\pgfqpoint{1.450907in}{0.500000in}}%
\pgfpathlineto{\pgfqpoint{1.450907in}{0.500000in}}%
\pgfpathlineto{\pgfqpoint{1.424445in}{0.500000in}}%
\pgfpathlineto{\pgfqpoint{1.424445in}{0.500000in}}%
\pgfpathclose%
\pgfusepath{fill}%
\end{pgfscope}%
\begin{pgfscope}%
\pgfpathrectangle{\pgfqpoint{0.750000in}{0.500000in}}{\pgfqpoint{4.650000in}{3.020000in}}%
\pgfusepath{clip}%
\pgfsetbuttcap%
\pgfsetmiterjoin%
\definecolor{currentfill}{rgb}{0.121569,0.466667,0.705882}%
\pgfsetfillcolor{currentfill}%
\pgfsetlinewidth{0.000000pt}%
\definecolor{currentstroke}{rgb}{0.000000,0.000000,0.000000}%
\pgfsetstrokecolor{currentstroke}%
\pgfsetstrokeopacity{0.000000}%
\pgfsetdash{}{0pt}%
\pgfpathmoveto{\pgfqpoint{1.457522in}{0.500000in}}%
\pgfpathlineto{\pgfqpoint{1.483984in}{0.500000in}}%
\pgfpathlineto{\pgfqpoint{1.483984in}{0.500000in}}%
\pgfpathlineto{\pgfqpoint{1.457522in}{0.500000in}}%
\pgfpathlineto{\pgfqpoint{1.457522in}{0.500000in}}%
\pgfpathclose%
\pgfusepath{fill}%
\end{pgfscope}%
\begin{pgfscope}%
\pgfpathrectangle{\pgfqpoint{0.750000in}{0.500000in}}{\pgfqpoint{4.650000in}{3.020000in}}%
\pgfusepath{clip}%
\pgfsetbuttcap%
\pgfsetmiterjoin%
\definecolor{currentfill}{rgb}{0.121569,0.466667,0.705882}%
\pgfsetfillcolor{currentfill}%
\pgfsetlinewidth{0.000000pt}%
\definecolor{currentstroke}{rgb}{0.000000,0.000000,0.000000}%
\pgfsetstrokecolor{currentstroke}%
\pgfsetstrokeopacity{0.000000}%
\pgfsetdash{}{0pt}%
\pgfpathmoveto{\pgfqpoint{1.490600in}{0.500000in}}%
\pgfpathlineto{\pgfqpoint{1.517061in}{0.500000in}}%
\pgfpathlineto{\pgfqpoint{1.517061in}{0.500000in}}%
\pgfpathlineto{\pgfqpoint{1.490600in}{0.500000in}}%
\pgfpathlineto{\pgfqpoint{1.490600in}{0.500000in}}%
\pgfpathclose%
\pgfusepath{fill}%
\end{pgfscope}%
\begin{pgfscope}%
\pgfpathrectangle{\pgfqpoint{0.750000in}{0.500000in}}{\pgfqpoint{4.650000in}{3.020000in}}%
\pgfusepath{clip}%
\pgfsetbuttcap%
\pgfsetmiterjoin%
\definecolor{currentfill}{rgb}{0.121569,0.466667,0.705882}%
\pgfsetfillcolor{currentfill}%
\pgfsetlinewidth{0.000000pt}%
\definecolor{currentstroke}{rgb}{0.000000,0.000000,0.000000}%
\pgfsetstrokecolor{currentstroke}%
\pgfsetstrokeopacity{0.000000}%
\pgfsetdash{}{0pt}%
\pgfpathmoveto{\pgfqpoint{1.523677in}{0.500000in}}%
\pgfpathlineto{\pgfqpoint{1.550139in}{0.500000in}}%
\pgfpathlineto{\pgfqpoint{1.550139in}{0.500000in}}%
\pgfpathlineto{\pgfqpoint{1.523677in}{0.500000in}}%
\pgfpathlineto{\pgfqpoint{1.523677in}{0.500000in}}%
\pgfpathclose%
\pgfusepath{fill}%
\end{pgfscope}%
\begin{pgfscope}%
\pgfpathrectangle{\pgfqpoint{0.750000in}{0.500000in}}{\pgfqpoint{4.650000in}{3.020000in}}%
\pgfusepath{clip}%
\pgfsetbuttcap%
\pgfsetmiterjoin%
\definecolor{currentfill}{rgb}{0.121569,0.466667,0.705882}%
\pgfsetfillcolor{currentfill}%
\pgfsetlinewidth{0.000000pt}%
\definecolor{currentstroke}{rgb}{0.000000,0.000000,0.000000}%
\pgfsetstrokecolor{currentstroke}%
\pgfsetstrokeopacity{0.000000}%
\pgfsetdash{}{0pt}%
\pgfpathmoveto{\pgfqpoint{1.556754in}{0.500000in}}%
\pgfpathlineto{\pgfqpoint{1.583216in}{0.500000in}}%
\pgfpathlineto{\pgfqpoint{1.583216in}{0.500000in}}%
\pgfpathlineto{\pgfqpoint{1.556754in}{0.500000in}}%
\pgfpathlineto{\pgfqpoint{1.556754in}{0.500000in}}%
\pgfpathclose%
\pgfusepath{fill}%
\end{pgfscope}%
\begin{pgfscope}%
\pgfpathrectangle{\pgfqpoint{0.750000in}{0.500000in}}{\pgfqpoint{4.650000in}{3.020000in}}%
\pgfusepath{clip}%
\pgfsetbuttcap%
\pgfsetmiterjoin%
\definecolor{currentfill}{rgb}{0.121569,0.466667,0.705882}%
\pgfsetfillcolor{currentfill}%
\pgfsetlinewidth{0.000000pt}%
\definecolor{currentstroke}{rgb}{0.000000,0.000000,0.000000}%
\pgfsetstrokecolor{currentstroke}%
\pgfsetstrokeopacity{0.000000}%
\pgfsetdash{}{0pt}%
\pgfpathmoveto{\pgfqpoint{1.589831in}{0.500000in}}%
\pgfpathlineto{\pgfqpoint{1.616293in}{0.500000in}}%
\pgfpathlineto{\pgfqpoint{1.616293in}{0.500000in}}%
\pgfpathlineto{\pgfqpoint{1.589831in}{0.500000in}}%
\pgfpathlineto{\pgfqpoint{1.589831in}{0.500000in}}%
\pgfpathclose%
\pgfusepath{fill}%
\end{pgfscope}%
\begin{pgfscope}%
\pgfpathrectangle{\pgfqpoint{0.750000in}{0.500000in}}{\pgfqpoint{4.650000in}{3.020000in}}%
\pgfusepath{clip}%
\pgfsetbuttcap%
\pgfsetmiterjoin%
\definecolor{currentfill}{rgb}{0.121569,0.466667,0.705882}%
\pgfsetfillcolor{currentfill}%
\pgfsetlinewidth{0.000000pt}%
\definecolor{currentstroke}{rgb}{0.000000,0.000000,0.000000}%
\pgfsetstrokecolor{currentstroke}%
\pgfsetstrokeopacity{0.000000}%
\pgfsetdash{}{0pt}%
\pgfpathmoveto{\pgfqpoint{1.622909in}{0.500000in}}%
\pgfpathlineto{\pgfqpoint{1.649370in}{0.500000in}}%
\pgfpathlineto{\pgfqpoint{1.649370in}{0.500000in}}%
\pgfpathlineto{\pgfqpoint{1.622909in}{0.500000in}}%
\pgfpathlineto{\pgfqpoint{1.622909in}{0.500000in}}%
\pgfpathclose%
\pgfusepath{fill}%
\end{pgfscope}%
\begin{pgfscope}%
\pgfpathrectangle{\pgfqpoint{0.750000in}{0.500000in}}{\pgfqpoint{4.650000in}{3.020000in}}%
\pgfusepath{clip}%
\pgfsetbuttcap%
\pgfsetmiterjoin%
\definecolor{currentfill}{rgb}{0.121569,0.466667,0.705882}%
\pgfsetfillcolor{currentfill}%
\pgfsetlinewidth{0.000000pt}%
\definecolor{currentstroke}{rgb}{0.000000,0.000000,0.000000}%
\pgfsetstrokecolor{currentstroke}%
\pgfsetstrokeopacity{0.000000}%
\pgfsetdash{}{0pt}%
\pgfpathmoveto{\pgfqpoint{1.655986in}{0.500000in}}%
\pgfpathlineto{\pgfqpoint{1.682448in}{0.500000in}}%
\pgfpathlineto{\pgfqpoint{1.682448in}{0.500000in}}%
\pgfpathlineto{\pgfqpoint{1.655986in}{0.500000in}}%
\pgfpathlineto{\pgfqpoint{1.655986in}{0.500000in}}%
\pgfpathclose%
\pgfusepath{fill}%
\end{pgfscope}%
\begin{pgfscope}%
\pgfpathrectangle{\pgfqpoint{0.750000in}{0.500000in}}{\pgfqpoint{4.650000in}{3.020000in}}%
\pgfusepath{clip}%
\pgfsetbuttcap%
\pgfsetmiterjoin%
\definecolor{currentfill}{rgb}{0.121569,0.466667,0.705882}%
\pgfsetfillcolor{currentfill}%
\pgfsetlinewidth{0.000000pt}%
\definecolor{currentstroke}{rgb}{0.000000,0.000000,0.000000}%
\pgfsetstrokecolor{currentstroke}%
\pgfsetstrokeopacity{0.000000}%
\pgfsetdash{}{0pt}%
\pgfpathmoveto{\pgfqpoint{1.689063in}{0.500000in}}%
\pgfpathlineto{\pgfqpoint{1.715525in}{0.500000in}}%
\pgfpathlineto{\pgfqpoint{1.715525in}{0.500000in}}%
\pgfpathlineto{\pgfqpoint{1.689063in}{0.500000in}}%
\pgfpathlineto{\pgfqpoint{1.689063in}{0.500000in}}%
\pgfpathclose%
\pgfusepath{fill}%
\end{pgfscope}%
\begin{pgfscope}%
\pgfpathrectangle{\pgfqpoint{0.750000in}{0.500000in}}{\pgfqpoint{4.650000in}{3.020000in}}%
\pgfusepath{clip}%
\pgfsetbuttcap%
\pgfsetmiterjoin%
\definecolor{currentfill}{rgb}{0.121569,0.466667,0.705882}%
\pgfsetfillcolor{currentfill}%
\pgfsetlinewidth{0.000000pt}%
\definecolor{currentstroke}{rgb}{0.000000,0.000000,0.000000}%
\pgfsetstrokecolor{currentstroke}%
\pgfsetstrokeopacity{0.000000}%
\pgfsetdash{}{0pt}%
\pgfpathmoveto{\pgfqpoint{1.722140in}{0.500000in}}%
\pgfpathlineto{\pgfqpoint{1.748602in}{0.500000in}}%
\pgfpathlineto{\pgfqpoint{1.748602in}{0.500000in}}%
\pgfpathlineto{\pgfqpoint{1.722140in}{0.500000in}}%
\pgfpathlineto{\pgfqpoint{1.722140in}{0.500000in}}%
\pgfpathclose%
\pgfusepath{fill}%
\end{pgfscope}%
\begin{pgfscope}%
\pgfpathrectangle{\pgfqpoint{0.750000in}{0.500000in}}{\pgfqpoint{4.650000in}{3.020000in}}%
\pgfusepath{clip}%
\pgfsetbuttcap%
\pgfsetmiterjoin%
\definecolor{currentfill}{rgb}{0.121569,0.466667,0.705882}%
\pgfsetfillcolor{currentfill}%
\pgfsetlinewidth{0.000000pt}%
\definecolor{currentstroke}{rgb}{0.000000,0.000000,0.000000}%
\pgfsetstrokecolor{currentstroke}%
\pgfsetstrokeopacity{0.000000}%
\pgfsetdash{}{0pt}%
\pgfpathmoveto{\pgfqpoint{1.755218in}{0.500000in}}%
\pgfpathlineto{\pgfqpoint{1.781679in}{0.500000in}}%
\pgfpathlineto{\pgfqpoint{1.781679in}{0.500000in}}%
\pgfpathlineto{\pgfqpoint{1.755218in}{0.500000in}}%
\pgfpathlineto{\pgfqpoint{1.755218in}{0.500000in}}%
\pgfpathclose%
\pgfusepath{fill}%
\end{pgfscope}%
\begin{pgfscope}%
\pgfpathrectangle{\pgfqpoint{0.750000in}{0.500000in}}{\pgfqpoint{4.650000in}{3.020000in}}%
\pgfusepath{clip}%
\pgfsetbuttcap%
\pgfsetmiterjoin%
\definecolor{currentfill}{rgb}{0.121569,0.466667,0.705882}%
\pgfsetfillcolor{currentfill}%
\pgfsetlinewidth{0.000000pt}%
\definecolor{currentstroke}{rgb}{0.000000,0.000000,0.000000}%
\pgfsetstrokecolor{currentstroke}%
\pgfsetstrokeopacity{0.000000}%
\pgfsetdash{}{0pt}%
\pgfpathmoveto{\pgfqpoint{1.788295in}{0.500000in}}%
\pgfpathlineto{\pgfqpoint{1.814757in}{0.500000in}}%
\pgfpathlineto{\pgfqpoint{1.814757in}{0.500000in}}%
\pgfpathlineto{\pgfqpoint{1.788295in}{0.500000in}}%
\pgfpathlineto{\pgfqpoint{1.788295in}{0.500000in}}%
\pgfpathclose%
\pgfusepath{fill}%
\end{pgfscope}%
\begin{pgfscope}%
\pgfpathrectangle{\pgfqpoint{0.750000in}{0.500000in}}{\pgfqpoint{4.650000in}{3.020000in}}%
\pgfusepath{clip}%
\pgfsetbuttcap%
\pgfsetmiterjoin%
\definecolor{currentfill}{rgb}{0.121569,0.466667,0.705882}%
\pgfsetfillcolor{currentfill}%
\pgfsetlinewidth{0.000000pt}%
\definecolor{currentstroke}{rgb}{0.000000,0.000000,0.000000}%
\pgfsetstrokecolor{currentstroke}%
\pgfsetstrokeopacity{0.000000}%
\pgfsetdash{}{0pt}%
\pgfpathmoveto{\pgfqpoint{1.821372in}{0.500000in}}%
\pgfpathlineto{\pgfqpoint{1.847834in}{0.500000in}}%
\pgfpathlineto{\pgfqpoint{1.847834in}{0.500000in}}%
\pgfpathlineto{\pgfqpoint{1.821372in}{0.500000in}}%
\pgfpathlineto{\pgfqpoint{1.821372in}{0.500000in}}%
\pgfpathclose%
\pgfusepath{fill}%
\end{pgfscope}%
\begin{pgfscope}%
\pgfpathrectangle{\pgfqpoint{0.750000in}{0.500000in}}{\pgfqpoint{4.650000in}{3.020000in}}%
\pgfusepath{clip}%
\pgfsetbuttcap%
\pgfsetmiterjoin%
\definecolor{currentfill}{rgb}{0.121569,0.466667,0.705882}%
\pgfsetfillcolor{currentfill}%
\pgfsetlinewidth{0.000000pt}%
\definecolor{currentstroke}{rgb}{0.000000,0.000000,0.000000}%
\pgfsetstrokecolor{currentstroke}%
\pgfsetstrokeopacity{0.000000}%
\pgfsetdash{}{0pt}%
\pgfpathmoveto{\pgfqpoint{1.854449in}{0.500000in}}%
\pgfpathlineto{\pgfqpoint{1.880911in}{0.500000in}}%
\pgfpathlineto{\pgfqpoint{1.880911in}{0.500000in}}%
\pgfpathlineto{\pgfqpoint{1.854449in}{0.500000in}}%
\pgfpathlineto{\pgfqpoint{1.854449in}{0.500000in}}%
\pgfpathclose%
\pgfusepath{fill}%
\end{pgfscope}%
\begin{pgfscope}%
\pgfpathrectangle{\pgfqpoint{0.750000in}{0.500000in}}{\pgfqpoint{4.650000in}{3.020000in}}%
\pgfusepath{clip}%
\pgfsetbuttcap%
\pgfsetmiterjoin%
\definecolor{currentfill}{rgb}{0.121569,0.466667,0.705882}%
\pgfsetfillcolor{currentfill}%
\pgfsetlinewidth{0.000000pt}%
\definecolor{currentstroke}{rgb}{0.000000,0.000000,0.000000}%
\pgfsetstrokecolor{currentstroke}%
\pgfsetstrokeopacity{0.000000}%
\pgfsetdash{}{0pt}%
\pgfpathmoveto{\pgfqpoint{1.887527in}{0.500000in}}%
\pgfpathlineto{\pgfqpoint{1.913988in}{0.500000in}}%
\pgfpathlineto{\pgfqpoint{1.913988in}{0.500000in}}%
\pgfpathlineto{\pgfqpoint{1.887527in}{0.500000in}}%
\pgfpathlineto{\pgfqpoint{1.887527in}{0.500000in}}%
\pgfpathclose%
\pgfusepath{fill}%
\end{pgfscope}%
\begin{pgfscope}%
\pgfpathrectangle{\pgfqpoint{0.750000in}{0.500000in}}{\pgfqpoint{4.650000in}{3.020000in}}%
\pgfusepath{clip}%
\pgfsetbuttcap%
\pgfsetmiterjoin%
\definecolor{currentfill}{rgb}{0.121569,0.466667,0.705882}%
\pgfsetfillcolor{currentfill}%
\pgfsetlinewidth{0.000000pt}%
\definecolor{currentstroke}{rgb}{0.000000,0.000000,0.000000}%
\pgfsetstrokecolor{currentstroke}%
\pgfsetstrokeopacity{0.000000}%
\pgfsetdash{}{0pt}%
\pgfpathmoveto{\pgfqpoint{1.920604in}{0.500000in}}%
\pgfpathlineto{\pgfqpoint{1.947066in}{0.500000in}}%
\pgfpathlineto{\pgfqpoint{1.947066in}{0.500000in}}%
\pgfpathlineto{\pgfqpoint{1.920604in}{0.500000in}}%
\pgfpathlineto{\pgfqpoint{1.920604in}{0.500000in}}%
\pgfpathclose%
\pgfusepath{fill}%
\end{pgfscope}%
\begin{pgfscope}%
\pgfpathrectangle{\pgfqpoint{0.750000in}{0.500000in}}{\pgfqpoint{4.650000in}{3.020000in}}%
\pgfusepath{clip}%
\pgfsetbuttcap%
\pgfsetmiterjoin%
\definecolor{currentfill}{rgb}{0.121569,0.466667,0.705882}%
\pgfsetfillcolor{currentfill}%
\pgfsetlinewidth{0.000000pt}%
\definecolor{currentstroke}{rgb}{0.000000,0.000000,0.000000}%
\pgfsetstrokecolor{currentstroke}%
\pgfsetstrokeopacity{0.000000}%
\pgfsetdash{}{0pt}%
\pgfpathmoveto{\pgfqpoint{1.953681in}{0.500000in}}%
\pgfpathlineto{\pgfqpoint{1.980143in}{0.500000in}}%
\pgfpathlineto{\pgfqpoint{1.980143in}{0.500000in}}%
\pgfpathlineto{\pgfqpoint{1.953681in}{0.500000in}}%
\pgfpathlineto{\pgfqpoint{1.953681in}{0.500000in}}%
\pgfpathclose%
\pgfusepath{fill}%
\end{pgfscope}%
\begin{pgfscope}%
\pgfpathrectangle{\pgfqpoint{0.750000in}{0.500000in}}{\pgfqpoint{4.650000in}{3.020000in}}%
\pgfusepath{clip}%
\pgfsetbuttcap%
\pgfsetmiterjoin%
\definecolor{currentfill}{rgb}{0.121569,0.466667,0.705882}%
\pgfsetfillcolor{currentfill}%
\pgfsetlinewidth{0.000000pt}%
\definecolor{currentstroke}{rgb}{0.000000,0.000000,0.000000}%
\pgfsetstrokecolor{currentstroke}%
\pgfsetstrokeopacity{0.000000}%
\pgfsetdash{}{0pt}%
\pgfpathmoveto{\pgfqpoint{1.986758in}{0.500000in}}%
\pgfpathlineto{\pgfqpoint{2.013220in}{0.500000in}}%
\pgfpathlineto{\pgfqpoint{2.013220in}{0.500000in}}%
\pgfpathlineto{\pgfqpoint{1.986758in}{0.500000in}}%
\pgfpathlineto{\pgfqpoint{1.986758in}{0.500000in}}%
\pgfpathclose%
\pgfusepath{fill}%
\end{pgfscope}%
\begin{pgfscope}%
\pgfpathrectangle{\pgfqpoint{0.750000in}{0.500000in}}{\pgfqpoint{4.650000in}{3.020000in}}%
\pgfusepath{clip}%
\pgfsetbuttcap%
\pgfsetmiterjoin%
\definecolor{currentfill}{rgb}{0.121569,0.466667,0.705882}%
\pgfsetfillcolor{currentfill}%
\pgfsetlinewidth{0.000000pt}%
\definecolor{currentstroke}{rgb}{0.000000,0.000000,0.000000}%
\pgfsetstrokecolor{currentstroke}%
\pgfsetstrokeopacity{0.000000}%
\pgfsetdash{}{0pt}%
\pgfpathmoveto{\pgfqpoint{2.019836in}{0.500000in}}%
\pgfpathlineto{\pgfqpoint{2.046297in}{0.500000in}}%
\pgfpathlineto{\pgfqpoint{2.046297in}{0.500000in}}%
\pgfpathlineto{\pgfqpoint{2.019836in}{0.500000in}}%
\pgfpathlineto{\pgfqpoint{2.019836in}{0.500000in}}%
\pgfpathclose%
\pgfusepath{fill}%
\end{pgfscope}%
\begin{pgfscope}%
\pgfpathrectangle{\pgfqpoint{0.750000in}{0.500000in}}{\pgfqpoint{4.650000in}{3.020000in}}%
\pgfusepath{clip}%
\pgfsetbuttcap%
\pgfsetmiterjoin%
\definecolor{currentfill}{rgb}{0.121569,0.466667,0.705882}%
\pgfsetfillcolor{currentfill}%
\pgfsetlinewidth{0.000000pt}%
\definecolor{currentstroke}{rgb}{0.000000,0.000000,0.000000}%
\pgfsetstrokecolor{currentstroke}%
\pgfsetstrokeopacity{0.000000}%
\pgfsetdash{}{0pt}%
\pgfpathmoveto{\pgfqpoint{2.052913in}{0.500000in}}%
\pgfpathlineto{\pgfqpoint{2.079375in}{0.500000in}}%
\pgfpathlineto{\pgfqpoint{2.079375in}{0.500001in}}%
\pgfpathlineto{\pgfqpoint{2.052913in}{0.500001in}}%
\pgfpathlineto{\pgfqpoint{2.052913in}{0.500000in}}%
\pgfpathclose%
\pgfusepath{fill}%
\end{pgfscope}%
\begin{pgfscope}%
\pgfpathrectangle{\pgfqpoint{0.750000in}{0.500000in}}{\pgfqpoint{4.650000in}{3.020000in}}%
\pgfusepath{clip}%
\pgfsetbuttcap%
\pgfsetmiterjoin%
\definecolor{currentfill}{rgb}{0.121569,0.466667,0.705882}%
\pgfsetfillcolor{currentfill}%
\pgfsetlinewidth{0.000000pt}%
\definecolor{currentstroke}{rgb}{0.000000,0.000000,0.000000}%
\pgfsetstrokecolor{currentstroke}%
\pgfsetstrokeopacity{0.000000}%
\pgfsetdash{}{0pt}%
\pgfpathmoveto{\pgfqpoint{2.085990in}{0.500000in}}%
\pgfpathlineto{\pgfqpoint{2.112452in}{0.500000in}}%
\pgfpathlineto{\pgfqpoint{2.112452in}{0.500001in}}%
\pgfpathlineto{\pgfqpoint{2.085990in}{0.500001in}}%
\pgfpathlineto{\pgfqpoint{2.085990in}{0.500000in}}%
\pgfpathclose%
\pgfusepath{fill}%
\end{pgfscope}%
\begin{pgfscope}%
\pgfpathrectangle{\pgfqpoint{0.750000in}{0.500000in}}{\pgfqpoint{4.650000in}{3.020000in}}%
\pgfusepath{clip}%
\pgfsetbuttcap%
\pgfsetmiterjoin%
\definecolor{currentfill}{rgb}{0.121569,0.466667,0.705882}%
\pgfsetfillcolor{currentfill}%
\pgfsetlinewidth{0.000000pt}%
\definecolor{currentstroke}{rgb}{0.000000,0.000000,0.000000}%
\pgfsetstrokecolor{currentstroke}%
\pgfsetstrokeopacity{0.000000}%
\pgfsetdash{}{0pt}%
\pgfpathmoveto{\pgfqpoint{2.119067in}{0.500000in}}%
\pgfpathlineto{\pgfqpoint{2.145529in}{0.500000in}}%
\pgfpathlineto{\pgfqpoint{2.145529in}{0.500004in}}%
\pgfpathlineto{\pgfqpoint{2.119067in}{0.500004in}}%
\pgfpathlineto{\pgfqpoint{2.119067in}{0.500000in}}%
\pgfpathclose%
\pgfusepath{fill}%
\end{pgfscope}%
\begin{pgfscope}%
\pgfpathrectangle{\pgfqpoint{0.750000in}{0.500000in}}{\pgfqpoint{4.650000in}{3.020000in}}%
\pgfusepath{clip}%
\pgfsetbuttcap%
\pgfsetmiterjoin%
\definecolor{currentfill}{rgb}{0.121569,0.466667,0.705882}%
\pgfsetfillcolor{currentfill}%
\pgfsetlinewidth{0.000000pt}%
\definecolor{currentstroke}{rgb}{0.000000,0.000000,0.000000}%
\pgfsetstrokecolor{currentstroke}%
\pgfsetstrokeopacity{0.000000}%
\pgfsetdash{}{0pt}%
\pgfpathmoveto{\pgfqpoint{2.152145in}{0.500000in}}%
\pgfpathlineto{\pgfqpoint{2.178606in}{0.500000in}}%
\pgfpathlineto{\pgfqpoint{2.178606in}{0.500010in}}%
\pgfpathlineto{\pgfqpoint{2.152145in}{0.500010in}}%
\pgfpathlineto{\pgfqpoint{2.152145in}{0.500000in}}%
\pgfpathclose%
\pgfusepath{fill}%
\end{pgfscope}%
\begin{pgfscope}%
\pgfpathrectangle{\pgfqpoint{0.750000in}{0.500000in}}{\pgfqpoint{4.650000in}{3.020000in}}%
\pgfusepath{clip}%
\pgfsetbuttcap%
\pgfsetmiterjoin%
\definecolor{currentfill}{rgb}{0.121569,0.466667,0.705882}%
\pgfsetfillcolor{currentfill}%
\pgfsetlinewidth{0.000000pt}%
\definecolor{currentstroke}{rgb}{0.000000,0.000000,0.000000}%
\pgfsetstrokecolor{currentstroke}%
\pgfsetstrokeopacity{0.000000}%
\pgfsetdash{}{0pt}%
\pgfpathmoveto{\pgfqpoint{2.185222in}{0.500000in}}%
\pgfpathlineto{\pgfqpoint{2.211684in}{0.500000in}}%
\pgfpathlineto{\pgfqpoint{2.211684in}{0.500025in}}%
\pgfpathlineto{\pgfqpoint{2.185222in}{0.500025in}}%
\pgfpathlineto{\pgfqpoint{2.185222in}{0.500000in}}%
\pgfpathclose%
\pgfusepath{fill}%
\end{pgfscope}%
\begin{pgfscope}%
\pgfpathrectangle{\pgfqpoint{0.750000in}{0.500000in}}{\pgfqpoint{4.650000in}{3.020000in}}%
\pgfusepath{clip}%
\pgfsetbuttcap%
\pgfsetmiterjoin%
\definecolor{currentfill}{rgb}{0.121569,0.466667,0.705882}%
\pgfsetfillcolor{currentfill}%
\pgfsetlinewidth{0.000000pt}%
\definecolor{currentstroke}{rgb}{0.000000,0.000000,0.000000}%
\pgfsetstrokecolor{currentstroke}%
\pgfsetstrokeopacity{0.000000}%
\pgfsetdash{}{0pt}%
\pgfpathmoveto{\pgfqpoint{2.218299in}{0.500000in}}%
\pgfpathlineto{\pgfqpoint{2.244761in}{0.500000in}}%
\pgfpathlineto{\pgfqpoint{2.244761in}{0.500060in}}%
\pgfpathlineto{\pgfqpoint{2.218299in}{0.500060in}}%
\pgfpathlineto{\pgfqpoint{2.218299in}{0.500000in}}%
\pgfpathclose%
\pgfusepath{fill}%
\end{pgfscope}%
\begin{pgfscope}%
\pgfpathrectangle{\pgfqpoint{0.750000in}{0.500000in}}{\pgfqpoint{4.650000in}{3.020000in}}%
\pgfusepath{clip}%
\pgfsetbuttcap%
\pgfsetmiterjoin%
\definecolor{currentfill}{rgb}{0.121569,0.466667,0.705882}%
\pgfsetfillcolor{currentfill}%
\pgfsetlinewidth{0.000000pt}%
\definecolor{currentstroke}{rgb}{0.000000,0.000000,0.000000}%
\pgfsetstrokecolor{currentstroke}%
\pgfsetstrokeopacity{0.000000}%
\pgfsetdash{}{0pt}%
\pgfpathmoveto{\pgfqpoint{2.251376in}{0.500000in}}%
\pgfpathlineto{\pgfqpoint{2.277838in}{0.500000in}}%
\pgfpathlineto{\pgfqpoint{2.277838in}{0.500138in}}%
\pgfpathlineto{\pgfqpoint{2.251376in}{0.500138in}}%
\pgfpathlineto{\pgfqpoint{2.251376in}{0.500000in}}%
\pgfpathclose%
\pgfusepath{fill}%
\end{pgfscope}%
\begin{pgfscope}%
\pgfpathrectangle{\pgfqpoint{0.750000in}{0.500000in}}{\pgfqpoint{4.650000in}{3.020000in}}%
\pgfusepath{clip}%
\pgfsetbuttcap%
\pgfsetmiterjoin%
\definecolor{currentfill}{rgb}{0.121569,0.466667,0.705882}%
\pgfsetfillcolor{currentfill}%
\pgfsetlinewidth{0.000000pt}%
\definecolor{currentstroke}{rgb}{0.000000,0.000000,0.000000}%
\pgfsetstrokecolor{currentstroke}%
\pgfsetstrokeopacity{0.000000}%
\pgfsetdash{}{0pt}%
\pgfpathmoveto{\pgfqpoint{2.284454in}{0.500000in}}%
\pgfpathlineto{\pgfqpoint{2.310915in}{0.500000in}}%
\pgfpathlineto{\pgfqpoint{2.310915in}{0.500306in}}%
\pgfpathlineto{\pgfqpoint{2.284454in}{0.500306in}}%
\pgfpathlineto{\pgfqpoint{2.284454in}{0.500000in}}%
\pgfpathclose%
\pgfusepath{fill}%
\end{pgfscope}%
\begin{pgfscope}%
\pgfpathrectangle{\pgfqpoint{0.750000in}{0.500000in}}{\pgfqpoint{4.650000in}{3.020000in}}%
\pgfusepath{clip}%
\pgfsetbuttcap%
\pgfsetmiterjoin%
\definecolor{currentfill}{rgb}{0.121569,0.466667,0.705882}%
\pgfsetfillcolor{currentfill}%
\pgfsetlinewidth{0.000000pt}%
\definecolor{currentstroke}{rgb}{0.000000,0.000000,0.000000}%
\pgfsetstrokecolor{currentstroke}%
\pgfsetstrokeopacity{0.000000}%
\pgfsetdash{}{0pt}%
\pgfpathmoveto{\pgfqpoint{2.317531in}{0.500000in}}%
\pgfpathlineto{\pgfqpoint{2.343993in}{0.500000in}}%
\pgfpathlineto{\pgfqpoint{2.343993in}{0.500657in}}%
\pgfpathlineto{\pgfqpoint{2.317531in}{0.500657in}}%
\pgfpathlineto{\pgfqpoint{2.317531in}{0.500000in}}%
\pgfpathclose%
\pgfusepath{fill}%
\end{pgfscope}%
\begin{pgfscope}%
\pgfpathrectangle{\pgfqpoint{0.750000in}{0.500000in}}{\pgfqpoint{4.650000in}{3.020000in}}%
\pgfusepath{clip}%
\pgfsetbuttcap%
\pgfsetmiterjoin%
\definecolor{currentfill}{rgb}{0.121569,0.466667,0.705882}%
\pgfsetfillcolor{currentfill}%
\pgfsetlinewidth{0.000000pt}%
\definecolor{currentstroke}{rgb}{0.000000,0.000000,0.000000}%
\pgfsetstrokecolor{currentstroke}%
\pgfsetstrokeopacity{0.000000}%
\pgfsetdash{}{0pt}%
\pgfpathmoveto{\pgfqpoint{2.350608in}{0.500000in}}%
\pgfpathlineto{\pgfqpoint{2.377070in}{0.500000in}}%
\pgfpathlineto{\pgfqpoint{2.377070in}{0.501360in}}%
\pgfpathlineto{\pgfqpoint{2.350608in}{0.501360in}}%
\pgfpathlineto{\pgfqpoint{2.350608in}{0.500000in}}%
\pgfpathclose%
\pgfusepath{fill}%
\end{pgfscope}%
\begin{pgfscope}%
\pgfpathrectangle{\pgfqpoint{0.750000in}{0.500000in}}{\pgfqpoint{4.650000in}{3.020000in}}%
\pgfusepath{clip}%
\pgfsetbuttcap%
\pgfsetmiterjoin%
\definecolor{currentfill}{rgb}{0.121569,0.466667,0.705882}%
\pgfsetfillcolor{currentfill}%
\pgfsetlinewidth{0.000000pt}%
\definecolor{currentstroke}{rgb}{0.000000,0.000000,0.000000}%
\pgfsetstrokecolor{currentstroke}%
\pgfsetstrokeopacity{0.000000}%
\pgfsetdash{}{0pt}%
\pgfpathmoveto{\pgfqpoint{2.383685in}{0.500000in}}%
\pgfpathlineto{\pgfqpoint{2.410147in}{0.500000in}}%
\pgfpathlineto{\pgfqpoint{2.410147in}{0.502721in}}%
\pgfpathlineto{\pgfqpoint{2.383685in}{0.502721in}}%
\pgfpathlineto{\pgfqpoint{2.383685in}{0.500000in}}%
\pgfpathclose%
\pgfusepath{fill}%
\end{pgfscope}%
\begin{pgfscope}%
\pgfpathrectangle{\pgfqpoint{0.750000in}{0.500000in}}{\pgfqpoint{4.650000in}{3.020000in}}%
\pgfusepath{clip}%
\pgfsetbuttcap%
\pgfsetmiterjoin%
\definecolor{currentfill}{rgb}{0.121569,0.466667,0.705882}%
\pgfsetfillcolor{currentfill}%
\pgfsetlinewidth{0.000000pt}%
\definecolor{currentstroke}{rgb}{0.000000,0.000000,0.000000}%
\pgfsetstrokecolor{currentstroke}%
\pgfsetstrokeopacity{0.000000}%
\pgfsetdash{}{0pt}%
\pgfpathmoveto{\pgfqpoint{2.416763in}{0.500000in}}%
\pgfpathlineto{\pgfqpoint{2.443224in}{0.500000in}}%
\pgfpathlineto{\pgfqpoint{2.443224in}{0.505256in}}%
\pgfpathlineto{\pgfqpoint{2.416763in}{0.505256in}}%
\pgfpathlineto{\pgfqpoint{2.416763in}{0.500000in}}%
\pgfpathclose%
\pgfusepath{fill}%
\end{pgfscope}%
\begin{pgfscope}%
\pgfpathrectangle{\pgfqpoint{0.750000in}{0.500000in}}{\pgfqpoint{4.650000in}{3.020000in}}%
\pgfusepath{clip}%
\pgfsetbuttcap%
\pgfsetmiterjoin%
\definecolor{currentfill}{rgb}{0.121569,0.466667,0.705882}%
\pgfsetfillcolor{currentfill}%
\pgfsetlinewidth{0.000000pt}%
\definecolor{currentstroke}{rgb}{0.000000,0.000000,0.000000}%
\pgfsetstrokecolor{currentstroke}%
\pgfsetstrokeopacity{0.000000}%
\pgfsetdash{}{0pt}%
\pgfpathmoveto{\pgfqpoint{2.449840in}{0.500000in}}%
\pgfpathlineto{\pgfqpoint{2.476302in}{0.500000in}}%
\pgfpathlineto{\pgfqpoint{2.476302in}{0.509811in}}%
\pgfpathlineto{\pgfqpoint{2.449840in}{0.509811in}}%
\pgfpathlineto{\pgfqpoint{2.449840in}{0.500000in}}%
\pgfpathclose%
\pgfusepath{fill}%
\end{pgfscope}%
\begin{pgfscope}%
\pgfpathrectangle{\pgfqpoint{0.750000in}{0.500000in}}{\pgfqpoint{4.650000in}{3.020000in}}%
\pgfusepath{clip}%
\pgfsetbuttcap%
\pgfsetmiterjoin%
\definecolor{currentfill}{rgb}{0.121569,0.466667,0.705882}%
\pgfsetfillcolor{currentfill}%
\pgfsetlinewidth{0.000000pt}%
\definecolor{currentstroke}{rgb}{0.000000,0.000000,0.000000}%
\pgfsetstrokecolor{currentstroke}%
\pgfsetstrokeopacity{0.000000}%
\pgfsetdash{}{0pt}%
\pgfpathmoveto{\pgfqpoint{2.482917in}{0.500000in}}%
\pgfpathlineto{\pgfqpoint{2.509379in}{0.500000in}}%
\pgfpathlineto{\pgfqpoint{2.509379in}{0.517703in}}%
\pgfpathlineto{\pgfqpoint{2.482917in}{0.517703in}}%
\pgfpathlineto{\pgfqpoint{2.482917in}{0.500000in}}%
\pgfpathclose%
\pgfusepath{fill}%
\end{pgfscope}%
\begin{pgfscope}%
\pgfpathrectangle{\pgfqpoint{0.750000in}{0.500000in}}{\pgfqpoint{4.650000in}{3.020000in}}%
\pgfusepath{clip}%
\pgfsetbuttcap%
\pgfsetmiterjoin%
\definecolor{currentfill}{rgb}{0.121569,0.466667,0.705882}%
\pgfsetfillcolor{currentfill}%
\pgfsetlinewidth{0.000000pt}%
\definecolor{currentstroke}{rgb}{0.000000,0.000000,0.000000}%
\pgfsetstrokecolor{currentstroke}%
\pgfsetstrokeopacity{0.000000}%
\pgfsetdash{}{0pt}%
\pgfpathmoveto{\pgfqpoint{2.515994in}{0.500000in}}%
\pgfpathlineto{\pgfqpoint{2.542456in}{0.500000in}}%
\pgfpathlineto{\pgfqpoint{2.542456in}{0.530887in}}%
\pgfpathlineto{\pgfqpoint{2.515994in}{0.530887in}}%
\pgfpathlineto{\pgfqpoint{2.515994in}{0.500000in}}%
\pgfpathclose%
\pgfusepath{fill}%
\end{pgfscope}%
\begin{pgfscope}%
\pgfpathrectangle{\pgfqpoint{0.750000in}{0.500000in}}{\pgfqpoint{4.650000in}{3.020000in}}%
\pgfusepath{clip}%
\pgfsetbuttcap%
\pgfsetmiterjoin%
\definecolor{currentfill}{rgb}{0.121569,0.466667,0.705882}%
\pgfsetfillcolor{currentfill}%
\pgfsetlinewidth{0.000000pt}%
\definecolor{currentstroke}{rgb}{0.000000,0.000000,0.000000}%
\pgfsetstrokecolor{currentstroke}%
\pgfsetstrokeopacity{0.000000}%
\pgfsetdash{}{0pt}%
\pgfpathmoveto{\pgfqpoint{2.549072in}{0.500000in}}%
\pgfpathlineto{\pgfqpoint{2.575534in}{0.500000in}}%
\pgfpathlineto{\pgfqpoint{2.575534in}{0.552121in}}%
\pgfpathlineto{\pgfqpoint{2.549072in}{0.552121in}}%
\pgfpathlineto{\pgfqpoint{2.549072in}{0.500000in}}%
\pgfpathclose%
\pgfusepath{fill}%
\end{pgfscope}%
\begin{pgfscope}%
\pgfpathrectangle{\pgfqpoint{0.750000in}{0.500000in}}{\pgfqpoint{4.650000in}{3.020000in}}%
\pgfusepath{clip}%
\pgfsetbuttcap%
\pgfsetmiterjoin%
\definecolor{currentfill}{rgb}{0.121569,0.466667,0.705882}%
\pgfsetfillcolor{currentfill}%
\pgfsetlinewidth{0.000000pt}%
\definecolor{currentstroke}{rgb}{0.000000,0.000000,0.000000}%
\pgfsetstrokecolor{currentstroke}%
\pgfsetstrokeopacity{0.000000}%
\pgfsetdash{}{0pt}%
\pgfpathmoveto{\pgfqpoint{2.582149in}{0.500000in}}%
\pgfpathlineto{\pgfqpoint{2.608611in}{0.500000in}}%
\pgfpathlineto{\pgfqpoint{2.608611in}{0.585096in}}%
\pgfpathlineto{\pgfqpoint{2.582149in}{0.585096in}}%
\pgfpathlineto{\pgfqpoint{2.582149in}{0.500000in}}%
\pgfpathclose%
\pgfusepath{fill}%
\end{pgfscope}%
\begin{pgfscope}%
\pgfpathrectangle{\pgfqpoint{0.750000in}{0.500000in}}{\pgfqpoint{4.650000in}{3.020000in}}%
\pgfusepath{clip}%
\pgfsetbuttcap%
\pgfsetmiterjoin%
\definecolor{currentfill}{rgb}{0.121569,0.466667,0.705882}%
\pgfsetfillcolor{currentfill}%
\pgfsetlinewidth{0.000000pt}%
\definecolor{currentstroke}{rgb}{0.000000,0.000000,0.000000}%
\pgfsetstrokecolor{currentstroke}%
\pgfsetstrokeopacity{0.000000}%
\pgfsetdash{}{0pt}%
\pgfpathmoveto{\pgfqpoint{2.615226in}{0.500000in}}%
\pgfpathlineto{\pgfqpoint{2.641688in}{0.500000in}}%
\pgfpathlineto{\pgfqpoint{2.641688in}{0.634452in}}%
\pgfpathlineto{\pgfqpoint{2.615226in}{0.634452in}}%
\pgfpathlineto{\pgfqpoint{2.615226in}{0.500000in}}%
\pgfpathclose%
\pgfusepath{fill}%
\end{pgfscope}%
\begin{pgfscope}%
\pgfpathrectangle{\pgfqpoint{0.750000in}{0.500000in}}{\pgfqpoint{4.650000in}{3.020000in}}%
\pgfusepath{clip}%
\pgfsetbuttcap%
\pgfsetmiterjoin%
\definecolor{currentfill}{rgb}{0.121569,0.466667,0.705882}%
\pgfsetfillcolor{currentfill}%
\pgfsetlinewidth{0.000000pt}%
\definecolor{currentstroke}{rgb}{0.000000,0.000000,0.000000}%
\pgfsetstrokecolor{currentstroke}%
\pgfsetstrokeopacity{0.000000}%
\pgfsetdash{}{0pt}%
\pgfpathmoveto{\pgfqpoint{2.648303in}{0.500000in}}%
\pgfpathlineto{\pgfqpoint{2.674765in}{0.500000in}}%
\pgfpathlineto{\pgfqpoint{2.674765in}{0.705632in}}%
\pgfpathlineto{\pgfqpoint{2.648303in}{0.705632in}}%
\pgfpathlineto{\pgfqpoint{2.648303in}{0.500000in}}%
\pgfpathclose%
\pgfusepath{fill}%
\end{pgfscope}%
\begin{pgfscope}%
\pgfpathrectangle{\pgfqpoint{0.750000in}{0.500000in}}{\pgfqpoint{4.650000in}{3.020000in}}%
\pgfusepath{clip}%
\pgfsetbuttcap%
\pgfsetmiterjoin%
\definecolor{currentfill}{rgb}{0.121569,0.466667,0.705882}%
\pgfsetfillcolor{currentfill}%
\pgfsetlinewidth{0.000000pt}%
\definecolor{currentstroke}{rgb}{0.000000,0.000000,0.000000}%
\pgfsetstrokecolor{currentstroke}%
\pgfsetstrokeopacity{0.000000}%
\pgfsetdash{}{0pt}%
\pgfpathmoveto{\pgfqpoint{2.681381in}{0.500000in}}%
\pgfpathlineto{\pgfqpoint{2.707843in}{0.500000in}}%
\pgfpathlineto{\pgfqpoint{2.707843in}{0.804493in}}%
\pgfpathlineto{\pgfqpoint{2.681381in}{0.804493in}}%
\pgfpathlineto{\pgfqpoint{2.681381in}{0.500000in}}%
\pgfpathclose%
\pgfusepath{fill}%
\end{pgfscope}%
\begin{pgfscope}%
\pgfpathrectangle{\pgfqpoint{0.750000in}{0.500000in}}{\pgfqpoint{4.650000in}{3.020000in}}%
\pgfusepath{clip}%
\pgfsetbuttcap%
\pgfsetmiterjoin%
\definecolor{currentfill}{rgb}{0.121569,0.466667,0.705882}%
\pgfsetfillcolor{currentfill}%
\pgfsetlinewidth{0.000000pt}%
\definecolor{currentstroke}{rgb}{0.000000,0.000000,0.000000}%
\pgfsetstrokecolor{currentstroke}%
\pgfsetstrokeopacity{0.000000}%
\pgfsetdash{}{0pt}%
\pgfpathmoveto{\pgfqpoint{2.714458in}{0.500000in}}%
\pgfpathlineto{\pgfqpoint{2.740920in}{0.500000in}}%
\pgfpathlineto{\pgfqpoint{2.740920in}{0.936632in}}%
\pgfpathlineto{\pgfqpoint{2.714458in}{0.936632in}}%
\pgfpathlineto{\pgfqpoint{2.714458in}{0.500000in}}%
\pgfpathclose%
\pgfusepath{fill}%
\end{pgfscope}%
\begin{pgfscope}%
\pgfpathrectangle{\pgfqpoint{0.750000in}{0.500000in}}{\pgfqpoint{4.650000in}{3.020000in}}%
\pgfusepath{clip}%
\pgfsetbuttcap%
\pgfsetmiterjoin%
\definecolor{currentfill}{rgb}{0.121569,0.466667,0.705882}%
\pgfsetfillcolor{currentfill}%
\pgfsetlinewidth{0.000000pt}%
\definecolor{currentstroke}{rgb}{0.000000,0.000000,0.000000}%
\pgfsetstrokecolor{currentstroke}%
\pgfsetstrokeopacity{0.000000}%
\pgfsetdash{}{0pt}%
\pgfpathmoveto{\pgfqpoint{2.747535in}{0.500000in}}%
\pgfpathlineto{\pgfqpoint{2.773997in}{0.500000in}}%
\pgfpathlineto{\pgfqpoint{2.773997in}{1.106433in}}%
\pgfpathlineto{\pgfqpoint{2.747535in}{1.106433in}}%
\pgfpathlineto{\pgfqpoint{2.747535in}{0.500000in}}%
\pgfpathclose%
\pgfusepath{fill}%
\end{pgfscope}%
\begin{pgfscope}%
\pgfpathrectangle{\pgfqpoint{0.750000in}{0.500000in}}{\pgfqpoint{4.650000in}{3.020000in}}%
\pgfusepath{clip}%
\pgfsetbuttcap%
\pgfsetmiterjoin%
\definecolor{currentfill}{rgb}{0.121569,0.466667,0.705882}%
\pgfsetfillcolor{currentfill}%
\pgfsetlinewidth{0.000000pt}%
\definecolor{currentstroke}{rgb}{0.000000,0.000000,0.000000}%
\pgfsetstrokecolor{currentstroke}%
\pgfsetstrokeopacity{0.000000}%
\pgfsetdash{}{0pt}%
\pgfpathmoveto{\pgfqpoint{2.780612in}{0.500000in}}%
\pgfpathlineto{\pgfqpoint{2.807074in}{0.500000in}}%
\pgfpathlineto{\pgfqpoint{2.807074in}{1.315928in}}%
\pgfpathlineto{\pgfqpoint{2.780612in}{1.315928in}}%
\pgfpathlineto{\pgfqpoint{2.780612in}{0.500000in}}%
\pgfpathclose%
\pgfusepath{fill}%
\end{pgfscope}%
\begin{pgfscope}%
\pgfpathrectangle{\pgfqpoint{0.750000in}{0.500000in}}{\pgfqpoint{4.650000in}{3.020000in}}%
\pgfusepath{clip}%
\pgfsetbuttcap%
\pgfsetmiterjoin%
\definecolor{currentfill}{rgb}{0.121569,0.466667,0.705882}%
\pgfsetfillcolor{currentfill}%
\pgfsetlinewidth{0.000000pt}%
\definecolor{currentstroke}{rgb}{0.000000,0.000000,0.000000}%
\pgfsetstrokecolor{currentstroke}%
\pgfsetstrokeopacity{0.000000}%
\pgfsetdash{}{0pt}%
\pgfpathmoveto{\pgfqpoint{2.813690in}{0.500000in}}%
\pgfpathlineto{\pgfqpoint{2.840152in}{0.500000in}}%
\pgfpathlineto{\pgfqpoint{2.840152in}{1.563621in}}%
\pgfpathlineto{\pgfqpoint{2.813690in}{1.563621in}}%
\pgfpathlineto{\pgfqpoint{2.813690in}{0.500000in}}%
\pgfpathclose%
\pgfusepath{fill}%
\end{pgfscope}%
\begin{pgfscope}%
\pgfpathrectangle{\pgfqpoint{0.750000in}{0.500000in}}{\pgfqpoint{4.650000in}{3.020000in}}%
\pgfusepath{clip}%
\pgfsetbuttcap%
\pgfsetmiterjoin%
\definecolor{currentfill}{rgb}{0.121569,0.466667,0.705882}%
\pgfsetfillcolor{currentfill}%
\pgfsetlinewidth{0.000000pt}%
\definecolor{currentstroke}{rgb}{0.000000,0.000000,0.000000}%
\pgfsetstrokecolor{currentstroke}%
\pgfsetstrokeopacity{0.000000}%
\pgfsetdash{}{0pt}%
\pgfpathmoveto{\pgfqpoint{2.846767in}{0.500000in}}%
\pgfpathlineto{\pgfqpoint{2.873229in}{0.500000in}}%
\pgfpathlineto{\pgfqpoint{2.873229in}{1.843521in}}%
\pgfpathlineto{\pgfqpoint{2.846767in}{1.843521in}}%
\pgfpathlineto{\pgfqpoint{2.846767in}{0.500000in}}%
\pgfpathclose%
\pgfusepath{fill}%
\end{pgfscope}%
\begin{pgfscope}%
\pgfpathrectangle{\pgfqpoint{0.750000in}{0.500000in}}{\pgfqpoint{4.650000in}{3.020000in}}%
\pgfusepath{clip}%
\pgfsetbuttcap%
\pgfsetmiterjoin%
\definecolor{currentfill}{rgb}{0.121569,0.466667,0.705882}%
\pgfsetfillcolor{currentfill}%
\pgfsetlinewidth{0.000000pt}%
\definecolor{currentstroke}{rgb}{0.000000,0.000000,0.000000}%
\pgfsetstrokecolor{currentstroke}%
\pgfsetstrokeopacity{0.000000}%
\pgfsetdash{}{0pt}%
\pgfpathmoveto{\pgfqpoint{2.879844in}{0.500000in}}%
\pgfpathlineto{\pgfqpoint{2.906306in}{0.500000in}}%
\pgfpathlineto{\pgfqpoint{2.906306in}{2.144655in}}%
\pgfpathlineto{\pgfqpoint{2.879844in}{2.144655in}}%
\pgfpathlineto{\pgfqpoint{2.879844in}{0.500000in}}%
\pgfpathclose%
\pgfusepath{fill}%
\end{pgfscope}%
\begin{pgfscope}%
\pgfpathrectangle{\pgfqpoint{0.750000in}{0.500000in}}{\pgfqpoint{4.650000in}{3.020000in}}%
\pgfusepath{clip}%
\pgfsetbuttcap%
\pgfsetmiterjoin%
\definecolor{currentfill}{rgb}{0.121569,0.466667,0.705882}%
\pgfsetfillcolor{currentfill}%
\pgfsetlinewidth{0.000000pt}%
\definecolor{currentstroke}{rgb}{0.000000,0.000000,0.000000}%
\pgfsetstrokecolor{currentstroke}%
\pgfsetstrokeopacity{0.000000}%
\pgfsetdash{}{0pt}%
\pgfpathmoveto{\pgfqpoint{2.912921in}{0.500000in}}%
\pgfpathlineto{\pgfqpoint{2.939383in}{0.500000in}}%
\pgfpathlineto{\pgfqpoint{2.939383in}{2.451286in}}%
\pgfpathlineto{\pgfqpoint{2.912921in}{2.451286in}}%
\pgfpathlineto{\pgfqpoint{2.912921in}{0.500000in}}%
\pgfpathclose%
\pgfusepath{fill}%
\end{pgfscope}%
\begin{pgfscope}%
\pgfpathrectangle{\pgfqpoint{0.750000in}{0.500000in}}{\pgfqpoint{4.650000in}{3.020000in}}%
\pgfusepath{clip}%
\pgfsetbuttcap%
\pgfsetmiterjoin%
\definecolor{currentfill}{rgb}{0.121569,0.466667,0.705882}%
\pgfsetfillcolor{currentfill}%
\pgfsetlinewidth{0.000000pt}%
\definecolor{currentstroke}{rgb}{0.000000,0.000000,0.000000}%
\pgfsetstrokecolor{currentstroke}%
\pgfsetstrokeopacity{0.000000}%
\pgfsetdash{}{0pt}%
\pgfpathmoveto{\pgfqpoint{2.945999in}{0.500000in}}%
\pgfpathlineto{\pgfqpoint{2.972461in}{0.500000in}}%
\pgfpathlineto{\pgfqpoint{2.972461in}{2.743978in}}%
\pgfpathlineto{\pgfqpoint{2.945999in}{2.743978in}}%
\pgfpathlineto{\pgfqpoint{2.945999in}{0.500000in}}%
\pgfpathclose%
\pgfusepath{fill}%
\end{pgfscope}%
\begin{pgfscope}%
\pgfpathrectangle{\pgfqpoint{0.750000in}{0.500000in}}{\pgfqpoint{4.650000in}{3.020000in}}%
\pgfusepath{clip}%
\pgfsetbuttcap%
\pgfsetmiterjoin%
\definecolor{currentfill}{rgb}{0.121569,0.466667,0.705882}%
\pgfsetfillcolor{currentfill}%
\pgfsetlinewidth{0.000000pt}%
\definecolor{currentstroke}{rgb}{0.000000,0.000000,0.000000}%
\pgfsetstrokecolor{currentstroke}%
\pgfsetstrokeopacity{0.000000}%
\pgfsetdash{}{0pt}%
\pgfpathmoveto{\pgfqpoint{2.979076in}{0.500000in}}%
\pgfpathlineto{\pgfqpoint{3.005538in}{0.500000in}}%
\pgfpathlineto{\pgfqpoint{3.005538in}{3.001484in}}%
\pgfpathlineto{\pgfqpoint{2.979076in}{3.001484in}}%
\pgfpathlineto{\pgfqpoint{2.979076in}{0.500000in}}%
\pgfpathclose%
\pgfusepath{fill}%
\end{pgfscope}%
\begin{pgfscope}%
\pgfpathrectangle{\pgfqpoint{0.750000in}{0.500000in}}{\pgfqpoint{4.650000in}{3.020000in}}%
\pgfusepath{clip}%
\pgfsetbuttcap%
\pgfsetmiterjoin%
\definecolor{currentfill}{rgb}{0.121569,0.466667,0.705882}%
\pgfsetfillcolor{currentfill}%
\pgfsetlinewidth{0.000000pt}%
\definecolor{currentstroke}{rgb}{0.000000,0.000000,0.000000}%
\pgfsetstrokecolor{currentstroke}%
\pgfsetstrokeopacity{0.000000}%
\pgfsetdash{}{0pt}%
\pgfpathmoveto{\pgfqpoint{3.012153in}{0.500000in}}%
\pgfpathlineto{\pgfqpoint{3.038615in}{0.500000in}}%
\pgfpathlineto{\pgfqpoint{3.038615in}{3.203217in}}%
\pgfpathlineto{\pgfqpoint{3.012153in}{3.203217in}}%
\pgfpathlineto{\pgfqpoint{3.012153in}{0.500000in}}%
\pgfpathclose%
\pgfusepath{fill}%
\end{pgfscope}%
\begin{pgfscope}%
\pgfpathrectangle{\pgfqpoint{0.750000in}{0.500000in}}{\pgfqpoint{4.650000in}{3.020000in}}%
\pgfusepath{clip}%
\pgfsetbuttcap%
\pgfsetmiterjoin%
\definecolor{currentfill}{rgb}{0.121569,0.466667,0.705882}%
\pgfsetfillcolor{currentfill}%
\pgfsetlinewidth{0.000000pt}%
\definecolor{currentstroke}{rgb}{0.000000,0.000000,0.000000}%
\pgfsetstrokecolor{currentstroke}%
\pgfsetstrokeopacity{0.000000}%
\pgfsetdash{}{0pt}%
\pgfpathmoveto{\pgfqpoint{3.045230in}{0.500000in}}%
\pgfpathlineto{\pgfqpoint{3.071692in}{0.500000in}}%
\pgfpathlineto{\pgfqpoint{3.071692in}{3.331941in}}%
\pgfpathlineto{\pgfqpoint{3.045230in}{3.331941in}}%
\pgfpathlineto{\pgfqpoint{3.045230in}{0.500000in}}%
\pgfpathclose%
\pgfusepath{fill}%
\end{pgfscope}%
\begin{pgfscope}%
\pgfpathrectangle{\pgfqpoint{0.750000in}{0.500000in}}{\pgfqpoint{4.650000in}{3.020000in}}%
\pgfusepath{clip}%
\pgfsetbuttcap%
\pgfsetmiterjoin%
\definecolor{currentfill}{rgb}{0.121569,0.466667,0.705882}%
\pgfsetfillcolor{currentfill}%
\pgfsetlinewidth{0.000000pt}%
\definecolor{currentstroke}{rgb}{0.000000,0.000000,0.000000}%
\pgfsetstrokecolor{currentstroke}%
\pgfsetstrokeopacity{0.000000}%
\pgfsetdash{}{0pt}%
\pgfpathmoveto{\pgfqpoint{3.078308in}{0.500000in}}%
\pgfpathlineto{\pgfqpoint{3.104770in}{0.500000in}}%
\pgfpathlineto{\pgfqpoint{3.104770in}{3.376190in}}%
\pgfpathlineto{\pgfqpoint{3.078308in}{3.376190in}}%
\pgfpathlineto{\pgfqpoint{3.078308in}{0.500000in}}%
\pgfpathclose%
\pgfusepath{fill}%
\end{pgfscope}%
\begin{pgfscope}%
\pgfpathrectangle{\pgfqpoint{0.750000in}{0.500000in}}{\pgfqpoint{4.650000in}{3.020000in}}%
\pgfusepath{clip}%
\pgfsetbuttcap%
\pgfsetmiterjoin%
\definecolor{currentfill}{rgb}{0.121569,0.466667,0.705882}%
\pgfsetfillcolor{currentfill}%
\pgfsetlinewidth{0.000000pt}%
\definecolor{currentstroke}{rgb}{0.000000,0.000000,0.000000}%
\pgfsetstrokecolor{currentstroke}%
\pgfsetstrokeopacity{0.000000}%
\pgfsetdash{}{0pt}%
\pgfpathmoveto{\pgfqpoint{3.111385in}{0.500000in}}%
\pgfpathlineto{\pgfqpoint{3.137847in}{0.500000in}}%
\pgfpathlineto{\pgfqpoint{3.137847in}{3.331941in}}%
\pgfpathlineto{\pgfqpoint{3.111385in}{3.331941in}}%
\pgfpathlineto{\pgfqpoint{3.111385in}{0.500000in}}%
\pgfpathclose%
\pgfusepath{fill}%
\end{pgfscope}%
\begin{pgfscope}%
\pgfpathrectangle{\pgfqpoint{0.750000in}{0.500000in}}{\pgfqpoint{4.650000in}{3.020000in}}%
\pgfusepath{clip}%
\pgfsetbuttcap%
\pgfsetmiterjoin%
\definecolor{currentfill}{rgb}{0.121569,0.466667,0.705882}%
\pgfsetfillcolor{currentfill}%
\pgfsetlinewidth{0.000000pt}%
\definecolor{currentstroke}{rgb}{0.000000,0.000000,0.000000}%
\pgfsetstrokecolor{currentstroke}%
\pgfsetstrokeopacity{0.000000}%
\pgfsetdash{}{0pt}%
\pgfpathmoveto{\pgfqpoint{3.144462in}{0.500000in}}%
\pgfpathlineto{\pgfqpoint{3.170924in}{0.500000in}}%
\pgfpathlineto{\pgfqpoint{3.170924in}{3.203217in}}%
\pgfpathlineto{\pgfqpoint{3.144462in}{3.203217in}}%
\pgfpathlineto{\pgfqpoint{3.144462in}{0.500000in}}%
\pgfpathclose%
\pgfusepath{fill}%
\end{pgfscope}%
\begin{pgfscope}%
\pgfpathrectangle{\pgfqpoint{0.750000in}{0.500000in}}{\pgfqpoint{4.650000in}{3.020000in}}%
\pgfusepath{clip}%
\pgfsetbuttcap%
\pgfsetmiterjoin%
\definecolor{currentfill}{rgb}{0.121569,0.466667,0.705882}%
\pgfsetfillcolor{currentfill}%
\pgfsetlinewidth{0.000000pt}%
\definecolor{currentstroke}{rgb}{0.000000,0.000000,0.000000}%
\pgfsetstrokecolor{currentstroke}%
\pgfsetstrokeopacity{0.000000}%
\pgfsetdash{}{0pt}%
\pgfpathmoveto{\pgfqpoint{3.177539in}{0.500000in}}%
\pgfpathlineto{\pgfqpoint{3.204001in}{0.500000in}}%
\pgfpathlineto{\pgfqpoint{3.204001in}{3.001484in}}%
\pgfpathlineto{\pgfqpoint{3.177539in}{3.001484in}}%
\pgfpathlineto{\pgfqpoint{3.177539in}{0.500000in}}%
\pgfpathclose%
\pgfusepath{fill}%
\end{pgfscope}%
\begin{pgfscope}%
\pgfpathrectangle{\pgfqpoint{0.750000in}{0.500000in}}{\pgfqpoint{4.650000in}{3.020000in}}%
\pgfusepath{clip}%
\pgfsetbuttcap%
\pgfsetmiterjoin%
\definecolor{currentfill}{rgb}{0.121569,0.466667,0.705882}%
\pgfsetfillcolor{currentfill}%
\pgfsetlinewidth{0.000000pt}%
\definecolor{currentstroke}{rgb}{0.000000,0.000000,0.000000}%
\pgfsetstrokecolor{currentstroke}%
\pgfsetstrokeopacity{0.000000}%
\pgfsetdash{}{0pt}%
\pgfpathmoveto{\pgfqpoint{3.210617in}{0.500000in}}%
\pgfpathlineto{\pgfqpoint{3.237079in}{0.500000in}}%
\pgfpathlineto{\pgfqpoint{3.237079in}{2.743978in}}%
\pgfpathlineto{\pgfqpoint{3.210617in}{2.743978in}}%
\pgfpathlineto{\pgfqpoint{3.210617in}{0.500000in}}%
\pgfpathclose%
\pgfusepath{fill}%
\end{pgfscope}%
\begin{pgfscope}%
\pgfpathrectangle{\pgfqpoint{0.750000in}{0.500000in}}{\pgfqpoint{4.650000in}{3.020000in}}%
\pgfusepath{clip}%
\pgfsetbuttcap%
\pgfsetmiterjoin%
\definecolor{currentfill}{rgb}{0.121569,0.466667,0.705882}%
\pgfsetfillcolor{currentfill}%
\pgfsetlinewidth{0.000000pt}%
\definecolor{currentstroke}{rgb}{0.000000,0.000000,0.000000}%
\pgfsetstrokecolor{currentstroke}%
\pgfsetstrokeopacity{0.000000}%
\pgfsetdash{}{0pt}%
\pgfpathmoveto{\pgfqpoint{3.243694in}{0.500000in}}%
\pgfpathlineto{\pgfqpoint{3.270156in}{0.500000in}}%
\pgfpathlineto{\pgfqpoint{3.270156in}{2.451286in}}%
\pgfpathlineto{\pgfqpoint{3.243694in}{2.451286in}}%
\pgfpathlineto{\pgfqpoint{3.243694in}{0.500000in}}%
\pgfpathclose%
\pgfusepath{fill}%
\end{pgfscope}%
\begin{pgfscope}%
\pgfpathrectangle{\pgfqpoint{0.750000in}{0.500000in}}{\pgfqpoint{4.650000in}{3.020000in}}%
\pgfusepath{clip}%
\pgfsetbuttcap%
\pgfsetmiterjoin%
\definecolor{currentfill}{rgb}{0.121569,0.466667,0.705882}%
\pgfsetfillcolor{currentfill}%
\pgfsetlinewidth{0.000000pt}%
\definecolor{currentstroke}{rgb}{0.000000,0.000000,0.000000}%
\pgfsetstrokecolor{currentstroke}%
\pgfsetstrokeopacity{0.000000}%
\pgfsetdash{}{0pt}%
\pgfpathmoveto{\pgfqpoint{3.276771in}{0.500000in}}%
\pgfpathlineto{\pgfqpoint{3.303233in}{0.500000in}}%
\pgfpathlineto{\pgfqpoint{3.303233in}{2.144655in}}%
\pgfpathlineto{\pgfqpoint{3.276771in}{2.144655in}}%
\pgfpathlineto{\pgfqpoint{3.276771in}{0.500000in}}%
\pgfpathclose%
\pgfusepath{fill}%
\end{pgfscope}%
\begin{pgfscope}%
\pgfpathrectangle{\pgfqpoint{0.750000in}{0.500000in}}{\pgfqpoint{4.650000in}{3.020000in}}%
\pgfusepath{clip}%
\pgfsetbuttcap%
\pgfsetmiterjoin%
\definecolor{currentfill}{rgb}{0.121569,0.466667,0.705882}%
\pgfsetfillcolor{currentfill}%
\pgfsetlinewidth{0.000000pt}%
\definecolor{currentstroke}{rgb}{0.000000,0.000000,0.000000}%
\pgfsetstrokecolor{currentstroke}%
\pgfsetstrokeopacity{0.000000}%
\pgfsetdash{}{0pt}%
\pgfpathmoveto{\pgfqpoint{3.309848in}{0.500000in}}%
\pgfpathlineto{\pgfqpoint{3.336310in}{0.500000in}}%
\pgfpathlineto{\pgfqpoint{3.336310in}{1.843521in}}%
\pgfpathlineto{\pgfqpoint{3.309848in}{1.843521in}}%
\pgfpathlineto{\pgfqpoint{3.309848in}{0.500000in}}%
\pgfpathclose%
\pgfusepath{fill}%
\end{pgfscope}%
\begin{pgfscope}%
\pgfpathrectangle{\pgfqpoint{0.750000in}{0.500000in}}{\pgfqpoint{4.650000in}{3.020000in}}%
\pgfusepath{clip}%
\pgfsetbuttcap%
\pgfsetmiterjoin%
\definecolor{currentfill}{rgb}{0.121569,0.466667,0.705882}%
\pgfsetfillcolor{currentfill}%
\pgfsetlinewidth{0.000000pt}%
\definecolor{currentstroke}{rgb}{0.000000,0.000000,0.000000}%
\pgfsetstrokecolor{currentstroke}%
\pgfsetstrokeopacity{0.000000}%
\pgfsetdash{}{0pt}%
\pgfpathmoveto{\pgfqpoint{3.342926in}{0.500000in}}%
\pgfpathlineto{\pgfqpoint{3.369388in}{0.500000in}}%
\pgfpathlineto{\pgfqpoint{3.369388in}{1.563621in}}%
\pgfpathlineto{\pgfqpoint{3.342926in}{1.563621in}}%
\pgfpathlineto{\pgfqpoint{3.342926in}{0.500000in}}%
\pgfpathclose%
\pgfusepath{fill}%
\end{pgfscope}%
\begin{pgfscope}%
\pgfpathrectangle{\pgfqpoint{0.750000in}{0.500000in}}{\pgfqpoint{4.650000in}{3.020000in}}%
\pgfusepath{clip}%
\pgfsetbuttcap%
\pgfsetmiterjoin%
\definecolor{currentfill}{rgb}{0.121569,0.466667,0.705882}%
\pgfsetfillcolor{currentfill}%
\pgfsetlinewidth{0.000000pt}%
\definecolor{currentstroke}{rgb}{0.000000,0.000000,0.000000}%
\pgfsetstrokecolor{currentstroke}%
\pgfsetstrokeopacity{0.000000}%
\pgfsetdash{}{0pt}%
\pgfpathmoveto{\pgfqpoint{3.376003in}{0.500000in}}%
\pgfpathlineto{\pgfqpoint{3.402465in}{0.500000in}}%
\pgfpathlineto{\pgfqpoint{3.402465in}{1.315928in}}%
\pgfpathlineto{\pgfqpoint{3.376003in}{1.315928in}}%
\pgfpathlineto{\pgfqpoint{3.376003in}{0.500000in}}%
\pgfpathclose%
\pgfusepath{fill}%
\end{pgfscope}%
\begin{pgfscope}%
\pgfpathrectangle{\pgfqpoint{0.750000in}{0.500000in}}{\pgfqpoint{4.650000in}{3.020000in}}%
\pgfusepath{clip}%
\pgfsetbuttcap%
\pgfsetmiterjoin%
\definecolor{currentfill}{rgb}{0.121569,0.466667,0.705882}%
\pgfsetfillcolor{currentfill}%
\pgfsetlinewidth{0.000000pt}%
\definecolor{currentstroke}{rgb}{0.000000,0.000000,0.000000}%
\pgfsetstrokecolor{currentstroke}%
\pgfsetstrokeopacity{0.000000}%
\pgfsetdash{}{0pt}%
\pgfpathmoveto{\pgfqpoint{3.409080in}{0.500000in}}%
\pgfpathlineto{\pgfqpoint{3.435542in}{0.500000in}}%
\pgfpathlineto{\pgfqpoint{3.435542in}{1.106433in}}%
\pgfpathlineto{\pgfqpoint{3.409080in}{1.106433in}}%
\pgfpathlineto{\pgfqpoint{3.409080in}{0.500000in}}%
\pgfpathclose%
\pgfusepath{fill}%
\end{pgfscope}%
\begin{pgfscope}%
\pgfpathrectangle{\pgfqpoint{0.750000in}{0.500000in}}{\pgfqpoint{4.650000in}{3.020000in}}%
\pgfusepath{clip}%
\pgfsetbuttcap%
\pgfsetmiterjoin%
\definecolor{currentfill}{rgb}{0.121569,0.466667,0.705882}%
\pgfsetfillcolor{currentfill}%
\pgfsetlinewidth{0.000000pt}%
\definecolor{currentstroke}{rgb}{0.000000,0.000000,0.000000}%
\pgfsetstrokecolor{currentstroke}%
\pgfsetstrokeopacity{0.000000}%
\pgfsetdash{}{0pt}%
\pgfpathmoveto{\pgfqpoint{3.442157in}{0.500000in}}%
\pgfpathlineto{\pgfqpoint{3.468619in}{0.500000in}}%
\pgfpathlineto{\pgfqpoint{3.468619in}{0.936632in}}%
\pgfpathlineto{\pgfqpoint{3.442157in}{0.936632in}}%
\pgfpathlineto{\pgfqpoint{3.442157in}{0.500000in}}%
\pgfpathclose%
\pgfusepath{fill}%
\end{pgfscope}%
\begin{pgfscope}%
\pgfpathrectangle{\pgfqpoint{0.750000in}{0.500000in}}{\pgfqpoint{4.650000in}{3.020000in}}%
\pgfusepath{clip}%
\pgfsetbuttcap%
\pgfsetmiterjoin%
\definecolor{currentfill}{rgb}{0.121569,0.466667,0.705882}%
\pgfsetfillcolor{currentfill}%
\pgfsetlinewidth{0.000000pt}%
\definecolor{currentstroke}{rgb}{0.000000,0.000000,0.000000}%
\pgfsetstrokecolor{currentstroke}%
\pgfsetstrokeopacity{0.000000}%
\pgfsetdash{}{0pt}%
\pgfpathmoveto{\pgfqpoint{3.475235in}{0.500000in}}%
\pgfpathlineto{\pgfqpoint{3.501697in}{0.500000in}}%
\pgfpathlineto{\pgfqpoint{3.501697in}{0.804493in}}%
\pgfpathlineto{\pgfqpoint{3.475235in}{0.804493in}}%
\pgfpathlineto{\pgfqpoint{3.475235in}{0.500000in}}%
\pgfpathclose%
\pgfusepath{fill}%
\end{pgfscope}%
\begin{pgfscope}%
\pgfpathrectangle{\pgfqpoint{0.750000in}{0.500000in}}{\pgfqpoint{4.650000in}{3.020000in}}%
\pgfusepath{clip}%
\pgfsetbuttcap%
\pgfsetmiterjoin%
\definecolor{currentfill}{rgb}{0.121569,0.466667,0.705882}%
\pgfsetfillcolor{currentfill}%
\pgfsetlinewidth{0.000000pt}%
\definecolor{currentstroke}{rgb}{0.000000,0.000000,0.000000}%
\pgfsetstrokecolor{currentstroke}%
\pgfsetstrokeopacity{0.000000}%
\pgfsetdash{}{0pt}%
\pgfpathmoveto{\pgfqpoint{3.508312in}{0.500000in}}%
\pgfpathlineto{\pgfqpoint{3.534774in}{0.500000in}}%
\pgfpathlineto{\pgfqpoint{3.534774in}{0.705632in}}%
\pgfpathlineto{\pgfqpoint{3.508312in}{0.705632in}}%
\pgfpathlineto{\pgfqpoint{3.508312in}{0.500000in}}%
\pgfpathclose%
\pgfusepath{fill}%
\end{pgfscope}%
\begin{pgfscope}%
\pgfpathrectangle{\pgfqpoint{0.750000in}{0.500000in}}{\pgfqpoint{4.650000in}{3.020000in}}%
\pgfusepath{clip}%
\pgfsetbuttcap%
\pgfsetmiterjoin%
\definecolor{currentfill}{rgb}{0.121569,0.466667,0.705882}%
\pgfsetfillcolor{currentfill}%
\pgfsetlinewidth{0.000000pt}%
\definecolor{currentstroke}{rgb}{0.000000,0.000000,0.000000}%
\pgfsetstrokecolor{currentstroke}%
\pgfsetstrokeopacity{0.000000}%
\pgfsetdash{}{0pt}%
\pgfpathmoveto{\pgfqpoint{3.541389in}{0.500000in}}%
\pgfpathlineto{\pgfqpoint{3.567851in}{0.500000in}}%
\pgfpathlineto{\pgfqpoint{3.567851in}{0.634452in}}%
\pgfpathlineto{\pgfqpoint{3.541389in}{0.634452in}}%
\pgfpathlineto{\pgfqpoint{3.541389in}{0.500000in}}%
\pgfpathclose%
\pgfusepath{fill}%
\end{pgfscope}%
\begin{pgfscope}%
\pgfpathrectangle{\pgfqpoint{0.750000in}{0.500000in}}{\pgfqpoint{4.650000in}{3.020000in}}%
\pgfusepath{clip}%
\pgfsetbuttcap%
\pgfsetmiterjoin%
\definecolor{currentfill}{rgb}{0.121569,0.466667,0.705882}%
\pgfsetfillcolor{currentfill}%
\pgfsetlinewidth{0.000000pt}%
\definecolor{currentstroke}{rgb}{0.000000,0.000000,0.000000}%
\pgfsetstrokecolor{currentstroke}%
\pgfsetstrokeopacity{0.000000}%
\pgfsetdash{}{0pt}%
\pgfpathmoveto{\pgfqpoint{3.574466in}{0.500000in}}%
\pgfpathlineto{\pgfqpoint{3.600928in}{0.500000in}}%
\pgfpathlineto{\pgfqpoint{3.600928in}{0.585096in}}%
\pgfpathlineto{\pgfqpoint{3.574466in}{0.585096in}}%
\pgfpathlineto{\pgfqpoint{3.574466in}{0.500000in}}%
\pgfpathclose%
\pgfusepath{fill}%
\end{pgfscope}%
\begin{pgfscope}%
\pgfpathrectangle{\pgfqpoint{0.750000in}{0.500000in}}{\pgfqpoint{4.650000in}{3.020000in}}%
\pgfusepath{clip}%
\pgfsetbuttcap%
\pgfsetmiterjoin%
\definecolor{currentfill}{rgb}{0.121569,0.466667,0.705882}%
\pgfsetfillcolor{currentfill}%
\pgfsetlinewidth{0.000000pt}%
\definecolor{currentstroke}{rgb}{0.000000,0.000000,0.000000}%
\pgfsetstrokecolor{currentstroke}%
\pgfsetstrokeopacity{0.000000}%
\pgfsetdash{}{0pt}%
\pgfpathmoveto{\pgfqpoint{3.607544in}{0.500000in}}%
\pgfpathlineto{\pgfqpoint{3.634006in}{0.500000in}}%
\pgfpathlineto{\pgfqpoint{3.634006in}{0.552121in}}%
\pgfpathlineto{\pgfqpoint{3.607544in}{0.552121in}}%
\pgfpathlineto{\pgfqpoint{3.607544in}{0.500000in}}%
\pgfpathclose%
\pgfusepath{fill}%
\end{pgfscope}%
\begin{pgfscope}%
\pgfpathrectangle{\pgfqpoint{0.750000in}{0.500000in}}{\pgfqpoint{4.650000in}{3.020000in}}%
\pgfusepath{clip}%
\pgfsetbuttcap%
\pgfsetmiterjoin%
\definecolor{currentfill}{rgb}{0.121569,0.466667,0.705882}%
\pgfsetfillcolor{currentfill}%
\pgfsetlinewidth{0.000000pt}%
\definecolor{currentstroke}{rgb}{0.000000,0.000000,0.000000}%
\pgfsetstrokecolor{currentstroke}%
\pgfsetstrokeopacity{0.000000}%
\pgfsetdash{}{0pt}%
\pgfpathmoveto{\pgfqpoint{3.640621in}{0.500000in}}%
\pgfpathlineto{\pgfqpoint{3.667083in}{0.500000in}}%
\pgfpathlineto{\pgfqpoint{3.667083in}{0.530887in}}%
\pgfpathlineto{\pgfqpoint{3.640621in}{0.530887in}}%
\pgfpathlineto{\pgfqpoint{3.640621in}{0.500000in}}%
\pgfpathclose%
\pgfusepath{fill}%
\end{pgfscope}%
\begin{pgfscope}%
\pgfpathrectangle{\pgfqpoint{0.750000in}{0.500000in}}{\pgfqpoint{4.650000in}{3.020000in}}%
\pgfusepath{clip}%
\pgfsetbuttcap%
\pgfsetmiterjoin%
\definecolor{currentfill}{rgb}{0.121569,0.466667,0.705882}%
\pgfsetfillcolor{currentfill}%
\pgfsetlinewidth{0.000000pt}%
\definecolor{currentstroke}{rgb}{0.000000,0.000000,0.000000}%
\pgfsetstrokecolor{currentstroke}%
\pgfsetstrokeopacity{0.000000}%
\pgfsetdash{}{0pt}%
\pgfpathmoveto{\pgfqpoint{3.673698in}{0.500000in}}%
\pgfpathlineto{\pgfqpoint{3.700160in}{0.500000in}}%
\pgfpathlineto{\pgfqpoint{3.700160in}{0.517703in}}%
\pgfpathlineto{\pgfqpoint{3.673698in}{0.517703in}}%
\pgfpathlineto{\pgfqpoint{3.673698in}{0.500000in}}%
\pgfpathclose%
\pgfusepath{fill}%
\end{pgfscope}%
\begin{pgfscope}%
\pgfpathrectangle{\pgfqpoint{0.750000in}{0.500000in}}{\pgfqpoint{4.650000in}{3.020000in}}%
\pgfusepath{clip}%
\pgfsetbuttcap%
\pgfsetmiterjoin%
\definecolor{currentfill}{rgb}{0.121569,0.466667,0.705882}%
\pgfsetfillcolor{currentfill}%
\pgfsetlinewidth{0.000000pt}%
\definecolor{currentstroke}{rgb}{0.000000,0.000000,0.000000}%
\pgfsetstrokecolor{currentstroke}%
\pgfsetstrokeopacity{0.000000}%
\pgfsetdash{}{0pt}%
\pgfpathmoveto{\pgfqpoint{3.706776in}{0.500000in}}%
\pgfpathlineto{\pgfqpoint{3.733237in}{0.500000in}}%
\pgfpathlineto{\pgfqpoint{3.733237in}{0.509811in}}%
\pgfpathlineto{\pgfqpoint{3.706776in}{0.509811in}}%
\pgfpathlineto{\pgfqpoint{3.706776in}{0.500000in}}%
\pgfpathclose%
\pgfusepath{fill}%
\end{pgfscope}%
\begin{pgfscope}%
\pgfpathrectangle{\pgfqpoint{0.750000in}{0.500000in}}{\pgfqpoint{4.650000in}{3.020000in}}%
\pgfusepath{clip}%
\pgfsetbuttcap%
\pgfsetmiterjoin%
\definecolor{currentfill}{rgb}{0.121569,0.466667,0.705882}%
\pgfsetfillcolor{currentfill}%
\pgfsetlinewidth{0.000000pt}%
\definecolor{currentstroke}{rgb}{0.000000,0.000000,0.000000}%
\pgfsetstrokecolor{currentstroke}%
\pgfsetstrokeopacity{0.000000}%
\pgfsetdash{}{0pt}%
\pgfpathmoveto{\pgfqpoint{3.739853in}{0.500000in}}%
\pgfpathlineto{\pgfqpoint{3.766315in}{0.500000in}}%
\pgfpathlineto{\pgfqpoint{3.766315in}{0.505256in}}%
\pgfpathlineto{\pgfqpoint{3.739853in}{0.505256in}}%
\pgfpathlineto{\pgfqpoint{3.739853in}{0.500000in}}%
\pgfpathclose%
\pgfusepath{fill}%
\end{pgfscope}%
\begin{pgfscope}%
\pgfpathrectangle{\pgfqpoint{0.750000in}{0.500000in}}{\pgfqpoint{4.650000in}{3.020000in}}%
\pgfusepath{clip}%
\pgfsetbuttcap%
\pgfsetmiterjoin%
\definecolor{currentfill}{rgb}{0.121569,0.466667,0.705882}%
\pgfsetfillcolor{currentfill}%
\pgfsetlinewidth{0.000000pt}%
\definecolor{currentstroke}{rgb}{0.000000,0.000000,0.000000}%
\pgfsetstrokecolor{currentstroke}%
\pgfsetstrokeopacity{0.000000}%
\pgfsetdash{}{0pt}%
\pgfpathmoveto{\pgfqpoint{3.772930in}{0.500000in}}%
\pgfpathlineto{\pgfqpoint{3.799392in}{0.500000in}}%
\pgfpathlineto{\pgfqpoint{3.799392in}{0.502721in}}%
\pgfpathlineto{\pgfqpoint{3.772930in}{0.502721in}}%
\pgfpathlineto{\pgfqpoint{3.772930in}{0.500000in}}%
\pgfpathclose%
\pgfusepath{fill}%
\end{pgfscope}%
\begin{pgfscope}%
\pgfpathrectangle{\pgfqpoint{0.750000in}{0.500000in}}{\pgfqpoint{4.650000in}{3.020000in}}%
\pgfusepath{clip}%
\pgfsetbuttcap%
\pgfsetmiterjoin%
\definecolor{currentfill}{rgb}{0.121569,0.466667,0.705882}%
\pgfsetfillcolor{currentfill}%
\pgfsetlinewidth{0.000000pt}%
\definecolor{currentstroke}{rgb}{0.000000,0.000000,0.000000}%
\pgfsetstrokecolor{currentstroke}%
\pgfsetstrokeopacity{0.000000}%
\pgfsetdash{}{0pt}%
\pgfpathmoveto{\pgfqpoint{3.806007in}{0.500000in}}%
\pgfpathlineto{\pgfqpoint{3.832469in}{0.500000in}}%
\pgfpathlineto{\pgfqpoint{3.832469in}{0.501360in}}%
\pgfpathlineto{\pgfqpoint{3.806007in}{0.501360in}}%
\pgfpathlineto{\pgfqpoint{3.806007in}{0.500000in}}%
\pgfpathclose%
\pgfusepath{fill}%
\end{pgfscope}%
\begin{pgfscope}%
\pgfpathrectangle{\pgfqpoint{0.750000in}{0.500000in}}{\pgfqpoint{4.650000in}{3.020000in}}%
\pgfusepath{clip}%
\pgfsetbuttcap%
\pgfsetmiterjoin%
\definecolor{currentfill}{rgb}{0.121569,0.466667,0.705882}%
\pgfsetfillcolor{currentfill}%
\pgfsetlinewidth{0.000000pt}%
\definecolor{currentstroke}{rgb}{0.000000,0.000000,0.000000}%
\pgfsetstrokecolor{currentstroke}%
\pgfsetstrokeopacity{0.000000}%
\pgfsetdash{}{0pt}%
\pgfpathmoveto{\pgfqpoint{3.839085in}{0.500000in}}%
\pgfpathlineto{\pgfqpoint{3.865546in}{0.500000in}}%
\pgfpathlineto{\pgfqpoint{3.865546in}{0.500657in}}%
\pgfpathlineto{\pgfqpoint{3.839085in}{0.500657in}}%
\pgfpathlineto{\pgfqpoint{3.839085in}{0.500000in}}%
\pgfpathclose%
\pgfusepath{fill}%
\end{pgfscope}%
\begin{pgfscope}%
\pgfpathrectangle{\pgfqpoint{0.750000in}{0.500000in}}{\pgfqpoint{4.650000in}{3.020000in}}%
\pgfusepath{clip}%
\pgfsetbuttcap%
\pgfsetmiterjoin%
\definecolor{currentfill}{rgb}{0.121569,0.466667,0.705882}%
\pgfsetfillcolor{currentfill}%
\pgfsetlinewidth{0.000000pt}%
\definecolor{currentstroke}{rgb}{0.000000,0.000000,0.000000}%
\pgfsetstrokecolor{currentstroke}%
\pgfsetstrokeopacity{0.000000}%
\pgfsetdash{}{0pt}%
\pgfpathmoveto{\pgfqpoint{3.872162in}{0.500000in}}%
\pgfpathlineto{\pgfqpoint{3.898624in}{0.500000in}}%
\pgfpathlineto{\pgfqpoint{3.898624in}{0.500306in}}%
\pgfpathlineto{\pgfqpoint{3.872162in}{0.500306in}}%
\pgfpathlineto{\pgfqpoint{3.872162in}{0.500000in}}%
\pgfpathclose%
\pgfusepath{fill}%
\end{pgfscope}%
\begin{pgfscope}%
\pgfpathrectangle{\pgfqpoint{0.750000in}{0.500000in}}{\pgfqpoint{4.650000in}{3.020000in}}%
\pgfusepath{clip}%
\pgfsetbuttcap%
\pgfsetmiterjoin%
\definecolor{currentfill}{rgb}{0.121569,0.466667,0.705882}%
\pgfsetfillcolor{currentfill}%
\pgfsetlinewidth{0.000000pt}%
\definecolor{currentstroke}{rgb}{0.000000,0.000000,0.000000}%
\pgfsetstrokecolor{currentstroke}%
\pgfsetstrokeopacity{0.000000}%
\pgfsetdash{}{0pt}%
\pgfpathmoveto{\pgfqpoint{3.905239in}{0.500000in}}%
\pgfpathlineto{\pgfqpoint{3.931701in}{0.500000in}}%
\pgfpathlineto{\pgfqpoint{3.931701in}{0.500138in}}%
\pgfpathlineto{\pgfqpoint{3.905239in}{0.500138in}}%
\pgfpathlineto{\pgfqpoint{3.905239in}{0.500000in}}%
\pgfpathclose%
\pgfusepath{fill}%
\end{pgfscope}%
\begin{pgfscope}%
\pgfpathrectangle{\pgfqpoint{0.750000in}{0.500000in}}{\pgfqpoint{4.650000in}{3.020000in}}%
\pgfusepath{clip}%
\pgfsetbuttcap%
\pgfsetmiterjoin%
\definecolor{currentfill}{rgb}{0.121569,0.466667,0.705882}%
\pgfsetfillcolor{currentfill}%
\pgfsetlinewidth{0.000000pt}%
\definecolor{currentstroke}{rgb}{0.000000,0.000000,0.000000}%
\pgfsetstrokecolor{currentstroke}%
\pgfsetstrokeopacity{0.000000}%
\pgfsetdash{}{0pt}%
\pgfpathmoveto{\pgfqpoint{3.938316in}{0.500000in}}%
\pgfpathlineto{\pgfqpoint{3.964778in}{0.500000in}}%
\pgfpathlineto{\pgfqpoint{3.964778in}{0.500060in}}%
\pgfpathlineto{\pgfqpoint{3.938316in}{0.500060in}}%
\pgfpathlineto{\pgfqpoint{3.938316in}{0.500000in}}%
\pgfpathclose%
\pgfusepath{fill}%
\end{pgfscope}%
\begin{pgfscope}%
\pgfpathrectangle{\pgfqpoint{0.750000in}{0.500000in}}{\pgfqpoint{4.650000in}{3.020000in}}%
\pgfusepath{clip}%
\pgfsetbuttcap%
\pgfsetmiterjoin%
\definecolor{currentfill}{rgb}{0.121569,0.466667,0.705882}%
\pgfsetfillcolor{currentfill}%
\pgfsetlinewidth{0.000000pt}%
\definecolor{currentstroke}{rgb}{0.000000,0.000000,0.000000}%
\pgfsetstrokecolor{currentstroke}%
\pgfsetstrokeopacity{0.000000}%
\pgfsetdash{}{0pt}%
\pgfpathmoveto{\pgfqpoint{3.971394in}{0.500000in}}%
\pgfpathlineto{\pgfqpoint{3.997855in}{0.500000in}}%
\pgfpathlineto{\pgfqpoint{3.997855in}{0.500025in}}%
\pgfpathlineto{\pgfqpoint{3.971394in}{0.500025in}}%
\pgfpathlineto{\pgfqpoint{3.971394in}{0.500000in}}%
\pgfpathclose%
\pgfusepath{fill}%
\end{pgfscope}%
\begin{pgfscope}%
\pgfpathrectangle{\pgfqpoint{0.750000in}{0.500000in}}{\pgfqpoint{4.650000in}{3.020000in}}%
\pgfusepath{clip}%
\pgfsetbuttcap%
\pgfsetmiterjoin%
\definecolor{currentfill}{rgb}{0.121569,0.466667,0.705882}%
\pgfsetfillcolor{currentfill}%
\pgfsetlinewidth{0.000000pt}%
\definecolor{currentstroke}{rgb}{0.000000,0.000000,0.000000}%
\pgfsetstrokecolor{currentstroke}%
\pgfsetstrokeopacity{0.000000}%
\pgfsetdash{}{0pt}%
\pgfpathmoveto{\pgfqpoint{4.004471in}{0.500000in}}%
\pgfpathlineto{\pgfqpoint{4.030933in}{0.500000in}}%
\pgfpathlineto{\pgfqpoint{4.030933in}{0.500010in}}%
\pgfpathlineto{\pgfqpoint{4.004471in}{0.500010in}}%
\pgfpathlineto{\pgfqpoint{4.004471in}{0.500000in}}%
\pgfpathclose%
\pgfusepath{fill}%
\end{pgfscope}%
\begin{pgfscope}%
\pgfpathrectangle{\pgfqpoint{0.750000in}{0.500000in}}{\pgfqpoint{4.650000in}{3.020000in}}%
\pgfusepath{clip}%
\pgfsetbuttcap%
\pgfsetmiterjoin%
\definecolor{currentfill}{rgb}{0.121569,0.466667,0.705882}%
\pgfsetfillcolor{currentfill}%
\pgfsetlinewidth{0.000000pt}%
\definecolor{currentstroke}{rgb}{0.000000,0.000000,0.000000}%
\pgfsetstrokecolor{currentstroke}%
\pgfsetstrokeopacity{0.000000}%
\pgfsetdash{}{0pt}%
\pgfpathmoveto{\pgfqpoint{4.037548in}{0.500000in}}%
\pgfpathlineto{\pgfqpoint{4.064010in}{0.500000in}}%
\pgfpathlineto{\pgfqpoint{4.064010in}{0.500004in}}%
\pgfpathlineto{\pgfqpoint{4.037548in}{0.500004in}}%
\pgfpathlineto{\pgfqpoint{4.037548in}{0.500000in}}%
\pgfpathclose%
\pgfusepath{fill}%
\end{pgfscope}%
\begin{pgfscope}%
\pgfpathrectangle{\pgfqpoint{0.750000in}{0.500000in}}{\pgfqpoint{4.650000in}{3.020000in}}%
\pgfusepath{clip}%
\pgfsetbuttcap%
\pgfsetmiterjoin%
\definecolor{currentfill}{rgb}{0.121569,0.466667,0.705882}%
\pgfsetfillcolor{currentfill}%
\pgfsetlinewidth{0.000000pt}%
\definecolor{currentstroke}{rgb}{0.000000,0.000000,0.000000}%
\pgfsetstrokecolor{currentstroke}%
\pgfsetstrokeopacity{0.000000}%
\pgfsetdash{}{0pt}%
\pgfpathmoveto{\pgfqpoint{4.070625in}{0.500000in}}%
\pgfpathlineto{\pgfqpoint{4.097087in}{0.500000in}}%
\pgfpathlineto{\pgfqpoint{4.097087in}{0.500001in}}%
\pgfpathlineto{\pgfqpoint{4.070625in}{0.500001in}}%
\pgfpathlineto{\pgfqpoint{4.070625in}{0.500000in}}%
\pgfpathclose%
\pgfusepath{fill}%
\end{pgfscope}%
\begin{pgfscope}%
\pgfpathrectangle{\pgfqpoint{0.750000in}{0.500000in}}{\pgfqpoint{4.650000in}{3.020000in}}%
\pgfusepath{clip}%
\pgfsetbuttcap%
\pgfsetmiterjoin%
\definecolor{currentfill}{rgb}{0.121569,0.466667,0.705882}%
\pgfsetfillcolor{currentfill}%
\pgfsetlinewidth{0.000000pt}%
\definecolor{currentstroke}{rgb}{0.000000,0.000000,0.000000}%
\pgfsetstrokecolor{currentstroke}%
\pgfsetstrokeopacity{0.000000}%
\pgfsetdash{}{0pt}%
\pgfpathmoveto{\pgfqpoint{4.103703in}{0.500000in}}%
\pgfpathlineto{\pgfqpoint{4.130164in}{0.500000in}}%
\pgfpathlineto{\pgfqpoint{4.130164in}{0.500001in}}%
\pgfpathlineto{\pgfqpoint{4.103703in}{0.500001in}}%
\pgfpathlineto{\pgfqpoint{4.103703in}{0.500000in}}%
\pgfpathclose%
\pgfusepath{fill}%
\end{pgfscope}%
\begin{pgfscope}%
\pgfpathrectangle{\pgfqpoint{0.750000in}{0.500000in}}{\pgfqpoint{4.650000in}{3.020000in}}%
\pgfusepath{clip}%
\pgfsetbuttcap%
\pgfsetmiterjoin%
\definecolor{currentfill}{rgb}{0.121569,0.466667,0.705882}%
\pgfsetfillcolor{currentfill}%
\pgfsetlinewidth{0.000000pt}%
\definecolor{currentstroke}{rgb}{0.000000,0.000000,0.000000}%
\pgfsetstrokecolor{currentstroke}%
\pgfsetstrokeopacity{0.000000}%
\pgfsetdash{}{0pt}%
\pgfpathmoveto{\pgfqpoint{4.136780in}{0.500000in}}%
\pgfpathlineto{\pgfqpoint{4.163242in}{0.500000in}}%
\pgfpathlineto{\pgfqpoint{4.163242in}{0.500000in}}%
\pgfpathlineto{\pgfqpoint{4.136780in}{0.500000in}}%
\pgfpathlineto{\pgfqpoint{4.136780in}{0.500000in}}%
\pgfpathclose%
\pgfusepath{fill}%
\end{pgfscope}%
\begin{pgfscope}%
\pgfpathrectangle{\pgfqpoint{0.750000in}{0.500000in}}{\pgfqpoint{4.650000in}{3.020000in}}%
\pgfusepath{clip}%
\pgfsetbuttcap%
\pgfsetmiterjoin%
\definecolor{currentfill}{rgb}{0.121569,0.466667,0.705882}%
\pgfsetfillcolor{currentfill}%
\pgfsetlinewidth{0.000000pt}%
\definecolor{currentstroke}{rgb}{0.000000,0.000000,0.000000}%
\pgfsetstrokecolor{currentstroke}%
\pgfsetstrokeopacity{0.000000}%
\pgfsetdash{}{0pt}%
\pgfpathmoveto{\pgfqpoint{4.169857in}{0.500000in}}%
\pgfpathlineto{\pgfqpoint{4.196319in}{0.500000in}}%
\pgfpathlineto{\pgfqpoint{4.196319in}{0.500000in}}%
\pgfpathlineto{\pgfqpoint{4.169857in}{0.500000in}}%
\pgfpathlineto{\pgfqpoint{4.169857in}{0.500000in}}%
\pgfpathclose%
\pgfusepath{fill}%
\end{pgfscope}%
\begin{pgfscope}%
\pgfpathrectangle{\pgfqpoint{0.750000in}{0.500000in}}{\pgfqpoint{4.650000in}{3.020000in}}%
\pgfusepath{clip}%
\pgfsetbuttcap%
\pgfsetmiterjoin%
\definecolor{currentfill}{rgb}{0.121569,0.466667,0.705882}%
\pgfsetfillcolor{currentfill}%
\pgfsetlinewidth{0.000000pt}%
\definecolor{currentstroke}{rgb}{0.000000,0.000000,0.000000}%
\pgfsetstrokecolor{currentstroke}%
\pgfsetstrokeopacity{0.000000}%
\pgfsetdash{}{0pt}%
\pgfpathmoveto{\pgfqpoint{4.202934in}{0.500000in}}%
\pgfpathlineto{\pgfqpoint{4.229396in}{0.500000in}}%
\pgfpathlineto{\pgfqpoint{4.229396in}{0.500000in}}%
\pgfpathlineto{\pgfqpoint{4.202934in}{0.500000in}}%
\pgfpathlineto{\pgfqpoint{4.202934in}{0.500000in}}%
\pgfpathclose%
\pgfusepath{fill}%
\end{pgfscope}%
\begin{pgfscope}%
\pgfpathrectangle{\pgfqpoint{0.750000in}{0.500000in}}{\pgfqpoint{4.650000in}{3.020000in}}%
\pgfusepath{clip}%
\pgfsetbuttcap%
\pgfsetmiterjoin%
\definecolor{currentfill}{rgb}{0.121569,0.466667,0.705882}%
\pgfsetfillcolor{currentfill}%
\pgfsetlinewidth{0.000000pt}%
\definecolor{currentstroke}{rgb}{0.000000,0.000000,0.000000}%
\pgfsetstrokecolor{currentstroke}%
\pgfsetstrokeopacity{0.000000}%
\pgfsetdash{}{0pt}%
\pgfpathmoveto{\pgfqpoint{4.236012in}{0.500000in}}%
\pgfpathlineto{\pgfqpoint{4.262473in}{0.500000in}}%
\pgfpathlineto{\pgfqpoint{4.262473in}{0.500000in}}%
\pgfpathlineto{\pgfqpoint{4.236012in}{0.500000in}}%
\pgfpathlineto{\pgfqpoint{4.236012in}{0.500000in}}%
\pgfpathclose%
\pgfusepath{fill}%
\end{pgfscope}%
\begin{pgfscope}%
\pgfpathrectangle{\pgfqpoint{0.750000in}{0.500000in}}{\pgfqpoint{4.650000in}{3.020000in}}%
\pgfusepath{clip}%
\pgfsetbuttcap%
\pgfsetmiterjoin%
\definecolor{currentfill}{rgb}{0.121569,0.466667,0.705882}%
\pgfsetfillcolor{currentfill}%
\pgfsetlinewidth{0.000000pt}%
\definecolor{currentstroke}{rgb}{0.000000,0.000000,0.000000}%
\pgfsetstrokecolor{currentstroke}%
\pgfsetstrokeopacity{0.000000}%
\pgfsetdash{}{0pt}%
\pgfpathmoveto{\pgfqpoint{4.269089in}{0.500000in}}%
\pgfpathlineto{\pgfqpoint{4.295551in}{0.500000in}}%
\pgfpathlineto{\pgfqpoint{4.295551in}{0.500000in}}%
\pgfpathlineto{\pgfqpoint{4.269089in}{0.500000in}}%
\pgfpathlineto{\pgfqpoint{4.269089in}{0.500000in}}%
\pgfpathclose%
\pgfusepath{fill}%
\end{pgfscope}%
\begin{pgfscope}%
\pgfpathrectangle{\pgfqpoint{0.750000in}{0.500000in}}{\pgfqpoint{4.650000in}{3.020000in}}%
\pgfusepath{clip}%
\pgfsetbuttcap%
\pgfsetmiterjoin%
\definecolor{currentfill}{rgb}{0.121569,0.466667,0.705882}%
\pgfsetfillcolor{currentfill}%
\pgfsetlinewidth{0.000000pt}%
\definecolor{currentstroke}{rgb}{0.000000,0.000000,0.000000}%
\pgfsetstrokecolor{currentstroke}%
\pgfsetstrokeopacity{0.000000}%
\pgfsetdash{}{0pt}%
\pgfpathmoveto{\pgfqpoint{4.302166in}{0.500000in}}%
\pgfpathlineto{\pgfqpoint{4.328628in}{0.500000in}}%
\pgfpathlineto{\pgfqpoint{4.328628in}{0.500000in}}%
\pgfpathlineto{\pgfqpoint{4.302166in}{0.500000in}}%
\pgfpathlineto{\pgfqpoint{4.302166in}{0.500000in}}%
\pgfpathclose%
\pgfusepath{fill}%
\end{pgfscope}%
\begin{pgfscope}%
\pgfpathrectangle{\pgfqpoint{0.750000in}{0.500000in}}{\pgfqpoint{4.650000in}{3.020000in}}%
\pgfusepath{clip}%
\pgfsetbuttcap%
\pgfsetmiterjoin%
\definecolor{currentfill}{rgb}{0.121569,0.466667,0.705882}%
\pgfsetfillcolor{currentfill}%
\pgfsetlinewidth{0.000000pt}%
\definecolor{currentstroke}{rgb}{0.000000,0.000000,0.000000}%
\pgfsetstrokecolor{currentstroke}%
\pgfsetstrokeopacity{0.000000}%
\pgfsetdash{}{0pt}%
\pgfpathmoveto{\pgfqpoint{4.335243in}{0.500000in}}%
\pgfpathlineto{\pgfqpoint{4.361705in}{0.500000in}}%
\pgfpathlineto{\pgfqpoint{4.361705in}{0.500000in}}%
\pgfpathlineto{\pgfqpoint{4.335243in}{0.500000in}}%
\pgfpathlineto{\pgfqpoint{4.335243in}{0.500000in}}%
\pgfpathclose%
\pgfusepath{fill}%
\end{pgfscope}%
\begin{pgfscope}%
\pgfpathrectangle{\pgfqpoint{0.750000in}{0.500000in}}{\pgfqpoint{4.650000in}{3.020000in}}%
\pgfusepath{clip}%
\pgfsetbuttcap%
\pgfsetmiterjoin%
\definecolor{currentfill}{rgb}{0.121569,0.466667,0.705882}%
\pgfsetfillcolor{currentfill}%
\pgfsetlinewidth{0.000000pt}%
\definecolor{currentstroke}{rgb}{0.000000,0.000000,0.000000}%
\pgfsetstrokecolor{currentstroke}%
\pgfsetstrokeopacity{0.000000}%
\pgfsetdash{}{0pt}%
\pgfpathmoveto{\pgfqpoint{4.368321in}{0.500000in}}%
\pgfpathlineto{\pgfqpoint{4.394782in}{0.500000in}}%
\pgfpathlineto{\pgfqpoint{4.394782in}{0.500000in}}%
\pgfpathlineto{\pgfqpoint{4.368321in}{0.500000in}}%
\pgfpathlineto{\pgfqpoint{4.368321in}{0.500000in}}%
\pgfpathclose%
\pgfusepath{fill}%
\end{pgfscope}%
\begin{pgfscope}%
\pgfpathrectangle{\pgfqpoint{0.750000in}{0.500000in}}{\pgfqpoint{4.650000in}{3.020000in}}%
\pgfusepath{clip}%
\pgfsetbuttcap%
\pgfsetmiterjoin%
\definecolor{currentfill}{rgb}{0.121569,0.466667,0.705882}%
\pgfsetfillcolor{currentfill}%
\pgfsetlinewidth{0.000000pt}%
\definecolor{currentstroke}{rgb}{0.000000,0.000000,0.000000}%
\pgfsetstrokecolor{currentstroke}%
\pgfsetstrokeopacity{0.000000}%
\pgfsetdash{}{0pt}%
\pgfpathmoveto{\pgfqpoint{4.401398in}{0.500000in}}%
\pgfpathlineto{\pgfqpoint{4.427860in}{0.500000in}}%
\pgfpathlineto{\pgfqpoint{4.427860in}{0.500000in}}%
\pgfpathlineto{\pgfqpoint{4.401398in}{0.500000in}}%
\pgfpathlineto{\pgfqpoint{4.401398in}{0.500000in}}%
\pgfpathclose%
\pgfusepath{fill}%
\end{pgfscope}%
\begin{pgfscope}%
\pgfpathrectangle{\pgfqpoint{0.750000in}{0.500000in}}{\pgfqpoint{4.650000in}{3.020000in}}%
\pgfusepath{clip}%
\pgfsetbuttcap%
\pgfsetmiterjoin%
\definecolor{currentfill}{rgb}{0.121569,0.466667,0.705882}%
\pgfsetfillcolor{currentfill}%
\pgfsetlinewidth{0.000000pt}%
\definecolor{currentstroke}{rgb}{0.000000,0.000000,0.000000}%
\pgfsetstrokecolor{currentstroke}%
\pgfsetstrokeopacity{0.000000}%
\pgfsetdash{}{0pt}%
\pgfpathmoveto{\pgfqpoint{4.434475in}{0.500000in}}%
\pgfpathlineto{\pgfqpoint{4.460937in}{0.500000in}}%
\pgfpathlineto{\pgfqpoint{4.460937in}{0.500000in}}%
\pgfpathlineto{\pgfqpoint{4.434475in}{0.500000in}}%
\pgfpathlineto{\pgfqpoint{4.434475in}{0.500000in}}%
\pgfpathclose%
\pgfusepath{fill}%
\end{pgfscope}%
\begin{pgfscope}%
\pgfpathrectangle{\pgfqpoint{0.750000in}{0.500000in}}{\pgfqpoint{4.650000in}{3.020000in}}%
\pgfusepath{clip}%
\pgfsetbuttcap%
\pgfsetmiterjoin%
\definecolor{currentfill}{rgb}{0.121569,0.466667,0.705882}%
\pgfsetfillcolor{currentfill}%
\pgfsetlinewidth{0.000000pt}%
\definecolor{currentstroke}{rgb}{0.000000,0.000000,0.000000}%
\pgfsetstrokecolor{currentstroke}%
\pgfsetstrokeopacity{0.000000}%
\pgfsetdash{}{0pt}%
\pgfpathmoveto{\pgfqpoint{4.467552in}{0.500000in}}%
\pgfpathlineto{\pgfqpoint{4.494014in}{0.500000in}}%
\pgfpathlineto{\pgfqpoint{4.494014in}{0.500000in}}%
\pgfpathlineto{\pgfqpoint{4.467552in}{0.500000in}}%
\pgfpathlineto{\pgfqpoint{4.467552in}{0.500000in}}%
\pgfpathclose%
\pgfusepath{fill}%
\end{pgfscope}%
\begin{pgfscope}%
\pgfpathrectangle{\pgfqpoint{0.750000in}{0.500000in}}{\pgfqpoint{4.650000in}{3.020000in}}%
\pgfusepath{clip}%
\pgfsetbuttcap%
\pgfsetmiterjoin%
\definecolor{currentfill}{rgb}{0.121569,0.466667,0.705882}%
\pgfsetfillcolor{currentfill}%
\pgfsetlinewidth{0.000000pt}%
\definecolor{currentstroke}{rgb}{0.000000,0.000000,0.000000}%
\pgfsetstrokecolor{currentstroke}%
\pgfsetstrokeopacity{0.000000}%
\pgfsetdash{}{0pt}%
\pgfpathmoveto{\pgfqpoint{4.500630in}{0.500000in}}%
\pgfpathlineto{\pgfqpoint{4.527091in}{0.500000in}}%
\pgfpathlineto{\pgfqpoint{4.527091in}{0.500000in}}%
\pgfpathlineto{\pgfqpoint{4.500630in}{0.500000in}}%
\pgfpathlineto{\pgfqpoint{4.500630in}{0.500000in}}%
\pgfpathclose%
\pgfusepath{fill}%
\end{pgfscope}%
\begin{pgfscope}%
\pgfpathrectangle{\pgfqpoint{0.750000in}{0.500000in}}{\pgfqpoint{4.650000in}{3.020000in}}%
\pgfusepath{clip}%
\pgfsetbuttcap%
\pgfsetmiterjoin%
\definecolor{currentfill}{rgb}{0.121569,0.466667,0.705882}%
\pgfsetfillcolor{currentfill}%
\pgfsetlinewidth{0.000000pt}%
\definecolor{currentstroke}{rgb}{0.000000,0.000000,0.000000}%
\pgfsetstrokecolor{currentstroke}%
\pgfsetstrokeopacity{0.000000}%
\pgfsetdash{}{0pt}%
\pgfpathmoveto{\pgfqpoint{4.533707in}{0.500000in}}%
\pgfpathlineto{\pgfqpoint{4.560169in}{0.500000in}}%
\pgfpathlineto{\pgfqpoint{4.560169in}{0.500000in}}%
\pgfpathlineto{\pgfqpoint{4.533707in}{0.500000in}}%
\pgfpathlineto{\pgfqpoint{4.533707in}{0.500000in}}%
\pgfpathclose%
\pgfusepath{fill}%
\end{pgfscope}%
\begin{pgfscope}%
\pgfpathrectangle{\pgfqpoint{0.750000in}{0.500000in}}{\pgfqpoint{4.650000in}{3.020000in}}%
\pgfusepath{clip}%
\pgfsetbuttcap%
\pgfsetmiterjoin%
\definecolor{currentfill}{rgb}{0.121569,0.466667,0.705882}%
\pgfsetfillcolor{currentfill}%
\pgfsetlinewidth{0.000000pt}%
\definecolor{currentstroke}{rgb}{0.000000,0.000000,0.000000}%
\pgfsetstrokecolor{currentstroke}%
\pgfsetstrokeopacity{0.000000}%
\pgfsetdash{}{0pt}%
\pgfpathmoveto{\pgfqpoint{4.566784in}{0.500000in}}%
\pgfpathlineto{\pgfqpoint{4.593246in}{0.500000in}}%
\pgfpathlineto{\pgfqpoint{4.593246in}{0.500000in}}%
\pgfpathlineto{\pgfqpoint{4.566784in}{0.500000in}}%
\pgfpathlineto{\pgfqpoint{4.566784in}{0.500000in}}%
\pgfpathclose%
\pgfusepath{fill}%
\end{pgfscope}%
\begin{pgfscope}%
\pgfpathrectangle{\pgfqpoint{0.750000in}{0.500000in}}{\pgfqpoint{4.650000in}{3.020000in}}%
\pgfusepath{clip}%
\pgfsetbuttcap%
\pgfsetmiterjoin%
\definecolor{currentfill}{rgb}{0.121569,0.466667,0.705882}%
\pgfsetfillcolor{currentfill}%
\pgfsetlinewidth{0.000000pt}%
\definecolor{currentstroke}{rgb}{0.000000,0.000000,0.000000}%
\pgfsetstrokecolor{currentstroke}%
\pgfsetstrokeopacity{0.000000}%
\pgfsetdash{}{0pt}%
\pgfpathmoveto{\pgfqpoint{4.599861in}{0.500000in}}%
\pgfpathlineto{\pgfqpoint{4.626323in}{0.500000in}}%
\pgfpathlineto{\pgfqpoint{4.626323in}{0.500000in}}%
\pgfpathlineto{\pgfqpoint{4.599861in}{0.500000in}}%
\pgfpathlineto{\pgfqpoint{4.599861in}{0.500000in}}%
\pgfpathclose%
\pgfusepath{fill}%
\end{pgfscope}%
\begin{pgfscope}%
\pgfpathrectangle{\pgfqpoint{0.750000in}{0.500000in}}{\pgfqpoint{4.650000in}{3.020000in}}%
\pgfusepath{clip}%
\pgfsetbuttcap%
\pgfsetmiterjoin%
\definecolor{currentfill}{rgb}{0.121569,0.466667,0.705882}%
\pgfsetfillcolor{currentfill}%
\pgfsetlinewidth{0.000000pt}%
\definecolor{currentstroke}{rgb}{0.000000,0.000000,0.000000}%
\pgfsetstrokecolor{currentstroke}%
\pgfsetstrokeopacity{0.000000}%
\pgfsetdash{}{0pt}%
\pgfpathmoveto{\pgfqpoint{4.632939in}{0.500000in}}%
\pgfpathlineto{\pgfqpoint{4.659400in}{0.500000in}}%
\pgfpathlineto{\pgfqpoint{4.659400in}{0.500000in}}%
\pgfpathlineto{\pgfqpoint{4.632939in}{0.500000in}}%
\pgfpathlineto{\pgfqpoint{4.632939in}{0.500000in}}%
\pgfpathclose%
\pgfusepath{fill}%
\end{pgfscope}%
\begin{pgfscope}%
\pgfpathrectangle{\pgfqpoint{0.750000in}{0.500000in}}{\pgfqpoint{4.650000in}{3.020000in}}%
\pgfusepath{clip}%
\pgfsetbuttcap%
\pgfsetmiterjoin%
\definecolor{currentfill}{rgb}{0.121569,0.466667,0.705882}%
\pgfsetfillcolor{currentfill}%
\pgfsetlinewidth{0.000000pt}%
\definecolor{currentstroke}{rgb}{0.000000,0.000000,0.000000}%
\pgfsetstrokecolor{currentstroke}%
\pgfsetstrokeopacity{0.000000}%
\pgfsetdash{}{0pt}%
\pgfpathmoveto{\pgfqpoint{4.666016in}{0.500000in}}%
\pgfpathlineto{\pgfqpoint{4.692478in}{0.500000in}}%
\pgfpathlineto{\pgfqpoint{4.692478in}{0.500000in}}%
\pgfpathlineto{\pgfqpoint{4.666016in}{0.500000in}}%
\pgfpathlineto{\pgfqpoint{4.666016in}{0.500000in}}%
\pgfpathclose%
\pgfusepath{fill}%
\end{pgfscope}%
\begin{pgfscope}%
\pgfpathrectangle{\pgfqpoint{0.750000in}{0.500000in}}{\pgfqpoint{4.650000in}{3.020000in}}%
\pgfusepath{clip}%
\pgfsetbuttcap%
\pgfsetmiterjoin%
\definecolor{currentfill}{rgb}{0.121569,0.466667,0.705882}%
\pgfsetfillcolor{currentfill}%
\pgfsetlinewidth{0.000000pt}%
\definecolor{currentstroke}{rgb}{0.000000,0.000000,0.000000}%
\pgfsetstrokecolor{currentstroke}%
\pgfsetstrokeopacity{0.000000}%
\pgfsetdash{}{0pt}%
\pgfpathmoveto{\pgfqpoint{4.699093in}{0.500000in}}%
\pgfpathlineto{\pgfqpoint{4.725555in}{0.500000in}}%
\pgfpathlineto{\pgfqpoint{4.725555in}{0.500000in}}%
\pgfpathlineto{\pgfqpoint{4.699093in}{0.500000in}}%
\pgfpathlineto{\pgfqpoint{4.699093in}{0.500000in}}%
\pgfpathclose%
\pgfusepath{fill}%
\end{pgfscope}%
\begin{pgfscope}%
\pgfpathrectangle{\pgfqpoint{0.750000in}{0.500000in}}{\pgfqpoint{4.650000in}{3.020000in}}%
\pgfusepath{clip}%
\pgfsetbuttcap%
\pgfsetmiterjoin%
\definecolor{currentfill}{rgb}{0.121569,0.466667,0.705882}%
\pgfsetfillcolor{currentfill}%
\pgfsetlinewidth{0.000000pt}%
\definecolor{currentstroke}{rgb}{0.000000,0.000000,0.000000}%
\pgfsetstrokecolor{currentstroke}%
\pgfsetstrokeopacity{0.000000}%
\pgfsetdash{}{0pt}%
\pgfpathmoveto{\pgfqpoint{4.732170in}{0.500000in}}%
\pgfpathlineto{\pgfqpoint{4.758632in}{0.500000in}}%
\pgfpathlineto{\pgfqpoint{4.758632in}{0.500000in}}%
\pgfpathlineto{\pgfqpoint{4.732170in}{0.500000in}}%
\pgfpathlineto{\pgfqpoint{4.732170in}{0.500000in}}%
\pgfpathclose%
\pgfusepath{fill}%
\end{pgfscope}%
\begin{pgfscope}%
\pgfpathrectangle{\pgfqpoint{0.750000in}{0.500000in}}{\pgfqpoint{4.650000in}{3.020000in}}%
\pgfusepath{clip}%
\pgfsetbuttcap%
\pgfsetmiterjoin%
\definecolor{currentfill}{rgb}{0.121569,0.466667,0.705882}%
\pgfsetfillcolor{currentfill}%
\pgfsetlinewidth{0.000000pt}%
\definecolor{currentstroke}{rgb}{0.000000,0.000000,0.000000}%
\pgfsetstrokecolor{currentstroke}%
\pgfsetstrokeopacity{0.000000}%
\pgfsetdash{}{0pt}%
\pgfpathmoveto{\pgfqpoint{4.765248in}{0.500000in}}%
\pgfpathlineto{\pgfqpoint{4.791709in}{0.500000in}}%
\pgfpathlineto{\pgfqpoint{4.791709in}{0.500000in}}%
\pgfpathlineto{\pgfqpoint{4.765248in}{0.500000in}}%
\pgfpathlineto{\pgfqpoint{4.765248in}{0.500000in}}%
\pgfpathclose%
\pgfusepath{fill}%
\end{pgfscope}%
\begin{pgfscope}%
\pgfpathrectangle{\pgfqpoint{0.750000in}{0.500000in}}{\pgfqpoint{4.650000in}{3.020000in}}%
\pgfusepath{clip}%
\pgfsetbuttcap%
\pgfsetmiterjoin%
\definecolor{currentfill}{rgb}{0.121569,0.466667,0.705882}%
\pgfsetfillcolor{currentfill}%
\pgfsetlinewidth{0.000000pt}%
\definecolor{currentstroke}{rgb}{0.000000,0.000000,0.000000}%
\pgfsetstrokecolor{currentstroke}%
\pgfsetstrokeopacity{0.000000}%
\pgfsetdash{}{0pt}%
\pgfpathmoveto{\pgfqpoint{4.798325in}{0.500000in}}%
\pgfpathlineto{\pgfqpoint{4.824787in}{0.500000in}}%
\pgfpathlineto{\pgfqpoint{4.824787in}{0.500000in}}%
\pgfpathlineto{\pgfqpoint{4.798325in}{0.500000in}}%
\pgfpathlineto{\pgfqpoint{4.798325in}{0.500000in}}%
\pgfpathclose%
\pgfusepath{fill}%
\end{pgfscope}%
\begin{pgfscope}%
\pgfpathrectangle{\pgfqpoint{0.750000in}{0.500000in}}{\pgfqpoint{4.650000in}{3.020000in}}%
\pgfusepath{clip}%
\pgfsetbuttcap%
\pgfsetmiterjoin%
\definecolor{currentfill}{rgb}{0.121569,0.466667,0.705882}%
\pgfsetfillcolor{currentfill}%
\pgfsetlinewidth{0.000000pt}%
\definecolor{currentstroke}{rgb}{0.000000,0.000000,0.000000}%
\pgfsetstrokecolor{currentstroke}%
\pgfsetstrokeopacity{0.000000}%
\pgfsetdash{}{0pt}%
\pgfpathmoveto{\pgfqpoint{4.831402in}{0.500000in}}%
\pgfpathlineto{\pgfqpoint{4.857864in}{0.500000in}}%
\pgfpathlineto{\pgfqpoint{4.857864in}{0.500000in}}%
\pgfpathlineto{\pgfqpoint{4.831402in}{0.500000in}}%
\pgfpathlineto{\pgfqpoint{4.831402in}{0.500000in}}%
\pgfpathclose%
\pgfusepath{fill}%
\end{pgfscope}%
\begin{pgfscope}%
\pgfpathrectangle{\pgfqpoint{0.750000in}{0.500000in}}{\pgfqpoint{4.650000in}{3.020000in}}%
\pgfusepath{clip}%
\pgfsetbuttcap%
\pgfsetmiterjoin%
\definecolor{currentfill}{rgb}{0.121569,0.466667,0.705882}%
\pgfsetfillcolor{currentfill}%
\pgfsetlinewidth{0.000000pt}%
\definecolor{currentstroke}{rgb}{0.000000,0.000000,0.000000}%
\pgfsetstrokecolor{currentstroke}%
\pgfsetstrokeopacity{0.000000}%
\pgfsetdash{}{0pt}%
\pgfpathmoveto{\pgfqpoint{4.864479in}{0.500000in}}%
\pgfpathlineto{\pgfqpoint{4.890941in}{0.500000in}}%
\pgfpathlineto{\pgfqpoint{4.890941in}{0.500000in}}%
\pgfpathlineto{\pgfqpoint{4.864479in}{0.500000in}}%
\pgfpathlineto{\pgfqpoint{4.864479in}{0.500000in}}%
\pgfpathclose%
\pgfusepath{fill}%
\end{pgfscope}%
\begin{pgfscope}%
\pgfpathrectangle{\pgfqpoint{0.750000in}{0.500000in}}{\pgfqpoint{4.650000in}{3.020000in}}%
\pgfusepath{clip}%
\pgfsetbuttcap%
\pgfsetmiterjoin%
\definecolor{currentfill}{rgb}{0.121569,0.466667,0.705882}%
\pgfsetfillcolor{currentfill}%
\pgfsetlinewidth{0.000000pt}%
\definecolor{currentstroke}{rgb}{0.000000,0.000000,0.000000}%
\pgfsetstrokecolor{currentstroke}%
\pgfsetstrokeopacity{0.000000}%
\pgfsetdash{}{0pt}%
\pgfpathmoveto{\pgfqpoint{4.897557in}{0.500000in}}%
\pgfpathlineto{\pgfqpoint{4.924018in}{0.500000in}}%
\pgfpathlineto{\pgfqpoint{4.924018in}{0.500000in}}%
\pgfpathlineto{\pgfqpoint{4.897557in}{0.500000in}}%
\pgfpathlineto{\pgfqpoint{4.897557in}{0.500000in}}%
\pgfpathclose%
\pgfusepath{fill}%
\end{pgfscope}%
\begin{pgfscope}%
\pgfpathrectangle{\pgfqpoint{0.750000in}{0.500000in}}{\pgfqpoint{4.650000in}{3.020000in}}%
\pgfusepath{clip}%
\pgfsetbuttcap%
\pgfsetmiterjoin%
\definecolor{currentfill}{rgb}{0.121569,0.466667,0.705882}%
\pgfsetfillcolor{currentfill}%
\pgfsetlinewidth{0.000000pt}%
\definecolor{currentstroke}{rgb}{0.000000,0.000000,0.000000}%
\pgfsetstrokecolor{currentstroke}%
\pgfsetstrokeopacity{0.000000}%
\pgfsetdash{}{0pt}%
\pgfpathmoveto{\pgfqpoint{4.930634in}{0.500000in}}%
\pgfpathlineto{\pgfqpoint{4.957096in}{0.500000in}}%
\pgfpathlineto{\pgfqpoint{4.957096in}{0.500000in}}%
\pgfpathlineto{\pgfqpoint{4.930634in}{0.500000in}}%
\pgfpathlineto{\pgfqpoint{4.930634in}{0.500000in}}%
\pgfpathclose%
\pgfusepath{fill}%
\end{pgfscope}%
\begin{pgfscope}%
\pgfpathrectangle{\pgfqpoint{0.750000in}{0.500000in}}{\pgfqpoint{4.650000in}{3.020000in}}%
\pgfusepath{clip}%
\pgfsetbuttcap%
\pgfsetmiterjoin%
\definecolor{currentfill}{rgb}{0.121569,0.466667,0.705882}%
\pgfsetfillcolor{currentfill}%
\pgfsetlinewidth{0.000000pt}%
\definecolor{currentstroke}{rgb}{0.000000,0.000000,0.000000}%
\pgfsetstrokecolor{currentstroke}%
\pgfsetstrokeopacity{0.000000}%
\pgfsetdash{}{0pt}%
\pgfpathmoveto{\pgfqpoint{4.963711in}{0.500000in}}%
\pgfpathlineto{\pgfqpoint{4.990173in}{0.500000in}}%
\pgfpathlineto{\pgfqpoint{4.990173in}{0.500000in}}%
\pgfpathlineto{\pgfqpoint{4.963711in}{0.500000in}}%
\pgfpathlineto{\pgfqpoint{4.963711in}{0.500000in}}%
\pgfpathclose%
\pgfusepath{fill}%
\end{pgfscope}%
\begin{pgfscope}%
\pgfpathrectangle{\pgfqpoint{0.750000in}{0.500000in}}{\pgfqpoint{4.650000in}{3.020000in}}%
\pgfusepath{clip}%
\pgfsetbuttcap%
\pgfsetmiterjoin%
\definecolor{currentfill}{rgb}{0.121569,0.466667,0.705882}%
\pgfsetfillcolor{currentfill}%
\pgfsetlinewidth{0.000000pt}%
\definecolor{currentstroke}{rgb}{0.000000,0.000000,0.000000}%
\pgfsetstrokecolor{currentstroke}%
\pgfsetstrokeopacity{0.000000}%
\pgfsetdash{}{0pt}%
\pgfpathmoveto{\pgfqpoint{4.996788in}{0.500000in}}%
\pgfpathlineto{\pgfqpoint{5.023250in}{0.500000in}}%
\pgfpathlineto{\pgfqpoint{5.023250in}{0.500000in}}%
\pgfpathlineto{\pgfqpoint{4.996788in}{0.500000in}}%
\pgfpathlineto{\pgfqpoint{4.996788in}{0.500000in}}%
\pgfpathclose%
\pgfusepath{fill}%
\end{pgfscope}%
\begin{pgfscope}%
\pgfpathrectangle{\pgfqpoint{0.750000in}{0.500000in}}{\pgfqpoint{4.650000in}{3.020000in}}%
\pgfusepath{clip}%
\pgfsetbuttcap%
\pgfsetmiterjoin%
\definecolor{currentfill}{rgb}{0.121569,0.466667,0.705882}%
\pgfsetfillcolor{currentfill}%
\pgfsetlinewidth{0.000000pt}%
\definecolor{currentstroke}{rgb}{0.000000,0.000000,0.000000}%
\pgfsetstrokecolor{currentstroke}%
\pgfsetstrokeopacity{0.000000}%
\pgfsetdash{}{0pt}%
\pgfpathmoveto{\pgfqpoint{5.029866in}{0.500000in}}%
\pgfpathlineto{\pgfqpoint{5.056327in}{0.500000in}}%
\pgfpathlineto{\pgfqpoint{5.056327in}{0.500000in}}%
\pgfpathlineto{\pgfqpoint{5.029866in}{0.500000in}}%
\pgfpathlineto{\pgfqpoint{5.029866in}{0.500000in}}%
\pgfpathclose%
\pgfusepath{fill}%
\end{pgfscope}%
\begin{pgfscope}%
\pgfpathrectangle{\pgfqpoint{0.750000in}{0.500000in}}{\pgfqpoint{4.650000in}{3.020000in}}%
\pgfusepath{clip}%
\pgfsetbuttcap%
\pgfsetmiterjoin%
\definecolor{currentfill}{rgb}{0.121569,0.466667,0.705882}%
\pgfsetfillcolor{currentfill}%
\pgfsetlinewidth{0.000000pt}%
\definecolor{currentstroke}{rgb}{0.000000,0.000000,0.000000}%
\pgfsetstrokecolor{currentstroke}%
\pgfsetstrokeopacity{0.000000}%
\pgfsetdash{}{0pt}%
\pgfpathmoveto{\pgfqpoint{5.062943in}{0.500000in}}%
\pgfpathlineto{\pgfqpoint{5.089405in}{0.500000in}}%
\pgfpathlineto{\pgfqpoint{5.089405in}{0.500000in}}%
\pgfpathlineto{\pgfqpoint{5.062943in}{0.500000in}}%
\pgfpathlineto{\pgfqpoint{5.062943in}{0.500000in}}%
\pgfpathclose%
\pgfusepath{fill}%
\end{pgfscope}%
\begin{pgfscope}%
\pgfpathrectangle{\pgfqpoint{0.750000in}{0.500000in}}{\pgfqpoint{4.650000in}{3.020000in}}%
\pgfusepath{clip}%
\pgfsetbuttcap%
\pgfsetmiterjoin%
\definecolor{currentfill}{rgb}{0.121569,0.466667,0.705882}%
\pgfsetfillcolor{currentfill}%
\pgfsetlinewidth{0.000000pt}%
\definecolor{currentstroke}{rgb}{0.000000,0.000000,0.000000}%
\pgfsetstrokecolor{currentstroke}%
\pgfsetstrokeopacity{0.000000}%
\pgfsetdash{}{0pt}%
\pgfpathmoveto{\pgfqpoint{5.096020in}{0.500000in}}%
\pgfpathlineto{\pgfqpoint{5.122482in}{0.500000in}}%
\pgfpathlineto{\pgfqpoint{5.122482in}{0.500000in}}%
\pgfpathlineto{\pgfqpoint{5.096020in}{0.500000in}}%
\pgfpathlineto{\pgfqpoint{5.096020in}{0.500000in}}%
\pgfpathclose%
\pgfusepath{fill}%
\end{pgfscope}%
\begin{pgfscope}%
\pgfpathrectangle{\pgfqpoint{0.750000in}{0.500000in}}{\pgfqpoint{4.650000in}{3.020000in}}%
\pgfusepath{clip}%
\pgfsetbuttcap%
\pgfsetmiterjoin%
\definecolor{currentfill}{rgb}{0.121569,0.466667,0.705882}%
\pgfsetfillcolor{currentfill}%
\pgfsetlinewidth{0.000000pt}%
\definecolor{currentstroke}{rgb}{0.000000,0.000000,0.000000}%
\pgfsetstrokecolor{currentstroke}%
\pgfsetstrokeopacity{0.000000}%
\pgfsetdash{}{0pt}%
\pgfpathmoveto{\pgfqpoint{5.129097in}{0.500000in}}%
\pgfpathlineto{\pgfqpoint{5.155559in}{0.500000in}}%
\pgfpathlineto{\pgfqpoint{5.155559in}{0.500000in}}%
\pgfpathlineto{\pgfqpoint{5.129097in}{0.500000in}}%
\pgfpathlineto{\pgfqpoint{5.129097in}{0.500000in}}%
\pgfpathclose%
\pgfusepath{fill}%
\end{pgfscope}%
\begin{pgfscope}%
\pgfpathrectangle{\pgfqpoint{0.750000in}{0.500000in}}{\pgfqpoint{4.650000in}{3.020000in}}%
\pgfusepath{clip}%
\pgfsetbuttcap%
\pgfsetmiterjoin%
\definecolor{currentfill}{rgb}{0.121569,0.466667,0.705882}%
\pgfsetfillcolor{currentfill}%
\pgfsetlinewidth{0.000000pt}%
\definecolor{currentstroke}{rgb}{0.000000,0.000000,0.000000}%
\pgfsetstrokecolor{currentstroke}%
\pgfsetstrokeopacity{0.000000}%
\pgfsetdash{}{0pt}%
\pgfpathmoveto{\pgfqpoint{5.162175in}{0.500000in}}%
\pgfpathlineto{\pgfqpoint{5.188636in}{0.500000in}}%
\pgfpathlineto{\pgfqpoint{5.188636in}{0.500000in}}%
\pgfpathlineto{\pgfqpoint{5.162175in}{0.500000in}}%
\pgfpathlineto{\pgfqpoint{5.162175in}{0.500000in}}%
\pgfpathclose%
\pgfusepath{fill}%
\end{pgfscope}%
\begin{pgfscope}%
\pgfpathrectangle{\pgfqpoint{0.750000in}{0.500000in}}{\pgfqpoint{4.650000in}{3.020000in}}%
\pgfusepath{clip}%
\pgfsetbuttcap%
\pgfsetroundjoin%
\definecolor{currentfill}{rgb}{1.000000,0.000000,0.000000}%
\pgfsetfillcolor{currentfill}%
\pgfsetlinewidth{1.003750pt}%
\definecolor{currentstroke}{rgb}{1.000000,0.000000,0.000000}%
\pgfsetstrokecolor{currentstroke}%
\pgfsetdash{}{0pt}%
\pgfsys@defobject{currentmarker}{\pgfqpoint{-0.041667in}{-0.041667in}}{\pgfqpoint{0.041667in}{0.041667in}}{%
\pgfpathmoveto{\pgfqpoint{0.000000in}{-0.041667in}}%
\pgfpathcurveto{\pgfqpoint{0.011050in}{-0.041667in}}{\pgfqpoint{0.021649in}{-0.037276in}}{\pgfqpoint{0.029463in}{-0.029463in}}%
\pgfpathcurveto{\pgfqpoint{0.037276in}{-0.021649in}}{\pgfqpoint{0.041667in}{-0.011050in}}{\pgfqpoint{0.041667in}{0.000000in}}%
\pgfpathcurveto{\pgfqpoint{0.041667in}{0.011050in}}{\pgfqpoint{0.037276in}{0.021649in}}{\pgfqpoint{0.029463in}{0.029463in}}%
\pgfpathcurveto{\pgfqpoint{0.021649in}{0.037276in}}{\pgfqpoint{0.011050in}{0.041667in}}{\pgfqpoint{0.000000in}{0.041667in}}%
\pgfpathcurveto{\pgfqpoint{-0.011050in}{0.041667in}}{\pgfqpoint{-0.021649in}{0.037276in}}{\pgfqpoint{-0.029463in}{0.029463in}}%
\pgfpathcurveto{\pgfqpoint{-0.037276in}{0.021649in}}{\pgfqpoint{-0.041667in}{0.011050in}}{\pgfqpoint{-0.041667in}{0.000000in}}%
\pgfpathcurveto{\pgfqpoint{-0.041667in}{-0.011050in}}{\pgfqpoint{-0.037276in}{-0.021649in}}{\pgfqpoint{-0.029463in}{-0.029463in}}%
\pgfpathcurveto{\pgfqpoint{-0.021649in}{-0.037276in}}{\pgfqpoint{-0.011050in}{-0.041667in}}{\pgfqpoint{0.000000in}{-0.041667in}}%
\pgfpathlineto{\pgfqpoint{0.000000in}{-0.041667in}}%
\pgfpathclose%
\pgfusepath{stroke,fill}%
}%
\begin{pgfscope}%
\pgfsys@transformshift{4.150011in}{0.500000in}%
\pgfsys@useobject{currentmarker}{}%
\end{pgfscope}%
\end{pgfscope}%
\begin{pgfscope}%
\pgfpathrectangle{\pgfqpoint{0.750000in}{0.500000in}}{\pgfqpoint{4.650000in}{3.020000in}}%
\pgfusepath{clip}%
\pgfsetbuttcap%
\pgfsetroundjoin%
\definecolor{currentfill}{rgb}{0.000000,0.500000,0.000000}%
\pgfsetfillcolor{currentfill}%
\pgfsetlinewidth{1.003750pt}%
\definecolor{currentstroke}{rgb}{0.000000,0.500000,0.000000}%
\pgfsetstrokecolor{currentstroke}%
\pgfsetdash{}{0pt}%
\pgfsys@defobject{currentmarker}{\pgfqpoint{-0.041667in}{-0.041667in}}{\pgfqpoint{0.041667in}{0.041667in}}{%
\pgfpathmoveto{\pgfqpoint{0.000000in}{-0.041667in}}%
\pgfpathcurveto{\pgfqpoint{0.011050in}{-0.041667in}}{\pgfqpoint{0.021649in}{-0.037276in}}{\pgfqpoint{0.029463in}{-0.029463in}}%
\pgfpathcurveto{\pgfqpoint{0.037276in}{-0.021649in}}{\pgfqpoint{0.041667in}{-0.011050in}}{\pgfqpoint{0.041667in}{0.000000in}}%
\pgfpathcurveto{\pgfqpoint{0.041667in}{0.011050in}}{\pgfqpoint{0.037276in}{0.021649in}}{\pgfqpoint{0.029463in}{0.029463in}}%
\pgfpathcurveto{\pgfqpoint{0.021649in}{0.037276in}}{\pgfqpoint{0.011050in}{0.041667in}}{\pgfqpoint{0.000000in}{0.041667in}}%
\pgfpathcurveto{\pgfqpoint{-0.011050in}{0.041667in}}{\pgfqpoint{-0.021649in}{0.037276in}}{\pgfqpoint{-0.029463in}{0.029463in}}%
\pgfpathcurveto{\pgfqpoint{-0.037276in}{0.021649in}}{\pgfqpoint{-0.041667in}{0.011050in}}{\pgfqpoint{-0.041667in}{0.000000in}}%
\pgfpathcurveto{\pgfqpoint{-0.041667in}{-0.011050in}}{\pgfqpoint{-0.037276in}{-0.021649in}}{\pgfqpoint{-0.029463in}{-0.029463in}}%
\pgfpathcurveto{\pgfqpoint{-0.021649in}{-0.037276in}}{\pgfqpoint{-0.011050in}{-0.041667in}}{\pgfqpoint{0.000000in}{-0.041667in}}%
\pgfpathlineto{\pgfqpoint{0.000000in}{-0.041667in}}%
\pgfpathclose%
\pgfusepath{stroke,fill}%
}%
\begin{pgfscope}%
\pgfsys@transformshift{4.216165in}{0.500000in}%
\pgfsys@useobject{currentmarker}{}%
\end{pgfscope}%
\end{pgfscope}%
\begin{pgfscope}%
\pgfsetbuttcap%
\pgfsetroundjoin%
\definecolor{currentfill}{rgb}{0.000000,0.000000,0.000000}%
\pgfsetfillcolor{currentfill}%
\pgfsetlinewidth{0.803000pt}%
\definecolor{currentstroke}{rgb}{0.000000,0.000000,0.000000}%
\pgfsetstrokecolor{currentstroke}%
\pgfsetdash{}{0pt}%
\pgfsys@defobject{currentmarker}{\pgfqpoint{0.000000in}{-0.048611in}}{\pgfqpoint{0.000000in}{0.000000in}}{%
\pgfpathmoveto{\pgfqpoint{0.000000in}{0.000000in}}%
\pgfpathlineto{\pgfqpoint{0.000000in}{-0.048611in}}%
\pgfusepath{stroke,fill}%
}%
\begin{pgfscope}%
\pgfsys@transformshift{0.974595in}{0.500000in}%
\pgfsys@useobject{currentmarker}{}%
\end{pgfscope}%
\end{pgfscope}%
\begin{pgfscope}%
\definecolor{textcolor}{rgb}{0.000000,0.000000,0.000000}%
\pgfsetstrokecolor{textcolor}%
\pgfsetfillcolor{textcolor}%
\pgftext[x=0.974595in,y=0.402778in,,top]{\color{textcolor}\sffamily\fontsize{10.000000}{12.000000}\selectfont 0}%
\end{pgfscope}%
\begin{pgfscope}%
\pgfsetbuttcap%
\pgfsetroundjoin%
\definecolor{currentfill}{rgb}{0.000000,0.000000,0.000000}%
\pgfsetfillcolor{currentfill}%
\pgfsetlinewidth{0.803000pt}%
\definecolor{currentstroke}{rgb}{0.000000,0.000000,0.000000}%
\pgfsetstrokecolor{currentstroke}%
\pgfsetdash{}{0pt}%
\pgfsys@defobject{currentmarker}{\pgfqpoint{0.000000in}{-0.048611in}}{\pgfqpoint{0.000000in}{0.000000in}}{%
\pgfpathmoveto{\pgfqpoint{0.000000in}{0.000000in}}%
\pgfpathlineto{\pgfqpoint{0.000000in}{-0.048611in}}%
\pgfusepath{stroke,fill}%
}%
\begin{pgfscope}%
\pgfsys@transformshift{1.636140in}{0.500000in}%
\pgfsys@useobject{currentmarker}{}%
\end{pgfscope}%
\end{pgfscope}%
\begin{pgfscope}%
\definecolor{textcolor}{rgb}{0.000000,0.000000,0.000000}%
\pgfsetstrokecolor{textcolor}%
\pgfsetfillcolor{textcolor}%
\pgftext[x=1.636140in,y=0.402778in,,top]{\color{textcolor}\sffamily\fontsize{10.000000}{12.000000}\selectfont 20}%
\end{pgfscope}%
\begin{pgfscope}%
\pgfsetbuttcap%
\pgfsetroundjoin%
\definecolor{currentfill}{rgb}{0.000000,0.000000,0.000000}%
\pgfsetfillcolor{currentfill}%
\pgfsetlinewidth{0.803000pt}%
\definecolor{currentstroke}{rgb}{0.000000,0.000000,0.000000}%
\pgfsetstrokecolor{currentstroke}%
\pgfsetdash{}{0pt}%
\pgfsys@defobject{currentmarker}{\pgfqpoint{0.000000in}{-0.048611in}}{\pgfqpoint{0.000000in}{0.000000in}}{%
\pgfpathmoveto{\pgfqpoint{0.000000in}{0.000000in}}%
\pgfpathlineto{\pgfqpoint{0.000000in}{-0.048611in}}%
\pgfusepath{stroke,fill}%
}%
\begin{pgfscope}%
\pgfsys@transformshift{2.297685in}{0.500000in}%
\pgfsys@useobject{currentmarker}{}%
\end{pgfscope}%
\end{pgfscope}%
\begin{pgfscope}%
\definecolor{textcolor}{rgb}{0.000000,0.000000,0.000000}%
\pgfsetstrokecolor{textcolor}%
\pgfsetfillcolor{textcolor}%
\pgftext[x=2.297685in,y=0.402778in,,top]{\color{textcolor}\sffamily\fontsize{10.000000}{12.000000}\selectfont 40}%
\end{pgfscope}%
\begin{pgfscope}%
\pgfsetbuttcap%
\pgfsetroundjoin%
\definecolor{currentfill}{rgb}{0.000000,0.000000,0.000000}%
\pgfsetfillcolor{currentfill}%
\pgfsetlinewidth{0.803000pt}%
\definecolor{currentstroke}{rgb}{0.000000,0.000000,0.000000}%
\pgfsetstrokecolor{currentstroke}%
\pgfsetdash{}{0pt}%
\pgfsys@defobject{currentmarker}{\pgfqpoint{0.000000in}{-0.048611in}}{\pgfqpoint{0.000000in}{0.000000in}}{%
\pgfpathmoveto{\pgfqpoint{0.000000in}{0.000000in}}%
\pgfpathlineto{\pgfqpoint{0.000000in}{-0.048611in}}%
\pgfusepath{stroke,fill}%
}%
\begin{pgfscope}%
\pgfsys@transformshift{2.959230in}{0.500000in}%
\pgfsys@useobject{currentmarker}{}%
\end{pgfscope}%
\end{pgfscope}%
\begin{pgfscope}%
\definecolor{textcolor}{rgb}{0.000000,0.000000,0.000000}%
\pgfsetstrokecolor{textcolor}%
\pgfsetfillcolor{textcolor}%
\pgftext[x=2.959230in,y=0.402778in,,top]{\color{textcolor}\sffamily\fontsize{10.000000}{12.000000}\selectfont 60}%
\end{pgfscope}%
\begin{pgfscope}%
\pgfsetbuttcap%
\pgfsetroundjoin%
\definecolor{currentfill}{rgb}{0.000000,0.000000,0.000000}%
\pgfsetfillcolor{currentfill}%
\pgfsetlinewidth{0.803000pt}%
\definecolor{currentstroke}{rgb}{0.000000,0.000000,0.000000}%
\pgfsetstrokecolor{currentstroke}%
\pgfsetdash{}{0pt}%
\pgfsys@defobject{currentmarker}{\pgfqpoint{0.000000in}{-0.048611in}}{\pgfqpoint{0.000000in}{0.000000in}}{%
\pgfpathmoveto{\pgfqpoint{0.000000in}{0.000000in}}%
\pgfpathlineto{\pgfqpoint{0.000000in}{-0.048611in}}%
\pgfusepath{stroke,fill}%
}%
\begin{pgfscope}%
\pgfsys@transformshift{3.620775in}{0.500000in}%
\pgfsys@useobject{currentmarker}{}%
\end{pgfscope}%
\end{pgfscope}%
\begin{pgfscope}%
\definecolor{textcolor}{rgb}{0.000000,0.000000,0.000000}%
\pgfsetstrokecolor{textcolor}%
\pgfsetfillcolor{textcolor}%
\pgftext[x=3.620775in,y=0.402778in,,top]{\color{textcolor}\sffamily\fontsize{10.000000}{12.000000}\selectfont 80}%
\end{pgfscope}%
\begin{pgfscope}%
\pgfsetbuttcap%
\pgfsetroundjoin%
\definecolor{currentfill}{rgb}{0.000000,0.000000,0.000000}%
\pgfsetfillcolor{currentfill}%
\pgfsetlinewidth{0.803000pt}%
\definecolor{currentstroke}{rgb}{0.000000,0.000000,0.000000}%
\pgfsetstrokecolor{currentstroke}%
\pgfsetdash{}{0pt}%
\pgfsys@defobject{currentmarker}{\pgfqpoint{0.000000in}{-0.048611in}}{\pgfqpoint{0.000000in}{0.000000in}}{%
\pgfpathmoveto{\pgfqpoint{0.000000in}{0.000000in}}%
\pgfpathlineto{\pgfqpoint{0.000000in}{-0.048611in}}%
\pgfusepath{stroke,fill}%
}%
\begin{pgfscope}%
\pgfsys@transformshift{4.282320in}{0.500000in}%
\pgfsys@useobject{currentmarker}{}%
\end{pgfscope}%
\end{pgfscope}%
\begin{pgfscope}%
\definecolor{textcolor}{rgb}{0.000000,0.000000,0.000000}%
\pgfsetstrokecolor{textcolor}%
\pgfsetfillcolor{textcolor}%
\pgftext[x=4.282320in,y=0.402778in,,top]{\color{textcolor}\sffamily\fontsize{10.000000}{12.000000}\selectfont 100}%
\end{pgfscope}%
\begin{pgfscope}%
\pgfsetbuttcap%
\pgfsetroundjoin%
\definecolor{currentfill}{rgb}{0.000000,0.000000,0.000000}%
\pgfsetfillcolor{currentfill}%
\pgfsetlinewidth{0.803000pt}%
\definecolor{currentstroke}{rgb}{0.000000,0.000000,0.000000}%
\pgfsetstrokecolor{currentstroke}%
\pgfsetdash{}{0pt}%
\pgfsys@defobject{currentmarker}{\pgfqpoint{0.000000in}{-0.048611in}}{\pgfqpoint{0.000000in}{0.000000in}}{%
\pgfpathmoveto{\pgfqpoint{0.000000in}{0.000000in}}%
\pgfpathlineto{\pgfqpoint{0.000000in}{-0.048611in}}%
\pgfusepath{stroke,fill}%
}%
\begin{pgfscope}%
\pgfsys@transformshift{4.943865in}{0.500000in}%
\pgfsys@useobject{currentmarker}{}%
\end{pgfscope}%
\end{pgfscope}%
\begin{pgfscope}%
\definecolor{textcolor}{rgb}{0.000000,0.000000,0.000000}%
\pgfsetstrokecolor{textcolor}%
\pgfsetfillcolor{textcolor}%
\pgftext[x=4.943865in,y=0.402778in,,top]{\color{textcolor}\sffamily\fontsize{10.000000}{12.000000}\selectfont 120}%
\end{pgfscope}%
\begin{pgfscope}%
\definecolor{textcolor}{rgb}{0.000000,0.000000,0.000000}%
\pgfsetstrokecolor{textcolor}%
\pgfsetfillcolor{textcolor}%
\pgftext[x=3.075000in,y=0.212809in,,top]{\color{textcolor}\sffamily\fontsize{10.000000}{12.000000}\selectfont Number of correct bits}%
\end{pgfscope}%
\begin{pgfscope}%
\pgfsetbuttcap%
\pgfsetroundjoin%
\definecolor{currentfill}{rgb}{0.000000,0.000000,0.000000}%
\pgfsetfillcolor{currentfill}%
\pgfsetlinewidth{0.803000pt}%
\definecolor{currentstroke}{rgb}{0.000000,0.000000,0.000000}%
\pgfsetstrokecolor{currentstroke}%
\pgfsetdash{}{0pt}%
\pgfsys@defobject{currentmarker}{\pgfqpoint{-0.048611in}{0.000000in}}{\pgfqpoint{-0.000000in}{0.000000in}}{%
\pgfpathmoveto{\pgfqpoint{-0.000000in}{0.000000in}}%
\pgfpathlineto{\pgfqpoint{-0.048611in}{0.000000in}}%
\pgfusepath{stroke,fill}%
}%
\begin{pgfscope}%
\pgfsys@transformshift{0.750000in}{0.500000in}%
\pgfsys@useobject{currentmarker}{}%
\end{pgfscope}%
\end{pgfscope}%
\begin{pgfscope}%
\definecolor{textcolor}{rgb}{0.000000,0.000000,0.000000}%
\pgfsetstrokecolor{textcolor}%
\pgfsetfillcolor{textcolor}%
\pgftext[x=0.343533in, y=0.447238in, left, base]{\color{textcolor}\sffamily\fontsize{10.000000}{12.000000}\selectfont 0.00}%
\end{pgfscope}%
\begin{pgfscope}%
\pgfsetbuttcap%
\pgfsetroundjoin%
\definecolor{currentfill}{rgb}{0.000000,0.000000,0.000000}%
\pgfsetfillcolor{currentfill}%
\pgfsetlinewidth{0.803000pt}%
\definecolor{currentstroke}{rgb}{0.000000,0.000000,0.000000}%
\pgfsetstrokecolor{currentstroke}%
\pgfsetdash{}{0pt}%
\pgfsys@defobject{currentmarker}{\pgfqpoint{-0.048611in}{0.000000in}}{\pgfqpoint{-0.000000in}{0.000000in}}{%
\pgfpathmoveto{\pgfqpoint{-0.000000in}{0.000000in}}%
\pgfpathlineto{\pgfqpoint{-0.048611in}{0.000000in}}%
\pgfusepath{stroke,fill}%
}%
\begin{pgfscope}%
\pgfsys@transformshift{0.750000in}{0.908631in}%
\pgfsys@useobject{currentmarker}{}%
\end{pgfscope}%
\end{pgfscope}%
\begin{pgfscope}%
\definecolor{textcolor}{rgb}{0.000000,0.000000,0.000000}%
\pgfsetstrokecolor{textcolor}%
\pgfsetfillcolor{textcolor}%
\pgftext[x=0.343533in, y=0.855869in, left, base]{\color{textcolor}\sffamily\fontsize{10.000000}{12.000000}\selectfont 0.01}%
\end{pgfscope}%
\begin{pgfscope}%
\pgfsetbuttcap%
\pgfsetroundjoin%
\definecolor{currentfill}{rgb}{0.000000,0.000000,0.000000}%
\pgfsetfillcolor{currentfill}%
\pgfsetlinewidth{0.803000pt}%
\definecolor{currentstroke}{rgb}{0.000000,0.000000,0.000000}%
\pgfsetstrokecolor{currentstroke}%
\pgfsetdash{}{0pt}%
\pgfsys@defobject{currentmarker}{\pgfqpoint{-0.048611in}{0.000000in}}{\pgfqpoint{-0.000000in}{0.000000in}}{%
\pgfpathmoveto{\pgfqpoint{-0.000000in}{0.000000in}}%
\pgfpathlineto{\pgfqpoint{-0.048611in}{0.000000in}}%
\pgfusepath{stroke,fill}%
}%
\begin{pgfscope}%
\pgfsys@transformshift{0.750000in}{1.317261in}%
\pgfsys@useobject{currentmarker}{}%
\end{pgfscope}%
\end{pgfscope}%
\begin{pgfscope}%
\definecolor{textcolor}{rgb}{0.000000,0.000000,0.000000}%
\pgfsetstrokecolor{textcolor}%
\pgfsetfillcolor{textcolor}%
\pgftext[x=0.343533in, y=1.264499in, left, base]{\color{textcolor}\sffamily\fontsize{10.000000}{12.000000}\selectfont 0.02}%
\end{pgfscope}%
\begin{pgfscope}%
\pgfsetbuttcap%
\pgfsetroundjoin%
\definecolor{currentfill}{rgb}{0.000000,0.000000,0.000000}%
\pgfsetfillcolor{currentfill}%
\pgfsetlinewidth{0.803000pt}%
\definecolor{currentstroke}{rgb}{0.000000,0.000000,0.000000}%
\pgfsetstrokecolor{currentstroke}%
\pgfsetdash{}{0pt}%
\pgfsys@defobject{currentmarker}{\pgfqpoint{-0.048611in}{0.000000in}}{\pgfqpoint{-0.000000in}{0.000000in}}{%
\pgfpathmoveto{\pgfqpoint{-0.000000in}{0.000000in}}%
\pgfpathlineto{\pgfqpoint{-0.048611in}{0.000000in}}%
\pgfusepath{stroke,fill}%
}%
\begin{pgfscope}%
\pgfsys@transformshift{0.750000in}{1.725892in}%
\pgfsys@useobject{currentmarker}{}%
\end{pgfscope}%
\end{pgfscope}%
\begin{pgfscope}%
\definecolor{textcolor}{rgb}{0.000000,0.000000,0.000000}%
\pgfsetstrokecolor{textcolor}%
\pgfsetfillcolor{textcolor}%
\pgftext[x=0.343533in, y=1.673130in, left, base]{\color{textcolor}\sffamily\fontsize{10.000000}{12.000000}\selectfont 0.03}%
\end{pgfscope}%
\begin{pgfscope}%
\pgfsetbuttcap%
\pgfsetroundjoin%
\definecolor{currentfill}{rgb}{0.000000,0.000000,0.000000}%
\pgfsetfillcolor{currentfill}%
\pgfsetlinewidth{0.803000pt}%
\definecolor{currentstroke}{rgb}{0.000000,0.000000,0.000000}%
\pgfsetstrokecolor{currentstroke}%
\pgfsetdash{}{0pt}%
\pgfsys@defobject{currentmarker}{\pgfqpoint{-0.048611in}{0.000000in}}{\pgfqpoint{-0.000000in}{0.000000in}}{%
\pgfpathmoveto{\pgfqpoint{-0.000000in}{0.000000in}}%
\pgfpathlineto{\pgfqpoint{-0.048611in}{0.000000in}}%
\pgfusepath{stroke,fill}%
}%
\begin{pgfscope}%
\pgfsys@transformshift{0.750000in}{2.134522in}%
\pgfsys@useobject{currentmarker}{}%
\end{pgfscope}%
\end{pgfscope}%
\begin{pgfscope}%
\definecolor{textcolor}{rgb}{0.000000,0.000000,0.000000}%
\pgfsetstrokecolor{textcolor}%
\pgfsetfillcolor{textcolor}%
\pgftext[x=0.343533in, y=2.081761in, left, base]{\color{textcolor}\sffamily\fontsize{10.000000}{12.000000}\selectfont 0.04}%
\end{pgfscope}%
\begin{pgfscope}%
\pgfsetbuttcap%
\pgfsetroundjoin%
\definecolor{currentfill}{rgb}{0.000000,0.000000,0.000000}%
\pgfsetfillcolor{currentfill}%
\pgfsetlinewidth{0.803000pt}%
\definecolor{currentstroke}{rgb}{0.000000,0.000000,0.000000}%
\pgfsetstrokecolor{currentstroke}%
\pgfsetdash{}{0pt}%
\pgfsys@defobject{currentmarker}{\pgfqpoint{-0.048611in}{0.000000in}}{\pgfqpoint{-0.000000in}{0.000000in}}{%
\pgfpathmoveto{\pgfqpoint{-0.000000in}{0.000000in}}%
\pgfpathlineto{\pgfqpoint{-0.048611in}{0.000000in}}%
\pgfusepath{stroke,fill}%
}%
\begin{pgfscope}%
\pgfsys@transformshift{0.750000in}{2.543153in}%
\pgfsys@useobject{currentmarker}{}%
\end{pgfscope}%
\end{pgfscope}%
\begin{pgfscope}%
\definecolor{textcolor}{rgb}{0.000000,0.000000,0.000000}%
\pgfsetstrokecolor{textcolor}%
\pgfsetfillcolor{textcolor}%
\pgftext[x=0.343533in, y=2.490391in, left, base]{\color{textcolor}\sffamily\fontsize{10.000000}{12.000000}\selectfont 0.05}%
\end{pgfscope}%
\begin{pgfscope}%
\pgfsetbuttcap%
\pgfsetroundjoin%
\definecolor{currentfill}{rgb}{0.000000,0.000000,0.000000}%
\pgfsetfillcolor{currentfill}%
\pgfsetlinewidth{0.803000pt}%
\definecolor{currentstroke}{rgb}{0.000000,0.000000,0.000000}%
\pgfsetstrokecolor{currentstroke}%
\pgfsetdash{}{0pt}%
\pgfsys@defobject{currentmarker}{\pgfqpoint{-0.048611in}{0.000000in}}{\pgfqpoint{-0.000000in}{0.000000in}}{%
\pgfpathmoveto{\pgfqpoint{-0.000000in}{0.000000in}}%
\pgfpathlineto{\pgfqpoint{-0.048611in}{0.000000in}}%
\pgfusepath{stroke,fill}%
}%
\begin{pgfscope}%
\pgfsys@transformshift{0.750000in}{2.951783in}%
\pgfsys@useobject{currentmarker}{}%
\end{pgfscope}%
\end{pgfscope}%
\begin{pgfscope}%
\definecolor{textcolor}{rgb}{0.000000,0.000000,0.000000}%
\pgfsetstrokecolor{textcolor}%
\pgfsetfillcolor{textcolor}%
\pgftext[x=0.343533in, y=2.899022in, left, base]{\color{textcolor}\sffamily\fontsize{10.000000}{12.000000}\selectfont 0.06}%
\end{pgfscope}%
\begin{pgfscope}%
\pgfsetbuttcap%
\pgfsetroundjoin%
\definecolor{currentfill}{rgb}{0.000000,0.000000,0.000000}%
\pgfsetfillcolor{currentfill}%
\pgfsetlinewidth{0.803000pt}%
\definecolor{currentstroke}{rgb}{0.000000,0.000000,0.000000}%
\pgfsetstrokecolor{currentstroke}%
\pgfsetdash{}{0pt}%
\pgfsys@defobject{currentmarker}{\pgfqpoint{-0.048611in}{0.000000in}}{\pgfqpoint{-0.000000in}{0.000000in}}{%
\pgfpathmoveto{\pgfqpoint{-0.000000in}{0.000000in}}%
\pgfpathlineto{\pgfqpoint{-0.048611in}{0.000000in}}%
\pgfusepath{stroke,fill}%
}%
\begin{pgfscope}%
\pgfsys@transformshift{0.750000in}{3.360414in}%
\pgfsys@useobject{currentmarker}{}%
\end{pgfscope}%
\end{pgfscope}%
\begin{pgfscope}%
\definecolor{textcolor}{rgb}{0.000000,0.000000,0.000000}%
\pgfsetstrokecolor{textcolor}%
\pgfsetfillcolor{textcolor}%
\pgftext[x=0.343533in, y=3.307652in, left, base]{\color{textcolor}\sffamily\fontsize{10.000000}{12.000000}\selectfont 0.07}%
\end{pgfscope}%
\begin{pgfscope}%
\definecolor{textcolor}{rgb}{0.000000,0.000000,0.000000}%
\pgfsetstrokecolor{textcolor}%
\pgfsetfillcolor{textcolor}%
\pgftext[x=0.287977in,y=2.010000in,,bottom,rotate=90.000000]{\color{textcolor}\sffamily\fontsize{10.000000}{12.000000}\selectfont Probability}%
\end{pgfscope}%
\begin{pgfscope}%
\pgfsetrectcap%
\pgfsetmiterjoin%
\pgfsetlinewidth{0.803000pt}%
\definecolor{currentstroke}{rgb}{0.000000,0.000000,0.000000}%
\pgfsetstrokecolor{currentstroke}%
\pgfsetdash{}{0pt}%
\pgfpathmoveto{\pgfqpoint{0.750000in}{0.500000in}}%
\pgfpathlineto{\pgfqpoint{0.750000in}{3.520000in}}%
\pgfusepath{stroke}%
\end{pgfscope}%
\begin{pgfscope}%
\pgfsetrectcap%
\pgfsetmiterjoin%
\pgfsetlinewidth{0.803000pt}%
\definecolor{currentstroke}{rgb}{0.000000,0.000000,0.000000}%
\pgfsetstrokecolor{currentstroke}%
\pgfsetdash{}{0pt}%
\pgfpathmoveto{\pgfqpoint{5.400000in}{0.500000in}}%
\pgfpathlineto{\pgfqpoint{5.400000in}{3.520000in}}%
\pgfusepath{stroke}%
\end{pgfscope}%
\begin{pgfscope}%
\pgfsetrectcap%
\pgfsetmiterjoin%
\pgfsetlinewidth{0.803000pt}%
\definecolor{currentstroke}{rgb}{0.000000,0.000000,0.000000}%
\pgfsetstrokecolor{currentstroke}%
\pgfsetdash{}{0pt}%
\pgfpathmoveto{\pgfqpoint{0.750000in}{0.500000in}}%
\pgfpathlineto{\pgfqpoint{5.400000in}{0.500000in}}%
\pgfusepath{stroke}%
\end{pgfscope}%
\begin{pgfscope}%
\pgfsetrectcap%
\pgfsetmiterjoin%
\pgfsetlinewidth{0.803000pt}%
\definecolor{currentstroke}{rgb}{0.000000,0.000000,0.000000}%
\pgfsetstrokecolor{currentstroke}%
\pgfsetdash{}{0pt}%
\pgfpathmoveto{\pgfqpoint{0.750000in}{3.520000in}}%
\pgfpathlineto{\pgfqpoint{5.400000in}{3.520000in}}%
\pgfusepath{stroke}%
\end{pgfscope}%
\begin{pgfscope}%
\definecolor{textcolor}{rgb}{0.000000,0.000000,0.000000}%
\pgfsetstrokecolor{textcolor}%
\pgfsetfillcolor{textcolor}%
\pgftext[x=3.075000in,y=3.603333in,,base]{\color{textcolor}\sffamily\fontsize{12.000000}{14.400000}\selectfont PDF for random bits}%
\end{pgfscope}%
\begin{pgfscope}%
\pgfsetbuttcap%
\pgfsetmiterjoin%
\definecolor{currentfill}{rgb}{1.000000,1.000000,1.000000}%
\pgfsetfillcolor{currentfill}%
\pgfsetfillopacity{0.800000}%
\pgfsetlinewidth{1.003750pt}%
\definecolor{currentstroke}{rgb}{0.800000,0.800000,0.800000}%
\pgfsetstrokecolor{currentstroke}%
\pgfsetstrokeopacity{0.800000}%
\pgfsetdash{}{0pt}%
\pgfpathmoveto{\pgfqpoint{3.456831in}{2.797317in}}%
\pgfpathlineto{\pgfqpoint{5.302778in}{2.797317in}}%
\pgfpathquadraticcurveto{\pgfqpoint{5.330556in}{2.797317in}}{\pgfqpoint{5.330556in}{2.825095in}}%
\pgfpathlineto{\pgfqpoint{5.330556in}{3.422778in}}%
\pgfpathquadraticcurveto{\pgfqpoint{5.330556in}{3.450556in}}{\pgfqpoint{5.302778in}{3.450556in}}%
\pgfpathlineto{\pgfqpoint{3.456831in}{3.450556in}}%
\pgfpathquadraticcurveto{\pgfqpoint{3.429053in}{3.450556in}}{\pgfqpoint{3.429053in}{3.422778in}}%
\pgfpathlineto{\pgfqpoint{3.429053in}{2.825095in}}%
\pgfpathquadraticcurveto{\pgfqpoint{3.429053in}{2.797317in}}{\pgfqpoint{3.456831in}{2.797317in}}%
\pgfpathlineto{\pgfqpoint{3.456831in}{2.797317in}}%
\pgfpathclose%
\pgfusepath{stroke,fill}%
\end{pgfscope}%
\begin{pgfscope}%
\pgfsetbuttcap%
\pgfsetroundjoin%
\definecolor{currentfill}{rgb}{1.000000,0.000000,0.000000}%
\pgfsetfillcolor{currentfill}%
\pgfsetlinewidth{1.003750pt}%
\definecolor{currentstroke}{rgb}{1.000000,0.000000,0.000000}%
\pgfsetstrokecolor{currentstroke}%
\pgfsetdash{}{0pt}%
\pgfsys@defobject{currentmarker}{\pgfqpoint{-0.041667in}{-0.041667in}}{\pgfqpoint{0.041667in}{0.041667in}}{%
\pgfpathmoveto{\pgfqpoint{0.000000in}{-0.041667in}}%
\pgfpathcurveto{\pgfqpoint{0.011050in}{-0.041667in}}{\pgfqpoint{0.021649in}{-0.037276in}}{\pgfqpoint{0.029463in}{-0.029463in}}%
\pgfpathcurveto{\pgfqpoint{0.037276in}{-0.021649in}}{\pgfqpoint{0.041667in}{-0.011050in}}{\pgfqpoint{0.041667in}{0.000000in}}%
\pgfpathcurveto{\pgfqpoint{0.041667in}{0.011050in}}{\pgfqpoint{0.037276in}{0.021649in}}{\pgfqpoint{0.029463in}{0.029463in}}%
\pgfpathcurveto{\pgfqpoint{0.021649in}{0.037276in}}{\pgfqpoint{0.011050in}{0.041667in}}{\pgfqpoint{0.000000in}{0.041667in}}%
\pgfpathcurveto{\pgfqpoint{-0.011050in}{0.041667in}}{\pgfqpoint{-0.021649in}{0.037276in}}{\pgfqpoint{-0.029463in}{0.029463in}}%
\pgfpathcurveto{\pgfqpoint{-0.037276in}{0.021649in}}{\pgfqpoint{-0.041667in}{0.011050in}}{\pgfqpoint{-0.041667in}{0.000000in}}%
\pgfpathcurveto{\pgfqpoint{-0.041667in}{-0.011050in}}{\pgfqpoint{-0.037276in}{-0.021649in}}{\pgfqpoint{-0.029463in}{-0.029463in}}%
\pgfpathcurveto{\pgfqpoint{-0.021649in}{-0.037276in}}{\pgfqpoint{-0.011050in}{-0.041667in}}{\pgfqpoint{0.000000in}{-0.041667in}}%
\pgfpathlineto{\pgfqpoint{0.000000in}{-0.041667in}}%
\pgfpathclose%
\pgfusepath{stroke,fill}%
}%
\begin{pgfscope}%
\pgfsys@transformshift{3.623497in}{3.325935in}%
\pgfsys@useobject{currentmarker}{}%
\end{pgfscope}%
\end{pgfscope}%
\begin{pgfscope}%
\definecolor{textcolor}{rgb}{0.000000,0.000000,0.000000}%
\pgfsetstrokecolor{textcolor}%
\pgfsetfillcolor{textcolor}%
\pgftext[x=3.873497in,y=3.289477in,left,base]{\color{textcolor}\sffamily\fontsize{10.000000}{12.000000}\selectfont SNN}%
\end{pgfscope}%
\begin{pgfscope}%
\pgfsetbuttcap%
\pgfsetroundjoin%
\definecolor{currentfill}{rgb}{0.000000,0.500000,0.000000}%
\pgfsetfillcolor{currentfill}%
\pgfsetlinewidth{1.003750pt}%
\definecolor{currentstroke}{rgb}{0.000000,0.500000,0.000000}%
\pgfsetstrokecolor{currentstroke}%
\pgfsetdash{}{0pt}%
\pgfsys@defobject{currentmarker}{\pgfqpoint{-0.041667in}{-0.041667in}}{\pgfqpoint{0.041667in}{0.041667in}}{%
\pgfpathmoveto{\pgfqpoint{0.000000in}{-0.041667in}}%
\pgfpathcurveto{\pgfqpoint{0.011050in}{-0.041667in}}{\pgfqpoint{0.021649in}{-0.037276in}}{\pgfqpoint{0.029463in}{-0.029463in}}%
\pgfpathcurveto{\pgfqpoint{0.037276in}{-0.021649in}}{\pgfqpoint{0.041667in}{-0.011050in}}{\pgfqpoint{0.041667in}{0.000000in}}%
\pgfpathcurveto{\pgfqpoint{0.041667in}{0.011050in}}{\pgfqpoint{0.037276in}{0.021649in}}{\pgfqpoint{0.029463in}{0.029463in}}%
\pgfpathcurveto{\pgfqpoint{0.021649in}{0.037276in}}{\pgfqpoint{0.011050in}{0.041667in}}{\pgfqpoint{0.000000in}{0.041667in}}%
\pgfpathcurveto{\pgfqpoint{-0.011050in}{0.041667in}}{\pgfqpoint{-0.021649in}{0.037276in}}{\pgfqpoint{-0.029463in}{0.029463in}}%
\pgfpathcurveto{\pgfqpoint{-0.037276in}{0.021649in}}{\pgfqpoint{-0.041667in}{0.011050in}}{\pgfqpoint{-0.041667in}{0.000000in}}%
\pgfpathcurveto{\pgfqpoint{-0.041667in}{-0.011050in}}{\pgfqpoint{-0.037276in}{-0.021649in}}{\pgfqpoint{-0.029463in}{-0.029463in}}%
\pgfpathcurveto{\pgfqpoint{-0.021649in}{-0.037276in}}{\pgfqpoint{-0.011050in}{-0.041667in}}{\pgfqpoint{0.000000in}{-0.041667in}}%
\pgfpathlineto{\pgfqpoint{0.000000in}{-0.041667in}}%
\pgfpathclose%
\pgfusepath{stroke,fill}%
}%
\begin{pgfscope}%
\pgfsys@transformshift{3.623497in}{3.122078in}%
\pgfsys@useobject{currentmarker}{}%
\end{pgfscope}%
\end{pgfscope}%
\begin{pgfscope}%
\definecolor{textcolor}{rgb}{0.000000,0.000000,0.000000}%
\pgfsetstrokecolor{textcolor}%
\pgfsetfillcolor{textcolor}%
\pgftext[x=3.873497in,y=3.085620in,left,base]{\color{textcolor}\sffamily\fontsize{10.000000}{12.000000}\selectfont NN}%
\end{pgfscope}%
\begin{pgfscope}%
\pgfsetbuttcap%
\pgfsetmiterjoin%
\definecolor{currentfill}{rgb}{0.121569,0.466667,0.705882}%
\pgfsetfillcolor{currentfill}%
\pgfsetlinewidth{0.000000pt}%
\definecolor{currentstroke}{rgb}{0.000000,0.000000,0.000000}%
\pgfsetstrokecolor{currentstroke}%
\pgfsetstrokeopacity{0.000000}%
\pgfsetdash{}{0pt}%
\pgfpathmoveto{\pgfqpoint{3.484608in}{2.881762in}}%
\pgfpathlineto{\pgfqpoint{3.762386in}{2.881762in}}%
\pgfpathlineto{\pgfqpoint{3.762386in}{2.978985in}}%
\pgfpathlineto{\pgfqpoint{3.484608in}{2.978985in}}%
\pgfpathlineto{\pgfqpoint{3.484608in}{2.881762in}}%
\pgfpathclose%
\pgfusepath{fill}%
\end{pgfscope}%
\begin{pgfscope}%
\definecolor{textcolor}{rgb}{0.000000,0.000000,0.000000}%
\pgfsetstrokecolor{textcolor}%
\pgfsetfillcolor{textcolor}%
\pgftext[x=3.873497in,y=2.881762in,left,base]{\color{textcolor}\sffamily\fontsize{10.000000}{12.000000}\selectfont Binomial distributon}%
\end{pgfscope}%
\end{pgfpicture}%
\makeatother%
\endgroup%

    \caption{Caption}
    \label{fig:my_label}
\end{figure}


\begin{figure}
%% Creator: Matplotlib, PGF backend
%%
%% To include the figure in your LaTeX document, write
%%   \input{<filename>.pgf}
%%
%% Make sure the required packages are loaded in your preamble
%%   \usepackage{pgf}
%%
%% Also ensure that all the required font packages are loaded; for instance,
%% the lmodern package is sometimes necessary when using math font.
%%   \usepackage{lmodern}
%%
%% Figures using additional raster images can only be included by \input if
%% they are in the same directory as the main LaTeX file. For loading figures
%% from other directories you can use the `import` package
%%   \usepackage{import}
%%
%% and then include the figures with
%%   \import{<path to file>}{<filename>.pgf}
%%
%% Matplotlib used the following preamble
%%
\begingroup%
\makeatletter%
\begin{pgfpicture}%
\pgfpathrectangle{\pgfpointorigin}{\pgfqpoint{6.000000in}{4.000000in}}%
\pgfusepath{use as bounding box, clip}%
\begin{pgfscope}%
\pgfsetbuttcap%
\pgfsetmiterjoin%
\pgfsetlinewidth{0.000000pt}%
\definecolor{currentstroke}{rgb}{1.000000,1.000000,1.000000}%
\pgfsetstrokecolor{currentstroke}%
\pgfsetstrokeopacity{0.000000}%
\pgfsetdash{}{0pt}%
\pgfpathmoveto{\pgfqpoint{0.000000in}{0.000000in}}%
\pgfpathlineto{\pgfqpoint{6.000000in}{0.000000in}}%
\pgfpathlineto{\pgfqpoint{6.000000in}{4.000000in}}%
\pgfpathlineto{\pgfqpoint{0.000000in}{4.000000in}}%
\pgfpathlineto{\pgfqpoint{0.000000in}{0.000000in}}%
\pgfpathclose%
\pgfusepath{}%
\end{pgfscope}%
\begin{pgfscope}%
\pgfsetbuttcap%
\pgfsetmiterjoin%
\definecolor{currentfill}{rgb}{1.000000,1.000000,1.000000}%
\pgfsetfillcolor{currentfill}%
\pgfsetlinewidth{0.000000pt}%
\definecolor{currentstroke}{rgb}{0.000000,0.000000,0.000000}%
\pgfsetstrokecolor{currentstroke}%
\pgfsetstrokeopacity{0.000000}%
\pgfsetdash{}{0pt}%
\pgfpathmoveto{\pgfqpoint{0.750000in}{0.500000in}}%
\pgfpathlineto{\pgfqpoint{5.400000in}{0.500000in}}%
\pgfpathlineto{\pgfqpoint{5.400000in}{3.520000in}}%
\pgfpathlineto{\pgfqpoint{0.750000in}{3.520000in}}%
\pgfpathlineto{\pgfqpoint{0.750000in}{0.500000in}}%
\pgfpathclose%
\pgfusepath{fill}%
\end{pgfscope}%
\begin{pgfscope}%
\pgfpathrectangle{\pgfqpoint{0.750000in}{0.500000in}}{\pgfqpoint{4.650000in}{3.020000in}}%
\pgfusepath{clip}%
\pgfsetbuttcap%
\pgfsetmiterjoin%
\definecolor{currentfill}{rgb}{0.121569,0.466667,0.705882}%
\pgfsetfillcolor{currentfill}%
\pgfsetlinewidth{0.000000pt}%
\definecolor{currentstroke}{rgb}{0.000000,0.000000,0.000000}%
\pgfsetstrokecolor{currentstroke}%
\pgfsetstrokeopacity{0.000000}%
\pgfsetdash{}{0pt}%
\pgfpathmoveto{\pgfqpoint{0.961364in}{0.500000in}}%
\pgfpathlineto{\pgfqpoint{0.987702in}{0.500000in}}%
\pgfpathlineto{\pgfqpoint{0.987702in}{0.500000in}}%
\pgfpathlineto{\pgfqpoint{0.961364in}{0.500000in}}%
\pgfpathlineto{\pgfqpoint{0.961364in}{0.500000in}}%
\pgfpathclose%
\pgfusepath{fill}%
\end{pgfscope}%
\begin{pgfscope}%
\pgfpathrectangle{\pgfqpoint{0.750000in}{0.500000in}}{\pgfqpoint{4.650000in}{3.020000in}}%
\pgfusepath{clip}%
\pgfsetbuttcap%
\pgfsetmiterjoin%
\definecolor{currentfill}{rgb}{0.121569,0.466667,0.705882}%
\pgfsetfillcolor{currentfill}%
\pgfsetlinewidth{0.000000pt}%
\definecolor{currentstroke}{rgb}{0.000000,0.000000,0.000000}%
\pgfsetstrokecolor{currentstroke}%
\pgfsetstrokeopacity{0.000000}%
\pgfsetdash{}{0pt}%
\pgfpathmoveto{\pgfqpoint{0.994286in}{0.500000in}}%
\pgfpathlineto{\pgfqpoint{1.020624in}{0.500000in}}%
\pgfpathlineto{\pgfqpoint{1.020624in}{0.500000in}}%
\pgfpathlineto{\pgfqpoint{0.994286in}{0.500000in}}%
\pgfpathlineto{\pgfqpoint{0.994286in}{0.500000in}}%
\pgfpathclose%
\pgfusepath{fill}%
\end{pgfscope}%
\begin{pgfscope}%
\pgfpathrectangle{\pgfqpoint{0.750000in}{0.500000in}}{\pgfqpoint{4.650000in}{3.020000in}}%
\pgfusepath{clip}%
\pgfsetbuttcap%
\pgfsetmiterjoin%
\definecolor{currentfill}{rgb}{0.121569,0.466667,0.705882}%
\pgfsetfillcolor{currentfill}%
\pgfsetlinewidth{0.000000pt}%
\definecolor{currentstroke}{rgb}{0.000000,0.000000,0.000000}%
\pgfsetstrokecolor{currentstroke}%
\pgfsetstrokeopacity{0.000000}%
\pgfsetdash{}{0pt}%
\pgfpathmoveto{\pgfqpoint{1.027209in}{0.500000in}}%
\pgfpathlineto{\pgfqpoint{1.053547in}{0.500000in}}%
\pgfpathlineto{\pgfqpoint{1.053547in}{0.500000in}}%
\pgfpathlineto{\pgfqpoint{1.027209in}{0.500000in}}%
\pgfpathlineto{\pgfqpoint{1.027209in}{0.500000in}}%
\pgfpathclose%
\pgfusepath{fill}%
\end{pgfscope}%
\begin{pgfscope}%
\pgfpathrectangle{\pgfqpoint{0.750000in}{0.500000in}}{\pgfqpoint{4.650000in}{3.020000in}}%
\pgfusepath{clip}%
\pgfsetbuttcap%
\pgfsetmiterjoin%
\definecolor{currentfill}{rgb}{0.121569,0.466667,0.705882}%
\pgfsetfillcolor{currentfill}%
\pgfsetlinewidth{0.000000pt}%
\definecolor{currentstroke}{rgb}{0.000000,0.000000,0.000000}%
\pgfsetstrokecolor{currentstroke}%
\pgfsetstrokeopacity{0.000000}%
\pgfsetdash{}{0pt}%
\pgfpathmoveto{\pgfqpoint{1.060132in}{0.500000in}}%
\pgfpathlineto{\pgfqpoint{1.086470in}{0.500000in}}%
\pgfpathlineto{\pgfqpoint{1.086470in}{0.500000in}}%
\pgfpathlineto{\pgfqpoint{1.060132in}{0.500000in}}%
\pgfpathlineto{\pgfqpoint{1.060132in}{0.500000in}}%
\pgfpathclose%
\pgfusepath{fill}%
\end{pgfscope}%
\begin{pgfscope}%
\pgfpathrectangle{\pgfqpoint{0.750000in}{0.500000in}}{\pgfqpoint{4.650000in}{3.020000in}}%
\pgfusepath{clip}%
\pgfsetbuttcap%
\pgfsetmiterjoin%
\definecolor{currentfill}{rgb}{0.121569,0.466667,0.705882}%
\pgfsetfillcolor{currentfill}%
\pgfsetlinewidth{0.000000pt}%
\definecolor{currentstroke}{rgb}{0.000000,0.000000,0.000000}%
\pgfsetstrokecolor{currentstroke}%
\pgfsetstrokeopacity{0.000000}%
\pgfsetdash{}{0pt}%
\pgfpathmoveto{\pgfqpoint{1.093054in}{0.500000in}}%
\pgfpathlineto{\pgfqpoint{1.119393in}{0.500000in}}%
\pgfpathlineto{\pgfqpoint{1.119393in}{0.500000in}}%
\pgfpathlineto{\pgfqpoint{1.093054in}{0.500000in}}%
\pgfpathlineto{\pgfqpoint{1.093054in}{0.500000in}}%
\pgfpathclose%
\pgfusepath{fill}%
\end{pgfscope}%
\begin{pgfscope}%
\pgfpathrectangle{\pgfqpoint{0.750000in}{0.500000in}}{\pgfqpoint{4.650000in}{3.020000in}}%
\pgfusepath{clip}%
\pgfsetbuttcap%
\pgfsetmiterjoin%
\definecolor{currentfill}{rgb}{0.121569,0.466667,0.705882}%
\pgfsetfillcolor{currentfill}%
\pgfsetlinewidth{0.000000pt}%
\definecolor{currentstroke}{rgb}{0.000000,0.000000,0.000000}%
\pgfsetstrokecolor{currentstroke}%
\pgfsetstrokeopacity{0.000000}%
\pgfsetdash{}{0pt}%
\pgfpathmoveto{\pgfqpoint{1.125977in}{0.500000in}}%
\pgfpathlineto{\pgfqpoint{1.152315in}{0.500000in}}%
\pgfpathlineto{\pgfqpoint{1.152315in}{0.500000in}}%
\pgfpathlineto{\pgfqpoint{1.125977in}{0.500000in}}%
\pgfpathlineto{\pgfqpoint{1.125977in}{0.500000in}}%
\pgfpathclose%
\pgfusepath{fill}%
\end{pgfscope}%
\begin{pgfscope}%
\pgfpathrectangle{\pgfqpoint{0.750000in}{0.500000in}}{\pgfqpoint{4.650000in}{3.020000in}}%
\pgfusepath{clip}%
\pgfsetbuttcap%
\pgfsetmiterjoin%
\definecolor{currentfill}{rgb}{0.121569,0.466667,0.705882}%
\pgfsetfillcolor{currentfill}%
\pgfsetlinewidth{0.000000pt}%
\definecolor{currentstroke}{rgb}{0.000000,0.000000,0.000000}%
\pgfsetstrokecolor{currentstroke}%
\pgfsetstrokeopacity{0.000000}%
\pgfsetdash{}{0pt}%
\pgfpathmoveto{\pgfqpoint{1.158900in}{0.500000in}}%
\pgfpathlineto{\pgfqpoint{1.185238in}{0.500000in}}%
\pgfpathlineto{\pgfqpoint{1.185238in}{0.500000in}}%
\pgfpathlineto{\pgfqpoint{1.158900in}{0.500000in}}%
\pgfpathlineto{\pgfqpoint{1.158900in}{0.500000in}}%
\pgfpathclose%
\pgfusepath{fill}%
\end{pgfscope}%
\begin{pgfscope}%
\pgfpathrectangle{\pgfqpoint{0.750000in}{0.500000in}}{\pgfqpoint{4.650000in}{3.020000in}}%
\pgfusepath{clip}%
\pgfsetbuttcap%
\pgfsetmiterjoin%
\definecolor{currentfill}{rgb}{0.121569,0.466667,0.705882}%
\pgfsetfillcolor{currentfill}%
\pgfsetlinewidth{0.000000pt}%
\definecolor{currentstroke}{rgb}{0.000000,0.000000,0.000000}%
\pgfsetstrokecolor{currentstroke}%
\pgfsetstrokeopacity{0.000000}%
\pgfsetdash{}{0pt}%
\pgfpathmoveto{\pgfqpoint{1.191822in}{0.500000in}}%
\pgfpathlineto{\pgfqpoint{1.218161in}{0.500000in}}%
\pgfpathlineto{\pgfqpoint{1.218161in}{0.500000in}}%
\pgfpathlineto{\pgfqpoint{1.191822in}{0.500000in}}%
\pgfpathlineto{\pgfqpoint{1.191822in}{0.500000in}}%
\pgfpathclose%
\pgfusepath{fill}%
\end{pgfscope}%
\begin{pgfscope}%
\pgfpathrectangle{\pgfqpoint{0.750000in}{0.500000in}}{\pgfqpoint{4.650000in}{3.020000in}}%
\pgfusepath{clip}%
\pgfsetbuttcap%
\pgfsetmiterjoin%
\definecolor{currentfill}{rgb}{0.121569,0.466667,0.705882}%
\pgfsetfillcolor{currentfill}%
\pgfsetlinewidth{0.000000pt}%
\definecolor{currentstroke}{rgb}{0.000000,0.000000,0.000000}%
\pgfsetstrokecolor{currentstroke}%
\pgfsetstrokeopacity{0.000000}%
\pgfsetdash{}{0pt}%
\pgfpathmoveto{\pgfqpoint{1.224745in}{0.500000in}}%
\pgfpathlineto{\pgfqpoint{1.251083in}{0.500000in}}%
\pgfpathlineto{\pgfqpoint{1.251083in}{0.500000in}}%
\pgfpathlineto{\pgfqpoint{1.224745in}{0.500000in}}%
\pgfpathlineto{\pgfqpoint{1.224745in}{0.500000in}}%
\pgfpathclose%
\pgfusepath{fill}%
\end{pgfscope}%
\begin{pgfscope}%
\pgfpathrectangle{\pgfqpoint{0.750000in}{0.500000in}}{\pgfqpoint{4.650000in}{3.020000in}}%
\pgfusepath{clip}%
\pgfsetbuttcap%
\pgfsetmiterjoin%
\definecolor{currentfill}{rgb}{0.121569,0.466667,0.705882}%
\pgfsetfillcolor{currentfill}%
\pgfsetlinewidth{0.000000pt}%
\definecolor{currentstroke}{rgb}{0.000000,0.000000,0.000000}%
\pgfsetstrokecolor{currentstroke}%
\pgfsetstrokeopacity{0.000000}%
\pgfsetdash{}{0pt}%
\pgfpathmoveto{\pgfqpoint{1.257668in}{0.500000in}}%
\pgfpathlineto{\pgfqpoint{1.284006in}{0.500000in}}%
\pgfpathlineto{\pgfqpoint{1.284006in}{0.500000in}}%
\pgfpathlineto{\pgfqpoint{1.257668in}{0.500000in}}%
\pgfpathlineto{\pgfqpoint{1.257668in}{0.500000in}}%
\pgfpathclose%
\pgfusepath{fill}%
\end{pgfscope}%
\begin{pgfscope}%
\pgfpathrectangle{\pgfqpoint{0.750000in}{0.500000in}}{\pgfqpoint{4.650000in}{3.020000in}}%
\pgfusepath{clip}%
\pgfsetbuttcap%
\pgfsetmiterjoin%
\definecolor{currentfill}{rgb}{0.121569,0.466667,0.705882}%
\pgfsetfillcolor{currentfill}%
\pgfsetlinewidth{0.000000pt}%
\definecolor{currentstroke}{rgb}{0.000000,0.000000,0.000000}%
\pgfsetstrokecolor{currentstroke}%
\pgfsetstrokeopacity{0.000000}%
\pgfsetdash{}{0pt}%
\pgfpathmoveto{\pgfqpoint{1.290590in}{0.500000in}}%
\pgfpathlineto{\pgfqpoint{1.316929in}{0.500000in}}%
\pgfpathlineto{\pgfqpoint{1.316929in}{0.500000in}}%
\pgfpathlineto{\pgfqpoint{1.290590in}{0.500000in}}%
\pgfpathlineto{\pgfqpoint{1.290590in}{0.500000in}}%
\pgfpathclose%
\pgfusepath{fill}%
\end{pgfscope}%
\begin{pgfscope}%
\pgfpathrectangle{\pgfqpoint{0.750000in}{0.500000in}}{\pgfqpoint{4.650000in}{3.020000in}}%
\pgfusepath{clip}%
\pgfsetbuttcap%
\pgfsetmiterjoin%
\definecolor{currentfill}{rgb}{0.121569,0.466667,0.705882}%
\pgfsetfillcolor{currentfill}%
\pgfsetlinewidth{0.000000pt}%
\definecolor{currentstroke}{rgb}{0.000000,0.000000,0.000000}%
\pgfsetstrokecolor{currentstroke}%
\pgfsetstrokeopacity{0.000000}%
\pgfsetdash{}{0pt}%
\pgfpathmoveto{\pgfqpoint{1.323513in}{0.500000in}}%
\pgfpathlineto{\pgfqpoint{1.349851in}{0.500000in}}%
\pgfpathlineto{\pgfqpoint{1.349851in}{0.500000in}}%
\pgfpathlineto{\pgfqpoint{1.323513in}{0.500000in}}%
\pgfpathlineto{\pgfqpoint{1.323513in}{0.500000in}}%
\pgfpathclose%
\pgfusepath{fill}%
\end{pgfscope}%
\begin{pgfscope}%
\pgfpathrectangle{\pgfqpoint{0.750000in}{0.500000in}}{\pgfqpoint{4.650000in}{3.020000in}}%
\pgfusepath{clip}%
\pgfsetbuttcap%
\pgfsetmiterjoin%
\definecolor{currentfill}{rgb}{0.121569,0.466667,0.705882}%
\pgfsetfillcolor{currentfill}%
\pgfsetlinewidth{0.000000pt}%
\definecolor{currentstroke}{rgb}{0.000000,0.000000,0.000000}%
\pgfsetstrokecolor{currentstroke}%
\pgfsetstrokeopacity{0.000000}%
\pgfsetdash{}{0pt}%
\pgfpathmoveto{\pgfqpoint{1.356436in}{0.500000in}}%
\pgfpathlineto{\pgfqpoint{1.382774in}{0.500000in}}%
\pgfpathlineto{\pgfqpoint{1.382774in}{0.500000in}}%
\pgfpathlineto{\pgfqpoint{1.356436in}{0.500000in}}%
\pgfpathlineto{\pgfqpoint{1.356436in}{0.500000in}}%
\pgfpathclose%
\pgfusepath{fill}%
\end{pgfscope}%
\begin{pgfscope}%
\pgfpathrectangle{\pgfqpoint{0.750000in}{0.500000in}}{\pgfqpoint{4.650000in}{3.020000in}}%
\pgfusepath{clip}%
\pgfsetbuttcap%
\pgfsetmiterjoin%
\definecolor{currentfill}{rgb}{0.121569,0.466667,0.705882}%
\pgfsetfillcolor{currentfill}%
\pgfsetlinewidth{0.000000pt}%
\definecolor{currentstroke}{rgb}{0.000000,0.000000,0.000000}%
\pgfsetstrokecolor{currentstroke}%
\pgfsetstrokeopacity{0.000000}%
\pgfsetdash{}{0pt}%
\pgfpathmoveto{\pgfqpoint{1.389359in}{0.500000in}}%
\pgfpathlineto{\pgfqpoint{1.415697in}{0.500000in}}%
\pgfpathlineto{\pgfqpoint{1.415697in}{0.500000in}}%
\pgfpathlineto{\pgfqpoint{1.389359in}{0.500000in}}%
\pgfpathlineto{\pgfqpoint{1.389359in}{0.500000in}}%
\pgfpathclose%
\pgfusepath{fill}%
\end{pgfscope}%
\begin{pgfscope}%
\pgfpathrectangle{\pgfqpoint{0.750000in}{0.500000in}}{\pgfqpoint{4.650000in}{3.020000in}}%
\pgfusepath{clip}%
\pgfsetbuttcap%
\pgfsetmiterjoin%
\definecolor{currentfill}{rgb}{0.121569,0.466667,0.705882}%
\pgfsetfillcolor{currentfill}%
\pgfsetlinewidth{0.000000pt}%
\definecolor{currentstroke}{rgb}{0.000000,0.000000,0.000000}%
\pgfsetstrokecolor{currentstroke}%
\pgfsetstrokeopacity{0.000000}%
\pgfsetdash{}{0pt}%
\pgfpathmoveto{\pgfqpoint{1.422281in}{0.500000in}}%
\pgfpathlineto{\pgfqpoint{1.448619in}{0.500000in}}%
\pgfpathlineto{\pgfqpoint{1.448619in}{0.500000in}}%
\pgfpathlineto{\pgfqpoint{1.422281in}{0.500000in}}%
\pgfpathlineto{\pgfqpoint{1.422281in}{0.500000in}}%
\pgfpathclose%
\pgfusepath{fill}%
\end{pgfscope}%
\begin{pgfscope}%
\pgfpathrectangle{\pgfqpoint{0.750000in}{0.500000in}}{\pgfqpoint{4.650000in}{3.020000in}}%
\pgfusepath{clip}%
\pgfsetbuttcap%
\pgfsetmiterjoin%
\definecolor{currentfill}{rgb}{0.121569,0.466667,0.705882}%
\pgfsetfillcolor{currentfill}%
\pgfsetlinewidth{0.000000pt}%
\definecolor{currentstroke}{rgb}{0.000000,0.000000,0.000000}%
\pgfsetstrokecolor{currentstroke}%
\pgfsetstrokeopacity{0.000000}%
\pgfsetdash{}{0pt}%
\pgfpathmoveto{\pgfqpoint{1.455204in}{0.500000in}}%
\pgfpathlineto{\pgfqpoint{1.481542in}{0.500000in}}%
\pgfpathlineto{\pgfqpoint{1.481542in}{0.500000in}}%
\pgfpathlineto{\pgfqpoint{1.455204in}{0.500000in}}%
\pgfpathlineto{\pgfqpoint{1.455204in}{0.500000in}}%
\pgfpathclose%
\pgfusepath{fill}%
\end{pgfscope}%
\begin{pgfscope}%
\pgfpathrectangle{\pgfqpoint{0.750000in}{0.500000in}}{\pgfqpoint{4.650000in}{3.020000in}}%
\pgfusepath{clip}%
\pgfsetbuttcap%
\pgfsetmiterjoin%
\definecolor{currentfill}{rgb}{0.121569,0.466667,0.705882}%
\pgfsetfillcolor{currentfill}%
\pgfsetlinewidth{0.000000pt}%
\definecolor{currentstroke}{rgb}{0.000000,0.000000,0.000000}%
\pgfsetstrokecolor{currentstroke}%
\pgfsetstrokeopacity{0.000000}%
\pgfsetdash{}{0pt}%
\pgfpathmoveto{\pgfqpoint{1.488127in}{0.500000in}}%
\pgfpathlineto{\pgfqpoint{1.514465in}{0.500000in}}%
\pgfpathlineto{\pgfqpoint{1.514465in}{0.500000in}}%
\pgfpathlineto{\pgfqpoint{1.488127in}{0.500000in}}%
\pgfpathlineto{\pgfqpoint{1.488127in}{0.500000in}}%
\pgfpathclose%
\pgfusepath{fill}%
\end{pgfscope}%
\begin{pgfscope}%
\pgfpathrectangle{\pgfqpoint{0.750000in}{0.500000in}}{\pgfqpoint{4.650000in}{3.020000in}}%
\pgfusepath{clip}%
\pgfsetbuttcap%
\pgfsetmiterjoin%
\definecolor{currentfill}{rgb}{0.121569,0.466667,0.705882}%
\pgfsetfillcolor{currentfill}%
\pgfsetlinewidth{0.000000pt}%
\definecolor{currentstroke}{rgb}{0.000000,0.000000,0.000000}%
\pgfsetstrokecolor{currentstroke}%
\pgfsetstrokeopacity{0.000000}%
\pgfsetdash{}{0pt}%
\pgfpathmoveto{\pgfqpoint{1.521049in}{0.500000in}}%
\pgfpathlineto{\pgfqpoint{1.547387in}{0.500000in}}%
\pgfpathlineto{\pgfqpoint{1.547387in}{0.500000in}}%
\pgfpathlineto{\pgfqpoint{1.521049in}{0.500000in}}%
\pgfpathlineto{\pgfqpoint{1.521049in}{0.500000in}}%
\pgfpathclose%
\pgfusepath{fill}%
\end{pgfscope}%
\begin{pgfscope}%
\pgfpathrectangle{\pgfqpoint{0.750000in}{0.500000in}}{\pgfqpoint{4.650000in}{3.020000in}}%
\pgfusepath{clip}%
\pgfsetbuttcap%
\pgfsetmiterjoin%
\definecolor{currentfill}{rgb}{0.121569,0.466667,0.705882}%
\pgfsetfillcolor{currentfill}%
\pgfsetlinewidth{0.000000pt}%
\definecolor{currentstroke}{rgb}{0.000000,0.000000,0.000000}%
\pgfsetstrokecolor{currentstroke}%
\pgfsetstrokeopacity{0.000000}%
\pgfsetdash{}{0pt}%
\pgfpathmoveto{\pgfqpoint{1.553972in}{0.500000in}}%
\pgfpathlineto{\pgfqpoint{1.580310in}{0.500000in}}%
\pgfpathlineto{\pgfqpoint{1.580310in}{0.500000in}}%
\pgfpathlineto{\pgfqpoint{1.553972in}{0.500000in}}%
\pgfpathlineto{\pgfqpoint{1.553972in}{0.500000in}}%
\pgfpathclose%
\pgfusepath{fill}%
\end{pgfscope}%
\begin{pgfscope}%
\pgfpathrectangle{\pgfqpoint{0.750000in}{0.500000in}}{\pgfqpoint{4.650000in}{3.020000in}}%
\pgfusepath{clip}%
\pgfsetbuttcap%
\pgfsetmiterjoin%
\definecolor{currentfill}{rgb}{0.121569,0.466667,0.705882}%
\pgfsetfillcolor{currentfill}%
\pgfsetlinewidth{0.000000pt}%
\definecolor{currentstroke}{rgb}{0.000000,0.000000,0.000000}%
\pgfsetstrokecolor{currentstroke}%
\pgfsetstrokeopacity{0.000000}%
\pgfsetdash{}{0pt}%
\pgfpathmoveto{\pgfqpoint{1.586895in}{0.500000in}}%
\pgfpathlineto{\pgfqpoint{1.613233in}{0.500000in}}%
\pgfpathlineto{\pgfqpoint{1.613233in}{0.500000in}}%
\pgfpathlineto{\pgfqpoint{1.586895in}{0.500000in}}%
\pgfpathlineto{\pgfqpoint{1.586895in}{0.500000in}}%
\pgfpathclose%
\pgfusepath{fill}%
\end{pgfscope}%
\begin{pgfscope}%
\pgfpathrectangle{\pgfqpoint{0.750000in}{0.500000in}}{\pgfqpoint{4.650000in}{3.020000in}}%
\pgfusepath{clip}%
\pgfsetbuttcap%
\pgfsetmiterjoin%
\definecolor{currentfill}{rgb}{0.121569,0.466667,0.705882}%
\pgfsetfillcolor{currentfill}%
\pgfsetlinewidth{0.000000pt}%
\definecolor{currentstroke}{rgb}{0.000000,0.000000,0.000000}%
\pgfsetstrokecolor{currentstroke}%
\pgfsetstrokeopacity{0.000000}%
\pgfsetdash{}{0pt}%
\pgfpathmoveto{\pgfqpoint{1.619817in}{0.500000in}}%
\pgfpathlineto{\pgfqpoint{1.646155in}{0.500000in}}%
\pgfpathlineto{\pgfqpoint{1.646155in}{0.500000in}}%
\pgfpathlineto{\pgfqpoint{1.619817in}{0.500000in}}%
\pgfpathlineto{\pgfqpoint{1.619817in}{0.500000in}}%
\pgfpathclose%
\pgfusepath{fill}%
\end{pgfscope}%
\begin{pgfscope}%
\pgfpathrectangle{\pgfqpoint{0.750000in}{0.500000in}}{\pgfqpoint{4.650000in}{3.020000in}}%
\pgfusepath{clip}%
\pgfsetbuttcap%
\pgfsetmiterjoin%
\definecolor{currentfill}{rgb}{0.121569,0.466667,0.705882}%
\pgfsetfillcolor{currentfill}%
\pgfsetlinewidth{0.000000pt}%
\definecolor{currentstroke}{rgb}{0.000000,0.000000,0.000000}%
\pgfsetstrokecolor{currentstroke}%
\pgfsetstrokeopacity{0.000000}%
\pgfsetdash{}{0pt}%
\pgfpathmoveto{\pgfqpoint{1.652740in}{0.500000in}}%
\pgfpathlineto{\pgfqpoint{1.679078in}{0.500000in}}%
\pgfpathlineto{\pgfqpoint{1.679078in}{0.500000in}}%
\pgfpathlineto{\pgfqpoint{1.652740in}{0.500000in}}%
\pgfpathlineto{\pgfqpoint{1.652740in}{0.500000in}}%
\pgfpathclose%
\pgfusepath{fill}%
\end{pgfscope}%
\begin{pgfscope}%
\pgfpathrectangle{\pgfqpoint{0.750000in}{0.500000in}}{\pgfqpoint{4.650000in}{3.020000in}}%
\pgfusepath{clip}%
\pgfsetbuttcap%
\pgfsetmiterjoin%
\definecolor{currentfill}{rgb}{0.121569,0.466667,0.705882}%
\pgfsetfillcolor{currentfill}%
\pgfsetlinewidth{0.000000pt}%
\definecolor{currentstroke}{rgb}{0.000000,0.000000,0.000000}%
\pgfsetstrokecolor{currentstroke}%
\pgfsetstrokeopacity{0.000000}%
\pgfsetdash{}{0pt}%
\pgfpathmoveto{\pgfqpoint{1.685663in}{0.500000in}}%
\pgfpathlineto{\pgfqpoint{1.712001in}{0.500000in}}%
\pgfpathlineto{\pgfqpoint{1.712001in}{0.500000in}}%
\pgfpathlineto{\pgfqpoint{1.685663in}{0.500000in}}%
\pgfpathlineto{\pgfqpoint{1.685663in}{0.500000in}}%
\pgfpathclose%
\pgfusepath{fill}%
\end{pgfscope}%
\begin{pgfscope}%
\pgfpathrectangle{\pgfqpoint{0.750000in}{0.500000in}}{\pgfqpoint{4.650000in}{3.020000in}}%
\pgfusepath{clip}%
\pgfsetbuttcap%
\pgfsetmiterjoin%
\definecolor{currentfill}{rgb}{0.121569,0.466667,0.705882}%
\pgfsetfillcolor{currentfill}%
\pgfsetlinewidth{0.000000pt}%
\definecolor{currentstroke}{rgb}{0.000000,0.000000,0.000000}%
\pgfsetstrokecolor{currentstroke}%
\pgfsetstrokeopacity{0.000000}%
\pgfsetdash{}{0pt}%
\pgfpathmoveto{\pgfqpoint{1.718585in}{0.500000in}}%
\pgfpathlineto{\pgfqpoint{1.744924in}{0.500000in}}%
\pgfpathlineto{\pgfqpoint{1.744924in}{0.500000in}}%
\pgfpathlineto{\pgfqpoint{1.718585in}{0.500000in}}%
\pgfpathlineto{\pgfqpoint{1.718585in}{0.500000in}}%
\pgfpathclose%
\pgfusepath{fill}%
\end{pgfscope}%
\begin{pgfscope}%
\pgfpathrectangle{\pgfqpoint{0.750000in}{0.500000in}}{\pgfqpoint{4.650000in}{3.020000in}}%
\pgfusepath{clip}%
\pgfsetbuttcap%
\pgfsetmiterjoin%
\definecolor{currentfill}{rgb}{0.121569,0.466667,0.705882}%
\pgfsetfillcolor{currentfill}%
\pgfsetlinewidth{0.000000pt}%
\definecolor{currentstroke}{rgb}{0.000000,0.000000,0.000000}%
\pgfsetstrokecolor{currentstroke}%
\pgfsetstrokeopacity{0.000000}%
\pgfsetdash{}{0pt}%
\pgfpathmoveto{\pgfqpoint{1.751508in}{0.500000in}}%
\pgfpathlineto{\pgfqpoint{1.777846in}{0.500000in}}%
\pgfpathlineto{\pgfqpoint{1.777846in}{0.500000in}}%
\pgfpathlineto{\pgfqpoint{1.751508in}{0.500000in}}%
\pgfpathlineto{\pgfqpoint{1.751508in}{0.500000in}}%
\pgfpathclose%
\pgfusepath{fill}%
\end{pgfscope}%
\begin{pgfscope}%
\pgfpathrectangle{\pgfqpoint{0.750000in}{0.500000in}}{\pgfqpoint{4.650000in}{3.020000in}}%
\pgfusepath{clip}%
\pgfsetbuttcap%
\pgfsetmiterjoin%
\definecolor{currentfill}{rgb}{0.121569,0.466667,0.705882}%
\pgfsetfillcolor{currentfill}%
\pgfsetlinewidth{0.000000pt}%
\definecolor{currentstroke}{rgb}{0.000000,0.000000,0.000000}%
\pgfsetstrokecolor{currentstroke}%
\pgfsetstrokeopacity{0.000000}%
\pgfsetdash{}{0pt}%
\pgfpathmoveto{\pgfqpoint{1.784431in}{0.500000in}}%
\pgfpathlineto{\pgfqpoint{1.810769in}{0.500000in}}%
\pgfpathlineto{\pgfqpoint{1.810769in}{0.500000in}}%
\pgfpathlineto{\pgfqpoint{1.784431in}{0.500000in}}%
\pgfpathlineto{\pgfqpoint{1.784431in}{0.500000in}}%
\pgfpathclose%
\pgfusepath{fill}%
\end{pgfscope}%
\begin{pgfscope}%
\pgfpathrectangle{\pgfqpoint{0.750000in}{0.500000in}}{\pgfqpoint{4.650000in}{3.020000in}}%
\pgfusepath{clip}%
\pgfsetbuttcap%
\pgfsetmiterjoin%
\definecolor{currentfill}{rgb}{0.121569,0.466667,0.705882}%
\pgfsetfillcolor{currentfill}%
\pgfsetlinewidth{0.000000pt}%
\definecolor{currentstroke}{rgb}{0.000000,0.000000,0.000000}%
\pgfsetstrokecolor{currentstroke}%
\pgfsetstrokeopacity{0.000000}%
\pgfsetdash{}{0pt}%
\pgfpathmoveto{\pgfqpoint{1.817353in}{0.500000in}}%
\pgfpathlineto{\pgfqpoint{1.843692in}{0.500000in}}%
\pgfpathlineto{\pgfqpoint{1.843692in}{0.500000in}}%
\pgfpathlineto{\pgfqpoint{1.817353in}{0.500000in}}%
\pgfpathlineto{\pgfqpoint{1.817353in}{0.500000in}}%
\pgfpathclose%
\pgfusepath{fill}%
\end{pgfscope}%
\begin{pgfscope}%
\pgfpathrectangle{\pgfqpoint{0.750000in}{0.500000in}}{\pgfqpoint{4.650000in}{3.020000in}}%
\pgfusepath{clip}%
\pgfsetbuttcap%
\pgfsetmiterjoin%
\definecolor{currentfill}{rgb}{0.121569,0.466667,0.705882}%
\pgfsetfillcolor{currentfill}%
\pgfsetlinewidth{0.000000pt}%
\definecolor{currentstroke}{rgb}{0.000000,0.000000,0.000000}%
\pgfsetstrokecolor{currentstroke}%
\pgfsetstrokeopacity{0.000000}%
\pgfsetdash{}{0pt}%
\pgfpathmoveto{\pgfqpoint{1.850276in}{0.500000in}}%
\pgfpathlineto{\pgfqpoint{1.876614in}{0.500000in}}%
\pgfpathlineto{\pgfqpoint{1.876614in}{0.500000in}}%
\pgfpathlineto{\pgfqpoint{1.850276in}{0.500000in}}%
\pgfpathlineto{\pgfqpoint{1.850276in}{0.500000in}}%
\pgfpathclose%
\pgfusepath{fill}%
\end{pgfscope}%
\begin{pgfscope}%
\pgfpathrectangle{\pgfqpoint{0.750000in}{0.500000in}}{\pgfqpoint{4.650000in}{3.020000in}}%
\pgfusepath{clip}%
\pgfsetbuttcap%
\pgfsetmiterjoin%
\definecolor{currentfill}{rgb}{0.121569,0.466667,0.705882}%
\pgfsetfillcolor{currentfill}%
\pgfsetlinewidth{0.000000pt}%
\definecolor{currentstroke}{rgb}{0.000000,0.000000,0.000000}%
\pgfsetstrokecolor{currentstroke}%
\pgfsetstrokeopacity{0.000000}%
\pgfsetdash{}{0pt}%
\pgfpathmoveto{\pgfqpoint{1.883199in}{0.500000in}}%
\pgfpathlineto{\pgfqpoint{1.909537in}{0.500000in}}%
\pgfpathlineto{\pgfqpoint{1.909537in}{0.500000in}}%
\pgfpathlineto{\pgfqpoint{1.883199in}{0.500000in}}%
\pgfpathlineto{\pgfqpoint{1.883199in}{0.500000in}}%
\pgfpathclose%
\pgfusepath{fill}%
\end{pgfscope}%
\begin{pgfscope}%
\pgfpathrectangle{\pgfqpoint{0.750000in}{0.500000in}}{\pgfqpoint{4.650000in}{3.020000in}}%
\pgfusepath{clip}%
\pgfsetbuttcap%
\pgfsetmiterjoin%
\definecolor{currentfill}{rgb}{0.121569,0.466667,0.705882}%
\pgfsetfillcolor{currentfill}%
\pgfsetlinewidth{0.000000pt}%
\definecolor{currentstroke}{rgb}{0.000000,0.000000,0.000000}%
\pgfsetstrokecolor{currentstroke}%
\pgfsetstrokeopacity{0.000000}%
\pgfsetdash{}{0pt}%
\pgfpathmoveto{\pgfqpoint{1.916121in}{0.500000in}}%
\pgfpathlineto{\pgfqpoint{1.942460in}{0.500000in}}%
\pgfpathlineto{\pgfqpoint{1.942460in}{0.500000in}}%
\pgfpathlineto{\pgfqpoint{1.916121in}{0.500000in}}%
\pgfpathlineto{\pgfqpoint{1.916121in}{0.500000in}}%
\pgfpathclose%
\pgfusepath{fill}%
\end{pgfscope}%
\begin{pgfscope}%
\pgfpathrectangle{\pgfqpoint{0.750000in}{0.500000in}}{\pgfqpoint{4.650000in}{3.020000in}}%
\pgfusepath{clip}%
\pgfsetbuttcap%
\pgfsetmiterjoin%
\definecolor{currentfill}{rgb}{0.121569,0.466667,0.705882}%
\pgfsetfillcolor{currentfill}%
\pgfsetlinewidth{0.000000pt}%
\definecolor{currentstroke}{rgb}{0.000000,0.000000,0.000000}%
\pgfsetstrokecolor{currentstroke}%
\pgfsetstrokeopacity{0.000000}%
\pgfsetdash{}{0pt}%
\pgfpathmoveto{\pgfqpoint{1.949044in}{0.500000in}}%
\pgfpathlineto{\pgfqpoint{1.975382in}{0.500000in}}%
\pgfpathlineto{\pgfqpoint{1.975382in}{0.500000in}}%
\pgfpathlineto{\pgfqpoint{1.949044in}{0.500000in}}%
\pgfpathlineto{\pgfqpoint{1.949044in}{0.500000in}}%
\pgfpathclose%
\pgfusepath{fill}%
\end{pgfscope}%
\begin{pgfscope}%
\pgfpathrectangle{\pgfqpoint{0.750000in}{0.500000in}}{\pgfqpoint{4.650000in}{3.020000in}}%
\pgfusepath{clip}%
\pgfsetbuttcap%
\pgfsetmiterjoin%
\definecolor{currentfill}{rgb}{0.121569,0.466667,0.705882}%
\pgfsetfillcolor{currentfill}%
\pgfsetlinewidth{0.000000pt}%
\definecolor{currentstroke}{rgb}{0.000000,0.000000,0.000000}%
\pgfsetstrokecolor{currentstroke}%
\pgfsetstrokeopacity{0.000000}%
\pgfsetdash{}{0pt}%
\pgfpathmoveto{\pgfqpoint{1.981967in}{0.500000in}}%
\pgfpathlineto{\pgfqpoint{2.008305in}{0.500000in}}%
\pgfpathlineto{\pgfqpoint{2.008305in}{0.500000in}}%
\pgfpathlineto{\pgfqpoint{1.981967in}{0.500000in}}%
\pgfpathlineto{\pgfqpoint{1.981967in}{0.500000in}}%
\pgfpathclose%
\pgfusepath{fill}%
\end{pgfscope}%
\begin{pgfscope}%
\pgfpathrectangle{\pgfqpoint{0.750000in}{0.500000in}}{\pgfqpoint{4.650000in}{3.020000in}}%
\pgfusepath{clip}%
\pgfsetbuttcap%
\pgfsetmiterjoin%
\definecolor{currentfill}{rgb}{0.121569,0.466667,0.705882}%
\pgfsetfillcolor{currentfill}%
\pgfsetlinewidth{0.000000pt}%
\definecolor{currentstroke}{rgb}{0.000000,0.000000,0.000000}%
\pgfsetstrokecolor{currentstroke}%
\pgfsetstrokeopacity{0.000000}%
\pgfsetdash{}{0pt}%
\pgfpathmoveto{\pgfqpoint{2.014890in}{0.500000in}}%
\pgfpathlineto{\pgfqpoint{2.041228in}{0.500000in}}%
\pgfpathlineto{\pgfqpoint{2.041228in}{0.500000in}}%
\pgfpathlineto{\pgfqpoint{2.014890in}{0.500000in}}%
\pgfpathlineto{\pgfqpoint{2.014890in}{0.500000in}}%
\pgfpathclose%
\pgfusepath{fill}%
\end{pgfscope}%
\begin{pgfscope}%
\pgfpathrectangle{\pgfqpoint{0.750000in}{0.500000in}}{\pgfqpoint{4.650000in}{3.020000in}}%
\pgfusepath{clip}%
\pgfsetbuttcap%
\pgfsetmiterjoin%
\definecolor{currentfill}{rgb}{0.121569,0.466667,0.705882}%
\pgfsetfillcolor{currentfill}%
\pgfsetlinewidth{0.000000pt}%
\definecolor{currentstroke}{rgb}{0.000000,0.000000,0.000000}%
\pgfsetstrokecolor{currentstroke}%
\pgfsetstrokeopacity{0.000000}%
\pgfsetdash{}{0pt}%
\pgfpathmoveto{\pgfqpoint{2.047812in}{0.500000in}}%
\pgfpathlineto{\pgfqpoint{2.074150in}{0.500000in}}%
\pgfpathlineto{\pgfqpoint{2.074150in}{0.500001in}}%
\pgfpathlineto{\pgfqpoint{2.047812in}{0.500001in}}%
\pgfpathlineto{\pgfqpoint{2.047812in}{0.500000in}}%
\pgfpathclose%
\pgfusepath{fill}%
\end{pgfscope}%
\begin{pgfscope}%
\pgfpathrectangle{\pgfqpoint{0.750000in}{0.500000in}}{\pgfqpoint{4.650000in}{3.020000in}}%
\pgfusepath{clip}%
\pgfsetbuttcap%
\pgfsetmiterjoin%
\definecolor{currentfill}{rgb}{0.121569,0.466667,0.705882}%
\pgfsetfillcolor{currentfill}%
\pgfsetlinewidth{0.000000pt}%
\definecolor{currentstroke}{rgb}{0.000000,0.000000,0.000000}%
\pgfsetstrokecolor{currentstroke}%
\pgfsetstrokeopacity{0.000000}%
\pgfsetdash{}{0pt}%
\pgfpathmoveto{\pgfqpoint{2.080735in}{0.500000in}}%
\pgfpathlineto{\pgfqpoint{2.107073in}{0.500000in}}%
\pgfpathlineto{\pgfqpoint{2.107073in}{0.500001in}}%
\pgfpathlineto{\pgfqpoint{2.080735in}{0.500001in}}%
\pgfpathlineto{\pgfqpoint{2.080735in}{0.500000in}}%
\pgfpathclose%
\pgfusepath{fill}%
\end{pgfscope}%
\begin{pgfscope}%
\pgfpathrectangle{\pgfqpoint{0.750000in}{0.500000in}}{\pgfqpoint{4.650000in}{3.020000in}}%
\pgfusepath{clip}%
\pgfsetbuttcap%
\pgfsetmiterjoin%
\definecolor{currentfill}{rgb}{0.121569,0.466667,0.705882}%
\pgfsetfillcolor{currentfill}%
\pgfsetlinewidth{0.000000pt}%
\definecolor{currentstroke}{rgb}{0.000000,0.000000,0.000000}%
\pgfsetstrokecolor{currentstroke}%
\pgfsetstrokeopacity{0.000000}%
\pgfsetdash{}{0pt}%
\pgfpathmoveto{\pgfqpoint{2.113658in}{0.500000in}}%
\pgfpathlineto{\pgfqpoint{2.139996in}{0.500000in}}%
\pgfpathlineto{\pgfqpoint{2.139996in}{0.500004in}}%
\pgfpathlineto{\pgfqpoint{2.113658in}{0.500004in}}%
\pgfpathlineto{\pgfqpoint{2.113658in}{0.500000in}}%
\pgfpathclose%
\pgfusepath{fill}%
\end{pgfscope}%
\begin{pgfscope}%
\pgfpathrectangle{\pgfqpoint{0.750000in}{0.500000in}}{\pgfqpoint{4.650000in}{3.020000in}}%
\pgfusepath{clip}%
\pgfsetbuttcap%
\pgfsetmiterjoin%
\definecolor{currentfill}{rgb}{0.121569,0.466667,0.705882}%
\pgfsetfillcolor{currentfill}%
\pgfsetlinewidth{0.000000pt}%
\definecolor{currentstroke}{rgb}{0.000000,0.000000,0.000000}%
\pgfsetstrokecolor{currentstroke}%
\pgfsetstrokeopacity{0.000000}%
\pgfsetdash{}{0pt}%
\pgfpathmoveto{\pgfqpoint{2.146580in}{0.500000in}}%
\pgfpathlineto{\pgfqpoint{2.172918in}{0.500000in}}%
\pgfpathlineto{\pgfqpoint{2.172918in}{0.500010in}}%
\pgfpathlineto{\pgfqpoint{2.146580in}{0.500010in}}%
\pgfpathlineto{\pgfqpoint{2.146580in}{0.500000in}}%
\pgfpathclose%
\pgfusepath{fill}%
\end{pgfscope}%
\begin{pgfscope}%
\pgfpathrectangle{\pgfqpoint{0.750000in}{0.500000in}}{\pgfqpoint{4.650000in}{3.020000in}}%
\pgfusepath{clip}%
\pgfsetbuttcap%
\pgfsetmiterjoin%
\definecolor{currentfill}{rgb}{0.121569,0.466667,0.705882}%
\pgfsetfillcolor{currentfill}%
\pgfsetlinewidth{0.000000pt}%
\definecolor{currentstroke}{rgb}{0.000000,0.000000,0.000000}%
\pgfsetstrokecolor{currentstroke}%
\pgfsetstrokeopacity{0.000000}%
\pgfsetdash{}{0pt}%
\pgfpathmoveto{\pgfqpoint{2.179503in}{0.500000in}}%
\pgfpathlineto{\pgfqpoint{2.205841in}{0.500000in}}%
\pgfpathlineto{\pgfqpoint{2.205841in}{0.500025in}}%
\pgfpathlineto{\pgfqpoint{2.179503in}{0.500025in}}%
\pgfpathlineto{\pgfqpoint{2.179503in}{0.500000in}}%
\pgfpathclose%
\pgfusepath{fill}%
\end{pgfscope}%
\begin{pgfscope}%
\pgfpathrectangle{\pgfqpoint{0.750000in}{0.500000in}}{\pgfqpoint{4.650000in}{3.020000in}}%
\pgfusepath{clip}%
\pgfsetbuttcap%
\pgfsetmiterjoin%
\definecolor{currentfill}{rgb}{0.121569,0.466667,0.705882}%
\pgfsetfillcolor{currentfill}%
\pgfsetlinewidth{0.000000pt}%
\definecolor{currentstroke}{rgb}{0.000000,0.000000,0.000000}%
\pgfsetstrokecolor{currentstroke}%
\pgfsetstrokeopacity{0.000000}%
\pgfsetdash{}{0pt}%
\pgfpathmoveto{\pgfqpoint{2.212426in}{0.500000in}}%
\pgfpathlineto{\pgfqpoint{2.238764in}{0.500000in}}%
\pgfpathlineto{\pgfqpoint{2.238764in}{0.500060in}}%
\pgfpathlineto{\pgfqpoint{2.212426in}{0.500060in}}%
\pgfpathlineto{\pgfqpoint{2.212426in}{0.500000in}}%
\pgfpathclose%
\pgfusepath{fill}%
\end{pgfscope}%
\begin{pgfscope}%
\pgfpathrectangle{\pgfqpoint{0.750000in}{0.500000in}}{\pgfqpoint{4.650000in}{3.020000in}}%
\pgfusepath{clip}%
\pgfsetbuttcap%
\pgfsetmiterjoin%
\definecolor{currentfill}{rgb}{0.121569,0.466667,0.705882}%
\pgfsetfillcolor{currentfill}%
\pgfsetlinewidth{0.000000pt}%
\definecolor{currentstroke}{rgb}{0.000000,0.000000,0.000000}%
\pgfsetstrokecolor{currentstroke}%
\pgfsetstrokeopacity{0.000000}%
\pgfsetdash{}{0pt}%
\pgfpathmoveto{\pgfqpoint{2.245348in}{0.500000in}}%
\pgfpathlineto{\pgfqpoint{2.271686in}{0.500000in}}%
\pgfpathlineto{\pgfqpoint{2.271686in}{0.500138in}}%
\pgfpathlineto{\pgfqpoint{2.245348in}{0.500138in}}%
\pgfpathlineto{\pgfqpoint{2.245348in}{0.500000in}}%
\pgfpathclose%
\pgfusepath{fill}%
\end{pgfscope}%
\begin{pgfscope}%
\pgfpathrectangle{\pgfqpoint{0.750000in}{0.500000in}}{\pgfqpoint{4.650000in}{3.020000in}}%
\pgfusepath{clip}%
\pgfsetbuttcap%
\pgfsetmiterjoin%
\definecolor{currentfill}{rgb}{0.121569,0.466667,0.705882}%
\pgfsetfillcolor{currentfill}%
\pgfsetlinewidth{0.000000pt}%
\definecolor{currentstroke}{rgb}{0.000000,0.000000,0.000000}%
\pgfsetstrokecolor{currentstroke}%
\pgfsetstrokeopacity{0.000000}%
\pgfsetdash{}{0pt}%
\pgfpathmoveto{\pgfqpoint{2.278271in}{0.500000in}}%
\pgfpathlineto{\pgfqpoint{2.304609in}{0.500000in}}%
\pgfpathlineto{\pgfqpoint{2.304609in}{0.500306in}}%
\pgfpathlineto{\pgfqpoint{2.278271in}{0.500306in}}%
\pgfpathlineto{\pgfqpoint{2.278271in}{0.500000in}}%
\pgfpathclose%
\pgfusepath{fill}%
\end{pgfscope}%
\begin{pgfscope}%
\pgfpathrectangle{\pgfqpoint{0.750000in}{0.500000in}}{\pgfqpoint{4.650000in}{3.020000in}}%
\pgfusepath{clip}%
\pgfsetbuttcap%
\pgfsetmiterjoin%
\definecolor{currentfill}{rgb}{0.121569,0.466667,0.705882}%
\pgfsetfillcolor{currentfill}%
\pgfsetlinewidth{0.000000pt}%
\definecolor{currentstroke}{rgb}{0.000000,0.000000,0.000000}%
\pgfsetstrokecolor{currentstroke}%
\pgfsetstrokeopacity{0.000000}%
\pgfsetdash{}{0pt}%
\pgfpathmoveto{\pgfqpoint{2.311194in}{0.500000in}}%
\pgfpathlineto{\pgfqpoint{2.337532in}{0.500000in}}%
\pgfpathlineto{\pgfqpoint{2.337532in}{0.500657in}}%
\pgfpathlineto{\pgfqpoint{2.311194in}{0.500657in}}%
\pgfpathlineto{\pgfqpoint{2.311194in}{0.500000in}}%
\pgfpathclose%
\pgfusepath{fill}%
\end{pgfscope}%
\begin{pgfscope}%
\pgfpathrectangle{\pgfqpoint{0.750000in}{0.500000in}}{\pgfqpoint{4.650000in}{3.020000in}}%
\pgfusepath{clip}%
\pgfsetbuttcap%
\pgfsetmiterjoin%
\definecolor{currentfill}{rgb}{0.121569,0.466667,0.705882}%
\pgfsetfillcolor{currentfill}%
\pgfsetlinewidth{0.000000pt}%
\definecolor{currentstroke}{rgb}{0.000000,0.000000,0.000000}%
\pgfsetstrokecolor{currentstroke}%
\pgfsetstrokeopacity{0.000000}%
\pgfsetdash{}{0pt}%
\pgfpathmoveto{\pgfqpoint{2.344116in}{0.500000in}}%
\pgfpathlineto{\pgfqpoint{2.370455in}{0.500000in}}%
\pgfpathlineto{\pgfqpoint{2.370455in}{0.501360in}}%
\pgfpathlineto{\pgfqpoint{2.344116in}{0.501360in}}%
\pgfpathlineto{\pgfqpoint{2.344116in}{0.500000in}}%
\pgfpathclose%
\pgfusepath{fill}%
\end{pgfscope}%
\begin{pgfscope}%
\pgfpathrectangle{\pgfqpoint{0.750000in}{0.500000in}}{\pgfqpoint{4.650000in}{3.020000in}}%
\pgfusepath{clip}%
\pgfsetbuttcap%
\pgfsetmiterjoin%
\definecolor{currentfill}{rgb}{0.121569,0.466667,0.705882}%
\pgfsetfillcolor{currentfill}%
\pgfsetlinewidth{0.000000pt}%
\definecolor{currentstroke}{rgb}{0.000000,0.000000,0.000000}%
\pgfsetstrokecolor{currentstroke}%
\pgfsetstrokeopacity{0.000000}%
\pgfsetdash{}{0pt}%
\pgfpathmoveto{\pgfqpoint{2.377039in}{0.500000in}}%
\pgfpathlineto{\pgfqpoint{2.403377in}{0.500000in}}%
\pgfpathlineto{\pgfqpoint{2.403377in}{0.502721in}}%
\pgfpathlineto{\pgfqpoint{2.377039in}{0.502721in}}%
\pgfpathlineto{\pgfqpoint{2.377039in}{0.500000in}}%
\pgfpathclose%
\pgfusepath{fill}%
\end{pgfscope}%
\begin{pgfscope}%
\pgfpathrectangle{\pgfqpoint{0.750000in}{0.500000in}}{\pgfqpoint{4.650000in}{3.020000in}}%
\pgfusepath{clip}%
\pgfsetbuttcap%
\pgfsetmiterjoin%
\definecolor{currentfill}{rgb}{0.121569,0.466667,0.705882}%
\pgfsetfillcolor{currentfill}%
\pgfsetlinewidth{0.000000pt}%
\definecolor{currentstroke}{rgb}{0.000000,0.000000,0.000000}%
\pgfsetstrokecolor{currentstroke}%
\pgfsetstrokeopacity{0.000000}%
\pgfsetdash{}{0pt}%
\pgfpathmoveto{\pgfqpoint{2.409962in}{0.500000in}}%
\pgfpathlineto{\pgfqpoint{2.436300in}{0.500000in}}%
\pgfpathlineto{\pgfqpoint{2.436300in}{0.505256in}}%
\pgfpathlineto{\pgfqpoint{2.409962in}{0.505256in}}%
\pgfpathlineto{\pgfqpoint{2.409962in}{0.500000in}}%
\pgfpathclose%
\pgfusepath{fill}%
\end{pgfscope}%
\begin{pgfscope}%
\pgfpathrectangle{\pgfqpoint{0.750000in}{0.500000in}}{\pgfqpoint{4.650000in}{3.020000in}}%
\pgfusepath{clip}%
\pgfsetbuttcap%
\pgfsetmiterjoin%
\definecolor{currentfill}{rgb}{0.121569,0.466667,0.705882}%
\pgfsetfillcolor{currentfill}%
\pgfsetlinewidth{0.000000pt}%
\definecolor{currentstroke}{rgb}{0.000000,0.000000,0.000000}%
\pgfsetstrokecolor{currentstroke}%
\pgfsetstrokeopacity{0.000000}%
\pgfsetdash{}{0pt}%
\pgfpathmoveto{\pgfqpoint{2.442884in}{0.500000in}}%
\pgfpathlineto{\pgfqpoint{2.469223in}{0.500000in}}%
\pgfpathlineto{\pgfqpoint{2.469223in}{0.509811in}}%
\pgfpathlineto{\pgfqpoint{2.442884in}{0.509811in}}%
\pgfpathlineto{\pgfqpoint{2.442884in}{0.500000in}}%
\pgfpathclose%
\pgfusepath{fill}%
\end{pgfscope}%
\begin{pgfscope}%
\pgfpathrectangle{\pgfqpoint{0.750000in}{0.500000in}}{\pgfqpoint{4.650000in}{3.020000in}}%
\pgfusepath{clip}%
\pgfsetbuttcap%
\pgfsetmiterjoin%
\definecolor{currentfill}{rgb}{0.121569,0.466667,0.705882}%
\pgfsetfillcolor{currentfill}%
\pgfsetlinewidth{0.000000pt}%
\definecolor{currentstroke}{rgb}{0.000000,0.000000,0.000000}%
\pgfsetstrokecolor{currentstroke}%
\pgfsetstrokeopacity{0.000000}%
\pgfsetdash{}{0pt}%
\pgfpathmoveto{\pgfqpoint{2.475807in}{0.500000in}}%
\pgfpathlineto{\pgfqpoint{2.502145in}{0.500000in}}%
\pgfpathlineto{\pgfqpoint{2.502145in}{0.517703in}}%
\pgfpathlineto{\pgfqpoint{2.475807in}{0.517703in}}%
\pgfpathlineto{\pgfqpoint{2.475807in}{0.500000in}}%
\pgfpathclose%
\pgfusepath{fill}%
\end{pgfscope}%
\begin{pgfscope}%
\pgfpathrectangle{\pgfqpoint{0.750000in}{0.500000in}}{\pgfqpoint{4.650000in}{3.020000in}}%
\pgfusepath{clip}%
\pgfsetbuttcap%
\pgfsetmiterjoin%
\definecolor{currentfill}{rgb}{0.121569,0.466667,0.705882}%
\pgfsetfillcolor{currentfill}%
\pgfsetlinewidth{0.000000pt}%
\definecolor{currentstroke}{rgb}{0.000000,0.000000,0.000000}%
\pgfsetstrokecolor{currentstroke}%
\pgfsetstrokeopacity{0.000000}%
\pgfsetdash{}{0pt}%
\pgfpathmoveto{\pgfqpoint{2.508730in}{0.500000in}}%
\pgfpathlineto{\pgfqpoint{2.535068in}{0.500000in}}%
\pgfpathlineto{\pgfqpoint{2.535068in}{0.530887in}}%
\pgfpathlineto{\pgfqpoint{2.508730in}{0.530887in}}%
\pgfpathlineto{\pgfqpoint{2.508730in}{0.500000in}}%
\pgfpathclose%
\pgfusepath{fill}%
\end{pgfscope}%
\begin{pgfscope}%
\pgfpathrectangle{\pgfqpoint{0.750000in}{0.500000in}}{\pgfqpoint{4.650000in}{3.020000in}}%
\pgfusepath{clip}%
\pgfsetbuttcap%
\pgfsetmiterjoin%
\definecolor{currentfill}{rgb}{0.121569,0.466667,0.705882}%
\pgfsetfillcolor{currentfill}%
\pgfsetlinewidth{0.000000pt}%
\definecolor{currentstroke}{rgb}{0.000000,0.000000,0.000000}%
\pgfsetstrokecolor{currentstroke}%
\pgfsetstrokeopacity{0.000000}%
\pgfsetdash{}{0pt}%
\pgfpathmoveto{\pgfqpoint{2.541653in}{0.500000in}}%
\pgfpathlineto{\pgfqpoint{2.567991in}{0.500000in}}%
\pgfpathlineto{\pgfqpoint{2.567991in}{0.552121in}}%
\pgfpathlineto{\pgfqpoint{2.541653in}{0.552121in}}%
\pgfpathlineto{\pgfqpoint{2.541653in}{0.500000in}}%
\pgfpathclose%
\pgfusepath{fill}%
\end{pgfscope}%
\begin{pgfscope}%
\pgfpathrectangle{\pgfqpoint{0.750000in}{0.500000in}}{\pgfqpoint{4.650000in}{3.020000in}}%
\pgfusepath{clip}%
\pgfsetbuttcap%
\pgfsetmiterjoin%
\definecolor{currentfill}{rgb}{0.121569,0.466667,0.705882}%
\pgfsetfillcolor{currentfill}%
\pgfsetlinewidth{0.000000pt}%
\definecolor{currentstroke}{rgb}{0.000000,0.000000,0.000000}%
\pgfsetstrokecolor{currentstroke}%
\pgfsetstrokeopacity{0.000000}%
\pgfsetdash{}{0pt}%
\pgfpathmoveto{\pgfqpoint{2.574575in}{0.500000in}}%
\pgfpathlineto{\pgfqpoint{2.600913in}{0.500000in}}%
\pgfpathlineto{\pgfqpoint{2.600913in}{0.585096in}}%
\pgfpathlineto{\pgfqpoint{2.574575in}{0.585096in}}%
\pgfpathlineto{\pgfqpoint{2.574575in}{0.500000in}}%
\pgfpathclose%
\pgfusepath{fill}%
\end{pgfscope}%
\begin{pgfscope}%
\pgfpathrectangle{\pgfqpoint{0.750000in}{0.500000in}}{\pgfqpoint{4.650000in}{3.020000in}}%
\pgfusepath{clip}%
\pgfsetbuttcap%
\pgfsetmiterjoin%
\definecolor{currentfill}{rgb}{0.121569,0.466667,0.705882}%
\pgfsetfillcolor{currentfill}%
\pgfsetlinewidth{0.000000pt}%
\definecolor{currentstroke}{rgb}{0.000000,0.000000,0.000000}%
\pgfsetstrokecolor{currentstroke}%
\pgfsetstrokeopacity{0.000000}%
\pgfsetdash{}{0pt}%
\pgfpathmoveto{\pgfqpoint{2.607498in}{0.500000in}}%
\pgfpathlineto{\pgfqpoint{2.633836in}{0.500000in}}%
\pgfpathlineto{\pgfqpoint{2.633836in}{0.634452in}}%
\pgfpathlineto{\pgfqpoint{2.607498in}{0.634452in}}%
\pgfpathlineto{\pgfqpoint{2.607498in}{0.500000in}}%
\pgfpathclose%
\pgfusepath{fill}%
\end{pgfscope}%
\begin{pgfscope}%
\pgfpathrectangle{\pgfqpoint{0.750000in}{0.500000in}}{\pgfqpoint{4.650000in}{3.020000in}}%
\pgfusepath{clip}%
\pgfsetbuttcap%
\pgfsetmiterjoin%
\definecolor{currentfill}{rgb}{0.121569,0.466667,0.705882}%
\pgfsetfillcolor{currentfill}%
\pgfsetlinewidth{0.000000pt}%
\definecolor{currentstroke}{rgb}{0.000000,0.000000,0.000000}%
\pgfsetstrokecolor{currentstroke}%
\pgfsetstrokeopacity{0.000000}%
\pgfsetdash{}{0pt}%
\pgfpathmoveto{\pgfqpoint{2.640421in}{0.500000in}}%
\pgfpathlineto{\pgfqpoint{2.666759in}{0.500000in}}%
\pgfpathlineto{\pgfqpoint{2.666759in}{0.705632in}}%
\pgfpathlineto{\pgfqpoint{2.640421in}{0.705632in}}%
\pgfpathlineto{\pgfqpoint{2.640421in}{0.500000in}}%
\pgfpathclose%
\pgfusepath{fill}%
\end{pgfscope}%
\begin{pgfscope}%
\pgfpathrectangle{\pgfqpoint{0.750000in}{0.500000in}}{\pgfqpoint{4.650000in}{3.020000in}}%
\pgfusepath{clip}%
\pgfsetbuttcap%
\pgfsetmiterjoin%
\definecolor{currentfill}{rgb}{0.121569,0.466667,0.705882}%
\pgfsetfillcolor{currentfill}%
\pgfsetlinewidth{0.000000pt}%
\definecolor{currentstroke}{rgb}{0.000000,0.000000,0.000000}%
\pgfsetstrokecolor{currentstroke}%
\pgfsetstrokeopacity{0.000000}%
\pgfsetdash{}{0pt}%
\pgfpathmoveto{\pgfqpoint{2.673343in}{0.500000in}}%
\pgfpathlineto{\pgfqpoint{2.699681in}{0.500000in}}%
\pgfpathlineto{\pgfqpoint{2.699681in}{0.804493in}}%
\pgfpathlineto{\pgfqpoint{2.673343in}{0.804493in}}%
\pgfpathlineto{\pgfqpoint{2.673343in}{0.500000in}}%
\pgfpathclose%
\pgfusepath{fill}%
\end{pgfscope}%
\begin{pgfscope}%
\pgfpathrectangle{\pgfqpoint{0.750000in}{0.500000in}}{\pgfqpoint{4.650000in}{3.020000in}}%
\pgfusepath{clip}%
\pgfsetbuttcap%
\pgfsetmiterjoin%
\definecolor{currentfill}{rgb}{0.121569,0.466667,0.705882}%
\pgfsetfillcolor{currentfill}%
\pgfsetlinewidth{0.000000pt}%
\definecolor{currentstroke}{rgb}{0.000000,0.000000,0.000000}%
\pgfsetstrokecolor{currentstroke}%
\pgfsetstrokeopacity{0.000000}%
\pgfsetdash{}{0pt}%
\pgfpathmoveto{\pgfqpoint{2.706266in}{0.500000in}}%
\pgfpathlineto{\pgfqpoint{2.732604in}{0.500000in}}%
\pgfpathlineto{\pgfqpoint{2.732604in}{0.936632in}}%
\pgfpathlineto{\pgfqpoint{2.706266in}{0.936632in}}%
\pgfpathlineto{\pgfqpoint{2.706266in}{0.500000in}}%
\pgfpathclose%
\pgfusepath{fill}%
\end{pgfscope}%
\begin{pgfscope}%
\pgfpathrectangle{\pgfqpoint{0.750000in}{0.500000in}}{\pgfqpoint{4.650000in}{3.020000in}}%
\pgfusepath{clip}%
\pgfsetbuttcap%
\pgfsetmiterjoin%
\definecolor{currentfill}{rgb}{0.121569,0.466667,0.705882}%
\pgfsetfillcolor{currentfill}%
\pgfsetlinewidth{0.000000pt}%
\definecolor{currentstroke}{rgb}{0.000000,0.000000,0.000000}%
\pgfsetstrokecolor{currentstroke}%
\pgfsetstrokeopacity{0.000000}%
\pgfsetdash{}{0pt}%
\pgfpathmoveto{\pgfqpoint{2.739189in}{0.500000in}}%
\pgfpathlineto{\pgfqpoint{2.765527in}{0.500000in}}%
\pgfpathlineto{\pgfqpoint{2.765527in}{1.106433in}}%
\pgfpathlineto{\pgfqpoint{2.739189in}{1.106433in}}%
\pgfpathlineto{\pgfqpoint{2.739189in}{0.500000in}}%
\pgfpathclose%
\pgfusepath{fill}%
\end{pgfscope}%
\begin{pgfscope}%
\pgfpathrectangle{\pgfqpoint{0.750000in}{0.500000in}}{\pgfqpoint{4.650000in}{3.020000in}}%
\pgfusepath{clip}%
\pgfsetbuttcap%
\pgfsetmiterjoin%
\definecolor{currentfill}{rgb}{0.121569,0.466667,0.705882}%
\pgfsetfillcolor{currentfill}%
\pgfsetlinewidth{0.000000pt}%
\definecolor{currentstroke}{rgb}{0.000000,0.000000,0.000000}%
\pgfsetstrokecolor{currentstroke}%
\pgfsetstrokeopacity{0.000000}%
\pgfsetdash{}{0pt}%
\pgfpathmoveto{\pgfqpoint{2.772111in}{0.500000in}}%
\pgfpathlineto{\pgfqpoint{2.798449in}{0.500000in}}%
\pgfpathlineto{\pgfqpoint{2.798449in}{1.315928in}}%
\pgfpathlineto{\pgfqpoint{2.772111in}{1.315928in}}%
\pgfpathlineto{\pgfqpoint{2.772111in}{0.500000in}}%
\pgfpathclose%
\pgfusepath{fill}%
\end{pgfscope}%
\begin{pgfscope}%
\pgfpathrectangle{\pgfqpoint{0.750000in}{0.500000in}}{\pgfqpoint{4.650000in}{3.020000in}}%
\pgfusepath{clip}%
\pgfsetbuttcap%
\pgfsetmiterjoin%
\definecolor{currentfill}{rgb}{0.121569,0.466667,0.705882}%
\pgfsetfillcolor{currentfill}%
\pgfsetlinewidth{0.000000pt}%
\definecolor{currentstroke}{rgb}{0.000000,0.000000,0.000000}%
\pgfsetstrokecolor{currentstroke}%
\pgfsetstrokeopacity{0.000000}%
\pgfsetdash{}{0pt}%
\pgfpathmoveto{\pgfqpoint{2.805034in}{0.500000in}}%
\pgfpathlineto{\pgfqpoint{2.831372in}{0.500000in}}%
\pgfpathlineto{\pgfqpoint{2.831372in}{1.563621in}}%
\pgfpathlineto{\pgfqpoint{2.805034in}{1.563621in}}%
\pgfpathlineto{\pgfqpoint{2.805034in}{0.500000in}}%
\pgfpathclose%
\pgfusepath{fill}%
\end{pgfscope}%
\begin{pgfscope}%
\pgfpathrectangle{\pgfqpoint{0.750000in}{0.500000in}}{\pgfqpoint{4.650000in}{3.020000in}}%
\pgfusepath{clip}%
\pgfsetbuttcap%
\pgfsetmiterjoin%
\definecolor{currentfill}{rgb}{0.121569,0.466667,0.705882}%
\pgfsetfillcolor{currentfill}%
\pgfsetlinewidth{0.000000pt}%
\definecolor{currentstroke}{rgb}{0.000000,0.000000,0.000000}%
\pgfsetstrokecolor{currentstroke}%
\pgfsetstrokeopacity{0.000000}%
\pgfsetdash{}{0pt}%
\pgfpathmoveto{\pgfqpoint{2.837957in}{0.500000in}}%
\pgfpathlineto{\pgfqpoint{2.864295in}{0.500000in}}%
\pgfpathlineto{\pgfqpoint{2.864295in}{1.843521in}}%
\pgfpathlineto{\pgfqpoint{2.837957in}{1.843521in}}%
\pgfpathlineto{\pgfqpoint{2.837957in}{0.500000in}}%
\pgfpathclose%
\pgfusepath{fill}%
\end{pgfscope}%
\begin{pgfscope}%
\pgfpathrectangle{\pgfqpoint{0.750000in}{0.500000in}}{\pgfqpoint{4.650000in}{3.020000in}}%
\pgfusepath{clip}%
\pgfsetbuttcap%
\pgfsetmiterjoin%
\definecolor{currentfill}{rgb}{0.121569,0.466667,0.705882}%
\pgfsetfillcolor{currentfill}%
\pgfsetlinewidth{0.000000pt}%
\definecolor{currentstroke}{rgb}{0.000000,0.000000,0.000000}%
\pgfsetstrokecolor{currentstroke}%
\pgfsetstrokeopacity{0.000000}%
\pgfsetdash{}{0pt}%
\pgfpathmoveto{\pgfqpoint{2.870879in}{0.500000in}}%
\pgfpathlineto{\pgfqpoint{2.897218in}{0.500000in}}%
\pgfpathlineto{\pgfqpoint{2.897218in}{2.144655in}}%
\pgfpathlineto{\pgfqpoint{2.870879in}{2.144655in}}%
\pgfpathlineto{\pgfqpoint{2.870879in}{0.500000in}}%
\pgfpathclose%
\pgfusepath{fill}%
\end{pgfscope}%
\begin{pgfscope}%
\pgfpathrectangle{\pgfqpoint{0.750000in}{0.500000in}}{\pgfqpoint{4.650000in}{3.020000in}}%
\pgfusepath{clip}%
\pgfsetbuttcap%
\pgfsetmiterjoin%
\definecolor{currentfill}{rgb}{0.121569,0.466667,0.705882}%
\pgfsetfillcolor{currentfill}%
\pgfsetlinewidth{0.000000pt}%
\definecolor{currentstroke}{rgb}{0.000000,0.000000,0.000000}%
\pgfsetstrokecolor{currentstroke}%
\pgfsetstrokeopacity{0.000000}%
\pgfsetdash{}{0pt}%
\pgfpathmoveto{\pgfqpoint{2.903802in}{0.500000in}}%
\pgfpathlineto{\pgfqpoint{2.930140in}{0.500000in}}%
\pgfpathlineto{\pgfqpoint{2.930140in}{2.451286in}}%
\pgfpathlineto{\pgfqpoint{2.903802in}{2.451286in}}%
\pgfpathlineto{\pgfqpoint{2.903802in}{0.500000in}}%
\pgfpathclose%
\pgfusepath{fill}%
\end{pgfscope}%
\begin{pgfscope}%
\pgfpathrectangle{\pgfqpoint{0.750000in}{0.500000in}}{\pgfqpoint{4.650000in}{3.020000in}}%
\pgfusepath{clip}%
\pgfsetbuttcap%
\pgfsetmiterjoin%
\definecolor{currentfill}{rgb}{0.121569,0.466667,0.705882}%
\pgfsetfillcolor{currentfill}%
\pgfsetlinewidth{0.000000pt}%
\definecolor{currentstroke}{rgb}{0.000000,0.000000,0.000000}%
\pgfsetstrokecolor{currentstroke}%
\pgfsetstrokeopacity{0.000000}%
\pgfsetdash{}{0pt}%
\pgfpathmoveto{\pgfqpoint{2.936725in}{0.500000in}}%
\pgfpathlineto{\pgfqpoint{2.963063in}{0.500000in}}%
\pgfpathlineto{\pgfqpoint{2.963063in}{2.743978in}}%
\pgfpathlineto{\pgfqpoint{2.936725in}{2.743978in}}%
\pgfpathlineto{\pgfqpoint{2.936725in}{0.500000in}}%
\pgfpathclose%
\pgfusepath{fill}%
\end{pgfscope}%
\begin{pgfscope}%
\pgfpathrectangle{\pgfqpoint{0.750000in}{0.500000in}}{\pgfqpoint{4.650000in}{3.020000in}}%
\pgfusepath{clip}%
\pgfsetbuttcap%
\pgfsetmiterjoin%
\definecolor{currentfill}{rgb}{0.121569,0.466667,0.705882}%
\pgfsetfillcolor{currentfill}%
\pgfsetlinewidth{0.000000pt}%
\definecolor{currentstroke}{rgb}{0.000000,0.000000,0.000000}%
\pgfsetstrokecolor{currentstroke}%
\pgfsetstrokeopacity{0.000000}%
\pgfsetdash{}{0pt}%
\pgfpathmoveto{\pgfqpoint{2.969647in}{0.500000in}}%
\pgfpathlineto{\pgfqpoint{2.995986in}{0.500000in}}%
\pgfpathlineto{\pgfqpoint{2.995986in}{3.001484in}}%
\pgfpathlineto{\pgfqpoint{2.969647in}{3.001484in}}%
\pgfpathlineto{\pgfqpoint{2.969647in}{0.500000in}}%
\pgfpathclose%
\pgfusepath{fill}%
\end{pgfscope}%
\begin{pgfscope}%
\pgfpathrectangle{\pgfqpoint{0.750000in}{0.500000in}}{\pgfqpoint{4.650000in}{3.020000in}}%
\pgfusepath{clip}%
\pgfsetbuttcap%
\pgfsetmiterjoin%
\definecolor{currentfill}{rgb}{0.121569,0.466667,0.705882}%
\pgfsetfillcolor{currentfill}%
\pgfsetlinewidth{0.000000pt}%
\definecolor{currentstroke}{rgb}{0.000000,0.000000,0.000000}%
\pgfsetstrokecolor{currentstroke}%
\pgfsetstrokeopacity{0.000000}%
\pgfsetdash{}{0pt}%
\pgfpathmoveto{\pgfqpoint{3.002570in}{0.500000in}}%
\pgfpathlineto{\pgfqpoint{3.028908in}{0.500000in}}%
\pgfpathlineto{\pgfqpoint{3.028908in}{3.203217in}}%
\pgfpathlineto{\pgfqpoint{3.002570in}{3.203217in}}%
\pgfpathlineto{\pgfqpoint{3.002570in}{0.500000in}}%
\pgfpathclose%
\pgfusepath{fill}%
\end{pgfscope}%
\begin{pgfscope}%
\pgfpathrectangle{\pgfqpoint{0.750000in}{0.500000in}}{\pgfqpoint{4.650000in}{3.020000in}}%
\pgfusepath{clip}%
\pgfsetbuttcap%
\pgfsetmiterjoin%
\definecolor{currentfill}{rgb}{0.121569,0.466667,0.705882}%
\pgfsetfillcolor{currentfill}%
\pgfsetlinewidth{0.000000pt}%
\definecolor{currentstroke}{rgb}{0.000000,0.000000,0.000000}%
\pgfsetstrokecolor{currentstroke}%
\pgfsetstrokeopacity{0.000000}%
\pgfsetdash{}{0pt}%
\pgfpathmoveto{\pgfqpoint{3.035493in}{0.500000in}}%
\pgfpathlineto{\pgfqpoint{3.061831in}{0.500000in}}%
\pgfpathlineto{\pgfqpoint{3.061831in}{3.331941in}}%
\pgfpathlineto{\pgfqpoint{3.035493in}{3.331941in}}%
\pgfpathlineto{\pgfqpoint{3.035493in}{0.500000in}}%
\pgfpathclose%
\pgfusepath{fill}%
\end{pgfscope}%
\begin{pgfscope}%
\pgfpathrectangle{\pgfqpoint{0.750000in}{0.500000in}}{\pgfqpoint{4.650000in}{3.020000in}}%
\pgfusepath{clip}%
\pgfsetbuttcap%
\pgfsetmiterjoin%
\definecolor{currentfill}{rgb}{0.121569,0.466667,0.705882}%
\pgfsetfillcolor{currentfill}%
\pgfsetlinewidth{0.000000pt}%
\definecolor{currentstroke}{rgb}{0.000000,0.000000,0.000000}%
\pgfsetstrokecolor{currentstroke}%
\pgfsetstrokeopacity{0.000000}%
\pgfsetdash{}{0pt}%
\pgfpathmoveto{\pgfqpoint{3.068415in}{0.500000in}}%
\pgfpathlineto{\pgfqpoint{3.094754in}{0.500000in}}%
\pgfpathlineto{\pgfqpoint{3.094754in}{3.376190in}}%
\pgfpathlineto{\pgfqpoint{3.068415in}{3.376190in}}%
\pgfpathlineto{\pgfqpoint{3.068415in}{0.500000in}}%
\pgfpathclose%
\pgfusepath{fill}%
\end{pgfscope}%
\begin{pgfscope}%
\pgfpathrectangle{\pgfqpoint{0.750000in}{0.500000in}}{\pgfqpoint{4.650000in}{3.020000in}}%
\pgfusepath{clip}%
\pgfsetbuttcap%
\pgfsetmiterjoin%
\definecolor{currentfill}{rgb}{0.121569,0.466667,0.705882}%
\pgfsetfillcolor{currentfill}%
\pgfsetlinewidth{0.000000pt}%
\definecolor{currentstroke}{rgb}{0.000000,0.000000,0.000000}%
\pgfsetstrokecolor{currentstroke}%
\pgfsetstrokeopacity{0.000000}%
\pgfsetdash{}{0pt}%
\pgfpathmoveto{\pgfqpoint{3.101338in}{0.500000in}}%
\pgfpathlineto{\pgfqpoint{3.127676in}{0.500000in}}%
\pgfpathlineto{\pgfqpoint{3.127676in}{3.331941in}}%
\pgfpathlineto{\pgfqpoint{3.101338in}{3.331941in}}%
\pgfpathlineto{\pgfqpoint{3.101338in}{0.500000in}}%
\pgfpathclose%
\pgfusepath{fill}%
\end{pgfscope}%
\begin{pgfscope}%
\pgfpathrectangle{\pgfqpoint{0.750000in}{0.500000in}}{\pgfqpoint{4.650000in}{3.020000in}}%
\pgfusepath{clip}%
\pgfsetbuttcap%
\pgfsetmiterjoin%
\definecolor{currentfill}{rgb}{0.121569,0.466667,0.705882}%
\pgfsetfillcolor{currentfill}%
\pgfsetlinewidth{0.000000pt}%
\definecolor{currentstroke}{rgb}{0.000000,0.000000,0.000000}%
\pgfsetstrokecolor{currentstroke}%
\pgfsetstrokeopacity{0.000000}%
\pgfsetdash{}{0pt}%
\pgfpathmoveto{\pgfqpoint{3.134261in}{0.500000in}}%
\pgfpathlineto{\pgfqpoint{3.160599in}{0.500000in}}%
\pgfpathlineto{\pgfqpoint{3.160599in}{3.203217in}}%
\pgfpathlineto{\pgfqpoint{3.134261in}{3.203217in}}%
\pgfpathlineto{\pgfqpoint{3.134261in}{0.500000in}}%
\pgfpathclose%
\pgfusepath{fill}%
\end{pgfscope}%
\begin{pgfscope}%
\pgfpathrectangle{\pgfqpoint{0.750000in}{0.500000in}}{\pgfqpoint{4.650000in}{3.020000in}}%
\pgfusepath{clip}%
\pgfsetbuttcap%
\pgfsetmiterjoin%
\definecolor{currentfill}{rgb}{0.121569,0.466667,0.705882}%
\pgfsetfillcolor{currentfill}%
\pgfsetlinewidth{0.000000pt}%
\definecolor{currentstroke}{rgb}{0.000000,0.000000,0.000000}%
\pgfsetstrokecolor{currentstroke}%
\pgfsetstrokeopacity{0.000000}%
\pgfsetdash{}{0pt}%
\pgfpathmoveto{\pgfqpoint{3.167184in}{0.500000in}}%
\pgfpathlineto{\pgfqpoint{3.193522in}{0.500000in}}%
\pgfpathlineto{\pgfqpoint{3.193522in}{3.001484in}}%
\pgfpathlineto{\pgfqpoint{3.167184in}{3.001484in}}%
\pgfpathlineto{\pgfqpoint{3.167184in}{0.500000in}}%
\pgfpathclose%
\pgfusepath{fill}%
\end{pgfscope}%
\begin{pgfscope}%
\pgfpathrectangle{\pgfqpoint{0.750000in}{0.500000in}}{\pgfqpoint{4.650000in}{3.020000in}}%
\pgfusepath{clip}%
\pgfsetbuttcap%
\pgfsetmiterjoin%
\definecolor{currentfill}{rgb}{0.121569,0.466667,0.705882}%
\pgfsetfillcolor{currentfill}%
\pgfsetlinewidth{0.000000pt}%
\definecolor{currentstroke}{rgb}{0.000000,0.000000,0.000000}%
\pgfsetstrokecolor{currentstroke}%
\pgfsetstrokeopacity{0.000000}%
\pgfsetdash{}{0pt}%
\pgfpathmoveto{\pgfqpoint{3.200106in}{0.500000in}}%
\pgfpathlineto{\pgfqpoint{3.226444in}{0.500000in}}%
\pgfpathlineto{\pgfqpoint{3.226444in}{2.743978in}}%
\pgfpathlineto{\pgfqpoint{3.200106in}{2.743978in}}%
\pgfpathlineto{\pgfqpoint{3.200106in}{0.500000in}}%
\pgfpathclose%
\pgfusepath{fill}%
\end{pgfscope}%
\begin{pgfscope}%
\pgfpathrectangle{\pgfqpoint{0.750000in}{0.500000in}}{\pgfqpoint{4.650000in}{3.020000in}}%
\pgfusepath{clip}%
\pgfsetbuttcap%
\pgfsetmiterjoin%
\definecolor{currentfill}{rgb}{0.121569,0.466667,0.705882}%
\pgfsetfillcolor{currentfill}%
\pgfsetlinewidth{0.000000pt}%
\definecolor{currentstroke}{rgb}{0.000000,0.000000,0.000000}%
\pgfsetstrokecolor{currentstroke}%
\pgfsetstrokeopacity{0.000000}%
\pgfsetdash{}{0pt}%
\pgfpathmoveto{\pgfqpoint{3.233029in}{0.500000in}}%
\pgfpathlineto{\pgfqpoint{3.259367in}{0.500000in}}%
\pgfpathlineto{\pgfqpoint{3.259367in}{2.451286in}}%
\pgfpathlineto{\pgfqpoint{3.233029in}{2.451286in}}%
\pgfpathlineto{\pgfqpoint{3.233029in}{0.500000in}}%
\pgfpathclose%
\pgfusepath{fill}%
\end{pgfscope}%
\begin{pgfscope}%
\pgfpathrectangle{\pgfqpoint{0.750000in}{0.500000in}}{\pgfqpoint{4.650000in}{3.020000in}}%
\pgfusepath{clip}%
\pgfsetbuttcap%
\pgfsetmiterjoin%
\definecolor{currentfill}{rgb}{0.121569,0.466667,0.705882}%
\pgfsetfillcolor{currentfill}%
\pgfsetlinewidth{0.000000pt}%
\definecolor{currentstroke}{rgb}{0.000000,0.000000,0.000000}%
\pgfsetstrokecolor{currentstroke}%
\pgfsetstrokeopacity{0.000000}%
\pgfsetdash{}{0pt}%
\pgfpathmoveto{\pgfqpoint{3.265952in}{0.500000in}}%
\pgfpathlineto{\pgfqpoint{3.292290in}{0.500000in}}%
\pgfpathlineto{\pgfqpoint{3.292290in}{2.144655in}}%
\pgfpathlineto{\pgfqpoint{3.265952in}{2.144655in}}%
\pgfpathlineto{\pgfqpoint{3.265952in}{0.500000in}}%
\pgfpathclose%
\pgfusepath{fill}%
\end{pgfscope}%
\begin{pgfscope}%
\pgfpathrectangle{\pgfqpoint{0.750000in}{0.500000in}}{\pgfqpoint{4.650000in}{3.020000in}}%
\pgfusepath{clip}%
\pgfsetbuttcap%
\pgfsetmiterjoin%
\definecolor{currentfill}{rgb}{0.121569,0.466667,0.705882}%
\pgfsetfillcolor{currentfill}%
\pgfsetlinewidth{0.000000pt}%
\definecolor{currentstroke}{rgb}{0.000000,0.000000,0.000000}%
\pgfsetstrokecolor{currentstroke}%
\pgfsetstrokeopacity{0.000000}%
\pgfsetdash{}{0pt}%
\pgfpathmoveto{\pgfqpoint{3.298874in}{0.500000in}}%
\pgfpathlineto{\pgfqpoint{3.325212in}{0.500000in}}%
\pgfpathlineto{\pgfqpoint{3.325212in}{1.843521in}}%
\pgfpathlineto{\pgfqpoint{3.298874in}{1.843521in}}%
\pgfpathlineto{\pgfqpoint{3.298874in}{0.500000in}}%
\pgfpathclose%
\pgfusepath{fill}%
\end{pgfscope}%
\begin{pgfscope}%
\pgfpathrectangle{\pgfqpoint{0.750000in}{0.500000in}}{\pgfqpoint{4.650000in}{3.020000in}}%
\pgfusepath{clip}%
\pgfsetbuttcap%
\pgfsetmiterjoin%
\definecolor{currentfill}{rgb}{0.121569,0.466667,0.705882}%
\pgfsetfillcolor{currentfill}%
\pgfsetlinewidth{0.000000pt}%
\definecolor{currentstroke}{rgb}{0.000000,0.000000,0.000000}%
\pgfsetstrokecolor{currentstroke}%
\pgfsetstrokeopacity{0.000000}%
\pgfsetdash{}{0pt}%
\pgfpathmoveto{\pgfqpoint{3.331797in}{0.500000in}}%
\pgfpathlineto{\pgfqpoint{3.358135in}{0.500000in}}%
\pgfpathlineto{\pgfqpoint{3.358135in}{1.563621in}}%
\pgfpathlineto{\pgfqpoint{3.331797in}{1.563621in}}%
\pgfpathlineto{\pgfqpoint{3.331797in}{0.500000in}}%
\pgfpathclose%
\pgfusepath{fill}%
\end{pgfscope}%
\begin{pgfscope}%
\pgfpathrectangle{\pgfqpoint{0.750000in}{0.500000in}}{\pgfqpoint{4.650000in}{3.020000in}}%
\pgfusepath{clip}%
\pgfsetbuttcap%
\pgfsetmiterjoin%
\definecolor{currentfill}{rgb}{0.121569,0.466667,0.705882}%
\pgfsetfillcolor{currentfill}%
\pgfsetlinewidth{0.000000pt}%
\definecolor{currentstroke}{rgb}{0.000000,0.000000,0.000000}%
\pgfsetstrokecolor{currentstroke}%
\pgfsetstrokeopacity{0.000000}%
\pgfsetdash{}{0pt}%
\pgfpathmoveto{\pgfqpoint{3.364720in}{0.500000in}}%
\pgfpathlineto{\pgfqpoint{3.391058in}{0.500000in}}%
\pgfpathlineto{\pgfqpoint{3.391058in}{1.315928in}}%
\pgfpathlineto{\pgfqpoint{3.364720in}{1.315928in}}%
\pgfpathlineto{\pgfqpoint{3.364720in}{0.500000in}}%
\pgfpathclose%
\pgfusepath{fill}%
\end{pgfscope}%
\begin{pgfscope}%
\pgfpathrectangle{\pgfqpoint{0.750000in}{0.500000in}}{\pgfqpoint{4.650000in}{3.020000in}}%
\pgfusepath{clip}%
\pgfsetbuttcap%
\pgfsetmiterjoin%
\definecolor{currentfill}{rgb}{0.121569,0.466667,0.705882}%
\pgfsetfillcolor{currentfill}%
\pgfsetlinewidth{0.000000pt}%
\definecolor{currentstroke}{rgb}{0.000000,0.000000,0.000000}%
\pgfsetstrokecolor{currentstroke}%
\pgfsetstrokeopacity{0.000000}%
\pgfsetdash{}{0pt}%
\pgfpathmoveto{\pgfqpoint{3.397642in}{0.500000in}}%
\pgfpathlineto{\pgfqpoint{3.423980in}{0.500000in}}%
\pgfpathlineto{\pgfqpoint{3.423980in}{1.106433in}}%
\pgfpathlineto{\pgfqpoint{3.397642in}{1.106433in}}%
\pgfpathlineto{\pgfqpoint{3.397642in}{0.500000in}}%
\pgfpathclose%
\pgfusepath{fill}%
\end{pgfscope}%
\begin{pgfscope}%
\pgfpathrectangle{\pgfqpoint{0.750000in}{0.500000in}}{\pgfqpoint{4.650000in}{3.020000in}}%
\pgfusepath{clip}%
\pgfsetbuttcap%
\pgfsetmiterjoin%
\definecolor{currentfill}{rgb}{0.121569,0.466667,0.705882}%
\pgfsetfillcolor{currentfill}%
\pgfsetlinewidth{0.000000pt}%
\definecolor{currentstroke}{rgb}{0.000000,0.000000,0.000000}%
\pgfsetstrokecolor{currentstroke}%
\pgfsetstrokeopacity{0.000000}%
\pgfsetdash{}{0pt}%
\pgfpathmoveto{\pgfqpoint{3.430565in}{0.500000in}}%
\pgfpathlineto{\pgfqpoint{3.456903in}{0.500000in}}%
\pgfpathlineto{\pgfqpoint{3.456903in}{0.936632in}}%
\pgfpathlineto{\pgfqpoint{3.430565in}{0.936632in}}%
\pgfpathlineto{\pgfqpoint{3.430565in}{0.500000in}}%
\pgfpathclose%
\pgfusepath{fill}%
\end{pgfscope}%
\begin{pgfscope}%
\pgfpathrectangle{\pgfqpoint{0.750000in}{0.500000in}}{\pgfqpoint{4.650000in}{3.020000in}}%
\pgfusepath{clip}%
\pgfsetbuttcap%
\pgfsetmiterjoin%
\definecolor{currentfill}{rgb}{0.121569,0.466667,0.705882}%
\pgfsetfillcolor{currentfill}%
\pgfsetlinewidth{0.000000pt}%
\definecolor{currentstroke}{rgb}{0.000000,0.000000,0.000000}%
\pgfsetstrokecolor{currentstroke}%
\pgfsetstrokeopacity{0.000000}%
\pgfsetdash{}{0pt}%
\pgfpathmoveto{\pgfqpoint{3.463488in}{0.500000in}}%
\pgfpathlineto{\pgfqpoint{3.489826in}{0.500000in}}%
\pgfpathlineto{\pgfqpoint{3.489826in}{0.804493in}}%
\pgfpathlineto{\pgfqpoint{3.463488in}{0.804493in}}%
\pgfpathlineto{\pgfqpoint{3.463488in}{0.500000in}}%
\pgfpathclose%
\pgfusepath{fill}%
\end{pgfscope}%
\begin{pgfscope}%
\pgfpathrectangle{\pgfqpoint{0.750000in}{0.500000in}}{\pgfqpoint{4.650000in}{3.020000in}}%
\pgfusepath{clip}%
\pgfsetbuttcap%
\pgfsetmiterjoin%
\definecolor{currentfill}{rgb}{0.121569,0.466667,0.705882}%
\pgfsetfillcolor{currentfill}%
\pgfsetlinewidth{0.000000pt}%
\definecolor{currentstroke}{rgb}{0.000000,0.000000,0.000000}%
\pgfsetstrokecolor{currentstroke}%
\pgfsetstrokeopacity{0.000000}%
\pgfsetdash{}{0pt}%
\pgfpathmoveto{\pgfqpoint{3.496410in}{0.500000in}}%
\pgfpathlineto{\pgfqpoint{3.522749in}{0.500000in}}%
\pgfpathlineto{\pgfqpoint{3.522749in}{0.705632in}}%
\pgfpathlineto{\pgfqpoint{3.496410in}{0.705632in}}%
\pgfpathlineto{\pgfqpoint{3.496410in}{0.500000in}}%
\pgfpathclose%
\pgfusepath{fill}%
\end{pgfscope}%
\begin{pgfscope}%
\pgfpathrectangle{\pgfqpoint{0.750000in}{0.500000in}}{\pgfqpoint{4.650000in}{3.020000in}}%
\pgfusepath{clip}%
\pgfsetbuttcap%
\pgfsetmiterjoin%
\definecolor{currentfill}{rgb}{0.121569,0.466667,0.705882}%
\pgfsetfillcolor{currentfill}%
\pgfsetlinewidth{0.000000pt}%
\definecolor{currentstroke}{rgb}{0.000000,0.000000,0.000000}%
\pgfsetstrokecolor{currentstroke}%
\pgfsetstrokeopacity{0.000000}%
\pgfsetdash{}{0pt}%
\pgfpathmoveto{\pgfqpoint{3.529333in}{0.500000in}}%
\pgfpathlineto{\pgfqpoint{3.555671in}{0.500000in}}%
\pgfpathlineto{\pgfqpoint{3.555671in}{0.634452in}}%
\pgfpathlineto{\pgfqpoint{3.529333in}{0.634452in}}%
\pgfpathlineto{\pgfqpoint{3.529333in}{0.500000in}}%
\pgfpathclose%
\pgfusepath{fill}%
\end{pgfscope}%
\begin{pgfscope}%
\pgfpathrectangle{\pgfqpoint{0.750000in}{0.500000in}}{\pgfqpoint{4.650000in}{3.020000in}}%
\pgfusepath{clip}%
\pgfsetbuttcap%
\pgfsetmiterjoin%
\definecolor{currentfill}{rgb}{0.121569,0.466667,0.705882}%
\pgfsetfillcolor{currentfill}%
\pgfsetlinewidth{0.000000pt}%
\definecolor{currentstroke}{rgb}{0.000000,0.000000,0.000000}%
\pgfsetstrokecolor{currentstroke}%
\pgfsetstrokeopacity{0.000000}%
\pgfsetdash{}{0pt}%
\pgfpathmoveto{\pgfqpoint{3.562256in}{0.500000in}}%
\pgfpathlineto{\pgfqpoint{3.588594in}{0.500000in}}%
\pgfpathlineto{\pgfqpoint{3.588594in}{0.585096in}}%
\pgfpathlineto{\pgfqpoint{3.562256in}{0.585096in}}%
\pgfpathlineto{\pgfqpoint{3.562256in}{0.500000in}}%
\pgfpathclose%
\pgfusepath{fill}%
\end{pgfscope}%
\begin{pgfscope}%
\pgfpathrectangle{\pgfqpoint{0.750000in}{0.500000in}}{\pgfqpoint{4.650000in}{3.020000in}}%
\pgfusepath{clip}%
\pgfsetbuttcap%
\pgfsetmiterjoin%
\definecolor{currentfill}{rgb}{0.121569,0.466667,0.705882}%
\pgfsetfillcolor{currentfill}%
\pgfsetlinewidth{0.000000pt}%
\definecolor{currentstroke}{rgb}{0.000000,0.000000,0.000000}%
\pgfsetstrokecolor{currentstroke}%
\pgfsetstrokeopacity{0.000000}%
\pgfsetdash{}{0pt}%
\pgfpathmoveto{\pgfqpoint{3.595178in}{0.500000in}}%
\pgfpathlineto{\pgfqpoint{3.621517in}{0.500000in}}%
\pgfpathlineto{\pgfqpoint{3.621517in}{0.552121in}}%
\pgfpathlineto{\pgfqpoint{3.595178in}{0.552121in}}%
\pgfpathlineto{\pgfqpoint{3.595178in}{0.500000in}}%
\pgfpathclose%
\pgfusepath{fill}%
\end{pgfscope}%
\begin{pgfscope}%
\pgfpathrectangle{\pgfqpoint{0.750000in}{0.500000in}}{\pgfqpoint{4.650000in}{3.020000in}}%
\pgfusepath{clip}%
\pgfsetbuttcap%
\pgfsetmiterjoin%
\definecolor{currentfill}{rgb}{0.121569,0.466667,0.705882}%
\pgfsetfillcolor{currentfill}%
\pgfsetlinewidth{0.000000pt}%
\definecolor{currentstroke}{rgb}{0.000000,0.000000,0.000000}%
\pgfsetstrokecolor{currentstroke}%
\pgfsetstrokeopacity{0.000000}%
\pgfsetdash{}{0pt}%
\pgfpathmoveto{\pgfqpoint{3.628101in}{0.500000in}}%
\pgfpathlineto{\pgfqpoint{3.654439in}{0.500000in}}%
\pgfpathlineto{\pgfqpoint{3.654439in}{0.530887in}}%
\pgfpathlineto{\pgfqpoint{3.628101in}{0.530887in}}%
\pgfpathlineto{\pgfqpoint{3.628101in}{0.500000in}}%
\pgfpathclose%
\pgfusepath{fill}%
\end{pgfscope}%
\begin{pgfscope}%
\pgfpathrectangle{\pgfqpoint{0.750000in}{0.500000in}}{\pgfqpoint{4.650000in}{3.020000in}}%
\pgfusepath{clip}%
\pgfsetbuttcap%
\pgfsetmiterjoin%
\definecolor{currentfill}{rgb}{0.121569,0.466667,0.705882}%
\pgfsetfillcolor{currentfill}%
\pgfsetlinewidth{0.000000pt}%
\definecolor{currentstroke}{rgb}{0.000000,0.000000,0.000000}%
\pgfsetstrokecolor{currentstroke}%
\pgfsetstrokeopacity{0.000000}%
\pgfsetdash{}{0pt}%
\pgfpathmoveto{\pgfqpoint{3.661024in}{0.500000in}}%
\pgfpathlineto{\pgfqpoint{3.687362in}{0.500000in}}%
\pgfpathlineto{\pgfqpoint{3.687362in}{0.517703in}}%
\pgfpathlineto{\pgfqpoint{3.661024in}{0.517703in}}%
\pgfpathlineto{\pgfqpoint{3.661024in}{0.500000in}}%
\pgfpathclose%
\pgfusepath{fill}%
\end{pgfscope}%
\begin{pgfscope}%
\pgfpathrectangle{\pgfqpoint{0.750000in}{0.500000in}}{\pgfqpoint{4.650000in}{3.020000in}}%
\pgfusepath{clip}%
\pgfsetbuttcap%
\pgfsetmiterjoin%
\definecolor{currentfill}{rgb}{0.121569,0.466667,0.705882}%
\pgfsetfillcolor{currentfill}%
\pgfsetlinewidth{0.000000pt}%
\definecolor{currentstroke}{rgb}{0.000000,0.000000,0.000000}%
\pgfsetstrokecolor{currentstroke}%
\pgfsetstrokeopacity{0.000000}%
\pgfsetdash{}{0pt}%
\pgfpathmoveto{\pgfqpoint{3.693946in}{0.500000in}}%
\pgfpathlineto{\pgfqpoint{3.720285in}{0.500000in}}%
\pgfpathlineto{\pgfqpoint{3.720285in}{0.509811in}}%
\pgfpathlineto{\pgfqpoint{3.693946in}{0.509811in}}%
\pgfpathlineto{\pgfqpoint{3.693946in}{0.500000in}}%
\pgfpathclose%
\pgfusepath{fill}%
\end{pgfscope}%
\begin{pgfscope}%
\pgfpathrectangle{\pgfqpoint{0.750000in}{0.500000in}}{\pgfqpoint{4.650000in}{3.020000in}}%
\pgfusepath{clip}%
\pgfsetbuttcap%
\pgfsetmiterjoin%
\definecolor{currentfill}{rgb}{0.121569,0.466667,0.705882}%
\pgfsetfillcolor{currentfill}%
\pgfsetlinewidth{0.000000pt}%
\definecolor{currentstroke}{rgb}{0.000000,0.000000,0.000000}%
\pgfsetstrokecolor{currentstroke}%
\pgfsetstrokeopacity{0.000000}%
\pgfsetdash{}{0pt}%
\pgfpathmoveto{\pgfqpoint{3.726869in}{0.500000in}}%
\pgfpathlineto{\pgfqpoint{3.753207in}{0.500000in}}%
\pgfpathlineto{\pgfqpoint{3.753207in}{0.505256in}}%
\pgfpathlineto{\pgfqpoint{3.726869in}{0.505256in}}%
\pgfpathlineto{\pgfqpoint{3.726869in}{0.500000in}}%
\pgfpathclose%
\pgfusepath{fill}%
\end{pgfscope}%
\begin{pgfscope}%
\pgfpathrectangle{\pgfqpoint{0.750000in}{0.500000in}}{\pgfqpoint{4.650000in}{3.020000in}}%
\pgfusepath{clip}%
\pgfsetbuttcap%
\pgfsetmiterjoin%
\definecolor{currentfill}{rgb}{0.121569,0.466667,0.705882}%
\pgfsetfillcolor{currentfill}%
\pgfsetlinewidth{0.000000pt}%
\definecolor{currentstroke}{rgb}{0.000000,0.000000,0.000000}%
\pgfsetstrokecolor{currentstroke}%
\pgfsetstrokeopacity{0.000000}%
\pgfsetdash{}{0pt}%
\pgfpathmoveto{\pgfqpoint{3.759792in}{0.500000in}}%
\pgfpathlineto{\pgfqpoint{3.786130in}{0.500000in}}%
\pgfpathlineto{\pgfqpoint{3.786130in}{0.502721in}}%
\pgfpathlineto{\pgfqpoint{3.759792in}{0.502721in}}%
\pgfpathlineto{\pgfqpoint{3.759792in}{0.500000in}}%
\pgfpathclose%
\pgfusepath{fill}%
\end{pgfscope}%
\begin{pgfscope}%
\pgfpathrectangle{\pgfqpoint{0.750000in}{0.500000in}}{\pgfqpoint{4.650000in}{3.020000in}}%
\pgfusepath{clip}%
\pgfsetbuttcap%
\pgfsetmiterjoin%
\definecolor{currentfill}{rgb}{0.121569,0.466667,0.705882}%
\pgfsetfillcolor{currentfill}%
\pgfsetlinewidth{0.000000pt}%
\definecolor{currentstroke}{rgb}{0.000000,0.000000,0.000000}%
\pgfsetstrokecolor{currentstroke}%
\pgfsetstrokeopacity{0.000000}%
\pgfsetdash{}{0pt}%
\pgfpathmoveto{\pgfqpoint{3.792715in}{0.500000in}}%
\pgfpathlineto{\pgfqpoint{3.819053in}{0.500000in}}%
\pgfpathlineto{\pgfqpoint{3.819053in}{0.501360in}}%
\pgfpathlineto{\pgfqpoint{3.792715in}{0.501360in}}%
\pgfpathlineto{\pgfqpoint{3.792715in}{0.500000in}}%
\pgfpathclose%
\pgfusepath{fill}%
\end{pgfscope}%
\begin{pgfscope}%
\pgfpathrectangle{\pgfqpoint{0.750000in}{0.500000in}}{\pgfqpoint{4.650000in}{3.020000in}}%
\pgfusepath{clip}%
\pgfsetbuttcap%
\pgfsetmiterjoin%
\definecolor{currentfill}{rgb}{0.121569,0.466667,0.705882}%
\pgfsetfillcolor{currentfill}%
\pgfsetlinewidth{0.000000pt}%
\definecolor{currentstroke}{rgb}{0.000000,0.000000,0.000000}%
\pgfsetstrokecolor{currentstroke}%
\pgfsetstrokeopacity{0.000000}%
\pgfsetdash{}{0pt}%
\pgfpathmoveto{\pgfqpoint{3.825637in}{0.500000in}}%
\pgfpathlineto{\pgfqpoint{3.851975in}{0.500000in}}%
\pgfpathlineto{\pgfqpoint{3.851975in}{0.500657in}}%
\pgfpathlineto{\pgfqpoint{3.825637in}{0.500657in}}%
\pgfpathlineto{\pgfqpoint{3.825637in}{0.500000in}}%
\pgfpathclose%
\pgfusepath{fill}%
\end{pgfscope}%
\begin{pgfscope}%
\pgfpathrectangle{\pgfqpoint{0.750000in}{0.500000in}}{\pgfqpoint{4.650000in}{3.020000in}}%
\pgfusepath{clip}%
\pgfsetbuttcap%
\pgfsetmiterjoin%
\definecolor{currentfill}{rgb}{0.121569,0.466667,0.705882}%
\pgfsetfillcolor{currentfill}%
\pgfsetlinewidth{0.000000pt}%
\definecolor{currentstroke}{rgb}{0.000000,0.000000,0.000000}%
\pgfsetstrokecolor{currentstroke}%
\pgfsetstrokeopacity{0.000000}%
\pgfsetdash{}{0pt}%
\pgfpathmoveto{\pgfqpoint{3.858560in}{0.500000in}}%
\pgfpathlineto{\pgfqpoint{3.884898in}{0.500000in}}%
\pgfpathlineto{\pgfqpoint{3.884898in}{0.500306in}}%
\pgfpathlineto{\pgfqpoint{3.858560in}{0.500306in}}%
\pgfpathlineto{\pgfqpoint{3.858560in}{0.500000in}}%
\pgfpathclose%
\pgfusepath{fill}%
\end{pgfscope}%
\begin{pgfscope}%
\pgfpathrectangle{\pgfqpoint{0.750000in}{0.500000in}}{\pgfqpoint{4.650000in}{3.020000in}}%
\pgfusepath{clip}%
\pgfsetbuttcap%
\pgfsetmiterjoin%
\definecolor{currentfill}{rgb}{0.121569,0.466667,0.705882}%
\pgfsetfillcolor{currentfill}%
\pgfsetlinewidth{0.000000pt}%
\definecolor{currentstroke}{rgb}{0.000000,0.000000,0.000000}%
\pgfsetstrokecolor{currentstroke}%
\pgfsetstrokeopacity{0.000000}%
\pgfsetdash{}{0pt}%
\pgfpathmoveto{\pgfqpoint{3.891483in}{0.500000in}}%
\pgfpathlineto{\pgfqpoint{3.917821in}{0.500000in}}%
\pgfpathlineto{\pgfqpoint{3.917821in}{0.500138in}}%
\pgfpathlineto{\pgfqpoint{3.891483in}{0.500138in}}%
\pgfpathlineto{\pgfqpoint{3.891483in}{0.500000in}}%
\pgfpathclose%
\pgfusepath{fill}%
\end{pgfscope}%
\begin{pgfscope}%
\pgfpathrectangle{\pgfqpoint{0.750000in}{0.500000in}}{\pgfqpoint{4.650000in}{3.020000in}}%
\pgfusepath{clip}%
\pgfsetbuttcap%
\pgfsetmiterjoin%
\definecolor{currentfill}{rgb}{0.121569,0.466667,0.705882}%
\pgfsetfillcolor{currentfill}%
\pgfsetlinewidth{0.000000pt}%
\definecolor{currentstroke}{rgb}{0.000000,0.000000,0.000000}%
\pgfsetstrokecolor{currentstroke}%
\pgfsetstrokeopacity{0.000000}%
\pgfsetdash{}{0pt}%
\pgfpathmoveto{\pgfqpoint{3.924405in}{0.500000in}}%
\pgfpathlineto{\pgfqpoint{3.950743in}{0.500000in}}%
\pgfpathlineto{\pgfqpoint{3.950743in}{0.500060in}}%
\pgfpathlineto{\pgfqpoint{3.924405in}{0.500060in}}%
\pgfpathlineto{\pgfqpoint{3.924405in}{0.500000in}}%
\pgfpathclose%
\pgfusepath{fill}%
\end{pgfscope}%
\begin{pgfscope}%
\pgfpathrectangle{\pgfqpoint{0.750000in}{0.500000in}}{\pgfqpoint{4.650000in}{3.020000in}}%
\pgfusepath{clip}%
\pgfsetbuttcap%
\pgfsetmiterjoin%
\definecolor{currentfill}{rgb}{0.121569,0.466667,0.705882}%
\pgfsetfillcolor{currentfill}%
\pgfsetlinewidth{0.000000pt}%
\definecolor{currentstroke}{rgb}{0.000000,0.000000,0.000000}%
\pgfsetstrokecolor{currentstroke}%
\pgfsetstrokeopacity{0.000000}%
\pgfsetdash{}{0pt}%
\pgfpathmoveto{\pgfqpoint{3.957328in}{0.500000in}}%
\pgfpathlineto{\pgfqpoint{3.983666in}{0.500000in}}%
\pgfpathlineto{\pgfqpoint{3.983666in}{0.500025in}}%
\pgfpathlineto{\pgfqpoint{3.957328in}{0.500025in}}%
\pgfpathlineto{\pgfqpoint{3.957328in}{0.500000in}}%
\pgfpathclose%
\pgfusepath{fill}%
\end{pgfscope}%
\begin{pgfscope}%
\pgfpathrectangle{\pgfqpoint{0.750000in}{0.500000in}}{\pgfqpoint{4.650000in}{3.020000in}}%
\pgfusepath{clip}%
\pgfsetbuttcap%
\pgfsetmiterjoin%
\definecolor{currentfill}{rgb}{0.121569,0.466667,0.705882}%
\pgfsetfillcolor{currentfill}%
\pgfsetlinewidth{0.000000pt}%
\definecolor{currentstroke}{rgb}{0.000000,0.000000,0.000000}%
\pgfsetstrokecolor{currentstroke}%
\pgfsetstrokeopacity{0.000000}%
\pgfsetdash{}{0pt}%
\pgfpathmoveto{\pgfqpoint{3.990251in}{0.500000in}}%
\pgfpathlineto{\pgfqpoint{4.016589in}{0.500000in}}%
\pgfpathlineto{\pgfqpoint{4.016589in}{0.500010in}}%
\pgfpathlineto{\pgfqpoint{3.990251in}{0.500010in}}%
\pgfpathlineto{\pgfqpoint{3.990251in}{0.500000in}}%
\pgfpathclose%
\pgfusepath{fill}%
\end{pgfscope}%
\begin{pgfscope}%
\pgfpathrectangle{\pgfqpoint{0.750000in}{0.500000in}}{\pgfqpoint{4.650000in}{3.020000in}}%
\pgfusepath{clip}%
\pgfsetbuttcap%
\pgfsetmiterjoin%
\definecolor{currentfill}{rgb}{0.121569,0.466667,0.705882}%
\pgfsetfillcolor{currentfill}%
\pgfsetlinewidth{0.000000pt}%
\definecolor{currentstroke}{rgb}{0.000000,0.000000,0.000000}%
\pgfsetstrokecolor{currentstroke}%
\pgfsetstrokeopacity{0.000000}%
\pgfsetdash{}{0pt}%
\pgfpathmoveto{\pgfqpoint{4.023173in}{0.500000in}}%
\pgfpathlineto{\pgfqpoint{4.049511in}{0.500000in}}%
\pgfpathlineto{\pgfqpoint{4.049511in}{0.500004in}}%
\pgfpathlineto{\pgfqpoint{4.023173in}{0.500004in}}%
\pgfpathlineto{\pgfqpoint{4.023173in}{0.500000in}}%
\pgfpathclose%
\pgfusepath{fill}%
\end{pgfscope}%
\begin{pgfscope}%
\pgfpathrectangle{\pgfqpoint{0.750000in}{0.500000in}}{\pgfqpoint{4.650000in}{3.020000in}}%
\pgfusepath{clip}%
\pgfsetbuttcap%
\pgfsetmiterjoin%
\definecolor{currentfill}{rgb}{0.121569,0.466667,0.705882}%
\pgfsetfillcolor{currentfill}%
\pgfsetlinewidth{0.000000pt}%
\definecolor{currentstroke}{rgb}{0.000000,0.000000,0.000000}%
\pgfsetstrokecolor{currentstroke}%
\pgfsetstrokeopacity{0.000000}%
\pgfsetdash{}{0pt}%
\pgfpathmoveto{\pgfqpoint{4.056096in}{0.500000in}}%
\pgfpathlineto{\pgfqpoint{4.082434in}{0.500000in}}%
\pgfpathlineto{\pgfqpoint{4.082434in}{0.500001in}}%
\pgfpathlineto{\pgfqpoint{4.056096in}{0.500001in}}%
\pgfpathlineto{\pgfqpoint{4.056096in}{0.500000in}}%
\pgfpathclose%
\pgfusepath{fill}%
\end{pgfscope}%
\begin{pgfscope}%
\pgfpathrectangle{\pgfqpoint{0.750000in}{0.500000in}}{\pgfqpoint{4.650000in}{3.020000in}}%
\pgfusepath{clip}%
\pgfsetbuttcap%
\pgfsetmiterjoin%
\definecolor{currentfill}{rgb}{0.121569,0.466667,0.705882}%
\pgfsetfillcolor{currentfill}%
\pgfsetlinewidth{0.000000pt}%
\definecolor{currentstroke}{rgb}{0.000000,0.000000,0.000000}%
\pgfsetstrokecolor{currentstroke}%
\pgfsetstrokeopacity{0.000000}%
\pgfsetdash{}{0pt}%
\pgfpathmoveto{\pgfqpoint{4.089019in}{0.500000in}}%
\pgfpathlineto{\pgfqpoint{4.115357in}{0.500000in}}%
\pgfpathlineto{\pgfqpoint{4.115357in}{0.500001in}}%
\pgfpathlineto{\pgfqpoint{4.089019in}{0.500001in}}%
\pgfpathlineto{\pgfqpoint{4.089019in}{0.500000in}}%
\pgfpathclose%
\pgfusepath{fill}%
\end{pgfscope}%
\begin{pgfscope}%
\pgfpathrectangle{\pgfqpoint{0.750000in}{0.500000in}}{\pgfqpoint{4.650000in}{3.020000in}}%
\pgfusepath{clip}%
\pgfsetbuttcap%
\pgfsetmiterjoin%
\definecolor{currentfill}{rgb}{0.121569,0.466667,0.705882}%
\pgfsetfillcolor{currentfill}%
\pgfsetlinewidth{0.000000pt}%
\definecolor{currentstroke}{rgb}{0.000000,0.000000,0.000000}%
\pgfsetstrokecolor{currentstroke}%
\pgfsetstrokeopacity{0.000000}%
\pgfsetdash{}{0pt}%
\pgfpathmoveto{\pgfqpoint{4.121941in}{0.500000in}}%
\pgfpathlineto{\pgfqpoint{4.148280in}{0.500000in}}%
\pgfpathlineto{\pgfqpoint{4.148280in}{0.500000in}}%
\pgfpathlineto{\pgfqpoint{4.121941in}{0.500000in}}%
\pgfpathlineto{\pgfqpoint{4.121941in}{0.500000in}}%
\pgfpathclose%
\pgfusepath{fill}%
\end{pgfscope}%
\begin{pgfscope}%
\pgfpathrectangle{\pgfqpoint{0.750000in}{0.500000in}}{\pgfqpoint{4.650000in}{3.020000in}}%
\pgfusepath{clip}%
\pgfsetbuttcap%
\pgfsetmiterjoin%
\definecolor{currentfill}{rgb}{0.121569,0.466667,0.705882}%
\pgfsetfillcolor{currentfill}%
\pgfsetlinewidth{0.000000pt}%
\definecolor{currentstroke}{rgb}{0.000000,0.000000,0.000000}%
\pgfsetstrokecolor{currentstroke}%
\pgfsetstrokeopacity{0.000000}%
\pgfsetdash{}{0pt}%
\pgfpathmoveto{\pgfqpoint{4.154864in}{0.500000in}}%
\pgfpathlineto{\pgfqpoint{4.181202in}{0.500000in}}%
\pgfpathlineto{\pgfqpoint{4.181202in}{0.500000in}}%
\pgfpathlineto{\pgfqpoint{4.154864in}{0.500000in}}%
\pgfpathlineto{\pgfqpoint{4.154864in}{0.500000in}}%
\pgfpathclose%
\pgfusepath{fill}%
\end{pgfscope}%
\begin{pgfscope}%
\pgfpathrectangle{\pgfqpoint{0.750000in}{0.500000in}}{\pgfqpoint{4.650000in}{3.020000in}}%
\pgfusepath{clip}%
\pgfsetbuttcap%
\pgfsetmiterjoin%
\definecolor{currentfill}{rgb}{0.121569,0.466667,0.705882}%
\pgfsetfillcolor{currentfill}%
\pgfsetlinewidth{0.000000pt}%
\definecolor{currentstroke}{rgb}{0.000000,0.000000,0.000000}%
\pgfsetstrokecolor{currentstroke}%
\pgfsetstrokeopacity{0.000000}%
\pgfsetdash{}{0pt}%
\pgfpathmoveto{\pgfqpoint{4.187787in}{0.500000in}}%
\pgfpathlineto{\pgfqpoint{4.214125in}{0.500000in}}%
\pgfpathlineto{\pgfqpoint{4.214125in}{0.500000in}}%
\pgfpathlineto{\pgfqpoint{4.187787in}{0.500000in}}%
\pgfpathlineto{\pgfqpoint{4.187787in}{0.500000in}}%
\pgfpathclose%
\pgfusepath{fill}%
\end{pgfscope}%
\begin{pgfscope}%
\pgfpathrectangle{\pgfqpoint{0.750000in}{0.500000in}}{\pgfqpoint{4.650000in}{3.020000in}}%
\pgfusepath{clip}%
\pgfsetbuttcap%
\pgfsetmiterjoin%
\definecolor{currentfill}{rgb}{0.121569,0.466667,0.705882}%
\pgfsetfillcolor{currentfill}%
\pgfsetlinewidth{0.000000pt}%
\definecolor{currentstroke}{rgb}{0.000000,0.000000,0.000000}%
\pgfsetstrokecolor{currentstroke}%
\pgfsetstrokeopacity{0.000000}%
\pgfsetdash{}{0pt}%
\pgfpathmoveto{\pgfqpoint{4.220709in}{0.500000in}}%
\pgfpathlineto{\pgfqpoint{4.247048in}{0.500000in}}%
\pgfpathlineto{\pgfqpoint{4.247048in}{0.500000in}}%
\pgfpathlineto{\pgfqpoint{4.220709in}{0.500000in}}%
\pgfpathlineto{\pgfqpoint{4.220709in}{0.500000in}}%
\pgfpathclose%
\pgfusepath{fill}%
\end{pgfscope}%
\begin{pgfscope}%
\pgfpathrectangle{\pgfqpoint{0.750000in}{0.500000in}}{\pgfqpoint{4.650000in}{3.020000in}}%
\pgfusepath{clip}%
\pgfsetbuttcap%
\pgfsetmiterjoin%
\definecolor{currentfill}{rgb}{0.121569,0.466667,0.705882}%
\pgfsetfillcolor{currentfill}%
\pgfsetlinewidth{0.000000pt}%
\definecolor{currentstroke}{rgb}{0.000000,0.000000,0.000000}%
\pgfsetstrokecolor{currentstroke}%
\pgfsetstrokeopacity{0.000000}%
\pgfsetdash{}{0pt}%
\pgfpathmoveto{\pgfqpoint{4.253632in}{0.500000in}}%
\pgfpathlineto{\pgfqpoint{4.279970in}{0.500000in}}%
\pgfpathlineto{\pgfqpoint{4.279970in}{0.500000in}}%
\pgfpathlineto{\pgfqpoint{4.253632in}{0.500000in}}%
\pgfpathlineto{\pgfqpoint{4.253632in}{0.500000in}}%
\pgfpathclose%
\pgfusepath{fill}%
\end{pgfscope}%
\begin{pgfscope}%
\pgfpathrectangle{\pgfqpoint{0.750000in}{0.500000in}}{\pgfqpoint{4.650000in}{3.020000in}}%
\pgfusepath{clip}%
\pgfsetbuttcap%
\pgfsetmiterjoin%
\definecolor{currentfill}{rgb}{0.121569,0.466667,0.705882}%
\pgfsetfillcolor{currentfill}%
\pgfsetlinewidth{0.000000pt}%
\definecolor{currentstroke}{rgb}{0.000000,0.000000,0.000000}%
\pgfsetstrokecolor{currentstroke}%
\pgfsetstrokeopacity{0.000000}%
\pgfsetdash{}{0pt}%
\pgfpathmoveto{\pgfqpoint{4.286555in}{0.500000in}}%
\pgfpathlineto{\pgfqpoint{4.312893in}{0.500000in}}%
\pgfpathlineto{\pgfqpoint{4.312893in}{0.500000in}}%
\pgfpathlineto{\pgfqpoint{4.286555in}{0.500000in}}%
\pgfpathlineto{\pgfqpoint{4.286555in}{0.500000in}}%
\pgfpathclose%
\pgfusepath{fill}%
\end{pgfscope}%
\begin{pgfscope}%
\pgfpathrectangle{\pgfqpoint{0.750000in}{0.500000in}}{\pgfqpoint{4.650000in}{3.020000in}}%
\pgfusepath{clip}%
\pgfsetbuttcap%
\pgfsetmiterjoin%
\definecolor{currentfill}{rgb}{0.121569,0.466667,0.705882}%
\pgfsetfillcolor{currentfill}%
\pgfsetlinewidth{0.000000pt}%
\definecolor{currentstroke}{rgb}{0.000000,0.000000,0.000000}%
\pgfsetstrokecolor{currentstroke}%
\pgfsetstrokeopacity{0.000000}%
\pgfsetdash{}{0pt}%
\pgfpathmoveto{\pgfqpoint{4.319477in}{0.500000in}}%
\pgfpathlineto{\pgfqpoint{4.345816in}{0.500000in}}%
\pgfpathlineto{\pgfqpoint{4.345816in}{0.500000in}}%
\pgfpathlineto{\pgfqpoint{4.319477in}{0.500000in}}%
\pgfpathlineto{\pgfqpoint{4.319477in}{0.500000in}}%
\pgfpathclose%
\pgfusepath{fill}%
\end{pgfscope}%
\begin{pgfscope}%
\pgfpathrectangle{\pgfqpoint{0.750000in}{0.500000in}}{\pgfqpoint{4.650000in}{3.020000in}}%
\pgfusepath{clip}%
\pgfsetbuttcap%
\pgfsetmiterjoin%
\definecolor{currentfill}{rgb}{0.121569,0.466667,0.705882}%
\pgfsetfillcolor{currentfill}%
\pgfsetlinewidth{0.000000pt}%
\definecolor{currentstroke}{rgb}{0.000000,0.000000,0.000000}%
\pgfsetstrokecolor{currentstroke}%
\pgfsetstrokeopacity{0.000000}%
\pgfsetdash{}{0pt}%
\pgfpathmoveto{\pgfqpoint{4.352400in}{0.500000in}}%
\pgfpathlineto{\pgfqpoint{4.378738in}{0.500000in}}%
\pgfpathlineto{\pgfqpoint{4.378738in}{0.500000in}}%
\pgfpathlineto{\pgfqpoint{4.352400in}{0.500000in}}%
\pgfpathlineto{\pgfqpoint{4.352400in}{0.500000in}}%
\pgfpathclose%
\pgfusepath{fill}%
\end{pgfscope}%
\begin{pgfscope}%
\pgfpathrectangle{\pgfqpoint{0.750000in}{0.500000in}}{\pgfqpoint{4.650000in}{3.020000in}}%
\pgfusepath{clip}%
\pgfsetbuttcap%
\pgfsetmiterjoin%
\definecolor{currentfill}{rgb}{0.121569,0.466667,0.705882}%
\pgfsetfillcolor{currentfill}%
\pgfsetlinewidth{0.000000pt}%
\definecolor{currentstroke}{rgb}{0.000000,0.000000,0.000000}%
\pgfsetstrokecolor{currentstroke}%
\pgfsetstrokeopacity{0.000000}%
\pgfsetdash{}{0pt}%
\pgfpathmoveto{\pgfqpoint{4.385323in}{0.500000in}}%
\pgfpathlineto{\pgfqpoint{4.411661in}{0.500000in}}%
\pgfpathlineto{\pgfqpoint{4.411661in}{0.500000in}}%
\pgfpathlineto{\pgfqpoint{4.385323in}{0.500000in}}%
\pgfpathlineto{\pgfqpoint{4.385323in}{0.500000in}}%
\pgfpathclose%
\pgfusepath{fill}%
\end{pgfscope}%
\begin{pgfscope}%
\pgfpathrectangle{\pgfqpoint{0.750000in}{0.500000in}}{\pgfqpoint{4.650000in}{3.020000in}}%
\pgfusepath{clip}%
\pgfsetbuttcap%
\pgfsetmiterjoin%
\definecolor{currentfill}{rgb}{0.121569,0.466667,0.705882}%
\pgfsetfillcolor{currentfill}%
\pgfsetlinewidth{0.000000pt}%
\definecolor{currentstroke}{rgb}{0.000000,0.000000,0.000000}%
\pgfsetstrokecolor{currentstroke}%
\pgfsetstrokeopacity{0.000000}%
\pgfsetdash{}{0pt}%
\pgfpathmoveto{\pgfqpoint{4.418246in}{0.500000in}}%
\pgfpathlineto{\pgfqpoint{4.444584in}{0.500000in}}%
\pgfpathlineto{\pgfqpoint{4.444584in}{0.500000in}}%
\pgfpathlineto{\pgfqpoint{4.418246in}{0.500000in}}%
\pgfpathlineto{\pgfqpoint{4.418246in}{0.500000in}}%
\pgfpathclose%
\pgfusepath{fill}%
\end{pgfscope}%
\begin{pgfscope}%
\pgfpathrectangle{\pgfqpoint{0.750000in}{0.500000in}}{\pgfqpoint{4.650000in}{3.020000in}}%
\pgfusepath{clip}%
\pgfsetbuttcap%
\pgfsetmiterjoin%
\definecolor{currentfill}{rgb}{0.121569,0.466667,0.705882}%
\pgfsetfillcolor{currentfill}%
\pgfsetlinewidth{0.000000pt}%
\definecolor{currentstroke}{rgb}{0.000000,0.000000,0.000000}%
\pgfsetstrokecolor{currentstroke}%
\pgfsetstrokeopacity{0.000000}%
\pgfsetdash{}{0pt}%
\pgfpathmoveto{\pgfqpoint{4.451168in}{0.500000in}}%
\pgfpathlineto{\pgfqpoint{4.477506in}{0.500000in}}%
\pgfpathlineto{\pgfqpoint{4.477506in}{0.500000in}}%
\pgfpathlineto{\pgfqpoint{4.451168in}{0.500000in}}%
\pgfpathlineto{\pgfqpoint{4.451168in}{0.500000in}}%
\pgfpathclose%
\pgfusepath{fill}%
\end{pgfscope}%
\begin{pgfscope}%
\pgfpathrectangle{\pgfqpoint{0.750000in}{0.500000in}}{\pgfqpoint{4.650000in}{3.020000in}}%
\pgfusepath{clip}%
\pgfsetbuttcap%
\pgfsetmiterjoin%
\definecolor{currentfill}{rgb}{0.121569,0.466667,0.705882}%
\pgfsetfillcolor{currentfill}%
\pgfsetlinewidth{0.000000pt}%
\definecolor{currentstroke}{rgb}{0.000000,0.000000,0.000000}%
\pgfsetstrokecolor{currentstroke}%
\pgfsetstrokeopacity{0.000000}%
\pgfsetdash{}{0pt}%
\pgfpathmoveto{\pgfqpoint{4.484091in}{0.500000in}}%
\pgfpathlineto{\pgfqpoint{4.510429in}{0.500000in}}%
\pgfpathlineto{\pgfqpoint{4.510429in}{0.500000in}}%
\pgfpathlineto{\pgfqpoint{4.484091in}{0.500000in}}%
\pgfpathlineto{\pgfqpoint{4.484091in}{0.500000in}}%
\pgfpathclose%
\pgfusepath{fill}%
\end{pgfscope}%
\begin{pgfscope}%
\pgfpathrectangle{\pgfqpoint{0.750000in}{0.500000in}}{\pgfqpoint{4.650000in}{3.020000in}}%
\pgfusepath{clip}%
\pgfsetbuttcap%
\pgfsetmiterjoin%
\definecolor{currentfill}{rgb}{0.121569,0.466667,0.705882}%
\pgfsetfillcolor{currentfill}%
\pgfsetlinewidth{0.000000pt}%
\definecolor{currentstroke}{rgb}{0.000000,0.000000,0.000000}%
\pgfsetstrokecolor{currentstroke}%
\pgfsetstrokeopacity{0.000000}%
\pgfsetdash{}{0pt}%
\pgfpathmoveto{\pgfqpoint{4.517014in}{0.500000in}}%
\pgfpathlineto{\pgfqpoint{4.543352in}{0.500000in}}%
\pgfpathlineto{\pgfqpoint{4.543352in}{0.500000in}}%
\pgfpathlineto{\pgfqpoint{4.517014in}{0.500000in}}%
\pgfpathlineto{\pgfqpoint{4.517014in}{0.500000in}}%
\pgfpathclose%
\pgfusepath{fill}%
\end{pgfscope}%
\begin{pgfscope}%
\pgfpathrectangle{\pgfqpoint{0.750000in}{0.500000in}}{\pgfqpoint{4.650000in}{3.020000in}}%
\pgfusepath{clip}%
\pgfsetbuttcap%
\pgfsetmiterjoin%
\definecolor{currentfill}{rgb}{0.121569,0.466667,0.705882}%
\pgfsetfillcolor{currentfill}%
\pgfsetlinewidth{0.000000pt}%
\definecolor{currentstroke}{rgb}{0.000000,0.000000,0.000000}%
\pgfsetstrokecolor{currentstroke}%
\pgfsetstrokeopacity{0.000000}%
\pgfsetdash{}{0pt}%
\pgfpathmoveto{\pgfqpoint{4.549936in}{0.500000in}}%
\pgfpathlineto{\pgfqpoint{4.576274in}{0.500000in}}%
\pgfpathlineto{\pgfqpoint{4.576274in}{0.500000in}}%
\pgfpathlineto{\pgfqpoint{4.549936in}{0.500000in}}%
\pgfpathlineto{\pgfqpoint{4.549936in}{0.500000in}}%
\pgfpathclose%
\pgfusepath{fill}%
\end{pgfscope}%
\begin{pgfscope}%
\pgfpathrectangle{\pgfqpoint{0.750000in}{0.500000in}}{\pgfqpoint{4.650000in}{3.020000in}}%
\pgfusepath{clip}%
\pgfsetbuttcap%
\pgfsetmiterjoin%
\definecolor{currentfill}{rgb}{0.121569,0.466667,0.705882}%
\pgfsetfillcolor{currentfill}%
\pgfsetlinewidth{0.000000pt}%
\definecolor{currentstroke}{rgb}{0.000000,0.000000,0.000000}%
\pgfsetstrokecolor{currentstroke}%
\pgfsetstrokeopacity{0.000000}%
\pgfsetdash{}{0pt}%
\pgfpathmoveto{\pgfqpoint{4.582859in}{0.500000in}}%
\pgfpathlineto{\pgfqpoint{4.609197in}{0.500000in}}%
\pgfpathlineto{\pgfqpoint{4.609197in}{0.500000in}}%
\pgfpathlineto{\pgfqpoint{4.582859in}{0.500000in}}%
\pgfpathlineto{\pgfqpoint{4.582859in}{0.500000in}}%
\pgfpathclose%
\pgfusepath{fill}%
\end{pgfscope}%
\begin{pgfscope}%
\pgfpathrectangle{\pgfqpoint{0.750000in}{0.500000in}}{\pgfqpoint{4.650000in}{3.020000in}}%
\pgfusepath{clip}%
\pgfsetbuttcap%
\pgfsetmiterjoin%
\definecolor{currentfill}{rgb}{0.121569,0.466667,0.705882}%
\pgfsetfillcolor{currentfill}%
\pgfsetlinewidth{0.000000pt}%
\definecolor{currentstroke}{rgb}{0.000000,0.000000,0.000000}%
\pgfsetstrokecolor{currentstroke}%
\pgfsetstrokeopacity{0.000000}%
\pgfsetdash{}{0pt}%
\pgfpathmoveto{\pgfqpoint{4.615782in}{0.500000in}}%
\pgfpathlineto{\pgfqpoint{4.642120in}{0.500000in}}%
\pgfpathlineto{\pgfqpoint{4.642120in}{0.500000in}}%
\pgfpathlineto{\pgfqpoint{4.615782in}{0.500000in}}%
\pgfpathlineto{\pgfqpoint{4.615782in}{0.500000in}}%
\pgfpathclose%
\pgfusepath{fill}%
\end{pgfscope}%
\begin{pgfscope}%
\pgfpathrectangle{\pgfqpoint{0.750000in}{0.500000in}}{\pgfqpoint{4.650000in}{3.020000in}}%
\pgfusepath{clip}%
\pgfsetbuttcap%
\pgfsetmiterjoin%
\definecolor{currentfill}{rgb}{0.121569,0.466667,0.705882}%
\pgfsetfillcolor{currentfill}%
\pgfsetlinewidth{0.000000pt}%
\definecolor{currentstroke}{rgb}{0.000000,0.000000,0.000000}%
\pgfsetstrokecolor{currentstroke}%
\pgfsetstrokeopacity{0.000000}%
\pgfsetdash{}{0pt}%
\pgfpathmoveto{\pgfqpoint{4.648704in}{0.500000in}}%
\pgfpathlineto{\pgfqpoint{4.675042in}{0.500000in}}%
\pgfpathlineto{\pgfqpoint{4.675042in}{0.500000in}}%
\pgfpathlineto{\pgfqpoint{4.648704in}{0.500000in}}%
\pgfpathlineto{\pgfqpoint{4.648704in}{0.500000in}}%
\pgfpathclose%
\pgfusepath{fill}%
\end{pgfscope}%
\begin{pgfscope}%
\pgfpathrectangle{\pgfqpoint{0.750000in}{0.500000in}}{\pgfqpoint{4.650000in}{3.020000in}}%
\pgfusepath{clip}%
\pgfsetbuttcap%
\pgfsetmiterjoin%
\definecolor{currentfill}{rgb}{0.121569,0.466667,0.705882}%
\pgfsetfillcolor{currentfill}%
\pgfsetlinewidth{0.000000pt}%
\definecolor{currentstroke}{rgb}{0.000000,0.000000,0.000000}%
\pgfsetstrokecolor{currentstroke}%
\pgfsetstrokeopacity{0.000000}%
\pgfsetdash{}{0pt}%
\pgfpathmoveto{\pgfqpoint{4.681627in}{0.500000in}}%
\pgfpathlineto{\pgfqpoint{4.707965in}{0.500000in}}%
\pgfpathlineto{\pgfqpoint{4.707965in}{0.500000in}}%
\pgfpathlineto{\pgfqpoint{4.681627in}{0.500000in}}%
\pgfpathlineto{\pgfqpoint{4.681627in}{0.500000in}}%
\pgfpathclose%
\pgfusepath{fill}%
\end{pgfscope}%
\begin{pgfscope}%
\pgfpathrectangle{\pgfqpoint{0.750000in}{0.500000in}}{\pgfqpoint{4.650000in}{3.020000in}}%
\pgfusepath{clip}%
\pgfsetbuttcap%
\pgfsetmiterjoin%
\definecolor{currentfill}{rgb}{0.121569,0.466667,0.705882}%
\pgfsetfillcolor{currentfill}%
\pgfsetlinewidth{0.000000pt}%
\definecolor{currentstroke}{rgb}{0.000000,0.000000,0.000000}%
\pgfsetstrokecolor{currentstroke}%
\pgfsetstrokeopacity{0.000000}%
\pgfsetdash{}{0pt}%
\pgfpathmoveto{\pgfqpoint{4.714550in}{0.500000in}}%
\pgfpathlineto{\pgfqpoint{4.740888in}{0.500000in}}%
\pgfpathlineto{\pgfqpoint{4.740888in}{0.500000in}}%
\pgfpathlineto{\pgfqpoint{4.714550in}{0.500000in}}%
\pgfpathlineto{\pgfqpoint{4.714550in}{0.500000in}}%
\pgfpathclose%
\pgfusepath{fill}%
\end{pgfscope}%
\begin{pgfscope}%
\pgfpathrectangle{\pgfqpoint{0.750000in}{0.500000in}}{\pgfqpoint{4.650000in}{3.020000in}}%
\pgfusepath{clip}%
\pgfsetbuttcap%
\pgfsetmiterjoin%
\definecolor{currentfill}{rgb}{0.121569,0.466667,0.705882}%
\pgfsetfillcolor{currentfill}%
\pgfsetlinewidth{0.000000pt}%
\definecolor{currentstroke}{rgb}{0.000000,0.000000,0.000000}%
\pgfsetstrokecolor{currentstroke}%
\pgfsetstrokeopacity{0.000000}%
\pgfsetdash{}{0pt}%
\pgfpathmoveto{\pgfqpoint{4.747472in}{0.500000in}}%
\pgfpathlineto{\pgfqpoint{4.773811in}{0.500000in}}%
\pgfpathlineto{\pgfqpoint{4.773811in}{0.500000in}}%
\pgfpathlineto{\pgfqpoint{4.747472in}{0.500000in}}%
\pgfpathlineto{\pgfqpoint{4.747472in}{0.500000in}}%
\pgfpathclose%
\pgfusepath{fill}%
\end{pgfscope}%
\begin{pgfscope}%
\pgfpathrectangle{\pgfqpoint{0.750000in}{0.500000in}}{\pgfqpoint{4.650000in}{3.020000in}}%
\pgfusepath{clip}%
\pgfsetbuttcap%
\pgfsetmiterjoin%
\definecolor{currentfill}{rgb}{0.121569,0.466667,0.705882}%
\pgfsetfillcolor{currentfill}%
\pgfsetlinewidth{0.000000pt}%
\definecolor{currentstroke}{rgb}{0.000000,0.000000,0.000000}%
\pgfsetstrokecolor{currentstroke}%
\pgfsetstrokeopacity{0.000000}%
\pgfsetdash{}{0pt}%
\pgfpathmoveto{\pgfqpoint{4.780395in}{0.500000in}}%
\pgfpathlineto{\pgfqpoint{4.806733in}{0.500000in}}%
\pgfpathlineto{\pgfqpoint{4.806733in}{0.500000in}}%
\pgfpathlineto{\pgfqpoint{4.780395in}{0.500000in}}%
\pgfpathlineto{\pgfqpoint{4.780395in}{0.500000in}}%
\pgfpathclose%
\pgfusepath{fill}%
\end{pgfscope}%
\begin{pgfscope}%
\pgfpathrectangle{\pgfqpoint{0.750000in}{0.500000in}}{\pgfqpoint{4.650000in}{3.020000in}}%
\pgfusepath{clip}%
\pgfsetbuttcap%
\pgfsetmiterjoin%
\definecolor{currentfill}{rgb}{0.121569,0.466667,0.705882}%
\pgfsetfillcolor{currentfill}%
\pgfsetlinewidth{0.000000pt}%
\definecolor{currentstroke}{rgb}{0.000000,0.000000,0.000000}%
\pgfsetstrokecolor{currentstroke}%
\pgfsetstrokeopacity{0.000000}%
\pgfsetdash{}{0pt}%
\pgfpathmoveto{\pgfqpoint{4.813318in}{0.500000in}}%
\pgfpathlineto{\pgfqpoint{4.839656in}{0.500000in}}%
\pgfpathlineto{\pgfqpoint{4.839656in}{0.500000in}}%
\pgfpathlineto{\pgfqpoint{4.813318in}{0.500000in}}%
\pgfpathlineto{\pgfqpoint{4.813318in}{0.500000in}}%
\pgfpathclose%
\pgfusepath{fill}%
\end{pgfscope}%
\begin{pgfscope}%
\pgfpathrectangle{\pgfqpoint{0.750000in}{0.500000in}}{\pgfqpoint{4.650000in}{3.020000in}}%
\pgfusepath{clip}%
\pgfsetbuttcap%
\pgfsetmiterjoin%
\definecolor{currentfill}{rgb}{0.121569,0.466667,0.705882}%
\pgfsetfillcolor{currentfill}%
\pgfsetlinewidth{0.000000pt}%
\definecolor{currentstroke}{rgb}{0.000000,0.000000,0.000000}%
\pgfsetstrokecolor{currentstroke}%
\pgfsetstrokeopacity{0.000000}%
\pgfsetdash{}{0pt}%
\pgfpathmoveto{\pgfqpoint{4.846240in}{0.500000in}}%
\pgfpathlineto{\pgfqpoint{4.872579in}{0.500000in}}%
\pgfpathlineto{\pgfqpoint{4.872579in}{0.500000in}}%
\pgfpathlineto{\pgfqpoint{4.846240in}{0.500000in}}%
\pgfpathlineto{\pgfqpoint{4.846240in}{0.500000in}}%
\pgfpathclose%
\pgfusepath{fill}%
\end{pgfscope}%
\begin{pgfscope}%
\pgfpathrectangle{\pgfqpoint{0.750000in}{0.500000in}}{\pgfqpoint{4.650000in}{3.020000in}}%
\pgfusepath{clip}%
\pgfsetbuttcap%
\pgfsetmiterjoin%
\definecolor{currentfill}{rgb}{0.121569,0.466667,0.705882}%
\pgfsetfillcolor{currentfill}%
\pgfsetlinewidth{0.000000pt}%
\definecolor{currentstroke}{rgb}{0.000000,0.000000,0.000000}%
\pgfsetstrokecolor{currentstroke}%
\pgfsetstrokeopacity{0.000000}%
\pgfsetdash{}{0pt}%
\pgfpathmoveto{\pgfqpoint{4.879163in}{0.500000in}}%
\pgfpathlineto{\pgfqpoint{4.905501in}{0.500000in}}%
\pgfpathlineto{\pgfqpoint{4.905501in}{0.500000in}}%
\pgfpathlineto{\pgfqpoint{4.879163in}{0.500000in}}%
\pgfpathlineto{\pgfqpoint{4.879163in}{0.500000in}}%
\pgfpathclose%
\pgfusepath{fill}%
\end{pgfscope}%
\begin{pgfscope}%
\pgfpathrectangle{\pgfqpoint{0.750000in}{0.500000in}}{\pgfqpoint{4.650000in}{3.020000in}}%
\pgfusepath{clip}%
\pgfsetbuttcap%
\pgfsetmiterjoin%
\definecolor{currentfill}{rgb}{0.121569,0.466667,0.705882}%
\pgfsetfillcolor{currentfill}%
\pgfsetlinewidth{0.000000pt}%
\definecolor{currentstroke}{rgb}{0.000000,0.000000,0.000000}%
\pgfsetstrokecolor{currentstroke}%
\pgfsetstrokeopacity{0.000000}%
\pgfsetdash{}{0pt}%
\pgfpathmoveto{\pgfqpoint{4.912086in}{0.500000in}}%
\pgfpathlineto{\pgfqpoint{4.938424in}{0.500000in}}%
\pgfpathlineto{\pgfqpoint{4.938424in}{0.500000in}}%
\pgfpathlineto{\pgfqpoint{4.912086in}{0.500000in}}%
\pgfpathlineto{\pgfqpoint{4.912086in}{0.500000in}}%
\pgfpathclose%
\pgfusepath{fill}%
\end{pgfscope}%
\begin{pgfscope}%
\pgfpathrectangle{\pgfqpoint{0.750000in}{0.500000in}}{\pgfqpoint{4.650000in}{3.020000in}}%
\pgfusepath{clip}%
\pgfsetbuttcap%
\pgfsetmiterjoin%
\definecolor{currentfill}{rgb}{0.121569,0.466667,0.705882}%
\pgfsetfillcolor{currentfill}%
\pgfsetlinewidth{0.000000pt}%
\definecolor{currentstroke}{rgb}{0.000000,0.000000,0.000000}%
\pgfsetstrokecolor{currentstroke}%
\pgfsetstrokeopacity{0.000000}%
\pgfsetdash{}{0pt}%
\pgfpathmoveto{\pgfqpoint{4.945008in}{0.500000in}}%
\pgfpathlineto{\pgfqpoint{4.971347in}{0.500000in}}%
\pgfpathlineto{\pgfqpoint{4.971347in}{0.500000in}}%
\pgfpathlineto{\pgfqpoint{4.945008in}{0.500000in}}%
\pgfpathlineto{\pgfqpoint{4.945008in}{0.500000in}}%
\pgfpathclose%
\pgfusepath{fill}%
\end{pgfscope}%
\begin{pgfscope}%
\pgfpathrectangle{\pgfqpoint{0.750000in}{0.500000in}}{\pgfqpoint{4.650000in}{3.020000in}}%
\pgfusepath{clip}%
\pgfsetbuttcap%
\pgfsetmiterjoin%
\definecolor{currentfill}{rgb}{0.121569,0.466667,0.705882}%
\pgfsetfillcolor{currentfill}%
\pgfsetlinewidth{0.000000pt}%
\definecolor{currentstroke}{rgb}{0.000000,0.000000,0.000000}%
\pgfsetstrokecolor{currentstroke}%
\pgfsetstrokeopacity{0.000000}%
\pgfsetdash{}{0pt}%
\pgfpathmoveto{\pgfqpoint{4.977931in}{0.500000in}}%
\pgfpathlineto{\pgfqpoint{5.004269in}{0.500000in}}%
\pgfpathlineto{\pgfqpoint{5.004269in}{0.500000in}}%
\pgfpathlineto{\pgfqpoint{4.977931in}{0.500000in}}%
\pgfpathlineto{\pgfqpoint{4.977931in}{0.500000in}}%
\pgfpathclose%
\pgfusepath{fill}%
\end{pgfscope}%
\begin{pgfscope}%
\pgfpathrectangle{\pgfqpoint{0.750000in}{0.500000in}}{\pgfqpoint{4.650000in}{3.020000in}}%
\pgfusepath{clip}%
\pgfsetbuttcap%
\pgfsetmiterjoin%
\definecolor{currentfill}{rgb}{0.121569,0.466667,0.705882}%
\pgfsetfillcolor{currentfill}%
\pgfsetlinewidth{0.000000pt}%
\definecolor{currentstroke}{rgb}{0.000000,0.000000,0.000000}%
\pgfsetstrokecolor{currentstroke}%
\pgfsetstrokeopacity{0.000000}%
\pgfsetdash{}{0pt}%
\pgfpathmoveto{\pgfqpoint{5.010854in}{0.500000in}}%
\pgfpathlineto{\pgfqpoint{5.037192in}{0.500000in}}%
\pgfpathlineto{\pgfqpoint{5.037192in}{0.500000in}}%
\pgfpathlineto{\pgfqpoint{5.010854in}{0.500000in}}%
\pgfpathlineto{\pgfqpoint{5.010854in}{0.500000in}}%
\pgfpathclose%
\pgfusepath{fill}%
\end{pgfscope}%
\begin{pgfscope}%
\pgfpathrectangle{\pgfqpoint{0.750000in}{0.500000in}}{\pgfqpoint{4.650000in}{3.020000in}}%
\pgfusepath{clip}%
\pgfsetbuttcap%
\pgfsetmiterjoin%
\definecolor{currentfill}{rgb}{0.121569,0.466667,0.705882}%
\pgfsetfillcolor{currentfill}%
\pgfsetlinewidth{0.000000pt}%
\definecolor{currentstroke}{rgb}{0.000000,0.000000,0.000000}%
\pgfsetstrokecolor{currentstroke}%
\pgfsetstrokeopacity{0.000000}%
\pgfsetdash{}{0pt}%
\pgfpathmoveto{\pgfqpoint{5.043777in}{0.500000in}}%
\pgfpathlineto{\pgfqpoint{5.070115in}{0.500000in}}%
\pgfpathlineto{\pgfqpoint{5.070115in}{0.500000in}}%
\pgfpathlineto{\pgfqpoint{5.043777in}{0.500000in}}%
\pgfpathlineto{\pgfqpoint{5.043777in}{0.500000in}}%
\pgfpathclose%
\pgfusepath{fill}%
\end{pgfscope}%
\begin{pgfscope}%
\pgfpathrectangle{\pgfqpoint{0.750000in}{0.500000in}}{\pgfqpoint{4.650000in}{3.020000in}}%
\pgfusepath{clip}%
\pgfsetbuttcap%
\pgfsetmiterjoin%
\definecolor{currentfill}{rgb}{0.121569,0.466667,0.705882}%
\pgfsetfillcolor{currentfill}%
\pgfsetlinewidth{0.000000pt}%
\definecolor{currentstroke}{rgb}{0.000000,0.000000,0.000000}%
\pgfsetstrokecolor{currentstroke}%
\pgfsetstrokeopacity{0.000000}%
\pgfsetdash{}{0pt}%
\pgfpathmoveto{\pgfqpoint{5.076699in}{0.500000in}}%
\pgfpathlineto{\pgfqpoint{5.103037in}{0.500000in}}%
\pgfpathlineto{\pgfqpoint{5.103037in}{0.500000in}}%
\pgfpathlineto{\pgfqpoint{5.076699in}{0.500000in}}%
\pgfpathlineto{\pgfqpoint{5.076699in}{0.500000in}}%
\pgfpathclose%
\pgfusepath{fill}%
\end{pgfscope}%
\begin{pgfscope}%
\pgfpathrectangle{\pgfqpoint{0.750000in}{0.500000in}}{\pgfqpoint{4.650000in}{3.020000in}}%
\pgfusepath{clip}%
\pgfsetbuttcap%
\pgfsetmiterjoin%
\definecolor{currentfill}{rgb}{0.121569,0.466667,0.705882}%
\pgfsetfillcolor{currentfill}%
\pgfsetlinewidth{0.000000pt}%
\definecolor{currentstroke}{rgb}{0.000000,0.000000,0.000000}%
\pgfsetstrokecolor{currentstroke}%
\pgfsetstrokeopacity{0.000000}%
\pgfsetdash{}{0pt}%
\pgfpathmoveto{\pgfqpoint{5.109622in}{0.500000in}}%
\pgfpathlineto{\pgfqpoint{5.135960in}{0.500000in}}%
\pgfpathlineto{\pgfqpoint{5.135960in}{0.500000in}}%
\pgfpathlineto{\pgfqpoint{5.109622in}{0.500000in}}%
\pgfpathlineto{\pgfqpoint{5.109622in}{0.500000in}}%
\pgfpathclose%
\pgfusepath{fill}%
\end{pgfscope}%
\begin{pgfscope}%
\pgfpathrectangle{\pgfqpoint{0.750000in}{0.500000in}}{\pgfqpoint{4.650000in}{3.020000in}}%
\pgfusepath{clip}%
\pgfsetbuttcap%
\pgfsetmiterjoin%
\definecolor{currentfill}{rgb}{0.121569,0.466667,0.705882}%
\pgfsetfillcolor{currentfill}%
\pgfsetlinewidth{0.000000pt}%
\definecolor{currentstroke}{rgb}{0.000000,0.000000,0.000000}%
\pgfsetstrokecolor{currentstroke}%
\pgfsetstrokeopacity{0.000000}%
\pgfsetdash{}{0pt}%
\pgfpathmoveto{\pgfqpoint{5.142545in}{0.500000in}}%
\pgfpathlineto{\pgfqpoint{5.168883in}{0.500000in}}%
\pgfpathlineto{\pgfqpoint{5.168883in}{0.500000in}}%
\pgfpathlineto{\pgfqpoint{5.142545in}{0.500000in}}%
\pgfpathlineto{\pgfqpoint{5.142545in}{0.500000in}}%
\pgfpathclose%
\pgfusepath{fill}%
\end{pgfscope}%
\begin{pgfscope}%
\pgfpathrectangle{\pgfqpoint{0.750000in}{0.500000in}}{\pgfqpoint{4.650000in}{3.020000in}}%
\pgfusepath{clip}%
\pgfsetbuttcap%
\pgfsetroundjoin%
\definecolor{currentfill}{rgb}{1.000000,0.000000,0.000000}%
\pgfsetfillcolor{currentfill}%
\pgfsetlinewidth{1.003750pt}%
\definecolor{currentstroke}{rgb}{1.000000,0.000000,0.000000}%
\pgfsetstrokecolor{currentstroke}%
\pgfsetdash{}{0pt}%
\pgfsys@defobject{currentmarker}{\pgfqpoint{-0.041667in}{-0.041667in}}{\pgfqpoint{0.041667in}{0.041667in}}{%
\pgfpathmoveto{\pgfqpoint{0.000000in}{-0.041667in}}%
\pgfpathcurveto{\pgfqpoint{0.011050in}{-0.041667in}}{\pgfqpoint{0.021649in}{-0.037276in}}{\pgfqpoint{0.029463in}{-0.029463in}}%
\pgfpathcurveto{\pgfqpoint{0.037276in}{-0.021649in}}{\pgfqpoint{0.041667in}{-0.011050in}}{\pgfqpoint{0.041667in}{0.000000in}}%
\pgfpathcurveto{\pgfqpoint{0.041667in}{0.011050in}}{\pgfqpoint{0.037276in}{0.021649in}}{\pgfqpoint{0.029463in}{0.029463in}}%
\pgfpathcurveto{\pgfqpoint{0.021649in}{0.037276in}}{\pgfqpoint{0.011050in}{0.041667in}}{\pgfqpoint{0.000000in}{0.041667in}}%
\pgfpathcurveto{\pgfqpoint{-0.011050in}{0.041667in}}{\pgfqpoint{-0.021649in}{0.037276in}}{\pgfqpoint{-0.029463in}{0.029463in}}%
\pgfpathcurveto{\pgfqpoint{-0.037276in}{0.021649in}}{\pgfqpoint{-0.041667in}{0.011050in}}{\pgfqpoint{-0.041667in}{0.000000in}}%
\pgfpathcurveto{\pgfqpoint{-0.041667in}{-0.011050in}}{\pgfqpoint{-0.037276in}{-0.021649in}}{\pgfqpoint{-0.029463in}{-0.029463in}}%
\pgfpathcurveto{\pgfqpoint{-0.021649in}{-0.037276in}}{\pgfqpoint{-0.011050in}{-0.041667in}}{\pgfqpoint{0.000000in}{-0.041667in}}%
\pgfpathlineto{\pgfqpoint{0.000000in}{-0.041667in}}%
\pgfpathclose%
\pgfusepath{stroke,fill}%
}%
\begin{pgfscope}%
\pgfsys@transformshift{4.135110in}{0.500000in}%
\pgfsys@useobject{currentmarker}{}%
\end{pgfscope}%
\end{pgfscope}%
\begin{pgfscope}%
\pgfpathrectangle{\pgfqpoint{0.750000in}{0.500000in}}{\pgfqpoint{4.650000in}{3.020000in}}%
\pgfusepath{clip}%
\pgfsetbuttcap%
\pgfsetroundjoin%
\definecolor{currentfill}{rgb}{0.000000,0.500000,0.000000}%
\pgfsetfillcolor{currentfill}%
\pgfsetlinewidth{1.003750pt}%
\definecolor{currentstroke}{rgb}{0.000000,0.500000,0.000000}%
\pgfsetstrokecolor{currentstroke}%
\pgfsetdash{}{0pt}%
\pgfsys@defobject{currentmarker}{\pgfqpoint{-0.041667in}{-0.041667in}}{\pgfqpoint{0.041667in}{0.041667in}}{%
\pgfpathmoveto{\pgfqpoint{0.000000in}{-0.041667in}}%
\pgfpathcurveto{\pgfqpoint{0.011050in}{-0.041667in}}{\pgfqpoint{0.021649in}{-0.037276in}}{\pgfqpoint{0.029463in}{-0.029463in}}%
\pgfpathcurveto{\pgfqpoint{0.037276in}{-0.021649in}}{\pgfqpoint{0.041667in}{-0.011050in}}{\pgfqpoint{0.041667in}{0.000000in}}%
\pgfpathcurveto{\pgfqpoint{0.041667in}{0.011050in}}{\pgfqpoint{0.037276in}{0.021649in}}{\pgfqpoint{0.029463in}{0.029463in}}%
\pgfpathcurveto{\pgfqpoint{0.021649in}{0.037276in}}{\pgfqpoint{0.011050in}{0.041667in}}{\pgfqpoint{0.000000in}{0.041667in}}%
\pgfpathcurveto{\pgfqpoint{-0.011050in}{0.041667in}}{\pgfqpoint{-0.021649in}{0.037276in}}{\pgfqpoint{-0.029463in}{0.029463in}}%
\pgfpathcurveto{\pgfqpoint{-0.037276in}{0.021649in}}{\pgfqpoint{-0.041667in}{0.011050in}}{\pgfqpoint{-0.041667in}{0.000000in}}%
\pgfpathcurveto{\pgfqpoint{-0.041667in}{-0.011050in}}{\pgfqpoint{-0.037276in}{-0.021649in}}{\pgfqpoint{-0.029463in}{-0.029463in}}%
\pgfpathcurveto{\pgfqpoint{-0.021649in}{-0.037276in}}{\pgfqpoint{-0.011050in}{-0.041667in}}{\pgfqpoint{0.000000in}{-0.041667in}}%
\pgfpathlineto{\pgfqpoint{0.000000in}{-0.041667in}}%
\pgfpathclose%
\pgfusepath{stroke,fill}%
}%
\begin{pgfscope}%
\pgfsys@transformshift{5.188636in}{0.500000in}%
\pgfsys@useobject{currentmarker}{}%
\end{pgfscope}%
\end{pgfscope}%
\begin{pgfscope}%
\pgfsetbuttcap%
\pgfsetroundjoin%
\definecolor{currentfill}{rgb}{0.000000,0.000000,0.000000}%
\pgfsetfillcolor{currentfill}%
\pgfsetlinewidth{0.803000pt}%
\definecolor{currentstroke}{rgb}{0.000000,0.000000,0.000000}%
\pgfsetstrokecolor{currentstroke}%
\pgfsetdash{}{0pt}%
\pgfsys@defobject{currentmarker}{\pgfqpoint{0.000000in}{-0.048611in}}{\pgfqpoint{0.000000in}{0.000000in}}{%
\pgfpathmoveto{\pgfqpoint{0.000000in}{0.000000in}}%
\pgfpathlineto{\pgfqpoint{0.000000in}{-0.048611in}}%
\pgfusepath{stroke,fill}%
}%
\begin{pgfscope}%
\pgfsys@transformshift{0.974533in}{0.500000in}%
\pgfsys@useobject{currentmarker}{}%
\end{pgfscope}%
\end{pgfscope}%
\begin{pgfscope}%
\definecolor{textcolor}{rgb}{0.000000,0.000000,0.000000}%
\pgfsetstrokecolor{textcolor}%
\pgfsetfillcolor{textcolor}%
\pgftext[x=0.974533in,y=0.402778in,,top]{\color{textcolor}\rmfamily\fontsize{13.000000}{15.600000}\selectfont \(\displaystyle {0}\)}%
\end{pgfscope}%
\begin{pgfscope}%
\pgfsetbuttcap%
\pgfsetroundjoin%
\definecolor{currentfill}{rgb}{0.000000,0.000000,0.000000}%
\pgfsetfillcolor{currentfill}%
\pgfsetlinewidth{0.803000pt}%
\definecolor{currentstroke}{rgb}{0.000000,0.000000,0.000000}%
\pgfsetstrokecolor{currentstroke}%
\pgfsetdash{}{0pt}%
\pgfsys@defobject{currentmarker}{\pgfqpoint{0.000000in}{-0.048611in}}{\pgfqpoint{0.000000in}{0.000000in}}{%
\pgfpathmoveto{\pgfqpoint{0.000000in}{0.000000in}}%
\pgfpathlineto{\pgfqpoint{0.000000in}{-0.048611in}}%
\pgfusepath{stroke,fill}%
}%
\begin{pgfscope}%
\pgfsys@transformshift{1.632986in}{0.500000in}%
\pgfsys@useobject{currentmarker}{}%
\end{pgfscope}%
\end{pgfscope}%
\begin{pgfscope}%
\definecolor{textcolor}{rgb}{0.000000,0.000000,0.000000}%
\pgfsetstrokecolor{textcolor}%
\pgfsetfillcolor{textcolor}%
\pgftext[x=1.632986in,y=0.402778in,,top]{\color{textcolor}\rmfamily\fontsize{13.000000}{15.600000}\selectfont \(\displaystyle {20}\)}%
\end{pgfscope}%
\begin{pgfscope}%
\pgfsetbuttcap%
\pgfsetroundjoin%
\definecolor{currentfill}{rgb}{0.000000,0.000000,0.000000}%
\pgfsetfillcolor{currentfill}%
\pgfsetlinewidth{0.803000pt}%
\definecolor{currentstroke}{rgb}{0.000000,0.000000,0.000000}%
\pgfsetstrokecolor{currentstroke}%
\pgfsetdash{}{0pt}%
\pgfsys@defobject{currentmarker}{\pgfqpoint{0.000000in}{-0.048611in}}{\pgfqpoint{0.000000in}{0.000000in}}{%
\pgfpathmoveto{\pgfqpoint{0.000000in}{0.000000in}}%
\pgfpathlineto{\pgfqpoint{0.000000in}{-0.048611in}}%
\pgfusepath{stroke,fill}%
}%
\begin{pgfscope}%
\pgfsys@transformshift{2.291440in}{0.500000in}%
\pgfsys@useobject{currentmarker}{}%
\end{pgfscope}%
\end{pgfscope}%
\begin{pgfscope}%
\definecolor{textcolor}{rgb}{0.000000,0.000000,0.000000}%
\pgfsetstrokecolor{textcolor}%
\pgfsetfillcolor{textcolor}%
\pgftext[x=2.291440in,y=0.402778in,,top]{\color{textcolor}\rmfamily\fontsize{13.000000}{15.600000}\selectfont \(\displaystyle {40}\)}%
\end{pgfscope}%
\begin{pgfscope}%
\pgfsetbuttcap%
\pgfsetroundjoin%
\definecolor{currentfill}{rgb}{0.000000,0.000000,0.000000}%
\pgfsetfillcolor{currentfill}%
\pgfsetlinewidth{0.803000pt}%
\definecolor{currentstroke}{rgb}{0.000000,0.000000,0.000000}%
\pgfsetstrokecolor{currentstroke}%
\pgfsetdash{}{0pt}%
\pgfsys@defobject{currentmarker}{\pgfqpoint{0.000000in}{-0.048611in}}{\pgfqpoint{0.000000in}{0.000000in}}{%
\pgfpathmoveto{\pgfqpoint{0.000000in}{0.000000in}}%
\pgfpathlineto{\pgfqpoint{0.000000in}{-0.048611in}}%
\pgfusepath{stroke,fill}%
}%
\begin{pgfscope}%
\pgfsys@transformshift{2.949894in}{0.500000in}%
\pgfsys@useobject{currentmarker}{}%
\end{pgfscope}%
\end{pgfscope}%
\begin{pgfscope}%
\definecolor{textcolor}{rgb}{0.000000,0.000000,0.000000}%
\pgfsetstrokecolor{textcolor}%
\pgfsetfillcolor{textcolor}%
\pgftext[x=2.949894in,y=0.402778in,,top]{\color{textcolor}\rmfamily\fontsize{13.000000}{15.600000}\selectfont \(\displaystyle {60}\)}%
\end{pgfscope}%
\begin{pgfscope}%
\pgfsetbuttcap%
\pgfsetroundjoin%
\definecolor{currentfill}{rgb}{0.000000,0.000000,0.000000}%
\pgfsetfillcolor{currentfill}%
\pgfsetlinewidth{0.803000pt}%
\definecolor{currentstroke}{rgb}{0.000000,0.000000,0.000000}%
\pgfsetstrokecolor{currentstroke}%
\pgfsetdash{}{0pt}%
\pgfsys@defobject{currentmarker}{\pgfqpoint{0.000000in}{-0.048611in}}{\pgfqpoint{0.000000in}{0.000000in}}{%
\pgfpathmoveto{\pgfqpoint{0.000000in}{0.000000in}}%
\pgfpathlineto{\pgfqpoint{0.000000in}{-0.048611in}}%
\pgfusepath{stroke,fill}%
}%
\begin{pgfscope}%
\pgfsys@transformshift{3.608347in}{0.500000in}%
\pgfsys@useobject{currentmarker}{}%
\end{pgfscope}%
\end{pgfscope}%
\begin{pgfscope}%
\definecolor{textcolor}{rgb}{0.000000,0.000000,0.000000}%
\pgfsetstrokecolor{textcolor}%
\pgfsetfillcolor{textcolor}%
\pgftext[x=3.608347in,y=0.402778in,,top]{\color{textcolor}\rmfamily\fontsize{13.000000}{15.600000}\selectfont \(\displaystyle {80}\)}%
\end{pgfscope}%
\begin{pgfscope}%
\pgfsetbuttcap%
\pgfsetroundjoin%
\definecolor{currentfill}{rgb}{0.000000,0.000000,0.000000}%
\pgfsetfillcolor{currentfill}%
\pgfsetlinewidth{0.803000pt}%
\definecolor{currentstroke}{rgb}{0.000000,0.000000,0.000000}%
\pgfsetstrokecolor{currentstroke}%
\pgfsetdash{}{0pt}%
\pgfsys@defobject{currentmarker}{\pgfqpoint{0.000000in}{-0.048611in}}{\pgfqpoint{0.000000in}{0.000000in}}{%
\pgfpathmoveto{\pgfqpoint{0.000000in}{0.000000in}}%
\pgfpathlineto{\pgfqpoint{0.000000in}{-0.048611in}}%
\pgfusepath{stroke,fill}%
}%
\begin{pgfscope}%
\pgfsys@transformshift{4.266801in}{0.500000in}%
\pgfsys@useobject{currentmarker}{}%
\end{pgfscope}%
\end{pgfscope}%
\begin{pgfscope}%
\definecolor{textcolor}{rgb}{0.000000,0.000000,0.000000}%
\pgfsetstrokecolor{textcolor}%
\pgfsetfillcolor{textcolor}%
\pgftext[x=4.266801in,y=0.402778in,,top]{\color{textcolor}\rmfamily\fontsize{13.000000}{15.600000}\selectfont \(\displaystyle {100}\)}%
\end{pgfscope}%
\begin{pgfscope}%
\pgfsetbuttcap%
\pgfsetroundjoin%
\definecolor{currentfill}{rgb}{0.000000,0.000000,0.000000}%
\pgfsetfillcolor{currentfill}%
\pgfsetlinewidth{0.803000pt}%
\definecolor{currentstroke}{rgb}{0.000000,0.000000,0.000000}%
\pgfsetstrokecolor{currentstroke}%
\pgfsetdash{}{0pt}%
\pgfsys@defobject{currentmarker}{\pgfqpoint{0.000000in}{-0.048611in}}{\pgfqpoint{0.000000in}{0.000000in}}{%
\pgfpathmoveto{\pgfqpoint{0.000000in}{0.000000in}}%
\pgfpathlineto{\pgfqpoint{0.000000in}{-0.048611in}}%
\pgfusepath{stroke,fill}%
}%
\begin{pgfscope}%
\pgfsys@transformshift{4.925255in}{0.500000in}%
\pgfsys@useobject{currentmarker}{}%
\end{pgfscope}%
\end{pgfscope}%
\begin{pgfscope}%
\definecolor{textcolor}{rgb}{0.000000,0.000000,0.000000}%
\pgfsetstrokecolor{textcolor}%
\pgfsetfillcolor{textcolor}%
\pgftext[x=4.925255in,y=0.402778in,,top]{\color{textcolor}\rmfamily\fontsize{13.000000}{15.600000}\selectfont \(\displaystyle {120}\)}%
\end{pgfscope}%
\begin{pgfscope}%
\definecolor{textcolor}{rgb}{0.000000,0.000000,0.000000}%
\pgfsetstrokecolor{textcolor}%
\pgfsetfillcolor{textcolor}%
\pgftext[x=3.075000in,y=0.199075in,,top]{\color{textcolor}\rmfamily\fontsize{13.000000}{15.600000}\selectfont Number of correct bits}%
\end{pgfscope}%
\begin{pgfscope}%
\pgfsetbuttcap%
\pgfsetroundjoin%
\definecolor{currentfill}{rgb}{0.000000,0.000000,0.000000}%
\pgfsetfillcolor{currentfill}%
\pgfsetlinewidth{0.803000pt}%
\definecolor{currentstroke}{rgb}{0.000000,0.000000,0.000000}%
\pgfsetstrokecolor{currentstroke}%
\pgfsetdash{}{0pt}%
\pgfsys@defobject{currentmarker}{\pgfqpoint{-0.048611in}{0.000000in}}{\pgfqpoint{-0.000000in}{0.000000in}}{%
\pgfpathmoveto{\pgfqpoint{-0.000000in}{0.000000in}}%
\pgfpathlineto{\pgfqpoint{-0.048611in}{0.000000in}}%
\pgfusepath{stroke,fill}%
}%
\begin{pgfscope}%
\pgfsys@transformshift{0.750000in}{0.500000in}%
\pgfsys@useobject{currentmarker}{}%
\end{pgfscope}%
\end{pgfscope}%
\begin{pgfscope}%
\definecolor{textcolor}{rgb}{0.000000,0.000000,0.000000}%
\pgfsetstrokecolor{textcolor}%
\pgfsetfillcolor{textcolor}%
\pgftext[x=0.362657in, y=0.442130in, left, base]{\color{textcolor}\rmfamily\fontsize{13.000000}{15.600000}\selectfont \(\displaystyle {0.00}\)}%
\end{pgfscope}%
\begin{pgfscope}%
\pgfsetbuttcap%
\pgfsetroundjoin%
\definecolor{currentfill}{rgb}{0.000000,0.000000,0.000000}%
\pgfsetfillcolor{currentfill}%
\pgfsetlinewidth{0.803000pt}%
\definecolor{currentstroke}{rgb}{0.000000,0.000000,0.000000}%
\pgfsetstrokecolor{currentstroke}%
\pgfsetdash{}{0pt}%
\pgfsys@defobject{currentmarker}{\pgfqpoint{-0.048611in}{0.000000in}}{\pgfqpoint{-0.000000in}{0.000000in}}{%
\pgfpathmoveto{\pgfqpoint{-0.000000in}{0.000000in}}%
\pgfpathlineto{\pgfqpoint{-0.048611in}{0.000000in}}%
\pgfusepath{stroke,fill}%
}%
\begin{pgfscope}%
\pgfsys@transformshift{0.750000in}{0.908631in}%
\pgfsys@useobject{currentmarker}{}%
\end{pgfscope}%
\end{pgfscope}%
\begin{pgfscope}%
\definecolor{textcolor}{rgb}{0.000000,0.000000,0.000000}%
\pgfsetstrokecolor{textcolor}%
\pgfsetfillcolor{textcolor}%
\pgftext[x=0.362657in, y=0.850760in, left, base]{\color{textcolor}\rmfamily\fontsize{13.000000}{15.600000}\selectfont \(\displaystyle {0.01}\)}%
\end{pgfscope}%
\begin{pgfscope}%
\pgfsetbuttcap%
\pgfsetroundjoin%
\definecolor{currentfill}{rgb}{0.000000,0.000000,0.000000}%
\pgfsetfillcolor{currentfill}%
\pgfsetlinewidth{0.803000pt}%
\definecolor{currentstroke}{rgb}{0.000000,0.000000,0.000000}%
\pgfsetstrokecolor{currentstroke}%
\pgfsetdash{}{0pt}%
\pgfsys@defobject{currentmarker}{\pgfqpoint{-0.048611in}{0.000000in}}{\pgfqpoint{-0.000000in}{0.000000in}}{%
\pgfpathmoveto{\pgfqpoint{-0.000000in}{0.000000in}}%
\pgfpathlineto{\pgfqpoint{-0.048611in}{0.000000in}}%
\pgfusepath{stroke,fill}%
}%
\begin{pgfscope}%
\pgfsys@transformshift{0.750000in}{1.317261in}%
\pgfsys@useobject{currentmarker}{}%
\end{pgfscope}%
\end{pgfscope}%
\begin{pgfscope}%
\definecolor{textcolor}{rgb}{0.000000,0.000000,0.000000}%
\pgfsetstrokecolor{textcolor}%
\pgfsetfillcolor{textcolor}%
\pgftext[x=0.362657in, y=1.259391in, left, base]{\color{textcolor}\rmfamily\fontsize{13.000000}{15.600000}\selectfont \(\displaystyle {0.02}\)}%
\end{pgfscope}%
\begin{pgfscope}%
\pgfsetbuttcap%
\pgfsetroundjoin%
\definecolor{currentfill}{rgb}{0.000000,0.000000,0.000000}%
\pgfsetfillcolor{currentfill}%
\pgfsetlinewidth{0.803000pt}%
\definecolor{currentstroke}{rgb}{0.000000,0.000000,0.000000}%
\pgfsetstrokecolor{currentstroke}%
\pgfsetdash{}{0pt}%
\pgfsys@defobject{currentmarker}{\pgfqpoint{-0.048611in}{0.000000in}}{\pgfqpoint{-0.000000in}{0.000000in}}{%
\pgfpathmoveto{\pgfqpoint{-0.000000in}{0.000000in}}%
\pgfpathlineto{\pgfqpoint{-0.048611in}{0.000000in}}%
\pgfusepath{stroke,fill}%
}%
\begin{pgfscope}%
\pgfsys@transformshift{0.750000in}{1.725892in}%
\pgfsys@useobject{currentmarker}{}%
\end{pgfscope}%
\end{pgfscope}%
\begin{pgfscope}%
\definecolor{textcolor}{rgb}{0.000000,0.000000,0.000000}%
\pgfsetstrokecolor{textcolor}%
\pgfsetfillcolor{textcolor}%
\pgftext[x=0.362657in, y=1.668021in, left, base]{\color{textcolor}\rmfamily\fontsize{13.000000}{15.600000}\selectfont \(\displaystyle {0.03}\)}%
\end{pgfscope}%
\begin{pgfscope}%
\pgfsetbuttcap%
\pgfsetroundjoin%
\definecolor{currentfill}{rgb}{0.000000,0.000000,0.000000}%
\pgfsetfillcolor{currentfill}%
\pgfsetlinewidth{0.803000pt}%
\definecolor{currentstroke}{rgb}{0.000000,0.000000,0.000000}%
\pgfsetstrokecolor{currentstroke}%
\pgfsetdash{}{0pt}%
\pgfsys@defobject{currentmarker}{\pgfqpoint{-0.048611in}{0.000000in}}{\pgfqpoint{-0.000000in}{0.000000in}}{%
\pgfpathmoveto{\pgfqpoint{-0.000000in}{0.000000in}}%
\pgfpathlineto{\pgfqpoint{-0.048611in}{0.000000in}}%
\pgfusepath{stroke,fill}%
}%
\begin{pgfscope}%
\pgfsys@transformshift{0.750000in}{2.134522in}%
\pgfsys@useobject{currentmarker}{}%
\end{pgfscope}%
\end{pgfscope}%
\begin{pgfscope}%
\definecolor{textcolor}{rgb}{0.000000,0.000000,0.000000}%
\pgfsetstrokecolor{textcolor}%
\pgfsetfillcolor{textcolor}%
\pgftext[x=0.362657in, y=2.076652in, left, base]{\color{textcolor}\rmfamily\fontsize{13.000000}{15.600000}\selectfont \(\displaystyle {0.04}\)}%
\end{pgfscope}%
\begin{pgfscope}%
\pgfsetbuttcap%
\pgfsetroundjoin%
\definecolor{currentfill}{rgb}{0.000000,0.000000,0.000000}%
\pgfsetfillcolor{currentfill}%
\pgfsetlinewidth{0.803000pt}%
\definecolor{currentstroke}{rgb}{0.000000,0.000000,0.000000}%
\pgfsetstrokecolor{currentstroke}%
\pgfsetdash{}{0pt}%
\pgfsys@defobject{currentmarker}{\pgfqpoint{-0.048611in}{0.000000in}}{\pgfqpoint{-0.000000in}{0.000000in}}{%
\pgfpathmoveto{\pgfqpoint{-0.000000in}{0.000000in}}%
\pgfpathlineto{\pgfqpoint{-0.048611in}{0.000000in}}%
\pgfusepath{stroke,fill}%
}%
\begin{pgfscope}%
\pgfsys@transformshift{0.750000in}{2.543153in}%
\pgfsys@useobject{currentmarker}{}%
\end{pgfscope}%
\end{pgfscope}%
\begin{pgfscope}%
\definecolor{textcolor}{rgb}{0.000000,0.000000,0.000000}%
\pgfsetstrokecolor{textcolor}%
\pgfsetfillcolor{textcolor}%
\pgftext[x=0.362657in, y=2.485282in, left, base]{\color{textcolor}\rmfamily\fontsize{13.000000}{15.600000}\selectfont \(\displaystyle {0.05}\)}%
\end{pgfscope}%
\begin{pgfscope}%
\pgfsetbuttcap%
\pgfsetroundjoin%
\definecolor{currentfill}{rgb}{0.000000,0.000000,0.000000}%
\pgfsetfillcolor{currentfill}%
\pgfsetlinewidth{0.803000pt}%
\definecolor{currentstroke}{rgb}{0.000000,0.000000,0.000000}%
\pgfsetstrokecolor{currentstroke}%
\pgfsetdash{}{0pt}%
\pgfsys@defobject{currentmarker}{\pgfqpoint{-0.048611in}{0.000000in}}{\pgfqpoint{-0.000000in}{0.000000in}}{%
\pgfpathmoveto{\pgfqpoint{-0.000000in}{0.000000in}}%
\pgfpathlineto{\pgfqpoint{-0.048611in}{0.000000in}}%
\pgfusepath{stroke,fill}%
}%
\begin{pgfscope}%
\pgfsys@transformshift{0.750000in}{2.951783in}%
\pgfsys@useobject{currentmarker}{}%
\end{pgfscope}%
\end{pgfscope}%
\begin{pgfscope}%
\definecolor{textcolor}{rgb}{0.000000,0.000000,0.000000}%
\pgfsetstrokecolor{textcolor}%
\pgfsetfillcolor{textcolor}%
\pgftext[x=0.362657in, y=2.893913in, left, base]{\color{textcolor}\rmfamily\fontsize{13.000000}{15.600000}\selectfont \(\displaystyle {0.06}\)}%
\end{pgfscope}%
\begin{pgfscope}%
\pgfsetbuttcap%
\pgfsetroundjoin%
\definecolor{currentfill}{rgb}{0.000000,0.000000,0.000000}%
\pgfsetfillcolor{currentfill}%
\pgfsetlinewidth{0.803000pt}%
\definecolor{currentstroke}{rgb}{0.000000,0.000000,0.000000}%
\pgfsetstrokecolor{currentstroke}%
\pgfsetdash{}{0pt}%
\pgfsys@defobject{currentmarker}{\pgfqpoint{-0.048611in}{0.000000in}}{\pgfqpoint{-0.000000in}{0.000000in}}{%
\pgfpathmoveto{\pgfqpoint{-0.000000in}{0.000000in}}%
\pgfpathlineto{\pgfqpoint{-0.048611in}{0.000000in}}%
\pgfusepath{stroke,fill}%
}%
\begin{pgfscope}%
\pgfsys@transformshift{0.750000in}{3.360414in}%
\pgfsys@useobject{currentmarker}{}%
\end{pgfscope}%
\end{pgfscope}%
\begin{pgfscope}%
\definecolor{textcolor}{rgb}{0.000000,0.000000,0.000000}%
\pgfsetstrokecolor{textcolor}%
\pgfsetfillcolor{textcolor}%
\pgftext[x=0.362657in, y=3.302543in, left, base]{\color{textcolor}\rmfamily\fontsize{13.000000}{15.600000}\selectfont \(\displaystyle {0.07}\)}%
\end{pgfscope}%
\begin{pgfscope}%
\definecolor{textcolor}{rgb}{0.000000,0.000000,0.000000}%
\pgfsetstrokecolor{textcolor}%
\pgfsetfillcolor{textcolor}%
\pgftext[x=0.307102in,y=2.010000in,,bottom,rotate=90.000000]{\color{textcolor}\rmfamily\fontsize{13.000000}{15.600000}\selectfont Probability}%
\end{pgfscope}%
\begin{pgfscope}%
\pgfsetrectcap%
\pgfsetmiterjoin%
\pgfsetlinewidth{0.803000pt}%
\definecolor{currentstroke}{rgb}{0.000000,0.000000,0.000000}%
\pgfsetstrokecolor{currentstroke}%
\pgfsetdash{}{0pt}%
\pgfpathmoveto{\pgfqpoint{0.750000in}{0.500000in}}%
\pgfpathlineto{\pgfqpoint{0.750000in}{3.520000in}}%
\pgfusepath{stroke}%
\end{pgfscope}%
\begin{pgfscope}%
\pgfsetrectcap%
\pgfsetmiterjoin%
\pgfsetlinewidth{0.803000pt}%
\definecolor{currentstroke}{rgb}{0.000000,0.000000,0.000000}%
\pgfsetstrokecolor{currentstroke}%
\pgfsetdash{}{0pt}%
\pgfpathmoveto{\pgfqpoint{5.400000in}{0.500000in}}%
\pgfpathlineto{\pgfqpoint{5.400000in}{3.520000in}}%
\pgfusepath{stroke}%
\end{pgfscope}%
\begin{pgfscope}%
\pgfsetrectcap%
\pgfsetmiterjoin%
\pgfsetlinewidth{0.803000pt}%
\definecolor{currentstroke}{rgb}{0.000000,0.000000,0.000000}%
\pgfsetstrokecolor{currentstroke}%
\pgfsetdash{}{0pt}%
\pgfpathmoveto{\pgfqpoint{0.750000in}{0.500000in}}%
\pgfpathlineto{\pgfqpoint{5.400000in}{0.500000in}}%
\pgfusepath{stroke}%
\end{pgfscope}%
\begin{pgfscope}%
\pgfsetrectcap%
\pgfsetmiterjoin%
\pgfsetlinewidth{0.803000pt}%
\definecolor{currentstroke}{rgb}{0.000000,0.000000,0.000000}%
\pgfsetstrokecolor{currentstroke}%
\pgfsetdash{}{0pt}%
\pgfpathmoveto{\pgfqpoint{0.750000in}{3.520000in}}%
\pgfpathlineto{\pgfqpoint{5.400000in}{3.520000in}}%
\pgfusepath{stroke}%
\end{pgfscope}%
\begin{pgfscope}%
\definecolor{textcolor}{rgb}{0.000000,0.000000,0.000000}%
\pgfsetstrokecolor{textcolor}%
\pgfsetfillcolor{textcolor}%
\pgftext[x=3.075000in,y=3.603333in,,base]{\color{textcolor}\rmfamily\fontsize{13.000000}{15.600000}\selectfont PDF for random bits \(\displaystyle 2x\)}%
\end{pgfscope}%
\begin{pgfscope}%
\pgfsetbuttcap%
\pgfsetmiterjoin%
\definecolor{currentfill}{rgb}{1.000000,1.000000,1.000000}%
\pgfsetfillcolor{currentfill}%
\pgfsetfillopacity{0.800000}%
\pgfsetlinewidth{1.003750pt}%
\definecolor{currentstroke}{rgb}{0.800000,0.800000,0.800000}%
\pgfsetstrokecolor{currentstroke}%
\pgfsetstrokeopacity{0.800000}%
\pgfsetdash{}{0pt}%
\pgfpathmoveto{\pgfqpoint{3.210355in}{2.628334in}}%
\pgfpathlineto{\pgfqpoint{5.273611in}{2.628334in}}%
\pgfpathquadraticcurveto{\pgfqpoint{5.309722in}{2.628334in}}{\pgfqpoint{5.309722in}{2.664445in}}%
\pgfpathlineto{\pgfqpoint{5.309722in}{3.393611in}}%
\pgfpathquadraticcurveto{\pgfqpoint{5.309722in}{3.429722in}}{\pgfqpoint{5.273611in}{3.429722in}}%
\pgfpathlineto{\pgfqpoint{3.210355in}{3.429722in}}%
\pgfpathquadraticcurveto{\pgfqpoint{3.174244in}{3.429722in}}{\pgfqpoint{3.174244in}{3.393611in}}%
\pgfpathlineto{\pgfqpoint{3.174244in}{2.664445in}}%
\pgfpathquadraticcurveto{\pgfqpoint{3.174244in}{2.628334in}}{\pgfqpoint{3.210355in}{2.628334in}}%
\pgfpathlineto{\pgfqpoint{3.210355in}{2.628334in}}%
\pgfpathclose%
\pgfusepath{stroke,fill}%
\end{pgfscope}%
\begin{pgfscope}%
\pgfsetbuttcap%
\pgfsetroundjoin%
\definecolor{currentfill}{rgb}{1.000000,0.000000,0.000000}%
\pgfsetfillcolor{currentfill}%
\pgfsetlinewidth{1.003750pt}%
\definecolor{currentstroke}{rgb}{1.000000,0.000000,0.000000}%
\pgfsetstrokecolor{currentstroke}%
\pgfsetdash{}{0pt}%
\pgfsys@defobject{currentmarker}{\pgfqpoint{-0.041667in}{-0.041667in}}{\pgfqpoint{0.041667in}{0.041667in}}{%
\pgfpathmoveto{\pgfqpoint{0.000000in}{-0.041667in}}%
\pgfpathcurveto{\pgfqpoint{0.011050in}{-0.041667in}}{\pgfqpoint{0.021649in}{-0.037276in}}{\pgfqpoint{0.029463in}{-0.029463in}}%
\pgfpathcurveto{\pgfqpoint{0.037276in}{-0.021649in}}{\pgfqpoint{0.041667in}{-0.011050in}}{\pgfqpoint{0.041667in}{0.000000in}}%
\pgfpathcurveto{\pgfqpoint{0.041667in}{0.011050in}}{\pgfqpoint{0.037276in}{0.021649in}}{\pgfqpoint{0.029463in}{0.029463in}}%
\pgfpathcurveto{\pgfqpoint{0.021649in}{0.037276in}}{\pgfqpoint{0.011050in}{0.041667in}}{\pgfqpoint{0.000000in}{0.041667in}}%
\pgfpathcurveto{\pgfqpoint{-0.011050in}{0.041667in}}{\pgfqpoint{-0.021649in}{0.037276in}}{\pgfqpoint{-0.029463in}{0.029463in}}%
\pgfpathcurveto{\pgfqpoint{-0.037276in}{0.021649in}}{\pgfqpoint{-0.041667in}{0.011050in}}{\pgfqpoint{-0.041667in}{0.000000in}}%
\pgfpathcurveto{\pgfqpoint{-0.041667in}{-0.011050in}}{\pgfqpoint{-0.037276in}{-0.021649in}}{\pgfqpoint{-0.029463in}{-0.029463in}}%
\pgfpathcurveto{\pgfqpoint{-0.021649in}{-0.037276in}}{\pgfqpoint{-0.011050in}{-0.041667in}}{\pgfqpoint{0.000000in}{-0.041667in}}%
\pgfpathlineto{\pgfqpoint{0.000000in}{-0.041667in}}%
\pgfpathclose%
\pgfusepath{stroke,fill}%
}%
\begin{pgfscope}%
\pgfsys@transformshift{3.427022in}{3.278507in}%
\pgfsys@useobject{currentmarker}{}%
\end{pgfscope}%
\end{pgfscope}%
\begin{pgfscope}%
\definecolor{textcolor}{rgb}{0.000000,0.000000,0.000000}%
\pgfsetstrokecolor{textcolor}%
\pgfsetfillcolor{textcolor}%
\pgftext[x=3.752022in,y=3.231111in,left,base]{\color{textcolor}\rmfamily\fontsize{13.000000}{15.600000}\selectfont SNN}%
\end{pgfscope}%
\begin{pgfscope}%
\pgfsetbuttcap%
\pgfsetroundjoin%
\definecolor{currentfill}{rgb}{0.000000,0.500000,0.000000}%
\pgfsetfillcolor{currentfill}%
\pgfsetlinewidth{1.003750pt}%
\definecolor{currentstroke}{rgb}{0.000000,0.500000,0.000000}%
\pgfsetstrokecolor{currentstroke}%
\pgfsetdash{}{0pt}%
\pgfsys@defobject{currentmarker}{\pgfqpoint{-0.041667in}{-0.041667in}}{\pgfqpoint{0.041667in}{0.041667in}}{%
\pgfpathmoveto{\pgfqpoint{0.000000in}{-0.041667in}}%
\pgfpathcurveto{\pgfqpoint{0.011050in}{-0.041667in}}{\pgfqpoint{0.021649in}{-0.037276in}}{\pgfqpoint{0.029463in}{-0.029463in}}%
\pgfpathcurveto{\pgfqpoint{0.037276in}{-0.021649in}}{\pgfqpoint{0.041667in}{-0.011050in}}{\pgfqpoint{0.041667in}{0.000000in}}%
\pgfpathcurveto{\pgfqpoint{0.041667in}{0.011050in}}{\pgfqpoint{0.037276in}{0.021649in}}{\pgfqpoint{0.029463in}{0.029463in}}%
\pgfpathcurveto{\pgfqpoint{0.021649in}{0.037276in}}{\pgfqpoint{0.011050in}{0.041667in}}{\pgfqpoint{0.000000in}{0.041667in}}%
\pgfpathcurveto{\pgfqpoint{-0.011050in}{0.041667in}}{\pgfqpoint{-0.021649in}{0.037276in}}{\pgfqpoint{-0.029463in}{0.029463in}}%
\pgfpathcurveto{\pgfqpoint{-0.037276in}{0.021649in}}{\pgfqpoint{-0.041667in}{0.011050in}}{\pgfqpoint{-0.041667in}{0.000000in}}%
\pgfpathcurveto{\pgfqpoint{-0.041667in}{-0.011050in}}{\pgfqpoint{-0.037276in}{-0.021649in}}{\pgfqpoint{-0.029463in}{-0.029463in}}%
\pgfpathcurveto{\pgfqpoint{-0.021649in}{-0.037276in}}{\pgfqpoint{-0.011050in}{-0.041667in}}{\pgfqpoint{0.000000in}{-0.041667in}}%
\pgfpathlineto{\pgfqpoint{0.000000in}{-0.041667in}}%
\pgfpathclose%
\pgfusepath{stroke,fill}%
}%
\begin{pgfscope}%
\pgfsys@transformshift{3.427022in}{3.029433in}%
\pgfsys@useobject{currentmarker}{}%
\end{pgfscope}%
\end{pgfscope}%
\begin{pgfscope}%
\definecolor{textcolor}{rgb}{0.000000,0.000000,0.000000}%
\pgfsetstrokecolor{textcolor}%
\pgfsetfillcolor{textcolor}%
\pgftext[x=3.752022in,y=2.982037in,left,base]{\color{textcolor}\rmfamily\fontsize{13.000000}{15.600000}\selectfont NN}%
\end{pgfscope}%
\begin{pgfscope}%
\pgfsetbuttcap%
\pgfsetmiterjoin%
\definecolor{currentfill}{rgb}{0.121569,0.466667,0.705882}%
\pgfsetfillcolor{currentfill}%
\pgfsetlinewidth{0.000000pt}%
\definecolor{currentstroke}{rgb}{0.000000,0.000000,0.000000}%
\pgfsetstrokecolor{currentstroke}%
\pgfsetstrokeopacity{0.000000}%
\pgfsetdash{}{0pt}%
\pgfpathmoveto{\pgfqpoint{3.246466in}{2.732964in}}%
\pgfpathlineto{\pgfqpoint{3.607577in}{2.732964in}}%
\pgfpathlineto{\pgfqpoint{3.607577in}{2.859352in}}%
\pgfpathlineto{\pgfqpoint{3.246466in}{2.859352in}}%
\pgfpathlineto{\pgfqpoint{3.246466in}{2.732964in}}%
\pgfpathclose%
\pgfusepath{fill}%
\end{pgfscope}%
\begin{pgfscope}%
\definecolor{textcolor}{rgb}{0.000000,0.000000,0.000000}%
\pgfsetstrokecolor{textcolor}%
\pgfsetfillcolor{textcolor}%
\pgftext[x=3.752022in,y=2.732964in,left,base]{\color{textcolor}\rmfamily\fontsize{13.000000}{15.600000}\selectfont Binomial distributon}%
\end{pgfscope}%
\end{pgfpicture}%
\makeatother%
\endgroup%

    \caption{Caption}
    \label{fig:my_label}
\end{figure}

\begin{figure}
%% Creator: Matplotlib, PGF backend
%%
%% To include the figure in your LaTeX document, write
%%   \input{<filename>.pgf}
%%
%% Make sure the required packages are loaded in your preamble
%%   \usepackage{pgf}
%%
%% Also ensure that all the required font packages are loaded; for instance,
%% the lmodern package is sometimes necessary when using math font.
%%   \usepackage{lmodern}
%%
%% Figures using additional raster images can only be included by \input if
%% they are in the same directory as the main LaTeX file. For loading figures
%% from other directories you can use the `import` package
%%   \usepackage{import}
%%
%% and then include the figures with
%%   \import{<path to file>}{<filename>.pgf}
%%
%% Matplotlib used the following preamble
%%
\begingroup%
\makeatletter%
\begin{pgfpicture}%
\pgfpathrectangle{\pgfpointorigin}{\pgfqpoint{6.000000in}{4.000000in}}%
\pgfusepath{use as bounding box, clip}%
\begin{pgfscope}%
\pgfsetbuttcap%
\pgfsetmiterjoin%
\pgfsetlinewidth{0.000000pt}%
\definecolor{currentstroke}{rgb}{1.000000,1.000000,1.000000}%
\pgfsetstrokecolor{currentstroke}%
\pgfsetstrokeopacity{0.000000}%
\pgfsetdash{}{0pt}%
\pgfpathmoveto{\pgfqpoint{0.000000in}{0.000000in}}%
\pgfpathlineto{\pgfqpoint{6.000000in}{0.000000in}}%
\pgfpathlineto{\pgfqpoint{6.000000in}{4.000000in}}%
\pgfpathlineto{\pgfqpoint{0.000000in}{4.000000in}}%
\pgfpathlineto{\pgfqpoint{0.000000in}{0.000000in}}%
\pgfpathclose%
\pgfusepath{}%
\end{pgfscope}%
\begin{pgfscope}%
\pgfsetbuttcap%
\pgfsetmiterjoin%
\definecolor{currentfill}{rgb}{1.000000,1.000000,1.000000}%
\pgfsetfillcolor{currentfill}%
\pgfsetlinewidth{0.000000pt}%
\definecolor{currentstroke}{rgb}{0.000000,0.000000,0.000000}%
\pgfsetstrokecolor{currentstroke}%
\pgfsetstrokeopacity{0.000000}%
\pgfsetdash{}{0pt}%
\pgfpathmoveto{\pgfqpoint{0.750000in}{0.500000in}}%
\pgfpathlineto{\pgfqpoint{5.400000in}{0.500000in}}%
\pgfpathlineto{\pgfqpoint{5.400000in}{3.520000in}}%
\pgfpathlineto{\pgfqpoint{0.750000in}{3.520000in}}%
\pgfpathlineto{\pgfqpoint{0.750000in}{0.500000in}}%
\pgfpathclose%
\pgfusepath{fill}%
\end{pgfscope}%
\begin{pgfscope}%
\pgfpathrectangle{\pgfqpoint{0.750000in}{0.500000in}}{\pgfqpoint{4.650000in}{3.020000in}}%
\pgfusepath{clip}%
\pgfsetbuttcap%
\pgfsetmiterjoin%
\definecolor{currentfill}{rgb}{1.000000,0.000000,0.000000}%
\pgfsetfillcolor{currentfill}%
\pgfsetlinewidth{0.000000pt}%
\definecolor{currentstroke}{rgb}{0.000000,0.000000,0.000000}%
\pgfsetstrokecolor{currentstroke}%
\pgfsetstrokeopacity{0.000000}%
\pgfsetdash{}{0pt}%
\pgfpathmoveto{\pgfqpoint{2.070332in}{0.500000in}}%
\pgfpathlineto{\pgfqpoint{2.094610in}{0.500000in}}%
\pgfpathlineto{\pgfqpoint{2.094610in}{0.506042in}}%
\pgfpathlineto{\pgfqpoint{2.070332in}{0.506042in}}%
\pgfpathlineto{\pgfqpoint{2.070332in}{0.500000in}}%
\pgfpathclose%
\pgfusepath{fill}%
\end{pgfscope}%
\begin{pgfscope}%
\pgfpathrectangle{\pgfqpoint{0.750000in}{0.500000in}}{\pgfqpoint{4.650000in}{3.020000in}}%
\pgfusepath{clip}%
\pgfsetbuttcap%
\pgfsetmiterjoin%
\definecolor{currentfill}{rgb}{1.000000,0.000000,0.000000}%
\pgfsetfillcolor{currentfill}%
\pgfsetlinewidth{0.000000pt}%
\definecolor{currentstroke}{rgb}{0.000000,0.000000,0.000000}%
\pgfsetstrokecolor{currentstroke}%
\pgfsetstrokeopacity{0.000000}%
\pgfsetdash{}{0pt}%
\pgfpathmoveto{\pgfqpoint{2.094610in}{0.500000in}}%
\pgfpathlineto{\pgfqpoint{2.118889in}{0.500000in}}%
\pgfpathlineto{\pgfqpoint{2.118889in}{0.500000in}}%
\pgfpathlineto{\pgfqpoint{2.094610in}{0.500000in}}%
\pgfpathlineto{\pgfqpoint{2.094610in}{0.500000in}}%
\pgfpathclose%
\pgfusepath{fill}%
\end{pgfscope}%
\begin{pgfscope}%
\pgfpathrectangle{\pgfqpoint{0.750000in}{0.500000in}}{\pgfqpoint{4.650000in}{3.020000in}}%
\pgfusepath{clip}%
\pgfsetbuttcap%
\pgfsetmiterjoin%
\definecolor{currentfill}{rgb}{1.000000,0.000000,0.000000}%
\pgfsetfillcolor{currentfill}%
\pgfsetlinewidth{0.000000pt}%
\definecolor{currentstroke}{rgb}{0.000000,0.000000,0.000000}%
\pgfsetstrokecolor{currentstroke}%
\pgfsetstrokeopacity{0.000000}%
\pgfsetdash{}{0pt}%
\pgfpathmoveto{\pgfqpoint{2.118889in}{0.500000in}}%
\pgfpathlineto{\pgfqpoint{2.143168in}{0.500000in}}%
\pgfpathlineto{\pgfqpoint{2.143168in}{0.500000in}}%
\pgfpathlineto{\pgfqpoint{2.118889in}{0.500000in}}%
\pgfpathlineto{\pgfqpoint{2.118889in}{0.500000in}}%
\pgfpathclose%
\pgfusepath{fill}%
\end{pgfscope}%
\begin{pgfscope}%
\pgfpathrectangle{\pgfqpoint{0.750000in}{0.500000in}}{\pgfqpoint{4.650000in}{3.020000in}}%
\pgfusepath{clip}%
\pgfsetbuttcap%
\pgfsetmiterjoin%
\definecolor{currentfill}{rgb}{1.000000,0.000000,0.000000}%
\pgfsetfillcolor{currentfill}%
\pgfsetlinewidth{0.000000pt}%
\definecolor{currentstroke}{rgb}{0.000000,0.000000,0.000000}%
\pgfsetstrokecolor{currentstroke}%
\pgfsetstrokeopacity{0.000000}%
\pgfsetdash{}{0pt}%
\pgfpathmoveto{\pgfqpoint{2.143168in}{0.500000in}}%
\pgfpathlineto{\pgfqpoint{2.167446in}{0.500000in}}%
\pgfpathlineto{\pgfqpoint{2.167446in}{0.503021in}}%
\pgfpathlineto{\pgfqpoint{2.143168in}{0.503021in}}%
\pgfpathlineto{\pgfqpoint{2.143168in}{0.500000in}}%
\pgfpathclose%
\pgfusepath{fill}%
\end{pgfscope}%
\begin{pgfscope}%
\pgfpathrectangle{\pgfqpoint{0.750000in}{0.500000in}}{\pgfqpoint{4.650000in}{3.020000in}}%
\pgfusepath{clip}%
\pgfsetbuttcap%
\pgfsetmiterjoin%
\definecolor{currentfill}{rgb}{1.000000,0.000000,0.000000}%
\pgfsetfillcolor{currentfill}%
\pgfsetlinewidth{0.000000pt}%
\definecolor{currentstroke}{rgb}{0.000000,0.000000,0.000000}%
\pgfsetstrokecolor{currentstroke}%
\pgfsetstrokeopacity{0.000000}%
\pgfsetdash{}{0pt}%
\pgfpathmoveto{\pgfqpoint{2.167446in}{0.500000in}}%
\pgfpathlineto{\pgfqpoint{2.191725in}{0.500000in}}%
\pgfpathlineto{\pgfqpoint{2.191725in}{0.500000in}}%
\pgfpathlineto{\pgfqpoint{2.167446in}{0.500000in}}%
\pgfpathlineto{\pgfqpoint{2.167446in}{0.500000in}}%
\pgfpathclose%
\pgfusepath{fill}%
\end{pgfscope}%
\begin{pgfscope}%
\pgfpathrectangle{\pgfqpoint{0.750000in}{0.500000in}}{\pgfqpoint{4.650000in}{3.020000in}}%
\pgfusepath{clip}%
\pgfsetbuttcap%
\pgfsetmiterjoin%
\definecolor{currentfill}{rgb}{1.000000,0.000000,0.000000}%
\pgfsetfillcolor{currentfill}%
\pgfsetlinewidth{0.000000pt}%
\definecolor{currentstroke}{rgb}{0.000000,0.000000,0.000000}%
\pgfsetstrokecolor{currentstroke}%
\pgfsetstrokeopacity{0.000000}%
\pgfsetdash{}{0pt}%
\pgfpathmoveto{\pgfqpoint{2.191725in}{0.500000in}}%
\pgfpathlineto{\pgfqpoint{2.216003in}{0.500000in}}%
\pgfpathlineto{\pgfqpoint{2.216003in}{0.509064in}}%
\pgfpathlineto{\pgfqpoint{2.191725in}{0.509064in}}%
\pgfpathlineto{\pgfqpoint{2.191725in}{0.500000in}}%
\pgfpathclose%
\pgfusepath{fill}%
\end{pgfscope}%
\begin{pgfscope}%
\pgfpathrectangle{\pgfqpoint{0.750000in}{0.500000in}}{\pgfqpoint{4.650000in}{3.020000in}}%
\pgfusepath{clip}%
\pgfsetbuttcap%
\pgfsetmiterjoin%
\definecolor{currentfill}{rgb}{1.000000,0.000000,0.000000}%
\pgfsetfillcolor{currentfill}%
\pgfsetlinewidth{0.000000pt}%
\definecolor{currentstroke}{rgb}{0.000000,0.000000,0.000000}%
\pgfsetstrokecolor{currentstroke}%
\pgfsetstrokeopacity{0.000000}%
\pgfsetdash{}{0pt}%
\pgfpathmoveto{\pgfqpoint{2.216003in}{0.500000in}}%
\pgfpathlineto{\pgfqpoint{2.240282in}{0.500000in}}%
\pgfpathlineto{\pgfqpoint{2.240282in}{0.500000in}}%
\pgfpathlineto{\pgfqpoint{2.216003in}{0.500000in}}%
\pgfpathlineto{\pgfqpoint{2.216003in}{0.500000in}}%
\pgfpathclose%
\pgfusepath{fill}%
\end{pgfscope}%
\begin{pgfscope}%
\pgfpathrectangle{\pgfqpoint{0.750000in}{0.500000in}}{\pgfqpoint{4.650000in}{3.020000in}}%
\pgfusepath{clip}%
\pgfsetbuttcap%
\pgfsetmiterjoin%
\definecolor{currentfill}{rgb}{1.000000,0.000000,0.000000}%
\pgfsetfillcolor{currentfill}%
\pgfsetlinewidth{0.000000pt}%
\definecolor{currentstroke}{rgb}{0.000000,0.000000,0.000000}%
\pgfsetstrokecolor{currentstroke}%
\pgfsetstrokeopacity{0.000000}%
\pgfsetdash{}{0pt}%
\pgfpathmoveto{\pgfqpoint{2.240282in}{0.500000in}}%
\pgfpathlineto{\pgfqpoint{2.264561in}{0.500000in}}%
\pgfpathlineto{\pgfqpoint{2.264561in}{0.500000in}}%
\pgfpathlineto{\pgfqpoint{2.240282in}{0.500000in}}%
\pgfpathlineto{\pgfqpoint{2.240282in}{0.500000in}}%
\pgfpathclose%
\pgfusepath{fill}%
\end{pgfscope}%
\begin{pgfscope}%
\pgfpathrectangle{\pgfqpoint{0.750000in}{0.500000in}}{\pgfqpoint{4.650000in}{3.020000in}}%
\pgfusepath{clip}%
\pgfsetbuttcap%
\pgfsetmiterjoin%
\definecolor{currentfill}{rgb}{1.000000,0.000000,0.000000}%
\pgfsetfillcolor{currentfill}%
\pgfsetlinewidth{0.000000pt}%
\definecolor{currentstroke}{rgb}{0.000000,0.000000,0.000000}%
\pgfsetstrokecolor{currentstroke}%
\pgfsetstrokeopacity{0.000000}%
\pgfsetdash{}{0pt}%
\pgfpathmoveto{\pgfqpoint{2.264561in}{0.500000in}}%
\pgfpathlineto{\pgfqpoint{2.288839in}{0.500000in}}%
\pgfpathlineto{\pgfqpoint{2.288839in}{0.536255in}}%
\pgfpathlineto{\pgfqpoint{2.264561in}{0.536255in}}%
\pgfpathlineto{\pgfqpoint{2.264561in}{0.500000in}}%
\pgfpathclose%
\pgfusepath{fill}%
\end{pgfscope}%
\begin{pgfscope}%
\pgfpathrectangle{\pgfqpoint{0.750000in}{0.500000in}}{\pgfqpoint{4.650000in}{3.020000in}}%
\pgfusepath{clip}%
\pgfsetbuttcap%
\pgfsetmiterjoin%
\definecolor{currentfill}{rgb}{1.000000,0.000000,0.000000}%
\pgfsetfillcolor{currentfill}%
\pgfsetlinewidth{0.000000pt}%
\definecolor{currentstroke}{rgb}{0.000000,0.000000,0.000000}%
\pgfsetstrokecolor{currentstroke}%
\pgfsetstrokeopacity{0.000000}%
\pgfsetdash{}{0pt}%
\pgfpathmoveto{\pgfqpoint{2.288839in}{0.500000in}}%
\pgfpathlineto{\pgfqpoint{2.313118in}{0.500000in}}%
\pgfpathlineto{\pgfqpoint{2.313118in}{0.500000in}}%
\pgfpathlineto{\pgfqpoint{2.288839in}{0.500000in}}%
\pgfpathlineto{\pgfqpoint{2.288839in}{0.500000in}}%
\pgfpathclose%
\pgfusepath{fill}%
\end{pgfscope}%
\begin{pgfscope}%
\pgfpathrectangle{\pgfqpoint{0.750000in}{0.500000in}}{\pgfqpoint{4.650000in}{3.020000in}}%
\pgfusepath{clip}%
\pgfsetbuttcap%
\pgfsetmiterjoin%
\definecolor{currentfill}{rgb}{1.000000,0.000000,0.000000}%
\pgfsetfillcolor{currentfill}%
\pgfsetlinewidth{0.000000pt}%
\definecolor{currentstroke}{rgb}{0.000000,0.000000,0.000000}%
\pgfsetstrokecolor{currentstroke}%
\pgfsetstrokeopacity{0.000000}%
\pgfsetdash{}{0pt}%
\pgfpathmoveto{\pgfqpoint{2.313118in}{0.500000in}}%
\pgfpathlineto{\pgfqpoint{2.337396in}{0.500000in}}%
\pgfpathlineto{\pgfqpoint{2.337396in}{0.566467in}}%
\pgfpathlineto{\pgfqpoint{2.313118in}{0.566467in}}%
\pgfpathlineto{\pgfqpoint{2.313118in}{0.500000in}}%
\pgfpathclose%
\pgfusepath{fill}%
\end{pgfscope}%
\begin{pgfscope}%
\pgfpathrectangle{\pgfqpoint{0.750000in}{0.500000in}}{\pgfqpoint{4.650000in}{3.020000in}}%
\pgfusepath{clip}%
\pgfsetbuttcap%
\pgfsetmiterjoin%
\definecolor{currentfill}{rgb}{1.000000,0.000000,0.000000}%
\pgfsetfillcolor{currentfill}%
\pgfsetlinewidth{0.000000pt}%
\definecolor{currentstroke}{rgb}{0.000000,0.000000,0.000000}%
\pgfsetstrokecolor{currentstroke}%
\pgfsetstrokeopacity{0.000000}%
\pgfsetdash{}{0pt}%
\pgfpathmoveto{\pgfqpoint{2.337396in}{0.500000in}}%
\pgfpathlineto{\pgfqpoint{2.361675in}{0.500000in}}%
\pgfpathlineto{\pgfqpoint{2.361675in}{0.503021in}}%
\pgfpathlineto{\pgfqpoint{2.337396in}{0.503021in}}%
\pgfpathlineto{\pgfqpoint{2.337396in}{0.500000in}}%
\pgfpathclose%
\pgfusepath{fill}%
\end{pgfscope}%
\begin{pgfscope}%
\pgfpathrectangle{\pgfqpoint{0.750000in}{0.500000in}}{\pgfqpoint{4.650000in}{3.020000in}}%
\pgfusepath{clip}%
\pgfsetbuttcap%
\pgfsetmiterjoin%
\definecolor{currentfill}{rgb}{1.000000,0.000000,0.000000}%
\pgfsetfillcolor{currentfill}%
\pgfsetlinewidth{0.000000pt}%
\definecolor{currentstroke}{rgb}{0.000000,0.000000,0.000000}%
\pgfsetstrokecolor{currentstroke}%
\pgfsetstrokeopacity{0.000000}%
\pgfsetdash{}{0pt}%
\pgfpathmoveto{\pgfqpoint{2.361675in}{0.500000in}}%
\pgfpathlineto{\pgfqpoint{2.385953in}{0.500000in}}%
\pgfpathlineto{\pgfqpoint{2.385953in}{0.512085in}}%
\pgfpathlineto{\pgfqpoint{2.361675in}{0.512085in}}%
\pgfpathlineto{\pgfqpoint{2.361675in}{0.500000in}}%
\pgfpathclose%
\pgfusepath{fill}%
\end{pgfscope}%
\begin{pgfscope}%
\pgfpathrectangle{\pgfqpoint{0.750000in}{0.500000in}}{\pgfqpoint{4.650000in}{3.020000in}}%
\pgfusepath{clip}%
\pgfsetbuttcap%
\pgfsetmiterjoin%
\definecolor{currentfill}{rgb}{1.000000,0.000000,0.000000}%
\pgfsetfillcolor{currentfill}%
\pgfsetlinewidth{0.000000pt}%
\definecolor{currentstroke}{rgb}{0.000000,0.000000,0.000000}%
\pgfsetstrokecolor{currentstroke}%
\pgfsetstrokeopacity{0.000000}%
\pgfsetdash{}{0pt}%
\pgfpathmoveto{\pgfqpoint{2.385953in}{0.500000in}}%
\pgfpathlineto{\pgfqpoint{2.410232in}{0.500000in}}%
\pgfpathlineto{\pgfqpoint{2.410232in}{0.611785in}}%
\pgfpathlineto{\pgfqpoint{2.385953in}{0.611785in}}%
\pgfpathlineto{\pgfqpoint{2.385953in}{0.500000in}}%
\pgfpathclose%
\pgfusepath{fill}%
\end{pgfscope}%
\begin{pgfscope}%
\pgfpathrectangle{\pgfqpoint{0.750000in}{0.500000in}}{\pgfqpoint{4.650000in}{3.020000in}}%
\pgfusepath{clip}%
\pgfsetbuttcap%
\pgfsetmiterjoin%
\definecolor{currentfill}{rgb}{1.000000,0.000000,0.000000}%
\pgfsetfillcolor{currentfill}%
\pgfsetlinewidth{0.000000pt}%
\definecolor{currentstroke}{rgb}{0.000000,0.000000,0.000000}%
\pgfsetstrokecolor{currentstroke}%
\pgfsetstrokeopacity{0.000000}%
\pgfsetdash{}{0pt}%
\pgfpathmoveto{\pgfqpoint{2.410232in}{0.500000in}}%
\pgfpathlineto{\pgfqpoint{2.434511in}{0.500000in}}%
\pgfpathlineto{\pgfqpoint{2.434511in}{0.506042in}}%
\pgfpathlineto{\pgfqpoint{2.410232in}{0.506042in}}%
\pgfpathlineto{\pgfqpoint{2.410232in}{0.500000in}}%
\pgfpathclose%
\pgfusepath{fill}%
\end{pgfscope}%
\begin{pgfscope}%
\pgfpathrectangle{\pgfqpoint{0.750000in}{0.500000in}}{\pgfqpoint{4.650000in}{3.020000in}}%
\pgfusepath{clip}%
\pgfsetbuttcap%
\pgfsetmiterjoin%
\definecolor{currentfill}{rgb}{1.000000,0.000000,0.000000}%
\pgfsetfillcolor{currentfill}%
\pgfsetlinewidth{0.000000pt}%
\definecolor{currentstroke}{rgb}{0.000000,0.000000,0.000000}%
\pgfsetstrokecolor{currentstroke}%
\pgfsetstrokeopacity{0.000000}%
\pgfsetdash{}{0pt}%
\pgfpathmoveto{\pgfqpoint{2.434511in}{0.500000in}}%
\pgfpathlineto{\pgfqpoint{2.458789in}{0.500000in}}%
\pgfpathlineto{\pgfqpoint{2.458789in}{0.503021in}}%
\pgfpathlineto{\pgfqpoint{2.434511in}{0.503021in}}%
\pgfpathlineto{\pgfqpoint{2.434511in}{0.500000in}}%
\pgfpathclose%
\pgfusepath{fill}%
\end{pgfscope}%
\begin{pgfscope}%
\pgfpathrectangle{\pgfqpoint{0.750000in}{0.500000in}}{\pgfqpoint{4.650000in}{3.020000in}}%
\pgfusepath{clip}%
\pgfsetbuttcap%
\pgfsetmiterjoin%
\definecolor{currentfill}{rgb}{1.000000,0.000000,0.000000}%
\pgfsetfillcolor{currentfill}%
\pgfsetlinewidth{0.000000pt}%
\definecolor{currentstroke}{rgb}{0.000000,0.000000,0.000000}%
\pgfsetstrokecolor{currentstroke}%
\pgfsetstrokeopacity{0.000000}%
\pgfsetdash{}{0pt}%
\pgfpathmoveto{\pgfqpoint{2.458789in}{0.500000in}}%
\pgfpathlineto{\pgfqpoint{2.483068in}{0.500000in}}%
\pgfpathlineto{\pgfqpoint{2.483068in}{0.759824in}}%
\pgfpathlineto{\pgfqpoint{2.458789in}{0.759824in}}%
\pgfpathlineto{\pgfqpoint{2.458789in}{0.500000in}}%
\pgfpathclose%
\pgfusepath{fill}%
\end{pgfscope}%
\begin{pgfscope}%
\pgfpathrectangle{\pgfqpoint{0.750000in}{0.500000in}}{\pgfqpoint{4.650000in}{3.020000in}}%
\pgfusepath{clip}%
\pgfsetbuttcap%
\pgfsetmiterjoin%
\definecolor{currentfill}{rgb}{1.000000,0.000000,0.000000}%
\pgfsetfillcolor{currentfill}%
\pgfsetlinewidth{0.000000pt}%
\definecolor{currentstroke}{rgb}{0.000000,0.000000,0.000000}%
\pgfsetstrokecolor{currentstroke}%
\pgfsetstrokeopacity{0.000000}%
\pgfsetdash{}{0pt}%
\pgfpathmoveto{\pgfqpoint{2.483068in}{0.500000in}}%
\pgfpathlineto{\pgfqpoint{2.507346in}{0.500000in}}%
\pgfpathlineto{\pgfqpoint{2.507346in}{0.527191in}}%
\pgfpathlineto{\pgfqpoint{2.483068in}{0.527191in}}%
\pgfpathlineto{\pgfqpoint{2.483068in}{0.500000in}}%
\pgfpathclose%
\pgfusepath{fill}%
\end{pgfscope}%
\begin{pgfscope}%
\pgfpathrectangle{\pgfqpoint{0.750000in}{0.500000in}}{\pgfqpoint{4.650000in}{3.020000in}}%
\pgfusepath{clip}%
\pgfsetbuttcap%
\pgfsetmiterjoin%
\definecolor{currentfill}{rgb}{1.000000,0.000000,0.000000}%
\pgfsetfillcolor{currentfill}%
\pgfsetlinewidth{0.000000pt}%
\definecolor{currentstroke}{rgb}{0.000000,0.000000,0.000000}%
\pgfsetstrokecolor{currentstroke}%
\pgfsetstrokeopacity{0.000000}%
\pgfsetdash{}{0pt}%
\pgfpathmoveto{\pgfqpoint{2.507346in}{0.500000in}}%
\pgfpathlineto{\pgfqpoint{2.531625in}{0.500000in}}%
\pgfpathlineto{\pgfqpoint{2.531625in}{0.527191in}}%
\pgfpathlineto{\pgfqpoint{2.507346in}{0.527191in}}%
\pgfpathlineto{\pgfqpoint{2.507346in}{0.500000in}}%
\pgfpathclose%
\pgfusepath{fill}%
\end{pgfscope}%
\begin{pgfscope}%
\pgfpathrectangle{\pgfqpoint{0.750000in}{0.500000in}}{\pgfqpoint{4.650000in}{3.020000in}}%
\pgfusepath{clip}%
\pgfsetbuttcap%
\pgfsetmiterjoin%
\definecolor{currentfill}{rgb}{1.000000,0.000000,0.000000}%
\pgfsetfillcolor{currentfill}%
\pgfsetlinewidth{0.000000pt}%
\definecolor{currentstroke}{rgb}{0.000000,0.000000,0.000000}%
\pgfsetstrokecolor{currentstroke}%
\pgfsetstrokeopacity{0.000000}%
\pgfsetdash{}{0pt}%
\pgfpathmoveto{\pgfqpoint{2.531625in}{0.500000in}}%
\pgfpathlineto{\pgfqpoint{2.555904in}{0.500000in}}%
\pgfpathlineto{\pgfqpoint{2.555904in}{0.880672in}}%
\pgfpathlineto{\pgfqpoint{2.531625in}{0.880672in}}%
\pgfpathlineto{\pgfqpoint{2.531625in}{0.500000in}}%
\pgfpathclose%
\pgfusepath{fill}%
\end{pgfscope}%
\begin{pgfscope}%
\pgfpathrectangle{\pgfqpoint{0.750000in}{0.500000in}}{\pgfqpoint{4.650000in}{3.020000in}}%
\pgfusepath{clip}%
\pgfsetbuttcap%
\pgfsetmiterjoin%
\definecolor{currentfill}{rgb}{1.000000,0.000000,0.000000}%
\pgfsetfillcolor{currentfill}%
\pgfsetlinewidth{0.000000pt}%
\definecolor{currentstroke}{rgb}{0.000000,0.000000,0.000000}%
\pgfsetstrokecolor{currentstroke}%
\pgfsetstrokeopacity{0.000000}%
\pgfsetdash{}{0pt}%
\pgfpathmoveto{\pgfqpoint{2.555904in}{0.500000in}}%
\pgfpathlineto{\pgfqpoint{2.580182in}{0.500000in}}%
\pgfpathlineto{\pgfqpoint{2.580182in}{0.512085in}}%
\pgfpathlineto{\pgfqpoint{2.555904in}{0.512085in}}%
\pgfpathlineto{\pgfqpoint{2.555904in}{0.500000in}}%
\pgfpathclose%
\pgfusepath{fill}%
\end{pgfscope}%
\begin{pgfscope}%
\pgfpathrectangle{\pgfqpoint{0.750000in}{0.500000in}}{\pgfqpoint{4.650000in}{3.020000in}}%
\pgfusepath{clip}%
\pgfsetbuttcap%
\pgfsetmiterjoin%
\definecolor{currentfill}{rgb}{1.000000,0.000000,0.000000}%
\pgfsetfillcolor{currentfill}%
\pgfsetlinewidth{0.000000pt}%
\definecolor{currentstroke}{rgb}{0.000000,0.000000,0.000000}%
\pgfsetstrokecolor{currentstroke}%
\pgfsetstrokeopacity{0.000000}%
\pgfsetdash{}{0pt}%
\pgfpathmoveto{\pgfqpoint{2.580182in}{0.500000in}}%
\pgfpathlineto{\pgfqpoint{2.604461in}{0.500000in}}%
\pgfpathlineto{\pgfqpoint{2.604461in}{1.071008in}}%
\pgfpathlineto{\pgfqpoint{2.580182in}{1.071008in}}%
\pgfpathlineto{\pgfqpoint{2.580182in}{0.500000in}}%
\pgfpathclose%
\pgfusepath{fill}%
\end{pgfscope}%
\begin{pgfscope}%
\pgfpathrectangle{\pgfqpoint{0.750000in}{0.500000in}}{\pgfqpoint{4.650000in}{3.020000in}}%
\pgfusepath{clip}%
\pgfsetbuttcap%
\pgfsetmiterjoin%
\definecolor{currentfill}{rgb}{1.000000,0.000000,0.000000}%
\pgfsetfillcolor{currentfill}%
\pgfsetlinewidth{0.000000pt}%
\definecolor{currentstroke}{rgb}{0.000000,0.000000,0.000000}%
\pgfsetstrokecolor{currentstroke}%
\pgfsetstrokeopacity{0.000000}%
\pgfsetdash{}{0pt}%
\pgfpathmoveto{\pgfqpoint{2.604461in}{0.500000in}}%
\pgfpathlineto{\pgfqpoint{2.628739in}{0.500000in}}%
\pgfpathlineto{\pgfqpoint{2.628739in}{0.521148in}}%
\pgfpathlineto{\pgfqpoint{2.604461in}{0.521148in}}%
\pgfpathlineto{\pgfqpoint{2.604461in}{0.500000in}}%
\pgfpathclose%
\pgfusepath{fill}%
\end{pgfscope}%
\begin{pgfscope}%
\pgfpathrectangle{\pgfqpoint{0.750000in}{0.500000in}}{\pgfqpoint{4.650000in}{3.020000in}}%
\pgfusepath{clip}%
\pgfsetbuttcap%
\pgfsetmiterjoin%
\definecolor{currentfill}{rgb}{1.000000,0.000000,0.000000}%
\pgfsetfillcolor{currentfill}%
\pgfsetlinewidth{0.000000pt}%
\definecolor{currentstroke}{rgb}{0.000000,0.000000,0.000000}%
\pgfsetstrokecolor{currentstroke}%
\pgfsetstrokeopacity{0.000000}%
\pgfsetdash{}{0pt}%
\pgfpathmoveto{\pgfqpoint{2.628739in}{0.500000in}}%
\pgfpathlineto{\pgfqpoint{2.653018in}{0.500000in}}%
\pgfpathlineto{\pgfqpoint{2.653018in}{0.524170in}}%
\pgfpathlineto{\pgfqpoint{2.628739in}{0.524170in}}%
\pgfpathlineto{\pgfqpoint{2.628739in}{0.500000in}}%
\pgfpathclose%
\pgfusepath{fill}%
\end{pgfscope}%
\begin{pgfscope}%
\pgfpathrectangle{\pgfqpoint{0.750000in}{0.500000in}}{\pgfqpoint{4.650000in}{3.020000in}}%
\pgfusepath{clip}%
\pgfsetbuttcap%
\pgfsetmiterjoin%
\definecolor{currentfill}{rgb}{1.000000,0.000000,0.000000}%
\pgfsetfillcolor{currentfill}%
\pgfsetlinewidth{0.000000pt}%
\definecolor{currentstroke}{rgb}{0.000000,0.000000,0.000000}%
\pgfsetstrokecolor{currentstroke}%
\pgfsetstrokeopacity{0.000000}%
\pgfsetdash{}{0pt}%
\pgfpathmoveto{\pgfqpoint{2.653018in}{0.500000in}}%
\pgfpathlineto{\pgfqpoint{2.677296in}{0.500000in}}%
\pgfpathlineto{\pgfqpoint{2.677296in}{1.436575in}}%
\pgfpathlineto{\pgfqpoint{2.653018in}{1.436575in}}%
\pgfpathlineto{\pgfqpoint{2.653018in}{0.500000in}}%
\pgfpathclose%
\pgfusepath{fill}%
\end{pgfscope}%
\begin{pgfscope}%
\pgfpathrectangle{\pgfqpoint{0.750000in}{0.500000in}}{\pgfqpoint{4.650000in}{3.020000in}}%
\pgfusepath{clip}%
\pgfsetbuttcap%
\pgfsetmiterjoin%
\definecolor{currentfill}{rgb}{1.000000,0.000000,0.000000}%
\pgfsetfillcolor{currentfill}%
\pgfsetlinewidth{0.000000pt}%
\definecolor{currentstroke}{rgb}{0.000000,0.000000,0.000000}%
\pgfsetstrokecolor{currentstroke}%
\pgfsetstrokeopacity{0.000000}%
\pgfsetdash{}{0pt}%
\pgfpathmoveto{\pgfqpoint{2.677296in}{0.500000in}}%
\pgfpathlineto{\pgfqpoint{2.701575in}{0.500000in}}%
\pgfpathlineto{\pgfqpoint{2.701575in}{0.530212in}}%
\pgfpathlineto{\pgfqpoint{2.677296in}{0.530212in}}%
\pgfpathlineto{\pgfqpoint{2.677296in}{0.500000in}}%
\pgfpathclose%
\pgfusepath{fill}%
\end{pgfscope}%
\begin{pgfscope}%
\pgfpathrectangle{\pgfqpoint{0.750000in}{0.500000in}}{\pgfqpoint{4.650000in}{3.020000in}}%
\pgfusepath{clip}%
\pgfsetbuttcap%
\pgfsetmiterjoin%
\definecolor{currentfill}{rgb}{1.000000,0.000000,0.000000}%
\pgfsetfillcolor{currentfill}%
\pgfsetlinewidth{0.000000pt}%
\definecolor{currentstroke}{rgb}{0.000000,0.000000,0.000000}%
\pgfsetstrokecolor{currentstroke}%
\pgfsetstrokeopacity{0.000000}%
\pgfsetdash{}{0pt}%
\pgfpathmoveto{\pgfqpoint{2.701575in}{0.500000in}}%
\pgfpathlineto{\pgfqpoint{2.725854in}{0.500000in}}%
\pgfpathlineto{\pgfqpoint{2.725854in}{0.533233in}}%
\pgfpathlineto{\pgfqpoint{2.701575in}{0.533233in}}%
\pgfpathlineto{\pgfqpoint{2.701575in}{0.500000in}}%
\pgfpathclose%
\pgfusepath{fill}%
\end{pgfscope}%
\begin{pgfscope}%
\pgfpathrectangle{\pgfqpoint{0.750000in}{0.500000in}}{\pgfqpoint{4.650000in}{3.020000in}}%
\pgfusepath{clip}%
\pgfsetbuttcap%
\pgfsetmiterjoin%
\definecolor{currentfill}{rgb}{1.000000,0.000000,0.000000}%
\pgfsetfillcolor{currentfill}%
\pgfsetlinewidth{0.000000pt}%
\definecolor{currentstroke}{rgb}{0.000000,0.000000,0.000000}%
\pgfsetstrokecolor{currentstroke}%
\pgfsetstrokeopacity{0.000000}%
\pgfsetdash{}{0pt}%
\pgfpathmoveto{\pgfqpoint{2.725854in}{0.500000in}}%
\pgfpathlineto{\pgfqpoint{2.750132in}{0.500000in}}%
\pgfpathlineto{\pgfqpoint{2.750132in}{1.880692in}}%
\pgfpathlineto{\pgfqpoint{2.725854in}{1.880692in}}%
\pgfpathlineto{\pgfqpoint{2.725854in}{0.500000in}}%
\pgfpathclose%
\pgfusepath{fill}%
\end{pgfscope}%
\begin{pgfscope}%
\pgfpathrectangle{\pgfqpoint{0.750000in}{0.500000in}}{\pgfqpoint{4.650000in}{3.020000in}}%
\pgfusepath{clip}%
\pgfsetbuttcap%
\pgfsetmiterjoin%
\definecolor{currentfill}{rgb}{1.000000,0.000000,0.000000}%
\pgfsetfillcolor{currentfill}%
\pgfsetlinewidth{0.000000pt}%
\definecolor{currentstroke}{rgb}{0.000000,0.000000,0.000000}%
\pgfsetstrokecolor{currentstroke}%
\pgfsetstrokeopacity{0.000000}%
\pgfsetdash{}{0pt}%
\pgfpathmoveto{\pgfqpoint{2.750132in}{0.500000in}}%
\pgfpathlineto{\pgfqpoint{2.774411in}{0.500000in}}%
\pgfpathlineto{\pgfqpoint{2.774411in}{0.530212in}}%
\pgfpathlineto{\pgfqpoint{2.750132in}{0.530212in}}%
\pgfpathlineto{\pgfqpoint{2.750132in}{0.500000in}}%
\pgfpathclose%
\pgfusepath{fill}%
\end{pgfscope}%
\begin{pgfscope}%
\pgfpathrectangle{\pgfqpoint{0.750000in}{0.500000in}}{\pgfqpoint{4.650000in}{3.020000in}}%
\pgfusepath{clip}%
\pgfsetbuttcap%
\pgfsetmiterjoin%
\definecolor{currentfill}{rgb}{1.000000,0.000000,0.000000}%
\pgfsetfillcolor{currentfill}%
\pgfsetlinewidth{0.000000pt}%
\definecolor{currentstroke}{rgb}{0.000000,0.000000,0.000000}%
\pgfsetstrokecolor{currentstroke}%
\pgfsetstrokeopacity{0.000000}%
\pgfsetdash{}{0pt}%
\pgfpathmoveto{\pgfqpoint{2.774411in}{0.500000in}}%
\pgfpathlineto{\pgfqpoint{2.798689in}{0.500000in}}%
\pgfpathlineto{\pgfqpoint{2.798689in}{2.119368in}}%
\pgfpathlineto{\pgfqpoint{2.774411in}{2.119368in}}%
\pgfpathlineto{\pgfqpoint{2.774411in}{0.500000in}}%
\pgfpathclose%
\pgfusepath{fill}%
\end{pgfscope}%
\begin{pgfscope}%
\pgfpathrectangle{\pgfqpoint{0.750000in}{0.500000in}}{\pgfqpoint{4.650000in}{3.020000in}}%
\pgfusepath{clip}%
\pgfsetbuttcap%
\pgfsetmiterjoin%
\definecolor{currentfill}{rgb}{1.000000,0.000000,0.000000}%
\pgfsetfillcolor{currentfill}%
\pgfsetlinewidth{0.000000pt}%
\definecolor{currentstroke}{rgb}{0.000000,0.000000,0.000000}%
\pgfsetstrokecolor{currentstroke}%
\pgfsetstrokeopacity{0.000000}%
\pgfsetdash{}{0pt}%
\pgfpathmoveto{\pgfqpoint{2.798689in}{0.500000in}}%
\pgfpathlineto{\pgfqpoint{2.822968in}{0.500000in}}%
\pgfpathlineto{\pgfqpoint{2.822968in}{0.690336in}}%
\pgfpathlineto{\pgfqpoint{2.798689in}{0.690336in}}%
\pgfpathlineto{\pgfqpoint{2.798689in}{0.500000in}}%
\pgfpathclose%
\pgfusepath{fill}%
\end{pgfscope}%
\begin{pgfscope}%
\pgfpathrectangle{\pgfqpoint{0.750000in}{0.500000in}}{\pgfqpoint{4.650000in}{3.020000in}}%
\pgfusepath{clip}%
\pgfsetbuttcap%
\pgfsetmiterjoin%
\definecolor{currentfill}{rgb}{1.000000,0.000000,0.000000}%
\pgfsetfillcolor{currentfill}%
\pgfsetlinewidth{0.000000pt}%
\definecolor{currentstroke}{rgb}{0.000000,0.000000,0.000000}%
\pgfsetstrokecolor{currentstroke}%
\pgfsetstrokeopacity{0.000000}%
\pgfsetdash{}{0pt}%
\pgfpathmoveto{\pgfqpoint{2.822968in}{0.500000in}}%
\pgfpathlineto{\pgfqpoint{2.847247in}{0.500000in}}%
\pgfpathlineto{\pgfqpoint{2.847247in}{0.572509in}}%
\pgfpathlineto{\pgfqpoint{2.822968in}{0.572509in}}%
\pgfpathlineto{\pgfqpoint{2.822968in}{0.500000in}}%
\pgfpathclose%
\pgfusepath{fill}%
\end{pgfscope}%
\begin{pgfscope}%
\pgfpathrectangle{\pgfqpoint{0.750000in}{0.500000in}}{\pgfqpoint{4.650000in}{3.020000in}}%
\pgfusepath{clip}%
\pgfsetbuttcap%
\pgfsetmiterjoin%
\definecolor{currentfill}{rgb}{1.000000,0.000000,0.000000}%
\pgfsetfillcolor{currentfill}%
\pgfsetlinewidth{0.000000pt}%
\definecolor{currentstroke}{rgb}{0.000000,0.000000,0.000000}%
\pgfsetstrokecolor{currentstroke}%
\pgfsetstrokeopacity{0.000000}%
\pgfsetdash{}{0pt}%
\pgfpathmoveto{\pgfqpoint{2.847247in}{0.500000in}}%
\pgfpathlineto{\pgfqpoint{2.871525in}{0.500000in}}%
\pgfpathlineto{\pgfqpoint{2.871525in}{2.584634in}}%
\pgfpathlineto{\pgfqpoint{2.847247in}{2.584634in}}%
\pgfpathlineto{\pgfqpoint{2.847247in}{0.500000in}}%
\pgfpathclose%
\pgfusepath{fill}%
\end{pgfscope}%
\begin{pgfscope}%
\pgfpathrectangle{\pgfqpoint{0.750000in}{0.500000in}}{\pgfqpoint{4.650000in}{3.020000in}}%
\pgfusepath{clip}%
\pgfsetbuttcap%
\pgfsetmiterjoin%
\definecolor{currentfill}{rgb}{1.000000,0.000000,0.000000}%
\pgfsetfillcolor{currentfill}%
\pgfsetlinewidth{0.000000pt}%
\definecolor{currentstroke}{rgb}{0.000000,0.000000,0.000000}%
\pgfsetstrokecolor{currentstroke}%
\pgfsetstrokeopacity{0.000000}%
\pgfsetdash{}{0pt}%
\pgfpathmoveto{\pgfqpoint{2.871525in}{0.500000in}}%
\pgfpathlineto{\pgfqpoint{2.895804in}{0.500000in}}%
\pgfpathlineto{\pgfqpoint{2.895804in}{0.590636in}}%
\pgfpathlineto{\pgfqpoint{2.871525in}{0.590636in}}%
\pgfpathlineto{\pgfqpoint{2.871525in}{0.500000in}}%
\pgfpathclose%
\pgfusepath{fill}%
\end{pgfscope}%
\begin{pgfscope}%
\pgfpathrectangle{\pgfqpoint{0.750000in}{0.500000in}}{\pgfqpoint{4.650000in}{3.020000in}}%
\pgfusepath{clip}%
\pgfsetbuttcap%
\pgfsetmiterjoin%
\definecolor{currentfill}{rgb}{1.000000,0.000000,0.000000}%
\pgfsetfillcolor{currentfill}%
\pgfsetlinewidth{0.000000pt}%
\definecolor{currentstroke}{rgb}{0.000000,0.000000,0.000000}%
\pgfsetstrokecolor{currentstroke}%
\pgfsetstrokeopacity{0.000000}%
\pgfsetdash{}{0pt}%
\pgfpathmoveto{\pgfqpoint{2.895804in}{0.500000in}}%
\pgfpathlineto{\pgfqpoint{2.920082in}{0.500000in}}%
\pgfpathlineto{\pgfqpoint{2.920082in}{0.596679in}}%
\pgfpathlineto{\pgfqpoint{2.895804in}{0.596679in}}%
\pgfpathlineto{\pgfqpoint{2.895804in}{0.500000in}}%
\pgfpathclose%
\pgfusepath{fill}%
\end{pgfscope}%
\begin{pgfscope}%
\pgfpathrectangle{\pgfqpoint{0.750000in}{0.500000in}}{\pgfqpoint{4.650000in}{3.020000in}}%
\pgfusepath{clip}%
\pgfsetbuttcap%
\pgfsetmiterjoin%
\definecolor{currentfill}{rgb}{1.000000,0.000000,0.000000}%
\pgfsetfillcolor{currentfill}%
\pgfsetlinewidth{0.000000pt}%
\definecolor{currentstroke}{rgb}{0.000000,0.000000,0.000000}%
\pgfsetstrokecolor{currentstroke}%
\pgfsetstrokeopacity{0.000000}%
\pgfsetdash{}{0pt}%
\pgfpathmoveto{\pgfqpoint{2.920082in}{0.500000in}}%
\pgfpathlineto{\pgfqpoint{2.944361in}{0.500000in}}%
\pgfpathlineto{\pgfqpoint{2.944361in}{3.037815in}}%
\pgfpathlineto{\pgfqpoint{2.920082in}{3.037815in}}%
\pgfpathlineto{\pgfqpoint{2.920082in}{0.500000in}}%
\pgfpathclose%
\pgfusepath{fill}%
\end{pgfscope}%
\begin{pgfscope}%
\pgfpathrectangle{\pgfqpoint{0.750000in}{0.500000in}}{\pgfqpoint{4.650000in}{3.020000in}}%
\pgfusepath{clip}%
\pgfsetbuttcap%
\pgfsetmiterjoin%
\definecolor{currentfill}{rgb}{1.000000,0.000000,0.000000}%
\pgfsetfillcolor{currentfill}%
\pgfsetlinewidth{0.000000pt}%
\definecolor{currentstroke}{rgb}{0.000000,0.000000,0.000000}%
\pgfsetstrokecolor{currentstroke}%
\pgfsetstrokeopacity{0.000000}%
\pgfsetdash{}{0pt}%
\pgfpathmoveto{\pgfqpoint{2.944361in}{0.500000in}}%
\pgfpathlineto{\pgfqpoint{2.968639in}{0.500000in}}%
\pgfpathlineto{\pgfqpoint{2.968639in}{0.575530in}}%
\pgfpathlineto{\pgfqpoint{2.944361in}{0.575530in}}%
\pgfpathlineto{\pgfqpoint{2.944361in}{0.500000in}}%
\pgfpathclose%
\pgfusepath{fill}%
\end{pgfscope}%
\begin{pgfscope}%
\pgfpathrectangle{\pgfqpoint{0.750000in}{0.500000in}}{\pgfqpoint{4.650000in}{3.020000in}}%
\pgfusepath{clip}%
\pgfsetbuttcap%
\pgfsetmiterjoin%
\definecolor{currentfill}{rgb}{1.000000,0.000000,0.000000}%
\pgfsetfillcolor{currentfill}%
\pgfsetlinewidth{0.000000pt}%
\definecolor{currentstroke}{rgb}{0.000000,0.000000,0.000000}%
\pgfsetstrokecolor{currentstroke}%
\pgfsetstrokeopacity{0.000000}%
\pgfsetdash{}{0pt}%
\pgfpathmoveto{\pgfqpoint{2.968639in}{0.500000in}}%
\pgfpathlineto{\pgfqpoint{2.992918in}{0.500000in}}%
\pgfpathlineto{\pgfqpoint{2.992918in}{0.608764in}}%
\pgfpathlineto{\pgfqpoint{2.968639in}{0.608764in}}%
\pgfpathlineto{\pgfqpoint{2.968639in}{0.500000in}}%
\pgfpathclose%
\pgfusepath{fill}%
\end{pgfscope}%
\begin{pgfscope}%
\pgfpathrectangle{\pgfqpoint{0.750000in}{0.500000in}}{\pgfqpoint{4.650000in}{3.020000in}}%
\pgfusepath{clip}%
\pgfsetbuttcap%
\pgfsetmiterjoin%
\definecolor{currentfill}{rgb}{1.000000,0.000000,0.000000}%
\pgfsetfillcolor{currentfill}%
\pgfsetlinewidth{0.000000pt}%
\definecolor{currentstroke}{rgb}{0.000000,0.000000,0.000000}%
\pgfsetstrokecolor{currentstroke}%
\pgfsetstrokeopacity{0.000000}%
\pgfsetdash{}{0pt}%
\pgfpathmoveto{\pgfqpoint{2.992918in}{0.500000in}}%
\pgfpathlineto{\pgfqpoint{3.017197in}{0.500000in}}%
\pgfpathlineto{\pgfqpoint{3.017197in}{3.267427in}}%
\pgfpathlineto{\pgfqpoint{2.992918in}{3.267427in}}%
\pgfpathlineto{\pgfqpoint{2.992918in}{0.500000in}}%
\pgfpathclose%
\pgfusepath{fill}%
\end{pgfscope}%
\begin{pgfscope}%
\pgfpathrectangle{\pgfqpoint{0.750000in}{0.500000in}}{\pgfqpoint{4.650000in}{3.020000in}}%
\pgfusepath{clip}%
\pgfsetbuttcap%
\pgfsetmiterjoin%
\definecolor{currentfill}{rgb}{1.000000,0.000000,0.000000}%
\pgfsetfillcolor{currentfill}%
\pgfsetlinewidth{0.000000pt}%
\definecolor{currentstroke}{rgb}{0.000000,0.000000,0.000000}%
\pgfsetstrokecolor{currentstroke}%
\pgfsetstrokeopacity{0.000000}%
\pgfsetdash{}{0pt}%
\pgfpathmoveto{\pgfqpoint{3.017197in}{0.500000in}}%
\pgfpathlineto{\pgfqpoint{3.041475in}{0.500000in}}%
\pgfpathlineto{\pgfqpoint{3.041475in}{0.587615in}}%
\pgfpathlineto{\pgfqpoint{3.017197in}{0.587615in}}%
\pgfpathlineto{\pgfqpoint{3.017197in}{0.500000in}}%
\pgfpathclose%
\pgfusepath{fill}%
\end{pgfscope}%
\begin{pgfscope}%
\pgfpathrectangle{\pgfqpoint{0.750000in}{0.500000in}}{\pgfqpoint{4.650000in}{3.020000in}}%
\pgfusepath{clip}%
\pgfsetbuttcap%
\pgfsetmiterjoin%
\definecolor{currentfill}{rgb}{1.000000,0.000000,0.000000}%
\pgfsetfillcolor{currentfill}%
\pgfsetlinewidth{0.000000pt}%
\definecolor{currentstroke}{rgb}{0.000000,0.000000,0.000000}%
\pgfsetstrokecolor{currentstroke}%
\pgfsetstrokeopacity{0.000000}%
\pgfsetdash{}{0pt}%
\pgfpathmoveto{\pgfqpoint{3.041475in}{0.500000in}}%
\pgfpathlineto{\pgfqpoint{3.065754in}{0.500000in}}%
\pgfpathlineto{\pgfqpoint{3.065754in}{3.376190in}}%
\pgfpathlineto{\pgfqpoint{3.041475in}{3.376190in}}%
\pgfpathlineto{\pgfqpoint{3.041475in}{0.500000in}}%
\pgfpathclose%
\pgfusepath{fill}%
\end{pgfscope}%
\begin{pgfscope}%
\pgfpathrectangle{\pgfqpoint{0.750000in}{0.500000in}}{\pgfqpoint{4.650000in}{3.020000in}}%
\pgfusepath{clip}%
\pgfsetbuttcap%
\pgfsetmiterjoin%
\definecolor{currentfill}{rgb}{1.000000,0.000000,0.000000}%
\pgfsetfillcolor{currentfill}%
\pgfsetlinewidth{0.000000pt}%
\definecolor{currentstroke}{rgb}{0.000000,0.000000,0.000000}%
\pgfsetstrokecolor{currentstroke}%
\pgfsetstrokeopacity{0.000000}%
\pgfsetdash{}{0pt}%
\pgfpathmoveto{\pgfqpoint{3.065754in}{0.500000in}}%
\pgfpathlineto{\pgfqpoint{3.090032in}{0.500000in}}%
\pgfpathlineto{\pgfqpoint{3.090032in}{0.645018in}}%
\pgfpathlineto{\pgfqpoint{3.065754in}{0.645018in}}%
\pgfpathlineto{\pgfqpoint{3.065754in}{0.500000in}}%
\pgfpathclose%
\pgfusepath{fill}%
\end{pgfscope}%
\begin{pgfscope}%
\pgfpathrectangle{\pgfqpoint{0.750000in}{0.500000in}}{\pgfqpoint{4.650000in}{3.020000in}}%
\pgfusepath{clip}%
\pgfsetbuttcap%
\pgfsetmiterjoin%
\definecolor{currentfill}{rgb}{1.000000,0.000000,0.000000}%
\pgfsetfillcolor{currentfill}%
\pgfsetlinewidth{0.000000pt}%
\definecolor{currentstroke}{rgb}{0.000000,0.000000,0.000000}%
\pgfsetstrokecolor{currentstroke}%
\pgfsetstrokeopacity{0.000000}%
\pgfsetdash{}{0pt}%
\pgfpathmoveto{\pgfqpoint{3.090032in}{0.500000in}}%
\pgfpathlineto{\pgfqpoint{3.114311in}{0.500000in}}%
\pgfpathlineto{\pgfqpoint{3.114311in}{0.590636in}}%
\pgfpathlineto{\pgfqpoint{3.090032in}{0.590636in}}%
\pgfpathlineto{\pgfqpoint{3.090032in}{0.500000in}}%
\pgfpathclose%
\pgfusepath{fill}%
\end{pgfscope}%
\begin{pgfscope}%
\pgfpathrectangle{\pgfqpoint{0.750000in}{0.500000in}}{\pgfqpoint{4.650000in}{3.020000in}}%
\pgfusepath{clip}%
\pgfsetbuttcap%
\pgfsetmiterjoin%
\definecolor{currentfill}{rgb}{1.000000,0.000000,0.000000}%
\pgfsetfillcolor{currentfill}%
\pgfsetlinewidth{0.000000pt}%
\definecolor{currentstroke}{rgb}{0.000000,0.000000,0.000000}%
\pgfsetstrokecolor{currentstroke}%
\pgfsetstrokeopacity{0.000000}%
\pgfsetdash{}{0pt}%
\pgfpathmoveto{\pgfqpoint{3.114311in}{0.500000in}}%
\pgfpathlineto{\pgfqpoint{3.138590in}{0.500000in}}%
\pgfpathlineto{\pgfqpoint{3.138590in}{3.273469in}}%
\pgfpathlineto{\pgfqpoint{3.114311in}{3.273469in}}%
\pgfpathlineto{\pgfqpoint{3.114311in}{0.500000in}}%
\pgfpathclose%
\pgfusepath{fill}%
\end{pgfscope}%
\begin{pgfscope}%
\pgfpathrectangle{\pgfqpoint{0.750000in}{0.500000in}}{\pgfqpoint{4.650000in}{3.020000in}}%
\pgfusepath{clip}%
\pgfsetbuttcap%
\pgfsetmiterjoin%
\definecolor{currentfill}{rgb}{1.000000,0.000000,0.000000}%
\pgfsetfillcolor{currentfill}%
\pgfsetlinewidth{0.000000pt}%
\definecolor{currentstroke}{rgb}{0.000000,0.000000,0.000000}%
\pgfsetstrokecolor{currentstroke}%
\pgfsetstrokeopacity{0.000000}%
\pgfsetdash{}{0pt}%
\pgfpathmoveto{\pgfqpoint{3.138590in}{0.500000in}}%
\pgfpathlineto{\pgfqpoint{3.162868in}{0.500000in}}%
\pgfpathlineto{\pgfqpoint{3.162868in}{0.593657in}}%
\pgfpathlineto{\pgfqpoint{3.138590in}{0.593657in}}%
\pgfpathlineto{\pgfqpoint{3.138590in}{0.500000in}}%
\pgfpathclose%
\pgfusepath{fill}%
\end{pgfscope}%
\begin{pgfscope}%
\pgfpathrectangle{\pgfqpoint{0.750000in}{0.500000in}}{\pgfqpoint{4.650000in}{3.020000in}}%
\pgfusepath{clip}%
\pgfsetbuttcap%
\pgfsetmiterjoin%
\definecolor{currentfill}{rgb}{1.000000,0.000000,0.000000}%
\pgfsetfillcolor{currentfill}%
\pgfsetlinewidth{0.000000pt}%
\definecolor{currentstroke}{rgb}{0.000000,0.000000,0.000000}%
\pgfsetstrokecolor{currentstroke}%
\pgfsetstrokeopacity{0.000000}%
\pgfsetdash{}{0pt}%
\pgfpathmoveto{\pgfqpoint{3.162868in}{0.500000in}}%
\pgfpathlineto{\pgfqpoint{3.187147in}{0.500000in}}%
\pgfpathlineto{\pgfqpoint{3.187147in}{0.593657in}}%
\pgfpathlineto{\pgfqpoint{3.162868in}{0.593657in}}%
\pgfpathlineto{\pgfqpoint{3.162868in}{0.500000in}}%
\pgfpathclose%
\pgfusepath{fill}%
\end{pgfscope}%
\begin{pgfscope}%
\pgfpathrectangle{\pgfqpoint{0.750000in}{0.500000in}}{\pgfqpoint{4.650000in}{3.020000in}}%
\pgfusepath{clip}%
\pgfsetbuttcap%
\pgfsetmiterjoin%
\definecolor{currentfill}{rgb}{1.000000,0.000000,0.000000}%
\pgfsetfillcolor{currentfill}%
\pgfsetlinewidth{0.000000pt}%
\definecolor{currentstroke}{rgb}{0.000000,0.000000,0.000000}%
\pgfsetstrokecolor{currentstroke}%
\pgfsetstrokeopacity{0.000000}%
\pgfsetdash{}{0pt}%
\pgfpathmoveto{\pgfqpoint{3.187147in}{0.500000in}}%
\pgfpathlineto{\pgfqpoint{3.211425in}{0.500000in}}%
\pgfpathlineto{\pgfqpoint{3.211425in}{3.016667in}}%
\pgfpathlineto{\pgfqpoint{3.187147in}{3.016667in}}%
\pgfpathlineto{\pgfqpoint{3.187147in}{0.500000in}}%
\pgfpathclose%
\pgfusepath{fill}%
\end{pgfscope}%
\begin{pgfscope}%
\pgfpathrectangle{\pgfqpoint{0.750000in}{0.500000in}}{\pgfqpoint{4.650000in}{3.020000in}}%
\pgfusepath{clip}%
\pgfsetbuttcap%
\pgfsetmiterjoin%
\definecolor{currentfill}{rgb}{1.000000,0.000000,0.000000}%
\pgfsetfillcolor{currentfill}%
\pgfsetlinewidth{0.000000pt}%
\definecolor{currentstroke}{rgb}{0.000000,0.000000,0.000000}%
\pgfsetstrokecolor{currentstroke}%
\pgfsetstrokeopacity{0.000000}%
\pgfsetdash{}{0pt}%
\pgfpathmoveto{\pgfqpoint{3.211425in}{0.500000in}}%
\pgfpathlineto{\pgfqpoint{3.235704in}{0.500000in}}%
\pgfpathlineto{\pgfqpoint{3.235704in}{0.563445in}}%
\pgfpathlineto{\pgfqpoint{3.211425in}{0.563445in}}%
\pgfpathlineto{\pgfqpoint{3.211425in}{0.500000in}}%
\pgfpathclose%
\pgfusepath{fill}%
\end{pgfscope}%
\begin{pgfscope}%
\pgfpathrectangle{\pgfqpoint{0.750000in}{0.500000in}}{\pgfqpoint{4.650000in}{3.020000in}}%
\pgfusepath{clip}%
\pgfsetbuttcap%
\pgfsetmiterjoin%
\definecolor{currentfill}{rgb}{1.000000,0.000000,0.000000}%
\pgfsetfillcolor{currentfill}%
\pgfsetlinewidth{0.000000pt}%
\definecolor{currentstroke}{rgb}{0.000000,0.000000,0.000000}%
\pgfsetstrokecolor{currentstroke}%
\pgfsetstrokeopacity{0.000000}%
\pgfsetdash{}{0pt}%
\pgfpathmoveto{\pgfqpoint{3.235704in}{0.500000in}}%
\pgfpathlineto{\pgfqpoint{3.259982in}{0.500000in}}%
\pgfpathlineto{\pgfqpoint{3.259982in}{0.635954in}}%
\pgfpathlineto{\pgfqpoint{3.235704in}{0.635954in}}%
\pgfpathlineto{\pgfqpoint{3.235704in}{0.500000in}}%
\pgfpathclose%
\pgfusepath{fill}%
\end{pgfscope}%
\begin{pgfscope}%
\pgfpathrectangle{\pgfqpoint{0.750000in}{0.500000in}}{\pgfqpoint{4.650000in}{3.020000in}}%
\pgfusepath{clip}%
\pgfsetbuttcap%
\pgfsetmiterjoin%
\definecolor{currentfill}{rgb}{1.000000,0.000000,0.000000}%
\pgfsetfillcolor{currentfill}%
\pgfsetlinewidth{0.000000pt}%
\definecolor{currentstroke}{rgb}{0.000000,0.000000,0.000000}%
\pgfsetstrokecolor{currentstroke}%
\pgfsetstrokeopacity{0.000000}%
\pgfsetdash{}{0pt}%
\pgfpathmoveto{\pgfqpoint{3.259982in}{0.500000in}}%
\pgfpathlineto{\pgfqpoint{3.284261in}{0.500000in}}%
\pgfpathlineto{\pgfqpoint{3.284261in}{2.509104in}}%
\pgfpathlineto{\pgfqpoint{3.259982in}{2.509104in}}%
\pgfpathlineto{\pgfqpoint{3.259982in}{0.500000in}}%
\pgfpathclose%
\pgfusepath{fill}%
\end{pgfscope}%
\begin{pgfscope}%
\pgfpathrectangle{\pgfqpoint{0.750000in}{0.500000in}}{\pgfqpoint{4.650000in}{3.020000in}}%
\pgfusepath{clip}%
\pgfsetbuttcap%
\pgfsetmiterjoin%
\definecolor{currentfill}{rgb}{1.000000,0.000000,0.000000}%
\pgfsetfillcolor{currentfill}%
\pgfsetlinewidth{0.000000pt}%
\definecolor{currentstroke}{rgb}{0.000000,0.000000,0.000000}%
\pgfsetstrokecolor{currentstroke}%
\pgfsetstrokeopacity{0.000000}%
\pgfsetdash{}{0pt}%
\pgfpathmoveto{\pgfqpoint{3.284261in}{0.500000in}}%
\pgfpathlineto{\pgfqpoint{3.308540in}{0.500000in}}%
\pgfpathlineto{\pgfqpoint{3.308540in}{0.569488in}}%
\pgfpathlineto{\pgfqpoint{3.284261in}{0.569488in}}%
\pgfpathlineto{\pgfqpoint{3.284261in}{0.500000in}}%
\pgfpathclose%
\pgfusepath{fill}%
\end{pgfscope}%
\begin{pgfscope}%
\pgfpathrectangle{\pgfqpoint{0.750000in}{0.500000in}}{\pgfqpoint{4.650000in}{3.020000in}}%
\pgfusepath{clip}%
\pgfsetbuttcap%
\pgfsetmiterjoin%
\definecolor{currentfill}{rgb}{1.000000,0.000000,0.000000}%
\pgfsetfillcolor{currentfill}%
\pgfsetlinewidth{0.000000pt}%
\definecolor{currentstroke}{rgb}{0.000000,0.000000,0.000000}%
\pgfsetstrokecolor{currentstroke}%
\pgfsetstrokeopacity{0.000000}%
\pgfsetdash{}{0pt}%
\pgfpathmoveto{\pgfqpoint{3.308540in}{0.500000in}}%
\pgfpathlineto{\pgfqpoint{3.332818in}{0.500000in}}%
\pgfpathlineto{\pgfqpoint{3.332818in}{2.203962in}}%
\pgfpathlineto{\pgfqpoint{3.308540in}{2.203962in}}%
\pgfpathlineto{\pgfqpoint{3.308540in}{0.500000in}}%
\pgfpathclose%
\pgfusepath{fill}%
\end{pgfscope}%
\begin{pgfscope}%
\pgfpathrectangle{\pgfqpoint{0.750000in}{0.500000in}}{\pgfqpoint{4.650000in}{3.020000in}}%
\pgfusepath{clip}%
\pgfsetbuttcap%
\pgfsetmiterjoin%
\definecolor{currentfill}{rgb}{1.000000,0.000000,0.000000}%
\pgfsetfillcolor{currentfill}%
\pgfsetlinewidth{0.000000pt}%
\definecolor{currentstroke}{rgb}{0.000000,0.000000,0.000000}%
\pgfsetstrokecolor{currentstroke}%
\pgfsetstrokeopacity{0.000000}%
\pgfsetdash{}{0pt}%
\pgfpathmoveto{\pgfqpoint{3.332818in}{0.500000in}}%
\pgfpathlineto{\pgfqpoint{3.357097in}{0.500000in}}%
\pgfpathlineto{\pgfqpoint{3.357097in}{0.566467in}}%
\pgfpathlineto{\pgfqpoint{3.332818in}{0.566467in}}%
\pgfpathlineto{\pgfqpoint{3.332818in}{0.500000in}}%
\pgfpathclose%
\pgfusepath{fill}%
\end{pgfscope}%
\begin{pgfscope}%
\pgfpathrectangle{\pgfqpoint{0.750000in}{0.500000in}}{\pgfqpoint{4.650000in}{3.020000in}}%
\pgfusepath{clip}%
\pgfsetbuttcap%
\pgfsetmiterjoin%
\definecolor{currentfill}{rgb}{1.000000,0.000000,0.000000}%
\pgfsetfillcolor{currentfill}%
\pgfsetlinewidth{0.000000pt}%
\definecolor{currentstroke}{rgb}{0.000000,0.000000,0.000000}%
\pgfsetstrokecolor{currentstroke}%
\pgfsetstrokeopacity{0.000000}%
\pgfsetdash{}{0pt}%
\pgfpathmoveto{\pgfqpoint{3.357097in}{0.500000in}}%
\pgfpathlineto{\pgfqpoint{3.381375in}{0.500000in}}%
\pgfpathlineto{\pgfqpoint{3.381375in}{0.563445in}}%
\pgfpathlineto{\pgfqpoint{3.357097in}{0.563445in}}%
\pgfpathlineto{\pgfqpoint{3.357097in}{0.500000in}}%
\pgfpathclose%
\pgfusepath{fill}%
\end{pgfscope}%
\begin{pgfscope}%
\pgfpathrectangle{\pgfqpoint{0.750000in}{0.500000in}}{\pgfqpoint{4.650000in}{3.020000in}}%
\pgfusepath{clip}%
\pgfsetbuttcap%
\pgfsetmiterjoin%
\definecolor{currentfill}{rgb}{1.000000,0.000000,0.000000}%
\pgfsetfillcolor{currentfill}%
\pgfsetlinewidth{0.000000pt}%
\definecolor{currentstroke}{rgb}{0.000000,0.000000,0.000000}%
\pgfsetstrokecolor{currentstroke}%
\pgfsetstrokeopacity{0.000000}%
\pgfsetdash{}{0pt}%
\pgfpathmoveto{\pgfqpoint{3.381375in}{0.500000in}}%
\pgfpathlineto{\pgfqpoint{3.405654in}{0.500000in}}%
\pgfpathlineto{\pgfqpoint{3.405654in}{1.711505in}}%
\pgfpathlineto{\pgfqpoint{3.381375in}{1.711505in}}%
\pgfpathlineto{\pgfqpoint{3.381375in}{0.500000in}}%
\pgfpathclose%
\pgfusepath{fill}%
\end{pgfscope}%
\begin{pgfscope}%
\pgfpathrectangle{\pgfqpoint{0.750000in}{0.500000in}}{\pgfqpoint{4.650000in}{3.020000in}}%
\pgfusepath{clip}%
\pgfsetbuttcap%
\pgfsetmiterjoin%
\definecolor{currentfill}{rgb}{1.000000,0.000000,0.000000}%
\pgfsetfillcolor{currentfill}%
\pgfsetlinewidth{0.000000pt}%
\definecolor{currentstroke}{rgb}{0.000000,0.000000,0.000000}%
\pgfsetstrokecolor{currentstroke}%
\pgfsetstrokeopacity{0.000000}%
\pgfsetdash{}{0pt}%
\pgfpathmoveto{\pgfqpoint{3.405654in}{0.500000in}}%
\pgfpathlineto{\pgfqpoint{3.429932in}{0.500000in}}%
\pgfpathlineto{\pgfqpoint{3.429932in}{0.518127in}}%
\pgfpathlineto{\pgfqpoint{3.405654in}{0.518127in}}%
\pgfpathlineto{\pgfqpoint{3.405654in}{0.500000in}}%
\pgfpathclose%
\pgfusepath{fill}%
\end{pgfscope}%
\begin{pgfscope}%
\pgfpathrectangle{\pgfqpoint{0.750000in}{0.500000in}}{\pgfqpoint{4.650000in}{3.020000in}}%
\pgfusepath{clip}%
\pgfsetbuttcap%
\pgfsetmiterjoin%
\definecolor{currentfill}{rgb}{1.000000,0.000000,0.000000}%
\pgfsetfillcolor{currentfill}%
\pgfsetlinewidth{0.000000pt}%
\definecolor{currentstroke}{rgb}{0.000000,0.000000,0.000000}%
\pgfsetstrokecolor{currentstroke}%
\pgfsetstrokeopacity{0.000000}%
\pgfsetdash{}{0pt}%
\pgfpathmoveto{\pgfqpoint{3.429932in}{0.500000in}}%
\pgfpathlineto{\pgfqpoint{3.454211in}{0.500000in}}%
\pgfpathlineto{\pgfqpoint{3.454211in}{0.545318in}}%
\pgfpathlineto{\pgfqpoint{3.429932in}{0.545318in}}%
\pgfpathlineto{\pgfqpoint{3.429932in}{0.500000in}}%
\pgfpathclose%
\pgfusepath{fill}%
\end{pgfscope}%
\begin{pgfscope}%
\pgfpathrectangle{\pgfqpoint{0.750000in}{0.500000in}}{\pgfqpoint{4.650000in}{3.020000in}}%
\pgfusepath{clip}%
\pgfsetbuttcap%
\pgfsetmiterjoin%
\definecolor{currentfill}{rgb}{1.000000,0.000000,0.000000}%
\pgfsetfillcolor{currentfill}%
\pgfsetlinewidth{0.000000pt}%
\definecolor{currentstroke}{rgb}{0.000000,0.000000,0.000000}%
\pgfsetstrokecolor{currentstroke}%
\pgfsetstrokeopacity{0.000000}%
\pgfsetdash{}{0pt}%
\pgfpathmoveto{\pgfqpoint{3.454211in}{0.500000in}}%
\pgfpathlineto{\pgfqpoint{3.478490in}{0.500000in}}%
\pgfpathlineto{\pgfqpoint{3.478490in}{1.469808in}}%
\pgfpathlineto{\pgfqpoint{3.454211in}{1.469808in}}%
\pgfpathlineto{\pgfqpoint{3.454211in}{0.500000in}}%
\pgfpathclose%
\pgfusepath{fill}%
\end{pgfscope}%
\begin{pgfscope}%
\pgfpathrectangle{\pgfqpoint{0.750000in}{0.500000in}}{\pgfqpoint{4.650000in}{3.020000in}}%
\pgfusepath{clip}%
\pgfsetbuttcap%
\pgfsetmiterjoin%
\definecolor{currentfill}{rgb}{1.000000,0.000000,0.000000}%
\pgfsetfillcolor{currentfill}%
\pgfsetlinewidth{0.000000pt}%
\definecolor{currentstroke}{rgb}{0.000000,0.000000,0.000000}%
\pgfsetstrokecolor{currentstroke}%
\pgfsetstrokeopacity{0.000000}%
\pgfsetdash{}{0pt}%
\pgfpathmoveto{\pgfqpoint{3.478490in}{0.500000in}}%
\pgfpathlineto{\pgfqpoint{3.502768in}{0.500000in}}%
\pgfpathlineto{\pgfqpoint{3.502768in}{0.530212in}}%
\pgfpathlineto{\pgfqpoint{3.478490in}{0.530212in}}%
\pgfpathlineto{\pgfqpoint{3.478490in}{0.500000in}}%
\pgfpathclose%
\pgfusepath{fill}%
\end{pgfscope}%
\begin{pgfscope}%
\pgfpathrectangle{\pgfqpoint{0.750000in}{0.500000in}}{\pgfqpoint{4.650000in}{3.020000in}}%
\pgfusepath{clip}%
\pgfsetbuttcap%
\pgfsetmiterjoin%
\definecolor{currentfill}{rgb}{1.000000,0.000000,0.000000}%
\pgfsetfillcolor{currentfill}%
\pgfsetlinewidth{0.000000pt}%
\definecolor{currentstroke}{rgb}{0.000000,0.000000,0.000000}%
\pgfsetstrokecolor{currentstroke}%
\pgfsetstrokeopacity{0.000000}%
\pgfsetdash{}{0pt}%
\pgfpathmoveto{\pgfqpoint{3.502768in}{0.500000in}}%
\pgfpathlineto{\pgfqpoint{3.527047in}{0.500000in}}%
\pgfpathlineto{\pgfqpoint{3.527047in}{1.110284in}}%
\pgfpathlineto{\pgfqpoint{3.502768in}{1.110284in}}%
\pgfpathlineto{\pgfqpoint{3.502768in}{0.500000in}}%
\pgfpathclose%
\pgfusepath{fill}%
\end{pgfscope}%
\begin{pgfscope}%
\pgfpathrectangle{\pgfqpoint{0.750000in}{0.500000in}}{\pgfqpoint{4.650000in}{3.020000in}}%
\pgfusepath{clip}%
\pgfsetbuttcap%
\pgfsetmiterjoin%
\definecolor{currentfill}{rgb}{1.000000,0.000000,0.000000}%
\pgfsetfillcolor{currentfill}%
\pgfsetlinewidth{0.000000pt}%
\definecolor{currentstroke}{rgb}{0.000000,0.000000,0.000000}%
\pgfsetstrokecolor{currentstroke}%
\pgfsetstrokeopacity{0.000000}%
\pgfsetdash{}{0pt}%
\pgfpathmoveto{\pgfqpoint{3.527047in}{0.500000in}}%
\pgfpathlineto{\pgfqpoint{3.551325in}{0.500000in}}%
\pgfpathlineto{\pgfqpoint{3.551325in}{0.539276in}}%
\pgfpathlineto{\pgfqpoint{3.527047in}{0.539276in}}%
\pgfpathlineto{\pgfqpoint{3.527047in}{0.500000in}}%
\pgfpathclose%
\pgfusepath{fill}%
\end{pgfscope}%
\begin{pgfscope}%
\pgfpathrectangle{\pgfqpoint{0.750000in}{0.500000in}}{\pgfqpoint{4.650000in}{3.020000in}}%
\pgfusepath{clip}%
\pgfsetbuttcap%
\pgfsetmiterjoin%
\definecolor{currentfill}{rgb}{1.000000,0.000000,0.000000}%
\pgfsetfillcolor{currentfill}%
\pgfsetlinewidth{0.000000pt}%
\definecolor{currentstroke}{rgb}{0.000000,0.000000,0.000000}%
\pgfsetstrokecolor{currentstroke}%
\pgfsetstrokeopacity{0.000000}%
\pgfsetdash{}{0pt}%
\pgfpathmoveto{\pgfqpoint{3.551325in}{0.500000in}}%
\pgfpathlineto{\pgfqpoint{3.575604in}{0.500000in}}%
\pgfpathlineto{\pgfqpoint{3.575604in}{0.518127in}}%
\pgfpathlineto{\pgfqpoint{3.551325in}{0.518127in}}%
\pgfpathlineto{\pgfqpoint{3.551325in}{0.500000in}}%
\pgfpathclose%
\pgfusepath{fill}%
\end{pgfscope}%
\begin{pgfscope}%
\pgfpathrectangle{\pgfqpoint{0.750000in}{0.500000in}}{\pgfqpoint{4.650000in}{3.020000in}}%
\pgfusepath{clip}%
\pgfsetbuttcap%
\pgfsetmiterjoin%
\definecolor{currentfill}{rgb}{1.000000,0.000000,0.000000}%
\pgfsetfillcolor{currentfill}%
\pgfsetlinewidth{0.000000pt}%
\definecolor{currentstroke}{rgb}{0.000000,0.000000,0.000000}%
\pgfsetstrokecolor{currentstroke}%
\pgfsetstrokeopacity{0.000000}%
\pgfsetdash{}{0pt}%
\pgfpathmoveto{\pgfqpoint{3.575604in}{0.500000in}}%
\pgfpathlineto{\pgfqpoint{3.599883in}{0.500000in}}%
\pgfpathlineto{\pgfqpoint{3.599883in}{0.856503in}}%
\pgfpathlineto{\pgfqpoint{3.575604in}{0.856503in}}%
\pgfpathlineto{\pgfqpoint{3.575604in}{0.500000in}}%
\pgfpathclose%
\pgfusepath{fill}%
\end{pgfscope}%
\begin{pgfscope}%
\pgfpathrectangle{\pgfqpoint{0.750000in}{0.500000in}}{\pgfqpoint{4.650000in}{3.020000in}}%
\pgfusepath{clip}%
\pgfsetbuttcap%
\pgfsetmiterjoin%
\definecolor{currentfill}{rgb}{1.000000,0.000000,0.000000}%
\pgfsetfillcolor{currentfill}%
\pgfsetlinewidth{0.000000pt}%
\definecolor{currentstroke}{rgb}{0.000000,0.000000,0.000000}%
\pgfsetstrokecolor{currentstroke}%
\pgfsetstrokeopacity{0.000000}%
\pgfsetdash{}{0pt}%
\pgfpathmoveto{\pgfqpoint{3.599883in}{0.500000in}}%
\pgfpathlineto{\pgfqpoint{3.624161in}{0.500000in}}%
\pgfpathlineto{\pgfqpoint{3.624161in}{0.503021in}}%
\pgfpathlineto{\pgfqpoint{3.599883in}{0.503021in}}%
\pgfpathlineto{\pgfqpoint{3.599883in}{0.500000in}}%
\pgfpathclose%
\pgfusepath{fill}%
\end{pgfscope}%
\begin{pgfscope}%
\pgfpathrectangle{\pgfqpoint{0.750000in}{0.500000in}}{\pgfqpoint{4.650000in}{3.020000in}}%
\pgfusepath{clip}%
\pgfsetbuttcap%
\pgfsetmiterjoin%
\definecolor{currentfill}{rgb}{1.000000,0.000000,0.000000}%
\pgfsetfillcolor{currentfill}%
\pgfsetlinewidth{0.000000pt}%
\definecolor{currentstroke}{rgb}{0.000000,0.000000,0.000000}%
\pgfsetstrokecolor{currentstroke}%
\pgfsetstrokeopacity{0.000000}%
\pgfsetdash{}{0pt}%
\pgfpathmoveto{\pgfqpoint{3.624161in}{0.500000in}}%
\pgfpathlineto{\pgfqpoint{3.648440in}{0.500000in}}%
\pgfpathlineto{\pgfqpoint{3.648440in}{0.506042in}}%
\pgfpathlineto{\pgfqpoint{3.624161in}{0.506042in}}%
\pgfpathlineto{\pgfqpoint{3.624161in}{0.500000in}}%
\pgfpathclose%
\pgfusepath{fill}%
\end{pgfscope}%
\begin{pgfscope}%
\pgfpathrectangle{\pgfqpoint{0.750000in}{0.500000in}}{\pgfqpoint{4.650000in}{3.020000in}}%
\pgfusepath{clip}%
\pgfsetbuttcap%
\pgfsetmiterjoin%
\definecolor{currentfill}{rgb}{1.000000,0.000000,0.000000}%
\pgfsetfillcolor{currentfill}%
\pgfsetlinewidth{0.000000pt}%
\definecolor{currentstroke}{rgb}{0.000000,0.000000,0.000000}%
\pgfsetstrokecolor{currentstroke}%
\pgfsetstrokeopacity{0.000000}%
\pgfsetdash{}{0pt}%
\pgfpathmoveto{\pgfqpoint{3.648440in}{0.500000in}}%
\pgfpathlineto{\pgfqpoint{3.672718in}{0.500000in}}%
\pgfpathlineto{\pgfqpoint{3.672718in}{0.723569in}}%
\pgfpathlineto{\pgfqpoint{3.648440in}{0.723569in}}%
\pgfpathlineto{\pgfqpoint{3.648440in}{0.500000in}}%
\pgfpathclose%
\pgfusepath{fill}%
\end{pgfscope}%
\begin{pgfscope}%
\pgfpathrectangle{\pgfqpoint{0.750000in}{0.500000in}}{\pgfqpoint{4.650000in}{3.020000in}}%
\pgfusepath{clip}%
\pgfsetbuttcap%
\pgfsetmiterjoin%
\definecolor{currentfill}{rgb}{1.000000,0.000000,0.000000}%
\pgfsetfillcolor{currentfill}%
\pgfsetlinewidth{0.000000pt}%
\definecolor{currentstroke}{rgb}{0.000000,0.000000,0.000000}%
\pgfsetstrokecolor{currentstroke}%
\pgfsetstrokeopacity{0.000000}%
\pgfsetdash{}{0pt}%
\pgfpathmoveto{\pgfqpoint{3.672718in}{0.500000in}}%
\pgfpathlineto{\pgfqpoint{3.696997in}{0.500000in}}%
\pgfpathlineto{\pgfqpoint{3.696997in}{0.509064in}}%
\pgfpathlineto{\pgfqpoint{3.672718in}{0.509064in}}%
\pgfpathlineto{\pgfqpoint{3.672718in}{0.500000in}}%
\pgfpathclose%
\pgfusepath{fill}%
\end{pgfscope}%
\begin{pgfscope}%
\pgfpathrectangle{\pgfqpoint{0.750000in}{0.500000in}}{\pgfqpoint{4.650000in}{3.020000in}}%
\pgfusepath{clip}%
\pgfsetbuttcap%
\pgfsetmiterjoin%
\definecolor{currentfill}{rgb}{1.000000,0.000000,0.000000}%
\pgfsetfillcolor{currentfill}%
\pgfsetlinewidth{0.000000pt}%
\definecolor{currentstroke}{rgb}{0.000000,0.000000,0.000000}%
\pgfsetstrokecolor{currentstroke}%
\pgfsetstrokeopacity{0.000000}%
\pgfsetdash{}{0pt}%
\pgfpathmoveto{\pgfqpoint{3.696997in}{0.500000in}}%
\pgfpathlineto{\pgfqpoint{3.721275in}{0.500000in}}%
\pgfpathlineto{\pgfqpoint{3.721275in}{0.500000in}}%
\pgfpathlineto{\pgfqpoint{3.696997in}{0.500000in}}%
\pgfpathlineto{\pgfqpoint{3.696997in}{0.500000in}}%
\pgfpathclose%
\pgfusepath{fill}%
\end{pgfscope}%
\begin{pgfscope}%
\pgfpathrectangle{\pgfqpoint{0.750000in}{0.500000in}}{\pgfqpoint{4.650000in}{3.020000in}}%
\pgfusepath{clip}%
\pgfsetbuttcap%
\pgfsetmiterjoin%
\definecolor{currentfill}{rgb}{1.000000,0.000000,0.000000}%
\pgfsetfillcolor{currentfill}%
\pgfsetlinewidth{0.000000pt}%
\definecolor{currentstroke}{rgb}{0.000000,0.000000,0.000000}%
\pgfsetstrokecolor{currentstroke}%
\pgfsetstrokeopacity{0.000000}%
\pgfsetdash{}{0pt}%
\pgfpathmoveto{\pgfqpoint{3.721275in}{0.500000in}}%
\pgfpathlineto{\pgfqpoint{3.745554in}{0.500000in}}%
\pgfpathlineto{\pgfqpoint{3.745554in}{0.605742in}}%
\pgfpathlineto{\pgfqpoint{3.721275in}{0.605742in}}%
\pgfpathlineto{\pgfqpoint{3.721275in}{0.500000in}}%
\pgfpathclose%
\pgfusepath{fill}%
\end{pgfscope}%
\begin{pgfscope}%
\pgfpathrectangle{\pgfqpoint{0.750000in}{0.500000in}}{\pgfqpoint{4.650000in}{3.020000in}}%
\pgfusepath{clip}%
\pgfsetbuttcap%
\pgfsetmiterjoin%
\definecolor{currentfill}{rgb}{1.000000,0.000000,0.000000}%
\pgfsetfillcolor{currentfill}%
\pgfsetlinewidth{0.000000pt}%
\definecolor{currentstroke}{rgb}{0.000000,0.000000,0.000000}%
\pgfsetstrokecolor{currentstroke}%
\pgfsetstrokeopacity{0.000000}%
\pgfsetdash{}{0pt}%
\pgfpathmoveto{\pgfqpoint{3.745554in}{0.500000in}}%
\pgfpathlineto{\pgfqpoint{3.769833in}{0.500000in}}%
\pgfpathlineto{\pgfqpoint{3.769833in}{0.500000in}}%
\pgfpathlineto{\pgfqpoint{3.745554in}{0.500000in}}%
\pgfpathlineto{\pgfqpoint{3.745554in}{0.500000in}}%
\pgfpathclose%
\pgfusepath{fill}%
\end{pgfscope}%
\begin{pgfscope}%
\pgfpathrectangle{\pgfqpoint{0.750000in}{0.500000in}}{\pgfqpoint{4.650000in}{3.020000in}}%
\pgfusepath{clip}%
\pgfsetbuttcap%
\pgfsetmiterjoin%
\definecolor{currentfill}{rgb}{1.000000,0.000000,0.000000}%
\pgfsetfillcolor{currentfill}%
\pgfsetlinewidth{0.000000pt}%
\definecolor{currentstroke}{rgb}{0.000000,0.000000,0.000000}%
\pgfsetstrokecolor{currentstroke}%
\pgfsetstrokeopacity{0.000000}%
\pgfsetdash{}{0pt}%
\pgfpathmoveto{\pgfqpoint{3.769833in}{0.500000in}}%
\pgfpathlineto{\pgfqpoint{3.794111in}{0.500000in}}%
\pgfpathlineto{\pgfqpoint{3.794111in}{0.557403in}}%
\pgfpathlineto{\pgfqpoint{3.769833in}{0.557403in}}%
\pgfpathlineto{\pgfqpoint{3.769833in}{0.500000in}}%
\pgfpathclose%
\pgfusepath{fill}%
\end{pgfscope}%
\begin{pgfscope}%
\pgfpathrectangle{\pgfqpoint{0.750000in}{0.500000in}}{\pgfqpoint{4.650000in}{3.020000in}}%
\pgfusepath{clip}%
\pgfsetbuttcap%
\pgfsetmiterjoin%
\definecolor{currentfill}{rgb}{1.000000,0.000000,0.000000}%
\pgfsetfillcolor{currentfill}%
\pgfsetlinewidth{0.000000pt}%
\definecolor{currentstroke}{rgb}{0.000000,0.000000,0.000000}%
\pgfsetstrokecolor{currentstroke}%
\pgfsetstrokeopacity{0.000000}%
\pgfsetdash{}{0pt}%
\pgfpathmoveto{\pgfqpoint{3.794111in}{0.500000in}}%
\pgfpathlineto{\pgfqpoint{3.818390in}{0.500000in}}%
\pgfpathlineto{\pgfqpoint{3.818390in}{0.500000in}}%
\pgfpathlineto{\pgfqpoint{3.794111in}{0.500000in}}%
\pgfpathlineto{\pgfqpoint{3.794111in}{0.500000in}}%
\pgfpathclose%
\pgfusepath{fill}%
\end{pgfscope}%
\begin{pgfscope}%
\pgfpathrectangle{\pgfqpoint{0.750000in}{0.500000in}}{\pgfqpoint{4.650000in}{3.020000in}}%
\pgfusepath{clip}%
\pgfsetbuttcap%
\pgfsetmiterjoin%
\definecolor{currentfill}{rgb}{1.000000,0.000000,0.000000}%
\pgfsetfillcolor{currentfill}%
\pgfsetlinewidth{0.000000pt}%
\definecolor{currentstroke}{rgb}{0.000000,0.000000,0.000000}%
\pgfsetstrokecolor{currentstroke}%
\pgfsetstrokeopacity{0.000000}%
\pgfsetdash{}{0pt}%
\pgfpathmoveto{\pgfqpoint{3.818390in}{0.500000in}}%
\pgfpathlineto{\pgfqpoint{3.842668in}{0.500000in}}%
\pgfpathlineto{\pgfqpoint{3.842668in}{0.503021in}}%
\pgfpathlineto{\pgfqpoint{3.818390in}{0.503021in}}%
\pgfpathlineto{\pgfqpoint{3.818390in}{0.500000in}}%
\pgfpathclose%
\pgfusepath{fill}%
\end{pgfscope}%
\begin{pgfscope}%
\pgfpathrectangle{\pgfqpoint{0.750000in}{0.500000in}}{\pgfqpoint{4.650000in}{3.020000in}}%
\pgfusepath{clip}%
\pgfsetbuttcap%
\pgfsetmiterjoin%
\definecolor{currentfill}{rgb}{1.000000,0.000000,0.000000}%
\pgfsetfillcolor{currentfill}%
\pgfsetlinewidth{0.000000pt}%
\definecolor{currentstroke}{rgb}{0.000000,0.000000,0.000000}%
\pgfsetstrokecolor{currentstroke}%
\pgfsetstrokeopacity{0.000000}%
\pgfsetdash{}{0pt}%
\pgfpathmoveto{\pgfqpoint{3.842668in}{0.500000in}}%
\pgfpathlineto{\pgfqpoint{3.866947in}{0.500000in}}%
\pgfpathlineto{\pgfqpoint{3.866947in}{0.512085in}}%
\pgfpathlineto{\pgfqpoint{3.842668in}{0.512085in}}%
\pgfpathlineto{\pgfqpoint{3.842668in}{0.500000in}}%
\pgfpathclose%
\pgfusepath{fill}%
\end{pgfscope}%
\begin{pgfscope}%
\pgfpathrectangle{\pgfqpoint{0.750000in}{0.500000in}}{\pgfqpoint{4.650000in}{3.020000in}}%
\pgfusepath{clip}%
\pgfsetbuttcap%
\pgfsetmiterjoin%
\definecolor{currentfill}{rgb}{1.000000,0.000000,0.000000}%
\pgfsetfillcolor{currentfill}%
\pgfsetlinewidth{0.000000pt}%
\definecolor{currentstroke}{rgb}{0.000000,0.000000,0.000000}%
\pgfsetstrokecolor{currentstroke}%
\pgfsetstrokeopacity{0.000000}%
\pgfsetdash{}{0pt}%
\pgfpathmoveto{\pgfqpoint{3.866947in}{0.500000in}}%
\pgfpathlineto{\pgfqpoint{3.891226in}{0.500000in}}%
\pgfpathlineto{\pgfqpoint{3.891226in}{0.500000in}}%
\pgfpathlineto{\pgfqpoint{3.866947in}{0.500000in}}%
\pgfpathlineto{\pgfqpoint{3.866947in}{0.500000in}}%
\pgfpathclose%
\pgfusepath{fill}%
\end{pgfscope}%
\begin{pgfscope}%
\pgfpathrectangle{\pgfqpoint{0.750000in}{0.500000in}}{\pgfqpoint{4.650000in}{3.020000in}}%
\pgfusepath{clip}%
\pgfsetbuttcap%
\pgfsetmiterjoin%
\definecolor{currentfill}{rgb}{1.000000,0.000000,0.000000}%
\pgfsetfillcolor{currentfill}%
\pgfsetlinewidth{0.000000pt}%
\definecolor{currentstroke}{rgb}{0.000000,0.000000,0.000000}%
\pgfsetstrokecolor{currentstroke}%
\pgfsetstrokeopacity{0.000000}%
\pgfsetdash{}{0pt}%
\pgfpathmoveto{\pgfqpoint{3.891226in}{0.500000in}}%
\pgfpathlineto{\pgfqpoint{3.915504in}{0.500000in}}%
\pgfpathlineto{\pgfqpoint{3.915504in}{0.500000in}}%
\pgfpathlineto{\pgfqpoint{3.891226in}{0.500000in}}%
\pgfpathlineto{\pgfqpoint{3.891226in}{0.500000in}}%
\pgfpathclose%
\pgfusepath{fill}%
\end{pgfscope}%
\begin{pgfscope}%
\pgfpathrectangle{\pgfqpoint{0.750000in}{0.500000in}}{\pgfqpoint{4.650000in}{3.020000in}}%
\pgfusepath{clip}%
\pgfsetbuttcap%
\pgfsetmiterjoin%
\definecolor{currentfill}{rgb}{1.000000,0.000000,0.000000}%
\pgfsetfillcolor{currentfill}%
\pgfsetlinewidth{0.000000pt}%
\definecolor{currentstroke}{rgb}{0.000000,0.000000,0.000000}%
\pgfsetstrokecolor{currentstroke}%
\pgfsetstrokeopacity{0.000000}%
\pgfsetdash{}{0pt}%
\pgfpathmoveto{\pgfqpoint{3.915504in}{0.500000in}}%
\pgfpathlineto{\pgfqpoint{3.939783in}{0.500000in}}%
\pgfpathlineto{\pgfqpoint{3.939783in}{0.509064in}}%
\pgfpathlineto{\pgfqpoint{3.915504in}{0.509064in}}%
\pgfpathlineto{\pgfqpoint{3.915504in}{0.500000in}}%
\pgfpathclose%
\pgfusepath{fill}%
\end{pgfscope}%
\begin{pgfscope}%
\pgfpathrectangle{\pgfqpoint{0.750000in}{0.500000in}}{\pgfqpoint{4.650000in}{3.020000in}}%
\pgfusepath{clip}%
\pgfsetbuttcap%
\pgfsetmiterjoin%
\definecolor{currentfill}{rgb}{1.000000,0.000000,0.000000}%
\pgfsetfillcolor{currentfill}%
\pgfsetlinewidth{0.000000pt}%
\definecolor{currentstroke}{rgb}{0.000000,0.000000,0.000000}%
\pgfsetstrokecolor{currentstroke}%
\pgfsetstrokeopacity{0.000000}%
\pgfsetdash{}{0pt}%
\pgfpathmoveto{\pgfqpoint{3.939783in}{0.500000in}}%
\pgfpathlineto{\pgfqpoint{3.964061in}{0.500000in}}%
\pgfpathlineto{\pgfqpoint{3.964061in}{0.500000in}}%
\pgfpathlineto{\pgfqpoint{3.939783in}{0.500000in}}%
\pgfpathlineto{\pgfqpoint{3.939783in}{0.500000in}}%
\pgfpathclose%
\pgfusepath{fill}%
\end{pgfscope}%
\begin{pgfscope}%
\pgfpathrectangle{\pgfqpoint{0.750000in}{0.500000in}}{\pgfqpoint{4.650000in}{3.020000in}}%
\pgfusepath{clip}%
\pgfsetbuttcap%
\pgfsetmiterjoin%
\definecolor{currentfill}{rgb}{1.000000,0.000000,0.000000}%
\pgfsetfillcolor{currentfill}%
\pgfsetlinewidth{0.000000pt}%
\definecolor{currentstroke}{rgb}{0.000000,0.000000,0.000000}%
\pgfsetstrokecolor{currentstroke}%
\pgfsetstrokeopacity{0.000000}%
\pgfsetdash{}{0pt}%
\pgfpathmoveto{\pgfqpoint{3.964061in}{0.500000in}}%
\pgfpathlineto{\pgfqpoint{3.988340in}{0.500000in}}%
\pgfpathlineto{\pgfqpoint{3.988340in}{0.500000in}}%
\pgfpathlineto{\pgfqpoint{3.964061in}{0.500000in}}%
\pgfpathlineto{\pgfqpoint{3.964061in}{0.500000in}}%
\pgfpathclose%
\pgfusepath{fill}%
\end{pgfscope}%
\begin{pgfscope}%
\pgfpathrectangle{\pgfqpoint{0.750000in}{0.500000in}}{\pgfqpoint{4.650000in}{3.020000in}}%
\pgfusepath{clip}%
\pgfsetbuttcap%
\pgfsetmiterjoin%
\definecolor{currentfill}{rgb}{1.000000,0.000000,0.000000}%
\pgfsetfillcolor{currentfill}%
\pgfsetlinewidth{0.000000pt}%
\definecolor{currentstroke}{rgb}{0.000000,0.000000,0.000000}%
\pgfsetstrokecolor{currentstroke}%
\pgfsetstrokeopacity{0.000000}%
\pgfsetdash{}{0pt}%
\pgfpathmoveto{\pgfqpoint{3.988340in}{0.500000in}}%
\pgfpathlineto{\pgfqpoint{4.012618in}{0.500000in}}%
\pgfpathlineto{\pgfqpoint{4.012618in}{0.503021in}}%
\pgfpathlineto{\pgfqpoint{3.988340in}{0.503021in}}%
\pgfpathlineto{\pgfqpoint{3.988340in}{0.500000in}}%
\pgfpathclose%
\pgfusepath{fill}%
\end{pgfscope}%
\begin{pgfscope}%
\pgfpathrectangle{\pgfqpoint{0.750000in}{0.500000in}}{\pgfqpoint{4.650000in}{3.020000in}}%
\pgfusepath{clip}%
\pgfsetbuttcap%
\pgfsetmiterjoin%
\definecolor{currentfill}{rgb}{1.000000,0.000000,0.000000}%
\pgfsetfillcolor{currentfill}%
\pgfsetlinewidth{0.000000pt}%
\definecolor{currentstroke}{rgb}{0.000000,0.000000,0.000000}%
\pgfsetstrokecolor{currentstroke}%
\pgfsetstrokeopacity{0.000000}%
\pgfsetdash{}{0pt}%
\pgfpathmoveto{\pgfqpoint{4.012618in}{0.500000in}}%
\pgfpathlineto{\pgfqpoint{4.036897in}{0.500000in}}%
\pgfpathlineto{\pgfqpoint{4.036897in}{0.503021in}}%
\pgfpathlineto{\pgfqpoint{4.012618in}{0.503021in}}%
\pgfpathlineto{\pgfqpoint{4.012618in}{0.500000in}}%
\pgfpathclose%
\pgfusepath{fill}%
\end{pgfscope}%
\begin{pgfscope}%
\pgfpathrectangle{\pgfqpoint{0.750000in}{0.500000in}}{\pgfqpoint{4.650000in}{3.020000in}}%
\pgfusepath{clip}%
\pgfsetbuttcap%
\pgfsetmiterjoin%
\definecolor{currentfill}{rgb}{1.000000,0.000000,0.000000}%
\pgfsetfillcolor{currentfill}%
\pgfsetlinewidth{0.000000pt}%
\definecolor{currentstroke}{rgb}{0.000000,0.000000,0.000000}%
\pgfsetstrokecolor{currentstroke}%
\pgfsetstrokeopacity{0.000000}%
\pgfsetdash{}{0pt}%
\pgfpathmoveto{\pgfqpoint{4.036897in}{0.500000in}}%
\pgfpathlineto{\pgfqpoint{4.061176in}{0.500000in}}%
\pgfpathlineto{\pgfqpoint{4.061176in}{0.503021in}}%
\pgfpathlineto{\pgfqpoint{4.036897in}{0.503021in}}%
\pgfpathlineto{\pgfqpoint{4.036897in}{0.500000in}}%
\pgfpathclose%
\pgfusepath{fill}%
\end{pgfscope}%
\begin{pgfscope}%
\pgfpathrectangle{\pgfqpoint{0.750000in}{0.500000in}}{\pgfqpoint{4.650000in}{3.020000in}}%
\pgfusepath{clip}%
\pgfsetbuttcap%
\pgfsetmiterjoin%
\definecolor{currentfill}{rgb}{1.000000,0.000000,0.000000}%
\pgfsetfillcolor{currentfill}%
\pgfsetlinewidth{0.000000pt}%
\definecolor{currentstroke}{rgb}{0.000000,0.000000,0.000000}%
\pgfsetstrokecolor{currentstroke}%
\pgfsetstrokeopacity{0.000000}%
\pgfsetdash{}{0pt}%
\pgfpathmoveto{\pgfqpoint{4.061176in}{0.500000in}}%
\pgfpathlineto{\pgfqpoint{4.085454in}{0.500000in}}%
\pgfpathlineto{\pgfqpoint{4.085454in}{0.503021in}}%
\pgfpathlineto{\pgfqpoint{4.061176in}{0.503021in}}%
\pgfpathlineto{\pgfqpoint{4.061176in}{0.500000in}}%
\pgfpathclose%
\pgfusepath{fill}%
\end{pgfscope}%
\begin{pgfscope}%
\pgfpathrectangle{\pgfqpoint{0.750000in}{0.500000in}}{\pgfqpoint{4.650000in}{3.020000in}}%
\pgfusepath{clip}%
\pgfsetbuttcap%
\pgfsetmiterjoin%
\definecolor{currentfill}{rgb}{1.000000,0.000000,0.000000}%
\pgfsetfillcolor{currentfill}%
\pgfsetlinewidth{0.000000pt}%
\definecolor{currentstroke}{rgb}{0.000000,0.000000,0.000000}%
\pgfsetstrokecolor{currentstroke}%
\pgfsetstrokeopacity{0.000000}%
\pgfsetdash{}{0pt}%
\pgfpathmoveto{\pgfqpoint{4.085454in}{0.500000in}}%
\pgfpathlineto{\pgfqpoint{4.109733in}{0.500000in}}%
\pgfpathlineto{\pgfqpoint{4.109733in}{0.500000in}}%
\pgfpathlineto{\pgfqpoint{4.085454in}{0.500000in}}%
\pgfpathlineto{\pgfqpoint{4.085454in}{0.500000in}}%
\pgfpathclose%
\pgfusepath{fill}%
\end{pgfscope}%
\begin{pgfscope}%
\pgfpathrectangle{\pgfqpoint{0.750000in}{0.500000in}}{\pgfqpoint{4.650000in}{3.020000in}}%
\pgfusepath{clip}%
\pgfsetbuttcap%
\pgfsetmiterjoin%
\definecolor{currentfill}{rgb}{1.000000,0.000000,0.000000}%
\pgfsetfillcolor{currentfill}%
\pgfsetlinewidth{0.000000pt}%
\definecolor{currentstroke}{rgb}{0.000000,0.000000,0.000000}%
\pgfsetstrokecolor{currentstroke}%
\pgfsetstrokeopacity{0.000000}%
\pgfsetdash{}{0pt}%
\pgfpathmoveto{\pgfqpoint{4.109733in}{0.500000in}}%
\pgfpathlineto{\pgfqpoint{4.134011in}{0.500000in}}%
\pgfpathlineto{\pgfqpoint{4.134011in}{0.500000in}}%
\pgfpathlineto{\pgfqpoint{4.109733in}{0.500000in}}%
\pgfpathlineto{\pgfqpoint{4.109733in}{0.500000in}}%
\pgfpathclose%
\pgfusepath{fill}%
\end{pgfscope}%
\begin{pgfscope}%
\pgfpathrectangle{\pgfqpoint{0.750000in}{0.500000in}}{\pgfqpoint{4.650000in}{3.020000in}}%
\pgfusepath{clip}%
\pgfsetbuttcap%
\pgfsetmiterjoin%
\definecolor{currentfill}{rgb}{1.000000,0.000000,0.000000}%
\pgfsetfillcolor{currentfill}%
\pgfsetlinewidth{0.000000pt}%
\definecolor{currentstroke}{rgb}{0.000000,0.000000,0.000000}%
\pgfsetstrokecolor{currentstroke}%
\pgfsetstrokeopacity{0.000000}%
\pgfsetdash{}{0pt}%
\pgfpathmoveto{\pgfqpoint{4.134011in}{0.500000in}}%
\pgfpathlineto{\pgfqpoint{4.158290in}{0.500000in}}%
\pgfpathlineto{\pgfqpoint{4.158290in}{0.500000in}}%
\pgfpathlineto{\pgfqpoint{4.134011in}{0.500000in}}%
\pgfpathlineto{\pgfqpoint{4.134011in}{0.500000in}}%
\pgfpathclose%
\pgfusepath{fill}%
\end{pgfscope}%
\begin{pgfscope}%
\pgfpathrectangle{\pgfqpoint{0.750000in}{0.500000in}}{\pgfqpoint{4.650000in}{3.020000in}}%
\pgfusepath{clip}%
\pgfsetbuttcap%
\pgfsetmiterjoin%
\definecolor{currentfill}{rgb}{1.000000,0.000000,0.000000}%
\pgfsetfillcolor{currentfill}%
\pgfsetlinewidth{0.000000pt}%
\definecolor{currentstroke}{rgb}{0.000000,0.000000,0.000000}%
\pgfsetstrokecolor{currentstroke}%
\pgfsetstrokeopacity{0.000000}%
\pgfsetdash{}{0pt}%
\pgfpathmoveto{\pgfqpoint{4.158290in}{0.500000in}}%
\pgfpathlineto{\pgfqpoint{4.182569in}{0.500000in}}%
\pgfpathlineto{\pgfqpoint{4.182569in}{0.500000in}}%
\pgfpathlineto{\pgfqpoint{4.158290in}{0.500000in}}%
\pgfpathlineto{\pgfqpoint{4.158290in}{0.500000in}}%
\pgfpathclose%
\pgfusepath{fill}%
\end{pgfscope}%
\begin{pgfscope}%
\pgfpathrectangle{\pgfqpoint{0.750000in}{0.500000in}}{\pgfqpoint{4.650000in}{3.020000in}}%
\pgfusepath{clip}%
\pgfsetbuttcap%
\pgfsetmiterjoin%
\definecolor{currentfill}{rgb}{1.000000,0.000000,0.000000}%
\pgfsetfillcolor{currentfill}%
\pgfsetlinewidth{0.000000pt}%
\definecolor{currentstroke}{rgb}{0.000000,0.000000,0.000000}%
\pgfsetstrokecolor{currentstroke}%
\pgfsetstrokeopacity{0.000000}%
\pgfsetdash{}{0pt}%
\pgfpathmoveto{\pgfqpoint{4.182569in}{0.500000in}}%
\pgfpathlineto{\pgfqpoint{4.206847in}{0.500000in}}%
\pgfpathlineto{\pgfqpoint{4.206847in}{0.500000in}}%
\pgfpathlineto{\pgfqpoint{4.182569in}{0.500000in}}%
\pgfpathlineto{\pgfqpoint{4.182569in}{0.500000in}}%
\pgfpathclose%
\pgfusepath{fill}%
\end{pgfscope}%
\begin{pgfscope}%
\pgfpathrectangle{\pgfqpoint{0.750000in}{0.500000in}}{\pgfqpoint{4.650000in}{3.020000in}}%
\pgfusepath{clip}%
\pgfsetbuttcap%
\pgfsetmiterjoin%
\definecolor{currentfill}{rgb}{1.000000,0.000000,0.000000}%
\pgfsetfillcolor{currentfill}%
\pgfsetlinewidth{0.000000pt}%
\definecolor{currentstroke}{rgb}{0.000000,0.000000,0.000000}%
\pgfsetstrokecolor{currentstroke}%
\pgfsetstrokeopacity{0.000000}%
\pgfsetdash{}{0pt}%
\pgfpathmoveto{\pgfqpoint{4.206847in}{0.500000in}}%
\pgfpathlineto{\pgfqpoint{4.231126in}{0.500000in}}%
\pgfpathlineto{\pgfqpoint{4.231126in}{0.500000in}}%
\pgfpathlineto{\pgfqpoint{4.206847in}{0.500000in}}%
\pgfpathlineto{\pgfqpoint{4.206847in}{0.500000in}}%
\pgfpathclose%
\pgfusepath{fill}%
\end{pgfscope}%
\begin{pgfscope}%
\pgfpathrectangle{\pgfqpoint{0.750000in}{0.500000in}}{\pgfqpoint{4.650000in}{3.020000in}}%
\pgfusepath{clip}%
\pgfsetbuttcap%
\pgfsetmiterjoin%
\definecolor{currentfill}{rgb}{1.000000,0.000000,0.000000}%
\pgfsetfillcolor{currentfill}%
\pgfsetlinewidth{0.000000pt}%
\definecolor{currentstroke}{rgb}{0.000000,0.000000,0.000000}%
\pgfsetstrokecolor{currentstroke}%
\pgfsetstrokeopacity{0.000000}%
\pgfsetdash{}{0pt}%
\pgfpathmoveto{\pgfqpoint{4.231126in}{0.500000in}}%
\pgfpathlineto{\pgfqpoint{4.255404in}{0.500000in}}%
\pgfpathlineto{\pgfqpoint{4.255404in}{0.503021in}}%
\pgfpathlineto{\pgfqpoint{4.231126in}{0.503021in}}%
\pgfpathlineto{\pgfqpoint{4.231126in}{0.500000in}}%
\pgfpathclose%
\pgfusepath{fill}%
\end{pgfscope}%
\begin{pgfscope}%
\pgfpathrectangle{\pgfqpoint{0.750000in}{0.500000in}}{\pgfqpoint{4.650000in}{3.020000in}}%
\pgfusepath{clip}%
\pgfsetbuttcap%
\pgfsetmiterjoin%
\definecolor{currentfill}{rgb}{1.000000,0.000000,0.000000}%
\pgfsetfillcolor{currentfill}%
\pgfsetlinewidth{0.000000pt}%
\definecolor{currentstroke}{rgb}{0.000000,0.000000,0.000000}%
\pgfsetstrokecolor{currentstroke}%
\pgfsetstrokeopacity{0.000000}%
\pgfsetdash{}{0pt}%
\pgfpathmoveto{\pgfqpoint{4.255404in}{0.500000in}}%
\pgfpathlineto{\pgfqpoint{4.279683in}{0.500000in}}%
\pgfpathlineto{\pgfqpoint{4.279683in}{0.500000in}}%
\pgfpathlineto{\pgfqpoint{4.255404in}{0.500000in}}%
\pgfpathlineto{\pgfqpoint{4.255404in}{0.500000in}}%
\pgfpathclose%
\pgfusepath{fill}%
\end{pgfscope}%
\begin{pgfscope}%
\pgfpathrectangle{\pgfqpoint{0.750000in}{0.500000in}}{\pgfqpoint{4.650000in}{3.020000in}}%
\pgfusepath{clip}%
\pgfsetbuttcap%
\pgfsetmiterjoin%
\definecolor{currentfill}{rgb}{1.000000,0.000000,0.000000}%
\pgfsetfillcolor{currentfill}%
\pgfsetlinewidth{0.000000pt}%
\definecolor{currentstroke}{rgb}{0.000000,0.000000,0.000000}%
\pgfsetstrokecolor{currentstroke}%
\pgfsetstrokeopacity{0.000000}%
\pgfsetdash{}{0pt}%
\pgfpathmoveto{\pgfqpoint{4.279683in}{0.500000in}}%
\pgfpathlineto{\pgfqpoint{4.303961in}{0.500000in}}%
\pgfpathlineto{\pgfqpoint{4.303961in}{0.500000in}}%
\pgfpathlineto{\pgfqpoint{4.279683in}{0.500000in}}%
\pgfpathlineto{\pgfqpoint{4.279683in}{0.500000in}}%
\pgfpathclose%
\pgfusepath{fill}%
\end{pgfscope}%
\begin{pgfscope}%
\pgfpathrectangle{\pgfqpoint{0.750000in}{0.500000in}}{\pgfqpoint{4.650000in}{3.020000in}}%
\pgfusepath{clip}%
\pgfsetbuttcap%
\pgfsetmiterjoin%
\definecolor{currentfill}{rgb}{1.000000,0.000000,0.000000}%
\pgfsetfillcolor{currentfill}%
\pgfsetlinewidth{0.000000pt}%
\definecolor{currentstroke}{rgb}{0.000000,0.000000,0.000000}%
\pgfsetstrokecolor{currentstroke}%
\pgfsetstrokeopacity{0.000000}%
\pgfsetdash{}{0pt}%
\pgfpathmoveto{\pgfqpoint{4.303961in}{0.500000in}}%
\pgfpathlineto{\pgfqpoint{4.328240in}{0.500000in}}%
\pgfpathlineto{\pgfqpoint{4.328240in}{0.500000in}}%
\pgfpathlineto{\pgfqpoint{4.303961in}{0.500000in}}%
\pgfpathlineto{\pgfqpoint{4.303961in}{0.500000in}}%
\pgfpathclose%
\pgfusepath{fill}%
\end{pgfscope}%
\begin{pgfscope}%
\pgfpathrectangle{\pgfqpoint{0.750000in}{0.500000in}}{\pgfqpoint{4.650000in}{3.020000in}}%
\pgfusepath{clip}%
\pgfsetbuttcap%
\pgfsetmiterjoin%
\definecolor{currentfill}{rgb}{1.000000,0.000000,0.000000}%
\pgfsetfillcolor{currentfill}%
\pgfsetlinewidth{0.000000pt}%
\definecolor{currentstroke}{rgb}{0.000000,0.000000,0.000000}%
\pgfsetstrokecolor{currentstroke}%
\pgfsetstrokeopacity{0.000000}%
\pgfsetdash{}{0pt}%
\pgfpathmoveto{\pgfqpoint{4.328240in}{0.500000in}}%
\pgfpathlineto{\pgfqpoint{4.352519in}{0.500000in}}%
\pgfpathlineto{\pgfqpoint{4.352519in}{0.500000in}}%
\pgfpathlineto{\pgfqpoint{4.328240in}{0.500000in}}%
\pgfpathlineto{\pgfqpoint{4.328240in}{0.500000in}}%
\pgfpathclose%
\pgfusepath{fill}%
\end{pgfscope}%
\begin{pgfscope}%
\pgfpathrectangle{\pgfqpoint{0.750000in}{0.500000in}}{\pgfqpoint{4.650000in}{3.020000in}}%
\pgfusepath{clip}%
\pgfsetbuttcap%
\pgfsetmiterjoin%
\definecolor{currentfill}{rgb}{1.000000,0.000000,0.000000}%
\pgfsetfillcolor{currentfill}%
\pgfsetlinewidth{0.000000pt}%
\definecolor{currentstroke}{rgb}{0.000000,0.000000,0.000000}%
\pgfsetstrokecolor{currentstroke}%
\pgfsetstrokeopacity{0.000000}%
\pgfsetdash{}{0pt}%
\pgfpathmoveto{\pgfqpoint{4.352519in}{0.500000in}}%
\pgfpathlineto{\pgfqpoint{4.376797in}{0.500000in}}%
\pgfpathlineto{\pgfqpoint{4.376797in}{0.500000in}}%
\pgfpathlineto{\pgfqpoint{4.352519in}{0.500000in}}%
\pgfpathlineto{\pgfqpoint{4.352519in}{0.500000in}}%
\pgfpathclose%
\pgfusepath{fill}%
\end{pgfscope}%
\begin{pgfscope}%
\pgfpathrectangle{\pgfqpoint{0.750000in}{0.500000in}}{\pgfqpoint{4.650000in}{3.020000in}}%
\pgfusepath{clip}%
\pgfsetbuttcap%
\pgfsetmiterjoin%
\definecolor{currentfill}{rgb}{1.000000,0.000000,0.000000}%
\pgfsetfillcolor{currentfill}%
\pgfsetlinewidth{0.000000pt}%
\definecolor{currentstroke}{rgb}{0.000000,0.000000,0.000000}%
\pgfsetstrokecolor{currentstroke}%
\pgfsetstrokeopacity{0.000000}%
\pgfsetdash{}{0pt}%
\pgfpathmoveto{\pgfqpoint{4.376797in}{0.500000in}}%
\pgfpathlineto{\pgfqpoint{4.401076in}{0.500000in}}%
\pgfpathlineto{\pgfqpoint{4.401076in}{0.500000in}}%
\pgfpathlineto{\pgfqpoint{4.376797in}{0.500000in}}%
\pgfpathlineto{\pgfqpoint{4.376797in}{0.500000in}}%
\pgfpathclose%
\pgfusepath{fill}%
\end{pgfscope}%
\begin{pgfscope}%
\pgfpathrectangle{\pgfqpoint{0.750000in}{0.500000in}}{\pgfqpoint{4.650000in}{3.020000in}}%
\pgfusepath{clip}%
\pgfsetbuttcap%
\pgfsetmiterjoin%
\definecolor{currentfill}{rgb}{1.000000,0.000000,0.000000}%
\pgfsetfillcolor{currentfill}%
\pgfsetlinewidth{0.000000pt}%
\definecolor{currentstroke}{rgb}{0.000000,0.000000,0.000000}%
\pgfsetstrokecolor{currentstroke}%
\pgfsetstrokeopacity{0.000000}%
\pgfsetdash{}{0pt}%
\pgfpathmoveto{\pgfqpoint{4.401076in}{0.500000in}}%
\pgfpathlineto{\pgfqpoint{4.425354in}{0.500000in}}%
\pgfpathlineto{\pgfqpoint{4.425354in}{0.500000in}}%
\pgfpathlineto{\pgfqpoint{4.401076in}{0.500000in}}%
\pgfpathlineto{\pgfqpoint{4.401076in}{0.500000in}}%
\pgfpathclose%
\pgfusepath{fill}%
\end{pgfscope}%
\begin{pgfscope}%
\pgfpathrectangle{\pgfqpoint{0.750000in}{0.500000in}}{\pgfqpoint{4.650000in}{3.020000in}}%
\pgfusepath{clip}%
\pgfsetbuttcap%
\pgfsetmiterjoin%
\definecolor{currentfill}{rgb}{1.000000,0.000000,0.000000}%
\pgfsetfillcolor{currentfill}%
\pgfsetlinewidth{0.000000pt}%
\definecolor{currentstroke}{rgb}{0.000000,0.000000,0.000000}%
\pgfsetstrokecolor{currentstroke}%
\pgfsetstrokeopacity{0.000000}%
\pgfsetdash{}{0pt}%
\pgfpathmoveto{\pgfqpoint{4.425354in}{0.500000in}}%
\pgfpathlineto{\pgfqpoint{4.449633in}{0.500000in}}%
\pgfpathlineto{\pgfqpoint{4.449633in}{0.500000in}}%
\pgfpathlineto{\pgfqpoint{4.425354in}{0.500000in}}%
\pgfpathlineto{\pgfqpoint{4.425354in}{0.500000in}}%
\pgfpathclose%
\pgfusepath{fill}%
\end{pgfscope}%
\begin{pgfscope}%
\pgfpathrectangle{\pgfqpoint{0.750000in}{0.500000in}}{\pgfqpoint{4.650000in}{3.020000in}}%
\pgfusepath{clip}%
\pgfsetbuttcap%
\pgfsetmiterjoin%
\definecolor{currentfill}{rgb}{1.000000,0.000000,0.000000}%
\pgfsetfillcolor{currentfill}%
\pgfsetlinewidth{0.000000pt}%
\definecolor{currentstroke}{rgb}{0.000000,0.000000,0.000000}%
\pgfsetstrokecolor{currentstroke}%
\pgfsetstrokeopacity{0.000000}%
\pgfsetdash{}{0pt}%
\pgfpathmoveto{\pgfqpoint{4.449633in}{0.500000in}}%
\pgfpathlineto{\pgfqpoint{4.473912in}{0.500000in}}%
\pgfpathlineto{\pgfqpoint{4.473912in}{0.500000in}}%
\pgfpathlineto{\pgfqpoint{4.449633in}{0.500000in}}%
\pgfpathlineto{\pgfqpoint{4.449633in}{0.500000in}}%
\pgfpathclose%
\pgfusepath{fill}%
\end{pgfscope}%
\begin{pgfscope}%
\pgfpathrectangle{\pgfqpoint{0.750000in}{0.500000in}}{\pgfqpoint{4.650000in}{3.020000in}}%
\pgfusepath{clip}%
\pgfsetbuttcap%
\pgfsetmiterjoin%
\definecolor{currentfill}{rgb}{1.000000,0.000000,0.000000}%
\pgfsetfillcolor{currentfill}%
\pgfsetlinewidth{0.000000pt}%
\definecolor{currentstroke}{rgb}{0.000000,0.000000,0.000000}%
\pgfsetstrokecolor{currentstroke}%
\pgfsetstrokeopacity{0.000000}%
\pgfsetdash{}{0pt}%
\pgfpathmoveto{\pgfqpoint{4.473912in}{0.500000in}}%
\pgfpathlineto{\pgfqpoint{4.498190in}{0.500000in}}%
\pgfpathlineto{\pgfqpoint{4.498190in}{0.500000in}}%
\pgfpathlineto{\pgfqpoint{4.473912in}{0.500000in}}%
\pgfpathlineto{\pgfqpoint{4.473912in}{0.500000in}}%
\pgfpathclose%
\pgfusepath{fill}%
\end{pgfscope}%
\begin{pgfscope}%
\pgfpathrectangle{\pgfqpoint{0.750000in}{0.500000in}}{\pgfqpoint{4.650000in}{3.020000in}}%
\pgfusepath{clip}%
\pgfsetbuttcap%
\pgfsetmiterjoin%
\definecolor{currentfill}{rgb}{1.000000,0.000000,0.000000}%
\pgfsetfillcolor{currentfill}%
\pgfsetlinewidth{0.000000pt}%
\definecolor{currentstroke}{rgb}{0.000000,0.000000,0.000000}%
\pgfsetstrokecolor{currentstroke}%
\pgfsetstrokeopacity{0.000000}%
\pgfsetdash{}{0pt}%
\pgfpathmoveto{\pgfqpoint{4.498190in}{0.500000in}}%
\pgfpathlineto{\pgfqpoint{4.522469in}{0.500000in}}%
\pgfpathlineto{\pgfqpoint{4.522469in}{0.500000in}}%
\pgfpathlineto{\pgfqpoint{4.498190in}{0.500000in}}%
\pgfpathlineto{\pgfqpoint{4.498190in}{0.500000in}}%
\pgfpathclose%
\pgfusepath{fill}%
\end{pgfscope}%
\begin{pgfscope}%
\pgfpathrectangle{\pgfqpoint{0.750000in}{0.500000in}}{\pgfqpoint{4.650000in}{3.020000in}}%
\pgfusepath{clip}%
\pgfsetbuttcap%
\pgfsetmiterjoin%
\definecolor{currentfill}{rgb}{1.000000,0.000000,0.000000}%
\pgfsetfillcolor{currentfill}%
\pgfsetlinewidth{0.000000pt}%
\definecolor{currentstroke}{rgb}{0.000000,0.000000,0.000000}%
\pgfsetstrokecolor{currentstroke}%
\pgfsetstrokeopacity{0.000000}%
\pgfsetdash{}{0pt}%
\pgfpathmoveto{\pgfqpoint{4.522469in}{0.500000in}}%
\pgfpathlineto{\pgfqpoint{4.546747in}{0.500000in}}%
\pgfpathlineto{\pgfqpoint{4.546747in}{0.500000in}}%
\pgfpathlineto{\pgfqpoint{4.522469in}{0.500000in}}%
\pgfpathlineto{\pgfqpoint{4.522469in}{0.500000in}}%
\pgfpathclose%
\pgfusepath{fill}%
\end{pgfscope}%
\begin{pgfscope}%
\pgfpathrectangle{\pgfqpoint{0.750000in}{0.500000in}}{\pgfqpoint{4.650000in}{3.020000in}}%
\pgfusepath{clip}%
\pgfsetbuttcap%
\pgfsetmiterjoin%
\definecolor{currentfill}{rgb}{1.000000,0.000000,0.000000}%
\pgfsetfillcolor{currentfill}%
\pgfsetlinewidth{0.000000pt}%
\definecolor{currentstroke}{rgb}{0.000000,0.000000,0.000000}%
\pgfsetstrokecolor{currentstroke}%
\pgfsetstrokeopacity{0.000000}%
\pgfsetdash{}{0pt}%
\pgfpathmoveto{\pgfqpoint{4.546747in}{0.500000in}}%
\pgfpathlineto{\pgfqpoint{4.571026in}{0.500000in}}%
\pgfpathlineto{\pgfqpoint{4.571026in}{0.503021in}}%
\pgfpathlineto{\pgfqpoint{4.546747in}{0.503021in}}%
\pgfpathlineto{\pgfqpoint{4.546747in}{0.500000in}}%
\pgfpathclose%
\pgfusepath{fill}%
\end{pgfscope}%
\begin{pgfscope}%
\pgfpathrectangle{\pgfqpoint{0.750000in}{0.500000in}}{\pgfqpoint{4.650000in}{3.020000in}}%
\pgfusepath{clip}%
\pgfsetbuttcap%
\pgfsetmiterjoin%
\definecolor{currentfill}{rgb}{1.000000,0.000000,0.000000}%
\pgfsetfillcolor{currentfill}%
\pgfsetlinewidth{0.000000pt}%
\definecolor{currentstroke}{rgb}{0.000000,0.000000,0.000000}%
\pgfsetstrokecolor{currentstroke}%
\pgfsetstrokeopacity{0.000000}%
\pgfsetdash{}{0pt}%
\pgfpathmoveto{\pgfqpoint{4.571026in}{0.500000in}}%
\pgfpathlineto{\pgfqpoint{4.595304in}{0.500000in}}%
\pgfpathlineto{\pgfqpoint{4.595304in}{0.500000in}}%
\pgfpathlineto{\pgfqpoint{4.571026in}{0.500000in}}%
\pgfpathlineto{\pgfqpoint{4.571026in}{0.500000in}}%
\pgfpathclose%
\pgfusepath{fill}%
\end{pgfscope}%
\begin{pgfscope}%
\pgfpathrectangle{\pgfqpoint{0.750000in}{0.500000in}}{\pgfqpoint{4.650000in}{3.020000in}}%
\pgfusepath{clip}%
\pgfsetbuttcap%
\pgfsetmiterjoin%
\definecolor{currentfill}{rgb}{1.000000,0.000000,0.000000}%
\pgfsetfillcolor{currentfill}%
\pgfsetlinewidth{0.000000pt}%
\definecolor{currentstroke}{rgb}{0.000000,0.000000,0.000000}%
\pgfsetstrokecolor{currentstroke}%
\pgfsetstrokeopacity{0.000000}%
\pgfsetdash{}{0pt}%
\pgfpathmoveto{\pgfqpoint{4.595304in}{0.500000in}}%
\pgfpathlineto{\pgfqpoint{4.619583in}{0.500000in}}%
\pgfpathlineto{\pgfqpoint{4.619583in}{0.500000in}}%
\pgfpathlineto{\pgfqpoint{4.595304in}{0.500000in}}%
\pgfpathlineto{\pgfqpoint{4.595304in}{0.500000in}}%
\pgfpathclose%
\pgfusepath{fill}%
\end{pgfscope}%
\begin{pgfscope}%
\pgfpathrectangle{\pgfqpoint{0.750000in}{0.500000in}}{\pgfqpoint{4.650000in}{3.020000in}}%
\pgfusepath{clip}%
\pgfsetbuttcap%
\pgfsetmiterjoin%
\definecolor{currentfill}{rgb}{1.000000,0.000000,0.000000}%
\pgfsetfillcolor{currentfill}%
\pgfsetlinewidth{0.000000pt}%
\definecolor{currentstroke}{rgb}{0.000000,0.000000,0.000000}%
\pgfsetstrokecolor{currentstroke}%
\pgfsetstrokeopacity{0.000000}%
\pgfsetdash{}{0pt}%
\pgfpathmoveto{\pgfqpoint{4.619583in}{0.500000in}}%
\pgfpathlineto{\pgfqpoint{4.643862in}{0.500000in}}%
\pgfpathlineto{\pgfqpoint{4.643862in}{0.503021in}}%
\pgfpathlineto{\pgfqpoint{4.619583in}{0.503021in}}%
\pgfpathlineto{\pgfqpoint{4.619583in}{0.500000in}}%
\pgfpathclose%
\pgfusepath{fill}%
\end{pgfscope}%
\begin{pgfscope}%
\pgfpathrectangle{\pgfqpoint{0.750000in}{0.500000in}}{\pgfqpoint{4.650000in}{3.020000in}}%
\pgfusepath{clip}%
\pgfsetbuttcap%
\pgfsetmiterjoin%
\definecolor{currentfill}{rgb}{1.000000,0.000000,0.000000}%
\pgfsetfillcolor{currentfill}%
\pgfsetlinewidth{0.000000pt}%
\definecolor{currentstroke}{rgb}{0.000000,0.000000,0.000000}%
\pgfsetstrokecolor{currentstroke}%
\pgfsetstrokeopacity{0.000000}%
\pgfsetdash{}{0pt}%
\pgfpathmoveto{\pgfqpoint{4.643862in}{0.500000in}}%
\pgfpathlineto{\pgfqpoint{4.668140in}{0.500000in}}%
\pgfpathlineto{\pgfqpoint{4.668140in}{0.500000in}}%
\pgfpathlineto{\pgfqpoint{4.643862in}{0.500000in}}%
\pgfpathlineto{\pgfqpoint{4.643862in}{0.500000in}}%
\pgfpathclose%
\pgfusepath{fill}%
\end{pgfscope}%
\begin{pgfscope}%
\pgfpathrectangle{\pgfqpoint{0.750000in}{0.500000in}}{\pgfqpoint{4.650000in}{3.020000in}}%
\pgfusepath{clip}%
\pgfsetbuttcap%
\pgfsetmiterjoin%
\definecolor{currentfill}{rgb}{1.000000,0.000000,0.000000}%
\pgfsetfillcolor{currentfill}%
\pgfsetlinewidth{0.000000pt}%
\definecolor{currentstroke}{rgb}{0.000000,0.000000,0.000000}%
\pgfsetstrokecolor{currentstroke}%
\pgfsetstrokeopacity{0.000000}%
\pgfsetdash{}{0pt}%
\pgfpathmoveto{\pgfqpoint{4.668140in}{0.500000in}}%
\pgfpathlineto{\pgfqpoint{4.692419in}{0.500000in}}%
\pgfpathlineto{\pgfqpoint{4.692419in}{0.503021in}}%
\pgfpathlineto{\pgfqpoint{4.668140in}{0.503021in}}%
\pgfpathlineto{\pgfqpoint{4.668140in}{0.500000in}}%
\pgfpathclose%
\pgfusepath{fill}%
\end{pgfscope}%
\begin{pgfscope}%
\pgfpathrectangle{\pgfqpoint{0.750000in}{0.500000in}}{\pgfqpoint{4.650000in}{3.020000in}}%
\pgfusepath{clip}%
\pgfsetbuttcap%
\pgfsetmiterjoin%
\definecolor{currentfill}{rgb}{1.000000,0.000000,0.000000}%
\pgfsetfillcolor{currentfill}%
\pgfsetlinewidth{0.000000pt}%
\definecolor{currentstroke}{rgb}{0.000000,0.000000,0.000000}%
\pgfsetstrokecolor{currentstroke}%
\pgfsetstrokeopacity{0.000000}%
\pgfsetdash{}{0pt}%
\pgfpathmoveto{\pgfqpoint{4.692419in}{0.500000in}}%
\pgfpathlineto{\pgfqpoint{4.716697in}{0.500000in}}%
\pgfpathlineto{\pgfqpoint{4.716697in}{0.500000in}}%
\pgfpathlineto{\pgfqpoint{4.692419in}{0.500000in}}%
\pgfpathlineto{\pgfqpoint{4.692419in}{0.500000in}}%
\pgfpathclose%
\pgfusepath{fill}%
\end{pgfscope}%
\begin{pgfscope}%
\pgfpathrectangle{\pgfqpoint{0.750000in}{0.500000in}}{\pgfqpoint{4.650000in}{3.020000in}}%
\pgfusepath{clip}%
\pgfsetbuttcap%
\pgfsetmiterjoin%
\definecolor{currentfill}{rgb}{1.000000,0.000000,0.000000}%
\pgfsetfillcolor{currentfill}%
\pgfsetlinewidth{0.000000pt}%
\definecolor{currentstroke}{rgb}{0.000000,0.000000,0.000000}%
\pgfsetstrokecolor{currentstroke}%
\pgfsetstrokeopacity{0.000000}%
\pgfsetdash{}{0pt}%
\pgfpathmoveto{\pgfqpoint{4.716697in}{0.500000in}}%
\pgfpathlineto{\pgfqpoint{4.740976in}{0.500000in}}%
\pgfpathlineto{\pgfqpoint{4.740976in}{0.500000in}}%
\pgfpathlineto{\pgfqpoint{4.716697in}{0.500000in}}%
\pgfpathlineto{\pgfqpoint{4.716697in}{0.500000in}}%
\pgfpathclose%
\pgfusepath{fill}%
\end{pgfscope}%
\begin{pgfscope}%
\pgfpathrectangle{\pgfqpoint{0.750000in}{0.500000in}}{\pgfqpoint{4.650000in}{3.020000in}}%
\pgfusepath{clip}%
\pgfsetbuttcap%
\pgfsetmiterjoin%
\definecolor{currentfill}{rgb}{1.000000,0.000000,0.000000}%
\pgfsetfillcolor{currentfill}%
\pgfsetlinewidth{0.000000pt}%
\definecolor{currentstroke}{rgb}{0.000000,0.000000,0.000000}%
\pgfsetstrokecolor{currentstroke}%
\pgfsetstrokeopacity{0.000000}%
\pgfsetdash{}{0pt}%
\pgfpathmoveto{\pgfqpoint{4.740976in}{0.500000in}}%
\pgfpathlineto{\pgfqpoint{4.765255in}{0.500000in}}%
\pgfpathlineto{\pgfqpoint{4.765255in}{0.500000in}}%
\pgfpathlineto{\pgfqpoint{4.740976in}{0.500000in}}%
\pgfpathlineto{\pgfqpoint{4.740976in}{0.500000in}}%
\pgfpathclose%
\pgfusepath{fill}%
\end{pgfscope}%
\begin{pgfscope}%
\pgfpathrectangle{\pgfqpoint{0.750000in}{0.500000in}}{\pgfqpoint{4.650000in}{3.020000in}}%
\pgfusepath{clip}%
\pgfsetbuttcap%
\pgfsetmiterjoin%
\definecolor{currentfill}{rgb}{1.000000,0.000000,0.000000}%
\pgfsetfillcolor{currentfill}%
\pgfsetlinewidth{0.000000pt}%
\definecolor{currentstroke}{rgb}{0.000000,0.000000,0.000000}%
\pgfsetstrokecolor{currentstroke}%
\pgfsetstrokeopacity{0.000000}%
\pgfsetdash{}{0pt}%
\pgfpathmoveto{\pgfqpoint{4.765255in}{0.500000in}}%
\pgfpathlineto{\pgfqpoint{4.789533in}{0.500000in}}%
\pgfpathlineto{\pgfqpoint{4.789533in}{0.500000in}}%
\pgfpathlineto{\pgfqpoint{4.765255in}{0.500000in}}%
\pgfpathlineto{\pgfqpoint{4.765255in}{0.500000in}}%
\pgfpathclose%
\pgfusepath{fill}%
\end{pgfscope}%
\begin{pgfscope}%
\pgfpathrectangle{\pgfqpoint{0.750000in}{0.500000in}}{\pgfqpoint{4.650000in}{3.020000in}}%
\pgfusepath{clip}%
\pgfsetbuttcap%
\pgfsetmiterjoin%
\definecolor{currentfill}{rgb}{1.000000,0.000000,0.000000}%
\pgfsetfillcolor{currentfill}%
\pgfsetlinewidth{0.000000pt}%
\definecolor{currentstroke}{rgb}{0.000000,0.000000,0.000000}%
\pgfsetstrokecolor{currentstroke}%
\pgfsetstrokeopacity{0.000000}%
\pgfsetdash{}{0pt}%
\pgfpathmoveto{\pgfqpoint{4.789533in}{0.500000in}}%
\pgfpathlineto{\pgfqpoint{4.813812in}{0.500000in}}%
\pgfpathlineto{\pgfqpoint{4.813812in}{0.503021in}}%
\pgfpathlineto{\pgfqpoint{4.789533in}{0.503021in}}%
\pgfpathlineto{\pgfqpoint{4.789533in}{0.500000in}}%
\pgfpathclose%
\pgfusepath{fill}%
\end{pgfscope}%
\begin{pgfscope}%
\pgfpathrectangle{\pgfqpoint{0.750000in}{0.500000in}}{\pgfqpoint{4.650000in}{3.020000in}}%
\pgfusepath{clip}%
\pgfsetbuttcap%
\pgfsetmiterjoin%
\definecolor{currentfill}{rgb}{1.000000,0.000000,0.000000}%
\pgfsetfillcolor{currentfill}%
\pgfsetlinewidth{0.000000pt}%
\definecolor{currentstroke}{rgb}{0.000000,0.000000,0.000000}%
\pgfsetstrokecolor{currentstroke}%
\pgfsetstrokeopacity{0.000000}%
\pgfsetdash{}{0pt}%
\pgfpathmoveto{\pgfqpoint{4.813812in}{0.500000in}}%
\pgfpathlineto{\pgfqpoint{4.838090in}{0.500000in}}%
\pgfpathlineto{\pgfqpoint{4.838090in}{0.500000in}}%
\pgfpathlineto{\pgfqpoint{4.813812in}{0.500000in}}%
\pgfpathlineto{\pgfqpoint{4.813812in}{0.500000in}}%
\pgfpathclose%
\pgfusepath{fill}%
\end{pgfscope}%
\begin{pgfscope}%
\pgfpathrectangle{\pgfqpoint{0.750000in}{0.500000in}}{\pgfqpoint{4.650000in}{3.020000in}}%
\pgfusepath{clip}%
\pgfsetbuttcap%
\pgfsetmiterjoin%
\definecolor{currentfill}{rgb}{1.000000,0.000000,0.000000}%
\pgfsetfillcolor{currentfill}%
\pgfsetlinewidth{0.000000pt}%
\definecolor{currentstroke}{rgb}{0.000000,0.000000,0.000000}%
\pgfsetstrokecolor{currentstroke}%
\pgfsetstrokeopacity{0.000000}%
\pgfsetdash{}{0pt}%
\pgfpathmoveto{\pgfqpoint{4.838090in}{0.500000in}}%
\pgfpathlineto{\pgfqpoint{4.862369in}{0.500000in}}%
\pgfpathlineto{\pgfqpoint{4.862369in}{0.500000in}}%
\pgfpathlineto{\pgfqpoint{4.838090in}{0.500000in}}%
\pgfpathlineto{\pgfqpoint{4.838090in}{0.500000in}}%
\pgfpathclose%
\pgfusepath{fill}%
\end{pgfscope}%
\begin{pgfscope}%
\pgfpathrectangle{\pgfqpoint{0.750000in}{0.500000in}}{\pgfqpoint{4.650000in}{3.020000in}}%
\pgfusepath{clip}%
\pgfsetbuttcap%
\pgfsetmiterjoin%
\definecolor{currentfill}{rgb}{1.000000,0.000000,0.000000}%
\pgfsetfillcolor{currentfill}%
\pgfsetlinewidth{0.000000pt}%
\definecolor{currentstroke}{rgb}{0.000000,0.000000,0.000000}%
\pgfsetstrokecolor{currentstroke}%
\pgfsetstrokeopacity{0.000000}%
\pgfsetdash{}{0pt}%
\pgfpathmoveto{\pgfqpoint{4.862369in}{0.500000in}}%
\pgfpathlineto{\pgfqpoint{4.886647in}{0.500000in}}%
\pgfpathlineto{\pgfqpoint{4.886647in}{0.503021in}}%
\pgfpathlineto{\pgfqpoint{4.862369in}{0.503021in}}%
\pgfpathlineto{\pgfqpoint{4.862369in}{0.500000in}}%
\pgfpathclose%
\pgfusepath{fill}%
\end{pgfscope}%
\begin{pgfscope}%
\pgfpathrectangle{\pgfqpoint{0.750000in}{0.500000in}}{\pgfqpoint{4.650000in}{3.020000in}}%
\pgfusepath{clip}%
\pgfsetbuttcap%
\pgfsetmiterjoin%
\definecolor{currentfill}{rgb}{1.000000,0.000000,0.000000}%
\pgfsetfillcolor{currentfill}%
\pgfsetlinewidth{0.000000pt}%
\definecolor{currentstroke}{rgb}{0.000000,0.000000,0.000000}%
\pgfsetstrokecolor{currentstroke}%
\pgfsetstrokeopacity{0.000000}%
\pgfsetdash{}{0pt}%
\pgfpathmoveto{\pgfqpoint{4.886647in}{0.500000in}}%
\pgfpathlineto{\pgfqpoint{4.910926in}{0.500000in}}%
\pgfpathlineto{\pgfqpoint{4.910926in}{0.500000in}}%
\pgfpathlineto{\pgfqpoint{4.886647in}{0.500000in}}%
\pgfpathlineto{\pgfqpoint{4.886647in}{0.500000in}}%
\pgfpathclose%
\pgfusepath{fill}%
\end{pgfscope}%
\begin{pgfscope}%
\pgfpathrectangle{\pgfqpoint{0.750000in}{0.500000in}}{\pgfqpoint{4.650000in}{3.020000in}}%
\pgfusepath{clip}%
\pgfsetbuttcap%
\pgfsetmiterjoin%
\definecolor{currentfill}{rgb}{1.000000,0.000000,0.000000}%
\pgfsetfillcolor{currentfill}%
\pgfsetlinewidth{0.000000pt}%
\definecolor{currentstroke}{rgb}{0.000000,0.000000,0.000000}%
\pgfsetstrokecolor{currentstroke}%
\pgfsetstrokeopacity{0.000000}%
\pgfsetdash{}{0pt}%
\pgfpathmoveto{\pgfqpoint{4.910926in}{0.500000in}}%
\pgfpathlineto{\pgfqpoint{4.935205in}{0.500000in}}%
\pgfpathlineto{\pgfqpoint{4.935205in}{0.500000in}}%
\pgfpathlineto{\pgfqpoint{4.910926in}{0.500000in}}%
\pgfpathlineto{\pgfqpoint{4.910926in}{0.500000in}}%
\pgfpathclose%
\pgfusepath{fill}%
\end{pgfscope}%
\begin{pgfscope}%
\pgfpathrectangle{\pgfqpoint{0.750000in}{0.500000in}}{\pgfqpoint{4.650000in}{3.020000in}}%
\pgfusepath{clip}%
\pgfsetbuttcap%
\pgfsetmiterjoin%
\definecolor{currentfill}{rgb}{1.000000,0.000000,0.000000}%
\pgfsetfillcolor{currentfill}%
\pgfsetlinewidth{0.000000pt}%
\definecolor{currentstroke}{rgb}{0.000000,0.000000,0.000000}%
\pgfsetstrokecolor{currentstroke}%
\pgfsetstrokeopacity{0.000000}%
\pgfsetdash{}{0pt}%
\pgfpathmoveto{\pgfqpoint{4.935205in}{0.500000in}}%
\pgfpathlineto{\pgfqpoint{4.959483in}{0.500000in}}%
\pgfpathlineto{\pgfqpoint{4.959483in}{0.506042in}}%
\pgfpathlineto{\pgfqpoint{4.935205in}{0.506042in}}%
\pgfpathlineto{\pgfqpoint{4.935205in}{0.500000in}}%
\pgfpathclose%
\pgfusepath{fill}%
\end{pgfscope}%
\begin{pgfscope}%
\pgfpathrectangle{\pgfqpoint{0.750000in}{0.500000in}}{\pgfqpoint{4.650000in}{3.020000in}}%
\pgfusepath{clip}%
\pgfsetbuttcap%
\pgfsetmiterjoin%
\definecolor{currentfill}{rgb}{1.000000,0.000000,0.000000}%
\pgfsetfillcolor{currentfill}%
\pgfsetlinewidth{0.000000pt}%
\definecolor{currentstroke}{rgb}{0.000000,0.000000,0.000000}%
\pgfsetstrokecolor{currentstroke}%
\pgfsetstrokeopacity{0.000000}%
\pgfsetdash{}{0pt}%
\pgfpathmoveto{\pgfqpoint{4.959483in}{0.500000in}}%
\pgfpathlineto{\pgfqpoint{4.983762in}{0.500000in}}%
\pgfpathlineto{\pgfqpoint{4.983762in}{0.500000in}}%
\pgfpathlineto{\pgfqpoint{4.959483in}{0.500000in}}%
\pgfpathlineto{\pgfqpoint{4.959483in}{0.500000in}}%
\pgfpathclose%
\pgfusepath{fill}%
\end{pgfscope}%
\begin{pgfscope}%
\pgfpathrectangle{\pgfqpoint{0.750000in}{0.500000in}}{\pgfqpoint{4.650000in}{3.020000in}}%
\pgfusepath{clip}%
\pgfsetbuttcap%
\pgfsetmiterjoin%
\definecolor{currentfill}{rgb}{1.000000,0.000000,0.000000}%
\pgfsetfillcolor{currentfill}%
\pgfsetlinewidth{0.000000pt}%
\definecolor{currentstroke}{rgb}{0.000000,0.000000,0.000000}%
\pgfsetstrokecolor{currentstroke}%
\pgfsetstrokeopacity{0.000000}%
\pgfsetdash{}{0pt}%
\pgfpathmoveto{\pgfqpoint{4.983762in}{0.500000in}}%
\pgfpathlineto{\pgfqpoint{5.008040in}{0.500000in}}%
\pgfpathlineto{\pgfqpoint{5.008040in}{0.500000in}}%
\pgfpathlineto{\pgfqpoint{4.983762in}{0.500000in}}%
\pgfpathlineto{\pgfqpoint{4.983762in}{0.500000in}}%
\pgfpathclose%
\pgfusepath{fill}%
\end{pgfscope}%
\begin{pgfscope}%
\pgfpathrectangle{\pgfqpoint{0.750000in}{0.500000in}}{\pgfqpoint{4.650000in}{3.020000in}}%
\pgfusepath{clip}%
\pgfsetbuttcap%
\pgfsetmiterjoin%
\definecolor{currentfill}{rgb}{1.000000,0.000000,0.000000}%
\pgfsetfillcolor{currentfill}%
\pgfsetlinewidth{0.000000pt}%
\definecolor{currentstroke}{rgb}{0.000000,0.000000,0.000000}%
\pgfsetstrokecolor{currentstroke}%
\pgfsetstrokeopacity{0.000000}%
\pgfsetdash{}{0pt}%
\pgfpathmoveto{\pgfqpoint{5.008040in}{0.500000in}}%
\pgfpathlineto{\pgfqpoint{5.032319in}{0.500000in}}%
\pgfpathlineto{\pgfqpoint{5.032319in}{0.503021in}}%
\pgfpathlineto{\pgfqpoint{5.008040in}{0.503021in}}%
\pgfpathlineto{\pgfqpoint{5.008040in}{0.500000in}}%
\pgfpathclose%
\pgfusepath{fill}%
\end{pgfscope}%
\begin{pgfscope}%
\pgfpathrectangle{\pgfqpoint{0.750000in}{0.500000in}}{\pgfqpoint{4.650000in}{3.020000in}}%
\pgfusepath{clip}%
\pgfsetbuttcap%
\pgfsetmiterjoin%
\definecolor{currentfill}{rgb}{1.000000,0.000000,0.000000}%
\pgfsetfillcolor{currentfill}%
\pgfsetlinewidth{0.000000pt}%
\definecolor{currentstroke}{rgb}{0.000000,0.000000,0.000000}%
\pgfsetstrokecolor{currentstroke}%
\pgfsetstrokeopacity{0.000000}%
\pgfsetdash{}{0pt}%
\pgfpathmoveto{\pgfqpoint{5.032319in}{0.500000in}}%
\pgfpathlineto{\pgfqpoint{5.056597in}{0.500000in}}%
\pgfpathlineto{\pgfqpoint{5.056597in}{0.506042in}}%
\pgfpathlineto{\pgfqpoint{5.032319in}{0.506042in}}%
\pgfpathlineto{\pgfqpoint{5.032319in}{0.500000in}}%
\pgfpathclose%
\pgfusepath{fill}%
\end{pgfscope}%
\begin{pgfscope}%
\pgfpathrectangle{\pgfqpoint{0.750000in}{0.500000in}}{\pgfqpoint{4.650000in}{3.020000in}}%
\pgfusepath{clip}%
\pgfsetbuttcap%
\pgfsetmiterjoin%
\definecolor{currentfill}{rgb}{1.000000,0.000000,0.000000}%
\pgfsetfillcolor{currentfill}%
\pgfsetlinewidth{0.000000pt}%
\definecolor{currentstroke}{rgb}{0.000000,0.000000,0.000000}%
\pgfsetstrokecolor{currentstroke}%
\pgfsetstrokeopacity{0.000000}%
\pgfsetdash{}{0pt}%
\pgfpathmoveto{\pgfqpoint{5.056597in}{0.500000in}}%
\pgfpathlineto{\pgfqpoint{5.080876in}{0.500000in}}%
\pgfpathlineto{\pgfqpoint{5.080876in}{0.500000in}}%
\pgfpathlineto{\pgfqpoint{5.056597in}{0.500000in}}%
\pgfpathlineto{\pgfqpoint{5.056597in}{0.500000in}}%
\pgfpathclose%
\pgfusepath{fill}%
\end{pgfscope}%
\begin{pgfscope}%
\pgfpathrectangle{\pgfqpoint{0.750000in}{0.500000in}}{\pgfqpoint{4.650000in}{3.020000in}}%
\pgfusepath{clip}%
\pgfsetbuttcap%
\pgfsetmiterjoin%
\definecolor{currentfill}{rgb}{1.000000,0.000000,0.000000}%
\pgfsetfillcolor{currentfill}%
\pgfsetlinewidth{0.000000pt}%
\definecolor{currentstroke}{rgb}{0.000000,0.000000,0.000000}%
\pgfsetstrokecolor{currentstroke}%
\pgfsetstrokeopacity{0.000000}%
\pgfsetdash{}{0pt}%
\pgfpathmoveto{\pgfqpoint{5.080876in}{0.500000in}}%
\pgfpathlineto{\pgfqpoint{5.105155in}{0.500000in}}%
\pgfpathlineto{\pgfqpoint{5.105155in}{0.500000in}}%
\pgfpathlineto{\pgfqpoint{5.080876in}{0.500000in}}%
\pgfpathlineto{\pgfqpoint{5.080876in}{0.500000in}}%
\pgfpathclose%
\pgfusepath{fill}%
\end{pgfscope}%
\begin{pgfscope}%
\pgfpathrectangle{\pgfqpoint{0.750000in}{0.500000in}}{\pgfqpoint{4.650000in}{3.020000in}}%
\pgfusepath{clip}%
\pgfsetbuttcap%
\pgfsetmiterjoin%
\definecolor{currentfill}{rgb}{1.000000,0.000000,0.000000}%
\pgfsetfillcolor{currentfill}%
\pgfsetlinewidth{0.000000pt}%
\definecolor{currentstroke}{rgb}{0.000000,0.000000,0.000000}%
\pgfsetstrokecolor{currentstroke}%
\pgfsetstrokeopacity{0.000000}%
\pgfsetdash{}{0pt}%
\pgfpathmoveto{\pgfqpoint{5.105155in}{0.500000in}}%
\pgfpathlineto{\pgfqpoint{5.129433in}{0.500000in}}%
\pgfpathlineto{\pgfqpoint{5.129433in}{0.503021in}}%
\pgfpathlineto{\pgfqpoint{5.105155in}{0.503021in}}%
\pgfpathlineto{\pgfqpoint{5.105155in}{0.500000in}}%
\pgfpathclose%
\pgfusepath{fill}%
\end{pgfscope}%
\begin{pgfscope}%
\pgfpathrectangle{\pgfqpoint{0.750000in}{0.500000in}}{\pgfqpoint{4.650000in}{3.020000in}}%
\pgfusepath{clip}%
\pgfsetbuttcap%
\pgfsetmiterjoin%
\definecolor{currentfill}{rgb}{1.000000,0.000000,0.000000}%
\pgfsetfillcolor{currentfill}%
\pgfsetlinewidth{0.000000pt}%
\definecolor{currentstroke}{rgb}{0.000000,0.000000,0.000000}%
\pgfsetstrokecolor{currentstroke}%
\pgfsetstrokeopacity{0.000000}%
\pgfsetdash{}{0pt}%
\pgfpathmoveto{\pgfqpoint{5.129433in}{0.500000in}}%
\pgfpathlineto{\pgfqpoint{5.153712in}{0.500000in}}%
\pgfpathlineto{\pgfqpoint{5.153712in}{0.503021in}}%
\pgfpathlineto{\pgfqpoint{5.129433in}{0.503021in}}%
\pgfpathlineto{\pgfqpoint{5.129433in}{0.500000in}}%
\pgfpathclose%
\pgfusepath{fill}%
\end{pgfscope}%
\begin{pgfscope}%
\pgfpathrectangle{\pgfqpoint{0.750000in}{0.500000in}}{\pgfqpoint{4.650000in}{3.020000in}}%
\pgfusepath{clip}%
\pgfsetbuttcap%
\pgfsetmiterjoin%
\definecolor{currentfill}{rgb}{1.000000,0.000000,0.000000}%
\pgfsetfillcolor{currentfill}%
\pgfsetlinewidth{0.000000pt}%
\definecolor{currentstroke}{rgb}{0.000000,0.000000,0.000000}%
\pgfsetstrokecolor{currentstroke}%
\pgfsetstrokeopacity{0.000000}%
\pgfsetdash{}{0pt}%
\pgfpathmoveto{\pgfqpoint{5.153712in}{0.500000in}}%
\pgfpathlineto{\pgfqpoint{5.177990in}{0.500000in}}%
\pgfpathlineto{\pgfqpoint{5.177990in}{0.509064in}}%
\pgfpathlineto{\pgfqpoint{5.153712in}{0.509064in}}%
\pgfpathlineto{\pgfqpoint{5.153712in}{0.500000in}}%
\pgfpathclose%
\pgfusepath{fill}%
\end{pgfscope}%
\begin{pgfscope}%
\pgfpathrectangle{\pgfqpoint{0.750000in}{0.500000in}}{\pgfqpoint{4.650000in}{3.020000in}}%
\pgfusepath{clip}%
\pgfsetbuttcap%
\pgfsetmiterjoin%
\definecolor{currentfill}{rgb}{0.000000,0.500000,0.000000}%
\pgfsetfillcolor{currentfill}%
\pgfsetlinewidth{0.000000pt}%
\definecolor{currentstroke}{rgb}{0.000000,0.000000,0.000000}%
\pgfsetstrokecolor{currentstroke}%
\pgfsetstrokeopacity{0.000000}%
\pgfsetdash{}{0pt}%
\pgfpathmoveto{\pgfqpoint{0.961364in}{0.500000in}}%
\pgfpathlineto{\pgfqpoint{0.994389in}{0.500000in}}%
\pgfpathlineto{\pgfqpoint{0.994389in}{0.635954in}}%
\pgfpathlineto{\pgfqpoint{0.961364in}{0.635954in}}%
\pgfpathlineto{\pgfqpoint{0.961364in}{0.500000in}}%
\pgfpathclose%
\pgfusepath{fill}%
\end{pgfscope}%
\begin{pgfscope}%
\pgfpathrectangle{\pgfqpoint{0.750000in}{0.500000in}}{\pgfqpoint{4.650000in}{3.020000in}}%
\pgfusepath{clip}%
\pgfsetbuttcap%
\pgfsetmiterjoin%
\definecolor{currentfill}{rgb}{0.000000,0.500000,0.000000}%
\pgfsetfillcolor{currentfill}%
\pgfsetlinewidth{0.000000pt}%
\definecolor{currentstroke}{rgb}{0.000000,0.000000,0.000000}%
\pgfsetstrokecolor{currentstroke}%
\pgfsetstrokeopacity{0.000000}%
\pgfsetdash{}{0pt}%
\pgfpathmoveto{\pgfqpoint{0.994389in}{0.500000in}}%
\pgfpathlineto{\pgfqpoint{1.027415in}{0.500000in}}%
\pgfpathlineto{\pgfqpoint{1.027415in}{0.965266in}}%
\pgfpathlineto{\pgfqpoint{0.994389in}{0.965266in}}%
\pgfpathlineto{\pgfqpoint{0.994389in}{0.500000in}}%
\pgfpathclose%
\pgfusepath{fill}%
\end{pgfscope}%
\begin{pgfscope}%
\pgfpathrectangle{\pgfqpoint{0.750000in}{0.500000in}}{\pgfqpoint{4.650000in}{3.020000in}}%
\pgfusepath{clip}%
\pgfsetbuttcap%
\pgfsetmiterjoin%
\definecolor{currentfill}{rgb}{0.000000,0.500000,0.000000}%
\pgfsetfillcolor{currentfill}%
\pgfsetlinewidth{0.000000pt}%
\definecolor{currentstroke}{rgb}{0.000000,0.000000,0.000000}%
\pgfsetstrokecolor{currentstroke}%
\pgfsetstrokeopacity{0.000000}%
\pgfsetdash{}{0pt}%
\pgfpathmoveto{\pgfqpoint{1.027415in}{0.500000in}}%
\pgfpathlineto{\pgfqpoint{1.060440in}{0.500000in}}%
\pgfpathlineto{\pgfqpoint{1.060440in}{1.415426in}}%
\pgfpathlineto{\pgfqpoint{1.027415in}{1.415426in}}%
\pgfpathlineto{\pgfqpoint{1.027415in}{0.500000in}}%
\pgfpathclose%
\pgfusepath{fill}%
\end{pgfscope}%
\begin{pgfscope}%
\pgfpathrectangle{\pgfqpoint{0.750000in}{0.500000in}}{\pgfqpoint{4.650000in}{3.020000in}}%
\pgfusepath{clip}%
\pgfsetbuttcap%
\pgfsetmiterjoin%
\definecolor{currentfill}{rgb}{0.000000,0.500000,0.000000}%
\pgfsetfillcolor{currentfill}%
\pgfsetlinewidth{0.000000pt}%
\definecolor{currentstroke}{rgb}{0.000000,0.000000,0.000000}%
\pgfsetstrokecolor{currentstroke}%
\pgfsetstrokeopacity{0.000000}%
\pgfsetdash{}{0pt}%
\pgfpathmoveto{\pgfqpoint{1.060440in}{0.500000in}}%
\pgfpathlineto{\pgfqpoint{1.093466in}{0.500000in}}%
\pgfpathlineto{\pgfqpoint{1.093466in}{1.496999in}}%
\pgfpathlineto{\pgfqpoint{1.060440in}{1.496999in}}%
\pgfpathlineto{\pgfqpoint{1.060440in}{0.500000in}}%
\pgfpathclose%
\pgfusepath{fill}%
\end{pgfscope}%
\begin{pgfscope}%
\pgfpathrectangle{\pgfqpoint{0.750000in}{0.500000in}}{\pgfqpoint{4.650000in}{3.020000in}}%
\pgfusepath{clip}%
\pgfsetbuttcap%
\pgfsetmiterjoin%
\definecolor{currentfill}{rgb}{0.000000,0.500000,0.000000}%
\pgfsetfillcolor{currentfill}%
\pgfsetlinewidth{0.000000pt}%
\definecolor{currentstroke}{rgb}{0.000000,0.000000,0.000000}%
\pgfsetstrokecolor{currentstroke}%
\pgfsetstrokeopacity{0.000000}%
\pgfsetdash{}{0pt}%
\pgfpathmoveto{\pgfqpoint{1.093466in}{0.500000in}}%
\pgfpathlineto{\pgfqpoint{1.126491in}{0.500000in}}%
\pgfpathlineto{\pgfqpoint{1.126491in}{1.654102in}}%
\pgfpathlineto{\pgfqpoint{1.093466in}{1.654102in}}%
\pgfpathlineto{\pgfqpoint{1.093466in}{0.500000in}}%
\pgfpathclose%
\pgfusepath{fill}%
\end{pgfscope}%
\begin{pgfscope}%
\pgfpathrectangle{\pgfqpoint{0.750000in}{0.500000in}}{\pgfqpoint{4.650000in}{3.020000in}}%
\pgfusepath{clip}%
\pgfsetbuttcap%
\pgfsetmiterjoin%
\definecolor{currentfill}{rgb}{0.000000,0.500000,0.000000}%
\pgfsetfillcolor{currentfill}%
\pgfsetlinewidth{0.000000pt}%
\definecolor{currentstroke}{rgb}{0.000000,0.000000,0.000000}%
\pgfsetstrokecolor{currentstroke}%
\pgfsetstrokeopacity{0.000000}%
\pgfsetdash{}{0pt}%
\pgfpathmoveto{\pgfqpoint{1.126491in}{0.500000in}}%
\pgfpathlineto{\pgfqpoint{1.159517in}{0.500000in}}%
\pgfpathlineto{\pgfqpoint{1.159517in}{1.669208in}}%
\pgfpathlineto{\pgfqpoint{1.126491in}{1.669208in}}%
\pgfpathlineto{\pgfqpoint{1.126491in}{0.500000in}}%
\pgfpathclose%
\pgfusepath{fill}%
\end{pgfscope}%
\begin{pgfscope}%
\pgfpathrectangle{\pgfqpoint{0.750000in}{0.500000in}}{\pgfqpoint{4.650000in}{3.020000in}}%
\pgfusepath{clip}%
\pgfsetbuttcap%
\pgfsetmiterjoin%
\definecolor{currentfill}{rgb}{0.000000,0.500000,0.000000}%
\pgfsetfillcolor{currentfill}%
\pgfsetlinewidth{0.000000pt}%
\definecolor{currentstroke}{rgb}{0.000000,0.000000,0.000000}%
\pgfsetstrokecolor{currentstroke}%
\pgfsetstrokeopacity{0.000000}%
\pgfsetdash{}{0pt}%
\pgfpathmoveto{\pgfqpoint{1.159517in}{0.500000in}}%
\pgfpathlineto{\pgfqpoint{1.192543in}{0.500000in}}%
\pgfpathlineto{\pgfqpoint{1.192543in}{1.678271in}}%
\pgfpathlineto{\pgfqpoint{1.159517in}{1.678271in}}%
\pgfpathlineto{\pgfqpoint{1.159517in}{0.500000in}}%
\pgfpathclose%
\pgfusepath{fill}%
\end{pgfscope}%
\begin{pgfscope}%
\pgfpathrectangle{\pgfqpoint{0.750000in}{0.500000in}}{\pgfqpoint{4.650000in}{3.020000in}}%
\pgfusepath{clip}%
\pgfsetbuttcap%
\pgfsetmiterjoin%
\definecolor{currentfill}{rgb}{0.000000,0.500000,0.000000}%
\pgfsetfillcolor{currentfill}%
\pgfsetlinewidth{0.000000pt}%
\definecolor{currentstroke}{rgb}{0.000000,0.000000,0.000000}%
\pgfsetstrokecolor{currentstroke}%
\pgfsetstrokeopacity{0.000000}%
\pgfsetdash{}{0pt}%
\pgfpathmoveto{\pgfqpoint{1.192543in}{0.500000in}}%
\pgfpathlineto{\pgfqpoint{1.225568in}{0.500000in}}%
\pgfpathlineto{\pgfqpoint{1.225568in}{1.868607in}}%
\pgfpathlineto{\pgfqpoint{1.192543in}{1.868607in}}%
\pgfpathlineto{\pgfqpoint{1.192543in}{0.500000in}}%
\pgfpathclose%
\pgfusepath{fill}%
\end{pgfscope}%
\begin{pgfscope}%
\pgfpathrectangle{\pgfqpoint{0.750000in}{0.500000in}}{\pgfqpoint{4.650000in}{3.020000in}}%
\pgfusepath{clip}%
\pgfsetbuttcap%
\pgfsetmiterjoin%
\definecolor{currentfill}{rgb}{0.000000,0.500000,0.000000}%
\pgfsetfillcolor{currentfill}%
\pgfsetlinewidth{0.000000pt}%
\definecolor{currentstroke}{rgb}{0.000000,0.000000,0.000000}%
\pgfsetstrokecolor{currentstroke}%
\pgfsetstrokeopacity{0.000000}%
\pgfsetdash{}{0pt}%
\pgfpathmoveto{\pgfqpoint{1.225568in}{0.500000in}}%
\pgfpathlineto{\pgfqpoint{1.258594in}{0.500000in}}%
\pgfpathlineto{\pgfqpoint{1.258594in}{1.602741in}}%
\pgfpathlineto{\pgfqpoint{1.225568in}{1.602741in}}%
\pgfpathlineto{\pgfqpoint{1.225568in}{0.500000in}}%
\pgfpathclose%
\pgfusepath{fill}%
\end{pgfscope}%
\begin{pgfscope}%
\pgfpathrectangle{\pgfqpoint{0.750000in}{0.500000in}}{\pgfqpoint{4.650000in}{3.020000in}}%
\pgfusepath{clip}%
\pgfsetbuttcap%
\pgfsetmiterjoin%
\definecolor{currentfill}{rgb}{0.000000,0.500000,0.000000}%
\pgfsetfillcolor{currentfill}%
\pgfsetlinewidth{0.000000pt}%
\definecolor{currentstroke}{rgb}{0.000000,0.000000,0.000000}%
\pgfsetstrokecolor{currentstroke}%
\pgfsetstrokeopacity{0.000000}%
\pgfsetdash{}{0pt}%
\pgfpathmoveto{\pgfqpoint{1.258594in}{0.500000in}}%
\pgfpathlineto{\pgfqpoint{1.291619in}{0.500000in}}%
\pgfpathlineto{\pgfqpoint{1.291619in}{1.629932in}}%
\pgfpathlineto{\pgfqpoint{1.258594in}{1.629932in}}%
\pgfpathlineto{\pgfqpoint{1.258594in}{0.500000in}}%
\pgfpathclose%
\pgfusepath{fill}%
\end{pgfscope}%
\begin{pgfscope}%
\pgfpathrectangle{\pgfqpoint{0.750000in}{0.500000in}}{\pgfqpoint{4.650000in}{3.020000in}}%
\pgfusepath{clip}%
\pgfsetbuttcap%
\pgfsetmiterjoin%
\definecolor{currentfill}{rgb}{0.000000,0.500000,0.000000}%
\pgfsetfillcolor{currentfill}%
\pgfsetlinewidth{0.000000pt}%
\definecolor{currentstroke}{rgb}{0.000000,0.000000,0.000000}%
\pgfsetstrokecolor{currentstroke}%
\pgfsetstrokeopacity{0.000000}%
\pgfsetdash{}{0pt}%
\pgfpathmoveto{\pgfqpoint{1.291619in}{0.500000in}}%
\pgfpathlineto{\pgfqpoint{1.324645in}{0.500000in}}%
\pgfpathlineto{\pgfqpoint{1.324645in}{1.539296in}}%
\pgfpathlineto{\pgfqpoint{1.291619in}{1.539296in}}%
\pgfpathlineto{\pgfqpoint{1.291619in}{0.500000in}}%
\pgfpathclose%
\pgfusepath{fill}%
\end{pgfscope}%
\begin{pgfscope}%
\pgfpathrectangle{\pgfqpoint{0.750000in}{0.500000in}}{\pgfqpoint{4.650000in}{3.020000in}}%
\pgfusepath{clip}%
\pgfsetbuttcap%
\pgfsetmiterjoin%
\definecolor{currentfill}{rgb}{0.000000,0.500000,0.000000}%
\pgfsetfillcolor{currentfill}%
\pgfsetlinewidth{0.000000pt}%
\definecolor{currentstroke}{rgb}{0.000000,0.000000,0.000000}%
\pgfsetstrokecolor{currentstroke}%
\pgfsetstrokeopacity{0.000000}%
\pgfsetdash{}{0pt}%
\pgfpathmoveto{\pgfqpoint{1.324645in}{0.500000in}}%
\pgfpathlineto{\pgfqpoint{1.357670in}{0.500000in}}%
\pgfpathlineto{\pgfqpoint{1.357670in}{1.527211in}}%
\pgfpathlineto{\pgfqpoint{1.324645in}{1.527211in}}%
\pgfpathlineto{\pgfqpoint{1.324645in}{0.500000in}}%
\pgfpathclose%
\pgfusepath{fill}%
\end{pgfscope}%
\begin{pgfscope}%
\pgfpathrectangle{\pgfqpoint{0.750000in}{0.500000in}}{\pgfqpoint{4.650000in}{3.020000in}}%
\pgfusepath{clip}%
\pgfsetbuttcap%
\pgfsetmiterjoin%
\definecolor{currentfill}{rgb}{0.000000,0.500000,0.000000}%
\pgfsetfillcolor{currentfill}%
\pgfsetlinewidth{0.000000pt}%
\definecolor{currentstroke}{rgb}{0.000000,0.000000,0.000000}%
\pgfsetstrokecolor{currentstroke}%
\pgfsetstrokeopacity{0.000000}%
\pgfsetdash{}{0pt}%
\pgfpathmoveto{\pgfqpoint{1.357670in}{0.500000in}}%
\pgfpathlineto{\pgfqpoint{1.390696in}{0.500000in}}%
\pgfpathlineto{\pgfqpoint{1.390696in}{1.336875in}}%
\pgfpathlineto{\pgfqpoint{1.357670in}{1.336875in}}%
\pgfpathlineto{\pgfqpoint{1.357670in}{0.500000in}}%
\pgfpathclose%
\pgfusepath{fill}%
\end{pgfscope}%
\begin{pgfscope}%
\pgfpathrectangle{\pgfqpoint{0.750000in}{0.500000in}}{\pgfqpoint{4.650000in}{3.020000in}}%
\pgfusepath{clip}%
\pgfsetbuttcap%
\pgfsetmiterjoin%
\definecolor{currentfill}{rgb}{0.000000,0.500000,0.000000}%
\pgfsetfillcolor{currentfill}%
\pgfsetlinewidth{0.000000pt}%
\definecolor{currentstroke}{rgb}{0.000000,0.000000,0.000000}%
\pgfsetstrokecolor{currentstroke}%
\pgfsetstrokeopacity{0.000000}%
\pgfsetdash{}{0pt}%
\pgfpathmoveto{\pgfqpoint{1.390696in}{0.500000in}}%
\pgfpathlineto{\pgfqpoint{1.423722in}{0.500000in}}%
\pgfpathlineto{\pgfqpoint{1.423722in}{1.351981in}}%
\pgfpathlineto{\pgfqpoint{1.390696in}{1.351981in}}%
\pgfpathlineto{\pgfqpoint{1.390696in}{0.500000in}}%
\pgfpathclose%
\pgfusepath{fill}%
\end{pgfscope}%
\begin{pgfscope}%
\pgfpathrectangle{\pgfqpoint{0.750000in}{0.500000in}}{\pgfqpoint{4.650000in}{3.020000in}}%
\pgfusepath{clip}%
\pgfsetbuttcap%
\pgfsetmiterjoin%
\definecolor{currentfill}{rgb}{0.000000,0.500000,0.000000}%
\pgfsetfillcolor{currentfill}%
\pgfsetlinewidth{0.000000pt}%
\definecolor{currentstroke}{rgb}{0.000000,0.000000,0.000000}%
\pgfsetstrokecolor{currentstroke}%
\pgfsetstrokeopacity{0.000000}%
\pgfsetdash{}{0pt}%
\pgfpathmoveto{\pgfqpoint{1.423722in}{0.500000in}}%
\pgfpathlineto{\pgfqpoint{1.456747in}{0.500000in}}%
\pgfpathlineto{\pgfqpoint{1.456747in}{1.255302in}}%
\pgfpathlineto{\pgfqpoint{1.423722in}{1.255302in}}%
\pgfpathlineto{\pgfqpoint{1.423722in}{0.500000in}}%
\pgfpathclose%
\pgfusepath{fill}%
\end{pgfscope}%
\begin{pgfscope}%
\pgfpathrectangle{\pgfqpoint{0.750000in}{0.500000in}}{\pgfqpoint{4.650000in}{3.020000in}}%
\pgfusepath{clip}%
\pgfsetbuttcap%
\pgfsetmiterjoin%
\definecolor{currentfill}{rgb}{0.000000,0.500000,0.000000}%
\pgfsetfillcolor{currentfill}%
\pgfsetlinewidth{0.000000pt}%
\definecolor{currentstroke}{rgb}{0.000000,0.000000,0.000000}%
\pgfsetstrokecolor{currentstroke}%
\pgfsetstrokeopacity{0.000000}%
\pgfsetdash{}{0pt}%
\pgfpathmoveto{\pgfqpoint{1.456747in}{0.500000in}}%
\pgfpathlineto{\pgfqpoint{1.489773in}{0.500000in}}%
\pgfpathlineto{\pgfqpoint{1.489773in}{1.209984in}}%
\pgfpathlineto{\pgfqpoint{1.456747in}{1.209984in}}%
\pgfpathlineto{\pgfqpoint{1.456747in}{0.500000in}}%
\pgfpathclose%
\pgfusepath{fill}%
\end{pgfscope}%
\begin{pgfscope}%
\pgfpathrectangle{\pgfqpoint{0.750000in}{0.500000in}}{\pgfqpoint{4.650000in}{3.020000in}}%
\pgfusepath{clip}%
\pgfsetbuttcap%
\pgfsetmiterjoin%
\definecolor{currentfill}{rgb}{0.000000,0.500000,0.000000}%
\pgfsetfillcolor{currentfill}%
\pgfsetlinewidth{0.000000pt}%
\definecolor{currentstroke}{rgb}{0.000000,0.000000,0.000000}%
\pgfsetstrokecolor{currentstroke}%
\pgfsetstrokeopacity{0.000000}%
\pgfsetdash{}{0pt}%
\pgfpathmoveto{\pgfqpoint{1.489773in}{0.500000in}}%
\pgfpathlineto{\pgfqpoint{1.522798in}{0.500000in}}%
\pgfpathlineto{\pgfqpoint{1.522798in}{1.222069in}}%
\pgfpathlineto{\pgfqpoint{1.489773in}{1.222069in}}%
\pgfpathlineto{\pgfqpoint{1.489773in}{0.500000in}}%
\pgfpathclose%
\pgfusepath{fill}%
\end{pgfscope}%
\begin{pgfscope}%
\pgfpathrectangle{\pgfqpoint{0.750000in}{0.500000in}}{\pgfqpoint{4.650000in}{3.020000in}}%
\pgfusepath{clip}%
\pgfsetbuttcap%
\pgfsetmiterjoin%
\definecolor{currentfill}{rgb}{0.000000,0.500000,0.000000}%
\pgfsetfillcolor{currentfill}%
\pgfsetlinewidth{0.000000pt}%
\definecolor{currentstroke}{rgb}{0.000000,0.000000,0.000000}%
\pgfsetstrokecolor{currentstroke}%
\pgfsetstrokeopacity{0.000000}%
\pgfsetdash{}{0pt}%
\pgfpathmoveto{\pgfqpoint{1.522798in}{0.500000in}}%
\pgfpathlineto{\pgfqpoint{1.555824in}{0.500000in}}%
\pgfpathlineto{\pgfqpoint{1.555824in}{1.152581in}}%
\pgfpathlineto{\pgfqpoint{1.522798in}{1.152581in}}%
\pgfpathlineto{\pgfqpoint{1.522798in}{0.500000in}}%
\pgfpathclose%
\pgfusepath{fill}%
\end{pgfscope}%
\begin{pgfscope}%
\pgfpathrectangle{\pgfqpoint{0.750000in}{0.500000in}}{\pgfqpoint{4.650000in}{3.020000in}}%
\pgfusepath{clip}%
\pgfsetbuttcap%
\pgfsetmiterjoin%
\definecolor{currentfill}{rgb}{0.000000,0.500000,0.000000}%
\pgfsetfillcolor{currentfill}%
\pgfsetlinewidth{0.000000pt}%
\definecolor{currentstroke}{rgb}{0.000000,0.000000,0.000000}%
\pgfsetstrokecolor{currentstroke}%
\pgfsetstrokeopacity{0.000000}%
\pgfsetdash{}{0pt}%
\pgfpathmoveto{\pgfqpoint{1.555824in}{0.500000in}}%
\pgfpathlineto{\pgfqpoint{1.588849in}{0.500000in}}%
\pgfpathlineto{\pgfqpoint{1.588849in}{1.149560in}}%
\pgfpathlineto{\pgfqpoint{1.555824in}{1.149560in}}%
\pgfpathlineto{\pgfqpoint{1.555824in}{0.500000in}}%
\pgfpathclose%
\pgfusepath{fill}%
\end{pgfscope}%
\begin{pgfscope}%
\pgfpathrectangle{\pgfqpoint{0.750000in}{0.500000in}}{\pgfqpoint{4.650000in}{3.020000in}}%
\pgfusepath{clip}%
\pgfsetbuttcap%
\pgfsetmiterjoin%
\definecolor{currentfill}{rgb}{0.000000,0.500000,0.000000}%
\pgfsetfillcolor{currentfill}%
\pgfsetlinewidth{0.000000pt}%
\definecolor{currentstroke}{rgb}{0.000000,0.000000,0.000000}%
\pgfsetstrokecolor{currentstroke}%
\pgfsetstrokeopacity{0.000000}%
\pgfsetdash{}{0pt}%
\pgfpathmoveto{\pgfqpoint{1.588849in}{0.500000in}}%
\pgfpathlineto{\pgfqpoint{1.621875in}{0.500000in}}%
\pgfpathlineto{\pgfqpoint{1.621875in}{0.995478in}}%
\pgfpathlineto{\pgfqpoint{1.588849in}{0.995478in}}%
\pgfpathlineto{\pgfqpoint{1.588849in}{0.500000in}}%
\pgfpathclose%
\pgfusepath{fill}%
\end{pgfscope}%
\begin{pgfscope}%
\pgfpathrectangle{\pgfqpoint{0.750000in}{0.500000in}}{\pgfqpoint{4.650000in}{3.020000in}}%
\pgfusepath{clip}%
\pgfsetbuttcap%
\pgfsetmiterjoin%
\definecolor{currentfill}{rgb}{0.000000,0.500000,0.000000}%
\pgfsetfillcolor{currentfill}%
\pgfsetlinewidth{0.000000pt}%
\definecolor{currentstroke}{rgb}{0.000000,0.000000,0.000000}%
\pgfsetstrokecolor{currentstroke}%
\pgfsetstrokeopacity{0.000000}%
\pgfsetdash{}{0pt}%
\pgfpathmoveto{\pgfqpoint{1.621875in}{0.500000in}}%
\pgfpathlineto{\pgfqpoint{1.654901in}{0.500000in}}%
\pgfpathlineto{\pgfqpoint{1.654901in}{1.061945in}}%
\pgfpathlineto{\pgfqpoint{1.621875in}{1.061945in}}%
\pgfpathlineto{\pgfqpoint{1.621875in}{0.500000in}}%
\pgfpathclose%
\pgfusepath{fill}%
\end{pgfscope}%
\begin{pgfscope}%
\pgfpathrectangle{\pgfqpoint{0.750000in}{0.500000in}}{\pgfqpoint{4.650000in}{3.020000in}}%
\pgfusepath{clip}%
\pgfsetbuttcap%
\pgfsetmiterjoin%
\definecolor{currentfill}{rgb}{0.000000,0.500000,0.000000}%
\pgfsetfillcolor{currentfill}%
\pgfsetlinewidth{0.000000pt}%
\definecolor{currentstroke}{rgb}{0.000000,0.000000,0.000000}%
\pgfsetstrokecolor{currentstroke}%
\pgfsetstrokeopacity{0.000000}%
\pgfsetdash{}{0pt}%
\pgfpathmoveto{\pgfqpoint{1.654901in}{0.500000in}}%
\pgfpathlineto{\pgfqpoint{1.687926in}{0.500000in}}%
\pgfpathlineto{\pgfqpoint{1.687926in}{1.089136in}}%
\pgfpathlineto{\pgfqpoint{1.654901in}{1.089136in}}%
\pgfpathlineto{\pgfqpoint{1.654901in}{0.500000in}}%
\pgfpathclose%
\pgfusepath{fill}%
\end{pgfscope}%
\begin{pgfscope}%
\pgfpathrectangle{\pgfqpoint{0.750000in}{0.500000in}}{\pgfqpoint{4.650000in}{3.020000in}}%
\pgfusepath{clip}%
\pgfsetbuttcap%
\pgfsetmiterjoin%
\definecolor{currentfill}{rgb}{0.000000,0.500000,0.000000}%
\pgfsetfillcolor{currentfill}%
\pgfsetlinewidth{0.000000pt}%
\definecolor{currentstroke}{rgb}{0.000000,0.000000,0.000000}%
\pgfsetstrokecolor{currentstroke}%
\pgfsetstrokeopacity{0.000000}%
\pgfsetdash{}{0pt}%
\pgfpathmoveto{\pgfqpoint{1.687926in}{0.500000in}}%
\pgfpathlineto{\pgfqpoint{1.720952in}{0.500000in}}%
\pgfpathlineto{\pgfqpoint{1.720952in}{0.953181in}}%
\pgfpathlineto{\pgfqpoint{1.687926in}{0.953181in}}%
\pgfpathlineto{\pgfqpoint{1.687926in}{0.500000in}}%
\pgfpathclose%
\pgfusepath{fill}%
\end{pgfscope}%
\begin{pgfscope}%
\pgfpathrectangle{\pgfqpoint{0.750000in}{0.500000in}}{\pgfqpoint{4.650000in}{3.020000in}}%
\pgfusepath{clip}%
\pgfsetbuttcap%
\pgfsetmiterjoin%
\definecolor{currentfill}{rgb}{0.000000,0.500000,0.000000}%
\pgfsetfillcolor{currentfill}%
\pgfsetlinewidth{0.000000pt}%
\definecolor{currentstroke}{rgb}{0.000000,0.000000,0.000000}%
\pgfsetstrokecolor{currentstroke}%
\pgfsetstrokeopacity{0.000000}%
\pgfsetdash{}{0pt}%
\pgfpathmoveto{\pgfqpoint{1.720952in}{0.500000in}}%
\pgfpathlineto{\pgfqpoint{1.753977in}{0.500000in}}%
\pgfpathlineto{\pgfqpoint{1.753977in}{1.004542in}}%
\pgfpathlineto{\pgfqpoint{1.720952in}{1.004542in}}%
\pgfpathlineto{\pgfqpoint{1.720952in}{0.500000in}}%
\pgfpathclose%
\pgfusepath{fill}%
\end{pgfscope}%
\begin{pgfscope}%
\pgfpathrectangle{\pgfqpoint{0.750000in}{0.500000in}}{\pgfqpoint{4.650000in}{3.020000in}}%
\pgfusepath{clip}%
\pgfsetbuttcap%
\pgfsetmiterjoin%
\definecolor{currentfill}{rgb}{0.000000,0.500000,0.000000}%
\pgfsetfillcolor{currentfill}%
\pgfsetlinewidth{0.000000pt}%
\definecolor{currentstroke}{rgb}{0.000000,0.000000,0.000000}%
\pgfsetstrokecolor{currentstroke}%
\pgfsetstrokeopacity{0.000000}%
\pgfsetdash{}{0pt}%
\pgfpathmoveto{\pgfqpoint{1.753977in}{0.500000in}}%
\pgfpathlineto{\pgfqpoint{1.787003in}{0.500000in}}%
\pgfpathlineto{\pgfqpoint{1.787003in}{0.995478in}}%
\pgfpathlineto{\pgfqpoint{1.753977in}{0.995478in}}%
\pgfpathlineto{\pgfqpoint{1.753977in}{0.500000in}}%
\pgfpathclose%
\pgfusepath{fill}%
\end{pgfscope}%
\begin{pgfscope}%
\pgfpathrectangle{\pgfqpoint{0.750000in}{0.500000in}}{\pgfqpoint{4.650000in}{3.020000in}}%
\pgfusepath{clip}%
\pgfsetbuttcap%
\pgfsetmiterjoin%
\definecolor{currentfill}{rgb}{0.000000,0.500000,0.000000}%
\pgfsetfillcolor{currentfill}%
\pgfsetlinewidth{0.000000pt}%
\definecolor{currentstroke}{rgb}{0.000000,0.000000,0.000000}%
\pgfsetstrokecolor{currentstroke}%
\pgfsetstrokeopacity{0.000000}%
\pgfsetdash{}{0pt}%
\pgfpathmoveto{\pgfqpoint{1.787003in}{0.500000in}}%
\pgfpathlineto{\pgfqpoint{1.820028in}{0.500000in}}%
\pgfpathlineto{\pgfqpoint{1.820028in}{0.901821in}}%
\pgfpathlineto{\pgfqpoint{1.787003in}{0.901821in}}%
\pgfpathlineto{\pgfqpoint{1.787003in}{0.500000in}}%
\pgfpathclose%
\pgfusepath{fill}%
\end{pgfscope}%
\begin{pgfscope}%
\pgfpathrectangle{\pgfqpoint{0.750000in}{0.500000in}}{\pgfqpoint{4.650000in}{3.020000in}}%
\pgfusepath{clip}%
\pgfsetbuttcap%
\pgfsetmiterjoin%
\definecolor{currentfill}{rgb}{0.000000,0.500000,0.000000}%
\pgfsetfillcolor{currentfill}%
\pgfsetlinewidth{0.000000pt}%
\definecolor{currentstroke}{rgb}{0.000000,0.000000,0.000000}%
\pgfsetstrokecolor{currentstroke}%
\pgfsetstrokeopacity{0.000000}%
\pgfsetdash{}{0pt}%
\pgfpathmoveto{\pgfqpoint{1.820028in}{0.500000in}}%
\pgfpathlineto{\pgfqpoint{1.853054in}{0.500000in}}%
\pgfpathlineto{\pgfqpoint{1.853054in}{0.880672in}}%
\pgfpathlineto{\pgfqpoint{1.820028in}{0.880672in}}%
\pgfpathlineto{\pgfqpoint{1.820028in}{0.500000in}}%
\pgfpathclose%
\pgfusepath{fill}%
\end{pgfscope}%
\begin{pgfscope}%
\pgfpathrectangle{\pgfqpoint{0.750000in}{0.500000in}}{\pgfqpoint{4.650000in}{3.020000in}}%
\pgfusepath{clip}%
\pgfsetbuttcap%
\pgfsetmiterjoin%
\definecolor{currentfill}{rgb}{0.000000,0.500000,0.000000}%
\pgfsetfillcolor{currentfill}%
\pgfsetlinewidth{0.000000pt}%
\definecolor{currentstroke}{rgb}{0.000000,0.000000,0.000000}%
\pgfsetstrokecolor{currentstroke}%
\pgfsetstrokeopacity{0.000000}%
\pgfsetdash{}{0pt}%
\pgfpathmoveto{\pgfqpoint{1.853054in}{0.500000in}}%
\pgfpathlineto{\pgfqpoint{1.886080in}{0.500000in}}%
\pgfpathlineto{\pgfqpoint{1.886080in}{0.907863in}}%
\pgfpathlineto{\pgfqpoint{1.853054in}{0.907863in}}%
\pgfpathlineto{\pgfqpoint{1.853054in}{0.500000in}}%
\pgfpathclose%
\pgfusepath{fill}%
\end{pgfscope}%
\begin{pgfscope}%
\pgfpathrectangle{\pgfqpoint{0.750000in}{0.500000in}}{\pgfqpoint{4.650000in}{3.020000in}}%
\pgfusepath{clip}%
\pgfsetbuttcap%
\pgfsetmiterjoin%
\definecolor{currentfill}{rgb}{0.000000,0.500000,0.000000}%
\pgfsetfillcolor{currentfill}%
\pgfsetlinewidth{0.000000pt}%
\definecolor{currentstroke}{rgb}{0.000000,0.000000,0.000000}%
\pgfsetstrokecolor{currentstroke}%
\pgfsetstrokeopacity{0.000000}%
\pgfsetdash{}{0pt}%
\pgfpathmoveto{\pgfqpoint{1.886080in}{0.500000in}}%
\pgfpathlineto{\pgfqpoint{1.919105in}{0.500000in}}%
\pgfpathlineto{\pgfqpoint{1.919105in}{0.841397in}}%
\pgfpathlineto{\pgfqpoint{1.886080in}{0.841397in}}%
\pgfpathlineto{\pgfqpoint{1.886080in}{0.500000in}}%
\pgfpathclose%
\pgfusepath{fill}%
\end{pgfscope}%
\begin{pgfscope}%
\pgfpathrectangle{\pgfqpoint{0.750000in}{0.500000in}}{\pgfqpoint{4.650000in}{3.020000in}}%
\pgfusepath{clip}%
\pgfsetbuttcap%
\pgfsetmiterjoin%
\definecolor{currentfill}{rgb}{0.000000,0.500000,0.000000}%
\pgfsetfillcolor{currentfill}%
\pgfsetlinewidth{0.000000pt}%
\definecolor{currentstroke}{rgb}{0.000000,0.000000,0.000000}%
\pgfsetstrokecolor{currentstroke}%
\pgfsetstrokeopacity{0.000000}%
\pgfsetdash{}{0pt}%
\pgfpathmoveto{\pgfqpoint{1.919105in}{0.500000in}}%
\pgfpathlineto{\pgfqpoint{1.952131in}{0.500000in}}%
\pgfpathlineto{\pgfqpoint{1.952131in}{0.889736in}}%
\pgfpathlineto{\pgfqpoint{1.919105in}{0.889736in}}%
\pgfpathlineto{\pgfqpoint{1.919105in}{0.500000in}}%
\pgfpathclose%
\pgfusepath{fill}%
\end{pgfscope}%
\begin{pgfscope}%
\pgfpathrectangle{\pgfqpoint{0.750000in}{0.500000in}}{\pgfqpoint{4.650000in}{3.020000in}}%
\pgfusepath{clip}%
\pgfsetbuttcap%
\pgfsetmiterjoin%
\definecolor{currentfill}{rgb}{0.000000,0.500000,0.000000}%
\pgfsetfillcolor{currentfill}%
\pgfsetlinewidth{0.000000pt}%
\definecolor{currentstroke}{rgb}{0.000000,0.000000,0.000000}%
\pgfsetstrokecolor{currentstroke}%
\pgfsetstrokeopacity{0.000000}%
\pgfsetdash{}{0pt}%
\pgfpathmoveto{\pgfqpoint{1.952131in}{0.500000in}}%
\pgfpathlineto{\pgfqpoint{1.985156in}{0.500000in}}%
\pgfpathlineto{\pgfqpoint{1.985156in}{0.808163in}}%
\pgfpathlineto{\pgfqpoint{1.952131in}{0.808163in}}%
\pgfpathlineto{\pgfqpoint{1.952131in}{0.500000in}}%
\pgfpathclose%
\pgfusepath{fill}%
\end{pgfscope}%
\begin{pgfscope}%
\pgfpathrectangle{\pgfqpoint{0.750000in}{0.500000in}}{\pgfqpoint{4.650000in}{3.020000in}}%
\pgfusepath{clip}%
\pgfsetbuttcap%
\pgfsetmiterjoin%
\definecolor{currentfill}{rgb}{0.000000,0.500000,0.000000}%
\pgfsetfillcolor{currentfill}%
\pgfsetlinewidth{0.000000pt}%
\definecolor{currentstroke}{rgb}{0.000000,0.000000,0.000000}%
\pgfsetstrokecolor{currentstroke}%
\pgfsetstrokeopacity{0.000000}%
\pgfsetdash{}{0pt}%
\pgfpathmoveto{\pgfqpoint{1.985156in}{0.500000in}}%
\pgfpathlineto{\pgfqpoint{2.018182in}{0.500000in}}%
\pgfpathlineto{\pgfqpoint{2.018182in}{0.823269in}}%
\pgfpathlineto{\pgfqpoint{1.985156in}{0.823269in}}%
\pgfpathlineto{\pgfqpoint{1.985156in}{0.500000in}}%
\pgfpathclose%
\pgfusepath{fill}%
\end{pgfscope}%
\begin{pgfscope}%
\pgfpathrectangle{\pgfqpoint{0.750000in}{0.500000in}}{\pgfqpoint{4.650000in}{3.020000in}}%
\pgfusepath{clip}%
\pgfsetbuttcap%
\pgfsetmiterjoin%
\definecolor{currentfill}{rgb}{0.000000,0.500000,0.000000}%
\pgfsetfillcolor{currentfill}%
\pgfsetlinewidth{0.000000pt}%
\definecolor{currentstroke}{rgb}{0.000000,0.000000,0.000000}%
\pgfsetstrokecolor{currentstroke}%
\pgfsetstrokeopacity{0.000000}%
\pgfsetdash{}{0pt}%
\pgfpathmoveto{\pgfqpoint{2.018182in}{0.500000in}}%
\pgfpathlineto{\pgfqpoint{2.051207in}{0.500000in}}%
\pgfpathlineto{\pgfqpoint{2.051207in}{0.811184in}}%
\pgfpathlineto{\pgfqpoint{2.018182in}{0.811184in}}%
\pgfpathlineto{\pgfqpoint{2.018182in}{0.500000in}}%
\pgfpathclose%
\pgfusepath{fill}%
\end{pgfscope}%
\begin{pgfscope}%
\pgfpathrectangle{\pgfqpoint{0.750000in}{0.500000in}}{\pgfqpoint{4.650000in}{3.020000in}}%
\pgfusepath{clip}%
\pgfsetbuttcap%
\pgfsetmiterjoin%
\definecolor{currentfill}{rgb}{0.000000,0.500000,0.000000}%
\pgfsetfillcolor{currentfill}%
\pgfsetlinewidth{0.000000pt}%
\definecolor{currentstroke}{rgb}{0.000000,0.000000,0.000000}%
\pgfsetstrokecolor{currentstroke}%
\pgfsetstrokeopacity{0.000000}%
\pgfsetdash{}{0pt}%
\pgfpathmoveto{\pgfqpoint{2.051207in}{0.500000in}}%
\pgfpathlineto{\pgfqpoint{2.084233in}{0.500000in}}%
\pgfpathlineto{\pgfqpoint{2.084233in}{0.787015in}}%
\pgfpathlineto{\pgfqpoint{2.051207in}{0.787015in}}%
\pgfpathlineto{\pgfqpoint{2.051207in}{0.500000in}}%
\pgfpathclose%
\pgfusepath{fill}%
\end{pgfscope}%
\begin{pgfscope}%
\pgfpathrectangle{\pgfqpoint{0.750000in}{0.500000in}}{\pgfqpoint{4.650000in}{3.020000in}}%
\pgfusepath{clip}%
\pgfsetbuttcap%
\pgfsetmiterjoin%
\definecolor{currentfill}{rgb}{0.000000,0.500000,0.000000}%
\pgfsetfillcolor{currentfill}%
\pgfsetlinewidth{0.000000pt}%
\definecolor{currentstroke}{rgb}{0.000000,0.000000,0.000000}%
\pgfsetstrokecolor{currentstroke}%
\pgfsetstrokeopacity{0.000000}%
\pgfsetdash{}{0pt}%
\pgfpathmoveto{\pgfqpoint{2.084233in}{0.500000in}}%
\pgfpathlineto{\pgfqpoint{2.117259in}{0.500000in}}%
\pgfpathlineto{\pgfqpoint{2.117259in}{0.723569in}}%
\pgfpathlineto{\pgfqpoint{2.084233in}{0.723569in}}%
\pgfpathlineto{\pgfqpoint{2.084233in}{0.500000in}}%
\pgfpathclose%
\pgfusepath{fill}%
\end{pgfscope}%
\begin{pgfscope}%
\pgfpathrectangle{\pgfqpoint{0.750000in}{0.500000in}}{\pgfqpoint{4.650000in}{3.020000in}}%
\pgfusepath{clip}%
\pgfsetbuttcap%
\pgfsetmiterjoin%
\definecolor{currentfill}{rgb}{0.000000,0.500000,0.000000}%
\pgfsetfillcolor{currentfill}%
\pgfsetlinewidth{0.000000pt}%
\definecolor{currentstroke}{rgb}{0.000000,0.000000,0.000000}%
\pgfsetstrokecolor{currentstroke}%
\pgfsetstrokeopacity{0.000000}%
\pgfsetdash{}{0pt}%
\pgfpathmoveto{\pgfqpoint{2.117259in}{0.500000in}}%
\pgfpathlineto{\pgfqpoint{2.150284in}{0.500000in}}%
\pgfpathlineto{\pgfqpoint{2.150284in}{0.796078in}}%
\pgfpathlineto{\pgfqpoint{2.117259in}{0.796078in}}%
\pgfpathlineto{\pgfqpoint{2.117259in}{0.500000in}}%
\pgfpathclose%
\pgfusepath{fill}%
\end{pgfscope}%
\begin{pgfscope}%
\pgfpathrectangle{\pgfqpoint{0.750000in}{0.500000in}}{\pgfqpoint{4.650000in}{3.020000in}}%
\pgfusepath{clip}%
\pgfsetbuttcap%
\pgfsetmiterjoin%
\definecolor{currentfill}{rgb}{0.000000,0.500000,0.000000}%
\pgfsetfillcolor{currentfill}%
\pgfsetlinewidth{0.000000pt}%
\definecolor{currentstroke}{rgb}{0.000000,0.000000,0.000000}%
\pgfsetstrokecolor{currentstroke}%
\pgfsetstrokeopacity{0.000000}%
\pgfsetdash{}{0pt}%
\pgfpathmoveto{\pgfqpoint{2.150284in}{0.500000in}}%
\pgfpathlineto{\pgfqpoint{2.183310in}{0.500000in}}%
\pgfpathlineto{\pgfqpoint{2.183310in}{0.735654in}}%
\pgfpathlineto{\pgfqpoint{2.150284in}{0.735654in}}%
\pgfpathlineto{\pgfqpoint{2.150284in}{0.500000in}}%
\pgfpathclose%
\pgfusepath{fill}%
\end{pgfscope}%
\begin{pgfscope}%
\pgfpathrectangle{\pgfqpoint{0.750000in}{0.500000in}}{\pgfqpoint{4.650000in}{3.020000in}}%
\pgfusepath{clip}%
\pgfsetbuttcap%
\pgfsetmiterjoin%
\definecolor{currentfill}{rgb}{0.000000,0.500000,0.000000}%
\pgfsetfillcolor{currentfill}%
\pgfsetlinewidth{0.000000pt}%
\definecolor{currentstroke}{rgb}{0.000000,0.000000,0.000000}%
\pgfsetstrokecolor{currentstroke}%
\pgfsetstrokeopacity{0.000000}%
\pgfsetdash{}{0pt}%
\pgfpathmoveto{\pgfqpoint{2.183310in}{0.500000in}}%
\pgfpathlineto{\pgfqpoint{2.216335in}{0.500000in}}%
\pgfpathlineto{\pgfqpoint{2.216335in}{0.765866in}}%
\pgfpathlineto{\pgfqpoint{2.183310in}{0.765866in}}%
\pgfpathlineto{\pgfqpoint{2.183310in}{0.500000in}}%
\pgfpathclose%
\pgfusepath{fill}%
\end{pgfscope}%
\begin{pgfscope}%
\pgfpathrectangle{\pgfqpoint{0.750000in}{0.500000in}}{\pgfqpoint{4.650000in}{3.020000in}}%
\pgfusepath{clip}%
\pgfsetbuttcap%
\pgfsetmiterjoin%
\definecolor{currentfill}{rgb}{0.000000,0.500000,0.000000}%
\pgfsetfillcolor{currentfill}%
\pgfsetlinewidth{0.000000pt}%
\definecolor{currentstroke}{rgb}{0.000000,0.000000,0.000000}%
\pgfsetstrokecolor{currentstroke}%
\pgfsetstrokeopacity{0.000000}%
\pgfsetdash{}{0pt}%
\pgfpathmoveto{\pgfqpoint{2.216335in}{0.500000in}}%
\pgfpathlineto{\pgfqpoint{2.249361in}{0.500000in}}%
\pgfpathlineto{\pgfqpoint{2.249361in}{0.741697in}}%
\pgfpathlineto{\pgfqpoint{2.216335in}{0.741697in}}%
\pgfpathlineto{\pgfqpoint{2.216335in}{0.500000in}}%
\pgfpathclose%
\pgfusepath{fill}%
\end{pgfscope}%
\begin{pgfscope}%
\pgfpathrectangle{\pgfqpoint{0.750000in}{0.500000in}}{\pgfqpoint{4.650000in}{3.020000in}}%
\pgfusepath{clip}%
\pgfsetbuttcap%
\pgfsetmiterjoin%
\definecolor{currentfill}{rgb}{0.000000,0.500000,0.000000}%
\pgfsetfillcolor{currentfill}%
\pgfsetlinewidth{0.000000pt}%
\definecolor{currentstroke}{rgb}{0.000000,0.000000,0.000000}%
\pgfsetstrokecolor{currentstroke}%
\pgfsetstrokeopacity{0.000000}%
\pgfsetdash{}{0pt}%
\pgfpathmoveto{\pgfqpoint{2.249361in}{0.500000in}}%
\pgfpathlineto{\pgfqpoint{2.282386in}{0.500000in}}%
\pgfpathlineto{\pgfqpoint{2.282386in}{0.711485in}}%
\pgfpathlineto{\pgfqpoint{2.249361in}{0.711485in}}%
\pgfpathlineto{\pgfqpoint{2.249361in}{0.500000in}}%
\pgfpathclose%
\pgfusepath{fill}%
\end{pgfscope}%
\begin{pgfscope}%
\pgfpathrectangle{\pgfqpoint{0.750000in}{0.500000in}}{\pgfqpoint{4.650000in}{3.020000in}}%
\pgfusepath{clip}%
\pgfsetbuttcap%
\pgfsetmiterjoin%
\definecolor{currentfill}{rgb}{0.000000,0.500000,0.000000}%
\pgfsetfillcolor{currentfill}%
\pgfsetlinewidth{0.000000pt}%
\definecolor{currentstroke}{rgb}{0.000000,0.000000,0.000000}%
\pgfsetstrokecolor{currentstroke}%
\pgfsetstrokeopacity{0.000000}%
\pgfsetdash{}{0pt}%
\pgfpathmoveto{\pgfqpoint{2.282386in}{0.500000in}}%
\pgfpathlineto{\pgfqpoint{2.315412in}{0.500000in}}%
\pgfpathlineto{\pgfqpoint{2.315412in}{0.684294in}}%
\pgfpathlineto{\pgfqpoint{2.282386in}{0.684294in}}%
\pgfpathlineto{\pgfqpoint{2.282386in}{0.500000in}}%
\pgfpathclose%
\pgfusepath{fill}%
\end{pgfscope}%
\begin{pgfscope}%
\pgfpathrectangle{\pgfqpoint{0.750000in}{0.500000in}}{\pgfqpoint{4.650000in}{3.020000in}}%
\pgfusepath{clip}%
\pgfsetbuttcap%
\pgfsetmiterjoin%
\definecolor{currentfill}{rgb}{0.000000,0.500000,0.000000}%
\pgfsetfillcolor{currentfill}%
\pgfsetlinewidth{0.000000pt}%
\definecolor{currentstroke}{rgb}{0.000000,0.000000,0.000000}%
\pgfsetstrokecolor{currentstroke}%
\pgfsetstrokeopacity{0.000000}%
\pgfsetdash{}{0pt}%
\pgfpathmoveto{\pgfqpoint{2.315412in}{0.500000in}}%
\pgfpathlineto{\pgfqpoint{2.348437in}{0.500000in}}%
\pgfpathlineto{\pgfqpoint{2.348437in}{0.690336in}}%
\pgfpathlineto{\pgfqpoint{2.315412in}{0.690336in}}%
\pgfpathlineto{\pgfqpoint{2.315412in}{0.500000in}}%
\pgfpathclose%
\pgfusepath{fill}%
\end{pgfscope}%
\begin{pgfscope}%
\pgfpathrectangle{\pgfqpoint{0.750000in}{0.500000in}}{\pgfqpoint{4.650000in}{3.020000in}}%
\pgfusepath{clip}%
\pgfsetbuttcap%
\pgfsetmiterjoin%
\definecolor{currentfill}{rgb}{0.000000,0.500000,0.000000}%
\pgfsetfillcolor{currentfill}%
\pgfsetlinewidth{0.000000pt}%
\definecolor{currentstroke}{rgb}{0.000000,0.000000,0.000000}%
\pgfsetstrokecolor{currentstroke}%
\pgfsetstrokeopacity{0.000000}%
\pgfsetdash{}{0pt}%
\pgfpathmoveto{\pgfqpoint{2.348437in}{0.500000in}}%
\pgfpathlineto{\pgfqpoint{2.381463in}{0.500000in}}%
\pgfpathlineto{\pgfqpoint{2.381463in}{0.678251in}}%
\pgfpathlineto{\pgfqpoint{2.348437in}{0.678251in}}%
\pgfpathlineto{\pgfqpoint{2.348437in}{0.500000in}}%
\pgfpathclose%
\pgfusepath{fill}%
\end{pgfscope}%
\begin{pgfscope}%
\pgfpathrectangle{\pgfqpoint{0.750000in}{0.500000in}}{\pgfqpoint{4.650000in}{3.020000in}}%
\pgfusepath{clip}%
\pgfsetbuttcap%
\pgfsetmiterjoin%
\definecolor{currentfill}{rgb}{0.000000,0.500000,0.000000}%
\pgfsetfillcolor{currentfill}%
\pgfsetlinewidth{0.000000pt}%
\definecolor{currentstroke}{rgb}{0.000000,0.000000,0.000000}%
\pgfsetstrokecolor{currentstroke}%
\pgfsetstrokeopacity{0.000000}%
\pgfsetdash{}{0pt}%
\pgfpathmoveto{\pgfqpoint{2.381463in}{0.500000in}}%
\pgfpathlineto{\pgfqpoint{2.414489in}{0.500000in}}%
\pgfpathlineto{\pgfqpoint{2.414489in}{0.696379in}}%
\pgfpathlineto{\pgfqpoint{2.381463in}{0.696379in}}%
\pgfpathlineto{\pgfqpoint{2.381463in}{0.500000in}}%
\pgfpathclose%
\pgfusepath{fill}%
\end{pgfscope}%
\begin{pgfscope}%
\pgfpathrectangle{\pgfqpoint{0.750000in}{0.500000in}}{\pgfqpoint{4.650000in}{3.020000in}}%
\pgfusepath{clip}%
\pgfsetbuttcap%
\pgfsetmiterjoin%
\definecolor{currentfill}{rgb}{0.000000,0.500000,0.000000}%
\pgfsetfillcolor{currentfill}%
\pgfsetlinewidth{0.000000pt}%
\definecolor{currentstroke}{rgb}{0.000000,0.000000,0.000000}%
\pgfsetstrokecolor{currentstroke}%
\pgfsetstrokeopacity{0.000000}%
\pgfsetdash{}{0pt}%
\pgfpathmoveto{\pgfqpoint{2.414489in}{0.500000in}}%
\pgfpathlineto{\pgfqpoint{2.447514in}{0.500000in}}%
\pgfpathlineto{\pgfqpoint{2.447514in}{0.708463in}}%
\pgfpathlineto{\pgfqpoint{2.414489in}{0.708463in}}%
\pgfpathlineto{\pgfqpoint{2.414489in}{0.500000in}}%
\pgfpathclose%
\pgfusepath{fill}%
\end{pgfscope}%
\begin{pgfscope}%
\pgfpathrectangle{\pgfqpoint{0.750000in}{0.500000in}}{\pgfqpoint{4.650000in}{3.020000in}}%
\pgfusepath{clip}%
\pgfsetbuttcap%
\pgfsetmiterjoin%
\definecolor{currentfill}{rgb}{0.000000,0.500000,0.000000}%
\pgfsetfillcolor{currentfill}%
\pgfsetlinewidth{0.000000pt}%
\definecolor{currentstroke}{rgb}{0.000000,0.000000,0.000000}%
\pgfsetstrokecolor{currentstroke}%
\pgfsetstrokeopacity{0.000000}%
\pgfsetdash{}{0pt}%
\pgfpathmoveto{\pgfqpoint{2.447514in}{0.500000in}}%
\pgfpathlineto{\pgfqpoint{2.480540in}{0.500000in}}%
\pgfpathlineto{\pgfqpoint{2.480540in}{0.684294in}}%
\pgfpathlineto{\pgfqpoint{2.447514in}{0.684294in}}%
\pgfpathlineto{\pgfqpoint{2.447514in}{0.500000in}}%
\pgfpathclose%
\pgfusepath{fill}%
\end{pgfscope}%
\begin{pgfscope}%
\pgfpathrectangle{\pgfqpoint{0.750000in}{0.500000in}}{\pgfqpoint{4.650000in}{3.020000in}}%
\pgfusepath{clip}%
\pgfsetbuttcap%
\pgfsetmiterjoin%
\definecolor{currentfill}{rgb}{0.000000,0.500000,0.000000}%
\pgfsetfillcolor{currentfill}%
\pgfsetlinewidth{0.000000pt}%
\definecolor{currentstroke}{rgb}{0.000000,0.000000,0.000000}%
\pgfsetstrokecolor{currentstroke}%
\pgfsetstrokeopacity{0.000000}%
\pgfsetdash{}{0pt}%
\pgfpathmoveto{\pgfqpoint{2.480540in}{0.500000in}}%
\pgfpathlineto{\pgfqpoint{2.513565in}{0.500000in}}%
\pgfpathlineto{\pgfqpoint{2.513565in}{0.651060in}}%
\pgfpathlineto{\pgfqpoint{2.480540in}{0.651060in}}%
\pgfpathlineto{\pgfqpoint{2.480540in}{0.500000in}}%
\pgfpathclose%
\pgfusepath{fill}%
\end{pgfscope}%
\begin{pgfscope}%
\pgfpathrectangle{\pgfqpoint{0.750000in}{0.500000in}}{\pgfqpoint{4.650000in}{3.020000in}}%
\pgfusepath{clip}%
\pgfsetbuttcap%
\pgfsetmiterjoin%
\definecolor{currentfill}{rgb}{0.000000,0.500000,0.000000}%
\pgfsetfillcolor{currentfill}%
\pgfsetlinewidth{0.000000pt}%
\definecolor{currentstroke}{rgb}{0.000000,0.000000,0.000000}%
\pgfsetstrokecolor{currentstroke}%
\pgfsetstrokeopacity{0.000000}%
\pgfsetdash{}{0pt}%
\pgfpathmoveto{\pgfqpoint{2.513565in}{0.500000in}}%
\pgfpathlineto{\pgfqpoint{2.546591in}{0.500000in}}%
\pgfpathlineto{\pgfqpoint{2.546591in}{0.690336in}}%
\pgfpathlineto{\pgfqpoint{2.513565in}{0.690336in}}%
\pgfpathlineto{\pgfqpoint{2.513565in}{0.500000in}}%
\pgfpathclose%
\pgfusepath{fill}%
\end{pgfscope}%
\begin{pgfscope}%
\pgfpathrectangle{\pgfqpoint{0.750000in}{0.500000in}}{\pgfqpoint{4.650000in}{3.020000in}}%
\pgfusepath{clip}%
\pgfsetbuttcap%
\pgfsetmiterjoin%
\definecolor{currentfill}{rgb}{0.000000,0.500000,0.000000}%
\pgfsetfillcolor{currentfill}%
\pgfsetlinewidth{0.000000pt}%
\definecolor{currentstroke}{rgb}{0.000000,0.000000,0.000000}%
\pgfsetstrokecolor{currentstroke}%
\pgfsetstrokeopacity{0.000000}%
\pgfsetdash{}{0pt}%
\pgfpathmoveto{\pgfqpoint{2.546591in}{0.500000in}}%
\pgfpathlineto{\pgfqpoint{2.579616in}{0.500000in}}%
\pgfpathlineto{\pgfqpoint{2.579616in}{0.666166in}}%
\pgfpathlineto{\pgfqpoint{2.546591in}{0.666166in}}%
\pgfpathlineto{\pgfqpoint{2.546591in}{0.500000in}}%
\pgfpathclose%
\pgfusepath{fill}%
\end{pgfscope}%
\begin{pgfscope}%
\pgfpathrectangle{\pgfqpoint{0.750000in}{0.500000in}}{\pgfqpoint{4.650000in}{3.020000in}}%
\pgfusepath{clip}%
\pgfsetbuttcap%
\pgfsetmiterjoin%
\definecolor{currentfill}{rgb}{0.000000,0.500000,0.000000}%
\pgfsetfillcolor{currentfill}%
\pgfsetlinewidth{0.000000pt}%
\definecolor{currentstroke}{rgb}{0.000000,0.000000,0.000000}%
\pgfsetstrokecolor{currentstroke}%
\pgfsetstrokeopacity{0.000000}%
\pgfsetdash{}{0pt}%
\pgfpathmoveto{\pgfqpoint{2.579616in}{0.500000in}}%
\pgfpathlineto{\pgfqpoint{2.612642in}{0.500000in}}%
\pgfpathlineto{\pgfqpoint{2.612642in}{0.669188in}}%
\pgfpathlineto{\pgfqpoint{2.579616in}{0.669188in}}%
\pgfpathlineto{\pgfqpoint{2.579616in}{0.500000in}}%
\pgfpathclose%
\pgfusepath{fill}%
\end{pgfscope}%
\begin{pgfscope}%
\pgfpathrectangle{\pgfqpoint{0.750000in}{0.500000in}}{\pgfqpoint{4.650000in}{3.020000in}}%
\pgfusepath{clip}%
\pgfsetbuttcap%
\pgfsetmiterjoin%
\definecolor{currentfill}{rgb}{0.000000,0.500000,0.000000}%
\pgfsetfillcolor{currentfill}%
\pgfsetlinewidth{0.000000pt}%
\definecolor{currentstroke}{rgb}{0.000000,0.000000,0.000000}%
\pgfsetstrokecolor{currentstroke}%
\pgfsetstrokeopacity{0.000000}%
\pgfsetdash{}{0pt}%
\pgfpathmoveto{\pgfqpoint{2.612642in}{0.500000in}}%
\pgfpathlineto{\pgfqpoint{2.645668in}{0.500000in}}%
\pgfpathlineto{\pgfqpoint{2.645668in}{0.654082in}}%
\pgfpathlineto{\pgfqpoint{2.612642in}{0.654082in}}%
\pgfpathlineto{\pgfqpoint{2.612642in}{0.500000in}}%
\pgfpathclose%
\pgfusepath{fill}%
\end{pgfscope}%
\begin{pgfscope}%
\pgfpathrectangle{\pgfqpoint{0.750000in}{0.500000in}}{\pgfqpoint{4.650000in}{3.020000in}}%
\pgfusepath{clip}%
\pgfsetbuttcap%
\pgfsetmiterjoin%
\definecolor{currentfill}{rgb}{0.000000,0.500000,0.000000}%
\pgfsetfillcolor{currentfill}%
\pgfsetlinewidth{0.000000pt}%
\definecolor{currentstroke}{rgb}{0.000000,0.000000,0.000000}%
\pgfsetstrokecolor{currentstroke}%
\pgfsetstrokeopacity{0.000000}%
\pgfsetdash{}{0pt}%
\pgfpathmoveto{\pgfqpoint{2.645668in}{0.500000in}}%
\pgfpathlineto{\pgfqpoint{2.678693in}{0.500000in}}%
\pgfpathlineto{\pgfqpoint{2.678693in}{0.723569in}}%
\pgfpathlineto{\pgfqpoint{2.645668in}{0.723569in}}%
\pgfpathlineto{\pgfqpoint{2.645668in}{0.500000in}}%
\pgfpathclose%
\pgfusepath{fill}%
\end{pgfscope}%
\begin{pgfscope}%
\pgfpathrectangle{\pgfqpoint{0.750000in}{0.500000in}}{\pgfqpoint{4.650000in}{3.020000in}}%
\pgfusepath{clip}%
\pgfsetbuttcap%
\pgfsetmiterjoin%
\definecolor{currentfill}{rgb}{0.000000,0.500000,0.000000}%
\pgfsetfillcolor{currentfill}%
\pgfsetlinewidth{0.000000pt}%
\definecolor{currentstroke}{rgb}{0.000000,0.000000,0.000000}%
\pgfsetstrokecolor{currentstroke}%
\pgfsetstrokeopacity{0.000000}%
\pgfsetdash{}{0pt}%
\pgfpathmoveto{\pgfqpoint{2.678693in}{0.500000in}}%
\pgfpathlineto{\pgfqpoint{2.711719in}{0.500000in}}%
\pgfpathlineto{\pgfqpoint{2.711719in}{0.669188in}}%
\pgfpathlineto{\pgfqpoint{2.678693in}{0.669188in}}%
\pgfpathlineto{\pgfqpoint{2.678693in}{0.500000in}}%
\pgfpathclose%
\pgfusepath{fill}%
\end{pgfscope}%
\begin{pgfscope}%
\pgfpathrectangle{\pgfqpoint{0.750000in}{0.500000in}}{\pgfqpoint{4.650000in}{3.020000in}}%
\pgfusepath{clip}%
\pgfsetbuttcap%
\pgfsetmiterjoin%
\definecolor{currentfill}{rgb}{0.000000,0.500000,0.000000}%
\pgfsetfillcolor{currentfill}%
\pgfsetlinewidth{0.000000pt}%
\definecolor{currentstroke}{rgb}{0.000000,0.000000,0.000000}%
\pgfsetstrokecolor{currentstroke}%
\pgfsetstrokeopacity{0.000000}%
\pgfsetdash{}{0pt}%
\pgfpathmoveto{\pgfqpoint{2.711719in}{0.500000in}}%
\pgfpathlineto{\pgfqpoint{2.744744in}{0.500000in}}%
\pgfpathlineto{\pgfqpoint{2.744744in}{0.648039in}}%
\pgfpathlineto{\pgfqpoint{2.711719in}{0.648039in}}%
\pgfpathlineto{\pgfqpoint{2.711719in}{0.500000in}}%
\pgfpathclose%
\pgfusepath{fill}%
\end{pgfscope}%
\begin{pgfscope}%
\pgfpathrectangle{\pgfqpoint{0.750000in}{0.500000in}}{\pgfqpoint{4.650000in}{3.020000in}}%
\pgfusepath{clip}%
\pgfsetbuttcap%
\pgfsetmiterjoin%
\definecolor{currentfill}{rgb}{0.000000,0.500000,0.000000}%
\pgfsetfillcolor{currentfill}%
\pgfsetlinewidth{0.000000pt}%
\definecolor{currentstroke}{rgb}{0.000000,0.000000,0.000000}%
\pgfsetstrokecolor{currentstroke}%
\pgfsetstrokeopacity{0.000000}%
\pgfsetdash{}{0pt}%
\pgfpathmoveto{\pgfqpoint{2.744744in}{0.500000in}}%
\pgfpathlineto{\pgfqpoint{2.777770in}{0.500000in}}%
\pgfpathlineto{\pgfqpoint{2.777770in}{0.702421in}}%
\pgfpathlineto{\pgfqpoint{2.744744in}{0.702421in}}%
\pgfpathlineto{\pgfqpoint{2.744744in}{0.500000in}}%
\pgfpathclose%
\pgfusepath{fill}%
\end{pgfscope}%
\begin{pgfscope}%
\pgfpathrectangle{\pgfqpoint{0.750000in}{0.500000in}}{\pgfqpoint{4.650000in}{3.020000in}}%
\pgfusepath{clip}%
\pgfsetbuttcap%
\pgfsetmiterjoin%
\definecolor{currentfill}{rgb}{0.000000,0.500000,0.000000}%
\pgfsetfillcolor{currentfill}%
\pgfsetlinewidth{0.000000pt}%
\definecolor{currentstroke}{rgb}{0.000000,0.000000,0.000000}%
\pgfsetstrokecolor{currentstroke}%
\pgfsetstrokeopacity{0.000000}%
\pgfsetdash{}{0pt}%
\pgfpathmoveto{\pgfqpoint{2.777770in}{0.500000in}}%
\pgfpathlineto{\pgfqpoint{2.810795in}{0.500000in}}%
\pgfpathlineto{\pgfqpoint{2.810795in}{0.672209in}}%
\pgfpathlineto{\pgfqpoint{2.777770in}{0.672209in}}%
\pgfpathlineto{\pgfqpoint{2.777770in}{0.500000in}}%
\pgfpathclose%
\pgfusepath{fill}%
\end{pgfscope}%
\begin{pgfscope}%
\pgfpathrectangle{\pgfqpoint{0.750000in}{0.500000in}}{\pgfqpoint{4.650000in}{3.020000in}}%
\pgfusepath{clip}%
\pgfsetbuttcap%
\pgfsetmiterjoin%
\definecolor{currentfill}{rgb}{0.000000,0.500000,0.000000}%
\pgfsetfillcolor{currentfill}%
\pgfsetlinewidth{0.000000pt}%
\definecolor{currentstroke}{rgb}{0.000000,0.000000,0.000000}%
\pgfsetstrokecolor{currentstroke}%
\pgfsetstrokeopacity{0.000000}%
\pgfsetdash{}{0pt}%
\pgfpathmoveto{\pgfqpoint{2.810795in}{0.500000in}}%
\pgfpathlineto{\pgfqpoint{2.843821in}{0.500000in}}%
\pgfpathlineto{\pgfqpoint{2.843821in}{0.714506in}}%
\pgfpathlineto{\pgfqpoint{2.810795in}{0.714506in}}%
\pgfpathlineto{\pgfqpoint{2.810795in}{0.500000in}}%
\pgfpathclose%
\pgfusepath{fill}%
\end{pgfscope}%
\begin{pgfscope}%
\pgfpathrectangle{\pgfqpoint{0.750000in}{0.500000in}}{\pgfqpoint{4.650000in}{3.020000in}}%
\pgfusepath{clip}%
\pgfsetbuttcap%
\pgfsetmiterjoin%
\definecolor{currentfill}{rgb}{0.000000,0.500000,0.000000}%
\pgfsetfillcolor{currentfill}%
\pgfsetlinewidth{0.000000pt}%
\definecolor{currentstroke}{rgb}{0.000000,0.000000,0.000000}%
\pgfsetstrokecolor{currentstroke}%
\pgfsetstrokeopacity{0.000000}%
\pgfsetdash{}{0pt}%
\pgfpathmoveto{\pgfqpoint{2.843821in}{0.500000in}}%
\pgfpathlineto{\pgfqpoint{2.876847in}{0.500000in}}%
\pgfpathlineto{\pgfqpoint{2.876847in}{0.666166in}}%
\pgfpathlineto{\pgfqpoint{2.843821in}{0.666166in}}%
\pgfpathlineto{\pgfqpoint{2.843821in}{0.500000in}}%
\pgfpathclose%
\pgfusepath{fill}%
\end{pgfscope}%
\begin{pgfscope}%
\pgfpathrectangle{\pgfqpoint{0.750000in}{0.500000in}}{\pgfqpoint{4.650000in}{3.020000in}}%
\pgfusepath{clip}%
\pgfsetbuttcap%
\pgfsetmiterjoin%
\definecolor{currentfill}{rgb}{0.000000,0.500000,0.000000}%
\pgfsetfillcolor{currentfill}%
\pgfsetlinewidth{0.000000pt}%
\definecolor{currentstroke}{rgb}{0.000000,0.000000,0.000000}%
\pgfsetstrokecolor{currentstroke}%
\pgfsetstrokeopacity{0.000000}%
\pgfsetdash{}{0pt}%
\pgfpathmoveto{\pgfqpoint{2.876847in}{0.500000in}}%
\pgfpathlineto{\pgfqpoint{2.909872in}{0.500000in}}%
\pgfpathlineto{\pgfqpoint{2.909872in}{0.732633in}}%
\pgfpathlineto{\pgfqpoint{2.876847in}{0.732633in}}%
\pgfpathlineto{\pgfqpoint{2.876847in}{0.500000in}}%
\pgfpathclose%
\pgfusepath{fill}%
\end{pgfscope}%
\begin{pgfscope}%
\pgfpathrectangle{\pgfqpoint{0.750000in}{0.500000in}}{\pgfqpoint{4.650000in}{3.020000in}}%
\pgfusepath{clip}%
\pgfsetbuttcap%
\pgfsetmiterjoin%
\definecolor{currentfill}{rgb}{0.000000,0.500000,0.000000}%
\pgfsetfillcolor{currentfill}%
\pgfsetlinewidth{0.000000pt}%
\definecolor{currentstroke}{rgb}{0.000000,0.000000,0.000000}%
\pgfsetstrokecolor{currentstroke}%
\pgfsetstrokeopacity{0.000000}%
\pgfsetdash{}{0pt}%
\pgfpathmoveto{\pgfqpoint{2.909872in}{0.500000in}}%
\pgfpathlineto{\pgfqpoint{2.942898in}{0.500000in}}%
\pgfpathlineto{\pgfqpoint{2.942898in}{0.717527in}}%
\pgfpathlineto{\pgfqpoint{2.909872in}{0.717527in}}%
\pgfpathlineto{\pgfqpoint{2.909872in}{0.500000in}}%
\pgfpathclose%
\pgfusepath{fill}%
\end{pgfscope}%
\begin{pgfscope}%
\pgfpathrectangle{\pgfqpoint{0.750000in}{0.500000in}}{\pgfqpoint{4.650000in}{3.020000in}}%
\pgfusepath{clip}%
\pgfsetbuttcap%
\pgfsetmiterjoin%
\definecolor{currentfill}{rgb}{0.000000,0.500000,0.000000}%
\pgfsetfillcolor{currentfill}%
\pgfsetlinewidth{0.000000pt}%
\definecolor{currentstroke}{rgb}{0.000000,0.000000,0.000000}%
\pgfsetstrokecolor{currentstroke}%
\pgfsetstrokeopacity{0.000000}%
\pgfsetdash{}{0pt}%
\pgfpathmoveto{\pgfqpoint{2.942898in}{0.500000in}}%
\pgfpathlineto{\pgfqpoint{2.975923in}{0.500000in}}%
\pgfpathlineto{\pgfqpoint{2.975923in}{0.699400in}}%
\pgfpathlineto{\pgfqpoint{2.942898in}{0.699400in}}%
\pgfpathlineto{\pgfqpoint{2.942898in}{0.500000in}}%
\pgfpathclose%
\pgfusepath{fill}%
\end{pgfscope}%
\begin{pgfscope}%
\pgfpathrectangle{\pgfqpoint{0.750000in}{0.500000in}}{\pgfqpoint{4.650000in}{3.020000in}}%
\pgfusepath{clip}%
\pgfsetbuttcap%
\pgfsetmiterjoin%
\definecolor{currentfill}{rgb}{0.000000,0.500000,0.000000}%
\pgfsetfillcolor{currentfill}%
\pgfsetlinewidth{0.000000pt}%
\definecolor{currentstroke}{rgb}{0.000000,0.000000,0.000000}%
\pgfsetstrokecolor{currentstroke}%
\pgfsetstrokeopacity{0.000000}%
\pgfsetdash{}{0pt}%
\pgfpathmoveto{\pgfqpoint{2.975923in}{0.500000in}}%
\pgfpathlineto{\pgfqpoint{3.008949in}{0.500000in}}%
\pgfpathlineto{\pgfqpoint{3.008949in}{0.714506in}}%
\pgfpathlineto{\pgfqpoint{2.975923in}{0.714506in}}%
\pgfpathlineto{\pgfqpoint{2.975923in}{0.500000in}}%
\pgfpathclose%
\pgfusepath{fill}%
\end{pgfscope}%
\begin{pgfscope}%
\pgfpathrectangle{\pgfqpoint{0.750000in}{0.500000in}}{\pgfqpoint{4.650000in}{3.020000in}}%
\pgfusepath{clip}%
\pgfsetbuttcap%
\pgfsetmiterjoin%
\definecolor{currentfill}{rgb}{0.000000,0.500000,0.000000}%
\pgfsetfillcolor{currentfill}%
\pgfsetlinewidth{0.000000pt}%
\definecolor{currentstroke}{rgb}{0.000000,0.000000,0.000000}%
\pgfsetstrokecolor{currentstroke}%
\pgfsetstrokeopacity{0.000000}%
\pgfsetdash{}{0pt}%
\pgfpathmoveto{\pgfqpoint{3.008949in}{0.500000in}}%
\pgfpathlineto{\pgfqpoint{3.041974in}{0.500000in}}%
\pgfpathlineto{\pgfqpoint{3.041974in}{0.699400in}}%
\pgfpathlineto{\pgfqpoint{3.008949in}{0.699400in}}%
\pgfpathlineto{\pgfqpoint{3.008949in}{0.500000in}}%
\pgfpathclose%
\pgfusepath{fill}%
\end{pgfscope}%
\begin{pgfscope}%
\pgfpathrectangle{\pgfqpoint{0.750000in}{0.500000in}}{\pgfqpoint{4.650000in}{3.020000in}}%
\pgfusepath{clip}%
\pgfsetbuttcap%
\pgfsetmiterjoin%
\definecolor{currentfill}{rgb}{0.000000,0.500000,0.000000}%
\pgfsetfillcolor{currentfill}%
\pgfsetlinewidth{0.000000pt}%
\definecolor{currentstroke}{rgb}{0.000000,0.000000,0.000000}%
\pgfsetstrokecolor{currentstroke}%
\pgfsetstrokeopacity{0.000000}%
\pgfsetdash{}{0pt}%
\pgfpathmoveto{\pgfqpoint{3.041974in}{0.500000in}}%
\pgfpathlineto{\pgfqpoint{3.075000in}{0.500000in}}%
\pgfpathlineto{\pgfqpoint{3.075000in}{0.660124in}}%
\pgfpathlineto{\pgfqpoint{3.041974in}{0.660124in}}%
\pgfpathlineto{\pgfqpoint{3.041974in}{0.500000in}}%
\pgfpathclose%
\pgfusepath{fill}%
\end{pgfscope}%
\begin{pgfscope}%
\pgfpathrectangle{\pgfqpoint{0.750000in}{0.500000in}}{\pgfqpoint{4.650000in}{3.020000in}}%
\pgfusepath{clip}%
\pgfsetbuttcap%
\pgfsetmiterjoin%
\definecolor{currentfill}{rgb}{0.000000,0.500000,0.000000}%
\pgfsetfillcolor{currentfill}%
\pgfsetlinewidth{0.000000pt}%
\definecolor{currentstroke}{rgb}{0.000000,0.000000,0.000000}%
\pgfsetstrokecolor{currentstroke}%
\pgfsetstrokeopacity{0.000000}%
\pgfsetdash{}{0pt}%
\pgfpathmoveto{\pgfqpoint{3.075000in}{0.500000in}}%
\pgfpathlineto{\pgfqpoint{3.108026in}{0.500000in}}%
\pgfpathlineto{\pgfqpoint{3.108026in}{0.684294in}}%
\pgfpathlineto{\pgfqpoint{3.075000in}{0.684294in}}%
\pgfpathlineto{\pgfqpoint{3.075000in}{0.500000in}}%
\pgfpathclose%
\pgfusepath{fill}%
\end{pgfscope}%
\begin{pgfscope}%
\pgfpathrectangle{\pgfqpoint{0.750000in}{0.500000in}}{\pgfqpoint{4.650000in}{3.020000in}}%
\pgfusepath{clip}%
\pgfsetbuttcap%
\pgfsetmiterjoin%
\definecolor{currentfill}{rgb}{0.000000,0.500000,0.000000}%
\pgfsetfillcolor{currentfill}%
\pgfsetlinewidth{0.000000pt}%
\definecolor{currentstroke}{rgb}{0.000000,0.000000,0.000000}%
\pgfsetstrokecolor{currentstroke}%
\pgfsetstrokeopacity{0.000000}%
\pgfsetdash{}{0pt}%
\pgfpathmoveto{\pgfqpoint{3.108026in}{0.500000in}}%
\pgfpathlineto{\pgfqpoint{3.141051in}{0.500000in}}%
\pgfpathlineto{\pgfqpoint{3.141051in}{0.654082in}}%
\pgfpathlineto{\pgfqpoint{3.108026in}{0.654082in}}%
\pgfpathlineto{\pgfqpoint{3.108026in}{0.500000in}}%
\pgfpathclose%
\pgfusepath{fill}%
\end{pgfscope}%
\begin{pgfscope}%
\pgfpathrectangle{\pgfqpoint{0.750000in}{0.500000in}}{\pgfqpoint{4.650000in}{3.020000in}}%
\pgfusepath{clip}%
\pgfsetbuttcap%
\pgfsetmiterjoin%
\definecolor{currentfill}{rgb}{0.000000,0.500000,0.000000}%
\pgfsetfillcolor{currentfill}%
\pgfsetlinewidth{0.000000pt}%
\definecolor{currentstroke}{rgb}{0.000000,0.000000,0.000000}%
\pgfsetstrokecolor{currentstroke}%
\pgfsetstrokeopacity{0.000000}%
\pgfsetdash{}{0pt}%
\pgfpathmoveto{\pgfqpoint{3.141051in}{0.500000in}}%
\pgfpathlineto{\pgfqpoint{3.174077in}{0.500000in}}%
\pgfpathlineto{\pgfqpoint{3.174077in}{0.645018in}}%
\pgfpathlineto{\pgfqpoint{3.141051in}{0.645018in}}%
\pgfpathlineto{\pgfqpoint{3.141051in}{0.500000in}}%
\pgfpathclose%
\pgfusepath{fill}%
\end{pgfscope}%
\begin{pgfscope}%
\pgfpathrectangle{\pgfqpoint{0.750000in}{0.500000in}}{\pgfqpoint{4.650000in}{3.020000in}}%
\pgfusepath{clip}%
\pgfsetbuttcap%
\pgfsetmiterjoin%
\definecolor{currentfill}{rgb}{0.000000,0.500000,0.000000}%
\pgfsetfillcolor{currentfill}%
\pgfsetlinewidth{0.000000pt}%
\definecolor{currentstroke}{rgb}{0.000000,0.000000,0.000000}%
\pgfsetstrokecolor{currentstroke}%
\pgfsetstrokeopacity{0.000000}%
\pgfsetdash{}{0pt}%
\pgfpathmoveto{\pgfqpoint{3.174077in}{0.500000in}}%
\pgfpathlineto{\pgfqpoint{3.207102in}{0.500000in}}%
\pgfpathlineto{\pgfqpoint{3.207102in}{0.645018in}}%
\pgfpathlineto{\pgfqpoint{3.174077in}{0.645018in}}%
\pgfpathlineto{\pgfqpoint{3.174077in}{0.500000in}}%
\pgfpathclose%
\pgfusepath{fill}%
\end{pgfscope}%
\begin{pgfscope}%
\pgfpathrectangle{\pgfqpoint{0.750000in}{0.500000in}}{\pgfqpoint{4.650000in}{3.020000in}}%
\pgfusepath{clip}%
\pgfsetbuttcap%
\pgfsetmiterjoin%
\definecolor{currentfill}{rgb}{0.000000,0.500000,0.000000}%
\pgfsetfillcolor{currentfill}%
\pgfsetlinewidth{0.000000pt}%
\definecolor{currentstroke}{rgb}{0.000000,0.000000,0.000000}%
\pgfsetstrokecolor{currentstroke}%
\pgfsetstrokeopacity{0.000000}%
\pgfsetdash{}{0pt}%
\pgfpathmoveto{\pgfqpoint{3.207102in}{0.500000in}}%
\pgfpathlineto{\pgfqpoint{3.240128in}{0.500000in}}%
\pgfpathlineto{\pgfqpoint{3.240128in}{0.593657in}}%
\pgfpathlineto{\pgfqpoint{3.207102in}{0.593657in}}%
\pgfpathlineto{\pgfqpoint{3.207102in}{0.500000in}}%
\pgfpathclose%
\pgfusepath{fill}%
\end{pgfscope}%
\begin{pgfscope}%
\pgfpathrectangle{\pgfqpoint{0.750000in}{0.500000in}}{\pgfqpoint{4.650000in}{3.020000in}}%
\pgfusepath{clip}%
\pgfsetbuttcap%
\pgfsetmiterjoin%
\definecolor{currentfill}{rgb}{0.000000,0.500000,0.000000}%
\pgfsetfillcolor{currentfill}%
\pgfsetlinewidth{0.000000pt}%
\definecolor{currentstroke}{rgb}{0.000000,0.000000,0.000000}%
\pgfsetstrokecolor{currentstroke}%
\pgfsetstrokeopacity{0.000000}%
\pgfsetdash{}{0pt}%
\pgfpathmoveto{\pgfqpoint{3.240128in}{0.500000in}}%
\pgfpathlineto{\pgfqpoint{3.273153in}{0.500000in}}%
\pgfpathlineto{\pgfqpoint{3.273153in}{0.587615in}}%
\pgfpathlineto{\pgfqpoint{3.240128in}{0.587615in}}%
\pgfpathlineto{\pgfqpoint{3.240128in}{0.500000in}}%
\pgfpathclose%
\pgfusepath{fill}%
\end{pgfscope}%
\begin{pgfscope}%
\pgfpathrectangle{\pgfqpoint{0.750000in}{0.500000in}}{\pgfqpoint{4.650000in}{3.020000in}}%
\pgfusepath{clip}%
\pgfsetbuttcap%
\pgfsetmiterjoin%
\definecolor{currentfill}{rgb}{0.000000,0.500000,0.000000}%
\pgfsetfillcolor{currentfill}%
\pgfsetlinewidth{0.000000pt}%
\definecolor{currentstroke}{rgb}{0.000000,0.000000,0.000000}%
\pgfsetstrokecolor{currentstroke}%
\pgfsetstrokeopacity{0.000000}%
\pgfsetdash{}{0pt}%
\pgfpathmoveto{\pgfqpoint{3.273153in}{0.500000in}}%
\pgfpathlineto{\pgfqpoint{3.306179in}{0.500000in}}%
\pgfpathlineto{\pgfqpoint{3.306179in}{0.551361in}}%
\pgfpathlineto{\pgfqpoint{3.273153in}{0.551361in}}%
\pgfpathlineto{\pgfqpoint{3.273153in}{0.500000in}}%
\pgfpathclose%
\pgfusepath{fill}%
\end{pgfscope}%
\begin{pgfscope}%
\pgfpathrectangle{\pgfqpoint{0.750000in}{0.500000in}}{\pgfqpoint{4.650000in}{3.020000in}}%
\pgfusepath{clip}%
\pgfsetbuttcap%
\pgfsetmiterjoin%
\definecolor{currentfill}{rgb}{0.000000,0.500000,0.000000}%
\pgfsetfillcolor{currentfill}%
\pgfsetlinewidth{0.000000pt}%
\definecolor{currentstroke}{rgb}{0.000000,0.000000,0.000000}%
\pgfsetstrokecolor{currentstroke}%
\pgfsetstrokeopacity{0.000000}%
\pgfsetdash{}{0pt}%
\pgfpathmoveto{\pgfqpoint{3.306179in}{0.500000in}}%
\pgfpathlineto{\pgfqpoint{3.339205in}{0.500000in}}%
\pgfpathlineto{\pgfqpoint{3.339205in}{0.533233in}}%
\pgfpathlineto{\pgfqpoint{3.306179in}{0.533233in}}%
\pgfpathlineto{\pgfqpoint{3.306179in}{0.500000in}}%
\pgfpathclose%
\pgfusepath{fill}%
\end{pgfscope}%
\begin{pgfscope}%
\pgfpathrectangle{\pgfqpoint{0.750000in}{0.500000in}}{\pgfqpoint{4.650000in}{3.020000in}}%
\pgfusepath{clip}%
\pgfsetbuttcap%
\pgfsetmiterjoin%
\definecolor{currentfill}{rgb}{0.000000,0.500000,0.000000}%
\pgfsetfillcolor{currentfill}%
\pgfsetlinewidth{0.000000pt}%
\definecolor{currentstroke}{rgb}{0.000000,0.000000,0.000000}%
\pgfsetstrokecolor{currentstroke}%
\pgfsetstrokeopacity{0.000000}%
\pgfsetdash{}{0pt}%
\pgfpathmoveto{\pgfqpoint{3.339205in}{0.500000in}}%
\pgfpathlineto{\pgfqpoint{3.372230in}{0.500000in}}%
\pgfpathlineto{\pgfqpoint{3.372230in}{0.539276in}}%
\pgfpathlineto{\pgfqpoint{3.339205in}{0.539276in}}%
\pgfpathlineto{\pgfqpoint{3.339205in}{0.500000in}}%
\pgfpathclose%
\pgfusepath{fill}%
\end{pgfscope}%
\begin{pgfscope}%
\pgfpathrectangle{\pgfqpoint{0.750000in}{0.500000in}}{\pgfqpoint{4.650000in}{3.020000in}}%
\pgfusepath{clip}%
\pgfsetbuttcap%
\pgfsetmiterjoin%
\definecolor{currentfill}{rgb}{0.000000,0.500000,0.000000}%
\pgfsetfillcolor{currentfill}%
\pgfsetlinewidth{0.000000pt}%
\definecolor{currentstroke}{rgb}{0.000000,0.000000,0.000000}%
\pgfsetstrokecolor{currentstroke}%
\pgfsetstrokeopacity{0.000000}%
\pgfsetdash{}{0pt}%
\pgfpathmoveto{\pgfqpoint{3.372230in}{0.500000in}}%
\pgfpathlineto{\pgfqpoint{3.405256in}{0.500000in}}%
\pgfpathlineto{\pgfqpoint{3.405256in}{0.515106in}}%
\pgfpathlineto{\pgfqpoint{3.372230in}{0.515106in}}%
\pgfpathlineto{\pgfqpoint{3.372230in}{0.500000in}}%
\pgfpathclose%
\pgfusepath{fill}%
\end{pgfscope}%
\begin{pgfscope}%
\pgfpathrectangle{\pgfqpoint{0.750000in}{0.500000in}}{\pgfqpoint{4.650000in}{3.020000in}}%
\pgfusepath{clip}%
\pgfsetbuttcap%
\pgfsetmiterjoin%
\definecolor{currentfill}{rgb}{0.000000,0.500000,0.000000}%
\pgfsetfillcolor{currentfill}%
\pgfsetlinewidth{0.000000pt}%
\definecolor{currentstroke}{rgb}{0.000000,0.000000,0.000000}%
\pgfsetstrokecolor{currentstroke}%
\pgfsetstrokeopacity{0.000000}%
\pgfsetdash{}{0pt}%
\pgfpathmoveto{\pgfqpoint{3.405256in}{0.500000in}}%
\pgfpathlineto{\pgfqpoint{3.438281in}{0.500000in}}%
\pgfpathlineto{\pgfqpoint{3.438281in}{0.527191in}}%
\pgfpathlineto{\pgfqpoint{3.405256in}{0.527191in}}%
\pgfpathlineto{\pgfqpoint{3.405256in}{0.500000in}}%
\pgfpathclose%
\pgfusepath{fill}%
\end{pgfscope}%
\begin{pgfscope}%
\pgfpathrectangle{\pgfqpoint{0.750000in}{0.500000in}}{\pgfqpoint{4.650000in}{3.020000in}}%
\pgfusepath{clip}%
\pgfsetbuttcap%
\pgfsetmiterjoin%
\definecolor{currentfill}{rgb}{0.000000,0.500000,0.000000}%
\pgfsetfillcolor{currentfill}%
\pgfsetlinewidth{0.000000pt}%
\definecolor{currentstroke}{rgb}{0.000000,0.000000,0.000000}%
\pgfsetstrokecolor{currentstroke}%
\pgfsetstrokeopacity{0.000000}%
\pgfsetdash{}{0pt}%
\pgfpathmoveto{\pgfqpoint{3.438281in}{0.500000in}}%
\pgfpathlineto{\pgfqpoint{3.471307in}{0.500000in}}%
\pgfpathlineto{\pgfqpoint{3.471307in}{0.506042in}}%
\pgfpathlineto{\pgfqpoint{3.438281in}{0.506042in}}%
\pgfpathlineto{\pgfqpoint{3.438281in}{0.500000in}}%
\pgfpathclose%
\pgfusepath{fill}%
\end{pgfscope}%
\begin{pgfscope}%
\pgfpathrectangle{\pgfqpoint{0.750000in}{0.500000in}}{\pgfqpoint{4.650000in}{3.020000in}}%
\pgfusepath{clip}%
\pgfsetbuttcap%
\pgfsetmiterjoin%
\definecolor{currentfill}{rgb}{0.000000,0.500000,0.000000}%
\pgfsetfillcolor{currentfill}%
\pgfsetlinewidth{0.000000pt}%
\definecolor{currentstroke}{rgb}{0.000000,0.000000,0.000000}%
\pgfsetstrokecolor{currentstroke}%
\pgfsetstrokeopacity{0.000000}%
\pgfsetdash{}{0pt}%
\pgfpathmoveto{\pgfqpoint{3.471307in}{0.500000in}}%
\pgfpathlineto{\pgfqpoint{3.504332in}{0.500000in}}%
\pgfpathlineto{\pgfqpoint{3.504332in}{0.503021in}}%
\pgfpathlineto{\pgfqpoint{3.471307in}{0.503021in}}%
\pgfpathlineto{\pgfqpoint{3.471307in}{0.500000in}}%
\pgfpathclose%
\pgfusepath{fill}%
\end{pgfscope}%
\begin{pgfscope}%
\pgfpathrectangle{\pgfqpoint{0.750000in}{0.500000in}}{\pgfqpoint{4.650000in}{3.020000in}}%
\pgfusepath{clip}%
\pgfsetbuttcap%
\pgfsetmiterjoin%
\definecolor{currentfill}{rgb}{0.000000,0.500000,0.000000}%
\pgfsetfillcolor{currentfill}%
\pgfsetlinewidth{0.000000pt}%
\definecolor{currentstroke}{rgb}{0.000000,0.000000,0.000000}%
\pgfsetstrokecolor{currentstroke}%
\pgfsetstrokeopacity{0.000000}%
\pgfsetdash{}{0pt}%
\pgfpathmoveto{\pgfqpoint{3.504332in}{0.500000in}}%
\pgfpathlineto{\pgfqpoint{3.537358in}{0.500000in}}%
\pgfpathlineto{\pgfqpoint{3.537358in}{0.524170in}}%
\pgfpathlineto{\pgfqpoint{3.504332in}{0.524170in}}%
\pgfpathlineto{\pgfqpoint{3.504332in}{0.500000in}}%
\pgfpathclose%
\pgfusepath{fill}%
\end{pgfscope}%
\begin{pgfscope}%
\pgfpathrectangle{\pgfqpoint{0.750000in}{0.500000in}}{\pgfqpoint{4.650000in}{3.020000in}}%
\pgfusepath{clip}%
\pgfsetbuttcap%
\pgfsetmiterjoin%
\definecolor{currentfill}{rgb}{0.000000,0.500000,0.000000}%
\pgfsetfillcolor{currentfill}%
\pgfsetlinewidth{0.000000pt}%
\definecolor{currentstroke}{rgb}{0.000000,0.000000,0.000000}%
\pgfsetstrokecolor{currentstroke}%
\pgfsetstrokeopacity{0.000000}%
\pgfsetdash{}{0pt}%
\pgfpathmoveto{\pgfqpoint{3.537358in}{0.500000in}}%
\pgfpathlineto{\pgfqpoint{3.570384in}{0.500000in}}%
\pgfpathlineto{\pgfqpoint{3.570384in}{0.509064in}}%
\pgfpathlineto{\pgfqpoint{3.537358in}{0.509064in}}%
\pgfpathlineto{\pgfqpoint{3.537358in}{0.500000in}}%
\pgfpathclose%
\pgfusepath{fill}%
\end{pgfscope}%
\begin{pgfscope}%
\pgfpathrectangle{\pgfqpoint{0.750000in}{0.500000in}}{\pgfqpoint{4.650000in}{3.020000in}}%
\pgfusepath{clip}%
\pgfsetbuttcap%
\pgfsetmiterjoin%
\definecolor{currentfill}{rgb}{0.000000,0.500000,0.000000}%
\pgfsetfillcolor{currentfill}%
\pgfsetlinewidth{0.000000pt}%
\definecolor{currentstroke}{rgb}{0.000000,0.000000,0.000000}%
\pgfsetstrokecolor{currentstroke}%
\pgfsetstrokeopacity{0.000000}%
\pgfsetdash{}{0pt}%
\pgfpathmoveto{\pgfqpoint{3.570384in}{0.500000in}}%
\pgfpathlineto{\pgfqpoint{3.603409in}{0.500000in}}%
\pgfpathlineto{\pgfqpoint{3.603409in}{0.503021in}}%
\pgfpathlineto{\pgfqpoint{3.570384in}{0.503021in}}%
\pgfpathlineto{\pgfqpoint{3.570384in}{0.500000in}}%
\pgfpathclose%
\pgfusepath{fill}%
\end{pgfscope}%
\begin{pgfscope}%
\pgfpathrectangle{\pgfqpoint{0.750000in}{0.500000in}}{\pgfqpoint{4.650000in}{3.020000in}}%
\pgfusepath{clip}%
\pgfsetbuttcap%
\pgfsetmiterjoin%
\definecolor{currentfill}{rgb}{0.000000,0.500000,0.000000}%
\pgfsetfillcolor{currentfill}%
\pgfsetlinewidth{0.000000pt}%
\definecolor{currentstroke}{rgb}{0.000000,0.000000,0.000000}%
\pgfsetstrokecolor{currentstroke}%
\pgfsetstrokeopacity{0.000000}%
\pgfsetdash{}{0pt}%
\pgfpathmoveto{\pgfqpoint{3.603409in}{0.500000in}}%
\pgfpathlineto{\pgfqpoint{3.636435in}{0.500000in}}%
\pgfpathlineto{\pgfqpoint{3.636435in}{0.500000in}}%
\pgfpathlineto{\pgfqpoint{3.603409in}{0.500000in}}%
\pgfpathlineto{\pgfqpoint{3.603409in}{0.500000in}}%
\pgfpathclose%
\pgfusepath{fill}%
\end{pgfscope}%
\begin{pgfscope}%
\pgfpathrectangle{\pgfqpoint{0.750000in}{0.500000in}}{\pgfqpoint{4.650000in}{3.020000in}}%
\pgfusepath{clip}%
\pgfsetbuttcap%
\pgfsetmiterjoin%
\definecolor{currentfill}{rgb}{0.000000,0.500000,0.000000}%
\pgfsetfillcolor{currentfill}%
\pgfsetlinewidth{0.000000pt}%
\definecolor{currentstroke}{rgb}{0.000000,0.000000,0.000000}%
\pgfsetstrokecolor{currentstroke}%
\pgfsetstrokeopacity{0.000000}%
\pgfsetdash{}{0pt}%
\pgfpathmoveto{\pgfqpoint{3.636435in}{0.500000in}}%
\pgfpathlineto{\pgfqpoint{3.669460in}{0.500000in}}%
\pgfpathlineto{\pgfqpoint{3.669460in}{0.503021in}}%
\pgfpathlineto{\pgfqpoint{3.636435in}{0.503021in}}%
\pgfpathlineto{\pgfqpoint{3.636435in}{0.500000in}}%
\pgfpathclose%
\pgfusepath{fill}%
\end{pgfscope}%
\begin{pgfscope}%
\pgfpathrectangle{\pgfqpoint{0.750000in}{0.500000in}}{\pgfqpoint{4.650000in}{3.020000in}}%
\pgfusepath{clip}%
\pgfsetbuttcap%
\pgfsetmiterjoin%
\definecolor{currentfill}{rgb}{0.000000,0.500000,0.000000}%
\pgfsetfillcolor{currentfill}%
\pgfsetlinewidth{0.000000pt}%
\definecolor{currentstroke}{rgb}{0.000000,0.000000,0.000000}%
\pgfsetstrokecolor{currentstroke}%
\pgfsetstrokeopacity{0.000000}%
\pgfsetdash{}{0pt}%
\pgfpathmoveto{\pgfqpoint{3.669460in}{0.500000in}}%
\pgfpathlineto{\pgfqpoint{3.702486in}{0.500000in}}%
\pgfpathlineto{\pgfqpoint{3.702486in}{0.503021in}}%
\pgfpathlineto{\pgfqpoint{3.669460in}{0.503021in}}%
\pgfpathlineto{\pgfqpoint{3.669460in}{0.500000in}}%
\pgfpathclose%
\pgfusepath{fill}%
\end{pgfscope}%
\begin{pgfscope}%
\pgfpathrectangle{\pgfqpoint{0.750000in}{0.500000in}}{\pgfqpoint{4.650000in}{3.020000in}}%
\pgfusepath{clip}%
\pgfsetbuttcap%
\pgfsetmiterjoin%
\definecolor{currentfill}{rgb}{0.000000,0.500000,0.000000}%
\pgfsetfillcolor{currentfill}%
\pgfsetlinewidth{0.000000pt}%
\definecolor{currentstroke}{rgb}{0.000000,0.000000,0.000000}%
\pgfsetstrokecolor{currentstroke}%
\pgfsetstrokeopacity{0.000000}%
\pgfsetdash{}{0pt}%
\pgfpathmoveto{\pgfqpoint{3.702486in}{0.500000in}}%
\pgfpathlineto{\pgfqpoint{3.735511in}{0.500000in}}%
\pgfpathlineto{\pgfqpoint{3.735511in}{0.500000in}}%
\pgfpathlineto{\pgfqpoint{3.702486in}{0.500000in}}%
\pgfpathlineto{\pgfqpoint{3.702486in}{0.500000in}}%
\pgfpathclose%
\pgfusepath{fill}%
\end{pgfscope}%
\begin{pgfscope}%
\pgfpathrectangle{\pgfqpoint{0.750000in}{0.500000in}}{\pgfqpoint{4.650000in}{3.020000in}}%
\pgfusepath{clip}%
\pgfsetbuttcap%
\pgfsetmiterjoin%
\definecolor{currentfill}{rgb}{0.000000,0.500000,0.000000}%
\pgfsetfillcolor{currentfill}%
\pgfsetlinewidth{0.000000pt}%
\definecolor{currentstroke}{rgb}{0.000000,0.000000,0.000000}%
\pgfsetstrokecolor{currentstroke}%
\pgfsetstrokeopacity{0.000000}%
\pgfsetdash{}{0pt}%
\pgfpathmoveto{\pgfqpoint{3.735511in}{0.500000in}}%
\pgfpathlineto{\pgfqpoint{3.768537in}{0.500000in}}%
\pgfpathlineto{\pgfqpoint{3.768537in}{0.500000in}}%
\pgfpathlineto{\pgfqpoint{3.735511in}{0.500000in}}%
\pgfpathlineto{\pgfqpoint{3.735511in}{0.500000in}}%
\pgfpathclose%
\pgfusepath{fill}%
\end{pgfscope}%
\begin{pgfscope}%
\pgfpathrectangle{\pgfqpoint{0.750000in}{0.500000in}}{\pgfqpoint{4.650000in}{3.020000in}}%
\pgfusepath{clip}%
\pgfsetbuttcap%
\pgfsetmiterjoin%
\definecolor{currentfill}{rgb}{0.000000,0.500000,0.000000}%
\pgfsetfillcolor{currentfill}%
\pgfsetlinewidth{0.000000pt}%
\definecolor{currentstroke}{rgb}{0.000000,0.000000,0.000000}%
\pgfsetstrokecolor{currentstroke}%
\pgfsetstrokeopacity{0.000000}%
\pgfsetdash{}{0pt}%
\pgfpathmoveto{\pgfqpoint{3.768537in}{0.500000in}}%
\pgfpathlineto{\pgfqpoint{3.801562in}{0.500000in}}%
\pgfpathlineto{\pgfqpoint{3.801562in}{0.503021in}}%
\pgfpathlineto{\pgfqpoint{3.768537in}{0.503021in}}%
\pgfpathlineto{\pgfqpoint{3.768537in}{0.500000in}}%
\pgfpathclose%
\pgfusepath{fill}%
\end{pgfscope}%
\begin{pgfscope}%
\pgfpathrectangle{\pgfqpoint{0.750000in}{0.500000in}}{\pgfqpoint{4.650000in}{3.020000in}}%
\pgfusepath{clip}%
\pgfsetbuttcap%
\pgfsetmiterjoin%
\definecolor{currentfill}{rgb}{0.000000,0.500000,0.000000}%
\pgfsetfillcolor{currentfill}%
\pgfsetlinewidth{0.000000pt}%
\definecolor{currentstroke}{rgb}{0.000000,0.000000,0.000000}%
\pgfsetstrokecolor{currentstroke}%
\pgfsetstrokeopacity{0.000000}%
\pgfsetdash{}{0pt}%
\pgfpathmoveto{\pgfqpoint{3.801562in}{0.500000in}}%
\pgfpathlineto{\pgfqpoint{3.834588in}{0.500000in}}%
\pgfpathlineto{\pgfqpoint{3.834588in}{0.503021in}}%
\pgfpathlineto{\pgfqpoint{3.801562in}{0.503021in}}%
\pgfpathlineto{\pgfqpoint{3.801562in}{0.500000in}}%
\pgfpathclose%
\pgfusepath{fill}%
\end{pgfscope}%
\begin{pgfscope}%
\pgfpathrectangle{\pgfqpoint{0.750000in}{0.500000in}}{\pgfqpoint{4.650000in}{3.020000in}}%
\pgfusepath{clip}%
\pgfsetbuttcap%
\pgfsetmiterjoin%
\definecolor{currentfill}{rgb}{0.000000,0.500000,0.000000}%
\pgfsetfillcolor{currentfill}%
\pgfsetlinewidth{0.000000pt}%
\definecolor{currentstroke}{rgb}{0.000000,0.000000,0.000000}%
\pgfsetstrokecolor{currentstroke}%
\pgfsetstrokeopacity{0.000000}%
\pgfsetdash{}{0pt}%
\pgfpathmoveto{\pgfqpoint{3.834588in}{0.500000in}}%
\pgfpathlineto{\pgfqpoint{3.867614in}{0.500000in}}%
\pgfpathlineto{\pgfqpoint{3.867614in}{0.500000in}}%
\pgfpathlineto{\pgfqpoint{3.834588in}{0.500000in}}%
\pgfpathlineto{\pgfqpoint{3.834588in}{0.500000in}}%
\pgfpathclose%
\pgfusepath{fill}%
\end{pgfscope}%
\begin{pgfscope}%
\pgfpathrectangle{\pgfqpoint{0.750000in}{0.500000in}}{\pgfqpoint{4.650000in}{3.020000in}}%
\pgfusepath{clip}%
\pgfsetbuttcap%
\pgfsetmiterjoin%
\definecolor{currentfill}{rgb}{0.000000,0.500000,0.000000}%
\pgfsetfillcolor{currentfill}%
\pgfsetlinewidth{0.000000pt}%
\definecolor{currentstroke}{rgb}{0.000000,0.000000,0.000000}%
\pgfsetstrokecolor{currentstroke}%
\pgfsetstrokeopacity{0.000000}%
\pgfsetdash{}{0pt}%
\pgfpathmoveto{\pgfqpoint{3.867614in}{0.500000in}}%
\pgfpathlineto{\pgfqpoint{3.900639in}{0.500000in}}%
\pgfpathlineto{\pgfqpoint{3.900639in}{0.500000in}}%
\pgfpathlineto{\pgfqpoint{3.867614in}{0.500000in}}%
\pgfpathlineto{\pgfqpoint{3.867614in}{0.500000in}}%
\pgfpathclose%
\pgfusepath{fill}%
\end{pgfscope}%
\begin{pgfscope}%
\pgfpathrectangle{\pgfqpoint{0.750000in}{0.500000in}}{\pgfqpoint{4.650000in}{3.020000in}}%
\pgfusepath{clip}%
\pgfsetbuttcap%
\pgfsetmiterjoin%
\definecolor{currentfill}{rgb}{0.000000,0.500000,0.000000}%
\pgfsetfillcolor{currentfill}%
\pgfsetlinewidth{0.000000pt}%
\definecolor{currentstroke}{rgb}{0.000000,0.000000,0.000000}%
\pgfsetstrokecolor{currentstroke}%
\pgfsetstrokeopacity{0.000000}%
\pgfsetdash{}{0pt}%
\pgfpathmoveto{\pgfqpoint{3.900639in}{0.500000in}}%
\pgfpathlineto{\pgfqpoint{3.933665in}{0.500000in}}%
\pgfpathlineto{\pgfqpoint{3.933665in}{0.503021in}}%
\pgfpathlineto{\pgfqpoint{3.900639in}{0.503021in}}%
\pgfpathlineto{\pgfqpoint{3.900639in}{0.500000in}}%
\pgfpathclose%
\pgfusepath{fill}%
\end{pgfscope}%
\begin{pgfscope}%
\pgfpathrectangle{\pgfqpoint{0.750000in}{0.500000in}}{\pgfqpoint{4.650000in}{3.020000in}}%
\pgfusepath{clip}%
\pgfsetbuttcap%
\pgfsetmiterjoin%
\definecolor{currentfill}{rgb}{0.000000,0.500000,0.000000}%
\pgfsetfillcolor{currentfill}%
\pgfsetlinewidth{0.000000pt}%
\definecolor{currentstroke}{rgb}{0.000000,0.000000,0.000000}%
\pgfsetstrokecolor{currentstroke}%
\pgfsetstrokeopacity{0.000000}%
\pgfsetdash{}{0pt}%
\pgfpathmoveto{\pgfqpoint{3.933665in}{0.500000in}}%
\pgfpathlineto{\pgfqpoint{3.966690in}{0.500000in}}%
\pgfpathlineto{\pgfqpoint{3.966690in}{0.500000in}}%
\pgfpathlineto{\pgfqpoint{3.933665in}{0.500000in}}%
\pgfpathlineto{\pgfqpoint{3.933665in}{0.500000in}}%
\pgfpathclose%
\pgfusepath{fill}%
\end{pgfscope}%
\begin{pgfscope}%
\pgfpathrectangle{\pgfqpoint{0.750000in}{0.500000in}}{\pgfqpoint{4.650000in}{3.020000in}}%
\pgfusepath{clip}%
\pgfsetbuttcap%
\pgfsetmiterjoin%
\definecolor{currentfill}{rgb}{0.000000,0.500000,0.000000}%
\pgfsetfillcolor{currentfill}%
\pgfsetlinewidth{0.000000pt}%
\definecolor{currentstroke}{rgb}{0.000000,0.000000,0.000000}%
\pgfsetstrokecolor{currentstroke}%
\pgfsetstrokeopacity{0.000000}%
\pgfsetdash{}{0pt}%
\pgfpathmoveto{\pgfqpoint{3.966690in}{0.500000in}}%
\pgfpathlineto{\pgfqpoint{3.999716in}{0.500000in}}%
\pgfpathlineto{\pgfqpoint{3.999716in}{0.500000in}}%
\pgfpathlineto{\pgfqpoint{3.966690in}{0.500000in}}%
\pgfpathlineto{\pgfqpoint{3.966690in}{0.500000in}}%
\pgfpathclose%
\pgfusepath{fill}%
\end{pgfscope}%
\begin{pgfscope}%
\pgfpathrectangle{\pgfqpoint{0.750000in}{0.500000in}}{\pgfqpoint{4.650000in}{3.020000in}}%
\pgfusepath{clip}%
\pgfsetbuttcap%
\pgfsetmiterjoin%
\definecolor{currentfill}{rgb}{0.000000,0.500000,0.000000}%
\pgfsetfillcolor{currentfill}%
\pgfsetlinewidth{0.000000pt}%
\definecolor{currentstroke}{rgb}{0.000000,0.000000,0.000000}%
\pgfsetstrokecolor{currentstroke}%
\pgfsetstrokeopacity{0.000000}%
\pgfsetdash{}{0pt}%
\pgfpathmoveto{\pgfqpoint{3.999716in}{0.500000in}}%
\pgfpathlineto{\pgfqpoint{4.032741in}{0.500000in}}%
\pgfpathlineto{\pgfqpoint{4.032741in}{0.503021in}}%
\pgfpathlineto{\pgfqpoint{3.999716in}{0.503021in}}%
\pgfpathlineto{\pgfqpoint{3.999716in}{0.500000in}}%
\pgfpathclose%
\pgfusepath{fill}%
\end{pgfscope}%
\begin{pgfscope}%
\pgfpathrectangle{\pgfqpoint{0.750000in}{0.500000in}}{\pgfqpoint{4.650000in}{3.020000in}}%
\pgfusepath{clip}%
\pgfsetbuttcap%
\pgfsetmiterjoin%
\definecolor{currentfill}{rgb}{0.000000,0.500000,0.000000}%
\pgfsetfillcolor{currentfill}%
\pgfsetlinewidth{0.000000pt}%
\definecolor{currentstroke}{rgb}{0.000000,0.000000,0.000000}%
\pgfsetstrokecolor{currentstroke}%
\pgfsetstrokeopacity{0.000000}%
\pgfsetdash{}{0pt}%
\pgfpathmoveto{\pgfqpoint{4.032741in}{0.500000in}}%
\pgfpathlineto{\pgfqpoint{4.065767in}{0.500000in}}%
\pgfpathlineto{\pgfqpoint{4.065767in}{0.503021in}}%
\pgfpathlineto{\pgfqpoint{4.032741in}{0.503021in}}%
\pgfpathlineto{\pgfqpoint{4.032741in}{0.500000in}}%
\pgfpathclose%
\pgfusepath{fill}%
\end{pgfscope}%
\begin{pgfscope}%
\pgfpathrectangle{\pgfqpoint{0.750000in}{0.500000in}}{\pgfqpoint{4.650000in}{3.020000in}}%
\pgfusepath{clip}%
\pgfsetbuttcap%
\pgfsetmiterjoin%
\definecolor{currentfill}{rgb}{0.000000,0.500000,0.000000}%
\pgfsetfillcolor{currentfill}%
\pgfsetlinewidth{0.000000pt}%
\definecolor{currentstroke}{rgb}{0.000000,0.000000,0.000000}%
\pgfsetstrokecolor{currentstroke}%
\pgfsetstrokeopacity{0.000000}%
\pgfsetdash{}{0pt}%
\pgfpathmoveto{\pgfqpoint{4.065767in}{0.500000in}}%
\pgfpathlineto{\pgfqpoint{4.098793in}{0.500000in}}%
\pgfpathlineto{\pgfqpoint{4.098793in}{0.500000in}}%
\pgfpathlineto{\pgfqpoint{4.065767in}{0.500000in}}%
\pgfpathlineto{\pgfqpoint{4.065767in}{0.500000in}}%
\pgfpathclose%
\pgfusepath{fill}%
\end{pgfscope}%
\begin{pgfscope}%
\pgfpathrectangle{\pgfqpoint{0.750000in}{0.500000in}}{\pgfqpoint{4.650000in}{3.020000in}}%
\pgfusepath{clip}%
\pgfsetbuttcap%
\pgfsetmiterjoin%
\definecolor{currentfill}{rgb}{0.000000,0.500000,0.000000}%
\pgfsetfillcolor{currentfill}%
\pgfsetlinewidth{0.000000pt}%
\definecolor{currentstroke}{rgb}{0.000000,0.000000,0.000000}%
\pgfsetstrokecolor{currentstroke}%
\pgfsetstrokeopacity{0.000000}%
\pgfsetdash{}{0pt}%
\pgfpathmoveto{\pgfqpoint{4.098793in}{0.500000in}}%
\pgfpathlineto{\pgfqpoint{4.131818in}{0.500000in}}%
\pgfpathlineto{\pgfqpoint{4.131818in}{0.500000in}}%
\pgfpathlineto{\pgfqpoint{4.098793in}{0.500000in}}%
\pgfpathlineto{\pgfqpoint{4.098793in}{0.500000in}}%
\pgfpathclose%
\pgfusepath{fill}%
\end{pgfscope}%
\begin{pgfscope}%
\pgfpathrectangle{\pgfqpoint{0.750000in}{0.500000in}}{\pgfqpoint{4.650000in}{3.020000in}}%
\pgfusepath{clip}%
\pgfsetbuttcap%
\pgfsetmiterjoin%
\definecolor{currentfill}{rgb}{0.000000,0.500000,0.000000}%
\pgfsetfillcolor{currentfill}%
\pgfsetlinewidth{0.000000pt}%
\definecolor{currentstroke}{rgb}{0.000000,0.000000,0.000000}%
\pgfsetstrokecolor{currentstroke}%
\pgfsetstrokeopacity{0.000000}%
\pgfsetdash{}{0pt}%
\pgfpathmoveto{\pgfqpoint{4.131818in}{0.500000in}}%
\pgfpathlineto{\pgfqpoint{4.164844in}{0.500000in}}%
\pgfpathlineto{\pgfqpoint{4.164844in}{0.503021in}}%
\pgfpathlineto{\pgfqpoint{4.131818in}{0.503021in}}%
\pgfpathlineto{\pgfqpoint{4.131818in}{0.500000in}}%
\pgfpathclose%
\pgfusepath{fill}%
\end{pgfscope}%
\begin{pgfscope}%
\pgfpathrectangle{\pgfqpoint{0.750000in}{0.500000in}}{\pgfqpoint{4.650000in}{3.020000in}}%
\pgfusepath{clip}%
\pgfsetbuttcap%
\pgfsetmiterjoin%
\definecolor{currentfill}{rgb}{0.000000,0.500000,0.000000}%
\pgfsetfillcolor{currentfill}%
\pgfsetlinewidth{0.000000pt}%
\definecolor{currentstroke}{rgb}{0.000000,0.000000,0.000000}%
\pgfsetstrokecolor{currentstroke}%
\pgfsetstrokeopacity{0.000000}%
\pgfsetdash{}{0pt}%
\pgfpathmoveto{\pgfqpoint{4.164844in}{0.500000in}}%
\pgfpathlineto{\pgfqpoint{4.197869in}{0.500000in}}%
\pgfpathlineto{\pgfqpoint{4.197869in}{0.500000in}}%
\pgfpathlineto{\pgfqpoint{4.164844in}{0.500000in}}%
\pgfpathlineto{\pgfqpoint{4.164844in}{0.500000in}}%
\pgfpathclose%
\pgfusepath{fill}%
\end{pgfscope}%
\begin{pgfscope}%
\pgfpathrectangle{\pgfqpoint{0.750000in}{0.500000in}}{\pgfqpoint{4.650000in}{3.020000in}}%
\pgfusepath{clip}%
\pgfsetbuttcap%
\pgfsetmiterjoin%
\definecolor{currentfill}{rgb}{0.000000,0.500000,0.000000}%
\pgfsetfillcolor{currentfill}%
\pgfsetlinewidth{0.000000pt}%
\definecolor{currentstroke}{rgb}{0.000000,0.000000,0.000000}%
\pgfsetstrokecolor{currentstroke}%
\pgfsetstrokeopacity{0.000000}%
\pgfsetdash{}{0pt}%
\pgfpathmoveto{\pgfqpoint{4.197869in}{0.500000in}}%
\pgfpathlineto{\pgfqpoint{4.230895in}{0.500000in}}%
\pgfpathlineto{\pgfqpoint{4.230895in}{0.500000in}}%
\pgfpathlineto{\pgfqpoint{4.197869in}{0.500000in}}%
\pgfpathlineto{\pgfqpoint{4.197869in}{0.500000in}}%
\pgfpathclose%
\pgfusepath{fill}%
\end{pgfscope}%
\begin{pgfscope}%
\pgfpathrectangle{\pgfqpoint{0.750000in}{0.500000in}}{\pgfqpoint{4.650000in}{3.020000in}}%
\pgfusepath{clip}%
\pgfsetbuttcap%
\pgfsetmiterjoin%
\definecolor{currentfill}{rgb}{0.000000,0.500000,0.000000}%
\pgfsetfillcolor{currentfill}%
\pgfsetlinewidth{0.000000pt}%
\definecolor{currentstroke}{rgb}{0.000000,0.000000,0.000000}%
\pgfsetstrokecolor{currentstroke}%
\pgfsetstrokeopacity{0.000000}%
\pgfsetdash{}{0pt}%
\pgfpathmoveto{\pgfqpoint{4.230895in}{0.500000in}}%
\pgfpathlineto{\pgfqpoint{4.263920in}{0.500000in}}%
\pgfpathlineto{\pgfqpoint{4.263920in}{0.500000in}}%
\pgfpathlineto{\pgfqpoint{4.230895in}{0.500000in}}%
\pgfpathlineto{\pgfqpoint{4.230895in}{0.500000in}}%
\pgfpathclose%
\pgfusepath{fill}%
\end{pgfscope}%
\begin{pgfscope}%
\pgfpathrectangle{\pgfqpoint{0.750000in}{0.500000in}}{\pgfqpoint{4.650000in}{3.020000in}}%
\pgfusepath{clip}%
\pgfsetbuttcap%
\pgfsetmiterjoin%
\definecolor{currentfill}{rgb}{0.000000,0.500000,0.000000}%
\pgfsetfillcolor{currentfill}%
\pgfsetlinewidth{0.000000pt}%
\definecolor{currentstroke}{rgb}{0.000000,0.000000,0.000000}%
\pgfsetstrokecolor{currentstroke}%
\pgfsetstrokeopacity{0.000000}%
\pgfsetdash{}{0pt}%
\pgfpathmoveto{\pgfqpoint{4.263920in}{0.500000in}}%
\pgfpathlineto{\pgfqpoint{4.296946in}{0.500000in}}%
\pgfpathlineto{\pgfqpoint{4.296946in}{0.503021in}}%
\pgfpathlineto{\pgfqpoint{4.263920in}{0.503021in}}%
\pgfpathlineto{\pgfqpoint{4.263920in}{0.500000in}}%
\pgfpathclose%
\pgfusepath{fill}%
\end{pgfscope}%
\begin{pgfscope}%
\pgfpathrectangle{\pgfqpoint{0.750000in}{0.500000in}}{\pgfqpoint{4.650000in}{3.020000in}}%
\pgfusepath{clip}%
\pgfsetbuttcap%
\pgfsetmiterjoin%
\definecolor{currentfill}{rgb}{0.000000,0.500000,0.000000}%
\pgfsetfillcolor{currentfill}%
\pgfsetlinewidth{0.000000pt}%
\definecolor{currentstroke}{rgb}{0.000000,0.000000,0.000000}%
\pgfsetstrokecolor{currentstroke}%
\pgfsetstrokeopacity{0.000000}%
\pgfsetdash{}{0pt}%
\pgfpathmoveto{\pgfqpoint{4.296946in}{0.500000in}}%
\pgfpathlineto{\pgfqpoint{4.329972in}{0.500000in}}%
\pgfpathlineto{\pgfqpoint{4.329972in}{0.500000in}}%
\pgfpathlineto{\pgfqpoint{4.296946in}{0.500000in}}%
\pgfpathlineto{\pgfqpoint{4.296946in}{0.500000in}}%
\pgfpathclose%
\pgfusepath{fill}%
\end{pgfscope}%
\begin{pgfscope}%
\pgfpathrectangle{\pgfqpoint{0.750000in}{0.500000in}}{\pgfqpoint{4.650000in}{3.020000in}}%
\pgfusepath{clip}%
\pgfsetbuttcap%
\pgfsetmiterjoin%
\definecolor{currentfill}{rgb}{0.000000,0.500000,0.000000}%
\pgfsetfillcolor{currentfill}%
\pgfsetlinewidth{0.000000pt}%
\definecolor{currentstroke}{rgb}{0.000000,0.000000,0.000000}%
\pgfsetstrokecolor{currentstroke}%
\pgfsetstrokeopacity{0.000000}%
\pgfsetdash{}{0pt}%
\pgfpathmoveto{\pgfqpoint{4.329972in}{0.500000in}}%
\pgfpathlineto{\pgfqpoint{4.362997in}{0.500000in}}%
\pgfpathlineto{\pgfqpoint{4.362997in}{0.500000in}}%
\pgfpathlineto{\pgfqpoint{4.329972in}{0.500000in}}%
\pgfpathlineto{\pgfqpoint{4.329972in}{0.500000in}}%
\pgfpathclose%
\pgfusepath{fill}%
\end{pgfscope}%
\begin{pgfscope}%
\pgfpathrectangle{\pgfqpoint{0.750000in}{0.500000in}}{\pgfqpoint{4.650000in}{3.020000in}}%
\pgfusepath{clip}%
\pgfsetbuttcap%
\pgfsetmiterjoin%
\definecolor{currentfill}{rgb}{0.000000,0.500000,0.000000}%
\pgfsetfillcolor{currentfill}%
\pgfsetlinewidth{0.000000pt}%
\definecolor{currentstroke}{rgb}{0.000000,0.000000,0.000000}%
\pgfsetstrokecolor{currentstroke}%
\pgfsetstrokeopacity{0.000000}%
\pgfsetdash{}{0pt}%
\pgfpathmoveto{\pgfqpoint{4.362997in}{0.500000in}}%
\pgfpathlineto{\pgfqpoint{4.396023in}{0.500000in}}%
\pgfpathlineto{\pgfqpoint{4.396023in}{0.503021in}}%
\pgfpathlineto{\pgfqpoint{4.362997in}{0.503021in}}%
\pgfpathlineto{\pgfqpoint{4.362997in}{0.500000in}}%
\pgfpathclose%
\pgfusepath{fill}%
\end{pgfscope}%
\begin{pgfscope}%
\pgfpathrectangle{\pgfqpoint{0.750000in}{0.500000in}}{\pgfqpoint{4.650000in}{3.020000in}}%
\pgfusepath{clip}%
\pgfsetbuttcap%
\pgfsetmiterjoin%
\definecolor{currentfill}{rgb}{0.000000,0.500000,0.000000}%
\pgfsetfillcolor{currentfill}%
\pgfsetlinewidth{0.000000pt}%
\definecolor{currentstroke}{rgb}{0.000000,0.000000,0.000000}%
\pgfsetstrokecolor{currentstroke}%
\pgfsetstrokeopacity{0.000000}%
\pgfsetdash{}{0pt}%
\pgfpathmoveto{\pgfqpoint{4.396023in}{0.500000in}}%
\pgfpathlineto{\pgfqpoint{4.429048in}{0.500000in}}%
\pgfpathlineto{\pgfqpoint{4.429048in}{0.500000in}}%
\pgfpathlineto{\pgfqpoint{4.396023in}{0.500000in}}%
\pgfpathlineto{\pgfqpoint{4.396023in}{0.500000in}}%
\pgfpathclose%
\pgfusepath{fill}%
\end{pgfscope}%
\begin{pgfscope}%
\pgfpathrectangle{\pgfqpoint{0.750000in}{0.500000in}}{\pgfqpoint{4.650000in}{3.020000in}}%
\pgfusepath{clip}%
\pgfsetbuttcap%
\pgfsetmiterjoin%
\definecolor{currentfill}{rgb}{0.000000,0.500000,0.000000}%
\pgfsetfillcolor{currentfill}%
\pgfsetlinewidth{0.000000pt}%
\definecolor{currentstroke}{rgb}{0.000000,0.000000,0.000000}%
\pgfsetstrokecolor{currentstroke}%
\pgfsetstrokeopacity{0.000000}%
\pgfsetdash{}{0pt}%
\pgfpathmoveto{\pgfqpoint{4.429048in}{0.500000in}}%
\pgfpathlineto{\pgfqpoint{4.462074in}{0.500000in}}%
\pgfpathlineto{\pgfqpoint{4.462074in}{0.500000in}}%
\pgfpathlineto{\pgfqpoint{4.429048in}{0.500000in}}%
\pgfpathlineto{\pgfqpoint{4.429048in}{0.500000in}}%
\pgfpathclose%
\pgfusepath{fill}%
\end{pgfscope}%
\begin{pgfscope}%
\pgfpathrectangle{\pgfqpoint{0.750000in}{0.500000in}}{\pgfqpoint{4.650000in}{3.020000in}}%
\pgfusepath{clip}%
\pgfsetbuttcap%
\pgfsetmiterjoin%
\definecolor{currentfill}{rgb}{0.000000,0.500000,0.000000}%
\pgfsetfillcolor{currentfill}%
\pgfsetlinewidth{0.000000pt}%
\definecolor{currentstroke}{rgb}{0.000000,0.000000,0.000000}%
\pgfsetstrokecolor{currentstroke}%
\pgfsetstrokeopacity{0.000000}%
\pgfsetdash{}{0pt}%
\pgfpathmoveto{\pgfqpoint{4.462074in}{0.500000in}}%
\pgfpathlineto{\pgfqpoint{4.495099in}{0.500000in}}%
\pgfpathlineto{\pgfqpoint{4.495099in}{0.500000in}}%
\pgfpathlineto{\pgfqpoint{4.462074in}{0.500000in}}%
\pgfpathlineto{\pgfqpoint{4.462074in}{0.500000in}}%
\pgfpathclose%
\pgfusepath{fill}%
\end{pgfscope}%
\begin{pgfscope}%
\pgfpathrectangle{\pgfqpoint{0.750000in}{0.500000in}}{\pgfqpoint{4.650000in}{3.020000in}}%
\pgfusepath{clip}%
\pgfsetbuttcap%
\pgfsetmiterjoin%
\definecolor{currentfill}{rgb}{0.000000,0.500000,0.000000}%
\pgfsetfillcolor{currentfill}%
\pgfsetlinewidth{0.000000pt}%
\definecolor{currentstroke}{rgb}{0.000000,0.000000,0.000000}%
\pgfsetstrokecolor{currentstroke}%
\pgfsetstrokeopacity{0.000000}%
\pgfsetdash{}{0pt}%
\pgfpathmoveto{\pgfqpoint{4.495099in}{0.500000in}}%
\pgfpathlineto{\pgfqpoint{4.528125in}{0.500000in}}%
\pgfpathlineto{\pgfqpoint{4.528125in}{0.503021in}}%
\pgfpathlineto{\pgfqpoint{4.495099in}{0.503021in}}%
\pgfpathlineto{\pgfqpoint{4.495099in}{0.500000in}}%
\pgfpathclose%
\pgfusepath{fill}%
\end{pgfscope}%
\begin{pgfscope}%
\pgfpathrectangle{\pgfqpoint{0.750000in}{0.500000in}}{\pgfqpoint{4.650000in}{3.020000in}}%
\pgfusepath{clip}%
\pgfsetbuttcap%
\pgfsetmiterjoin%
\definecolor{currentfill}{rgb}{0.000000,0.500000,0.000000}%
\pgfsetfillcolor{currentfill}%
\pgfsetlinewidth{0.000000pt}%
\definecolor{currentstroke}{rgb}{0.000000,0.000000,0.000000}%
\pgfsetstrokecolor{currentstroke}%
\pgfsetstrokeopacity{0.000000}%
\pgfsetdash{}{0pt}%
\pgfpathmoveto{\pgfqpoint{4.528125in}{0.500000in}}%
\pgfpathlineto{\pgfqpoint{4.561151in}{0.500000in}}%
\pgfpathlineto{\pgfqpoint{4.561151in}{0.503021in}}%
\pgfpathlineto{\pgfqpoint{4.528125in}{0.503021in}}%
\pgfpathlineto{\pgfqpoint{4.528125in}{0.500000in}}%
\pgfpathclose%
\pgfusepath{fill}%
\end{pgfscope}%
\begin{pgfscope}%
\pgfpathrectangle{\pgfqpoint{0.750000in}{0.500000in}}{\pgfqpoint{4.650000in}{3.020000in}}%
\pgfusepath{clip}%
\pgfsetbuttcap%
\pgfsetmiterjoin%
\definecolor{currentfill}{rgb}{0.000000,0.500000,0.000000}%
\pgfsetfillcolor{currentfill}%
\pgfsetlinewidth{0.000000pt}%
\definecolor{currentstroke}{rgb}{0.000000,0.000000,0.000000}%
\pgfsetstrokecolor{currentstroke}%
\pgfsetstrokeopacity{0.000000}%
\pgfsetdash{}{0pt}%
\pgfpathmoveto{\pgfqpoint{4.561151in}{0.500000in}}%
\pgfpathlineto{\pgfqpoint{4.594176in}{0.500000in}}%
\pgfpathlineto{\pgfqpoint{4.594176in}{0.503021in}}%
\pgfpathlineto{\pgfqpoint{4.561151in}{0.503021in}}%
\pgfpathlineto{\pgfqpoint{4.561151in}{0.500000in}}%
\pgfpathclose%
\pgfusepath{fill}%
\end{pgfscope}%
\begin{pgfscope}%
\pgfpathrectangle{\pgfqpoint{0.750000in}{0.500000in}}{\pgfqpoint{4.650000in}{3.020000in}}%
\pgfusepath{clip}%
\pgfsetbuttcap%
\pgfsetmiterjoin%
\definecolor{currentfill}{rgb}{0.000000,0.500000,0.000000}%
\pgfsetfillcolor{currentfill}%
\pgfsetlinewidth{0.000000pt}%
\definecolor{currentstroke}{rgb}{0.000000,0.000000,0.000000}%
\pgfsetstrokecolor{currentstroke}%
\pgfsetstrokeopacity{0.000000}%
\pgfsetdash{}{0pt}%
\pgfpathmoveto{\pgfqpoint{4.594176in}{0.500000in}}%
\pgfpathlineto{\pgfqpoint{4.627202in}{0.500000in}}%
\pgfpathlineto{\pgfqpoint{4.627202in}{0.500000in}}%
\pgfpathlineto{\pgfqpoint{4.594176in}{0.500000in}}%
\pgfpathlineto{\pgfqpoint{4.594176in}{0.500000in}}%
\pgfpathclose%
\pgfusepath{fill}%
\end{pgfscope}%
\begin{pgfscope}%
\pgfpathrectangle{\pgfqpoint{0.750000in}{0.500000in}}{\pgfqpoint{4.650000in}{3.020000in}}%
\pgfusepath{clip}%
\pgfsetbuttcap%
\pgfsetmiterjoin%
\definecolor{currentfill}{rgb}{0.000000,0.500000,0.000000}%
\pgfsetfillcolor{currentfill}%
\pgfsetlinewidth{0.000000pt}%
\definecolor{currentstroke}{rgb}{0.000000,0.000000,0.000000}%
\pgfsetstrokecolor{currentstroke}%
\pgfsetstrokeopacity{0.000000}%
\pgfsetdash{}{0pt}%
\pgfpathmoveto{\pgfqpoint{4.627202in}{0.500000in}}%
\pgfpathlineto{\pgfqpoint{4.660227in}{0.500000in}}%
\pgfpathlineto{\pgfqpoint{4.660227in}{0.503021in}}%
\pgfpathlineto{\pgfqpoint{4.627202in}{0.503021in}}%
\pgfpathlineto{\pgfqpoint{4.627202in}{0.500000in}}%
\pgfpathclose%
\pgfusepath{fill}%
\end{pgfscope}%
\begin{pgfscope}%
\pgfpathrectangle{\pgfqpoint{0.750000in}{0.500000in}}{\pgfqpoint{4.650000in}{3.020000in}}%
\pgfusepath{clip}%
\pgfsetbuttcap%
\pgfsetmiterjoin%
\definecolor{currentfill}{rgb}{0.000000,0.500000,0.000000}%
\pgfsetfillcolor{currentfill}%
\pgfsetlinewidth{0.000000pt}%
\definecolor{currentstroke}{rgb}{0.000000,0.000000,0.000000}%
\pgfsetstrokecolor{currentstroke}%
\pgfsetstrokeopacity{0.000000}%
\pgfsetdash{}{0pt}%
\pgfpathmoveto{\pgfqpoint{4.660227in}{0.500000in}}%
\pgfpathlineto{\pgfqpoint{4.693253in}{0.500000in}}%
\pgfpathlineto{\pgfqpoint{4.693253in}{0.503021in}}%
\pgfpathlineto{\pgfqpoint{4.660227in}{0.503021in}}%
\pgfpathlineto{\pgfqpoint{4.660227in}{0.500000in}}%
\pgfpathclose%
\pgfusepath{fill}%
\end{pgfscope}%
\begin{pgfscope}%
\pgfpathrectangle{\pgfqpoint{0.750000in}{0.500000in}}{\pgfqpoint{4.650000in}{3.020000in}}%
\pgfusepath{clip}%
\pgfsetbuttcap%
\pgfsetmiterjoin%
\definecolor{currentfill}{rgb}{0.000000,0.500000,0.000000}%
\pgfsetfillcolor{currentfill}%
\pgfsetlinewidth{0.000000pt}%
\definecolor{currentstroke}{rgb}{0.000000,0.000000,0.000000}%
\pgfsetstrokecolor{currentstroke}%
\pgfsetstrokeopacity{0.000000}%
\pgfsetdash{}{0pt}%
\pgfpathmoveto{\pgfqpoint{4.693253in}{0.500000in}}%
\pgfpathlineto{\pgfqpoint{4.726278in}{0.500000in}}%
\pgfpathlineto{\pgfqpoint{4.726278in}{0.503021in}}%
\pgfpathlineto{\pgfqpoint{4.693253in}{0.503021in}}%
\pgfpathlineto{\pgfqpoint{4.693253in}{0.500000in}}%
\pgfpathclose%
\pgfusepath{fill}%
\end{pgfscope}%
\begin{pgfscope}%
\pgfpathrectangle{\pgfqpoint{0.750000in}{0.500000in}}{\pgfqpoint{4.650000in}{3.020000in}}%
\pgfusepath{clip}%
\pgfsetbuttcap%
\pgfsetmiterjoin%
\definecolor{currentfill}{rgb}{0.000000,0.500000,0.000000}%
\pgfsetfillcolor{currentfill}%
\pgfsetlinewidth{0.000000pt}%
\definecolor{currentstroke}{rgb}{0.000000,0.000000,0.000000}%
\pgfsetstrokecolor{currentstroke}%
\pgfsetstrokeopacity{0.000000}%
\pgfsetdash{}{0pt}%
\pgfpathmoveto{\pgfqpoint{4.726278in}{0.500000in}}%
\pgfpathlineto{\pgfqpoint{4.759304in}{0.500000in}}%
\pgfpathlineto{\pgfqpoint{4.759304in}{0.500000in}}%
\pgfpathlineto{\pgfqpoint{4.726278in}{0.500000in}}%
\pgfpathlineto{\pgfqpoint{4.726278in}{0.500000in}}%
\pgfpathclose%
\pgfusepath{fill}%
\end{pgfscope}%
\begin{pgfscope}%
\pgfpathrectangle{\pgfqpoint{0.750000in}{0.500000in}}{\pgfqpoint{4.650000in}{3.020000in}}%
\pgfusepath{clip}%
\pgfsetbuttcap%
\pgfsetmiterjoin%
\definecolor{currentfill}{rgb}{0.000000,0.500000,0.000000}%
\pgfsetfillcolor{currentfill}%
\pgfsetlinewidth{0.000000pt}%
\definecolor{currentstroke}{rgb}{0.000000,0.000000,0.000000}%
\pgfsetstrokecolor{currentstroke}%
\pgfsetstrokeopacity{0.000000}%
\pgfsetdash{}{0pt}%
\pgfpathmoveto{\pgfqpoint{4.759304in}{0.500000in}}%
\pgfpathlineto{\pgfqpoint{4.792330in}{0.500000in}}%
\pgfpathlineto{\pgfqpoint{4.792330in}{0.500000in}}%
\pgfpathlineto{\pgfqpoint{4.759304in}{0.500000in}}%
\pgfpathlineto{\pgfqpoint{4.759304in}{0.500000in}}%
\pgfpathclose%
\pgfusepath{fill}%
\end{pgfscope}%
\begin{pgfscope}%
\pgfpathrectangle{\pgfqpoint{0.750000in}{0.500000in}}{\pgfqpoint{4.650000in}{3.020000in}}%
\pgfusepath{clip}%
\pgfsetbuttcap%
\pgfsetmiterjoin%
\definecolor{currentfill}{rgb}{0.000000,0.500000,0.000000}%
\pgfsetfillcolor{currentfill}%
\pgfsetlinewidth{0.000000pt}%
\definecolor{currentstroke}{rgb}{0.000000,0.000000,0.000000}%
\pgfsetstrokecolor{currentstroke}%
\pgfsetstrokeopacity{0.000000}%
\pgfsetdash{}{0pt}%
\pgfpathmoveto{\pgfqpoint{4.792330in}{0.500000in}}%
\pgfpathlineto{\pgfqpoint{4.825355in}{0.500000in}}%
\pgfpathlineto{\pgfqpoint{4.825355in}{0.506042in}}%
\pgfpathlineto{\pgfqpoint{4.792330in}{0.506042in}}%
\pgfpathlineto{\pgfqpoint{4.792330in}{0.500000in}}%
\pgfpathclose%
\pgfusepath{fill}%
\end{pgfscope}%
\begin{pgfscope}%
\pgfpathrectangle{\pgfqpoint{0.750000in}{0.500000in}}{\pgfqpoint{4.650000in}{3.020000in}}%
\pgfusepath{clip}%
\pgfsetbuttcap%
\pgfsetmiterjoin%
\definecolor{currentfill}{rgb}{0.000000,0.500000,0.000000}%
\pgfsetfillcolor{currentfill}%
\pgfsetlinewidth{0.000000pt}%
\definecolor{currentstroke}{rgb}{0.000000,0.000000,0.000000}%
\pgfsetstrokecolor{currentstroke}%
\pgfsetstrokeopacity{0.000000}%
\pgfsetdash{}{0pt}%
\pgfpathmoveto{\pgfqpoint{4.825355in}{0.500000in}}%
\pgfpathlineto{\pgfqpoint{4.858381in}{0.500000in}}%
\pgfpathlineto{\pgfqpoint{4.858381in}{0.500000in}}%
\pgfpathlineto{\pgfqpoint{4.825355in}{0.500000in}}%
\pgfpathlineto{\pgfqpoint{4.825355in}{0.500000in}}%
\pgfpathclose%
\pgfusepath{fill}%
\end{pgfscope}%
\begin{pgfscope}%
\pgfpathrectangle{\pgfqpoint{0.750000in}{0.500000in}}{\pgfqpoint{4.650000in}{3.020000in}}%
\pgfusepath{clip}%
\pgfsetbuttcap%
\pgfsetmiterjoin%
\definecolor{currentfill}{rgb}{0.000000,0.500000,0.000000}%
\pgfsetfillcolor{currentfill}%
\pgfsetlinewidth{0.000000pt}%
\definecolor{currentstroke}{rgb}{0.000000,0.000000,0.000000}%
\pgfsetstrokecolor{currentstroke}%
\pgfsetstrokeopacity{0.000000}%
\pgfsetdash{}{0pt}%
\pgfpathmoveto{\pgfqpoint{4.858381in}{0.500000in}}%
\pgfpathlineto{\pgfqpoint{4.891406in}{0.500000in}}%
\pgfpathlineto{\pgfqpoint{4.891406in}{0.506042in}}%
\pgfpathlineto{\pgfqpoint{4.858381in}{0.506042in}}%
\pgfpathlineto{\pgfqpoint{4.858381in}{0.500000in}}%
\pgfpathclose%
\pgfusepath{fill}%
\end{pgfscope}%
\begin{pgfscope}%
\pgfpathrectangle{\pgfqpoint{0.750000in}{0.500000in}}{\pgfqpoint{4.650000in}{3.020000in}}%
\pgfusepath{clip}%
\pgfsetbuttcap%
\pgfsetmiterjoin%
\definecolor{currentfill}{rgb}{0.000000,0.500000,0.000000}%
\pgfsetfillcolor{currentfill}%
\pgfsetlinewidth{0.000000pt}%
\definecolor{currentstroke}{rgb}{0.000000,0.000000,0.000000}%
\pgfsetstrokecolor{currentstroke}%
\pgfsetstrokeopacity{0.000000}%
\pgfsetdash{}{0pt}%
\pgfpathmoveto{\pgfqpoint{4.891406in}{0.500000in}}%
\pgfpathlineto{\pgfqpoint{4.924432in}{0.500000in}}%
\pgfpathlineto{\pgfqpoint{4.924432in}{0.503021in}}%
\pgfpathlineto{\pgfqpoint{4.891406in}{0.503021in}}%
\pgfpathlineto{\pgfqpoint{4.891406in}{0.500000in}}%
\pgfpathclose%
\pgfusepath{fill}%
\end{pgfscope}%
\begin{pgfscope}%
\pgfpathrectangle{\pgfqpoint{0.750000in}{0.500000in}}{\pgfqpoint{4.650000in}{3.020000in}}%
\pgfusepath{clip}%
\pgfsetbuttcap%
\pgfsetmiterjoin%
\definecolor{currentfill}{rgb}{0.000000,0.500000,0.000000}%
\pgfsetfillcolor{currentfill}%
\pgfsetlinewidth{0.000000pt}%
\definecolor{currentstroke}{rgb}{0.000000,0.000000,0.000000}%
\pgfsetstrokecolor{currentstroke}%
\pgfsetstrokeopacity{0.000000}%
\pgfsetdash{}{0pt}%
\pgfpathmoveto{\pgfqpoint{4.924432in}{0.500000in}}%
\pgfpathlineto{\pgfqpoint{4.957457in}{0.500000in}}%
\pgfpathlineto{\pgfqpoint{4.957457in}{0.500000in}}%
\pgfpathlineto{\pgfqpoint{4.924432in}{0.500000in}}%
\pgfpathlineto{\pgfqpoint{4.924432in}{0.500000in}}%
\pgfpathclose%
\pgfusepath{fill}%
\end{pgfscope}%
\begin{pgfscope}%
\pgfpathrectangle{\pgfqpoint{0.750000in}{0.500000in}}{\pgfqpoint{4.650000in}{3.020000in}}%
\pgfusepath{clip}%
\pgfsetbuttcap%
\pgfsetmiterjoin%
\definecolor{currentfill}{rgb}{0.000000,0.500000,0.000000}%
\pgfsetfillcolor{currentfill}%
\pgfsetlinewidth{0.000000pt}%
\definecolor{currentstroke}{rgb}{0.000000,0.000000,0.000000}%
\pgfsetstrokecolor{currentstroke}%
\pgfsetstrokeopacity{0.000000}%
\pgfsetdash{}{0pt}%
\pgfpathmoveto{\pgfqpoint{4.957457in}{0.500000in}}%
\pgfpathlineto{\pgfqpoint{4.990483in}{0.500000in}}%
\pgfpathlineto{\pgfqpoint{4.990483in}{0.506042in}}%
\pgfpathlineto{\pgfqpoint{4.957457in}{0.506042in}}%
\pgfpathlineto{\pgfqpoint{4.957457in}{0.500000in}}%
\pgfpathclose%
\pgfusepath{fill}%
\end{pgfscope}%
\begin{pgfscope}%
\pgfpathrectangle{\pgfqpoint{0.750000in}{0.500000in}}{\pgfqpoint{4.650000in}{3.020000in}}%
\pgfusepath{clip}%
\pgfsetbuttcap%
\pgfsetmiterjoin%
\definecolor{currentfill}{rgb}{0.000000,0.500000,0.000000}%
\pgfsetfillcolor{currentfill}%
\pgfsetlinewidth{0.000000pt}%
\definecolor{currentstroke}{rgb}{0.000000,0.000000,0.000000}%
\pgfsetstrokecolor{currentstroke}%
\pgfsetstrokeopacity{0.000000}%
\pgfsetdash{}{0pt}%
\pgfpathmoveto{\pgfqpoint{4.990483in}{0.500000in}}%
\pgfpathlineto{\pgfqpoint{5.023509in}{0.500000in}}%
\pgfpathlineto{\pgfqpoint{5.023509in}{0.500000in}}%
\pgfpathlineto{\pgfqpoint{4.990483in}{0.500000in}}%
\pgfpathlineto{\pgfqpoint{4.990483in}{0.500000in}}%
\pgfpathclose%
\pgfusepath{fill}%
\end{pgfscope}%
\begin{pgfscope}%
\pgfpathrectangle{\pgfqpoint{0.750000in}{0.500000in}}{\pgfqpoint{4.650000in}{3.020000in}}%
\pgfusepath{clip}%
\pgfsetbuttcap%
\pgfsetmiterjoin%
\definecolor{currentfill}{rgb}{0.000000,0.500000,0.000000}%
\pgfsetfillcolor{currentfill}%
\pgfsetlinewidth{0.000000pt}%
\definecolor{currentstroke}{rgb}{0.000000,0.000000,0.000000}%
\pgfsetstrokecolor{currentstroke}%
\pgfsetstrokeopacity{0.000000}%
\pgfsetdash{}{0pt}%
\pgfpathmoveto{\pgfqpoint{5.023509in}{0.500000in}}%
\pgfpathlineto{\pgfqpoint{5.056534in}{0.500000in}}%
\pgfpathlineto{\pgfqpoint{5.056534in}{0.509064in}}%
\pgfpathlineto{\pgfqpoint{5.023509in}{0.509064in}}%
\pgfpathlineto{\pgfqpoint{5.023509in}{0.500000in}}%
\pgfpathclose%
\pgfusepath{fill}%
\end{pgfscope}%
\begin{pgfscope}%
\pgfpathrectangle{\pgfqpoint{0.750000in}{0.500000in}}{\pgfqpoint{4.650000in}{3.020000in}}%
\pgfusepath{clip}%
\pgfsetbuttcap%
\pgfsetmiterjoin%
\definecolor{currentfill}{rgb}{0.000000,0.500000,0.000000}%
\pgfsetfillcolor{currentfill}%
\pgfsetlinewidth{0.000000pt}%
\definecolor{currentstroke}{rgb}{0.000000,0.000000,0.000000}%
\pgfsetstrokecolor{currentstroke}%
\pgfsetstrokeopacity{0.000000}%
\pgfsetdash{}{0pt}%
\pgfpathmoveto{\pgfqpoint{5.056534in}{0.500000in}}%
\pgfpathlineto{\pgfqpoint{5.089560in}{0.500000in}}%
\pgfpathlineto{\pgfqpoint{5.089560in}{0.509064in}}%
\pgfpathlineto{\pgfqpoint{5.056534in}{0.509064in}}%
\pgfpathlineto{\pgfqpoint{5.056534in}{0.500000in}}%
\pgfpathclose%
\pgfusepath{fill}%
\end{pgfscope}%
\begin{pgfscope}%
\pgfpathrectangle{\pgfqpoint{0.750000in}{0.500000in}}{\pgfqpoint{4.650000in}{3.020000in}}%
\pgfusepath{clip}%
\pgfsetbuttcap%
\pgfsetmiterjoin%
\definecolor{currentfill}{rgb}{0.000000,0.500000,0.000000}%
\pgfsetfillcolor{currentfill}%
\pgfsetlinewidth{0.000000pt}%
\definecolor{currentstroke}{rgb}{0.000000,0.000000,0.000000}%
\pgfsetstrokecolor{currentstroke}%
\pgfsetstrokeopacity{0.000000}%
\pgfsetdash{}{0pt}%
\pgfpathmoveto{\pgfqpoint{5.089560in}{0.500000in}}%
\pgfpathlineto{\pgfqpoint{5.122585in}{0.500000in}}%
\pgfpathlineto{\pgfqpoint{5.122585in}{0.503021in}}%
\pgfpathlineto{\pgfqpoint{5.089560in}{0.503021in}}%
\pgfpathlineto{\pgfqpoint{5.089560in}{0.500000in}}%
\pgfpathclose%
\pgfusepath{fill}%
\end{pgfscope}%
\begin{pgfscope}%
\pgfpathrectangle{\pgfqpoint{0.750000in}{0.500000in}}{\pgfqpoint{4.650000in}{3.020000in}}%
\pgfusepath{clip}%
\pgfsetbuttcap%
\pgfsetmiterjoin%
\definecolor{currentfill}{rgb}{0.000000,0.500000,0.000000}%
\pgfsetfillcolor{currentfill}%
\pgfsetlinewidth{0.000000pt}%
\definecolor{currentstroke}{rgb}{0.000000,0.000000,0.000000}%
\pgfsetstrokecolor{currentstroke}%
\pgfsetstrokeopacity{0.000000}%
\pgfsetdash{}{0pt}%
\pgfpathmoveto{\pgfqpoint{5.122585in}{0.500000in}}%
\pgfpathlineto{\pgfqpoint{5.155611in}{0.500000in}}%
\pgfpathlineto{\pgfqpoint{5.155611in}{0.506042in}}%
\pgfpathlineto{\pgfqpoint{5.122585in}{0.506042in}}%
\pgfpathlineto{\pgfqpoint{5.122585in}{0.500000in}}%
\pgfpathclose%
\pgfusepath{fill}%
\end{pgfscope}%
\begin{pgfscope}%
\pgfpathrectangle{\pgfqpoint{0.750000in}{0.500000in}}{\pgfqpoint{4.650000in}{3.020000in}}%
\pgfusepath{clip}%
\pgfsetbuttcap%
\pgfsetmiterjoin%
\definecolor{currentfill}{rgb}{0.000000,0.500000,0.000000}%
\pgfsetfillcolor{currentfill}%
\pgfsetlinewidth{0.000000pt}%
\definecolor{currentstroke}{rgb}{0.000000,0.000000,0.000000}%
\pgfsetstrokecolor{currentstroke}%
\pgfsetstrokeopacity{0.000000}%
\pgfsetdash{}{0pt}%
\pgfpathmoveto{\pgfqpoint{5.155611in}{0.500000in}}%
\pgfpathlineto{\pgfqpoint{5.188636in}{0.500000in}}%
\pgfpathlineto{\pgfqpoint{5.188636in}{0.515106in}}%
\pgfpathlineto{\pgfqpoint{5.155611in}{0.515106in}}%
\pgfpathlineto{\pgfqpoint{5.155611in}{0.500000in}}%
\pgfpathclose%
\pgfusepath{fill}%
\end{pgfscope}%
\begin{pgfscope}%
\pgfsetbuttcap%
\pgfsetroundjoin%
\definecolor{currentfill}{rgb}{0.000000,0.000000,0.000000}%
\pgfsetfillcolor{currentfill}%
\pgfsetlinewidth{0.803000pt}%
\definecolor{currentstroke}{rgb}{0.000000,0.000000,0.000000}%
\pgfsetstrokecolor{currentstroke}%
\pgfsetdash{}{0pt}%
\pgfsys@defobject{currentmarker}{\pgfqpoint{0.000000in}{-0.048611in}}{\pgfqpoint{0.000000in}{0.000000in}}{%
\pgfpathmoveto{\pgfqpoint{0.000000in}{0.000000in}}%
\pgfpathlineto{\pgfqpoint{0.000000in}{-0.048611in}}%
\pgfusepath{stroke,fill}%
}%
\begin{pgfscope}%
\pgfsys@transformshift{0.945286in}{0.500000in}%
\pgfsys@useobject{currentmarker}{}%
\end{pgfscope}%
\end{pgfscope}%
\begin{pgfscope}%
\definecolor{textcolor}{rgb}{0.000000,0.000000,0.000000}%
\pgfsetstrokecolor{textcolor}%
\pgfsetfillcolor{textcolor}%
\pgftext[x=0.945286in,y=0.402778in,,top]{\color{textcolor}\rmfamily\fontsize{13.000000}{15.600000}\selectfont \(\displaystyle {0.0}\)}%
\end{pgfscope}%
\begin{pgfscope}%
\pgfsetbuttcap%
\pgfsetroundjoin%
\definecolor{currentfill}{rgb}{0.000000,0.000000,0.000000}%
\pgfsetfillcolor{currentfill}%
\pgfsetlinewidth{0.803000pt}%
\definecolor{currentstroke}{rgb}{0.000000,0.000000,0.000000}%
\pgfsetstrokecolor{currentstroke}%
\pgfsetdash{}{0pt}%
\pgfsys@defobject{currentmarker}{\pgfqpoint{0.000000in}{-0.048611in}}{\pgfqpoint{0.000000in}{0.000000in}}{%
\pgfpathmoveto{\pgfqpoint{0.000000in}{0.000000in}}%
\pgfpathlineto{\pgfqpoint{0.000000in}{-0.048611in}}%
\pgfusepath{stroke,fill}%
}%
\begin{pgfscope}%
\pgfsys@transformshift{1.792379in}{0.500000in}%
\pgfsys@useobject{currentmarker}{}%
\end{pgfscope}%
\end{pgfscope}%
\begin{pgfscope}%
\definecolor{textcolor}{rgb}{0.000000,0.000000,0.000000}%
\pgfsetstrokecolor{textcolor}%
\pgfsetfillcolor{textcolor}%
\pgftext[x=1.792379in,y=0.402778in,,top]{\color{textcolor}\rmfamily\fontsize{13.000000}{15.600000}\selectfont \(\displaystyle {0.1}\)}%
\end{pgfscope}%
\begin{pgfscope}%
\pgfsetbuttcap%
\pgfsetroundjoin%
\definecolor{currentfill}{rgb}{0.000000,0.000000,0.000000}%
\pgfsetfillcolor{currentfill}%
\pgfsetlinewidth{0.803000pt}%
\definecolor{currentstroke}{rgb}{0.000000,0.000000,0.000000}%
\pgfsetstrokecolor{currentstroke}%
\pgfsetdash{}{0pt}%
\pgfsys@defobject{currentmarker}{\pgfqpoint{0.000000in}{-0.048611in}}{\pgfqpoint{0.000000in}{0.000000in}}{%
\pgfpathmoveto{\pgfqpoint{0.000000in}{0.000000in}}%
\pgfpathlineto{\pgfqpoint{0.000000in}{-0.048611in}}%
\pgfusepath{stroke,fill}%
}%
\begin{pgfscope}%
\pgfsys@transformshift{2.639473in}{0.500000in}%
\pgfsys@useobject{currentmarker}{}%
\end{pgfscope}%
\end{pgfscope}%
\begin{pgfscope}%
\definecolor{textcolor}{rgb}{0.000000,0.000000,0.000000}%
\pgfsetstrokecolor{textcolor}%
\pgfsetfillcolor{textcolor}%
\pgftext[x=2.639473in,y=0.402778in,,top]{\color{textcolor}\rmfamily\fontsize{13.000000}{15.600000}\selectfont \(\displaystyle {0.2}\)}%
\end{pgfscope}%
\begin{pgfscope}%
\pgfsetbuttcap%
\pgfsetroundjoin%
\definecolor{currentfill}{rgb}{0.000000,0.000000,0.000000}%
\pgfsetfillcolor{currentfill}%
\pgfsetlinewidth{0.803000pt}%
\definecolor{currentstroke}{rgb}{0.000000,0.000000,0.000000}%
\pgfsetstrokecolor{currentstroke}%
\pgfsetdash{}{0pt}%
\pgfsys@defobject{currentmarker}{\pgfqpoint{0.000000in}{-0.048611in}}{\pgfqpoint{0.000000in}{0.000000in}}{%
\pgfpathmoveto{\pgfqpoint{0.000000in}{0.000000in}}%
\pgfpathlineto{\pgfqpoint{0.000000in}{-0.048611in}}%
\pgfusepath{stroke,fill}%
}%
\begin{pgfscope}%
\pgfsys@transformshift{3.486567in}{0.500000in}%
\pgfsys@useobject{currentmarker}{}%
\end{pgfscope}%
\end{pgfscope}%
\begin{pgfscope}%
\definecolor{textcolor}{rgb}{0.000000,0.000000,0.000000}%
\pgfsetstrokecolor{textcolor}%
\pgfsetfillcolor{textcolor}%
\pgftext[x=3.486567in,y=0.402778in,,top]{\color{textcolor}\rmfamily\fontsize{13.000000}{15.600000}\selectfont \(\displaystyle {0.3}\)}%
\end{pgfscope}%
\begin{pgfscope}%
\pgfsetbuttcap%
\pgfsetroundjoin%
\definecolor{currentfill}{rgb}{0.000000,0.000000,0.000000}%
\pgfsetfillcolor{currentfill}%
\pgfsetlinewidth{0.803000pt}%
\definecolor{currentstroke}{rgb}{0.000000,0.000000,0.000000}%
\pgfsetstrokecolor{currentstroke}%
\pgfsetdash{}{0pt}%
\pgfsys@defobject{currentmarker}{\pgfqpoint{0.000000in}{-0.048611in}}{\pgfqpoint{0.000000in}{0.000000in}}{%
\pgfpathmoveto{\pgfqpoint{0.000000in}{0.000000in}}%
\pgfpathlineto{\pgfqpoint{0.000000in}{-0.048611in}}%
\pgfusepath{stroke,fill}%
}%
\begin{pgfscope}%
\pgfsys@transformshift{4.333661in}{0.500000in}%
\pgfsys@useobject{currentmarker}{}%
\end{pgfscope}%
\end{pgfscope}%
\begin{pgfscope}%
\definecolor{textcolor}{rgb}{0.000000,0.000000,0.000000}%
\pgfsetstrokecolor{textcolor}%
\pgfsetfillcolor{textcolor}%
\pgftext[x=4.333661in,y=0.402778in,,top]{\color{textcolor}\rmfamily\fontsize{13.000000}{15.600000}\selectfont \(\displaystyle {0.4}\)}%
\end{pgfscope}%
\begin{pgfscope}%
\pgfsetbuttcap%
\pgfsetroundjoin%
\definecolor{currentfill}{rgb}{0.000000,0.000000,0.000000}%
\pgfsetfillcolor{currentfill}%
\pgfsetlinewidth{0.803000pt}%
\definecolor{currentstroke}{rgb}{0.000000,0.000000,0.000000}%
\pgfsetstrokecolor{currentstroke}%
\pgfsetdash{}{0pt}%
\pgfsys@defobject{currentmarker}{\pgfqpoint{0.000000in}{-0.048611in}}{\pgfqpoint{0.000000in}{0.000000in}}{%
\pgfpathmoveto{\pgfqpoint{0.000000in}{0.000000in}}%
\pgfpathlineto{\pgfqpoint{0.000000in}{-0.048611in}}%
\pgfusepath{stroke,fill}%
}%
\begin{pgfscope}%
\pgfsys@transformshift{5.180754in}{0.500000in}%
\pgfsys@useobject{currentmarker}{}%
\end{pgfscope}%
\end{pgfscope}%
\begin{pgfscope}%
\definecolor{textcolor}{rgb}{0.000000,0.000000,0.000000}%
\pgfsetstrokecolor{textcolor}%
\pgfsetfillcolor{textcolor}%
\pgftext[x=5.180754in,y=0.402778in,,top]{\color{textcolor}\rmfamily\fontsize{13.000000}{15.600000}\selectfont \(\displaystyle {0.5}\)}%
\end{pgfscope}%
\begin{pgfscope}%
\definecolor{textcolor}{rgb}{0.000000,0.000000,0.000000}%
\pgfsetstrokecolor{textcolor}%
\pgfsetfillcolor{textcolor}%
\pgftext[x=3.075000in,y=0.199075in,,top]{\color{textcolor}\rmfamily\fontsize{13.000000}{15.600000}\selectfont Loss}%
\end{pgfscope}%
\begin{pgfscope}%
\pgfsetbuttcap%
\pgfsetroundjoin%
\definecolor{currentfill}{rgb}{0.000000,0.000000,0.000000}%
\pgfsetfillcolor{currentfill}%
\pgfsetlinewidth{0.803000pt}%
\definecolor{currentstroke}{rgb}{0.000000,0.000000,0.000000}%
\pgfsetstrokecolor{currentstroke}%
\pgfsetdash{}{0pt}%
\pgfsys@defobject{currentmarker}{\pgfqpoint{-0.048611in}{0.000000in}}{\pgfqpoint{-0.000000in}{0.000000in}}{%
\pgfpathmoveto{\pgfqpoint{-0.000000in}{0.000000in}}%
\pgfpathlineto{\pgfqpoint{-0.048611in}{0.000000in}}%
\pgfusepath{stroke,fill}%
}%
\begin{pgfscope}%
\pgfsys@transformshift{0.750000in}{0.500000in}%
\pgfsys@useobject{currentmarker}{}%
\end{pgfscope}%
\end{pgfscope}%
\begin{pgfscope}%
\definecolor{textcolor}{rgb}{0.000000,0.000000,0.000000}%
\pgfsetstrokecolor{textcolor}%
\pgfsetfillcolor{textcolor}%
\pgftext[x=0.571181in, y=0.442130in, left, base]{\color{textcolor}\rmfamily\fontsize{13.000000}{15.600000}\selectfont \(\displaystyle {0}\)}%
\end{pgfscope}%
\begin{pgfscope}%
\pgfsetbuttcap%
\pgfsetroundjoin%
\definecolor{currentfill}{rgb}{0.000000,0.000000,0.000000}%
\pgfsetfillcolor{currentfill}%
\pgfsetlinewidth{0.803000pt}%
\definecolor{currentstroke}{rgb}{0.000000,0.000000,0.000000}%
\pgfsetstrokecolor{currentstroke}%
\pgfsetdash{}{0pt}%
\pgfsys@defobject{currentmarker}{\pgfqpoint{-0.048611in}{0.000000in}}{\pgfqpoint{-0.000000in}{0.000000in}}{%
\pgfpathmoveto{\pgfqpoint{-0.000000in}{0.000000in}}%
\pgfpathlineto{\pgfqpoint{-0.048611in}{0.000000in}}%
\pgfusepath{stroke,fill}%
}%
\begin{pgfscope}%
\pgfsys@transformshift{0.750000in}{1.104242in}%
\pgfsys@useobject{currentmarker}{}%
\end{pgfscope}%
\end{pgfscope}%
\begin{pgfscope}%
\definecolor{textcolor}{rgb}{0.000000,0.000000,0.000000}%
\pgfsetstrokecolor{textcolor}%
\pgfsetfillcolor{textcolor}%
\pgftext[x=0.407989in, y=1.046371in, left, base]{\color{textcolor}\rmfamily\fontsize{13.000000}{15.600000}\selectfont \(\displaystyle {200}\)}%
\end{pgfscope}%
\begin{pgfscope}%
\pgfsetbuttcap%
\pgfsetroundjoin%
\definecolor{currentfill}{rgb}{0.000000,0.000000,0.000000}%
\pgfsetfillcolor{currentfill}%
\pgfsetlinewidth{0.803000pt}%
\definecolor{currentstroke}{rgb}{0.000000,0.000000,0.000000}%
\pgfsetstrokecolor{currentstroke}%
\pgfsetdash{}{0pt}%
\pgfsys@defobject{currentmarker}{\pgfqpoint{-0.048611in}{0.000000in}}{\pgfqpoint{-0.000000in}{0.000000in}}{%
\pgfpathmoveto{\pgfqpoint{-0.000000in}{0.000000in}}%
\pgfpathlineto{\pgfqpoint{-0.048611in}{0.000000in}}%
\pgfusepath{stroke,fill}%
}%
\begin{pgfscope}%
\pgfsys@transformshift{0.750000in}{1.708483in}%
\pgfsys@useobject{currentmarker}{}%
\end{pgfscope}%
\end{pgfscope}%
\begin{pgfscope}%
\definecolor{textcolor}{rgb}{0.000000,0.000000,0.000000}%
\pgfsetstrokecolor{textcolor}%
\pgfsetfillcolor{textcolor}%
\pgftext[x=0.407989in, y=1.650613in, left, base]{\color{textcolor}\rmfamily\fontsize{13.000000}{15.600000}\selectfont \(\displaystyle {400}\)}%
\end{pgfscope}%
\begin{pgfscope}%
\pgfsetbuttcap%
\pgfsetroundjoin%
\definecolor{currentfill}{rgb}{0.000000,0.000000,0.000000}%
\pgfsetfillcolor{currentfill}%
\pgfsetlinewidth{0.803000pt}%
\definecolor{currentstroke}{rgb}{0.000000,0.000000,0.000000}%
\pgfsetstrokecolor{currentstroke}%
\pgfsetdash{}{0pt}%
\pgfsys@defobject{currentmarker}{\pgfqpoint{-0.048611in}{0.000000in}}{\pgfqpoint{-0.000000in}{0.000000in}}{%
\pgfpathmoveto{\pgfqpoint{-0.000000in}{0.000000in}}%
\pgfpathlineto{\pgfqpoint{-0.048611in}{0.000000in}}%
\pgfusepath{stroke,fill}%
}%
\begin{pgfscope}%
\pgfsys@transformshift{0.750000in}{2.312725in}%
\pgfsys@useobject{currentmarker}{}%
\end{pgfscope}%
\end{pgfscope}%
\begin{pgfscope}%
\definecolor{textcolor}{rgb}{0.000000,0.000000,0.000000}%
\pgfsetstrokecolor{textcolor}%
\pgfsetfillcolor{textcolor}%
\pgftext[x=0.407989in, y=2.254855in, left, base]{\color{textcolor}\rmfamily\fontsize{13.000000}{15.600000}\selectfont \(\displaystyle {600}\)}%
\end{pgfscope}%
\begin{pgfscope}%
\pgfsetbuttcap%
\pgfsetroundjoin%
\definecolor{currentfill}{rgb}{0.000000,0.000000,0.000000}%
\pgfsetfillcolor{currentfill}%
\pgfsetlinewidth{0.803000pt}%
\definecolor{currentstroke}{rgb}{0.000000,0.000000,0.000000}%
\pgfsetstrokecolor{currentstroke}%
\pgfsetdash{}{0pt}%
\pgfsys@defobject{currentmarker}{\pgfqpoint{-0.048611in}{0.000000in}}{\pgfqpoint{-0.000000in}{0.000000in}}{%
\pgfpathmoveto{\pgfqpoint{-0.000000in}{0.000000in}}%
\pgfpathlineto{\pgfqpoint{-0.048611in}{0.000000in}}%
\pgfusepath{stroke,fill}%
}%
\begin{pgfscope}%
\pgfsys@transformshift{0.750000in}{2.916967in}%
\pgfsys@useobject{currentmarker}{}%
\end{pgfscope}%
\end{pgfscope}%
\begin{pgfscope}%
\definecolor{textcolor}{rgb}{0.000000,0.000000,0.000000}%
\pgfsetstrokecolor{textcolor}%
\pgfsetfillcolor{textcolor}%
\pgftext[x=0.407989in, y=2.859097in, left, base]{\color{textcolor}\rmfamily\fontsize{13.000000}{15.600000}\selectfont \(\displaystyle {800}\)}%
\end{pgfscope}%
\begin{pgfscope}%
\definecolor{textcolor}{rgb}{0.000000,0.000000,0.000000}%
\pgfsetstrokecolor{textcolor}%
\pgfsetfillcolor{textcolor}%
\pgftext[x=0.352433in,y=2.010000in,,bottom,rotate=90.000000]{\color{textcolor}\rmfamily\fontsize{13.000000}{15.600000}\selectfont Count}%
\end{pgfscope}%
\begin{pgfscope}%
\pgfsetrectcap%
\pgfsetmiterjoin%
\pgfsetlinewidth{0.803000pt}%
\definecolor{currentstroke}{rgb}{0.000000,0.000000,0.000000}%
\pgfsetstrokecolor{currentstroke}%
\pgfsetdash{}{0pt}%
\pgfpathmoveto{\pgfqpoint{0.750000in}{0.500000in}}%
\pgfpathlineto{\pgfqpoint{0.750000in}{3.520000in}}%
\pgfusepath{stroke}%
\end{pgfscope}%
\begin{pgfscope}%
\pgfsetrectcap%
\pgfsetmiterjoin%
\pgfsetlinewidth{0.803000pt}%
\definecolor{currentstroke}{rgb}{0.000000,0.000000,0.000000}%
\pgfsetstrokecolor{currentstroke}%
\pgfsetdash{}{0pt}%
\pgfpathmoveto{\pgfqpoint{5.400000in}{0.500000in}}%
\pgfpathlineto{\pgfqpoint{5.400000in}{3.520000in}}%
\pgfusepath{stroke}%
\end{pgfscope}%
\begin{pgfscope}%
\pgfsetrectcap%
\pgfsetmiterjoin%
\pgfsetlinewidth{0.803000pt}%
\definecolor{currentstroke}{rgb}{0.000000,0.000000,0.000000}%
\pgfsetstrokecolor{currentstroke}%
\pgfsetdash{}{0pt}%
\pgfpathmoveto{\pgfqpoint{0.750000in}{0.500000in}}%
\pgfpathlineto{\pgfqpoint{5.400000in}{0.500000in}}%
\pgfusepath{stroke}%
\end{pgfscope}%
\begin{pgfscope}%
\pgfsetrectcap%
\pgfsetmiterjoin%
\pgfsetlinewidth{0.803000pt}%
\definecolor{currentstroke}{rgb}{0.000000,0.000000,0.000000}%
\pgfsetstrokecolor{currentstroke}%
\pgfsetdash{}{0pt}%
\pgfpathmoveto{\pgfqpoint{0.750000in}{3.520000in}}%
\pgfpathlineto{\pgfqpoint{5.400000in}{3.520000in}}%
\pgfusepath{stroke}%
\end{pgfscope}%
\begin{pgfscope}%
\definecolor{textcolor}{rgb}{0.000000,0.000000,0.000000}%
\pgfsetstrokecolor{textcolor}%
\pgfsetfillcolor{textcolor}%
\pgftext[x=3.075000in,y=3.603333in,,base]{\color{textcolor}\rmfamily\fontsize{13.000000}{15.600000}\selectfont Loss Histogram for \(\displaystyle f(x)=2x\)}%
\end{pgfscope}%
\begin{pgfscope}%
\pgfsetbuttcap%
\pgfsetmiterjoin%
\definecolor{currentfill}{rgb}{1.000000,1.000000,1.000000}%
\pgfsetfillcolor{currentfill}%
\pgfsetfillopacity{0.800000}%
\pgfsetlinewidth{1.003750pt}%
\definecolor{currentstroke}{rgb}{0.800000,0.800000,0.800000}%
\pgfsetstrokecolor{currentstroke}%
\pgfsetstrokeopacity{0.800000}%
\pgfsetdash{}{0pt}%
\pgfpathmoveto{\pgfqpoint{4.360497in}{2.877408in}}%
\pgfpathlineto{\pgfqpoint{5.273611in}{2.877408in}}%
\pgfpathquadraticcurveto{\pgfqpoint{5.309722in}{2.877408in}}{\pgfqpoint{5.309722in}{2.913519in}}%
\pgfpathlineto{\pgfqpoint{5.309722in}{3.393611in}}%
\pgfpathquadraticcurveto{\pgfqpoint{5.309722in}{3.429722in}}{\pgfqpoint{5.273611in}{3.429722in}}%
\pgfpathlineto{\pgfqpoint{4.360497in}{3.429722in}}%
\pgfpathquadraticcurveto{\pgfqpoint{4.324386in}{3.429722in}}{\pgfqpoint{4.324386in}{3.393611in}}%
\pgfpathlineto{\pgfqpoint{4.324386in}{2.913519in}}%
\pgfpathquadraticcurveto{\pgfqpoint{4.324386in}{2.877408in}}{\pgfqpoint{4.360497in}{2.877408in}}%
\pgfpathlineto{\pgfqpoint{4.360497in}{2.877408in}}%
\pgfpathclose%
\pgfusepath{stroke,fill}%
\end{pgfscope}%
\begin{pgfscope}%
\pgfsetbuttcap%
\pgfsetmiterjoin%
\definecolor{currentfill}{rgb}{1.000000,0.000000,0.000000}%
\pgfsetfillcolor{currentfill}%
\pgfsetlinewidth{0.000000pt}%
\definecolor{currentstroke}{rgb}{0.000000,0.000000,0.000000}%
\pgfsetstrokecolor{currentstroke}%
\pgfsetstrokeopacity{0.000000}%
\pgfsetdash{}{0pt}%
\pgfpathmoveto{\pgfqpoint{4.396608in}{3.231111in}}%
\pgfpathlineto{\pgfqpoint{4.757719in}{3.231111in}}%
\pgfpathlineto{\pgfqpoint{4.757719in}{3.357500in}}%
\pgfpathlineto{\pgfqpoint{4.396608in}{3.357500in}}%
\pgfpathlineto{\pgfqpoint{4.396608in}{3.231111in}}%
\pgfpathclose%
\pgfusepath{fill}%
\end{pgfscope}%
\begin{pgfscope}%
\definecolor{textcolor}{rgb}{0.000000,0.000000,0.000000}%
\pgfsetstrokecolor{textcolor}%
\pgfsetfillcolor{textcolor}%
\pgftext[x=4.902164in,y=3.231111in,left,base]{\color{textcolor}\rmfamily\fontsize{13.000000}{15.600000}\selectfont SNN}%
\end{pgfscope}%
\begin{pgfscope}%
\pgfsetbuttcap%
\pgfsetmiterjoin%
\definecolor{currentfill}{rgb}{0.000000,0.500000,0.000000}%
\pgfsetfillcolor{currentfill}%
\pgfsetlinewidth{0.000000pt}%
\definecolor{currentstroke}{rgb}{0.000000,0.000000,0.000000}%
\pgfsetstrokecolor{currentstroke}%
\pgfsetstrokeopacity{0.000000}%
\pgfsetdash{}{0pt}%
\pgfpathmoveto{\pgfqpoint{4.396608in}{2.982037in}}%
\pgfpathlineto{\pgfqpoint{4.757719in}{2.982037in}}%
\pgfpathlineto{\pgfqpoint{4.757719in}{3.108426in}}%
\pgfpathlineto{\pgfqpoint{4.396608in}{3.108426in}}%
\pgfpathlineto{\pgfqpoint{4.396608in}{2.982037in}}%
\pgfpathclose%
\pgfusepath{fill}%
\end{pgfscope}%
\begin{pgfscope}%
\definecolor{textcolor}{rgb}{0.000000,0.000000,0.000000}%
\pgfsetstrokecolor{textcolor}%
\pgfsetfillcolor{textcolor}%
\pgftext[x=4.902164in,y=2.982037in,left,base]{\color{textcolor}\rmfamily\fontsize{13.000000}{15.600000}\selectfont NN}%
\end{pgfscope}%
\end{pgfpicture}%
\makeatother%
\endgroup%

    \caption{Caption}
    \label{fig:my_label}
\end{figure}

\begin{figure}
%% Creator: Matplotlib, PGF backend
%%
%% To include the figure in your LaTeX document, write
%%   \input{<filename>.pgf}
%%
%% Make sure the required packages are loaded in your preamble
%%   \usepackage{pgf}
%%
%% Also ensure that all the required font packages are loaded; for instance,
%% the lmodern package is sometimes necessary when using math font.
%%   \usepackage{lmodern}
%%
%% Figures using additional raster images can only be included by \input if
%% they are in the same directory as the main LaTeX file. For loading figures
%% from other directories you can use the `import` package
%%   \usepackage{import}
%%
%% and then include the figures with
%%   \import{<path to file>}{<filename>.pgf}
%%
%% Matplotlib used the following preamble
%%
\begingroup%
\makeatletter%
\begin{pgfpicture}%
\pgfpathrectangle{\pgfpointorigin}{\pgfqpoint{6.000000in}{4.000000in}}%
\pgfusepath{use as bounding box, clip}%
\begin{pgfscope}%
\pgfsetbuttcap%
\pgfsetmiterjoin%
\pgfsetlinewidth{0.000000pt}%
\definecolor{currentstroke}{rgb}{1.000000,1.000000,1.000000}%
\pgfsetstrokecolor{currentstroke}%
\pgfsetstrokeopacity{0.000000}%
\pgfsetdash{}{0pt}%
\pgfpathmoveto{\pgfqpoint{0.000000in}{0.000000in}}%
\pgfpathlineto{\pgfqpoint{6.000000in}{0.000000in}}%
\pgfpathlineto{\pgfqpoint{6.000000in}{4.000000in}}%
\pgfpathlineto{\pgfqpoint{0.000000in}{4.000000in}}%
\pgfpathlineto{\pgfqpoint{0.000000in}{0.000000in}}%
\pgfpathclose%
\pgfusepath{}%
\end{pgfscope}%
\begin{pgfscope}%
\pgfsetbuttcap%
\pgfsetmiterjoin%
\definecolor{currentfill}{rgb}{1.000000,1.000000,1.000000}%
\pgfsetfillcolor{currentfill}%
\pgfsetlinewidth{0.000000pt}%
\definecolor{currentstroke}{rgb}{0.000000,0.000000,0.000000}%
\pgfsetstrokecolor{currentstroke}%
\pgfsetstrokeopacity{0.000000}%
\pgfsetdash{}{0pt}%
\pgfpathmoveto{\pgfqpoint{0.750000in}{0.500000in}}%
\pgfpathlineto{\pgfqpoint{5.400000in}{0.500000in}}%
\pgfpathlineto{\pgfqpoint{5.400000in}{3.520000in}}%
\pgfpathlineto{\pgfqpoint{0.750000in}{3.520000in}}%
\pgfpathlineto{\pgfqpoint{0.750000in}{0.500000in}}%
\pgfpathclose%
\pgfusepath{fill}%
\end{pgfscope}%
\begin{pgfscope}%
\pgfpathrectangle{\pgfqpoint{0.750000in}{0.500000in}}{\pgfqpoint{4.650000in}{3.020000in}}%
\pgfusepath{clip}%
\pgfsetbuttcap%
\pgfsetmiterjoin%
\definecolor{currentfill}{rgb}{1.000000,0.000000,0.000000}%
\pgfsetfillcolor{currentfill}%
\pgfsetlinewidth{0.000000pt}%
\definecolor{currentstroke}{rgb}{0.000000,0.000000,0.000000}%
\pgfsetstrokecolor{currentstroke}%
\pgfsetstrokeopacity{0.000000}%
\pgfsetdash{}{0pt}%
\pgfpathmoveto{\pgfqpoint{0.961364in}{0.500000in}}%
\pgfpathlineto{\pgfqpoint{0.994389in}{0.500000in}}%
\pgfpathlineto{\pgfqpoint{0.994389in}{0.506042in}}%
\pgfpathlineto{\pgfqpoint{0.961364in}{0.506042in}}%
\pgfpathlineto{\pgfqpoint{0.961364in}{0.500000in}}%
\pgfpathclose%
\pgfusepath{fill}%
\end{pgfscope}%
\begin{pgfscope}%
\pgfpathrectangle{\pgfqpoint{0.750000in}{0.500000in}}{\pgfqpoint{4.650000in}{3.020000in}}%
\pgfusepath{clip}%
\pgfsetbuttcap%
\pgfsetmiterjoin%
\definecolor{currentfill}{rgb}{1.000000,0.000000,0.000000}%
\pgfsetfillcolor{currentfill}%
\pgfsetlinewidth{0.000000pt}%
\definecolor{currentstroke}{rgb}{0.000000,0.000000,0.000000}%
\pgfsetstrokecolor{currentstroke}%
\pgfsetstrokeopacity{0.000000}%
\pgfsetdash{}{0pt}%
\pgfpathmoveto{\pgfqpoint{0.994389in}{0.500000in}}%
\pgfpathlineto{\pgfqpoint{1.027415in}{0.500000in}}%
\pgfpathlineto{\pgfqpoint{1.027415in}{0.500000in}}%
\pgfpathlineto{\pgfqpoint{0.994389in}{0.500000in}}%
\pgfpathlineto{\pgfqpoint{0.994389in}{0.500000in}}%
\pgfpathclose%
\pgfusepath{fill}%
\end{pgfscope}%
\begin{pgfscope}%
\pgfpathrectangle{\pgfqpoint{0.750000in}{0.500000in}}{\pgfqpoint{4.650000in}{3.020000in}}%
\pgfusepath{clip}%
\pgfsetbuttcap%
\pgfsetmiterjoin%
\definecolor{currentfill}{rgb}{1.000000,0.000000,0.000000}%
\pgfsetfillcolor{currentfill}%
\pgfsetlinewidth{0.000000pt}%
\definecolor{currentstroke}{rgb}{0.000000,0.000000,0.000000}%
\pgfsetstrokecolor{currentstroke}%
\pgfsetstrokeopacity{0.000000}%
\pgfsetdash{}{0pt}%
\pgfpathmoveto{\pgfqpoint{1.027415in}{0.500000in}}%
\pgfpathlineto{\pgfqpoint{1.060440in}{0.500000in}}%
\pgfpathlineto{\pgfqpoint{1.060440in}{0.500000in}}%
\pgfpathlineto{\pgfqpoint{1.027415in}{0.500000in}}%
\pgfpathlineto{\pgfqpoint{1.027415in}{0.500000in}}%
\pgfpathclose%
\pgfusepath{fill}%
\end{pgfscope}%
\begin{pgfscope}%
\pgfpathrectangle{\pgfqpoint{0.750000in}{0.500000in}}{\pgfqpoint{4.650000in}{3.020000in}}%
\pgfusepath{clip}%
\pgfsetbuttcap%
\pgfsetmiterjoin%
\definecolor{currentfill}{rgb}{1.000000,0.000000,0.000000}%
\pgfsetfillcolor{currentfill}%
\pgfsetlinewidth{0.000000pt}%
\definecolor{currentstroke}{rgb}{0.000000,0.000000,0.000000}%
\pgfsetstrokecolor{currentstroke}%
\pgfsetstrokeopacity{0.000000}%
\pgfsetdash{}{0pt}%
\pgfpathmoveto{\pgfqpoint{1.060440in}{0.500000in}}%
\pgfpathlineto{\pgfqpoint{1.093466in}{0.500000in}}%
\pgfpathlineto{\pgfqpoint{1.093466in}{0.503021in}}%
\pgfpathlineto{\pgfqpoint{1.060440in}{0.503021in}}%
\pgfpathlineto{\pgfqpoint{1.060440in}{0.500000in}}%
\pgfpathclose%
\pgfusepath{fill}%
\end{pgfscope}%
\begin{pgfscope}%
\pgfpathrectangle{\pgfqpoint{0.750000in}{0.500000in}}{\pgfqpoint{4.650000in}{3.020000in}}%
\pgfusepath{clip}%
\pgfsetbuttcap%
\pgfsetmiterjoin%
\definecolor{currentfill}{rgb}{1.000000,0.000000,0.000000}%
\pgfsetfillcolor{currentfill}%
\pgfsetlinewidth{0.000000pt}%
\definecolor{currentstroke}{rgb}{0.000000,0.000000,0.000000}%
\pgfsetstrokecolor{currentstroke}%
\pgfsetstrokeopacity{0.000000}%
\pgfsetdash{}{0pt}%
\pgfpathmoveto{\pgfqpoint{1.093466in}{0.500000in}}%
\pgfpathlineto{\pgfqpoint{1.126491in}{0.500000in}}%
\pgfpathlineto{\pgfqpoint{1.126491in}{0.500000in}}%
\pgfpathlineto{\pgfqpoint{1.093466in}{0.500000in}}%
\pgfpathlineto{\pgfqpoint{1.093466in}{0.500000in}}%
\pgfpathclose%
\pgfusepath{fill}%
\end{pgfscope}%
\begin{pgfscope}%
\pgfpathrectangle{\pgfqpoint{0.750000in}{0.500000in}}{\pgfqpoint{4.650000in}{3.020000in}}%
\pgfusepath{clip}%
\pgfsetbuttcap%
\pgfsetmiterjoin%
\definecolor{currentfill}{rgb}{1.000000,0.000000,0.000000}%
\pgfsetfillcolor{currentfill}%
\pgfsetlinewidth{0.000000pt}%
\definecolor{currentstroke}{rgb}{0.000000,0.000000,0.000000}%
\pgfsetstrokecolor{currentstroke}%
\pgfsetstrokeopacity{0.000000}%
\pgfsetdash{}{0pt}%
\pgfpathmoveto{\pgfqpoint{1.126491in}{0.500000in}}%
\pgfpathlineto{\pgfqpoint{1.159517in}{0.500000in}}%
\pgfpathlineto{\pgfqpoint{1.159517in}{0.509064in}}%
\pgfpathlineto{\pgfqpoint{1.126491in}{0.509064in}}%
\pgfpathlineto{\pgfqpoint{1.126491in}{0.500000in}}%
\pgfpathclose%
\pgfusepath{fill}%
\end{pgfscope}%
\begin{pgfscope}%
\pgfpathrectangle{\pgfqpoint{0.750000in}{0.500000in}}{\pgfqpoint{4.650000in}{3.020000in}}%
\pgfusepath{clip}%
\pgfsetbuttcap%
\pgfsetmiterjoin%
\definecolor{currentfill}{rgb}{1.000000,0.000000,0.000000}%
\pgfsetfillcolor{currentfill}%
\pgfsetlinewidth{0.000000pt}%
\definecolor{currentstroke}{rgb}{0.000000,0.000000,0.000000}%
\pgfsetstrokecolor{currentstroke}%
\pgfsetstrokeopacity{0.000000}%
\pgfsetdash{}{0pt}%
\pgfpathmoveto{\pgfqpoint{1.159517in}{0.500000in}}%
\pgfpathlineto{\pgfqpoint{1.192543in}{0.500000in}}%
\pgfpathlineto{\pgfqpoint{1.192543in}{0.500000in}}%
\pgfpathlineto{\pgfqpoint{1.159517in}{0.500000in}}%
\pgfpathlineto{\pgfqpoint{1.159517in}{0.500000in}}%
\pgfpathclose%
\pgfusepath{fill}%
\end{pgfscope}%
\begin{pgfscope}%
\pgfpathrectangle{\pgfqpoint{0.750000in}{0.500000in}}{\pgfqpoint{4.650000in}{3.020000in}}%
\pgfusepath{clip}%
\pgfsetbuttcap%
\pgfsetmiterjoin%
\definecolor{currentfill}{rgb}{1.000000,0.000000,0.000000}%
\pgfsetfillcolor{currentfill}%
\pgfsetlinewidth{0.000000pt}%
\definecolor{currentstroke}{rgb}{0.000000,0.000000,0.000000}%
\pgfsetstrokecolor{currentstroke}%
\pgfsetstrokeopacity{0.000000}%
\pgfsetdash{}{0pt}%
\pgfpathmoveto{\pgfqpoint{1.192543in}{0.500000in}}%
\pgfpathlineto{\pgfqpoint{1.225568in}{0.500000in}}%
\pgfpathlineto{\pgfqpoint{1.225568in}{0.500000in}}%
\pgfpathlineto{\pgfqpoint{1.192543in}{0.500000in}}%
\pgfpathlineto{\pgfqpoint{1.192543in}{0.500000in}}%
\pgfpathclose%
\pgfusepath{fill}%
\end{pgfscope}%
\begin{pgfscope}%
\pgfpathrectangle{\pgfqpoint{0.750000in}{0.500000in}}{\pgfqpoint{4.650000in}{3.020000in}}%
\pgfusepath{clip}%
\pgfsetbuttcap%
\pgfsetmiterjoin%
\definecolor{currentfill}{rgb}{1.000000,0.000000,0.000000}%
\pgfsetfillcolor{currentfill}%
\pgfsetlinewidth{0.000000pt}%
\definecolor{currentstroke}{rgb}{0.000000,0.000000,0.000000}%
\pgfsetstrokecolor{currentstroke}%
\pgfsetstrokeopacity{0.000000}%
\pgfsetdash{}{0pt}%
\pgfpathmoveto{\pgfqpoint{1.225568in}{0.500000in}}%
\pgfpathlineto{\pgfqpoint{1.258594in}{0.500000in}}%
\pgfpathlineto{\pgfqpoint{1.258594in}{0.536255in}}%
\pgfpathlineto{\pgfqpoint{1.225568in}{0.536255in}}%
\pgfpathlineto{\pgfqpoint{1.225568in}{0.500000in}}%
\pgfpathclose%
\pgfusepath{fill}%
\end{pgfscope}%
\begin{pgfscope}%
\pgfpathrectangle{\pgfqpoint{0.750000in}{0.500000in}}{\pgfqpoint{4.650000in}{3.020000in}}%
\pgfusepath{clip}%
\pgfsetbuttcap%
\pgfsetmiterjoin%
\definecolor{currentfill}{rgb}{1.000000,0.000000,0.000000}%
\pgfsetfillcolor{currentfill}%
\pgfsetlinewidth{0.000000pt}%
\definecolor{currentstroke}{rgb}{0.000000,0.000000,0.000000}%
\pgfsetstrokecolor{currentstroke}%
\pgfsetstrokeopacity{0.000000}%
\pgfsetdash{}{0pt}%
\pgfpathmoveto{\pgfqpoint{1.258594in}{0.500000in}}%
\pgfpathlineto{\pgfqpoint{1.291619in}{0.500000in}}%
\pgfpathlineto{\pgfqpoint{1.291619in}{0.500000in}}%
\pgfpathlineto{\pgfqpoint{1.258594in}{0.500000in}}%
\pgfpathlineto{\pgfqpoint{1.258594in}{0.500000in}}%
\pgfpathclose%
\pgfusepath{fill}%
\end{pgfscope}%
\begin{pgfscope}%
\pgfpathrectangle{\pgfqpoint{0.750000in}{0.500000in}}{\pgfqpoint{4.650000in}{3.020000in}}%
\pgfusepath{clip}%
\pgfsetbuttcap%
\pgfsetmiterjoin%
\definecolor{currentfill}{rgb}{1.000000,0.000000,0.000000}%
\pgfsetfillcolor{currentfill}%
\pgfsetlinewidth{0.000000pt}%
\definecolor{currentstroke}{rgb}{0.000000,0.000000,0.000000}%
\pgfsetstrokecolor{currentstroke}%
\pgfsetstrokeopacity{0.000000}%
\pgfsetdash{}{0pt}%
\pgfpathmoveto{\pgfqpoint{1.291619in}{0.500000in}}%
\pgfpathlineto{\pgfqpoint{1.324645in}{0.500000in}}%
\pgfpathlineto{\pgfqpoint{1.324645in}{0.566467in}}%
\pgfpathlineto{\pgfqpoint{1.291619in}{0.566467in}}%
\pgfpathlineto{\pgfqpoint{1.291619in}{0.500000in}}%
\pgfpathclose%
\pgfusepath{fill}%
\end{pgfscope}%
\begin{pgfscope}%
\pgfpathrectangle{\pgfqpoint{0.750000in}{0.500000in}}{\pgfqpoint{4.650000in}{3.020000in}}%
\pgfusepath{clip}%
\pgfsetbuttcap%
\pgfsetmiterjoin%
\definecolor{currentfill}{rgb}{1.000000,0.000000,0.000000}%
\pgfsetfillcolor{currentfill}%
\pgfsetlinewidth{0.000000pt}%
\definecolor{currentstroke}{rgb}{0.000000,0.000000,0.000000}%
\pgfsetstrokecolor{currentstroke}%
\pgfsetstrokeopacity{0.000000}%
\pgfsetdash{}{0pt}%
\pgfpathmoveto{\pgfqpoint{1.324645in}{0.500000in}}%
\pgfpathlineto{\pgfqpoint{1.357670in}{0.500000in}}%
\pgfpathlineto{\pgfqpoint{1.357670in}{0.503021in}}%
\pgfpathlineto{\pgfqpoint{1.324645in}{0.503021in}}%
\pgfpathlineto{\pgfqpoint{1.324645in}{0.500000in}}%
\pgfpathclose%
\pgfusepath{fill}%
\end{pgfscope}%
\begin{pgfscope}%
\pgfpathrectangle{\pgfqpoint{0.750000in}{0.500000in}}{\pgfqpoint{4.650000in}{3.020000in}}%
\pgfusepath{clip}%
\pgfsetbuttcap%
\pgfsetmiterjoin%
\definecolor{currentfill}{rgb}{1.000000,0.000000,0.000000}%
\pgfsetfillcolor{currentfill}%
\pgfsetlinewidth{0.000000pt}%
\definecolor{currentstroke}{rgb}{0.000000,0.000000,0.000000}%
\pgfsetstrokecolor{currentstroke}%
\pgfsetstrokeopacity{0.000000}%
\pgfsetdash{}{0pt}%
\pgfpathmoveto{\pgfqpoint{1.357670in}{0.500000in}}%
\pgfpathlineto{\pgfqpoint{1.390696in}{0.500000in}}%
\pgfpathlineto{\pgfqpoint{1.390696in}{0.512085in}}%
\pgfpathlineto{\pgfqpoint{1.357670in}{0.512085in}}%
\pgfpathlineto{\pgfqpoint{1.357670in}{0.500000in}}%
\pgfpathclose%
\pgfusepath{fill}%
\end{pgfscope}%
\begin{pgfscope}%
\pgfpathrectangle{\pgfqpoint{0.750000in}{0.500000in}}{\pgfqpoint{4.650000in}{3.020000in}}%
\pgfusepath{clip}%
\pgfsetbuttcap%
\pgfsetmiterjoin%
\definecolor{currentfill}{rgb}{1.000000,0.000000,0.000000}%
\pgfsetfillcolor{currentfill}%
\pgfsetlinewidth{0.000000pt}%
\definecolor{currentstroke}{rgb}{0.000000,0.000000,0.000000}%
\pgfsetstrokecolor{currentstroke}%
\pgfsetstrokeopacity{0.000000}%
\pgfsetdash{}{0pt}%
\pgfpathmoveto{\pgfqpoint{1.390696in}{0.500000in}}%
\pgfpathlineto{\pgfqpoint{1.423722in}{0.500000in}}%
\pgfpathlineto{\pgfqpoint{1.423722in}{0.611785in}}%
\pgfpathlineto{\pgfqpoint{1.390696in}{0.611785in}}%
\pgfpathlineto{\pgfqpoint{1.390696in}{0.500000in}}%
\pgfpathclose%
\pgfusepath{fill}%
\end{pgfscope}%
\begin{pgfscope}%
\pgfpathrectangle{\pgfqpoint{0.750000in}{0.500000in}}{\pgfqpoint{4.650000in}{3.020000in}}%
\pgfusepath{clip}%
\pgfsetbuttcap%
\pgfsetmiterjoin%
\definecolor{currentfill}{rgb}{1.000000,0.000000,0.000000}%
\pgfsetfillcolor{currentfill}%
\pgfsetlinewidth{0.000000pt}%
\definecolor{currentstroke}{rgb}{0.000000,0.000000,0.000000}%
\pgfsetstrokecolor{currentstroke}%
\pgfsetstrokeopacity{0.000000}%
\pgfsetdash{}{0pt}%
\pgfpathmoveto{\pgfqpoint{1.423722in}{0.500000in}}%
\pgfpathlineto{\pgfqpoint{1.456747in}{0.500000in}}%
\pgfpathlineto{\pgfqpoint{1.456747in}{0.506042in}}%
\pgfpathlineto{\pgfqpoint{1.423722in}{0.506042in}}%
\pgfpathlineto{\pgfqpoint{1.423722in}{0.500000in}}%
\pgfpathclose%
\pgfusepath{fill}%
\end{pgfscope}%
\begin{pgfscope}%
\pgfpathrectangle{\pgfqpoint{0.750000in}{0.500000in}}{\pgfqpoint{4.650000in}{3.020000in}}%
\pgfusepath{clip}%
\pgfsetbuttcap%
\pgfsetmiterjoin%
\definecolor{currentfill}{rgb}{1.000000,0.000000,0.000000}%
\pgfsetfillcolor{currentfill}%
\pgfsetlinewidth{0.000000pt}%
\definecolor{currentstroke}{rgb}{0.000000,0.000000,0.000000}%
\pgfsetstrokecolor{currentstroke}%
\pgfsetstrokeopacity{0.000000}%
\pgfsetdash{}{0pt}%
\pgfpathmoveto{\pgfqpoint{1.456747in}{0.500000in}}%
\pgfpathlineto{\pgfqpoint{1.489773in}{0.500000in}}%
\pgfpathlineto{\pgfqpoint{1.489773in}{0.503021in}}%
\pgfpathlineto{\pgfqpoint{1.456747in}{0.503021in}}%
\pgfpathlineto{\pgfqpoint{1.456747in}{0.500000in}}%
\pgfpathclose%
\pgfusepath{fill}%
\end{pgfscope}%
\begin{pgfscope}%
\pgfpathrectangle{\pgfqpoint{0.750000in}{0.500000in}}{\pgfqpoint{4.650000in}{3.020000in}}%
\pgfusepath{clip}%
\pgfsetbuttcap%
\pgfsetmiterjoin%
\definecolor{currentfill}{rgb}{1.000000,0.000000,0.000000}%
\pgfsetfillcolor{currentfill}%
\pgfsetlinewidth{0.000000pt}%
\definecolor{currentstroke}{rgb}{0.000000,0.000000,0.000000}%
\pgfsetstrokecolor{currentstroke}%
\pgfsetstrokeopacity{0.000000}%
\pgfsetdash{}{0pt}%
\pgfpathmoveto{\pgfqpoint{1.489773in}{0.500000in}}%
\pgfpathlineto{\pgfqpoint{1.522798in}{0.500000in}}%
\pgfpathlineto{\pgfqpoint{1.522798in}{0.759824in}}%
\pgfpathlineto{\pgfqpoint{1.489773in}{0.759824in}}%
\pgfpathlineto{\pgfqpoint{1.489773in}{0.500000in}}%
\pgfpathclose%
\pgfusepath{fill}%
\end{pgfscope}%
\begin{pgfscope}%
\pgfpathrectangle{\pgfqpoint{0.750000in}{0.500000in}}{\pgfqpoint{4.650000in}{3.020000in}}%
\pgfusepath{clip}%
\pgfsetbuttcap%
\pgfsetmiterjoin%
\definecolor{currentfill}{rgb}{1.000000,0.000000,0.000000}%
\pgfsetfillcolor{currentfill}%
\pgfsetlinewidth{0.000000pt}%
\definecolor{currentstroke}{rgb}{0.000000,0.000000,0.000000}%
\pgfsetstrokecolor{currentstroke}%
\pgfsetstrokeopacity{0.000000}%
\pgfsetdash{}{0pt}%
\pgfpathmoveto{\pgfqpoint{1.522798in}{0.500000in}}%
\pgfpathlineto{\pgfqpoint{1.555824in}{0.500000in}}%
\pgfpathlineto{\pgfqpoint{1.555824in}{0.527191in}}%
\pgfpathlineto{\pgfqpoint{1.522798in}{0.527191in}}%
\pgfpathlineto{\pgfqpoint{1.522798in}{0.500000in}}%
\pgfpathclose%
\pgfusepath{fill}%
\end{pgfscope}%
\begin{pgfscope}%
\pgfpathrectangle{\pgfqpoint{0.750000in}{0.500000in}}{\pgfqpoint{4.650000in}{3.020000in}}%
\pgfusepath{clip}%
\pgfsetbuttcap%
\pgfsetmiterjoin%
\definecolor{currentfill}{rgb}{1.000000,0.000000,0.000000}%
\pgfsetfillcolor{currentfill}%
\pgfsetlinewidth{0.000000pt}%
\definecolor{currentstroke}{rgb}{0.000000,0.000000,0.000000}%
\pgfsetstrokecolor{currentstroke}%
\pgfsetstrokeopacity{0.000000}%
\pgfsetdash{}{0pt}%
\pgfpathmoveto{\pgfqpoint{1.555824in}{0.500000in}}%
\pgfpathlineto{\pgfqpoint{1.588849in}{0.500000in}}%
\pgfpathlineto{\pgfqpoint{1.588849in}{0.527191in}}%
\pgfpathlineto{\pgfqpoint{1.555824in}{0.527191in}}%
\pgfpathlineto{\pgfqpoint{1.555824in}{0.500000in}}%
\pgfpathclose%
\pgfusepath{fill}%
\end{pgfscope}%
\begin{pgfscope}%
\pgfpathrectangle{\pgfqpoint{0.750000in}{0.500000in}}{\pgfqpoint{4.650000in}{3.020000in}}%
\pgfusepath{clip}%
\pgfsetbuttcap%
\pgfsetmiterjoin%
\definecolor{currentfill}{rgb}{1.000000,0.000000,0.000000}%
\pgfsetfillcolor{currentfill}%
\pgfsetlinewidth{0.000000pt}%
\definecolor{currentstroke}{rgb}{0.000000,0.000000,0.000000}%
\pgfsetstrokecolor{currentstroke}%
\pgfsetstrokeopacity{0.000000}%
\pgfsetdash{}{0pt}%
\pgfpathmoveto{\pgfqpoint{1.588849in}{0.500000in}}%
\pgfpathlineto{\pgfqpoint{1.621875in}{0.500000in}}%
\pgfpathlineto{\pgfqpoint{1.621875in}{0.880672in}}%
\pgfpathlineto{\pgfqpoint{1.588849in}{0.880672in}}%
\pgfpathlineto{\pgfqpoint{1.588849in}{0.500000in}}%
\pgfpathclose%
\pgfusepath{fill}%
\end{pgfscope}%
\begin{pgfscope}%
\pgfpathrectangle{\pgfqpoint{0.750000in}{0.500000in}}{\pgfqpoint{4.650000in}{3.020000in}}%
\pgfusepath{clip}%
\pgfsetbuttcap%
\pgfsetmiterjoin%
\definecolor{currentfill}{rgb}{1.000000,0.000000,0.000000}%
\pgfsetfillcolor{currentfill}%
\pgfsetlinewidth{0.000000pt}%
\definecolor{currentstroke}{rgb}{0.000000,0.000000,0.000000}%
\pgfsetstrokecolor{currentstroke}%
\pgfsetstrokeopacity{0.000000}%
\pgfsetdash{}{0pt}%
\pgfpathmoveto{\pgfqpoint{1.621875in}{0.500000in}}%
\pgfpathlineto{\pgfqpoint{1.654901in}{0.500000in}}%
\pgfpathlineto{\pgfqpoint{1.654901in}{0.512085in}}%
\pgfpathlineto{\pgfqpoint{1.621875in}{0.512085in}}%
\pgfpathlineto{\pgfqpoint{1.621875in}{0.500000in}}%
\pgfpathclose%
\pgfusepath{fill}%
\end{pgfscope}%
\begin{pgfscope}%
\pgfpathrectangle{\pgfqpoint{0.750000in}{0.500000in}}{\pgfqpoint{4.650000in}{3.020000in}}%
\pgfusepath{clip}%
\pgfsetbuttcap%
\pgfsetmiterjoin%
\definecolor{currentfill}{rgb}{1.000000,0.000000,0.000000}%
\pgfsetfillcolor{currentfill}%
\pgfsetlinewidth{0.000000pt}%
\definecolor{currentstroke}{rgb}{0.000000,0.000000,0.000000}%
\pgfsetstrokecolor{currentstroke}%
\pgfsetstrokeopacity{0.000000}%
\pgfsetdash{}{0pt}%
\pgfpathmoveto{\pgfqpoint{1.654901in}{0.500000in}}%
\pgfpathlineto{\pgfqpoint{1.687926in}{0.500000in}}%
\pgfpathlineto{\pgfqpoint{1.687926in}{1.071008in}}%
\pgfpathlineto{\pgfqpoint{1.654901in}{1.071008in}}%
\pgfpathlineto{\pgfqpoint{1.654901in}{0.500000in}}%
\pgfpathclose%
\pgfusepath{fill}%
\end{pgfscope}%
\begin{pgfscope}%
\pgfpathrectangle{\pgfqpoint{0.750000in}{0.500000in}}{\pgfqpoint{4.650000in}{3.020000in}}%
\pgfusepath{clip}%
\pgfsetbuttcap%
\pgfsetmiterjoin%
\definecolor{currentfill}{rgb}{1.000000,0.000000,0.000000}%
\pgfsetfillcolor{currentfill}%
\pgfsetlinewidth{0.000000pt}%
\definecolor{currentstroke}{rgb}{0.000000,0.000000,0.000000}%
\pgfsetstrokecolor{currentstroke}%
\pgfsetstrokeopacity{0.000000}%
\pgfsetdash{}{0pt}%
\pgfpathmoveto{\pgfqpoint{1.687926in}{0.500000in}}%
\pgfpathlineto{\pgfqpoint{1.720952in}{0.500000in}}%
\pgfpathlineto{\pgfqpoint{1.720952in}{0.521148in}}%
\pgfpathlineto{\pgfqpoint{1.687926in}{0.521148in}}%
\pgfpathlineto{\pgfqpoint{1.687926in}{0.500000in}}%
\pgfpathclose%
\pgfusepath{fill}%
\end{pgfscope}%
\begin{pgfscope}%
\pgfpathrectangle{\pgfqpoint{0.750000in}{0.500000in}}{\pgfqpoint{4.650000in}{3.020000in}}%
\pgfusepath{clip}%
\pgfsetbuttcap%
\pgfsetmiterjoin%
\definecolor{currentfill}{rgb}{1.000000,0.000000,0.000000}%
\pgfsetfillcolor{currentfill}%
\pgfsetlinewidth{0.000000pt}%
\definecolor{currentstroke}{rgb}{0.000000,0.000000,0.000000}%
\pgfsetstrokecolor{currentstroke}%
\pgfsetstrokeopacity{0.000000}%
\pgfsetdash{}{0pt}%
\pgfpathmoveto{\pgfqpoint{1.720952in}{0.500000in}}%
\pgfpathlineto{\pgfqpoint{1.753977in}{0.500000in}}%
\pgfpathlineto{\pgfqpoint{1.753977in}{0.524170in}}%
\pgfpathlineto{\pgfqpoint{1.720952in}{0.524170in}}%
\pgfpathlineto{\pgfqpoint{1.720952in}{0.500000in}}%
\pgfpathclose%
\pgfusepath{fill}%
\end{pgfscope}%
\begin{pgfscope}%
\pgfpathrectangle{\pgfqpoint{0.750000in}{0.500000in}}{\pgfqpoint{4.650000in}{3.020000in}}%
\pgfusepath{clip}%
\pgfsetbuttcap%
\pgfsetmiterjoin%
\definecolor{currentfill}{rgb}{1.000000,0.000000,0.000000}%
\pgfsetfillcolor{currentfill}%
\pgfsetlinewidth{0.000000pt}%
\definecolor{currentstroke}{rgb}{0.000000,0.000000,0.000000}%
\pgfsetstrokecolor{currentstroke}%
\pgfsetstrokeopacity{0.000000}%
\pgfsetdash{}{0pt}%
\pgfpathmoveto{\pgfqpoint{1.753977in}{0.500000in}}%
\pgfpathlineto{\pgfqpoint{1.787003in}{0.500000in}}%
\pgfpathlineto{\pgfqpoint{1.787003in}{1.436575in}}%
\pgfpathlineto{\pgfqpoint{1.753977in}{1.436575in}}%
\pgfpathlineto{\pgfqpoint{1.753977in}{0.500000in}}%
\pgfpathclose%
\pgfusepath{fill}%
\end{pgfscope}%
\begin{pgfscope}%
\pgfpathrectangle{\pgfqpoint{0.750000in}{0.500000in}}{\pgfqpoint{4.650000in}{3.020000in}}%
\pgfusepath{clip}%
\pgfsetbuttcap%
\pgfsetmiterjoin%
\definecolor{currentfill}{rgb}{1.000000,0.000000,0.000000}%
\pgfsetfillcolor{currentfill}%
\pgfsetlinewidth{0.000000pt}%
\definecolor{currentstroke}{rgb}{0.000000,0.000000,0.000000}%
\pgfsetstrokecolor{currentstroke}%
\pgfsetstrokeopacity{0.000000}%
\pgfsetdash{}{0pt}%
\pgfpathmoveto{\pgfqpoint{1.787003in}{0.500000in}}%
\pgfpathlineto{\pgfqpoint{1.820028in}{0.500000in}}%
\pgfpathlineto{\pgfqpoint{1.820028in}{0.530212in}}%
\pgfpathlineto{\pgfqpoint{1.787003in}{0.530212in}}%
\pgfpathlineto{\pgfqpoint{1.787003in}{0.500000in}}%
\pgfpathclose%
\pgfusepath{fill}%
\end{pgfscope}%
\begin{pgfscope}%
\pgfpathrectangle{\pgfqpoint{0.750000in}{0.500000in}}{\pgfqpoint{4.650000in}{3.020000in}}%
\pgfusepath{clip}%
\pgfsetbuttcap%
\pgfsetmiterjoin%
\definecolor{currentfill}{rgb}{1.000000,0.000000,0.000000}%
\pgfsetfillcolor{currentfill}%
\pgfsetlinewidth{0.000000pt}%
\definecolor{currentstroke}{rgb}{0.000000,0.000000,0.000000}%
\pgfsetstrokecolor{currentstroke}%
\pgfsetstrokeopacity{0.000000}%
\pgfsetdash{}{0pt}%
\pgfpathmoveto{\pgfqpoint{1.820028in}{0.500000in}}%
\pgfpathlineto{\pgfqpoint{1.853054in}{0.500000in}}%
\pgfpathlineto{\pgfqpoint{1.853054in}{0.533233in}}%
\pgfpathlineto{\pgfqpoint{1.820028in}{0.533233in}}%
\pgfpathlineto{\pgfqpoint{1.820028in}{0.500000in}}%
\pgfpathclose%
\pgfusepath{fill}%
\end{pgfscope}%
\begin{pgfscope}%
\pgfpathrectangle{\pgfqpoint{0.750000in}{0.500000in}}{\pgfqpoint{4.650000in}{3.020000in}}%
\pgfusepath{clip}%
\pgfsetbuttcap%
\pgfsetmiterjoin%
\definecolor{currentfill}{rgb}{1.000000,0.000000,0.000000}%
\pgfsetfillcolor{currentfill}%
\pgfsetlinewidth{0.000000pt}%
\definecolor{currentstroke}{rgb}{0.000000,0.000000,0.000000}%
\pgfsetstrokecolor{currentstroke}%
\pgfsetstrokeopacity{0.000000}%
\pgfsetdash{}{0pt}%
\pgfpathmoveto{\pgfqpoint{1.853054in}{0.500000in}}%
\pgfpathlineto{\pgfqpoint{1.886080in}{0.500000in}}%
\pgfpathlineto{\pgfqpoint{1.886080in}{1.880692in}}%
\pgfpathlineto{\pgfqpoint{1.853054in}{1.880692in}}%
\pgfpathlineto{\pgfqpoint{1.853054in}{0.500000in}}%
\pgfpathclose%
\pgfusepath{fill}%
\end{pgfscope}%
\begin{pgfscope}%
\pgfpathrectangle{\pgfqpoint{0.750000in}{0.500000in}}{\pgfqpoint{4.650000in}{3.020000in}}%
\pgfusepath{clip}%
\pgfsetbuttcap%
\pgfsetmiterjoin%
\definecolor{currentfill}{rgb}{1.000000,0.000000,0.000000}%
\pgfsetfillcolor{currentfill}%
\pgfsetlinewidth{0.000000pt}%
\definecolor{currentstroke}{rgb}{0.000000,0.000000,0.000000}%
\pgfsetstrokecolor{currentstroke}%
\pgfsetstrokeopacity{0.000000}%
\pgfsetdash{}{0pt}%
\pgfpathmoveto{\pgfqpoint{1.886080in}{0.500000in}}%
\pgfpathlineto{\pgfqpoint{1.919105in}{0.500000in}}%
\pgfpathlineto{\pgfqpoint{1.919105in}{0.530212in}}%
\pgfpathlineto{\pgfqpoint{1.886080in}{0.530212in}}%
\pgfpathlineto{\pgfqpoint{1.886080in}{0.500000in}}%
\pgfpathclose%
\pgfusepath{fill}%
\end{pgfscope}%
\begin{pgfscope}%
\pgfpathrectangle{\pgfqpoint{0.750000in}{0.500000in}}{\pgfqpoint{4.650000in}{3.020000in}}%
\pgfusepath{clip}%
\pgfsetbuttcap%
\pgfsetmiterjoin%
\definecolor{currentfill}{rgb}{1.000000,0.000000,0.000000}%
\pgfsetfillcolor{currentfill}%
\pgfsetlinewidth{0.000000pt}%
\definecolor{currentstroke}{rgb}{0.000000,0.000000,0.000000}%
\pgfsetstrokecolor{currentstroke}%
\pgfsetstrokeopacity{0.000000}%
\pgfsetdash{}{0pt}%
\pgfpathmoveto{\pgfqpoint{1.919105in}{0.500000in}}%
\pgfpathlineto{\pgfqpoint{1.952131in}{0.500000in}}%
\pgfpathlineto{\pgfqpoint{1.952131in}{2.119368in}}%
\pgfpathlineto{\pgfqpoint{1.919105in}{2.119368in}}%
\pgfpathlineto{\pgfqpoint{1.919105in}{0.500000in}}%
\pgfpathclose%
\pgfusepath{fill}%
\end{pgfscope}%
\begin{pgfscope}%
\pgfpathrectangle{\pgfqpoint{0.750000in}{0.500000in}}{\pgfqpoint{4.650000in}{3.020000in}}%
\pgfusepath{clip}%
\pgfsetbuttcap%
\pgfsetmiterjoin%
\definecolor{currentfill}{rgb}{1.000000,0.000000,0.000000}%
\pgfsetfillcolor{currentfill}%
\pgfsetlinewidth{0.000000pt}%
\definecolor{currentstroke}{rgb}{0.000000,0.000000,0.000000}%
\pgfsetstrokecolor{currentstroke}%
\pgfsetstrokeopacity{0.000000}%
\pgfsetdash{}{0pt}%
\pgfpathmoveto{\pgfqpoint{1.952131in}{0.500000in}}%
\pgfpathlineto{\pgfqpoint{1.985156in}{0.500000in}}%
\pgfpathlineto{\pgfqpoint{1.985156in}{0.690336in}}%
\pgfpathlineto{\pgfqpoint{1.952131in}{0.690336in}}%
\pgfpathlineto{\pgfqpoint{1.952131in}{0.500000in}}%
\pgfpathclose%
\pgfusepath{fill}%
\end{pgfscope}%
\begin{pgfscope}%
\pgfpathrectangle{\pgfqpoint{0.750000in}{0.500000in}}{\pgfqpoint{4.650000in}{3.020000in}}%
\pgfusepath{clip}%
\pgfsetbuttcap%
\pgfsetmiterjoin%
\definecolor{currentfill}{rgb}{1.000000,0.000000,0.000000}%
\pgfsetfillcolor{currentfill}%
\pgfsetlinewidth{0.000000pt}%
\definecolor{currentstroke}{rgb}{0.000000,0.000000,0.000000}%
\pgfsetstrokecolor{currentstroke}%
\pgfsetstrokeopacity{0.000000}%
\pgfsetdash{}{0pt}%
\pgfpathmoveto{\pgfqpoint{1.985156in}{0.500000in}}%
\pgfpathlineto{\pgfqpoint{2.018182in}{0.500000in}}%
\pgfpathlineto{\pgfqpoint{2.018182in}{0.572509in}}%
\pgfpathlineto{\pgfqpoint{1.985156in}{0.572509in}}%
\pgfpathlineto{\pgfqpoint{1.985156in}{0.500000in}}%
\pgfpathclose%
\pgfusepath{fill}%
\end{pgfscope}%
\begin{pgfscope}%
\pgfpathrectangle{\pgfqpoint{0.750000in}{0.500000in}}{\pgfqpoint{4.650000in}{3.020000in}}%
\pgfusepath{clip}%
\pgfsetbuttcap%
\pgfsetmiterjoin%
\definecolor{currentfill}{rgb}{1.000000,0.000000,0.000000}%
\pgfsetfillcolor{currentfill}%
\pgfsetlinewidth{0.000000pt}%
\definecolor{currentstroke}{rgb}{0.000000,0.000000,0.000000}%
\pgfsetstrokecolor{currentstroke}%
\pgfsetstrokeopacity{0.000000}%
\pgfsetdash{}{0pt}%
\pgfpathmoveto{\pgfqpoint{2.018182in}{0.500000in}}%
\pgfpathlineto{\pgfqpoint{2.051207in}{0.500000in}}%
\pgfpathlineto{\pgfqpoint{2.051207in}{2.584634in}}%
\pgfpathlineto{\pgfqpoint{2.018182in}{2.584634in}}%
\pgfpathlineto{\pgfqpoint{2.018182in}{0.500000in}}%
\pgfpathclose%
\pgfusepath{fill}%
\end{pgfscope}%
\begin{pgfscope}%
\pgfpathrectangle{\pgfqpoint{0.750000in}{0.500000in}}{\pgfqpoint{4.650000in}{3.020000in}}%
\pgfusepath{clip}%
\pgfsetbuttcap%
\pgfsetmiterjoin%
\definecolor{currentfill}{rgb}{1.000000,0.000000,0.000000}%
\pgfsetfillcolor{currentfill}%
\pgfsetlinewidth{0.000000pt}%
\definecolor{currentstroke}{rgb}{0.000000,0.000000,0.000000}%
\pgfsetstrokecolor{currentstroke}%
\pgfsetstrokeopacity{0.000000}%
\pgfsetdash{}{0pt}%
\pgfpathmoveto{\pgfqpoint{2.051207in}{0.500000in}}%
\pgfpathlineto{\pgfqpoint{2.084233in}{0.500000in}}%
\pgfpathlineto{\pgfqpoint{2.084233in}{0.590636in}}%
\pgfpathlineto{\pgfqpoint{2.051207in}{0.590636in}}%
\pgfpathlineto{\pgfqpoint{2.051207in}{0.500000in}}%
\pgfpathclose%
\pgfusepath{fill}%
\end{pgfscope}%
\begin{pgfscope}%
\pgfpathrectangle{\pgfqpoint{0.750000in}{0.500000in}}{\pgfqpoint{4.650000in}{3.020000in}}%
\pgfusepath{clip}%
\pgfsetbuttcap%
\pgfsetmiterjoin%
\definecolor{currentfill}{rgb}{1.000000,0.000000,0.000000}%
\pgfsetfillcolor{currentfill}%
\pgfsetlinewidth{0.000000pt}%
\definecolor{currentstroke}{rgb}{0.000000,0.000000,0.000000}%
\pgfsetstrokecolor{currentstroke}%
\pgfsetstrokeopacity{0.000000}%
\pgfsetdash{}{0pt}%
\pgfpathmoveto{\pgfqpoint{2.084233in}{0.500000in}}%
\pgfpathlineto{\pgfqpoint{2.117259in}{0.500000in}}%
\pgfpathlineto{\pgfqpoint{2.117259in}{0.596679in}}%
\pgfpathlineto{\pgfqpoint{2.084233in}{0.596679in}}%
\pgfpathlineto{\pgfqpoint{2.084233in}{0.500000in}}%
\pgfpathclose%
\pgfusepath{fill}%
\end{pgfscope}%
\begin{pgfscope}%
\pgfpathrectangle{\pgfqpoint{0.750000in}{0.500000in}}{\pgfqpoint{4.650000in}{3.020000in}}%
\pgfusepath{clip}%
\pgfsetbuttcap%
\pgfsetmiterjoin%
\definecolor{currentfill}{rgb}{1.000000,0.000000,0.000000}%
\pgfsetfillcolor{currentfill}%
\pgfsetlinewidth{0.000000pt}%
\definecolor{currentstroke}{rgb}{0.000000,0.000000,0.000000}%
\pgfsetstrokecolor{currentstroke}%
\pgfsetstrokeopacity{0.000000}%
\pgfsetdash{}{0pt}%
\pgfpathmoveto{\pgfqpoint{2.117259in}{0.500000in}}%
\pgfpathlineto{\pgfqpoint{2.150284in}{0.500000in}}%
\pgfpathlineto{\pgfqpoint{2.150284in}{3.037815in}}%
\pgfpathlineto{\pgfqpoint{2.117259in}{3.037815in}}%
\pgfpathlineto{\pgfqpoint{2.117259in}{0.500000in}}%
\pgfpathclose%
\pgfusepath{fill}%
\end{pgfscope}%
\begin{pgfscope}%
\pgfpathrectangle{\pgfqpoint{0.750000in}{0.500000in}}{\pgfqpoint{4.650000in}{3.020000in}}%
\pgfusepath{clip}%
\pgfsetbuttcap%
\pgfsetmiterjoin%
\definecolor{currentfill}{rgb}{1.000000,0.000000,0.000000}%
\pgfsetfillcolor{currentfill}%
\pgfsetlinewidth{0.000000pt}%
\definecolor{currentstroke}{rgb}{0.000000,0.000000,0.000000}%
\pgfsetstrokecolor{currentstroke}%
\pgfsetstrokeopacity{0.000000}%
\pgfsetdash{}{0pt}%
\pgfpathmoveto{\pgfqpoint{2.150284in}{0.500000in}}%
\pgfpathlineto{\pgfqpoint{2.183310in}{0.500000in}}%
\pgfpathlineto{\pgfqpoint{2.183310in}{0.575530in}}%
\pgfpathlineto{\pgfqpoint{2.150284in}{0.575530in}}%
\pgfpathlineto{\pgfqpoint{2.150284in}{0.500000in}}%
\pgfpathclose%
\pgfusepath{fill}%
\end{pgfscope}%
\begin{pgfscope}%
\pgfpathrectangle{\pgfqpoint{0.750000in}{0.500000in}}{\pgfqpoint{4.650000in}{3.020000in}}%
\pgfusepath{clip}%
\pgfsetbuttcap%
\pgfsetmiterjoin%
\definecolor{currentfill}{rgb}{1.000000,0.000000,0.000000}%
\pgfsetfillcolor{currentfill}%
\pgfsetlinewidth{0.000000pt}%
\definecolor{currentstroke}{rgb}{0.000000,0.000000,0.000000}%
\pgfsetstrokecolor{currentstroke}%
\pgfsetstrokeopacity{0.000000}%
\pgfsetdash{}{0pt}%
\pgfpathmoveto{\pgfqpoint{2.183310in}{0.500000in}}%
\pgfpathlineto{\pgfqpoint{2.216335in}{0.500000in}}%
\pgfpathlineto{\pgfqpoint{2.216335in}{0.608764in}}%
\pgfpathlineto{\pgfqpoint{2.183310in}{0.608764in}}%
\pgfpathlineto{\pgfqpoint{2.183310in}{0.500000in}}%
\pgfpathclose%
\pgfusepath{fill}%
\end{pgfscope}%
\begin{pgfscope}%
\pgfpathrectangle{\pgfqpoint{0.750000in}{0.500000in}}{\pgfqpoint{4.650000in}{3.020000in}}%
\pgfusepath{clip}%
\pgfsetbuttcap%
\pgfsetmiterjoin%
\definecolor{currentfill}{rgb}{1.000000,0.000000,0.000000}%
\pgfsetfillcolor{currentfill}%
\pgfsetlinewidth{0.000000pt}%
\definecolor{currentstroke}{rgb}{0.000000,0.000000,0.000000}%
\pgfsetstrokecolor{currentstroke}%
\pgfsetstrokeopacity{0.000000}%
\pgfsetdash{}{0pt}%
\pgfpathmoveto{\pgfqpoint{2.216335in}{0.500000in}}%
\pgfpathlineto{\pgfqpoint{2.249361in}{0.500000in}}%
\pgfpathlineto{\pgfqpoint{2.249361in}{3.267427in}}%
\pgfpathlineto{\pgfqpoint{2.216335in}{3.267427in}}%
\pgfpathlineto{\pgfqpoint{2.216335in}{0.500000in}}%
\pgfpathclose%
\pgfusepath{fill}%
\end{pgfscope}%
\begin{pgfscope}%
\pgfpathrectangle{\pgfqpoint{0.750000in}{0.500000in}}{\pgfqpoint{4.650000in}{3.020000in}}%
\pgfusepath{clip}%
\pgfsetbuttcap%
\pgfsetmiterjoin%
\definecolor{currentfill}{rgb}{1.000000,0.000000,0.000000}%
\pgfsetfillcolor{currentfill}%
\pgfsetlinewidth{0.000000pt}%
\definecolor{currentstroke}{rgb}{0.000000,0.000000,0.000000}%
\pgfsetstrokecolor{currentstroke}%
\pgfsetstrokeopacity{0.000000}%
\pgfsetdash{}{0pt}%
\pgfpathmoveto{\pgfqpoint{2.249361in}{0.500000in}}%
\pgfpathlineto{\pgfqpoint{2.282386in}{0.500000in}}%
\pgfpathlineto{\pgfqpoint{2.282386in}{0.587615in}}%
\pgfpathlineto{\pgfqpoint{2.249361in}{0.587615in}}%
\pgfpathlineto{\pgfqpoint{2.249361in}{0.500000in}}%
\pgfpathclose%
\pgfusepath{fill}%
\end{pgfscope}%
\begin{pgfscope}%
\pgfpathrectangle{\pgfqpoint{0.750000in}{0.500000in}}{\pgfqpoint{4.650000in}{3.020000in}}%
\pgfusepath{clip}%
\pgfsetbuttcap%
\pgfsetmiterjoin%
\definecolor{currentfill}{rgb}{1.000000,0.000000,0.000000}%
\pgfsetfillcolor{currentfill}%
\pgfsetlinewidth{0.000000pt}%
\definecolor{currentstroke}{rgb}{0.000000,0.000000,0.000000}%
\pgfsetstrokecolor{currentstroke}%
\pgfsetstrokeopacity{0.000000}%
\pgfsetdash{}{0pt}%
\pgfpathmoveto{\pgfqpoint{2.282386in}{0.500000in}}%
\pgfpathlineto{\pgfqpoint{2.315412in}{0.500000in}}%
\pgfpathlineto{\pgfqpoint{2.315412in}{3.376190in}}%
\pgfpathlineto{\pgfqpoint{2.282386in}{3.376190in}}%
\pgfpathlineto{\pgfqpoint{2.282386in}{0.500000in}}%
\pgfpathclose%
\pgfusepath{fill}%
\end{pgfscope}%
\begin{pgfscope}%
\pgfpathrectangle{\pgfqpoint{0.750000in}{0.500000in}}{\pgfqpoint{4.650000in}{3.020000in}}%
\pgfusepath{clip}%
\pgfsetbuttcap%
\pgfsetmiterjoin%
\definecolor{currentfill}{rgb}{1.000000,0.000000,0.000000}%
\pgfsetfillcolor{currentfill}%
\pgfsetlinewidth{0.000000pt}%
\definecolor{currentstroke}{rgb}{0.000000,0.000000,0.000000}%
\pgfsetstrokecolor{currentstroke}%
\pgfsetstrokeopacity{0.000000}%
\pgfsetdash{}{0pt}%
\pgfpathmoveto{\pgfqpoint{2.315412in}{0.500000in}}%
\pgfpathlineto{\pgfqpoint{2.348437in}{0.500000in}}%
\pgfpathlineto{\pgfqpoint{2.348437in}{0.645018in}}%
\pgfpathlineto{\pgfqpoint{2.315412in}{0.645018in}}%
\pgfpathlineto{\pgfqpoint{2.315412in}{0.500000in}}%
\pgfpathclose%
\pgfusepath{fill}%
\end{pgfscope}%
\begin{pgfscope}%
\pgfpathrectangle{\pgfqpoint{0.750000in}{0.500000in}}{\pgfqpoint{4.650000in}{3.020000in}}%
\pgfusepath{clip}%
\pgfsetbuttcap%
\pgfsetmiterjoin%
\definecolor{currentfill}{rgb}{1.000000,0.000000,0.000000}%
\pgfsetfillcolor{currentfill}%
\pgfsetlinewidth{0.000000pt}%
\definecolor{currentstroke}{rgb}{0.000000,0.000000,0.000000}%
\pgfsetstrokecolor{currentstroke}%
\pgfsetstrokeopacity{0.000000}%
\pgfsetdash{}{0pt}%
\pgfpathmoveto{\pgfqpoint{2.348437in}{0.500000in}}%
\pgfpathlineto{\pgfqpoint{2.381463in}{0.500000in}}%
\pgfpathlineto{\pgfqpoint{2.381463in}{0.590636in}}%
\pgfpathlineto{\pgfqpoint{2.348437in}{0.590636in}}%
\pgfpathlineto{\pgfqpoint{2.348437in}{0.500000in}}%
\pgfpathclose%
\pgfusepath{fill}%
\end{pgfscope}%
\begin{pgfscope}%
\pgfpathrectangle{\pgfqpoint{0.750000in}{0.500000in}}{\pgfqpoint{4.650000in}{3.020000in}}%
\pgfusepath{clip}%
\pgfsetbuttcap%
\pgfsetmiterjoin%
\definecolor{currentfill}{rgb}{1.000000,0.000000,0.000000}%
\pgfsetfillcolor{currentfill}%
\pgfsetlinewidth{0.000000pt}%
\definecolor{currentstroke}{rgb}{0.000000,0.000000,0.000000}%
\pgfsetstrokecolor{currentstroke}%
\pgfsetstrokeopacity{0.000000}%
\pgfsetdash{}{0pt}%
\pgfpathmoveto{\pgfqpoint{2.381463in}{0.500000in}}%
\pgfpathlineto{\pgfqpoint{2.414489in}{0.500000in}}%
\pgfpathlineto{\pgfqpoint{2.414489in}{3.273469in}}%
\pgfpathlineto{\pgfqpoint{2.381463in}{3.273469in}}%
\pgfpathlineto{\pgfqpoint{2.381463in}{0.500000in}}%
\pgfpathclose%
\pgfusepath{fill}%
\end{pgfscope}%
\begin{pgfscope}%
\pgfpathrectangle{\pgfqpoint{0.750000in}{0.500000in}}{\pgfqpoint{4.650000in}{3.020000in}}%
\pgfusepath{clip}%
\pgfsetbuttcap%
\pgfsetmiterjoin%
\definecolor{currentfill}{rgb}{1.000000,0.000000,0.000000}%
\pgfsetfillcolor{currentfill}%
\pgfsetlinewidth{0.000000pt}%
\definecolor{currentstroke}{rgb}{0.000000,0.000000,0.000000}%
\pgfsetstrokecolor{currentstroke}%
\pgfsetstrokeopacity{0.000000}%
\pgfsetdash{}{0pt}%
\pgfpathmoveto{\pgfqpoint{2.414489in}{0.500000in}}%
\pgfpathlineto{\pgfqpoint{2.447514in}{0.500000in}}%
\pgfpathlineto{\pgfqpoint{2.447514in}{0.593657in}}%
\pgfpathlineto{\pgfqpoint{2.414489in}{0.593657in}}%
\pgfpathlineto{\pgfqpoint{2.414489in}{0.500000in}}%
\pgfpathclose%
\pgfusepath{fill}%
\end{pgfscope}%
\begin{pgfscope}%
\pgfpathrectangle{\pgfqpoint{0.750000in}{0.500000in}}{\pgfqpoint{4.650000in}{3.020000in}}%
\pgfusepath{clip}%
\pgfsetbuttcap%
\pgfsetmiterjoin%
\definecolor{currentfill}{rgb}{1.000000,0.000000,0.000000}%
\pgfsetfillcolor{currentfill}%
\pgfsetlinewidth{0.000000pt}%
\definecolor{currentstroke}{rgb}{0.000000,0.000000,0.000000}%
\pgfsetstrokecolor{currentstroke}%
\pgfsetstrokeopacity{0.000000}%
\pgfsetdash{}{0pt}%
\pgfpathmoveto{\pgfqpoint{2.447514in}{0.500000in}}%
\pgfpathlineto{\pgfqpoint{2.480540in}{0.500000in}}%
\pgfpathlineto{\pgfqpoint{2.480540in}{0.593657in}}%
\pgfpathlineto{\pgfqpoint{2.447514in}{0.593657in}}%
\pgfpathlineto{\pgfqpoint{2.447514in}{0.500000in}}%
\pgfpathclose%
\pgfusepath{fill}%
\end{pgfscope}%
\begin{pgfscope}%
\pgfpathrectangle{\pgfqpoint{0.750000in}{0.500000in}}{\pgfqpoint{4.650000in}{3.020000in}}%
\pgfusepath{clip}%
\pgfsetbuttcap%
\pgfsetmiterjoin%
\definecolor{currentfill}{rgb}{1.000000,0.000000,0.000000}%
\pgfsetfillcolor{currentfill}%
\pgfsetlinewidth{0.000000pt}%
\definecolor{currentstroke}{rgb}{0.000000,0.000000,0.000000}%
\pgfsetstrokecolor{currentstroke}%
\pgfsetstrokeopacity{0.000000}%
\pgfsetdash{}{0pt}%
\pgfpathmoveto{\pgfqpoint{2.480540in}{0.500000in}}%
\pgfpathlineto{\pgfqpoint{2.513565in}{0.500000in}}%
\pgfpathlineto{\pgfqpoint{2.513565in}{3.016667in}}%
\pgfpathlineto{\pgfqpoint{2.480540in}{3.016667in}}%
\pgfpathlineto{\pgfqpoint{2.480540in}{0.500000in}}%
\pgfpathclose%
\pgfusepath{fill}%
\end{pgfscope}%
\begin{pgfscope}%
\pgfpathrectangle{\pgfqpoint{0.750000in}{0.500000in}}{\pgfqpoint{4.650000in}{3.020000in}}%
\pgfusepath{clip}%
\pgfsetbuttcap%
\pgfsetmiterjoin%
\definecolor{currentfill}{rgb}{1.000000,0.000000,0.000000}%
\pgfsetfillcolor{currentfill}%
\pgfsetlinewidth{0.000000pt}%
\definecolor{currentstroke}{rgb}{0.000000,0.000000,0.000000}%
\pgfsetstrokecolor{currentstroke}%
\pgfsetstrokeopacity{0.000000}%
\pgfsetdash{}{0pt}%
\pgfpathmoveto{\pgfqpoint{2.513565in}{0.500000in}}%
\pgfpathlineto{\pgfqpoint{2.546591in}{0.500000in}}%
\pgfpathlineto{\pgfqpoint{2.546591in}{0.563445in}}%
\pgfpathlineto{\pgfqpoint{2.513565in}{0.563445in}}%
\pgfpathlineto{\pgfqpoint{2.513565in}{0.500000in}}%
\pgfpathclose%
\pgfusepath{fill}%
\end{pgfscope}%
\begin{pgfscope}%
\pgfpathrectangle{\pgfqpoint{0.750000in}{0.500000in}}{\pgfqpoint{4.650000in}{3.020000in}}%
\pgfusepath{clip}%
\pgfsetbuttcap%
\pgfsetmiterjoin%
\definecolor{currentfill}{rgb}{1.000000,0.000000,0.000000}%
\pgfsetfillcolor{currentfill}%
\pgfsetlinewidth{0.000000pt}%
\definecolor{currentstroke}{rgb}{0.000000,0.000000,0.000000}%
\pgfsetstrokecolor{currentstroke}%
\pgfsetstrokeopacity{0.000000}%
\pgfsetdash{}{0pt}%
\pgfpathmoveto{\pgfqpoint{2.546591in}{0.500000in}}%
\pgfpathlineto{\pgfqpoint{2.579616in}{0.500000in}}%
\pgfpathlineto{\pgfqpoint{2.579616in}{0.635954in}}%
\pgfpathlineto{\pgfqpoint{2.546591in}{0.635954in}}%
\pgfpathlineto{\pgfqpoint{2.546591in}{0.500000in}}%
\pgfpathclose%
\pgfusepath{fill}%
\end{pgfscope}%
\begin{pgfscope}%
\pgfpathrectangle{\pgfqpoint{0.750000in}{0.500000in}}{\pgfqpoint{4.650000in}{3.020000in}}%
\pgfusepath{clip}%
\pgfsetbuttcap%
\pgfsetmiterjoin%
\definecolor{currentfill}{rgb}{1.000000,0.000000,0.000000}%
\pgfsetfillcolor{currentfill}%
\pgfsetlinewidth{0.000000pt}%
\definecolor{currentstroke}{rgb}{0.000000,0.000000,0.000000}%
\pgfsetstrokecolor{currentstroke}%
\pgfsetstrokeopacity{0.000000}%
\pgfsetdash{}{0pt}%
\pgfpathmoveto{\pgfqpoint{2.579616in}{0.500000in}}%
\pgfpathlineto{\pgfqpoint{2.612642in}{0.500000in}}%
\pgfpathlineto{\pgfqpoint{2.612642in}{2.509104in}}%
\pgfpathlineto{\pgfqpoint{2.579616in}{2.509104in}}%
\pgfpathlineto{\pgfqpoint{2.579616in}{0.500000in}}%
\pgfpathclose%
\pgfusepath{fill}%
\end{pgfscope}%
\begin{pgfscope}%
\pgfpathrectangle{\pgfqpoint{0.750000in}{0.500000in}}{\pgfqpoint{4.650000in}{3.020000in}}%
\pgfusepath{clip}%
\pgfsetbuttcap%
\pgfsetmiterjoin%
\definecolor{currentfill}{rgb}{1.000000,0.000000,0.000000}%
\pgfsetfillcolor{currentfill}%
\pgfsetlinewidth{0.000000pt}%
\definecolor{currentstroke}{rgb}{0.000000,0.000000,0.000000}%
\pgfsetstrokecolor{currentstroke}%
\pgfsetstrokeopacity{0.000000}%
\pgfsetdash{}{0pt}%
\pgfpathmoveto{\pgfqpoint{2.612642in}{0.500000in}}%
\pgfpathlineto{\pgfqpoint{2.645668in}{0.500000in}}%
\pgfpathlineto{\pgfqpoint{2.645668in}{0.569488in}}%
\pgfpathlineto{\pgfqpoint{2.612642in}{0.569488in}}%
\pgfpathlineto{\pgfqpoint{2.612642in}{0.500000in}}%
\pgfpathclose%
\pgfusepath{fill}%
\end{pgfscope}%
\begin{pgfscope}%
\pgfpathrectangle{\pgfqpoint{0.750000in}{0.500000in}}{\pgfqpoint{4.650000in}{3.020000in}}%
\pgfusepath{clip}%
\pgfsetbuttcap%
\pgfsetmiterjoin%
\definecolor{currentfill}{rgb}{1.000000,0.000000,0.000000}%
\pgfsetfillcolor{currentfill}%
\pgfsetlinewidth{0.000000pt}%
\definecolor{currentstroke}{rgb}{0.000000,0.000000,0.000000}%
\pgfsetstrokecolor{currentstroke}%
\pgfsetstrokeopacity{0.000000}%
\pgfsetdash{}{0pt}%
\pgfpathmoveto{\pgfqpoint{2.645668in}{0.500000in}}%
\pgfpathlineto{\pgfqpoint{2.678693in}{0.500000in}}%
\pgfpathlineto{\pgfqpoint{2.678693in}{2.203962in}}%
\pgfpathlineto{\pgfqpoint{2.645668in}{2.203962in}}%
\pgfpathlineto{\pgfqpoint{2.645668in}{0.500000in}}%
\pgfpathclose%
\pgfusepath{fill}%
\end{pgfscope}%
\begin{pgfscope}%
\pgfpathrectangle{\pgfqpoint{0.750000in}{0.500000in}}{\pgfqpoint{4.650000in}{3.020000in}}%
\pgfusepath{clip}%
\pgfsetbuttcap%
\pgfsetmiterjoin%
\definecolor{currentfill}{rgb}{1.000000,0.000000,0.000000}%
\pgfsetfillcolor{currentfill}%
\pgfsetlinewidth{0.000000pt}%
\definecolor{currentstroke}{rgb}{0.000000,0.000000,0.000000}%
\pgfsetstrokecolor{currentstroke}%
\pgfsetstrokeopacity{0.000000}%
\pgfsetdash{}{0pt}%
\pgfpathmoveto{\pgfqpoint{2.678693in}{0.500000in}}%
\pgfpathlineto{\pgfqpoint{2.711719in}{0.500000in}}%
\pgfpathlineto{\pgfqpoint{2.711719in}{0.566467in}}%
\pgfpathlineto{\pgfqpoint{2.678693in}{0.566467in}}%
\pgfpathlineto{\pgfqpoint{2.678693in}{0.500000in}}%
\pgfpathclose%
\pgfusepath{fill}%
\end{pgfscope}%
\begin{pgfscope}%
\pgfpathrectangle{\pgfqpoint{0.750000in}{0.500000in}}{\pgfqpoint{4.650000in}{3.020000in}}%
\pgfusepath{clip}%
\pgfsetbuttcap%
\pgfsetmiterjoin%
\definecolor{currentfill}{rgb}{1.000000,0.000000,0.000000}%
\pgfsetfillcolor{currentfill}%
\pgfsetlinewidth{0.000000pt}%
\definecolor{currentstroke}{rgb}{0.000000,0.000000,0.000000}%
\pgfsetstrokecolor{currentstroke}%
\pgfsetstrokeopacity{0.000000}%
\pgfsetdash{}{0pt}%
\pgfpathmoveto{\pgfqpoint{2.711719in}{0.500000in}}%
\pgfpathlineto{\pgfqpoint{2.744744in}{0.500000in}}%
\pgfpathlineto{\pgfqpoint{2.744744in}{0.563445in}}%
\pgfpathlineto{\pgfqpoint{2.711719in}{0.563445in}}%
\pgfpathlineto{\pgfqpoint{2.711719in}{0.500000in}}%
\pgfpathclose%
\pgfusepath{fill}%
\end{pgfscope}%
\begin{pgfscope}%
\pgfpathrectangle{\pgfqpoint{0.750000in}{0.500000in}}{\pgfqpoint{4.650000in}{3.020000in}}%
\pgfusepath{clip}%
\pgfsetbuttcap%
\pgfsetmiterjoin%
\definecolor{currentfill}{rgb}{1.000000,0.000000,0.000000}%
\pgfsetfillcolor{currentfill}%
\pgfsetlinewidth{0.000000pt}%
\definecolor{currentstroke}{rgb}{0.000000,0.000000,0.000000}%
\pgfsetstrokecolor{currentstroke}%
\pgfsetstrokeopacity{0.000000}%
\pgfsetdash{}{0pt}%
\pgfpathmoveto{\pgfqpoint{2.744744in}{0.500000in}}%
\pgfpathlineto{\pgfqpoint{2.777770in}{0.500000in}}%
\pgfpathlineto{\pgfqpoint{2.777770in}{1.711505in}}%
\pgfpathlineto{\pgfqpoint{2.744744in}{1.711505in}}%
\pgfpathlineto{\pgfqpoint{2.744744in}{0.500000in}}%
\pgfpathclose%
\pgfusepath{fill}%
\end{pgfscope}%
\begin{pgfscope}%
\pgfpathrectangle{\pgfqpoint{0.750000in}{0.500000in}}{\pgfqpoint{4.650000in}{3.020000in}}%
\pgfusepath{clip}%
\pgfsetbuttcap%
\pgfsetmiterjoin%
\definecolor{currentfill}{rgb}{1.000000,0.000000,0.000000}%
\pgfsetfillcolor{currentfill}%
\pgfsetlinewidth{0.000000pt}%
\definecolor{currentstroke}{rgb}{0.000000,0.000000,0.000000}%
\pgfsetstrokecolor{currentstroke}%
\pgfsetstrokeopacity{0.000000}%
\pgfsetdash{}{0pt}%
\pgfpathmoveto{\pgfqpoint{2.777770in}{0.500000in}}%
\pgfpathlineto{\pgfqpoint{2.810795in}{0.500000in}}%
\pgfpathlineto{\pgfqpoint{2.810795in}{0.518127in}}%
\pgfpathlineto{\pgfqpoint{2.777770in}{0.518127in}}%
\pgfpathlineto{\pgfqpoint{2.777770in}{0.500000in}}%
\pgfpathclose%
\pgfusepath{fill}%
\end{pgfscope}%
\begin{pgfscope}%
\pgfpathrectangle{\pgfqpoint{0.750000in}{0.500000in}}{\pgfqpoint{4.650000in}{3.020000in}}%
\pgfusepath{clip}%
\pgfsetbuttcap%
\pgfsetmiterjoin%
\definecolor{currentfill}{rgb}{1.000000,0.000000,0.000000}%
\pgfsetfillcolor{currentfill}%
\pgfsetlinewidth{0.000000pt}%
\definecolor{currentstroke}{rgb}{0.000000,0.000000,0.000000}%
\pgfsetstrokecolor{currentstroke}%
\pgfsetstrokeopacity{0.000000}%
\pgfsetdash{}{0pt}%
\pgfpathmoveto{\pgfqpoint{2.810795in}{0.500000in}}%
\pgfpathlineto{\pgfqpoint{2.843821in}{0.500000in}}%
\pgfpathlineto{\pgfqpoint{2.843821in}{0.545318in}}%
\pgfpathlineto{\pgfqpoint{2.810795in}{0.545318in}}%
\pgfpathlineto{\pgfqpoint{2.810795in}{0.500000in}}%
\pgfpathclose%
\pgfusepath{fill}%
\end{pgfscope}%
\begin{pgfscope}%
\pgfpathrectangle{\pgfqpoint{0.750000in}{0.500000in}}{\pgfqpoint{4.650000in}{3.020000in}}%
\pgfusepath{clip}%
\pgfsetbuttcap%
\pgfsetmiterjoin%
\definecolor{currentfill}{rgb}{1.000000,0.000000,0.000000}%
\pgfsetfillcolor{currentfill}%
\pgfsetlinewidth{0.000000pt}%
\definecolor{currentstroke}{rgb}{0.000000,0.000000,0.000000}%
\pgfsetstrokecolor{currentstroke}%
\pgfsetstrokeopacity{0.000000}%
\pgfsetdash{}{0pt}%
\pgfpathmoveto{\pgfqpoint{2.843821in}{0.500000in}}%
\pgfpathlineto{\pgfqpoint{2.876847in}{0.500000in}}%
\pgfpathlineto{\pgfqpoint{2.876847in}{1.469808in}}%
\pgfpathlineto{\pgfqpoint{2.843821in}{1.469808in}}%
\pgfpathlineto{\pgfqpoint{2.843821in}{0.500000in}}%
\pgfpathclose%
\pgfusepath{fill}%
\end{pgfscope}%
\begin{pgfscope}%
\pgfpathrectangle{\pgfqpoint{0.750000in}{0.500000in}}{\pgfqpoint{4.650000in}{3.020000in}}%
\pgfusepath{clip}%
\pgfsetbuttcap%
\pgfsetmiterjoin%
\definecolor{currentfill}{rgb}{1.000000,0.000000,0.000000}%
\pgfsetfillcolor{currentfill}%
\pgfsetlinewidth{0.000000pt}%
\definecolor{currentstroke}{rgb}{0.000000,0.000000,0.000000}%
\pgfsetstrokecolor{currentstroke}%
\pgfsetstrokeopacity{0.000000}%
\pgfsetdash{}{0pt}%
\pgfpathmoveto{\pgfqpoint{2.876847in}{0.500000in}}%
\pgfpathlineto{\pgfqpoint{2.909872in}{0.500000in}}%
\pgfpathlineto{\pgfqpoint{2.909872in}{0.530212in}}%
\pgfpathlineto{\pgfqpoint{2.876847in}{0.530212in}}%
\pgfpathlineto{\pgfqpoint{2.876847in}{0.500000in}}%
\pgfpathclose%
\pgfusepath{fill}%
\end{pgfscope}%
\begin{pgfscope}%
\pgfpathrectangle{\pgfqpoint{0.750000in}{0.500000in}}{\pgfqpoint{4.650000in}{3.020000in}}%
\pgfusepath{clip}%
\pgfsetbuttcap%
\pgfsetmiterjoin%
\definecolor{currentfill}{rgb}{1.000000,0.000000,0.000000}%
\pgfsetfillcolor{currentfill}%
\pgfsetlinewidth{0.000000pt}%
\definecolor{currentstroke}{rgb}{0.000000,0.000000,0.000000}%
\pgfsetstrokecolor{currentstroke}%
\pgfsetstrokeopacity{0.000000}%
\pgfsetdash{}{0pt}%
\pgfpathmoveto{\pgfqpoint{2.909872in}{0.500000in}}%
\pgfpathlineto{\pgfqpoint{2.942898in}{0.500000in}}%
\pgfpathlineto{\pgfqpoint{2.942898in}{1.110284in}}%
\pgfpathlineto{\pgfqpoint{2.909872in}{1.110284in}}%
\pgfpathlineto{\pgfqpoint{2.909872in}{0.500000in}}%
\pgfpathclose%
\pgfusepath{fill}%
\end{pgfscope}%
\begin{pgfscope}%
\pgfpathrectangle{\pgfqpoint{0.750000in}{0.500000in}}{\pgfqpoint{4.650000in}{3.020000in}}%
\pgfusepath{clip}%
\pgfsetbuttcap%
\pgfsetmiterjoin%
\definecolor{currentfill}{rgb}{1.000000,0.000000,0.000000}%
\pgfsetfillcolor{currentfill}%
\pgfsetlinewidth{0.000000pt}%
\definecolor{currentstroke}{rgb}{0.000000,0.000000,0.000000}%
\pgfsetstrokecolor{currentstroke}%
\pgfsetstrokeopacity{0.000000}%
\pgfsetdash{}{0pt}%
\pgfpathmoveto{\pgfqpoint{2.942898in}{0.500000in}}%
\pgfpathlineto{\pgfqpoint{2.975923in}{0.500000in}}%
\pgfpathlineto{\pgfqpoint{2.975923in}{0.539276in}}%
\pgfpathlineto{\pgfqpoint{2.942898in}{0.539276in}}%
\pgfpathlineto{\pgfqpoint{2.942898in}{0.500000in}}%
\pgfpathclose%
\pgfusepath{fill}%
\end{pgfscope}%
\begin{pgfscope}%
\pgfpathrectangle{\pgfqpoint{0.750000in}{0.500000in}}{\pgfqpoint{4.650000in}{3.020000in}}%
\pgfusepath{clip}%
\pgfsetbuttcap%
\pgfsetmiterjoin%
\definecolor{currentfill}{rgb}{1.000000,0.000000,0.000000}%
\pgfsetfillcolor{currentfill}%
\pgfsetlinewidth{0.000000pt}%
\definecolor{currentstroke}{rgb}{0.000000,0.000000,0.000000}%
\pgfsetstrokecolor{currentstroke}%
\pgfsetstrokeopacity{0.000000}%
\pgfsetdash{}{0pt}%
\pgfpathmoveto{\pgfqpoint{2.975923in}{0.500000in}}%
\pgfpathlineto{\pgfqpoint{3.008949in}{0.500000in}}%
\pgfpathlineto{\pgfqpoint{3.008949in}{0.518127in}}%
\pgfpathlineto{\pgfqpoint{2.975923in}{0.518127in}}%
\pgfpathlineto{\pgfqpoint{2.975923in}{0.500000in}}%
\pgfpathclose%
\pgfusepath{fill}%
\end{pgfscope}%
\begin{pgfscope}%
\pgfpathrectangle{\pgfqpoint{0.750000in}{0.500000in}}{\pgfqpoint{4.650000in}{3.020000in}}%
\pgfusepath{clip}%
\pgfsetbuttcap%
\pgfsetmiterjoin%
\definecolor{currentfill}{rgb}{1.000000,0.000000,0.000000}%
\pgfsetfillcolor{currentfill}%
\pgfsetlinewidth{0.000000pt}%
\definecolor{currentstroke}{rgb}{0.000000,0.000000,0.000000}%
\pgfsetstrokecolor{currentstroke}%
\pgfsetstrokeopacity{0.000000}%
\pgfsetdash{}{0pt}%
\pgfpathmoveto{\pgfqpoint{3.008949in}{0.500000in}}%
\pgfpathlineto{\pgfqpoint{3.041974in}{0.500000in}}%
\pgfpathlineto{\pgfqpoint{3.041974in}{0.856503in}}%
\pgfpathlineto{\pgfqpoint{3.008949in}{0.856503in}}%
\pgfpathlineto{\pgfqpoint{3.008949in}{0.500000in}}%
\pgfpathclose%
\pgfusepath{fill}%
\end{pgfscope}%
\begin{pgfscope}%
\pgfpathrectangle{\pgfqpoint{0.750000in}{0.500000in}}{\pgfqpoint{4.650000in}{3.020000in}}%
\pgfusepath{clip}%
\pgfsetbuttcap%
\pgfsetmiterjoin%
\definecolor{currentfill}{rgb}{1.000000,0.000000,0.000000}%
\pgfsetfillcolor{currentfill}%
\pgfsetlinewidth{0.000000pt}%
\definecolor{currentstroke}{rgb}{0.000000,0.000000,0.000000}%
\pgfsetstrokecolor{currentstroke}%
\pgfsetstrokeopacity{0.000000}%
\pgfsetdash{}{0pt}%
\pgfpathmoveto{\pgfqpoint{3.041974in}{0.500000in}}%
\pgfpathlineto{\pgfqpoint{3.075000in}{0.500000in}}%
\pgfpathlineto{\pgfqpoint{3.075000in}{0.503021in}}%
\pgfpathlineto{\pgfqpoint{3.041974in}{0.503021in}}%
\pgfpathlineto{\pgfqpoint{3.041974in}{0.500000in}}%
\pgfpathclose%
\pgfusepath{fill}%
\end{pgfscope}%
\begin{pgfscope}%
\pgfpathrectangle{\pgfqpoint{0.750000in}{0.500000in}}{\pgfqpoint{4.650000in}{3.020000in}}%
\pgfusepath{clip}%
\pgfsetbuttcap%
\pgfsetmiterjoin%
\definecolor{currentfill}{rgb}{1.000000,0.000000,0.000000}%
\pgfsetfillcolor{currentfill}%
\pgfsetlinewidth{0.000000pt}%
\definecolor{currentstroke}{rgb}{0.000000,0.000000,0.000000}%
\pgfsetstrokecolor{currentstroke}%
\pgfsetstrokeopacity{0.000000}%
\pgfsetdash{}{0pt}%
\pgfpathmoveto{\pgfqpoint{3.075000in}{0.500000in}}%
\pgfpathlineto{\pgfqpoint{3.108026in}{0.500000in}}%
\pgfpathlineto{\pgfqpoint{3.108026in}{0.506042in}}%
\pgfpathlineto{\pgfqpoint{3.075000in}{0.506042in}}%
\pgfpathlineto{\pgfqpoint{3.075000in}{0.500000in}}%
\pgfpathclose%
\pgfusepath{fill}%
\end{pgfscope}%
\begin{pgfscope}%
\pgfpathrectangle{\pgfqpoint{0.750000in}{0.500000in}}{\pgfqpoint{4.650000in}{3.020000in}}%
\pgfusepath{clip}%
\pgfsetbuttcap%
\pgfsetmiterjoin%
\definecolor{currentfill}{rgb}{1.000000,0.000000,0.000000}%
\pgfsetfillcolor{currentfill}%
\pgfsetlinewidth{0.000000pt}%
\definecolor{currentstroke}{rgb}{0.000000,0.000000,0.000000}%
\pgfsetstrokecolor{currentstroke}%
\pgfsetstrokeopacity{0.000000}%
\pgfsetdash{}{0pt}%
\pgfpathmoveto{\pgfqpoint{3.108026in}{0.500000in}}%
\pgfpathlineto{\pgfqpoint{3.141051in}{0.500000in}}%
\pgfpathlineto{\pgfqpoint{3.141051in}{0.723569in}}%
\pgfpathlineto{\pgfqpoint{3.108026in}{0.723569in}}%
\pgfpathlineto{\pgfqpoint{3.108026in}{0.500000in}}%
\pgfpathclose%
\pgfusepath{fill}%
\end{pgfscope}%
\begin{pgfscope}%
\pgfpathrectangle{\pgfqpoint{0.750000in}{0.500000in}}{\pgfqpoint{4.650000in}{3.020000in}}%
\pgfusepath{clip}%
\pgfsetbuttcap%
\pgfsetmiterjoin%
\definecolor{currentfill}{rgb}{1.000000,0.000000,0.000000}%
\pgfsetfillcolor{currentfill}%
\pgfsetlinewidth{0.000000pt}%
\definecolor{currentstroke}{rgb}{0.000000,0.000000,0.000000}%
\pgfsetstrokecolor{currentstroke}%
\pgfsetstrokeopacity{0.000000}%
\pgfsetdash{}{0pt}%
\pgfpathmoveto{\pgfqpoint{3.141051in}{0.500000in}}%
\pgfpathlineto{\pgfqpoint{3.174077in}{0.500000in}}%
\pgfpathlineto{\pgfqpoint{3.174077in}{0.509064in}}%
\pgfpathlineto{\pgfqpoint{3.141051in}{0.509064in}}%
\pgfpathlineto{\pgfqpoint{3.141051in}{0.500000in}}%
\pgfpathclose%
\pgfusepath{fill}%
\end{pgfscope}%
\begin{pgfscope}%
\pgfpathrectangle{\pgfqpoint{0.750000in}{0.500000in}}{\pgfqpoint{4.650000in}{3.020000in}}%
\pgfusepath{clip}%
\pgfsetbuttcap%
\pgfsetmiterjoin%
\definecolor{currentfill}{rgb}{1.000000,0.000000,0.000000}%
\pgfsetfillcolor{currentfill}%
\pgfsetlinewidth{0.000000pt}%
\definecolor{currentstroke}{rgb}{0.000000,0.000000,0.000000}%
\pgfsetstrokecolor{currentstroke}%
\pgfsetstrokeopacity{0.000000}%
\pgfsetdash{}{0pt}%
\pgfpathmoveto{\pgfqpoint{3.174077in}{0.500000in}}%
\pgfpathlineto{\pgfqpoint{3.207102in}{0.500000in}}%
\pgfpathlineto{\pgfqpoint{3.207102in}{0.500000in}}%
\pgfpathlineto{\pgfqpoint{3.174077in}{0.500000in}}%
\pgfpathlineto{\pgfqpoint{3.174077in}{0.500000in}}%
\pgfpathclose%
\pgfusepath{fill}%
\end{pgfscope}%
\begin{pgfscope}%
\pgfpathrectangle{\pgfqpoint{0.750000in}{0.500000in}}{\pgfqpoint{4.650000in}{3.020000in}}%
\pgfusepath{clip}%
\pgfsetbuttcap%
\pgfsetmiterjoin%
\definecolor{currentfill}{rgb}{1.000000,0.000000,0.000000}%
\pgfsetfillcolor{currentfill}%
\pgfsetlinewidth{0.000000pt}%
\definecolor{currentstroke}{rgb}{0.000000,0.000000,0.000000}%
\pgfsetstrokecolor{currentstroke}%
\pgfsetstrokeopacity{0.000000}%
\pgfsetdash{}{0pt}%
\pgfpathmoveto{\pgfqpoint{3.207102in}{0.500000in}}%
\pgfpathlineto{\pgfqpoint{3.240128in}{0.500000in}}%
\pgfpathlineto{\pgfqpoint{3.240128in}{0.605742in}}%
\pgfpathlineto{\pgfqpoint{3.207102in}{0.605742in}}%
\pgfpathlineto{\pgfqpoint{3.207102in}{0.500000in}}%
\pgfpathclose%
\pgfusepath{fill}%
\end{pgfscope}%
\begin{pgfscope}%
\pgfpathrectangle{\pgfqpoint{0.750000in}{0.500000in}}{\pgfqpoint{4.650000in}{3.020000in}}%
\pgfusepath{clip}%
\pgfsetbuttcap%
\pgfsetmiterjoin%
\definecolor{currentfill}{rgb}{1.000000,0.000000,0.000000}%
\pgfsetfillcolor{currentfill}%
\pgfsetlinewidth{0.000000pt}%
\definecolor{currentstroke}{rgb}{0.000000,0.000000,0.000000}%
\pgfsetstrokecolor{currentstroke}%
\pgfsetstrokeopacity{0.000000}%
\pgfsetdash{}{0pt}%
\pgfpathmoveto{\pgfqpoint{3.240128in}{0.500000in}}%
\pgfpathlineto{\pgfqpoint{3.273153in}{0.500000in}}%
\pgfpathlineto{\pgfqpoint{3.273153in}{0.500000in}}%
\pgfpathlineto{\pgfqpoint{3.240128in}{0.500000in}}%
\pgfpathlineto{\pgfqpoint{3.240128in}{0.500000in}}%
\pgfpathclose%
\pgfusepath{fill}%
\end{pgfscope}%
\begin{pgfscope}%
\pgfpathrectangle{\pgfqpoint{0.750000in}{0.500000in}}{\pgfqpoint{4.650000in}{3.020000in}}%
\pgfusepath{clip}%
\pgfsetbuttcap%
\pgfsetmiterjoin%
\definecolor{currentfill}{rgb}{1.000000,0.000000,0.000000}%
\pgfsetfillcolor{currentfill}%
\pgfsetlinewidth{0.000000pt}%
\definecolor{currentstroke}{rgb}{0.000000,0.000000,0.000000}%
\pgfsetstrokecolor{currentstroke}%
\pgfsetstrokeopacity{0.000000}%
\pgfsetdash{}{0pt}%
\pgfpathmoveto{\pgfqpoint{3.273153in}{0.500000in}}%
\pgfpathlineto{\pgfqpoint{3.306179in}{0.500000in}}%
\pgfpathlineto{\pgfqpoint{3.306179in}{0.557403in}}%
\pgfpathlineto{\pgfqpoint{3.273153in}{0.557403in}}%
\pgfpathlineto{\pgfqpoint{3.273153in}{0.500000in}}%
\pgfpathclose%
\pgfusepath{fill}%
\end{pgfscope}%
\begin{pgfscope}%
\pgfpathrectangle{\pgfqpoint{0.750000in}{0.500000in}}{\pgfqpoint{4.650000in}{3.020000in}}%
\pgfusepath{clip}%
\pgfsetbuttcap%
\pgfsetmiterjoin%
\definecolor{currentfill}{rgb}{1.000000,0.000000,0.000000}%
\pgfsetfillcolor{currentfill}%
\pgfsetlinewidth{0.000000pt}%
\definecolor{currentstroke}{rgb}{0.000000,0.000000,0.000000}%
\pgfsetstrokecolor{currentstroke}%
\pgfsetstrokeopacity{0.000000}%
\pgfsetdash{}{0pt}%
\pgfpathmoveto{\pgfqpoint{3.306179in}{0.500000in}}%
\pgfpathlineto{\pgfqpoint{3.339205in}{0.500000in}}%
\pgfpathlineto{\pgfqpoint{3.339205in}{0.500000in}}%
\pgfpathlineto{\pgfqpoint{3.306179in}{0.500000in}}%
\pgfpathlineto{\pgfqpoint{3.306179in}{0.500000in}}%
\pgfpathclose%
\pgfusepath{fill}%
\end{pgfscope}%
\begin{pgfscope}%
\pgfpathrectangle{\pgfqpoint{0.750000in}{0.500000in}}{\pgfqpoint{4.650000in}{3.020000in}}%
\pgfusepath{clip}%
\pgfsetbuttcap%
\pgfsetmiterjoin%
\definecolor{currentfill}{rgb}{1.000000,0.000000,0.000000}%
\pgfsetfillcolor{currentfill}%
\pgfsetlinewidth{0.000000pt}%
\definecolor{currentstroke}{rgb}{0.000000,0.000000,0.000000}%
\pgfsetstrokecolor{currentstroke}%
\pgfsetstrokeopacity{0.000000}%
\pgfsetdash{}{0pt}%
\pgfpathmoveto{\pgfqpoint{3.339205in}{0.500000in}}%
\pgfpathlineto{\pgfqpoint{3.372230in}{0.500000in}}%
\pgfpathlineto{\pgfqpoint{3.372230in}{0.503021in}}%
\pgfpathlineto{\pgfqpoint{3.339205in}{0.503021in}}%
\pgfpathlineto{\pgfqpoint{3.339205in}{0.500000in}}%
\pgfpathclose%
\pgfusepath{fill}%
\end{pgfscope}%
\begin{pgfscope}%
\pgfpathrectangle{\pgfqpoint{0.750000in}{0.500000in}}{\pgfqpoint{4.650000in}{3.020000in}}%
\pgfusepath{clip}%
\pgfsetbuttcap%
\pgfsetmiterjoin%
\definecolor{currentfill}{rgb}{1.000000,0.000000,0.000000}%
\pgfsetfillcolor{currentfill}%
\pgfsetlinewidth{0.000000pt}%
\definecolor{currentstroke}{rgb}{0.000000,0.000000,0.000000}%
\pgfsetstrokecolor{currentstroke}%
\pgfsetstrokeopacity{0.000000}%
\pgfsetdash{}{0pt}%
\pgfpathmoveto{\pgfqpoint{3.372230in}{0.500000in}}%
\pgfpathlineto{\pgfqpoint{3.405256in}{0.500000in}}%
\pgfpathlineto{\pgfqpoint{3.405256in}{0.512085in}}%
\pgfpathlineto{\pgfqpoint{3.372230in}{0.512085in}}%
\pgfpathlineto{\pgfqpoint{3.372230in}{0.500000in}}%
\pgfpathclose%
\pgfusepath{fill}%
\end{pgfscope}%
\begin{pgfscope}%
\pgfpathrectangle{\pgfqpoint{0.750000in}{0.500000in}}{\pgfqpoint{4.650000in}{3.020000in}}%
\pgfusepath{clip}%
\pgfsetbuttcap%
\pgfsetmiterjoin%
\definecolor{currentfill}{rgb}{1.000000,0.000000,0.000000}%
\pgfsetfillcolor{currentfill}%
\pgfsetlinewidth{0.000000pt}%
\definecolor{currentstroke}{rgb}{0.000000,0.000000,0.000000}%
\pgfsetstrokecolor{currentstroke}%
\pgfsetstrokeopacity{0.000000}%
\pgfsetdash{}{0pt}%
\pgfpathmoveto{\pgfqpoint{3.405256in}{0.500000in}}%
\pgfpathlineto{\pgfqpoint{3.438281in}{0.500000in}}%
\pgfpathlineto{\pgfqpoint{3.438281in}{0.500000in}}%
\pgfpathlineto{\pgfqpoint{3.405256in}{0.500000in}}%
\pgfpathlineto{\pgfqpoint{3.405256in}{0.500000in}}%
\pgfpathclose%
\pgfusepath{fill}%
\end{pgfscope}%
\begin{pgfscope}%
\pgfpathrectangle{\pgfqpoint{0.750000in}{0.500000in}}{\pgfqpoint{4.650000in}{3.020000in}}%
\pgfusepath{clip}%
\pgfsetbuttcap%
\pgfsetmiterjoin%
\definecolor{currentfill}{rgb}{1.000000,0.000000,0.000000}%
\pgfsetfillcolor{currentfill}%
\pgfsetlinewidth{0.000000pt}%
\definecolor{currentstroke}{rgb}{0.000000,0.000000,0.000000}%
\pgfsetstrokecolor{currentstroke}%
\pgfsetstrokeopacity{0.000000}%
\pgfsetdash{}{0pt}%
\pgfpathmoveto{\pgfqpoint{3.438281in}{0.500000in}}%
\pgfpathlineto{\pgfqpoint{3.471307in}{0.500000in}}%
\pgfpathlineto{\pgfqpoint{3.471307in}{0.500000in}}%
\pgfpathlineto{\pgfqpoint{3.438281in}{0.500000in}}%
\pgfpathlineto{\pgfqpoint{3.438281in}{0.500000in}}%
\pgfpathclose%
\pgfusepath{fill}%
\end{pgfscope}%
\begin{pgfscope}%
\pgfpathrectangle{\pgfqpoint{0.750000in}{0.500000in}}{\pgfqpoint{4.650000in}{3.020000in}}%
\pgfusepath{clip}%
\pgfsetbuttcap%
\pgfsetmiterjoin%
\definecolor{currentfill}{rgb}{1.000000,0.000000,0.000000}%
\pgfsetfillcolor{currentfill}%
\pgfsetlinewidth{0.000000pt}%
\definecolor{currentstroke}{rgb}{0.000000,0.000000,0.000000}%
\pgfsetstrokecolor{currentstroke}%
\pgfsetstrokeopacity{0.000000}%
\pgfsetdash{}{0pt}%
\pgfpathmoveto{\pgfqpoint{3.471307in}{0.500000in}}%
\pgfpathlineto{\pgfqpoint{3.504332in}{0.500000in}}%
\pgfpathlineto{\pgfqpoint{3.504332in}{0.509064in}}%
\pgfpathlineto{\pgfqpoint{3.471307in}{0.509064in}}%
\pgfpathlineto{\pgfqpoint{3.471307in}{0.500000in}}%
\pgfpathclose%
\pgfusepath{fill}%
\end{pgfscope}%
\begin{pgfscope}%
\pgfpathrectangle{\pgfqpoint{0.750000in}{0.500000in}}{\pgfqpoint{4.650000in}{3.020000in}}%
\pgfusepath{clip}%
\pgfsetbuttcap%
\pgfsetmiterjoin%
\definecolor{currentfill}{rgb}{1.000000,0.000000,0.000000}%
\pgfsetfillcolor{currentfill}%
\pgfsetlinewidth{0.000000pt}%
\definecolor{currentstroke}{rgb}{0.000000,0.000000,0.000000}%
\pgfsetstrokecolor{currentstroke}%
\pgfsetstrokeopacity{0.000000}%
\pgfsetdash{}{0pt}%
\pgfpathmoveto{\pgfqpoint{3.504332in}{0.500000in}}%
\pgfpathlineto{\pgfqpoint{3.537358in}{0.500000in}}%
\pgfpathlineto{\pgfqpoint{3.537358in}{0.500000in}}%
\pgfpathlineto{\pgfqpoint{3.504332in}{0.500000in}}%
\pgfpathlineto{\pgfqpoint{3.504332in}{0.500000in}}%
\pgfpathclose%
\pgfusepath{fill}%
\end{pgfscope}%
\begin{pgfscope}%
\pgfpathrectangle{\pgfqpoint{0.750000in}{0.500000in}}{\pgfqpoint{4.650000in}{3.020000in}}%
\pgfusepath{clip}%
\pgfsetbuttcap%
\pgfsetmiterjoin%
\definecolor{currentfill}{rgb}{1.000000,0.000000,0.000000}%
\pgfsetfillcolor{currentfill}%
\pgfsetlinewidth{0.000000pt}%
\definecolor{currentstroke}{rgb}{0.000000,0.000000,0.000000}%
\pgfsetstrokecolor{currentstroke}%
\pgfsetstrokeopacity{0.000000}%
\pgfsetdash{}{0pt}%
\pgfpathmoveto{\pgfqpoint{3.537358in}{0.500000in}}%
\pgfpathlineto{\pgfqpoint{3.570384in}{0.500000in}}%
\pgfpathlineto{\pgfqpoint{3.570384in}{0.500000in}}%
\pgfpathlineto{\pgfqpoint{3.537358in}{0.500000in}}%
\pgfpathlineto{\pgfqpoint{3.537358in}{0.500000in}}%
\pgfpathclose%
\pgfusepath{fill}%
\end{pgfscope}%
\begin{pgfscope}%
\pgfpathrectangle{\pgfqpoint{0.750000in}{0.500000in}}{\pgfqpoint{4.650000in}{3.020000in}}%
\pgfusepath{clip}%
\pgfsetbuttcap%
\pgfsetmiterjoin%
\definecolor{currentfill}{rgb}{1.000000,0.000000,0.000000}%
\pgfsetfillcolor{currentfill}%
\pgfsetlinewidth{0.000000pt}%
\definecolor{currentstroke}{rgb}{0.000000,0.000000,0.000000}%
\pgfsetstrokecolor{currentstroke}%
\pgfsetstrokeopacity{0.000000}%
\pgfsetdash{}{0pt}%
\pgfpathmoveto{\pgfqpoint{3.570384in}{0.500000in}}%
\pgfpathlineto{\pgfqpoint{3.603409in}{0.500000in}}%
\pgfpathlineto{\pgfqpoint{3.603409in}{0.503021in}}%
\pgfpathlineto{\pgfqpoint{3.570384in}{0.503021in}}%
\pgfpathlineto{\pgfqpoint{3.570384in}{0.500000in}}%
\pgfpathclose%
\pgfusepath{fill}%
\end{pgfscope}%
\begin{pgfscope}%
\pgfpathrectangle{\pgfqpoint{0.750000in}{0.500000in}}{\pgfqpoint{4.650000in}{3.020000in}}%
\pgfusepath{clip}%
\pgfsetbuttcap%
\pgfsetmiterjoin%
\definecolor{currentfill}{rgb}{1.000000,0.000000,0.000000}%
\pgfsetfillcolor{currentfill}%
\pgfsetlinewidth{0.000000pt}%
\definecolor{currentstroke}{rgb}{0.000000,0.000000,0.000000}%
\pgfsetstrokecolor{currentstroke}%
\pgfsetstrokeopacity{0.000000}%
\pgfsetdash{}{0pt}%
\pgfpathmoveto{\pgfqpoint{3.603409in}{0.500000in}}%
\pgfpathlineto{\pgfqpoint{3.636435in}{0.500000in}}%
\pgfpathlineto{\pgfqpoint{3.636435in}{0.503021in}}%
\pgfpathlineto{\pgfqpoint{3.603409in}{0.503021in}}%
\pgfpathlineto{\pgfqpoint{3.603409in}{0.500000in}}%
\pgfpathclose%
\pgfusepath{fill}%
\end{pgfscope}%
\begin{pgfscope}%
\pgfpathrectangle{\pgfqpoint{0.750000in}{0.500000in}}{\pgfqpoint{4.650000in}{3.020000in}}%
\pgfusepath{clip}%
\pgfsetbuttcap%
\pgfsetmiterjoin%
\definecolor{currentfill}{rgb}{1.000000,0.000000,0.000000}%
\pgfsetfillcolor{currentfill}%
\pgfsetlinewidth{0.000000pt}%
\definecolor{currentstroke}{rgb}{0.000000,0.000000,0.000000}%
\pgfsetstrokecolor{currentstroke}%
\pgfsetstrokeopacity{0.000000}%
\pgfsetdash{}{0pt}%
\pgfpathmoveto{\pgfqpoint{3.636435in}{0.500000in}}%
\pgfpathlineto{\pgfqpoint{3.669460in}{0.500000in}}%
\pgfpathlineto{\pgfqpoint{3.669460in}{0.503021in}}%
\pgfpathlineto{\pgfqpoint{3.636435in}{0.503021in}}%
\pgfpathlineto{\pgfqpoint{3.636435in}{0.500000in}}%
\pgfpathclose%
\pgfusepath{fill}%
\end{pgfscope}%
\begin{pgfscope}%
\pgfpathrectangle{\pgfqpoint{0.750000in}{0.500000in}}{\pgfqpoint{4.650000in}{3.020000in}}%
\pgfusepath{clip}%
\pgfsetbuttcap%
\pgfsetmiterjoin%
\definecolor{currentfill}{rgb}{1.000000,0.000000,0.000000}%
\pgfsetfillcolor{currentfill}%
\pgfsetlinewidth{0.000000pt}%
\definecolor{currentstroke}{rgb}{0.000000,0.000000,0.000000}%
\pgfsetstrokecolor{currentstroke}%
\pgfsetstrokeopacity{0.000000}%
\pgfsetdash{}{0pt}%
\pgfpathmoveto{\pgfqpoint{3.669460in}{0.500000in}}%
\pgfpathlineto{\pgfqpoint{3.702486in}{0.500000in}}%
\pgfpathlineto{\pgfqpoint{3.702486in}{0.503021in}}%
\pgfpathlineto{\pgfqpoint{3.669460in}{0.503021in}}%
\pgfpathlineto{\pgfqpoint{3.669460in}{0.500000in}}%
\pgfpathclose%
\pgfusepath{fill}%
\end{pgfscope}%
\begin{pgfscope}%
\pgfpathrectangle{\pgfqpoint{0.750000in}{0.500000in}}{\pgfqpoint{4.650000in}{3.020000in}}%
\pgfusepath{clip}%
\pgfsetbuttcap%
\pgfsetmiterjoin%
\definecolor{currentfill}{rgb}{1.000000,0.000000,0.000000}%
\pgfsetfillcolor{currentfill}%
\pgfsetlinewidth{0.000000pt}%
\definecolor{currentstroke}{rgb}{0.000000,0.000000,0.000000}%
\pgfsetstrokecolor{currentstroke}%
\pgfsetstrokeopacity{0.000000}%
\pgfsetdash{}{0pt}%
\pgfpathmoveto{\pgfqpoint{3.702486in}{0.500000in}}%
\pgfpathlineto{\pgfqpoint{3.735511in}{0.500000in}}%
\pgfpathlineto{\pgfqpoint{3.735511in}{0.500000in}}%
\pgfpathlineto{\pgfqpoint{3.702486in}{0.500000in}}%
\pgfpathlineto{\pgfqpoint{3.702486in}{0.500000in}}%
\pgfpathclose%
\pgfusepath{fill}%
\end{pgfscope}%
\begin{pgfscope}%
\pgfpathrectangle{\pgfqpoint{0.750000in}{0.500000in}}{\pgfqpoint{4.650000in}{3.020000in}}%
\pgfusepath{clip}%
\pgfsetbuttcap%
\pgfsetmiterjoin%
\definecolor{currentfill}{rgb}{1.000000,0.000000,0.000000}%
\pgfsetfillcolor{currentfill}%
\pgfsetlinewidth{0.000000pt}%
\definecolor{currentstroke}{rgb}{0.000000,0.000000,0.000000}%
\pgfsetstrokecolor{currentstroke}%
\pgfsetstrokeopacity{0.000000}%
\pgfsetdash{}{0pt}%
\pgfpathmoveto{\pgfqpoint{3.735511in}{0.500000in}}%
\pgfpathlineto{\pgfqpoint{3.768537in}{0.500000in}}%
\pgfpathlineto{\pgfqpoint{3.768537in}{0.500000in}}%
\pgfpathlineto{\pgfqpoint{3.735511in}{0.500000in}}%
\pgfpathlineto{\pgfqpoint{3.735511in}{0.500000in}}%
\pgfpathclose%
\pgfusepath{fill}%
\end{pgfscope}%
\begin{pgfscope}%
\pgfpathrectangle{\pgfqpoint{0.750000in}{0.500000in}}{\pgfqpoint{4.650000in}{3.020000in}}%
\pgfusepath{clip}%
\pgfsetbuttcap%
\pgfsetmiterjoin%
\definecolor{currentfill}{rgb}{1.000000,0.000000,0.000000}%
\pgfsetfillcolor{currentfill}%
\pgfsetlinewidth{0.000000pt}%
\definecolor{currentstroke}{rgb}{0.000000,0.000000,0.000000}%
\pgfsetstrokecolor{currentstroke}%
\pgfsetstrokeopacity{0.000000}%
\pgfsetdash{}{0pt}%
\pgfpathmoveto{\pgfqpoint{3.768537in}{0.500000in}}%
\pgfpathlineto{\pgfqpoint{3.801563in}{0.500000in}}%
\pgfpathlineto{\pgfqpoint{3.801563in}{0.500000in}}%
\pgfpathlineto{\pgfqpoint{3.768537in}{0.500000in}}%
\pgfpathlineto{\pgfqpoint{3.768537in}{0.500000in}}%
\pgfpathclose%
\pgfusepath{fill}%
\end{pgfscope}%
\begin{pgfscope}%
\pgfpathrectangle{\pgfqpoint{0.750000in}{0.500000in}}{\pgfqpoint{4.650000in}{3.020000in}}%
\pgfusepath{clip}%
\pgfsetbuttcap%
\pgfsetmiterjoin%
\definecolor{currentfill}{rgb}{1.000000,0.000000,0.000000}%
\pgfsetfillcolor{currentfill}%
\pgfsetlinewidth{0.000000pt}%
\definecolor{currentstroke}{rgb}{0.000000,0.000000,0.000000}%
\pgfsetstrokecolor{currentstroke}%
\pgfsetstrokeopacity{0.000000}%
\pgfsetdash{}{0pt}%
\pgfpathmoveto{\pgfqpoint{3.801563in}{0.500000in}}%
\pgfpathlineto{\pgfqpoint{3.834588in}{0.500000in}}%
\pgfpathlineto{\pgfqpoint{3.834588in}{0.500000in}}%
\pgfpathlineto{\pgfqpoint{3.801563in}{0.500000in}}%
\pgfpathlineto{\pgfqpoint{3.801563in}{0.500000in}}%
\pgfpathclose%
\pgfusepath{fill}%
\end{pgfscope}%
\begin{pgfscope}%
\pgfpathrectangle{\pgfqpoint{0.750000in}{0.500000in}}{\pgfqpoint{4.650000in}{3.020000in}}%
\pgfusepath{clip}%
\pgfsetbuttcap%
\pgfsetmiterjoin%
\definecolor{currentfill}{rgb}{1.000000,0.000000,0.000000}%
\pgfsetfillcolor{currentfill}%
\pgfsetlinewidth{0.000000pt}%
\definecolor{currentstroke}{rgb}{0.000000,0.000000,0.000000}%
\pgfsetstrokecolor{currentstroke}%
\pgfsetstrokeopacity{0.000000}%
\pgfsetdash{}{0pt}%
\pgfpathmoveto{\pgfqpoint{3.834588in}{0.500000in}}%
\pgfpathlineto{\pgfqpoint{3.867614in}{0.500000in}}%
\pgfpathlineto{\pgfqpoint{3.867614in}{0.500000in}}%
\pgfpathlineto{\pgfqpoint{3.834588in}{0.500000in}}%
\pgfpathlineto{\pgfqpoint{3.834588in}{0.500000in}}%
\pgfpathclose%
\pgfusepath{fill}%
\end{pgfscope}%
\begin{pgfscope}%
\pgfpathrectangle{\pgfqpoint{0.750000in}{0.500000in}}{\pgfqpoint{4.650000in}{3.020000in}}%
\pgfusepath{clip}%
\pgfsetbuttcap%
\pgfsetmiterjoin%
\definecolor{currentfill}{rgb}{1.000000,0.000000,0.000000}%
\pgfsetfillcolor{currentfill}%
\pgfsetlinewidth{0.000000pt}%
\definecolor{currentstroke}{rgb}{0.000000,0.000000,0.000000}%
\pgfsetstrokecolor{currentstroke}%
\pgfsetstrokeopacity{0.000000}%
\pgfsetdash{}{0pt}%
\pgfpathmoveto{\pgfqpoint{3.867614in}{0.500000in}}%
\pgfpathlineto{\pgfqpoint{3.900639in}{0.500000in}}%
\pgfpathlineto{\pgfqpoint{3.900639in}{0.500000in}}%
\pgfpathlineto{\pgfqpoint{3.867614in}{0.500000in}}%
\pgfpathlineto{\pgfqpoint{3.867614in}{0.500000in}}%
\pgfpathclose%
\pgfusepath{fill}%
\end{pgfscope}%
\begin{pgfscope}%
\pgfpathrectangle{\pgfqpoint{0.750000in}{0.500000in}}{\pgfqpoint{4.650000in}{3.020000in}}%
\pgfusepath{clip}%
\pgfsetbuttcap%
\pgfsetmiterjoin%
\definecolor{currentfill}{rgb}{1.000000,0.000000,0.000000}%
\pgfsetfillcolor{currentfill}%
\pgfsetlinewidth{0.000000pt}%
\definecolor{currentstroke}{rgb}{0.000000,0.000000,0.000000}%
\pgfsetstrokecolor{currentstroke}%
\pgfsetstrokeopacity{0.000000}%
\pgfsetdash{}{0pt}%
\pgfpathmoveto{\pgfqpoint{3.900639in}{0.500000in}}%
\pgfpathlineto{\pgfqpoint{3.933665in}{0.500000in}}%
\pgfpathlineto{\pgfqpoint{3.933665in}{0.503021in}}%
\pgfpathlineto{\pgfqpoint{3.900639in}{0.503021in}}%
\pgfpathlineto{\pgfqpoint{3.900639in}{0.500000in}}%
\pgfpathclose%
\pgfusepath{fill}%
\end{pgfscope}%
\begin{pgfscope}%
\pgfpathrectangle{\pgfqpoint{0.750000in}{0.500000in}}{\pgfqpoint{4.650000in}{3.020000in}}%
\pgfusepath{clip}%
\pgfsetbuttcap%
\pgfsetmiterjoin%
\definecolor{currentfill}{rgb}{1.000000,0.000000,0.000000}%
\pgfsetfillcolor{currentfill}%
\pgfsetlinewidth{0.000000pt}%
\definecolor{currentstroke}{rgb}{0.000000,0.000000,0.000000}%
\pgfsetstrokecolor{currentstroke}%
\pgfsetstrokeopacity{0.000000}%
\pgfsetdash{}{0pt}%
\pgfpathmoveto{\pgfqpoint{3.933665in}{0.500000in}}%
\pgfpathlineto{\pgfqpoint{3.966690in}{0.500000in}}%
\pgfpathlineto{\pgfqpoint{3.966690in}{0.500000in}}%
\pgfpathlineto{\pgfqpoint{3.933665in}{0.500000in}}%
\pgfpathlineto{\pgfqpoint{3.933665in}{0.500000in}}%
\pgfpathclose%
\pgfusepath{fill}%
\end{pgfscope}%
\begin{pgfscope}%
\pgfpathrectangle{\pgfqpoint{0.750000in}{0.500000in}}{\pgfqpoint{4.650000in}{3.020000in}}%
\pgfusepath{clip}%
\pgfsetbuttcap%
\pgfsetmiterjoin%
\definecolor{currentfill}{rgb}{1.000000,0.000000,0.000000}%
\pgfsetfillcolor{currentfill}%
\pgfsetlinewidth{0.000000pt}%
\definecolor{currentstroke}{rgb}{0.000000,0.000000,0.000000}%
\pgfsetstrokecolor{currentstroke}%
\pgfsetstrokeopacity{0.000000}%
\pgfsetdash{}{0pt}%
\pgfpathmoveto{\pgfqpoint{3.966690in}{0.500000in}}%
\pgfpathlineto{\pgfqpoint{3.999716in}{0.500000in}}%
\pgfpathlineto{\pgfqpoint{3.999716in}{0.500000in}}%
\pgfpathlineto{\pgfqpoint{3.966690in}{0.500000in}}%
\pgfpathlineto{\pgfqpoint{3.966690in}{0.500000in}}%
\pgfpathclose%
\pgfusepath{fill}%
\end{pgfscope}%
\begin{pgfscope}%
\pgfpathrectangle{\pgfqpoint{0.750000in}{0.500000in}}{\pgfqpoint{4.650000in}{3.020000in}}%
\pgfusepath{clip}%
\pgfsetbuttcap%
\pgfsetmiterjoin%
\definecolor{currentfill}{rgb}{1.000000,0.000000,0.000000}%
\pgfsetfillcolor{currentfill}%
\pgfsetlinewidth{0.000000pt}%
\definecolor{currentstroke}{rgb}{0.000000,0.000000,0.000000}%
\pgfsetstrokecolor{currentstroke}%
\pgfsetstrokeopacity{0.000000}%
\pgfsetdash{}{0pt}%
\pgfpathmoveto{\pgfqpoint{3.999716in}{0.500000in}}%
\pgfpathlineto{\pgfqpoint{4.032741in}{0.500000in}}%
\pgfpathlineto{\pgfqpoint{4.032741in}{0.500000in}}%
\pgfpathlineto{\pgfqpoint{3.999716in}{0.500000in}}%
\pgfpathlineto{\pgfqpoint{3.999716in}{0.500000in}}%
\pgfpathclose%
\pgfusepath{fill}%
\end{pgfscope}%
\begin{pgfscope}%
\pgfpathrectangle{\pgfqpoint{0.750000in}{0.500000in}}{\pgfqpoint{4.650000in}{3.020000in}}%
\pgfusepath{clip}%
\pgfsetbuttcap%
\pgfsetmiterjoin%
\definecolor{currentfill}{rgb}{1.000000,0.000000,0.000000}%
\pgfsetfillcolor{currentfill}%
\pgfsetlinewidth{0.000000pt}%
\definecolor{currentstroke}{rgb}{0.000000,0.000000,0.000000}%
\pgfsetstrokecolor{currentstroke}%
\pgfsetstrokeopacity{0.000000}%
\pgfsetdash{}{0pt}%
\pgfpathmoveto{\pgfqpoint{4.032741in}{0.500000in}}%
\pgfpathlineto{\pgfqpoint{4.065767in}{0.500000in}}%
\pgfpathlineto{\pgfqpoint{4.065767in}{0.500000in}}%
\pgfpathlineto{\pgfqpoint{4.032741in}{0.500000in}}%
\pgfpathlineto{\pgfqpoint{4.032741in}{0.500000in}}%
\pgfpathclose%
\pgfusepath{fill}%
\end{pgfscope}%
\begin{pgfscope}%
\pgfpathrectangle{\pgfqpoint{0.750000in}{0.500000in}}{\pgfqpoint{4.650000in}{3.020000in}}%
\pgfusepath{clip}%
\pgfsetbuttcap%
\pgfsetmiterjoin%
\definecolor{currentfill}{rgb}{1.000000,0.000000,0.000000}%
\pgfsetfillcolor{currentfill}%
\pgfsetlinewidth{0.000000pt}%
\definecolor{currentstroke}{rgb}{0.000000,0.000000,0.000000}%
\pgfsetstrokecolor{currentstroke}%
\pgfsetstrokeopacity{0.000000}%
\pgfsetdash{}{0pt}%
\pgfpathmoveto{\pgfqpoint{4.065767in}{0.500000in}}%
\pgfpathlineto{\pgfqpoint{4.098793in}{0.500000in}}%
\pgfpathlineto{\pgfqpoint{4.098793in}{0.500000in}}%
\pgfpathlineto{\pgfqpoint{4.065767in}{0.500000in}}%
\pgfpathlineto{\pgfqpoint{4.065767in}{0.500000in}}%
\pgfpathclose%
\pgfusepath{fill}%
\end{pgfscope}%
\begin{pgfscope}%
\pgfpathrectangle{\pgfqpoint{0.750000in}{0.500000in}}{\pgfqpoint{4.650000in}{3.020000in}}%
\pgfusepath{clip}%
\pgfsetbuttcap%
\pgfsetmiterjoin%
\definecolor{currentfill}{rgb}{1.000000,0.000000,0.000000}%
\pgfsetfillcolor{currentfill}%
\pgfsetlinewidth{0.000000pt}%
\definecolor{currentstroke}{rgb}{0.000000,0.000000,0.000000}%
\pgfsetstrokecolor{currentstroke}%
\pgfsetstrokeopacity{0.000000}%
\pgfsetdash{}{0pt}%
\pgfpathmoveto{\pgfqpoint{4.098793in}{0.500000in}}%
\pgfpathlineto{\pgfqpoint{4.131818in}{0.500000in}}%
\pgfpathlineto{\pgfqpoint{4.131818in}{0.500000in}}%
\pgfpathlineto{\pgfqpoint{4.098793in}{0.500000in}}%
\pgfpathlineto{\pgfqpoint{4.098793in}{0.500000in}}%
\pgfpathclose%
\pgfusepath{fill}%
\end{pgfscope}%
\begin{pgfscope}%
\pgfpathrectangle{\pgfqpoint{0.750000in}{0.500000in}}{\pgfqpoint{4.650000in}{3.020000in}}%
\pgfusepath{clip}%
\pgfsetbuttcap%
\pgfsetmiterjoin%
\definecolor{currentfill}{rgb}{1.000000,0.000000,0.000000}%
\pgfsetfillcolor{currentfill}%
\pgfsetlinewidth{0.000000pt}%
\definecolor{currentstroke}{rgb}{0.000000,0.000000,0.000000}%
\pgfsetstrokecolor{currentstroke}%
\pgfsetstrokeopacity{0.000000}%
\pgfsetdash{}{0pt}%
\pgfpathmoveto{\pgfqpoint{4.131818in}{0.500000in}}%
\pgfpathlineto{\pgfqpoint{4.164844in}{0.500000in}}%
\pgfpathlineto{\pgfqpoint{4.164844in}{0.500000in}}%
\pgfpathlineto{\pgfqpoint{4.131818in}{0.500000in}}%
\pgfpathlineto{\pgfqpoint{4.131818in}{0.500000in}}%
\pgfpathclose%
\pgfusepath{fill}%
\end{pgfscope}%
\begin{pgfscope}%
\pgfpathrectangle{\pgfqpoint{0.750000in}{0.500000in}}{\pgfqpoint{4.650000in}{3.020000in}}%
\pgfusepath{clip}%
\pgfsetbuttcap%
\pgfsetmiterjoin%
\definecolor{currentfill}{rgb}{1.000000,0.000000,0.000000}%
\pgfsetfillcolor{currentfill}%
\pgfsetlinewidth{0.000000pt}%
\definecolor{currentstroke}{rgb}{0.000000,0.000000,0.000000}%
\pgfsetstrokecolor{currentstroke}%
\pgfsetstrokeopacity{0.000000}%
\pgfsetdash{}{0pt}%
\pgfpathmoveto{\pgfqpoint{4.164844in}{0.500000in}}%
\pgfpathlineto{\pgfqpoint{4.197869in}{0.500000in}}%
\pgfpathlineto{\pgfqpoint{4.197869in}{0.500000in}}%
\pgfpathlineto{\pgfqpoint{4.164844in}{0.500000in}}%
\pgfpathlineto{\pgfqpoint{4.164844in}{0.500000in}}%
\pgfpathclose%
\pgfusepath{fill}%
\end{pgfscope}%
\begin{pgfscope}%
\pgfpathrectangle{\pgfqpoint{0.750000in}{0.500000in}}{\pgfqpoint{4.650000in}{3.020000in}}%
\pgfusepath{clip}%
\pgfsetbuttcap%
\pgfsetmiterjoin%
\definecolor{currentfill}{rgb}{1.000000,0.000000,0.000000}%
\pgfsetfillcolor{currentfill}%
\pgfsetlinewidth{0.000000pt}%
\definecolor{currentstroke}{rgb}{0.000000,0.000000,0.000000}%
\pgfsetstrokecolor{currentstroke}%
\pgfsetstrokeopacity{0.000000}%
\pgfsetdash{}{0pt}%
\pgfpathmoveto{\pgfqpoint{4.197869in}{0.500000in}}%
\pgfpathlineto{\pgfqpoint{4.230895in}{0.500000in}}%
\pgfpathlineto{\pgfqpoint{4.230895in}{0.500000in}}%
\pgfpathlineto{\pgfqpoint{4.197869in}{0.500000in}}%
\pgfpathlineto{\pgfqpoint{4.197869in}{0.500000in}}%
\pgfpathclose%
\pgfusepath{fill}%
\end{pgfscope}%
\begin{pgfscope}%
\pgfpathrectangle{\pgfqpoint{0.750000in}{0.500000in}}{\pgfqpoint{4.650000in}{3.020000in}}%
\pgfusepath{clip}%
\pgfsetbuttcap%
\pgfsetmiterjoin%
\definecolor{currentfill}{rgb}{1.000000,0.000000,0.000000}%
\pgfsetfillcolor{currentfill}%
\pgfsetlinewidth{0.000000pt}%
\definecolor{currentstroke}{rgb}{0.000000,0.000000,0.000000}%
\pgfsetstrokecolor{currentstroke}%
\pgfsetstrokeopacity{0.000000}%
\pgfsetdash{}{0pt}%
\pgfpathmoveto{\pgfqpoint{4.230895in}{0.500000in}}%
\pgfpathlineto{\pgfqpoint{4.263920in}{0.500000in}}%
\pgfpathlineto{\pgfqpoint{4.263920in}{0.500000in}}%
\pgfpathlineto{\pgfqpoint{4.230895in}{0.500000in}}%
\pgfpathlineto{\pgfqpoint{4.230895in}{0.500000in}}%
\pgfpathclose%
\pgfusepath{fill}%
\end{pgfscope}%
\begin{pgfscope}%
\pgfpathrectangle{\pgfqpoint{0.750000in}{0.500000in}}{\pgfqpoint{4.650000in}{3.020000in}}%
\pgfusepath{clip}%
\pgfsetbuttcap%
\pgfsetmiterjoin%
\definecolor{currentfill}{rgb}{1.000000,0.000000,0.000000}%
\pgfsetfillcolor{currentfill}%
\pgfsetlinewidth{0.000000pt}%
\definecolor{currentstroke}{rgb}{0.000000,0.000000,0.000000}%
\pgfsetstrokecolor{currentstroke}%
\pgfsetstrokeopacity{0.000000}%
\pgfsetdash{}{0pt}%
\pgfpathmoveto{\pgfqpoint{4.263920in}{0.500000in}}%
\pgfpathlineto{\pgfqpoint{4.296946in}{0.500000in}}%
\pgfpathlineto{\pgfqpoint{4.296946in}{0.500000in}}%
\pgfpathlineto{\pgfqpoint{4.263920in}{0.500000in}}%
\pgfpathlineto{\pgfqpoint{4.263920in}{0.500000in}}%
\pgfpathclose%
\pgfusepath{fill}%
\end{pgfscope}%
\begin{pgfscope}%
\pgfpathrectangle{\pgfqpoint{0.750000in}{0.500000in}}{\pgfqpoint{4.650000in}{3.020000in}}%
\pgfusepath{clip}%
\pgfsetbuttcap%
\pgfsetmiterjoin%
\definecolor{currentfill}{rgb}{1.000000,0.000000,0.000000}%
\pgfsetfillcolor{currentfill}%
\pgfsetlinewidth{0.000000pt}%
\definecolor{currentstroke}{rgb}{0.000000,0.000000,0.000000}%
\pgfsetstrokecolor{currentstroke}%
\pgfsetstrokeopacity{0.000000}%
\pgfsetdash{}{0pt}%
\pgfpathmoveto{\pgfqpoint{4.296946in}{0.500000in}}%
\pgfpathlineto{\pgfqpoint{4.329972in}{0.500000in}}%
\pgfpathlineto{\pgfqpoint{4.329972in}{0.500000in}}%
\pgfpathlineto{\pgfqpoint{4.296946in}{0.500000in}}%
\pgfpathlineto{\pgfqpoint{4.296946in}{0.500000in}}%
\pgfpathclose%
\pgfusepath{fill}%
\end{pgfscope}%
\begin{pgfscope}%
\pgfpathrectangle{\pgfqpoint{0.750000in}{0.500000in}}{\pgfqpoint{4.650000in}{3.020000in}}%
\pgfusepath{clip}%
\pgfsetbuttcap%
\pgfsetmiterjoin%
\definecolor{currentfill}{rgb}{1.000000,0.000000,0.000000}%
\pgfsetfillcolor{currentfill}%
\pgfsetlinewidth{0.000000pt}%
\definecolor{currentstroke}{rgb}{0.000000,0.000000,0.000000}%
\pgfsetstrokecolor{currentstroke}%
\pgfsetstrokeopacity{0.000000}%
\pgfsetdash{}{0pt}%
\pgfpathmoveto{\pgfqpoint{4.329972in}{0.500000in}}%
\pgfpathlineto{\pgfqpoint{4.362997in}{0.500000in}}%
\pgfpathlineto{\pgfqpoint{4.362997in}{0.503021in}}%
\pgfpathlineto{\pgfqpoint{4.329972in}{0.503021in}}%
\pgfpathlineto{\pgfqpoint{4.329972in}{0.500000in}}%
\pgfpathclose%
\pgfusepath{fill}%
\end{pgfscope}%
\begin{pgfscope}%
\pgfpathrectangle{\pgfqpoint{0.750000in}{0.500000in}}{\pgfqpoint{4.650000in}{3.020000in}}%
\pgfusepath{clip}%
\pgfsetbuttcap%
\pgfsetmiterjoin%
\definecolor{currentfill}{rgb}{1.000000,0.000000,0.000000}%
\pgfsetfillcolor{currentfill}%
\pgfsetlinewidth{0.000000pt}%
\definecolor{currentstroke}{rgb}{0.000000,0.000000,0.000000}%
\pgfsetstrokecolor{currentstroke}%
\pgfsetstrokeopacity{0.000000}%
\pgfsetdash{}{0pt}%
\pgfpathmoveto{\pgfqpoint{4.362997in}{0.500000in}}%
\pgfpathlineto{\pgfqpoint{4.396023in}{0.500000in}}%
\pgfpathlineto{\pgfqpoint{4.396023in}{0.500000in}}%
\pgfpathlineto{\pgfqpoint{4.362997in}{0.500000in}}%
\pgfpathlineto{\pgfqpoint{4.362997in}{0.500000in}}%
\pgfpathclose%
\pgfusepath{fill}%
\end{pgfscope}%
\begin{pgfscope}%
\pgfpathrectangle{\pgfqpoint{0.750000in}{0.500000in}}{\pgfqpoint{4.650000in}{3.020000in}}%
\pgfusepath{clip}%
\pgfsetbuttcap%
\pgfsetmiterjoin%
\definecolor{currentfill}{rgb}{1.000000,0.000000,0.000000}%
\pgfsetfillcolor{currentfill}%
\pgfsetlinewidth{0.000000pt}%
\definecolor{currentstroke}{rgb}{0.000000,0.000000,0.000000}%
\pgfsetstrokecolor{currentstroke}%
\pgfsetstrokeopacity{0.000000}%
\pgfsetdash{}{0pt}%
\pgfpathmoveto{\pgfqpoint{4.396023in}{0.500000in}}%
\pgfpathlineto{\pgfqpoint{4.429048in}{0.500000in}}%
\pgfpathlineto{\pgfqpoint{4.429048in}{0.500000in}}%
\pgfpathlineto{\pgfqpoint{4.396023in}{0.500000in}}%
\pgfpathlineto{\pgfqpoint{4.396023in}{0.500000in}}%
\pgfpathclose%
\pgfusepath{fill}%
\end{pgfscope}%
\begin{pgfscope}%
\pgfpathrectangle{\pgfqpoint{0.750000in}{0.500000in}}{\pgfqpoint{4.650000in}{3.020000in}}%
\pgfusepath{clip}%
\pgfsetbuttcap%
\pgfsetmiterjoin%
\definecolor{currentfill}{rgb}{1.000000,0.000000,0.000000}%
\pgfsetfillcolor{currentfill}%
\pgfsetlinewidth{0.000000pt}%
\definecolor{currentstroke}{rgb}{0.000000,0.000000,0.000000}%
\pgfsetstrokecolor{currentstroke}%
\pgfsetstrokeopacity{0.000000}%
\pgfsetdash{}{0pt}%
\pgfpathmoveto{\pgfqpoint{4.429048in}{0.500000in}}%
\pgfpathlineto{\pgfqpoint{4.462074in}{0.500000in}}%
\pgfpathlineto{\pgfqpoint{4.462074in}{0.503021in}}%
\pgfpathlineto{\pgfqpoint{4.429048in}{0.503021in}}%
\pgfpathlineto{\pgfqpoint{4.429048in}{0.500000in}}%
\pgfpathclose%
\pgfusepath{fill}%
\end{pgfscope}%
\begin{pgfscope}%
\pgfpathrectangle{\pgfqpoint{0.750000in}{0.500000in}}{\pgfqpoint{4.650000in}{3.020000in}}%
\pgfusepath{clip}%
\pgfsetbuttcap%
\pgfsetmiterjoin%
\definecolor{currentfill}{rgb}{1.000000,0.000000,0.000000}%
\pgfsetfillcolor{currentfill}%
\pgfsetlinewidth{0.000000pt}%
\definecolor{currentstroke}{rgb}{0.000000,0.000000,0.000000}%
\pgfsetstrokecolor{currentstroke}%
\pgfsetstrokeopacity{0.000000}%
\pgfsetdash{}{0pt}%
\pgfpathmoveto{\pgfqpoint{4.462074in}{0.500000in}}%
\pgfpathlineto{\pgfqpoint{4.495099in}{0.500000in}}%
\pgfpathlineto{\pgfqpoint{4.495099in}{0.500000in}}%
\pgfpathlineto{\pgfqpoint{4.462074in}{0.500000in}}%
\pgfpathlineto{\pgfqpoint{4.462074in}{0.500000in}}%
\pgfpathclose%
\pgfusepath{fill}%
\end{pgfscope}%
\begin{pgfscope}%
\pgfpathrectangle{\pgfqpoint{0.750000in}{0.500000in}}{\pgfqpoint{4.650000in}{3.020000in}}%
\pgfusepath{clip}%
\pgfsetbuttcap%
\pgfsetmiterjoin%
\definecolor{currentfill}{rgb}{1.000000,0.000000,0.000000}%
\pgfsetfillcolor{currentfill}%
\pgfsetlinewidth{0.000000pt}%
\definecolor{currentstroke}{rgb}{0.000000,0.000000,0.000000}%
\pgfsetstrokecolor{currentstroke}%
\pgfsetstrokeopacity{0.000000}%
\pgfsetdash{}{0pt}%
\pgfpathmoveto{\pgfqpoint{4.495099in}{0.500000in}}%
\pgfpathlineto{\pgfqpoint{4.528125in}{0.500000in}}%
\pgfpathlineto{\pgfqpoint{4.528125in}{0.503021in}}%
\pgfpathlineto{\pgfqpoint{4.495099in}{0.503021in}}%
\pgfpathlineto{\pgfqpoint{4.495099in}{0.500000in}}%
\pgfpathclose%
\pgfusepath{fill}%
\end{pgfscope}%
\begin{pgfscope}%
\pgfpathrectangle{\pgfqpoint{0.750000in}{0.500000in}}{\pgfqpoint{4.650000in}{3.020000in}}%
\pgfusepath{clip}%
\pgfsetbuttcap%
\pgfsetmiterjoin%
\definecolor{currentfill}{rgb}{1.000000,0.000000,0.000000}%
\pgfsetfillcolor{currentfill}%
\pgfsetlinewidth{0.000000pt}%
\definecolor{currentstroke}{rgb}{0.000000,0.000000,0.000000}%
\pgfsetstrokecolor{currentstroke}%
\pgfsetstrokeopacity{0.000000}%
\pgfsetdash{}{0pt}%
\pgfpathmoveto{\pgfqpoint{4.528125in}{0.500000in}}%
\pgfpathlineto{\pgfqpoint{4.561151in}{0.500000in}}%
\pgfpathlineto{\pgfqpoint{4.561151in}{0.500000in}}%
\pgfpathlineto{\pgfqpoint{4.528125in}{0.500000in}}%
\pgfpathlineto{\pgfqpoint{4.528125in}{0.500000in}}%
\pgfpathclose%
\pgfusepath{fill}%
\end{pgfscope}%
\begin{pgfscope}%
\pgfpathrectangle{\pgfqpoint{0.750000in}{0.500000in}}{\pgfqpoint{4.650000in}{3.020000in}}%
\pgfusepath{clip}%
\pgfsetbuttcap%
\pgfsetmiterjoin%
\definecolor{currentfill}{rgb}{1.000000,0.000000,0.000000}%
\pgfsetfillcolor{currentfill}%
\pgfsetlinewidth{0.000000pt}%
\definecolor{currentstroke}{rgb}{0.000000,0.000000,0.000000}%
\pgfsetstrokecolor{currentstroke}%
\pgfsetstrokeopacity{0.000000}%
\pgfsetdash{}{0pt}%
\pgfpathmoveto{\pgfqpoint{4.561151in}{0.500000in}}%
\pgfpathlineto{\pgfqpoint{4.594176in}{0.500000in}}%
\pgfpathlineto{\pgfqpoint{4.594176in}{0.500000in}}%
\pgfpathlineto{\pgfqpoint{4.561151in}{0.500000in}}%
\pgfpathlineto{\pgfqpoint{4.561151in}{0.500000in}}%
\pgfpathclose%
\pgfusepath{fill}%
\end{pgfscope}%
\begin{pgfscope}%
\pgfpathrectangle{\pgfqpoint{0.750000in}{0.500000in}}{\pgfqpoint{4.650000in}{3.020000in}}%
\pgfusepath{clip}%
\pgfsetbuttcap%
\pgfsetmiterjoin%
\definecolor{currentfill}{rgb}{1.000000,0.000000,0.000000}%
\pgfsetfillcolor{currentfill}%
\pgfsetlinewidth{0.000000pt}%
\definecolor{currentstroke}{rgb}{0.000000,0.000000,0.000000}%
\pgfsetstrokecolor{currentstroke}%
\pgfsetstrokeopacity{0.000000}%
\pgfsetdash{}{0pt}%
\pgfpathmoveto{\pgfqpoint{4.594176in}{0.500000in}}%
\pgfpathlineto{\pgfqpoint{4.627202in}{0.500000in}}%
\pgfpathlineto{\pgfqpoint{4.627202in}{0.500000in}}%
\pgfpathlineto{\pgfqpoint{4.594176in}{0.500000in}}%
\pgfpathlineto{\pgfqpoint{4.594176in}{0.500000in}}%
\pgfpathclose%
\pgfusepath{fill}%
\end{pgfscope}%
\begin{pgfscope}%
\pgfpathrectangle{\pgfqpoint{0.750000in}{0.500000in}}{\pgfqpoint{4.650000in}{3.020000in}}%
\pgfusepath{clip}%
\pgfsetbuttcap%
\pgfsetmiterjoin%
\definecolor{currentfill}{rgb}{1.000000,0.000000,0.000000}%
\pgfsetfillcolor{currentfill}%
\pgfsetlinewidth{0.000000pt}%
\definecolor{currentstroke}{rgb}{0.000000,0.000000,0.000000}%
\pgfsetstrokecolor{currentstroke}%
\pgfsetstrokeopacity{0.000000}%
\pgfsetdash{}{0pt}%
\pgfpathmoveto{\pgfqpoint{4.627202in}{0.500000in}}%
\pgfpathlineto{\pgfqpoint{4.660227in}{0.500000in}}%
\pgfpathlineto{\pgfqpoint{4.660227in}{0.500000in}}%
\pgfpathlineto{\pgfqpoint{4.627202in}{0.500000in}}%
\pgfpathlineto{\pgfqpoint{4.627202in}{0.500000in}}%
\pgfpathclose%
\pgfusepath{fill}%
\end{pgfscope}%
\begin{pgfscope}%
\pgfpathrectangle{\pgfqpoint{0.750000in}{0.500000in}}{\pgfqpoint{4.650000in}{3.020000in}}%
\pgfusepath{clip}%
\pgfsetbuttcap%
\pgfsetmiterjoin%
\definecolor{currentfill}{rgb}{1.000000,0.000000,0.000000}%
\pgfsetfillcolor{currentfill}%
\pgfsetlinewidth{0.000000pt}%
\definecolor{currentstroke}{rgb}{0.000000,0.000000,0.000000}%
\pgfsetstrokecolor{currentstroke}%
\pgfsetstrokeopacity{0.000000}%
\pgfsetdash{}{0pt}%
\pgfpathmoveto{\pgfqpoint{4.660227in}{0.500000in}}%
\pgfpathlineto{\pgfqpoint{4.693253in}{0.500000in}}%
\pgfpathlineto{\pgfqpoint{4.693253in}{0.503021in}}%
\pgfpathlineto{\pgfqpoint{4.660227in}{0.503021in}}%
\pgfpathlineto{\pgfqpoint{4.660227in}{0.500000in}}%
\pgfpathclose%
\pgfusepath{fill}%
\end{pgfscope}%
\begin{pgfscope}%
\pgfpathrectangle{\pgfqpoint{0.750000in}{0.500000in}}{\pgfqpoint{4.650000in}{3.020000in}}%
\pgfusepath{clip}%
\pgfsetbuttcap%
\pgfsetmiterjoin%
\definecolor{currentfill}{rgb}{1.000000,0.000000,0.000000}%
\pgfsetfillcolor{currentfill}%
\pgfsetlinewidth{0.000000pt}%
\definecolor{currentstroke}{rgb}{0.000000,0.000000,0.000000}%
\pgfsetstrokecolor{currentstroke}%
\pgfsetstrokeopacity{0.000000}%
\pgfsetdash{}{0pt}%
\pgfpathmoveto{\pgfqpoint{4.693253in}{0.500000in}}%
\pgfpathlineto{\pgfqpoint{4.726278in}{0.500000in}}%
\pgfpathlineto{\pgfqpoint{4.726278in}{0.500000in}}%
\pgfpathlineto{\pgfqpoint{4.693253in}{0.500000in}}%
\pgfpathlineto{\pgfqpoint{4.693253in}{0.500000in}}%
\pgfpathclose%
\pgfusepath{fill}%
\end{pgfscope}%
\begin{pgfscope}%
\pgfpathrectangle{\pgfqpoint{0.750000in}{0.500000in}}{\pgfqpoint{4.650000in}{3.020000in}}%
\pgfusepath{clip}%
\pgfsetbuttcap%
\pgfsetmiterjoin%
\definecolor{currentfill}{rgb}{1.000000,0.000000,0.000000}%
\pgfsetfillcolor{currentfill}%
\pgfsetlinewidth{0.000000pt}%
\definecolor{currentstroke}{rgb}{0.000000,0.000000,0.000000}%
\pgfsetstrokecolor{currentstroke}%
\pgfsetstrokeopacity{0.000000}%
\pgfsetdash{}{0pt}%
\pgfpathmoveto{\pgfqpoint{4.726278in}{0.500000in}}%
\pgfpathlineto{\pgfqpoint{4.759304in}{0.500000in}}%
\pgfpathlineto{\pgfqpoint{4.759304in}{0.500000in}}%
\pgfpathlineto{\pgfqpoint{4.726278in}{0.500000in}}%
\pgfpathlineto{\pgfqpoint{4.726278in}{0.500000in}}%
\pgfpathclose%
\pgfusepath{fill}%
\end{pgfscope}%
\begin{pgfscope}%
\pgfpathrectangle{\pgfqpoint{0.750000in}{0.500000in}}{\pgfqpoint{4.650000in}{3.020000in}}%
\pgfusepath{clip}%
\pgfsetbuttcap%
\pgfsetmiterjoin%
\definecolor{currentfill}{rgb}{1.000000,0.000000,0.000000}%
\pgfsetfillcolor{currentfill}%
\pgfsetlinewidth{0.000000pt}%
\definecolor{currentstroke}{rgb}{0.000000,0.000000,0.000000}%
\pgfsetstrokecolor{currentstroke}%
\pgfsetstrokeopacity{0.000000}%
\pgfsetdash{}{0pt}%
\pgfpathmoveto{\pgfqpoint{4.759304in}{0.500000in}}%
\pgfpathlineto{\pgfqpoint{4.792330in}{0.500000in}}%
\pgfpathlineto{\pgfqpoint{4.792330in}{0.503021in}}%
\pgfpathlineto{\pgfqpoint{4.759304in}{0.503021in}}%
\pgfpathlineto{\pgfqpoint{4.759304in}{0.500000in}}%
\pgfpathclose%
\pgfusepath{fill}%
\end{pgfscope}%
\begin{pgfscope}%
\pgfpathrectangle{\pgfqpoint{0.750000in}{0.500000in}}{\pgfqpoint{4.650000in}{3.020000in}}%
\pgfusepath{clip}%
\pgfsetbuttcap%
\pgfsetmiterjoin%
\definecolor{currentfill}{rgb}{1.000000,0.000000,0.000000}%
\pgfsetfillcolor{currentfill}%
\pgfsetlinewidth{0.000000pt}%
\definecolor{currentstroke}{rgb}{0.000000,0.000000,0.000000}%
\pgfsetstrokecolor{currentstroke}%
\pgfsetstrokeopacity{0.000000}%
\pgfsetdash{}{0pt}%
\pgfpathmoveto{\pgfqpoint{4.792330in}{0.500000in}}%
\pgfpathlineto{\pgfqpoint{4.825355in}{0.500000in}}%
\pgfpathlineto{\pgfqpoint{4.825355in}{0.500000in}}%
\pgfpathlineto{\pgfqpoint{4.792330in}{0.500000in}}%
\pgfpathlineto{\pgfqpoint{4.792330in}{0.500000in}}%
\pgfpathclose%
\pgfusepath{fill}%
\end{pgfscope}%
\begin{pgfscope}%
\pgfpathrectangle{\pgfqpoint{0.750000in}{0.500000in}}{\pgfqpoint{4.650000in}{3.020000in}}%
\pgfusepath{clip}%
\pgfsetbuttcap%
\pgfsetmiterjoin%
\definecolor{currentfill}{rgb}{1.000000,0.000000,0.000000}%
\pgfsetfillcolor{currentfill}%
\pgfsetlinewidth{0.000000pt}%
\definecolor{currentstroke}{rgb}{0.000000,0.000000,0.000000}%
\pgfsetstrokecolor{currentstroke}%
\pgfsetstrokeopacity{0.000000}%
\pgfsetdash{}{0pt}%
\pgfpathmoveto{\pgfqpoint{4.825355in}{0.500000in}}%
\pgfpathlineto{\pgfqpoint{4.858381in}{0.500000in}}%
\pgfpathlineto{\pgfqpoint{4.858381in}{0.500000in}}%
\pgfpathlineto{\pgfqpoint{4.825355in}{0.500000in}}%
\pgfpathlineto{\pgfqpoint{4.825355in}{0.500000in}}%
\pgfpathclose%
\pgfusepath{fill}%
\end{pgfscope}%
\begin{pgfscope}%
\pgfpathrectangle{\pgfqpoint{0.750000in}{0.500000in}}{\pgfqpoint{4.650000in}{3.020000in}}%
\pgfusepath{clip}%
\pgfsetbuttcap%
\pgfsetmiterjoin%
\definecolor{currentfill}{rgb}{1.000000,0.000000,0.000000}%
\pgfsetfillcolor{currentfill}%
\pgfsetlinewidth{0.000000pt}%
\definecolor{currentstroke}{rgb}{0.000000,0.000000,0.000000}%
\pgfsetstrokecolor{currentstroke}%
\pgfsetstrokeopacity{0.000000}%
\pgfsetdash{}{0pt}%
\pgfpathmoveto{\pgfqpoint{4.858381in}{0.500000in}}%
\pgfpathlineto{\pgfqpoint{4.891406in}{0.500000in}}%
\pgfpathlineto{\pgfqpoint{4.891406in}{0.506042in}}%
\pgfpathlineto{\pgfqpoint{4.858381in}{0.506042in}}%
\pgfpathlineto{\pgfqpoint{4.858381in}{0.500000in}}%
\pgfpathclose%
\pgfusepath{fill}%
\end{pgfscope}%
\begin{pgfscope}%
\pgfpathrectangle{\pgfqpoint{0.750000in}{0.500000in}}{\pgfqpoint{4.650000in}{3.020000in}}%
\pgfusepath{clip}%
\pgfsetbuttcap%
\pgfsetmiterjoin%
\definecolor{currentfill}{rgb}{1.000000,0.000000,0.000000}%
\pgfsetfillcolor{currentfill}%
\pgfsetlinewidth{0.000000pt}%
\definecolor{currentstroke}{rgb}{0.000000,0.000000,0.000000}%
\pgfsetstrokecolor{currentstroke}%
\pgfsetstrokeopacity{0.000000}%
\pgfsetdash{}{0pt}%
\pgfpathmoveto{\pgfqpoint{4.891406in}{0.500000in}}%
\pgfpathlineto{\pgfqpoint{4.924432in}{0.500000in}}%
\pgfpathlineto{\pgfqpoint{4.924432in}{0.500000in}}%
\pgfpathlineto{\pgfqpoint{4.891406in}{0.500000in}}%
\pgfpathlineto{\pgfqpoint{4.891406in}{0.500000in}}%
\pgfpathclose%
\pgfusepath{fill}%
\end{pgfscope}%
\begin{pgfscope}%
\pgfpathrectangle{\pgfqpoint{0.750000in}{0.500000in}}{\pgfqpoint{4.650000in}{3.020000in}}%
\pgfusepath{clip}%
\pgfsetbuttcap%
\pgfsetmiterjoin%
\definecolor{currentfill}{rgb}{1.000000,0.000000,0.000000}%
\pgfsetfillcolor{currentfill}%
\pgfsetlinewidth{0.000000pt}%
\definecolor{currentstroke}{rgb}{0.000000,0.000000,0.000000}%
\pgfsetstrokecolor{currentstroke}%
\pgfsetstrokeopacity{0.000000}%
\pgfsetdash{}{0pt}%
\pgfpathmoveto{\pgfqpoint{4.924432in}{0.500000in}}%
\pgfpathlineto{\pgfqpoint{4.957457in}{0.500000in}}%
\pgfpathlineto{\pgfqpoint{4.957457in}{0.500000in}}%
\pgfpathlineto{\pgfqpoint{4.924432in}{0.500000in}}%
\pgfpathlineto{\pgfqpoint{4.924432in}{0.500000in}}%
\pgfpathclose%
\pgfusepath{fill}%
\end{pgfscope}%
\begin{pgfscope}%
\pgfpathrectangle{\pgfqpoint{0.750000in}{0.500000in}}{\pgfqpoint{4.650000in}{3.020000in}}%
\pgfusepath{clip}%
\pgfsetbuttcap%
\pgfsetmiterjoin%
\definecolor{currentfill}{rgb}{1.000000,0.000000,0.000000}%
\pgfsetfillcolor{currentfill}%
\pgfsetlinewidth{0.000000pt}%
\definecolor{currentstroke}{rgb}{0.000000,0.000000,0.000000}%
\pgfsetstrokecolor{currentstroke}%
\pgfsetstrokeopacity{0.000000}%
\pgfsetdash{}{0pt}%
\pgfpathmoveto{\pgfqpoint{4.957457in}{0.500000in}}%
\pgfpathlineto{\pgfqpoint{4.990483in}{0.500000in}}%
\pgfpathlineto{\pgfqpoint{4.990483in}{0.503021in}}%
\pgfpathlineto{\pgfqpoint{4.957457in}{0.503021in}}%
\pgfpathlineto{\pgfqpoint{4.957457in}{0.500000in}}%
\pgfpathclose%
\pgfusepath{fill}%
\end{pgfscope}%
\begin{pgfscope}%
\pgfpathrectangle{\pgfqpoint{0.750000in}{0.500000in}}{\pgfqpoint{4.650000in}{3.020000in}}%
\pgfusepath{clip}%
\pgfsetbuttcap%
\pgfsetmiterjoin%
\definecolor{currentfill}{rgb}{1.000000,0.000000,0.000000}%
\pgfsetfillcolor{currentfill}%
\pgfsetlinewidth{0.000000pt}%
\definecolor{currentstroke}{rgb}{0.000000,0.000000,0.000000}%
\pgfsetstrokecolor{currentstroke}%
\pgfsetstrokeopacity{0.000000}%
\pgfsetdash{}{0pt}%
\pgfpathmoveto{\pgfqpoint{4.990483in}{0.500000in}}%
\pgfpathlineto{\pgfqpoint{5.023509in}{0.500000in}}%
\pgfpathlineto{\pgfqpoint{5.023509in}{0.506042in}}%
\pgfpathlineto{\pgfqpoint{4.990483in}{0.506042in}}%
\pgfpathlineto{\pgfqpoint{4.990483in}{0.500000in}}%
\pgfpathclose%
\pgfusepath{fill}%
\end{pgfscope}%
\begin{pgfscope}%
\pgfpathrectangle{\pgfqpoint{0.750000in}{0.500000in}}{\pgfqpoint{4.650000in}{3.020000in}}%
\pgfusepath{clip}%
\pgfsetbuttcap%
\pgfsetmiterjoin%
\definecolor{currentfill}{rgb}{1.000000,0.000000,0.000000}%
\pgfsetfillcolor{currentfill}%
\pgfsetlinewidth{0.000000pt}%
\definecolor{currentstroke}{rgb}{0.000000,0.000000,0.000000}%
\pgfsetstrokecolor{currentstroke}%
\pgfsetstrokeopacity{0.000000}%
\pgfsetdash{}{0pt}%
\pgfpathmoveto{\pgfqpoint{5.023509in}{0.500000in}}%
\pgfpathlineto{\pgfqpoint{5.056534in}{0.500000in}}%
\pgfpathlineto{\pgfqpoint{5.056534in}{0.500000in}}%
\pgfpathlineto{\pgfqpoint{5.023509in}{0.500000in}}%
\pgfpathlineto{\pgfqpoint{5.023509in}{0.500000in}}%
\pgfpathclose%
\pgfusepath{fill}%
\end{pgfscope}%
\begin{pgfscope}%
\pgfpathrectangle{\pgfqpoint{0.750000in}{0.500000in}}{\pgfqpoint{4.650000in}{3.020000in}}%
\pgfusepath{clip}%
\pgfsetbuttcap%
\pgfsetmiterjoin%
\definecolor{currentfill}{rgb}{1.000000,0.000000,0.000000}%
\pgfsetfillcolor{currentfill}%
\pgfsetlinewidth{0.000000pt}%
\definecolor{currentstroke}{rgb}{0.000000,0.000000,0.000000}%
\pgfsetstrokecolor{currentstroke}%
\pgfsetstrokeopacity{0.000000}%
\pgfsetdash{}{0pt}%
\pgfpathmoveto{\pgfqpoint{5.056534in}{0.500000in}}%
\pgfpathlineto{\pgfqpoint{5.089560in}{0.500000in}}%
\pgfpathlineto{\pgfqpoint{5.089560in}{0.500000in}}%
\pgfpathlineto{\pgfqpoint{5.056534in}{0.500000in}}%
\pgfpathlineto{\pgfqpoint{5.056534in}{0.500000in}}%
\pgfpathclose%
\pgfusepath{fill}%
\end{pgfscope}%
\begin{pgfscope}%
\pgfpathrectangle{\pgfqpoint{0.750000in}{0.500000in}}{\pgfqpoint{4.650000in}{3.020000in}}%
\pgfusepath{clip}%
\pgfsetbuttcap%
\pgfsetmiterjoin%
\definecolor{currentfill}{rgb}{1.000000,0.000000,0.000000}%
\pgfsetfillcolor{currentfill}%
\pgfsetlinewidth{0.000000pt}%
\definecolor{currentstroke}{rgb}{0.000000,0.000000,0.000000}%
\pgfsetstrokecolor{currentstroke}%
\pgfsetstrokeopacity{0.000000}%
\pgfsetdash{}{0pt}%
\pgfpathmoveto{\pgfqpoint{5.089560in}{0.500000in}}%
\pgfpathlineto{\pgfqpoint{5.122585in}{0.500000in}}%
\pgfpathlineto{\pgfqpoint{5.122585in}{0.503021in}}%
\pgfpathlineto{\pgfqpoint{5.089560in}{0.503021in}}%
\pgfpathlineto{\pgfqpoint{5.089560in}{0.500000in}}%
\pgfpathclose%
\pgfusepath{fill}%
\end{pgfscope}%
\begin{pgfscope}%
\pgfpathrectangle{\pgfqpoint{0.750000in}{0.500000in}}{\pgfqpoint{4.650000in}{3.020000in}}%
\pgfusepath{clip}%
\pgfsetbuttcap%
\pgfsetmiterjoin%
\definecolor{currentfill}{rgb}{1.000000,0.000000,0.000000}%
\pgfsetfillcolor{currentfill}%
\pgfsetlinewidth{0.000000pt}%
\definecolor{currentstroke}{rgb}{0.000000,0.000000,0.000000}%
\pgfsetstrokecolor{currentstroke}%
\pgfsetstrokeopacity{0.000000}%
\pgfsetdash{}{0pt}%
\pgfpathmoveto{\pgfqpoint{5.122585in}{0.500000in}}%
\pgfpathlineto{\pgfqpoint{5.155611in}{0.500000in}}%
\pgfpathlineto{\pgfqpoint{5.155611in}{0.503021in}}%
\pgfpathlineto{\pgfqpoint{5.122585in}{0.503021in}}%
\pgfpathlineto{\pgfqpoint{5.122585in}{0.500000in}}%
\pgfpathclose%
\pgfusepath{fill}%
\end{pgfscope}%
\begin{pgfscope}%
\pgfpathrectangle{\pgfqpoint{0.750000in}{0.500000in}}{\pgfqpoint{4.650000in}{3.020000in}}%
\pgfusepath{clip}%
\pgfsetbuttcap%
\pgfsetmiterjoin%
\definecolor{currentfill}{rgb}{1.000000,0.000000,0.000000}%
\pgfsetfillcolor{currentfill}%
\pgfsetlinewidth{0.000000pt}%
\definecolor{currentstroke}{rgb}{0.000000,0.000000,0.000000}%
\pgfsetstrokecolor{currentstroke}%
\pgfsetstrokeopacity{0.000000}%
\pgfsetdash{}{0pt}%
\pgfpathmoveto{\pgfqpoint{5.155611in}{0.500000in}}%
\pgfpathlineto{\pgfqpoint{5.188636in}{0.500000in}}%
\pgfpathlineto{\pgfqpoint{5.188636in}{0.509064in}}%
\pgfpathlineto{\pgfqpoint{5.155611in}{0.509064in}}%
\pgfpathlineto{\pgfqpoint{5.155611in}{0.500000in}}%
\pgfpathclose%
\pgfusepath{fill}%
\end{pgfscope}%
\begin{pgfscope}%
\pgfsetbuttcap%
\pgfsetroundjoin%
\definecolor{currentfill}{rgb}{0.000000,0.000000,0.000000}%
\pgfsetfillcolor{currentfill}%
\pgfsetlinewidth{0.803000pt}%
\definecolor{currentstroke}{rgb}{0.000000,0.000000,0.000000}%
\pgfsetstrokecolor{currentstroke}%
\pgfsetdash{}{0pt}%
\pgfsys@defobject{currentmarker}{\pgfqpoint{0.000000in}{-0.048611in}}{\pgfqpoint{0.000000in}{0.000000in}}{%
\pgfpathmoveto{\pgfqpoint{0.000000in}{0.000000in}}%
\pgfpathlineto{\pgfqpoint{0.000000in}{-0.048611in}}%
\pgfusepath{stroke,fill}%
}%
\begin{pgfscope}%
\pgfsys@transformshift{1.735553in}{0.500000in}%
\pgfsys@useobject{currentmarker}{}%
\end{pgfscope}%
\end{pgfscope}%
\begin{pgfscope}%
\definecolor{textcolor}{rgb}{0.000000,0.000000,0.000000}%
\pgfsetstrokecolor{textcolor}%
\pgfsetfillcolor{textcolor}%
\pgftext[x=1.735553in,y=0.402778in,,top]{\color{textcolor}\rmfamily\fontsize{13.000000}{15.600000}\selectfont \(\displaystyle {0.2}\)}%
\end{pgfscope}%
\begin{pgfscope}%
\pgfsetbuttcap%
\pgfsetroundjoin%
\definecolor{currentfill}{rgb}{0.000000,0.000000,0.000000}%
\pgfsetfillcolor{currentfill}%
\pgfsetlinewidth{0.803000pt}%
\definecolor{currentstroke}{rgb}{0.000000,0.000000,0.000000}%
\pgfsetstrokecolor{currentstroke}%
\pgfsetdash{}{0pt}%
\pgfsys@defobject{currentmarker}{\pgfqpoint{0.000000in}{-0.048611in}}{\pgfqpoint{0.000000in}{0.000000in}}{%
\pgfpathmoveto{\pgfqpoint{0.000000in}{0.000000in}}%
\pgfpathlineto{\pgfqpoint{0.000000in}{-0.048611in}}%
\pgfusepath{stroke,fill}%
}%
\begin{pgfscope}%
\pgfsys@transformshift{2.887834in}{0.500000in}%
\pgfsys@useobject{currentmarker}{}%
\end{pgfscope}%
\end{pgfscope}%
\begin{pgfscope}%
\definecolor{textcolor}{rgb}{0.000000,0.000000,0.000000}%
\pgfsetstrokecolor{textcolor}%
\pgfsetfillcolor{textcolor}%
\pgftext[x=2.887834in,y=0.402778in,,top]{\color{textcolor}\rmfamily\fontsize{13.000000}{15.600000}\selectfont \(\displaystyle {0.3}\)}%
\end{pgfscope}%
\begin{pgfscope}%
\pgfsetbuttcap%
\pgfsetroundjoin%
\definecolor{currentfill}{rgb}{0.000000,0.000000,0.000000}%
\pgfsetfillcolor{currentfill}%
\pgfsetlinewidth{0.803000pt}%
\definecolor{currentstroke}{rgb}{0.000000,0.000000,0.000000}%
\pgfsetstrokecolor{currentstroke}%
\pgfsetdash{}{0pt}%
\pgfsys@defobject{currentmarker}{\pgfqpoint{0.000000in}{-0.048611in}}{\pgfqpoint{0.000000in}{0.000000in}}{%
\pgfpathmoveto{\pgfqpoint{0.000000in}{0.000000in}}%
\pgfpathlineto{\pgfqpoint{0.000000in}{-0.048611in}}%
\pgfusepath{stroke,fill}%
}%
\begin{pgfscope}%
\pgfsys@transformshift{4.040115in}{0.500000in}%
\pgfsys@useobject{currentmarker}{}%
\end{pgfscope}%
\end{pgfscope}%
\begin{pgfscope}%
\definecolor{textcolor}{rgb}{0.000000,0.000000,0.000000}%
\pgfsetstrokecolor{textcolor}%
\pgfsetfillcolor{textcolor}%
\pgftext[x=4.040115in,y=0.402778in,,top]{\color{textcolor}\rmfamily\fontsize{13.000000}{15.600000}\selectfont \(\displaystyle {0.4}\)}%
\end{pgfscope}%
\begin{pgfscope}%
\pgfsetbuttcap%
\pgfsetroundjoin%
\definecolor{currentfill}{rgb}{0.000000,0.000000,0.000000}%
\pgfsetfillcolor{currentfill}%
\pgfsetlinewidth{0.803000pt}%
\definecolor{currentstroke}{rgb}{0.000000,0.000000,0.000000}%
\pgfsetstrokecolor{currentstroke}%
\pgfsetdash{}{0pt}%
\pgfsys@defobject{currentmarker}{\pgfqpoint{0.000000in}{-0.048611in}}{\pgfqpoint{0.000000in}{0.000000in}}{%
\pgfpathmoveto{\pgfqpoint{0.000000in}{0.000000in}}%
\pgfpathlineto{\pgfqpoint{0.000000in}{-0.048611in}}%
\pgfusepath{stroke,fill}%
}%
\begin{pgfscope}%
\pgfsys@transformshift{5.192396in}{0.500000in}%
\pgfsys@useobject{currentmarker}{}%
\end{pgfscope}%
\end{pgfscope}%
\begin{pgfscope}%
\definecolor{textcolor}{rgb}{0.000000,0.000000,0.000000}%
\pgfsetstrokecolor{textcolor}%
\pgfsetfillcolor{textcolor}%
\pgftext[x=5.192396in,y=0.402778in,,top]{\color{textcolor}\rmfamily\fontsize{13.000000}{15.600000}\selectfont \(\displaystyle {0.5}\)}%
\end{pgfscope}%
\begin{pgfscope}%
\definecolor{textcolor}{rgb}{0.000000,0.000000,0.000000}%
\pgfsetstrokecolor{textcolor}%
\pgfsetfillcolor{textcolor}%
\pgftext[x=3.075000in,y=0.199075in,,top]{\color{textcolor}\rmfamily\fontsize{13.000000}{15.600000}\selectfont Loss}%
\end{pgfscope}%
\begin{pgfscope}%
\pgfsetbuttcap%
\pgfsetroundjoin%
\definecolor{currentfill}{rgb}{0.000000,0.000000,0.000000}%
\pgfsetfillcolor{currentfill}%
\pgfsetlinewidth{0.803000pt}%
\definecolor{currentstroke}{rgb}{0.000000,0.000000,0.000000}%
\pgfsetstrokecolor{currentstroke}%
\pgfsetdash{}{0pt}%
\pgfsys@defobject{currentmarker}{\pgfqpoint{-0.048611in}{0.000000in}}{\pgfqpoint{-0.000000in}{0.000000in}}{%
\pgfpathmoveto{\pgfqpoint{-0.000000in}{0.000000in}}%
\pgfpathlineto{\pgfqpoint{-0.048611in}{0.000000in}}%
\pgfusepath{stroke,fill}%
}%
\begin{pgfscope}%
\pgfsys@transformshift{0.750000in}{0.500000in}%
\pgfsys@useobject{currentmarker}{}%
\end{pgfscope}%
\end{pgfscope}%
\begin{pgfscope}%
\definecolor{textcolor}{rgb}{0.000000,0.000000,0.000000}%
\pgfsetstrokecolor{textcolor}%
\pgfsetfillcolor{textcolor}%
\pgftext[x=0.571181in, y=0.442130in, left, base]{\color{textcolor}\rmfamily\fontsize{13.000000}{15.600000}\selectfont \(\displaystyle {0}\)}%
\end{pgfscope}%
\begin{pgfscope}%
\pgfsetbuttcap%
\pgfsetroundjoin%
\definecolor{currentfill}{rgb}{0.000000,0.000000,0.000000}%
\pgfsetfillcolor{currentfill}%
\pgfsetlinewidth{0.803000pt}%
\definecolor{currentstroke}{rgb}{0.000000,0.000000,0.000000}%
\pgfsetstrokecolor{currentstroke}%
\pgfsetdash{}{0pt}%
\pgfsys@defobject{currentmarker}{\pgfqpoint{-0.048611in}{0.000000in}}{\pgfqpoint{-0.000000in}{0.000000in}}{%
\pgfpathmoveto{\pgfqpoint{-0.000000in}{0.000000in}}%
\pgfpathlineto{\pgfqpoint{-0.048611in}{0.000000in}}%
\pgfusepath{stroke,fill}%
}%
\begin{pgfscope}%
\pgfsys@transformshift{0.750000in}{1.104242in}%
\pgfsys@useobject{currentmarker}{}%
\end{pgfscope}%
\end{pgfscope}%
\begin{pgfscope}%
\definecolor{textcolor}{rgb}{0.000000,0.000000,0.000000}%
\pgfsetstrokecolor{textcolor}%
\pgfsetfillcolor{textcolor}%
\pgftext[x=0.407989in, y=1.046371in, left, base]{\color{textcolor}\rmfamily\fontsize{13.000000}{15.600000}\selectfont \(\displaystyle {200}\)}%
\end{pgfscope}%
\begin{pgfscope}%
\pgfsetbuttcap%
\pgfsetroundjoin%
\definecolor{currentfill}{rgb}{0.000000,0.000000,0.000000}%
\pgfsetfillcolor{currentfill}%
\pgfsetlinewidth{0.803000pt}%
\definecolor{currentstroke}{rgb}{0.000000,0.000000,0.000000}%
\pgfsetstrokecolor{currentstroke}%
\pgfsetdash{}{0pt}%
\pgfsys@defobject{currentmarker}{\pgfqpoint{-0.048611in}{0.000000in}}{\pgfqpoint{-0.000000in}{0.000000in}}{%
\pgfpathmoveto{\pgfqpoint{-0.000000in}{0.000000in}}%
\pgfpathlineto{\pgfqpoint{-0.048611in}{0.000000in}}%
\pgfusepath{stroke,fill}%
}%
\begin{pgfscope}%
\pgfsys@transformshift{0.750000in}{1.708483in}%
\pgfsys@useobject{currentmarker}{}%
\end{pgfscope}%
\end{pgfscope}%
\begin{pgfscope}%
\definecolor{textcolor}{rgb}{0.000000,0.000000,0.000000}%
\pgfsetstrokecolor{textcolor}%
\pgfsetfillcolor{textcolor}%
\pgftext[x=0.407989in, y=1.650613in, left, base]{\color{textcolor}\rmfamily\fontsize{13.000000}{15.600000}\selectfont \(\displaystyle {400}\)}%
\end{pgfscope}%
\begin{pgfscope}%
\pgfsetbuttcap%
\pgfsetroundjoin%
\definecolor{currentfill}{rgb}{0.000000,0.000000,0.000000}%
\pgfsetfillcolor{currentfill}%
\pgfsetlinewidth{0.803000pt}%
\definecolor{currentstroke}{rgb}{0.000000,0.000000,0.000000}%
\pgfsetstrokecolor{currentstroke}%
\pgfsetdash{}{0pt}%
\pgfsys@defobject{currentmarker}{\pgfqpoint{-0.048611in}{0.000000in}}{\pgfqpoint{-0.000000in}{0.000000in}}{%
\pgfpathmoveto{\pgfqpoint{-0.000000in}{0.000000in}}%
\pgfpathlineto{\pgfqpoint{-0.048611in}{0.000000in}}%
\pgfusepath{stroke,fill}%
}%
\begin{pgfscope}%
\pgfsys@transformshift{0.750000in}{2.312725in}%
\pgfsys@useobject{currentmarker}{}%
\end{pgfscope}%
\end{pgfscope}%
\begin{pgfscope}%
\definecolor{textcolor}{rgb}{0.000000,0.000000,0.000000}%
\pgfsetstrokecolor{textcolor}%
\pgfsetfillcolor{textcolor}%
\pgftext[x=0.407989in, y=2.254855in, left, base]{\color{textcolor}\rmfamily\fontsize{13.000000}{15.600000}\selectfont \(\displaystyle {600}\)}%
\end{pgfscope}%
\begin{pgfscope}%
\pgfsetbuttcap%
\pgfsetroundjoin%
\definecolor{currentfill}{rgb}{0.000000,0.000000,0.000000}%
\pgfsetfillcolor{currentfill}%
\pgfsetlinewidth{0.803000pt}%
\definecolor{currentstroke}{rgb}{0.000000,0.000000,0.000000}%
\pgfsetstrokecolor{currentstroke}%
\pgfsetdash{}{0pt}%
\pgfsys@defobject{currentmarker}{\pgfqpoint{-0.048611in}{0.000000in}}{\pgfqpoint{-0.000000in}{0.000000in}}{%
\pgfpathmoveto{\pgfqpoint{-0.000000in}{0.000000in}}%
\pgfpathlineto{\pgfqpoint{-0.048611in}{0.000000in}}%
\pgfusepath{stroke,fill}%
}%
\begin{pgfscope}%
\pgfsys@transformshift{0.750000in}{2.916967in}%
\pgfsys@useobject{currentmarker}{}%
\end{pgfscope}%
\end{pgfscope}%
\begin{pgfscope}%
\definecolor{textcolor}{rgb}{0.000000,0.000000,0.000000}%
\pgfsetstrokecolor{textcolor}%
\pgfsetfillcolor{textcolor}%
\pgftext[x=0.407989in, y=2.859097in, left, base]{\color{textcolor}\rmfamily\fontsize{13.000000}{15.600000}\selectfont \(\displaystyle {800}\)}%
\end{pgfscope}%
\begin{pgfscope}%
\definecolor{textcolor}{rgb}{0.000000,0.000000,0.000000}%
\pgfsetstrokecolor{textcolor}%
\pgfsetfillcolor{textcolor}%
\pgftext[x=0.352433in,y=2.010000in,,bottom,rotate=90.000000]{\color{textcolor}\rmfamily\fontsize{13.000000}{15.600000}\selectfont Count}%
\end{pgfscope}%
\begin{pgfscope}%
\pgfsetrectcap%
\pgfsetmiterjoin%
\pgfsetlinewidth{0.803000pt}%
\definecolor{currentstroke}{rgb}{0.000000,0.000000,0.000000}%
\pgfsetstrokecolor{currentstroke}%
\pgfsetdash{}{0pt}%
\pgfpathmoveto{\pgfqpoint{0.750000in}{0.500000in}}%
\pgfpathlineto{\pgfqpoint{0.750000in}{3.520000in}}%
\pgfusepath{stroke}%
\end{pgfscope}%
\begin{pgfscope}%
\pgfsetrectcap%
\pgfsetmiterjoin%
\pgfsetlinewidth{0.803000pt}%
\definecolor{currentstroke}{rgb}{0.000000,0.000000,0.000000}%
\pgfsetstrokecolor{currentstroke}%
\pgfsetdash{}{0pt}%
\pgfpathmoveto{\pgfqpoint{5.400000in}{0.500000in}}%
\pgfpathlineto{\pgfqpoint{5.400000in}{3.520000in}}%
\pgfusepath{stroke}%
\end{pgfscope}%
\begin{pgfscope}%
\pgfsetrectcap%
\pgfsetmiterjoin%
\pgfsetlinewidth{0.803000pt}%
\definecolor{currentstroke}{rgb}{0.000000,0.000000,0.000000}%
\pgfsetstrokecolor{currentstroke}%
\pgfsetdash{}{0pt}%
\pgfpathmoveto{\pgfqpoint{0.750000in}{0.500000in}}%
\pgfpathlineto{\pgfqpoint{5.400000in}{0.500000in}}%
\pgfusepath{stroke}%
\end{pgfscope}%
\begin{pgfscope}%
\pgfsetrectcap%
\pgfsetmiterjoin%
\pgfsetlinewidth{0.803000pt}%
\definecolor{currentstroke}{rgb}{0.000000,0.000000,0.000000}%
\pgfsetstrokecolor{currentstroke}%
\pgfsetdash{}{0pt}%
\pgfpathmoveto{\pgfqpoint{0.750000in}{3.520000in}}%
\pgfpathlineto{\pgfqpoint{5.400000in}{3.520000in}}%
\pgfusepath{stroke}%
\end{pgfscope}%
\begin{pgfscope}%
\definecolor{textcolor}{rgb}{0.000000,0.000000,0.000000}%
\pgfsetstrokecolor{textcolor}%
\pgfsetfillcolor{textcolor}%
\pgftext[x=3.075000in,y=3.603333in,,base]{\color{textcolor}\rmfamily\fontsize{13.000000}{15.600000}\selectfont Loss Histogram for \(\displaystyle f(x)=2x\)}%
\end{pgfscope}%
\begin{pgfscope}%
\pgfsetbuttcap%
\pgfsetmiterjoin%
\definecolor{currentfill}{rgb}{1.000000,1.000000,1.000000}%
\pgfsetfillcolor{currentfill}%
\pgfsetfillopacity{0.800000}%
\pgfsetlinewidth{1.003750pt}%
\definecolor{currentstroke}{rgb}{0.800000,0.800000,0.800000}%
\pgfsetstrokecolor{currentstroke}%
\pgfsetstrokeopacity{0.800000}%
\pgfsetdash{}{0pt}%
\pgfpathmoveto{\pgfqpoint{4.360497in}{3.126482in}}%
\pgfpathlineto{\pgfqpoint{5.273611in}{3.126482in}}%
\pgfpathquadraticcurveto{\pgfqpoint{5.309722in}{3.126482in}}{\pgfqpoint{5.309722in}{3.162593in}}%
\pgfpathlineto{\pgfqpoint{5.309722in}{3.393611in}}%
\pgfpathquadraticcurveto{\pgfqpoint{5.309722in}{3.429722in}}{\pgfqpoint{5.273611in}{3.429722in}}%
\pgfpathlineto{\pgfqpoint{4.360497in}{3.429722in}}%
\pgfpathquadraticcurveto{\pgfqpoint{4.324386in}{3.429722in}}{\pgfqpoint{4.324386in}{3.393611in}}%
\pgfpathlineto{\pgfqpoint{4.324386in}{3.162593in}}%
\pgfpathquadraticcurveto{\pgfqpoint{4.324386in}{3.126482in}}{\pgfqpoint{4.360497in}{3.126482in}}%
\pgfpathlineto{\pgfqpoint{4.360497in}{3.126482in}}%
\pgfpathclose%
\pgfusepath{stroke,fill}%
\end{pgfscope}%
\begin{pgfscope}%
\pgfsetbuttcap%
\pgfsetmiterjoin%
\definecolor{currentfill}{rgb}{1.000000,0.000000,0.000000}%
\pgfsetfillcolor{currentfill}%
\pgfsetlinewidth{0.000000pt}%
\definecolor{currentstroke}{rgb}{0.000000,0.000000,0.000000}%
\pgfsetstrokecolor{currentstroke}%
\pgfsetstrokeopacity{0.000000}%
\pgfsetdash{}{0pt}%
\pgfpathmoveto{\pgfqpoint{4.396608in}{3.231111in}}%
\pgfpathlineto{\pgfqpoint{4.757719in}{3.231111in}}%
\pgfpathlineto{\pgfqpoint{4.757719in}{3.357500in}}%
\pgfpathlineto{\pgfqpoint{4.396608in}{3.357500in}}%
\pgfpathlineto{\pgfqpoint{4.396608in}{3.231111in}}%
\pgfpathclose%
\pgfusepath{fill}%
\end{pgfscope}%
\begin{pgfscope}%
\definecolor{textcolor}{rgb}{0.000000,0.000000,0.000000}%
\pgfsetstrokecolor{textcolor}%
\pgfsetfillcolor{textcolor}%
\pgftext[x=4.902164in,y=3.231111in,left,base]{\color{textcolor}\rmfamily\fontsize{13.000000}{15.600000}\selectfont SNN}%
\end{pgfscope}%
\end{pgfpicture}%
\makeatother%
\endgroup%

    \caption{Caption}
    \label{fig:my_label}
\end{figure}

\begin{figure}
\input{./plots/2x_nn_loss_histogram.pgf}
    \caption{Caption}
    \label{fig:my_label}
\end{figure}

\begin{figure}
%% Creator: Matplotlib, PGF backend
%%
%% To include the figure in your LaTeX document, write
%%   \input{<filename>.pgf}
%%
%% Make sure the required packages are loaded in your preamble
%%   \usepackage{pgf}
%%
%% Also ensure that all the required font packages are loaded; for instance,
%% the lmodern package is sometimes necessary when using math font.
%%   \usepackage{lmodern}
%%
%% Figures using additional raster images can only be included by \input if
%% they are in the same directory as the main LaTeX file. For loading figures
%% from other directories you can use the `import` package
%%   \usepackage{import}
%%
%% and then include the figures with
%%   \import{<path to file>}{<filename>.pgf}
%%
%% Matplotlib used the following preamble
%%
\begingroup%
\makeatletter%
\begin{pgfpicture}%
\pgfpathrectangle{\pgfpointorigin}{\pgfqpoint{6.000000in}{4.000000in}}%
\pgfusepath{use as bounding box, clip}%
\begin{pgfscope}%
\pgfsetbuttcap%
\pgfsetmiterjoin%
\pgfsetlinewidth{0.000000pt}%
\definecolor{currentstroke}{rgb}{1.000000,1.000000,1.000000}%
\pgfsetstrokecolor{currentstroke}%
\pgfsetstrokeopacity{0.000000}%
\pgfsetdash{}{0pt}%
\pgfpathmoveto{\pgfqpoint{0.000000in}{0.000000in}}%
\pgfpathlineto{\pgfqpoint{6.000000in}{0.000000in}}%
\pgfpathlineto{\pgfqpoint{6.000000in}{4.000000in}}%
\pgfpathlineto{\pgfqpoint{0.000000in}{4.000000in}}%
\pgfpathlineto{\pgfqpoint{0.000000in}{0.000000in}}%
\pgfpathclose%
\pgfusepath{}%
\end{pgfscope}%
\begin{pgfscope}%
\pgfsetbuttcap%
\pgfsetmiterjoin%
\definecolor{currentfill}{rgb}{1.000000,1.000000,1.000000}%
\pgfsetfillcolor{currentfill}%
\pgfsetlinewidth{0.000000pt}%
\definecolor{currentstroke}{rgb}{0.000000,0.000000,0.000000}%
\pgfsetstrokecolor{currentstroke}%
\pgfsetstrokeopacity{0.000000}%
\pgfsetdash{}{0pt}%
\pgfpathmoveto{\pgfqpoint{0.750000in}{0.500000in}}%
\pgfpathlineto{\pgfqpoint{5.400000in}{0.500000in}}%
\pgfpathlineto{\pgfqpoint{5.400000in}{3.520000in}}%
\pgfpathlineto{\pgfqpoint{0.750000in}{3.520000in}}%
\pgfpathlineto{\pgfqpoint{0.750000in}{0.500000in}}%
\pgfpathclose%
\pgfusepath{fill}%
\end{pgfscope}%
\begin{pgfscope}%
\pgfpathrectangle{\pgfqpoint{0.750000in}{0.500000in}}{\pgfqpoint{4.650000in}{3.020000in}}%
\pgfusepath{clip}%
\pgfsetbuttcap%
\pgfsetmiterjoin%
\definecolor{currentfill}{rgb}{1.000000,0.000000,0.000000}%
\pgfsetfillcolor{currentfill}%
\pgfsetlinewidth{0.000000pt}%
\definecolor{currentstroke}{rgb}{0.000000,0.000000,0.000000}%
\pgfsetstrokecolor{currentstroke}%
\pgfsetstrokeopacity{0.000000}%
\pgfsetdash{}{0pt}%
\pgfpathmoveto{\pgfqpoint{2.070332in}{0.500000in}}%
\pgfpathlineto{\pgfqpoint{2.094610in}{0.500000in}}%
\pgfpathlineto{\pgfqpoint{2.094610in}{0.506042in}}%
\pgfpathlineto{\pgfqpoint{2.070332in}{0.506042in}}%
\pgfpathlineto{\pgfqpoint{2.070332in}{0.500000in}}%
\pgfpathclose%
\pgfusepath{fill}%
\end{pgfscope}%
\begin{pgfscope}%
\pgfpathrectangle{\pgfqpoint{0.750000in}{0.500000in}}{\pgfqpoint{4.650000in}{3.020000in}}%
\pgfusepath{clip}%
\pgfsetbuttcap%
\pgfsetmiterjoin%
\definecolor{currentfill}{rgb}{1.000000,0.000000,0.000000}%
\pgfsetfillcolor{currentfill}%
\pgfsetlinewidth{0.000000pt}%
\definecolor{currentstroke}{rgb}{0.000000,0.000000,0.000000}%
\pgfsetstrokecolor{currentstroke}%
\pgfsetstrokeopacity{0.000000}%
\pgfsetdash{}{0pt}%
\pgfpathmoveto{\pgfqpoint{2.094610in}{0.500000in}}%
\pgfpathlineto{\pgfqpoint{2.118889in}{0.500000in}}%
\pgfpathlineto{\pgfqpoint{2.118889in}{0.500000in}}%
\pgfpathlineto{\pgfqpoint{2.094610in}{0.500000in}}%
\pgfpathlineto{\pgfqpoint{2.094610in}{0.500000in}}%
\pgfpathclose%
\pgfusepath{fill}%
\end{pgfscope}%
\begin{pgfscope}%
\pgfpathrectangle{\pgfqpoint{0.750000in}{0.500000in}}{\pgfqpoint{4.650000in}{3.020000in}}%
\pgfusepath{clip}%
\pgfsetbuttcap%
\pgfsetmiterjoin%
\definecolor{currentfill}{rgb}{1.000000,0.000000,0.000000}%
\pgfsetfillcolor{currentfill}%
\pgfsetlinewidth{0.000000pt}%
\definecolor{currentstroke}{rgb}{0.000000,0.000000,0.000000}%
\pgfsetstrokecolor{currentstroke}%
\pgfsetstrokeopacity{0.000000}%
\pgfsetdash{}{0pt}%
\pgfpathmoveto{\pgfqpoint{2.118889in}{0.500000in}}%
\pgfpathlineto{\pgfqpoint{2.143168in}{0.500000in}}%
\pgfpathlineto{\pgfqpoint{2.143168in}{0.500000in}}%
\pgfpathlineto{\pgfqpoint{2.118889in}{0.500000in}}%
\pgfpathlineto{\pgfqpoint{2.118889in}{0.500000in}}%
\pgfpathclose%
\pgfusepath{fill}%
\end{pgfscope}%
\begin{pgfscope}%
\pgfpathrectangle{\pgfqpoint{0.750000in}{0.500000in}}{\pgfqpoint{4.650000in}{3.020000in}}%
\pgfusepath{clip}%
\pgfsetbuttcap%
\pgfsetmiterjoin%
\definecolor{currentfill}{rgb}{1.000000,0.000000,0.000000}%
\pgfsetfillcolor{currentfill}%
\pgfsetlinewidth{0.000000pt}%
\definecolor{currentstroke}{rgb}{0.000000,0.000000,0.000000}%
\pgfsetstrokecolor{currentstroke}%
\pgfsetstrokeopacity{0.000000}%
\pgfsetdash{}{0pt}%
\pgfpathmoveto{\pgfqpoint{2.143168in}{0.500000in}}%
\pgfpathlineto{\pgfqpoint{2.167446in}{0.500000in}}%
\pgfpathlineto{\pgfqpoint{2.167446in}{0.503021in}}%
\pgfpathlineto{\pgfqpoint{2.143168in}{0.503021in}}%
\pgfpathlineto{\pgfqpoint{2.143168in}{0.500000in}}%
\pgfpathclose%
\pgfusepath{fill}%
\end{pgfscope}%
\begin{pgfscope}%
\pgfpathrectangle{\pgfqpoint{0.750000in}{0.500000in}}{\pgfqpoint{4.650000in}{3.020000in}}%
\pgfusepath{clip}%
\pgfsetbuttcap%
\pgfsetmiterjoin%
\definecolor{currentfill}{rgb}{1.000000,0.000000,0.000000}%
\pgfsetfillcolor{currentfill}%
\pgfsetlinewidth{0.000000pt}%
\definecolor{currentstroke}{rgb}{0.000000,0.000000,0.000000}%
\pgfsetstrokecolor{currentstroke}%
\pgfsetstrokeopacity{0.000000}%
\pgfsetdash{}{0pt}%
\pgfpathmoveto{\pgfqpoint{2.167446in}{0.500000in}}%
\pgfpathlineto{\pgfqpoint{2.191725in}{0.500000in}}%
\pgfpathlineto{\pgfqpoint{2.191725in}{0.500000in}}%
\pgfpathlineto{\pgfqpoint{2.167446in}{0.500000in}}%
\pgfpathlineto{\pgfqpoint{2.167446in}{0.500000in}}%
\pgfpathclose%
\pgfusepath{fill}%
\end{pgfscope}%
\begin{pgfscope}%
\pgfpathrectangle{\pgfqpoint{0.750000in}{0.500000in}}{\pgfqpoint{4.650000in}{3.020000in}}%
\pgfusepath{clip}%
\pgfsetbuttcap%
\pgfsetmiterjoin%
\definecolor{currentfill}{rgb}{1.000000,0.000000,0.000000}%
\pgfsetfillcolor{currentfill}%
\pgfsetlinewidth{0.000000pt}%
\definecolor{currentstroke}{rgb}{0.000000,0.000000,0.000000}%
\pgfsetstrokecolor{currentstroke}%
\pgfsetstrokeopacity{0.000000}%
\pgfsetdash{}{0pt}%
\pgfpathmoveto{\pgfqpoint{2.191725in}{0.500000in}}%
\pgfpathlineto{\pgfqpoint{2.216003in}{0.500000in}}%
\pgfpathlineto{\pgfqpoint{2.216003in}{0.509064in}}%
\pgfpathlineto{\pgfqpoint{2.191725in}{0.509064in}}%
\pgfpathlineto{\pgfqpoint{2.191725in}{0.500000in}}%
\pgfpathclose%
\pgfusepath{fill}%
\end{pgfscope}%
\begin{pgfscope}%
\pgfpathrectangle{\pgfqpoint{0.750000in}{0.500000in}}{\pgfqpoint{4.650000in}{3.020000in}}%
\pgfusepath{clip}%
\pgfsetbuttcap%
\pgfsetmiterjoin%
\definecolor{currentfill}{rgb}{1.000000,0.000000,0.000000}%
\pgfsetfillcolor{currentfill}%
\pgfsetlinewidth{0.000000pt}%
\definecolor{currentstroke}{rgb}{0.000000,0.000000,0.000000}%
\pgfsetstrokecolor{currentstroke}%
\pgfsetstrokeopacity{0.000000}%
\pgfsetdash{}{0pt}%
\pgfpathmoveto{\pgfqpoint{2.216003in}{0.500000in}}%
\pgfpathlineto{\pgfqpoint{2.240282in}{0.500000in}}%
\pgfpathlineto{\pgfqpoint{2.240282in}{0.500000in}}%
\pgfpathlineto{\pgfqpoint{2.216003in}{0.500000in}}%
\pgfpathlineto{\pgfqpoint{2.216003in}{0.500000in}}%
\pgfpathclose%
\pgfusepath{fill}%
\end{pgfscope}%
\begin{pgfscope}%
\pgfpathrectangle{\pgfqpoint{0.750000in}{0.500000in}}{\pgfqpoint{4.650000in}{3.020000in}}%
\pgfusepath{clip}%
\pgfsetbuttcap%
\pgfsetmiterjoin%
\definecolor{currentfill}{rgb}{1.000000,0.000000,0.000000}%
\pgfsetfillcolor{currentfill}%
\pgfsetlinewidth{0.000000pt}%
\definecolor{currentstroke}{rgb}{0.000000,0.000000,0.000000}%
\pgfsetstrokecolor{currentstroke}%
\pgfsetstrokeopacity{0.000000}%
\pgfsetdash{}{0pt}%
\pgfpathmoveto{\pgfqpoint{2.240282in}{0.500000in}}%
\pgfpathlineto{\pgfqpoint{2.264561in}{0.500000in}}%
\pgfpathlineto{\pgfqpoint{2.264561in}{0.500000in}}%
\pgfpathlineto{\pgfqpoint{2.240282in}{0.500000in}}%
\pgfpathlineto{\pgfqpoint{2.240282in}{0.500000in}}%
\pgfpathclose%
\pgfusepath{fill}%
\end{pgfscope}%
\begin{pgfscope}%
\pgfpathrectangle{\pgfqpoint{0.750000in}{0.500000in}}{\pgfqpoint{4.650000in}{3.020000in}}%
\pgfusepath{clip}%
\pgfsetbuttcap%
\pgfsetmiterjoin%
\definecolor{currentfill}{rgb}{1.000000,0.000000,0.000000}%
\pgfsetfillcolor{currentfill}%
\pgfsetlinewidth{0.000000pt}%
\definecolor{currentstroke}{rgb}{0.000000,0.000000,0.000000}%
\pgfsetstrokecolor{currentstroke}%
\pgfsetstrokeopacity{0.000000}%
\pgfsetdash{}{0pt}%
\pgfpathmoveto{\pgfqpoint{2.264561in}{0.500000in}}%
\pgfpathlineto{\pgfqpoint{2.288839in}{0.500000in}}%
\pgfpathlineto{\pgfqpoint{2.288839in}{0.536255in}}%
\pgfpathlineto{\pgfqpoint{2.264561in}{0.536255in}}%
\pgfpathlineto{\pgfqpoint{2.264561in}{0.500000in}}%
\pgfpathclose%
\pgfusepath{fill}%
\end{pgfscope}%
\begin{pgfscope}%
\pgfpathrectangle{\pgfqpoint{0.750000in}{0.500000in}}{\pgfqpoint{4.650000in}{3.020000in}}%
\pgfusepath{clip}%
\pgfsetbuttcap%
\pgfsetmiterjoin%
\definecolor{currentfill}{rgb}{1.000000,0.000000,0.000000}%
\pgfsetfillcolor{currentfill}%
\pgfsetlinewidth{0.000000pt}%
\definecolor{currentstroke}{rgb}{0.000000,0.000000,0.000000}%
\pgfsetstrokecolor{currentstroke}%
\pgfsetstrokeopacity{0.000000}%
\pgfsetdash{}{0pt}%
\pgfpathmoveto{\pgfqpoint{2.288839in}{0.500000in}}%
\pgfpathlineto{\pgfqpoint{2.313118in}{0.500000in}}%
\pgfpathlineto{\pgfqpoint{2.313118in}{0.500000in}}%
\pgfpathlineto{\pgfqpoint{2.288839in}{0.500000in}}%
\pgfpathlineto{\pgfqpoint{2.288839in}{0.500000in}}%
\pgfpathclose%
\pgfusepath{fill}%
\end{pgfscope}%
\begin{pgfscope}%
\pgfpathrectangle{\pgfqpoint{0.750000in}{0.500000in}}{\pgfqpoint{4.650000in}{3.020000in}}%
\pgfusepath{clip}%
\pgfsetbuttcap%
\pgfsetmiterjoin%
\definecolor{currentfill}{rgb}{1.000000,0.000000,0.000000}%
\pgfsetfillcolor{currentfill}%
\pgfsetlinewidth{0.000000pt}%
\definecolor{currentstroke}{rgb}{0.000000,0.000000,0.000000}%
\pgfsetstrokecolor{currentstroke}%
\pgfsetstrokeopacity{0.000000}%
\pgfsetdash{}{0pt}%
\pgfpathmoveto{\pgfqpoint{2.313118in}{0.500000in}}%
\pgfpathlineto{\pgfqpoint{2.337396in}{0.500000in}}%
\pgfpathlineto{\pgfqpoint{2.337396in}{0.566467in}}%
\pgfpathlineto{\pgfqpoint{2.313118in}{0.566467in}}%
\pgfpathlineto{\pgfqpoint{2.313118in}{0.500000in}}%
\pgfpathclose%
\pgfusepath{fill}%
\end{pgfscope}%
\begin{pgfscope}%
\pgfpathrectangle{\pgfqpoint{0.750000in}{0.500000in}}{\pgfqpoint{4.650000in}{3.020000in}}%
\pgfusepath{clip}%
\pgfsetbuttcap%
\pgfsetmiterjoin%
\definecolor{currentfill}{rgb}{1.000000,0.000000,0.000000}%
\pgfsetfillcolor{currentfill}%
\pgfsetlinewidth{0.000000pt}%
\definecolor{currentstroke}{rgb}{0.000000,0.000000,0.000000}%
\pgfsetstrokecolor{currentstroke}%
\pgfsetstrokeopacity{0.000000}%
\pgfsetdash{}{0pt}%
\pgfpathmoveto{\pgfqpoint{2.337396in}{0.500000in}}%
\pgfpathlineto{\pgfqpoint{2.361675in}{0.500000in}}%
\pgfpathlineto{\pgfqpoint{2.361675in}{0.503021in}}%
\pgfpathlineto{\pgfqpoint{2.337396in}{0.503021in}}%
\pgfpathlineto{\pgfqpoint{2.337396in}{0.500000in}}%
\pgfpathclose%
\pgfusepath{fill}%
\end{pgfscope}%
\begin{pgfscope}%
\pgfpathrectangle{\pgfqpoint{0.750000in}{0.500000in}}{\pgfqpoint{4.650000in}{3.020000in}}%
\pgfusepath{clip}%
\pgfsetbuttcap%
\pgfsetmiterjoin%
\definecolor{currentfill}{rgb}{1.000000,0.000000,0.000000}%
\pgfsetfillcolor{currentfill}%
\pgfsetlinewidth{0.000000pt}%
\definecolor{currentstroke}{rgb}{0.000000,0.000000,0.000000}%
\pgfsetstrokecolor{currentstroke}%
\pgfsetstrokeopacity{0.000000}%
\pgfsetdash{}{0pt}%
\pgfpathmoveto{\pgfqpoint{2.361675in}{0.500000in}}%
\pgfpathlineto{\pgfqpoint{2.385953in}{0.500000in}}%
\pgfpathlineto{\pgfqpoint{2.385953in}{0.512085in}}%
\pgfpathlineto{\pgfqpoint{2.361675in}{0.512085in}}%
\pgfpathlineto{\pgfqpoint{2.361675in}{0.500000in}}%
\pgfpathclose%
\pgfusepath{fill}%
\end{pgfscope}%
\begin{pgfscope}%
\pgfpathrectangle{\pgfqpoint{0.750000in}{0.500000in}}{\pgfqpoint{4.650000in}{3.020000in}}%
\pgfusepath{clip}%
\pgfsetbuttcap%
\pgfsetmiterjoin%
\definecolor{currentfill}{rgb}{1.000000,0.000000,0.000000}%
\pgfsetfillcolor{currentfill}%
\pgfsetlinewidth{0.000000pt}%
\definecolor{currentstroke}{rgb}{0.000000,0.000000,0.000000}%
\pgfsetstrokecolor{currentstroke}%
\pgfsetstrokeopacity{0.000000}%
\pgfsetdash{}{0pt}%
\pgfpathmoveto{\pgfqpoint{2.385953in}{0.500000in}}%
\pgfpathlineto{\pgfqpoint{2.410232in}{0.500000in}}%
\pgfpathlineto{\pgfqpoint{2.410232in}{0.611785in}}%
\pgfpathlineto{\pgfqpoint{2.385953in}{0.611785in}}%
\pgfpathlineto{\pgfqpoint{2.385953in}{0.500000in}}%
\pgfpathclose%
\pgfusepath{fill}%
\end{pgfscope}%
\begin{pgfscope}%
\pgfpathrectangle{\pgfqpoint{0.750000in}{0.500000in}}{\pgfqpoint{4.650000in}{3.020000in}}%
\pgfusepath{clip}%
\pgfsetbuttcap%
\pgfsetmiterjoin%
\definecolor{currentfill}{rgb}{1.000000,0.000000,0.000000}%
\pgfsetfillcolor{currentfill}%
\pgfsetlinewidth{0.000000pt}%
\definecolor{currentstroke}{rgb}{0.000000,0.000000,0.000000}%
\pgfsetstrokecolor{currentstroke}%
\pgfsetstrokeopacity{0.000000}%
\pgfsetdash{}{0pt}%
\pgfpathmoveto{\pgfqpoint{2.410232in}{0.500000in}}%
\pgfpathlineto{\pgfqpoint{2.434511in}{0.500000in}}%
\pgfpathlineto{\pgfqpoint{2.434511in}{0.506042in}}%
\pgfpathlineto{\pgfqpoint{2.410232in}{0.506042in}}%
\pgfpathlineto{\pgfqpoint{2.410232in}{0.500000in}}%
\pgfpathclose%
\pgfusepath{fill}%
\end{pgfscope}%
\begin{pgfscope}%
\pgfpathrectangle{\pgfqpoint{0.750000in}{0.500000in}}{\pgfqpoint{4.650000in}{3.020000in}}%
\pgfusepath{clip}%
\pgfsetbuttcap%
\pgfsetmiterjoin%
\definecolor{currentfill}{rgb}{1.000000,0.000000,0.000000}%
\pgfsetfillcolor{currentfill}%
\pgfsetlinewidth{0.000000pt}%
\definecolor{currentstroke}{rgb}{0.000000,0.000000,0.000000}%
\pgfsetstrokecolor{currentstroke}%
\pgfsetstrokeopacity{0.000000}%
\pgfsetdash{}{0pt}%
\pgfpathmoveto{\pgfqpoint{2.434511in}{0.500000in}}%
\pgfpathlineto{\pgfqpoint{2.458789in}{0.500000in}}%
\pgfpathlineto{\pgfqpoint{2.458789in}{0.503021in}}%
\pgfpathlineto{\pgfqpoint{2.434511in}{0.503021in}}%
\pgfpathlineto{\pgfqpoint{2.434511in}{0.500000in}}%
\pgfpathclose%
\pgfusepath{fill}%
\end{pgfscope}%
\begin{pgfscope}%
\pgfpathrectangle{\pgfqpoint{0.750000in}{0.500000in}}{\pgfqpoint{4.650000in}{3.020000in}}%
\pgfusepath{clip}%
\pgfsetbuttcap%
\pgfsetmiterjoin%
\definecolor{currentfill}{rgb}{1.000000,0.000000,0.000000}%
\pgfsetfillcolor{currentfill}%
\pgfsetlinewidth{0.000000pt}%
\definecolor{currentstroke}{rgb}{0.000000,0.000000,0.000000}%
\pgfsetstrokecolor{currentstroke}%
\pgfsetstrokeopacity{0.000000}%
\pgfsetdash{}{0pt}%
\pgfpathmoveto{\pgfqpoint{2.458789in}{0.500000in}}%
\pgfpathlineto{\pgfqpoint{2.483068in}{0.500000in}}%
\pgfpathlineto{\pgfqpoint{2.483068in}{0.759824in}}%
\pgfpathlineto{\pgfqpoint{2.458789in}{0.759824in}}%
\pgfpathlineto{\pgfqpoint{2.458789in}{0.500000in}}%
\pgfpathclose%
\pgfusepath{fill}%
\end{pgfscope}%
\begin{pgfscope}%
\pgfpathrectangle{\pgfqpoint{0.750000in}{0.500000in}}{\pgfqpoint{4.650000in}{3.020000in}}%
\pgfusepath{clip}%
\pgfsetbuttcap%
\pgfsetmiterjoin%
\definecolor{currentfill}{rgb}{1.000000,0.000000,0.000000}%
\pgfsetfillcolor{currentfill}%
\pgfsetlinewidth{0.000000pt}%
\definecolor{currentstroke}{rgb}{0.000000,0.000000,0.000000}%
\pgfsetstrokecolor{currentstroke}%
\pgfsetstrokeopacity{0.000000}%
\pgfsetdash{}{0pt}%
\pgfpathmoveto{\pgfqpoint{2.483068in}{0.500000in}}%
\pgfpathlineto{\pgfqpoint{2.507346in}{0.500000in}}%
\pgfpathlineto{\pgfqpoint{2.507346in}{0.527191in}}%
\pgfpathlineto{\pgfqpoint{2.483068in}{0.527191in}}%
\pgfpathlineto{\pgfqpoint{2.483068in}{0.500000in}}%
\pgfpathclose%
\pgfusepath{fill}%
\end{pgfscope}%
\begin{pgfscope}%
\pgfpathrectangle{\pgfqpoint{0.750000in}{0.500000in}}{\pgfqpoint{4.650000in}{3.020000in}}%
\pgfusepath{clip}%
\pgfsetbuttcap%
\pgfsetmiterjoin%
\definecolor{currentfill}{rgb}{1.000000,0.000000,0.000000}%
\pgfsetfillcolor{currentfill}%
\pgfsetlinewidth{0.000000pt}%
\definecolor{currentstroke}{rgb}{0.000000,0.000000,0.000000}%
\pgfsetstrokecolor{currentstroke}%
\pgfsetstrokeopacity{0.000000}%
\pgfsetdash{}{0pt}%
\pgfpathmoveto{\pgfqpoint{2.507346in}{0.500000in}}%
\pgfpathlineto{\pgfqpoint{2.531625in}{0.500000in}}%
\pgfpathlineto{\pgfqpoint{2.531625in}{0.527191in}}%
\pgfpathlineto{\pgfqpoint{2.507346in}{0.527191in}}%
\pgfpathlineto{\pgfqpoint{2.507346in}{0.500000in}}%
\pgfpathclose%
\pgfusepath{fill}%
\end{pgfscope}%
\begin{pgfscope}%
\pgfpathrectangle{\pgfqpoint{0.750000in}{0.500000in}}{\pgfqpoint{4.650000in}{3.020000in}}%
\pgfusepath{clip}%
\pgfsetbuttcap%
\pgfsetmiterjoin%
\definecolor{currentfill}{rgb}{1.000000,0.000000,0.000000}%
\pgfsetfillcolor{currentfill}%
\pgfsetlinewidth{0.000000pt}%
\definecolor{currentstroke}{rgb}{0.000000,0.000000,0.000000}%
\pgfsetstrokecolor{currentstroke}%
\pgfsetstrokeopacity{0.000000}%
\pgfsetdash{}{0pt}%
\pgfpathmoveto{\pgfqpoint{2.531625in}{0.500000in}}%
\pgfpathlineto{\pgfqpoint{2.555904in}{0.500000in}}%
\pgfpathlineto{\pgfqpoint{2.555904in}{0.880672in}}%
\pgfpathlineto{\pgfqpoint{2.531625in}{0.880672in}}%
\pgfpathlineto{\pgfqpoint{2.531625in}{0.500000in}}%
\pgfpathclose%
\pgfusepath{fill}%
\end{pgfscope}%
\begin{pgfscope}%
\pgfpathrectangle{\pgfqpoint{0.750000in}{0.500000in}}{\pgfqpoint{4.650000in}{3.020000in}}%
\pgfusepath{clip}%
\pgfsetbuttcap%
\pgfsetmiterjoin%
\definecolor{currentfill}{rgb}{1.000000,0.000000,0.000000}%
\pgfsetfillcolor{currentfill}%
\pgfsetlinewidth{0.000000pt}%
\definecolor{currentstroke}{rgb}{0.000000,0.000000,0.000000}%
\pgfsetstrokecolor{currentstroke}%
\pgfsetstrokeopacity{0.000000}%
\pgfsetdash{}{0pt}%
\pgfpathmoveto{\pgfqpoint{2.555904in}{0.500000in}}%
\pgfpathlineto{\pgfqpoint{2.580182in}{0.500000in}}%
\pgfpathlineto{\pgfqpoint{2.580182in}{0.512085in}}%
\pgfpathlineto{\pgfqpoint{2.555904in}{0.512085in}}%
\pgfpathlineto{\pgfqpoint{2.555904in}{0.500000in}}%
\pgfpathclose%
\pgfusepath{fill}%
\end{pgfscope}%
\begin{pgfscope}%
\pgfpathrectangle{\pgfqpoint{0.750000in}{0.500000in}}{\pgfqpoint{4.650000in}{3.020000in}}%
\pgfusepath{clip}%
\pgfsetbuttcap%
\pgfsetmiterjoin%
\definecolor{currentfill}{rgb}{1.000000,0.000000,0.000000}%
\pgfsetfillcolor{currentfill}%
\pgfsetlinewidth{0.000000pt}%
\definecolor{currentstroke}{rgb}{0.000000,0.000000,0.000000}%
\pgfsetstrokecolor{currentstroke}%
\pgfsetstrokeopacity{0.000000}%
\pgfsetdash{}{0pt}%
\pgfpathmoveto{\pgfqpoint{2.580182in}{0.500000in}}%
\pgfpathlineto{\pgfqpoint{2.604461in}{0.500000in}}%
\pgfpathlineto{\pgfqpoint{2.604461in}{1.071008in}}%
\pgfpathlineto{\pgfqpoint{2.580182in}{1.071008in}}%
\pgfpathlineto{\pgfqpoint{2.580182in}{0.500000in}}%
\pgfpathclose%
\pgfusepath{fill}%
\end{pgfscope}%
\begin{pgfscope}%
\pgfpathrectangle{\pgfqpoint{0.750000in}{0.500000in}}{\pgfqpoint{4.650000in}{3.020000in}}%
\pgfusepath{clip}%
\pgfsetbuttcap%
\pgfsetmiterjoin%
\definecolor{currentfill}{rgb}{1.000000,0.000000,0.000000}%
\pgfsetfillcolor{currentfill}%
\pgfsetlinewidth{0.000000pt}%
\definecolor{currentstroke}{rgb}{0.000000,0.000000,0.000000}%
\pgfsetstrokecolor{currentstroke}%
\pgfsetstrokeopacity{0.000000}%
\pgfsetdash{}{0pt}%
\pgfpathmoveto{\pgfqpoint{2.604461in}{0.500000in}}%
\pgfpathlineto{\pgfqpoint{2.628739in}{0.500000in}}%
\pgfpathlineto{\pgfqpoint{2.628739in}{0.521148in}}%
\pgfpathlineto{\pgfqpoint{2.604461in}{0.521148in}}%
\pgfpathlineto{\pgfqpoint{2.604461in}{0.500000in}}%
\pgfpathclose%
\pgfusepath{fill}%
\end{pgfscope}%
\begin{pgfscope}%
\pgfpathrectangle{\pgfqpoint{0.750000in}{0.500000in}}{\pgfqpoint{4.650000in}{3.020000in}}%
\pgfusepath{clip}%
\pgfsetbuttcap%
\pgfsetmiterjoin%
\definecolor{currentfill}{rgb}{1.000000,0.000000,0.000000}%
\pgfsetfillcolor{currentfill}%
\pgfsetlinewidth{0.000000pt}%
\definecolor{currentstroke}{rgb}{0.000000,0.000000,0.000000}%
\pgfsetstrokecolor{currentstroke}%
\pgfsetstrokeopacity{0.000000}%
\pgfsetdash{}{0pt}%
\pgfpathmoveto{\pgfqpoint{2.628739in}{0.500000in}}%
\pgfpathlineto{\pgfqpoint{2.653018in}{0.500000in}}%
\pgfpathlineto{\pgfqpoint{2.653018in}{0.524170in}}%
\pgfpathlineto{\pgfqpoint{2.628739in}{0.524170in}}%
\pgfpathlineto{\pgfqpoint{2.628739in}{0.500000in}}%
\pgfpathclose%
\pgfusepath{fill}%
\end{pgfscope}%
\begin{pgfscope}%
\pgfpathrectangle{\pgfqpoint{0.750000in}{0.500000in}}{\pgfqpoint{4.650000in}{3.020000in}}%
\pgfusepath{clip}%
\pgfsetbuttcap%
\pgfsetmiterjoin%
\definecolor{currentfill}{rgb}{1.000000,0.000000,0.000000}%
\pgfsetfillcolor{currentfill}%
\pgfsetlinewidth{0.000000pt}%
\definecolor{currentstroke}{rgb}{0.000000,0.000000,0.000000}%
\pgfsetstrokecolor{currentstroke}%
\pgfsetstrokeopacity{0.000000}%
\pgfsetdash{}{0pt}%
\pgfpathmoveto{\pgfqpoint{2.653018in}{0.500000in}}%
\pgfpathlineto{\pgfqpoint{2.677296in}{0.500000in}}%
\pgfpathlineto{\pgfqpoint{2.677296in}{1.436575in}}%
\pgfpathlineto{\pgfqpoint{2.653018in}{1.436575in}}%
\pgfpathlineto{\pgfqpoint{2.653018in}{0.500000in}}%
\pgfpathclose%
\pgfusepath{fill}%
\end{pgfscope}%
\begin{pgfscope}%
\pgfpathrectangle{\pgfqpoint{0.750000in}{0.500000in}}{\pgfqpoint{4.650000in}{3.020000in}}%
\pgfusepath{clip}%
\pgfsetbuttcap%
\pgfsetmiterjoin%
\definecolor{currentfill}{rgb}{1.000000,0.000000,0.000000}%
\pgfsetfillcolor{currentfill}%
\pgfsetlinewidth{0.000000pt}%
\definecolor{currentstroke}{rgb}{0.000000,0.000000,0.000000}%
\pgfsetstrokecolor{currentstroke}%
\pgfsetstrokeopacity{0.000000}%
\pgfsetdash{}{0pt}%
\pgfpathmoveto{\pgfqpoint{2.677296in}{0.500000in}}%
\pgfpathlineto{\pgfqpoint{2.701575in}{0.500000in}}%
\pgfpathlineto{\pgfqpoint{2.701575in}{0.530212in}}%
\pgfpathlineto{\pgfqpoint{2.677296in}{0.530212in}}%
\pgfpathlineto{\pgfqpoint{2.677296in}{0.500000in}}%
\pgfpathclose%
\pgfusepath{fill}%
\end{pgfscope}%
\begin{pgfscope}%
\pgfpathrectangle{\pgfqpoint{0.750000in}{0.500000in}}{\pgfqpoint{4.650000in}{3.020000in}}%
\pgfusepath{clip}%
\pgfsetbuttcap%
\pgfsetmiterjoin%
\definecolor{currentfill}{rgb}{1.000000,0.000000,0.000000}%
\pgfsetfillcolor{currentfill}%
\pgfsetlinewidth{0.000000pt}%
\definecolor{currentstroke}{rgb}{0.000000,0.000000,0.000000}%
\pgfsetstrokecolor{currentstroke}%
\pgfsetstrokeopacity{0.000000}%
\pgfsetdash{}{0pt}%
\pgfpathmoveto{\pgfqpoint{2.701575in}{0.500000in}}%
\pgfpathlineto{\pgfqpoint{2.725854in}{0.500000in}}%
\pgfpathlineto{\pgfqpoint{2.725854in}{0.533233in}}%
\pgfpathlineto{\pgfqpoint{2.701575in}{0.533233in}}%
\pgfpathlineto{\pgfqpoint{2.701575in}{0.500000in}}%
\pgfpathclose%
\pgfusepath{fill}%
\end{pgfscope}%
\begin{pgfscope}%
\pgfpathrectangle{\pgfqpoint{0.750000in}{0.500000in}}{\pgfqpoint{4.650000in}{3.020000in}}%
\pgfusepath{clip}%
\pgfsetbuttcap%
\pgfsetmiterjoin%
\definecolor{currentfill}{rgb}{1.000000,0.000000,0.000000}%
\pgfsetfillcolor{currentfill}%
\pgfsetlinewidth{0.000000pt}%
\definecolor{currentstroke}{rgb}{0.000000,0.000000,0.000000}%
\pgfsetstrokecolor{currentstroke}%
\pgfsetstrokeopacity{0.000000}%
\pgfsetdash{}{0pt}%
\pgfpathmoveto{\pgfqpoint{2.725854in}{0.500000in}}%
\pgfpathlineto{\pgfqpoint{2.750132in}{0.500000in}}%
\pgfpathlineto{\pgfqpoint{2.750132in}{1.880692in}}%
\pgfpathlineto{\pgfqpoint{2.725854in}{1.880692in}}%
\pgfpathlineto{\pgfqpoint{2.725854in}{0.500000in}}%
\pgfpathclose%
\pgfusepath{fill}%
\end{pgfscope}%
\begin{pgfscope}%
\pgfpathrectangle{\pgfqpoint{0.750000in}{0.500000in}}{\pgfqpoint{4.650000in}{3.020000in}}%
\pgfusepath{clip}%
\pgfsetbuttcap%
\pgfsetmiterjoin%
\definecolor{currentfill}{rgb}{1.000000,0.000000,0.000000}%
\pgfsetfillcolor{currentfill}%
\pgfsetlinewidth{0.000000pt}%
\definecolor{currentstroke}{rgb}{0.000000,0.000000,0.000000}%
\pgfsetstrokecolor{currentstroke}%
\pgfsetstrokeopacity{0.000000}%
\pgfsetdash{}{0pt}%
\pgfpathmoveto{\pgfqpoint{2.750132in}{0.500000in}}%
\pgfpathlineto{\pgfqpoint{2.774411in}{0.500000in}}%
\pgfpathlineto{\pgfqpoint{2.774411in}{0.530212in}}%
\pgfpathlineto{\pgfqpoint{2.750132in}{0.530212in}}%
\pgfpathlineto{\pgfqpoint{2.750132in}{0.500000in}}%
\pgfpathclose%
\pgfusepath{fill}%
\end{pgfscope}%
\begin{pgfscope}%
\pgfpathrectangle{\pgfqpoint{0.750000in}{0.500000in}}{\pgfqpoint{4.650000in}{3.020000in}}%
\pgfusepath{clip}%
\pgfsetbuttcap%
\pgfsetmiterjoin%
\definecolor{currentfill}{rgb}{1.000000,0.000000,0.000000}%
\pgfsetfillcolor{currentfill}%
\pgfsetlinewidth{0.000000pt}%
\definecolor{currentstroke}{rgb}{0.000000,0.000000,0.000000}%
\pgfsetstrokecolor{currentstroke}%
\pgfsetstrokeopacity{0.000000}%
\pgfsetdash{}{0pt}%
\pgfpathmoveto{\pgfqpoint{2.774411in}{0.500000in}}%
\pgfpathlineto{\pgfqpoint{2.798689in}{0.500000in}}%
\pgfpathlineto{\pgfqpoint{2.798689in}{2.119368in}}%
\pgfpathlineto{\pgfqpoint{2.774411in}{2.119368in}}%
\pgfpathlineto{\pgfqpoint{2.774411in}{0.500000in}}%
\pgfpathclose%
\pgfusepath{fill}%
\end{pgfscope}%
\begin{pgfscope}%
\pgfpathrectangle{\pgfqpoint{0.750000in}{0.500000in}}{\pgfqpoint{4.650000in}{3.020000in}}%
\pgfusepath{clip}%
\pgfsetbuttcap%
\pgfsetmiterjoin%
\definecolor{currentfill}{rgb}{1.000000,0.000000,0.000000}%
\pgfsetfillcolor{currentfill}%
\pgfsetlinewidth{0.000000pt}%
\definecolor{currentstroke}{rgb}{0.000000,0.000000,0.000000}%
\pgfsetstrokecolor{currentstroke}%
\pgfsetstrokeopacity{0.000000}%
\pgfsetdash{}{0pt}%
\pgfpathmoveto{\pgfqpoint{2.798689in}{0.500000in}}%
\pgfpathlineto{\pgfqpoint{2.822968in}{0.500000in}}%
\pgfpathlineto{\pgfqpoint{2.822968in}{0.690336in}}%
\pgfpathlineto{\pgfqpoint{2.798689in}{0.690336in}}%
\pgfpathlineto{\pgfqpoint{2.798689in}{0.500000in}}%
\pgfpathclose%
\pgfusepath{fill}%
\end{pgfscope}%
\begin{pgfscope}%
\pgfpathrectangle{\pgfqpoint{0.750000in}{0.500000in}}{\pgfqpoint{4.650000in}{3.020000in}}%
\pgfusepath{clip}%
\pgfsetbuttcap%
\pgfsetmiterjoin%
\definecolor{currentfill}{rgb}{1.000000,0.000000,0.000000}%
\pgfsetfillcolor{currentfill}%
\pgfsetlinewidth{0.000000pt}%
\definecolor{currentstroke}{rgb}{0.000000,0.000000,0.000000}%
\pgfsetstrokecolor{currentstroke}%
\pgfsetstrokeopacity{0.000000}%
\pgfsetdash{}{0pt}%
\pgfpathmoveto{\pgfqpoint{2.822968in}{0.500000in}}%
\pgfpathlineto{\pgfqpoint{2.847247in}{0.500000in}}%
\pgfpathlineto{\pgfqpoint{2.847247in}{0.572509in}}%
\pgfpathlineto{\pgfqpoint{2.822968in}{0.572509in}}%
\pgfpathlineto{\pgfqpoint{2.822968in}{0.500000in}}%
\pgfpathclose%
\pgfusepath{fill}%
\end{pgfscope}%
\begin{pgfscope}%
\pgfpathrectangle{\pgfqpoint{0.750000in}{0.500000in}}{\pgfqpoint{4.650000in}{3.020000in}}%
\pgfusepath{clip}%
\pgfsetbuttcap%
\pgfsetmiterjoin%
\definecolor{currentfill}{rgb}{1.000000,0.000000,0.000000}%
\pgfsetfillcolor{currentfill}%
\pgfsetlinewidth{0.000000pt}%
\definecolor{currentstroke}{rgb}{0.000000,0.000000,0.000000}%
\pgfsetstrokecolor{currentstroke}%
\pgfsetstrokeopacity{0.000000}%
\pgfsetdash{}{0pt}%
\pgfpathmoveto{\pgfqpoint{2.847247in}{0.500000in}}%
\pgfpathlineto{\pgfqpoint{2.871525in}{0.500000in}}%
\pgfpathlineto{\pgfqpoint{2.871525in}{2.584634in}}%
\pgfpathlineto{\pgfqpoint{2.847247in}{2.584634in}}%
\pgfpathlineto{\pgfqpoint{2.847247in}{0.500000in}}%
\pgfpathclose%
\pgfusepath{fill}%
\end{pgfscope}%
\begin{pgfscope}%
\pgfpathrectangle{\pgfqpoint{0.750000in}{0.500000in}}{\pgfqpoint{4.650000in}{3.020000in}}%
\pgfusepath{clip}%
\pgfsetbuttcap%
\pgfsetmiterjoin%
\definecolor{currentfill}{rgb}{1.000000,0.000000,0.000000}%
\pgfsetfillcolor{currentfill}%
\pgfsetlinewidth{0.000000pt}%
\definecolor{currentstroke}{rgb}{0.000000,0.000000,0.000000}%
\pgfsetstrokecolor{currentstroke}%
\pgfsetstrokeopacity{0.000000}%
\pgfsetdash{}{0pt}%
\pgfpathmoveto{\pgfqpoint{2.871525in}{0.500000in}}%
\pgfpathlineto{\pgfqpoint{2.895804in}{0.500000in}}%
\pgfpathlineto{\pgfqpoint{2.895804in}{0.590636in}}%
\pgfpathlineto{\pgfqpoint{2.871525in}{0.590636in}}%
\pgfpathlineto{\pgfqpoint{2.871525in}{0.500000in}}%
\pgfpathclose%
\pgfusepath{fill}%
\end{pgfscope}%
\begin{pgfscope}%
\pgfpathrectangle{\pgfqpoint{0.750000in}{0.500000in}}{\pgfqpoint{4.650000in}{3.020000in}}%
\pgfusepath{clip}%
\pgfsetbuttcap%
\pgfsetmiterjoin%
\definecolor{currentfill}{rgb}{1.000000,0.000000,0.000000}%
\pgfsetfillcolor{currentfill}%
\pgfsetlinewidth{0.000000pt}%
\definecolor{currentstroke}{rgb}{0.000000,0.000000,0.000000}%
\pgfsetstrokecolor{currentstroke}%
\pgfsetstrokeopacity{0.000000}%
\pgfsetdash{}{0pt}%
\pgfpathmoveto{\pgfqpoint{2.895804in}{0.500000in}}%
\pgfpathlineto{\pgfqpoint{2.920082in}{0.500000in}}%
\pgfpathlineto{\pgfqpoint{2.920082in}{0.596679in}}%
\pgfpathlineto{\pgfqpoint{2.895804in}{0.596679in}}%
\pgfpathlineto{\pgfqpoint{2.895804in}{0.500000in}}%
\pgfpathclose%
\pgfusepath{fill}%
\end{pgfscope}%
\begin{pgfscope}%
\pgfpathrectangle{\pgfqpoint{0.750000in}{0.500000in}}{\pgfqpoint{4.650000in}{3.020000in}}%
\pgfusepath{clip}%
\pgfsetbuttcap%
\pgfsetmiterjoin%
\definecolor{currentfill}{rgb}{1.000000,0.000000,0.000000}%
\pgfsetfillcolor{currentfill}%
\pgfsetlinewidth{0.000000pt}%
\definecolor{currentstroke}{rgb}{0.000000,0.000000,0.000000}%
\pgfsetstrokecolor{currentstroke}%
\pgfsetstrokeopacity{0.000000}%
\pgfsetdash{}{0pt}%
\pgfpathmoveto{\pgfqpoint{2.920082in}{0.500000in}}%
\pgfpathlineto{\pgfqpoint{2.944361in}{0.500000in}}%
\pgfpathlineto{\pgfqpoint{2.944361in}{3.037815in}}%
\pgfpathlineto{\pgfqpoint{2.920082in}{3.037815in}}%
\pgfpathlineto{\pgfqpoint{2.920082in}{0.500000in}}%
\pgfpathclose%
\pgfusepath{fill}%
\end{pgfscope}%
\begin{pgfscope}%
\pgfpathrectangle{\pgfqpoint{0.750000in}{0.500000in}}{\pgfqpoint{4.650000in}{3.020000in}}%
\pgfusepath{clip}%
\pgfsetbuttcap%
\pgfsetmiterjoin%
\definecolor{currentfill}{rgb}{1.000000,0.000000,0.000000}%
\pgfsetfillcolor{currentfill}%
\pgfsetlinewidth{0.000000pt}%
\definecolor{currentstroke}{rgb}{0.000000,0.000000,0.000000}%
\pgfsetstrokecolor{currentstroke}%
\pgfsetstrokeopacity{0.000000}%
\pgfsetdash{}{0pt}%
\pgfpathmoveto{\pgfqpoint{2.944361in}{0.500000in}}%
\pgfpathlineto{\pgfqpoint{2.968639in}{0.500000in}}%
\pgfpathlineto{\pgfqpoint{2.968639in}{0.575530in}}%
\pgfpathlineto{\pgfqpoint{2.944361in}{0.575530in}}%
\pgfpathlineto{\pgfqpoint{2.944361in}{0.500000in}}%
\pgfpathclose%
\pgfusepath{fill}%
\end{pgfscope}%
\begin{pgfscope}%
\pgfpathrectangle{\pgfqpoint{0.750000in}{0.500000in}}{\pgfqpoint{4.650000in}{3.020000in}}%
\pgfusepath{clip}%
\pgfsetbuttcap%
\pgfsetmiterjoin%
\definecolor{currentfill}{rgb}{1.000000,0.000000,0.000000}%
\pgfsetfillcolor{currentfill}%
\pgfsetlinewidth{0.000000pt}%
\definecolor{currentstroke}{rgb}{0.000000,0.000000,0.000000}%
\pgfsetstrokecolor{currentstroke}%
\pgfsetstrokeopacity{0.000000}%
\pgfsetdash{}{0pt}%
\pgfpathmoveto{\pgfqpoint{2.968639in}{0.500000in}}%
\pgfpathlineto{\pgfqpoint{2.992918in}{0.500000in}}%
\pgfpathlineto{\pgfqpoint{2.992918in}{0.608764in}}%
\pgfpathlineto{\pgfqpoint{2.968639in}{0.608764in}}%
\pgfpathlineto{\pgfqpoint{2.968639in}{0.500000in}}%
\pgfpathclose%
\pgfusepath{fill}%
\end{pgfscope}%
\begin{pgfscope}%
\pgfpathrectangle{\pgfqpoint{0.750000in}{0.500000in}}{\pgfqpoint{4.650000in}{3.020000in}}%
\pgfusepath{clip}%
\pgfsetbuttcap%
\pgfsetmiterjoin%
\definecolor{currentfill}{rgb}{1.000000,0.000000,0.000000}%
\pgfsetfillcolor{currentfill}%
\pgfsetlinewidth{0.000000pt}%
\definecolor{currentstroke}{rgb}{0.000000,0.000000,0.000000}%
\pgfsetstrokecolor{currentstroke}%
\pgfsetstrokeopacity{0.000000}%
\pgfsetdash{}{0pt}%
\pgfpathmoveto{\pgfqpoint{2.992918in}{0.500000in}}%
\pgfpathlineto{\pgfqpoint{3.017197in}{0.500000in}}%
\pgfpathlineto{\pgfqpoint{3.017197in}{3.267427in}}%
\pgfpathlineto{\pgfqpoint{2.992918in}{3.267427in}}%
\pgfpathlineto{\pgfqpoint{2.992918in}{0.500000in}}%
\pgfpathclose%
\pgfusepath{fill}%
\end{pgfscope}%
\begin{pgfscope}%
\pgfpathrectangle{\pgfqpoint{0.750000in}{0.500000in}}{\pgfqpoint{4.650000in}{3.020000in}}%
\pgfusepath{clip}%
\pgfsetbuttcap%
\pgfsetmiterjoin%
\definecolor{currentfill}{rgb}{1.000000,0.000000,0.000000}%
\pgfsetfillcolor{currentfill}%
\pgfsetlinewidth{0.000000pt}%
\definecolor{currentstroke}{rgb}{0.000000,0.000000,0.000000}%
\pgfsetstrokecolor{currentstroke}%
\pgfsetstrokeopacity{0.000000}%
\pgfsetdash{}{0pt}%
\pgfpathmoveto{\pgfqpoint{3.017197in}{0.500000in}}%
\pgfpathlineto{\pgfqpoint{3.041475in}{0.500000in}}%
\pgfpathlineto{\pgfqpoint{3.041475in}{0.587615in}}%
\pgfpathlineto{\pgfqpoint{3.017197in}{0.587615in}}%
\pgfpathlineto{\pgfqpoint{3.017197in}{0.500000in}}%
\pgfpathclose%
\pgfusepath{fill}%
\end{pgfscope}%
\begin{pgfscope}%
\pgfpathrectangle{\pgfqpoint{0.750000in}{0.500000in}}{\pgfqpoint{4.650000in}{3.020000in}}%
\pgfusepath{clip}%
\pgfsetbuttcap%
\pgfsetmiterjoin%
\definecolor{currentfill}{rgb}{1.000000,0.000000,0.000000}%
\pgfsetfillcolor{currentfill}%
\pgfsetlinewidth{0.000000pt}%
\definecolor{currentstroke}{rgb}{0.000000,0.000000,0.000000}%
\pgfsetstrokecolor{currentstroke}%
\pgfsetstrokeopacity{0.000000}%
\pgfsetdash{}{0pt}%
\pgfpathmoveto{\pgfqpoint{3.041475in}{0.500000in}}%
\pgfpathlineto{\pgfqpoint{3.065754in}{0.500000in}}%
\pgfpathlineto{\pgfqpoint{3.065754in}{3.376190in}}%
\pgfpathlineto{\pgfqpoint{3.041475in}{3.376190in}}%
\pgfpathlineto{\pgfqpoint{3.041475in}{0.500000in}}%
\pgfpathclose%
\pgfusepath{fill}%
\end{pgfscope}%
\begin{pgfscope}%
\pgfpathrectangle{\pgfqpoint{0.750000in}{0.500000in}}{\pgfqpoint{4.650000in}{3.020000in}}%
\pgfusepath{clip}%
\pgfsetbuttcap%
\pgfsetmiterjoin%
\definecolor{currentfill}{rgb}{1.000000,0.000000,0.000000}%
\pgfsetfillcolor{currentfill}%
\pgfsetlinewidth{0.000000pt}%
\definecolor{currentstroke}{rgb}{0.000000,0.000000,0.000000}%
\pgfsetstrokecolor{currentstroke}%
\pgfsetstrokeopacity{0.000000}%
\pgfsetdash{}{0pt}%
\pgfpathmoveto{\pgfqpoint{3.065754in}{0.500000in}}%
\pgfpathlineto{\pgfqpoint{3.090032in}{0.500000in}}%
\pgfpathlineto{\pgfqpoint{3.090032in}{0.645018in}}%
\pgfpathlineto{\pgfqpoint{3.065754in}{0.645018in}}%
\pgfpathlineto{\pgfqpoint{3.065754in}{0.500000in}}%
\pgfpathclose%
\pgfusepath{fill}%
\end{pgfscope}%
\begin{pgfscope}%
\pgfpathrectangle{\pgfqpoint{0.750000in}{0.500000in}}{\pgfqpoint{4.650000in}{3.020000in}}%
\pgfusepath{clip}%
\pgfsetbuttcap%
\pgfsetmiterjoin%
\definecolor{currentfill}{rgb}{1.000000,0.000000,0.000000}%
\pgfsetfillcolor{currentfill}%
\pgfsetlinewidth{0.000000pt}%
\definecolor{currentstroke}{rgb}{0.000000,0.000000,0.000000}%
\pgfsetstrokecolor{currentstroke}%
\pgfsetstrokeopacity{0.000000}%
\pgfsetdash{}{0pt}%
\pgfpathmoveto{\pgfqpoint{3.090032in}{0.500000in}}%
\pgfpathlineto{\pgfqpoint{3.114311in}{0.500000in}}%
\pgfpathlineto{\pgfqpoint{3.114311in}{0.590636in}}%
\pgfpathlineto{\pgfqpoint{3.090032in}{0.590636in}}%
\pgfpathlineto{\pgfqpoint{3.090032in}{0.500000in}}%
\pgfpathclose%
\pgfusepath{fill}%
\end{pgfscope}%
\begin{pgfscope}%
\pgfpathrectangle{\pgfqpoint{0.750000in}{0.500000in}}{\pgfqpoint{4.650000in}{3.020000in}}%
\pgfusepath{clip}%
\pgfsetbuttcap%
\pgfsetmiterjoin%
\definecolor{currentfill}{rgb}{1.000000,0.000000,0.000000}%
\pgfsetfillcolor{currentfill}%
\pgfsetlinewidth{0.000000pt}%
\definecolor{currentstroke}{rgb}{0.000000,0.000000,0.000000}%
\pgfsetstrokecolor{currentstroke}%
\pgfsetstrokeopacity{0.000000}%
\pgfsetdash{}{0pt}%
\pgfpathmoveto{\pgfqpoint{3.114311in}{0.500000in}}%
\pgfpathlineto{\pgfqpoint{3.138590in}{0.500000in}}%
\pgfpathlineto{\pgfqpoint{3.138590in}{3.273469in}}%
\pgfpathlineto{\pgfqpoint{3.114311in}{3.273469in}}%
\pgfpathlineto{\pgfqpoint{3.114311in}{0.500000in}}%
\pgfpathclose%
\pgfusepath{fill}%
\end{pgfscope}%
\begin{pgfscope}%
\pgfpathrectangle{\pgfqpoint{0.750000in}{0.500000in}}{\pgfqpoint{4.650000in}{3.020000in}}%
\pgfusepath{clip}%
\pgfsetbuttcap%
\pgfsetmiterjoin%
\definecolor{currentfill}{rgb}{1.000000,0.000000,0.000000}%
\pgfsetfillcolor{currentfill}%
\pgfsetlinewidth{0.000000pt}%
\definecolor{currentstroke}{rgb}{0.000000,0.000000,0.000000}%
\pgfsetstrokecolor{currentstroke}%
\pgfsetstrokeopacity{0.000000}%
\pgfsetdash{}{0pt}%
\pgfpathmoveto{\pgfqpoint{3.138590in}{0.500000in}}%
\pgfpathlineto{\pgfqpoint{3.162868in}{0.500000in}}%
\pgfpathlineto{\pgfqpoint{3.162868in}{0.593657in}}%
\pgfpathlineto{\pgfqpoint{3.138590in}{0.593657in}}%
\pgfpathlineto{\pgfqpoint{3.138590in}{0.500000in}}%
\pgfpathclose%
\pgfusepath{fill}%
\end{pgfscope}%
\begin{pgfscope}%
\pgfpathrectangle{\pgfqpoint{0.750000in}{0.500000in}}{\pgfqpoint{4.650000in}{3.020000in}}%
\pgfusepath{clip}%
\pgfsetbuttcap%
\pgfsetmiterjoin%
\definecolor{currentfill}{rgb}{1.000000,0.000000,0.000000}%
\pgfsetfillcolor{currentfill}%
\pgfsetlinewidth{0.000000pt}%
\definecolor{currentstroke}{rgb}{0.000000,0.000000,0.000000}%
\pgfsetstrokecolor{currentstroke}%
\pgfsetstrokeopacity{0.000000}%
\pgfsetdash{}{0pt}%
\pgfpathmoveto{\pgfqpoint{3.162868in}{0.500000in}}%
\pgfpathlineto{\pgfqpoint{3.187147in}{0.500000in}}%
\pgfpathlineto{\pgfqpoint{3.187147in}{0.593657in}}%
\pgfpathlineto{\pgfqpoint{3.162868in}{0.593657in}}%
\pgfpathlineto{\pgfqpoint{3.162868in}{0.500000in}}%
\pgfpathclose%
\pgfusepath{fill}%
\end{pgfscope}%
\begin{pgfscope}%
\pgfpathrectangle{\pgfqpoint{0.750000in}{0.500000in}}{\pgfqpoint{4.650000in}{3.020000in}}%
\pgfusepath{clip}%
\pgfsetbuttcap%
\pgfsetmiterjoin%
\definecolor{currentfill}{rgb}{1.000000,0.000000,0.000000}%
\pgfsetfillcolor{currentfill}%
\pgfsetlinewidth{0.000000pt}%
\definecolor{currentstroke}{rgb}{0.000000,0.000000,0.000000}%
\pgfsetstrokecolor{currentstroke}%
\pgfsetstrokeopacity{0.000000}%
\pgfsetdash{}{0pt}%
\pgfpathmoveto{\pgfqpoint{3.187147in}{0.500000in}}%
\pgfpathlineto{\pgfqpoint{3.211425in}{0.500000in}}%
\pgfpathlineto{\pgfqpoint{3.211425in}{3.016667in}}%
\pgfpathlineto{\pgfqpoint{3.187147in}{3.016667in}}%
\pgfpathlineto{\pgfqpoint{3.187147in}{0.500000in}}%
\pgfpathclose%
\pgfusepath{fill}%
\end{pgfscope}%
\begin{pgfscope}%
\pgfpathrectangle{\pgfqpoint{0.750000in}{0.500000in}}{\pgfqpoint{4.650000in}{3.020000in}}%
\pgfusepath{clip}%
\pgfsetbuttcap%
\pgfsetmiterjoin%
\definecolor{currentfill}{rgb}{1.000000,0.000000,0.000000}%
\pgfsetfillcolor{currentfill}%
\pgfsetlinewidth{0.000000pt}%
\definecolor{currentstroke}{rgb}{0.000000,0.000000,0.000000}%
\pgfsetstrokecolor{currentstroke}%
\pgfsetstrokeopacity{0.000000}%
\pgfsetdash{}{0pt}%
\pgfpathmoveto{\pgfqpoint{3.211425in}{0.500000in}}%
\pgfpathlineto{\pgfqpoint{3.235704in}{0.500000in}}%
\pgfpathlineto{\pgfqpoint{3.235704in}{0.563445in}}%
\pgfpathlineto{\pgfqpoint{3.211425in}{0.563445in}}%
\pgfpathlineto{\pgfqpoint{3.211425in}{0.500000in}}%
\pgfpathclose%
\pgfusepath{fill}%
\end{pgfscope}%
\begin{pgfscope}%
\pgfpathrectangle{\pgfqpoint{0.750000in}{0.500000in}}{\pgfqpoint{4.650000in}{3.020000in}}%
\pgfusepath{clip}%
\pgfsetbuttcap%
\pgfsetmiterjoin%
\definecolor{currentfill}{rgb}{1.000000,0.000000,0.000000}%
\pgfsetfillcolor{currentfill}%
\pgfsetlinewidth{0.000000pt}%
\definecolor{currentstroke}{rgb}{0.000000,0.000000,0.000000}%
\pgfsetstrokecolor{currentstroke}%
\pgfsetstrokeopacity{0.000000}%
\pgfsetdash{}{0pt}%
\pgfpathmoveto{\pgfqpoint{3.235704in}{0.500000in}}%
\pgfpathlineto{\pgfqpoint{3.259982in}{0.500000in}}%
\pgfpathlineto{\pgfqpoint{3.259982in}{0.635954in}}%
\pgfpathlineto{\pgfqpoint{3.235704in}{0.635954in}}%
\pgfpathlineto{\pgfqpoint{3.235704in}{0.500000in}}%
\pgfpathclose%
\pgfusepath{fill}%
\end{pgfscope}%
\begin{pgfscope}%
\pgfpathrectangle{\pgfqpoint{0.750000in}{0.500000in}}{\pgfqpoint{4.650000in}{3.020000in}}%
\pgfusepath{clip}%
\pgfsetbuttcap%
\pgfsetmiterjoin%
\definecolor{currentfill}{rgb}{1.000000,0.000000,0.000000}%
\pgfsetfillcolor{currentfill}%
\pgfsetlinewidth{0.000000pt}%
\definecolor{currentstroke}{rgb}{0.000000,0.000000,0.000000}%
\pgfsetstrokecolor{currentstroke}%
\pgfsetstrokeopacity{0.000000}%
\pgfsetdash{}{0pt}%
\pgfpathmoveto{\pgfqpoint{3.259982in}{0.500000in}}%
\pgfpathlineto{\pgfqpoint{3.284261in}{0.500000in}}%
\pgfpathlineto{\pgfqpoint{3.284261in}{2.509104in}}%
\pgfpathlineto{\pgfqpoint{3.259982in}{2.509104in}}%
\pgfpathlineto{\pgfqpoint{3.259982in}{0.500000in}}%
\pgfpathclose%
\pgfusepath{fill}%
\end{pgfscope}%
\begin{pgfscope}%
\pgfpathrectangle{\pgfqpoint{0.750000in}{0.500000in}}{\pgfqpoint{4.650000in}{3.020000in}}%
\pgfusepath{clip}%
\pgfsetbuttcap%
\pgfsetmiterjoin%
\definecolor{currentfill}{rgb}{1.000000,0.000000,0.000000}%
\pgfsetfillcolor{currentfill}%
\pgfsetlinewidth{0.000000pt}%
\definecolor{currentstroke}{rgb}{0.000000,0.000000,0.000000}%
\pgfsetstrokecolor{currentstroke}%
\pgfsetstrokeopacity{0.000000}%
\pgfsetdash{}{0pt}%
\pgfpathmoveto{\pgfqpoint{3.284261in}{0.500000in}}%
\pgfpathlineto{\pgfqpoint{3.308540in}{0.500000in}}%
\pgfpathlineto{\pgfqpoint{3.308540in}{0.569488in}}%
\pgfpathlineto{\pgfqpoint{3.284261in}{0.569488in}}%
\pgfpathlineto{\pgfqpoint{3.284261in}{0.500000in}}%
\pgfpathclose%
\pgfusepath{fill}%
\end{pgfscope}%
\begin{pgfscope}%
\pgfpathrectangle{\pgfqpoint{0.750000in}{0.500000in}}{\pgfqpoint{4.650000in}{3.020000in}}%
\pgfusepath{clip}%
\pgfsetbuttcap%
\pgfsetmiterjoin%
\definecolor{currentfill}{rgb}{1.000000,0.000000,0.000000}%
\pgfsetfillcolor{currentfill}%
\pgfsetlinewidth{0.000000pt}%
\definecolor{currentstroke}{rgb}{0.000000,0.000000,0.000000}%
\pgfsetstrokecolor{currentstroke}%
\pgfsetstrokeopacity{0.000000}%
\pgfsetdash{}{0pt}%
\pgfpathmoveto{\pgfqpoint{3.308540in}{0.500000in}}%
\pgfpathlineto{\pgfqpoint{3.332818in}{0.500000in}}%
\pgfpathlineto{\pgfqpoint{3.332818in}{2.203962in}}%
\pgfpathlineto{\pgfqpoint{3.308540in}{2.203962in}}%
\pgfpathlineto{\pgfqpoint{3.308540in}{0.500000in}}%
\pgfpathclose%
\pgfusepath{fill}%
\end{pgfscope}%
\begin{pgfscope}%
\pgfpathrectangle{\pgfqpoint{0.750000in}{0.500000in}}{\pgfqpoint{4.650000in}{3.020000in}}%
\pgfusepath{clip}%
\pgfsetbuttcap%
\pgfsetmiterjoin%
\definecolor{currentfill}{rgb}{1.000000,0.000000,0.000000}%
\pgfsetfillcolor{currentfill}%
\pgfsetlinewidth{0.000000pt}%
\definecolor{currentstroke}{rgb}{0.000000,0.000000,0.000000}%
\pgfsetstrokecolor{currentstroke}%
\pgfsetstrokeopacity{0.000000}%
\pgfsetdash{}{0pt}%
\pgfpathmoveto{\pgfqpoint{3.332818in}{0.500000in}}%
\pgfpathlineto{\pgfqpoint{3.357097in}{0.500000in}}%
\pgfpathlineto{\pgfqpoint{3.357097in}{0.566467in}}%
\pgfpathlineto{\pgfqpoint{3.332818in}{0.566467in}}%
\pgfpathlineto{\pgfqpoint{3.332818in}{0.500000in}}%
\pgfpathclose%
\pgfusepath{fill}%
\end{pgfscope}%
\begin{pgfscope}%
\pgfpathrectangle{\pgfqpoint{0.750000in}{0.500000in}}{\pgfqpoint{4.650000in}{3.020000in}}%
\pgfusepath{clip}%
\pgfsetbuttcap%
\pgfsetmiterjoin%
\definecolor{currentfill}{rgb}{1.000000,0.000000,0.000000}%
\pgfsetfillcolor{currentfill}%
\pgfsetlinewidth{0.000000pt}%
\definecolor{currentstroke}{rgb}{0.000000,0.000000,0.000000}%
\pgfsetstrokecolor{currentstroke}%
\pgfsetstrokeopacity{0.000000}%
\pgfsetdash{}{0pt}%
\pgfpathmoveto{\pgfqpoint{3.357097in}{0.500000in}}%
\pgfpathlineto{\pgfqpoint{3.381375in}{0.500000in}}%
\pgfpathlineto{\pgfqpoint{3.381375in}{0.563445in}}%
\pgfpathlineto{\pgfqpoint{3.357097in}{0.563445in}}%
\pgfpathlineto{\pgfqpoint{3.357097in}{0.500000in}}%
\pgfpathclose%
\pgfusepath{fill}%
\end{pgfscope}%
\begin{pgfscope}%
\pgfpathrectangle{\pgfqpoint{0.750000in}{0.500000in}}{\pgfqpoint{4.650000in}{3.020000in}}%
\pgfusepath{clip}%
\pgfsetbuttcap%
\pgfsetmiterjoin%
\definecolor{currentfill}{rgb}{1.000000,0.000000,0.000000}%
\pgfsetfillcolor{currentfill}%
\pgfsetlinewidth{0.000000pt}%
\definecolor{currentstroke}{rgb}{0.000000,0.000000,0.000000}%
\pgfsetstrokecolor{currentstroke}%
\pgfsetstrokeopacity{0.000000}%
\pgfsetdash{}{0pt}%
\pgfpathmoveto{\pgfqpoint{3.381375in}{0.500000in}}%
\pgfpathlineto{\pgfqpoint{3.405654in}{0.500000in}}%
\pgfpathlineto{\pgfqpoint{3.405654in}{1.711505in}}%
\pgfpathlineto{\pgfqpoint{3.381375in}{1.711505in}}%
\pgfpathlineto{\pgfqpoint{3.381375in}{0.500000in}}%
\pgfpathclose%
\pgfusepath{fill}%
\end{pgfscope}%
\begin{pgfscope}%
\pgfpathrectangle{\pgfqpoint{0.750000in}{0.500000in}}{\pgfqpoint{4.650000in}{3.020000in}}%
\pgfusepath{clip}%
\pgfsetbuttcap%
\pgfsetmiterjoin%
\definecolor{currentfill}{rgb}{1.000000,0.000000,0.000000}%
\pgfsetfillcolor{currentfill}%
\pgfsetlinewidth{0.000000pt}%
\definecolor{currentstroke}{rgb}{0.000000,0.000000,0.000000}%
\pgfsetstrokecolor{currentstroke}%
\pgfsetstrokeopacity{0.000000}%
\pgfsetdash{}{0pt}%
\pgfpathmoveto{\pgfqpoint{3.405654in}{0.500000in}}%
\pgfpathlineto{\pgfqpoint{3.429932in}{0.500000in}}%
\pgfpathlineto{\pgfqpoint{3.429932in}{0.518127in}}%
\pgfpathlineto{\pgfqpoint{3.405654in}{0.518127in}}%
\pgfpathlineto{\pgfqpoint{3.405654in}{0.500000in}}%
\pgfpathclose%
\pgfusepath{fill}%
\end{pgfscope}%
\begin{pgfscope}%
\pgfpathrectangle{\pgfqpoint{0.750000in}{0.500000in}}{\pgfqpoint{4.650000in}{3.020000in}}%
\pgfusepath{clip}%
\pgfsetbuttcap%
\pgfsetmiterjoin%
\definecolor{currentfill}{rgb}{1.000000,0.000000,0.000000}%
\pgfsetfillcolor{currentfill}%
\pgfsetlinewidth{0.000000pt}%
\definecolor{currentstroke}{rgb}{0.000000,0.000000,0.000000}%
\pgfsetstrokecolor{currentstroke}%
\pgfsetstrokeopacity{0.000000}%
\pgfsetdash{}{0pt}%
\pgfpathmoveto{\pgfqpoint{3.429932in}{0.500000in}}%
\pgfpathlineto{\pgfqpoint{3.454211in}{0.500000in}}%
\pgfpathlineto{\pgfqpoint{3.454211in}{0.545318in}}%
\pgfpathlineto{\pgfqpoint{3.429932in}{0.545318in}}%
\pgfpathlineto{\pgfqpoint{3.429932in}{0.500000in}}%
\pgfpathclose%
\pgfusepath{fill}%
\end{pgfscope}%
\begin{pgfscope}%
\pgfpathrectangle{\pgfqpoint{0.750000in}{0.500000in}}{\pgfqpoint{4.650000in}{3.020000in}}%
\pgfusepath{clip}%
\pgfsetbuttcap%
\pgfsetmiterjoin%
\definecolor{currentfill}{rgb}{1.000000,0.000000,0.000000}%
\pgfsetfillcolor{currentfill}%
\pgfsetlinewidth{0.000000pt}%
\definecolor{currentstroke}{rgb}{0.000000,0.000000,0.000000}%
\pgfsetstrokecolor{currentstroke}%
\pgfsetstrokeopacity{0.000000}%
\pgfsetdash{}{0pt}%
\pgfpathmoveto{\pgfqpoint{3.454211in}{0.500000in}}%
\pgfpathlineto{\pgfqpoint{3.478490in}{0.500000in}}%
\pgfpathlineto{\pgfqpoint{3.478490in}{1.469808in}}%
\pgfpathlineto{\pgfqpoint{3.454211in}{1.469808in}}%
\pgfpathlineto{\pgfqpoint{3.454211in}{0.500000in}}%
\pgfpathclose%
\pgfusepath{fill}%
\end{pgfscope}%
\begin{pgfscope}%
\pgfpathrectangle{\pgfqpoint{0.750000in}{0.500000in}}{\pgfqpoint{4.650000in}{3.020000in}}%
\pgfusepath{clip}%
\pgfsetbuttcap%
\pgfsetmiterjoin%
\definecolor{currentfill}{rgb}{1.000000,0.000000,0.000000}%
\pgfsetfillcolor{currentfill}%
\pgfsetlinewidth{0.000000pt}%
\definecolor{currentstroke}{rgb}{0.000000,0.000000,0.000000}%
\pgfsetstrokecolor{currentstroke}%
\pgfsetstrokeopacity{0.000000}%
\pgfsetdash{}{0pt}%
\pgfpathmoveto{\pgfqpoint{3.478490in}{0.500000in}}%
\pgfpathlineto{\pgfqpoint{3.502768in}{0.500000in}}%
\pgfpathlineto{\pgfqpoint{3.502768in}{0.530212in}}%
\pgfpathlineto{\pgfqpoint{3.478490in}{0.530212in}}%
\pgfpathlineto{\pgfqpoint{3.478490in}{0.500000in}}%
\pgfpathclose%
\pgfusepath{fill}%
\end{pgfscope}%
\begin{pgfscope}%
\pgfpathrectangle{\pgfqpoint{0.750000in}{0.500000in}}{\pgfqpoint{4.650000in}{3.020000in}}%
\pgfusepath{clip}%
\pgfsetbuttcap%
\pgfsetmiterjoin%
\definecolor{currentfill}{rgb}{1.000000,0.000000,0.000000}%
\pgfsetfillcolor{currentfill}%
\pgfsetlinewidth{0.000000pt}%
\definecolor{currentstroke}{rgb}{0.000000,0.000000,0.000000}%
\pgfsetstrokecolor{currentstroke}%
\pgfsetstrokeopacity{0.000000}%
\pgfsetdash{}{0pt}%
\pgfpathmoveto{\pgfqpoint{3.502768in}{0.500000in}}%
\pgfpathlineto{\pgfqpoint{3.527047in}{0.500000in}}%
\pgfpathlineto{\pgfqpoint{3.527047in}{1.110284in}}%
\pgfpathlineto{\pgfqpoint{3.502768in}{1.110284in}}%
\pgfpathlineto{\pgfqpoint{3.502768in}{0.500000in}}%
\pgfpathclose%
\pgfusepath{fill}%
\end{pgfscope}%
\begin{pgfscope}%
\pgfpathrectangle{\pgfqpoint{0.750000in}{0.500000in}}{\pgfqpoint{4.650000in}{3.020000in}}%
\pgfusepath{clip}%
\pgfsetbuttcap%
\pgfsetmiterjoin%
\definecolor{currentfill}{rgb}{1.000000,0.000000,0.000000}%
\pgfsetfillcolor{currentfill}%
\pgfsetlinewidth{0.000000pt}%
\definecolor{currentstroke}{rgb}{0.000000,0.000000,0.000000}%
\pgfsetstrokecolor{currentstroke}%
\pgfsetstrokeopacity{0.000000}%
\pgfsetdash{}{0pt}%
\pgfpathmoveto{\pgfqpoint{3.527047in}{0.500000in}}%
\pgfpathlineto{\pgfqpoint{3.551325in}{0.500000in}}%
\pgfpathlineto{\pgfqpoint{3.551325in}{0.539276in}}%
\pgfpathlineto{\pgfqpoint{3.527047in}{0.539276in}}%
\pgfpathlineto{\pgfqpoint{3.527047in}{0.500000in}}%
\pgfpathclose%
\pgfusepath{fill}%
\end{pgfscope}%
\begin{pgfscope}%
\pgfpathrectangle{\pgfqpoint{0.750000in}{0.500000in}}{\pgfqpoint{4.650000in}{3.020000in}}%
\pgfusepath{clip}%
\pgfsetbuttcap%
\pgfsetmiterjoin%
\definecolor{currentfill}{rgb}{1.000000,0.000000,0.000000}%
\pgfsetfillcolor{currentfill}%
\pgfsetlinewidth{0.000000pt}%
\definecolor{currentstroke}{rgb}{0.000000,0.000000,0.000000}%
\pgfsetstrokecolor{currentstroke}%
\pgfsetstrokeopacity{0.000000}%
\pgfsetdash{}{0pt}%
\pgfpathmoveto{\pgfqpoint{3.551325in}{0.500000in}}%
\pgfpathlineto{\pgfqpoint{3.575604in}{0.500000in}}%
\pgfpathlineto{\pgfqpoint{3.575604in}{0.518127in}}%
\pgfpathlineto{\pgfqpoint{3.551325in}{0.518127in}}%
\pgfpathlineto{\pgfqpoint{3.551325in}{0.500000in}}%
\pgfpathclose%
\pgfusepath{fill}%
\end{pgfscope}%
\begin{pgfscope}%
\pgfpathrectangle{\pgfqpoint{0.750000in}{0.500000in}}{\pgfqpoint{4.650000in}{3.020000in}}%
\pgfusepath{clip}%
\pgfsetbuttcap%
\pgfsetmiterjoin%
\definecolor{currentfill}{rgb}{1.000000,0.000000,0.000000}%
\pgfsetfillcolor{currentfill}%
\pgfsetlinewidth{0.000000pt}%
\definecolor{currentstroke}{rgb}{0.000000,0.000000,0.000000}%
\pgfsetstrokecolor{currentstroke}%
\pgfsetstrokeopacity{0.000000}%
\pgfsetdash{}{0pt}%
\pgfpathmoveto{\pgfqpoint{3.575604in}{0.500000in}}%
\pgfpathlineto{\pgfqpoint{3.599883in}{0.500000in}}%
\pgfpathlineto{\pgfqpoint{3.599883in}{0.856503in}}%
\pgfpathlineto{\pgfqpoint{3.575604in}{0.856503in}}%
\pgfpathlineto{\pgfqpoint{3.575604in}{0.500000in}}%
\pgfpathclose%
\pgfusepath{fill}%
\end{pgfscope}%
\begin{pgfscope}%
\pgfpathrectangle{\pgfqpoint{0.750000in}{0.500000in}}{\pgfqpoint{4.650000in}{3.020000in}}%
\pgfusepath{clip}%
\pgfsetbuttcap%
\pgfsetmiterjoin%
\definecolor{currentfill}{rgb}{1.000000,0.000000,0.000000}%
\pgfsetfillcolor{currentfill}%
\pgfsetlinewidth{0.000000pt}%
\definecolor{currentstroke}{rgb}{0.000000,0.000000,0.000000}%
\pgfsetstrokecolor{currentstroke}%
\pgfsetstrokeopacity{0.000000}%
\pgfsetdash{}{0pt}%
\pgfpathmoveto{\pgfqpoint{3.599883in}{0.500000in}}%
\pgfpathlineto{\pgfqpoint{3.624161in}{0.500000in}}%
\pgfpathlineto{\pgfqpoint{3.624161in}{0.503021in}}%
\pgfpathlineto{\pgfqpoint{3.599883in}{0.503021in}}%
\pgfpathlineto{\pgfqpoint{3.599883in}{0.500000in}}%
\pgfpathclose%
\pgfusepath{fill}%
\end{pgfscope}%
\begin{pgfscope}%
\pgfpathrectangle{\pgfqpoint{0.750000in}{0.500000in}}{\pgfqpoint{4.650000in}{3.020000in}}%
\pgfusepath{clip}%
\pgfsetbuttcap%
\pgfsetmiterjoin%
\definecolor{currentfill}{rgb}{1.000000,0.000000,0.000000}%
\pgfsetfillcolor{currentfill}%
\pgfsetlinewidth{0.000000pt}%
\definecolor{currentstroke}{rgb}{0.000000,0.000000,0.000000}%
\pgfsetstrokecolor{currentstroke}%
\pgfsetstrokeopacity{0.000000}%
\pgfsetdash{}{0pt}%
\pgfpathmoveto{\pgfqpoint{3.624161in}{0.500000in}}%
\pgfpathlineto{\pgfqpoint{3.648440in}{0.500000in}}%
\pgfpathlineto{\pgfqpoint{3.648440in}{0.506042in}}%
\pgfpathlineto{\pgfqpoint{3.624161in}{0.506042in}}%
\pgfpathlineto{\pgfqpoint{3.624161in}{0.500000in}}%
\pgfpathclose%
\pgfusepath{fill}%
\end{pgfscope}%
\begin{pgfscope}%
\pgfpathrectangle{\pgfqpoint{0.750000in}{0.500000in}}{\pgfqpoint{4.650000in}{3.020000in}}%
\pgfusepath{clip}%
\pgfsetbuttcap%
\pgfsetmiterjoin%
\definecolor{currentfill}{rgb}{1.000000,0.000000,0.000000}%
\pgfsetfillcolor{currentfill}%
\pgfsetlinewidth{0.000000pt}%
\definecolor{currentstroke}{rgb}{0.000000,0.000000,0.000000}%
\pgfsetstrokecolor{currentstroke}%
\pgfsetstrokeopacity{0.000000}%
\pgfsetdash{}{0pt}%
\pgfpathmoveto{\pgfqpoint{3.648440in}{0.500000in}}%
\pgfpathlineto{\pgfqpoint{3.672718in}{0.500000in}}%
\pgfpathlineto{\pgfqpoint{3.672718in}{0.723569in}}%
\pgfpathlineto{\pgfqpoint{3.648440in}{0.723569in}}%
\pgfpathlineto{\pgfqpoint{3.648440in}{0.500000in}}%
\pgfpathclose%
\pgfusepath{fill}%
\end{pgfscope}%
\begin{pgfscope}%
\pgfpathrectangle{\pgfqpoint{0.750000in}{0.500000in}}{\pgfqpoint{4.650000in}{3.020000in}}%
\pgfusepath{clip}%
\pgfsetbuttcap%
\pgfsetmiterjoin%
\definecolor{currentfill}{rgb}{1.000000,0.000000,0.000000}%
\pgfsetfillcolor{currentfill}%
\pgfsetlinewidth{0.000000pt}%
\definecolor{currentstroke}{rgb}{0.000000,0.000000,0.000000}%
\pgfsetstrokecolor{currentstroke}%
\pgfsetstrokeopacity{0.000000}%
\pgfsetdash{}{0pt}%
\pgfpathmoveto{\pgfqpoint{3.672718in}{0.500000in}}%
\pgfpathlineto{\pgfqpoint{3.696997in}{0.500000in}}%
\pgfpathlineto{\pgfqpoint{3.696997in}{0.509064in}}%
\pgfpathlineto{\pgfqpoint{3.672718in}{0.509064in}}%
\pgfpathlineto{\pgfqpoint{3.672718in}{0.500000in}}%
\pgfpathclose%
\pgfusepath{fill}%
\end{pgfscope}%
\begin{pgfscope}%
\pgfpathrectangle{\pgfqpoint{0.750000in}{0.500000in}}{\pgfqpoint{4.650000in}{3.020000in}}%
\pgfusepath{clip}%
\pgfsetbuttcap%
\pgfsetmiterjoin%
\definecolor{currentfill}{rgb}{1.000000,0.000000,0.000000}%
\pgfsetfillcolor{currentfill}%
\pgfsetlinewidth{0.000000pt}%
\definecolor{currentstroke}{rgb}{0.000000,0.000000,0.000000}%
\pgfsetstrokecolor{currentstroke}%
\pgfsetstrokeopacity{0.000000}%
\pgfsetdash{}{0pt}%
\pgfpathmoveto{\pgfqpoint{3.696997in}{0.500000in}}%
\pgfpathlineto{\pgfqpoint{3.721275in}{0.500000in}}%
\pgfpathlineto{\pgfqpoint{3.721275in}{0.500000in}}%
\pgfpathlineto{\pgfqpoint{3.696997in}{0.500000in}}%
\pgfpathlineto{\pgfqpoint{3.696997in}{0.500000in}}%
\pgfpathclose%
\pgfusepath{fill}%
\end{pgfscope}%
\begin{pgfscope}%
\pgfpathrectangle{\pgfqpoint{0.750000in}{0.500000in}}{\pgfqpoint{4.650000in}{3.020000in}}%
\pgfusepath{clip}%
\pgfsetbuttcap%
\pgfsetmiterjoin%
\definecolor{currentfill}{rgb}{1.000000,0.000000,0.000000}%
\pgfsetfillcolor{currentfill}%
\pgfsetlinewidth{0.000000pt}%
\definecolor{currentstroke}{rgb}{0.000000,0.000000,0.000000}%
\pgfsetstrokecolor{currentstroke}%
\pgfsetstrokeopacity{0.000000}%
\pgfsetdash{}{0pt}%
\pgfpathmoveto{\pgfqpoint{3.721275in}{0.500000in}}%
\pgfpathlineto{\pgfqpoint{3.745554in}{0.500000in}}%
\pgfpathlineto{\pgfqpoint{3.745554in}{0.605742in}}%
\pgfpathlineto{\pgfqpoint{3.721275in}{0.605742in}}%
\pgfpathlineto{\pgfqpoint{3.721275in}{0.500000in}}%
\pgfpathclose%
\pgfusepath{fill}%
\end{pgfscope}%
\begin{pgfscope}%
\pgfpathrectangle{\pgfqpoint{0.750000in}{0.500000in}}{\pgfqpoint{4.650000in}{3.020000in}}%
\pgfusepath{clip}%
\pgfsetbuttcap%
\pgfsetmiterjoin%
\definecolor{currentfill}{rgb}{1.000000,0.000000,0.000000}%
\pgfsetfillcolor{currentfill}%
\pgfsetlinewidth{0.000000pt}%
\definecolor{currentstroke}{rgb}{0.000000,0.000000,0.000000}%
\pgfsetstrokecolor{currentstroke}%
\pgfsetstrokeopacity{0.000000}%
\pgfsetdash{}{0pt}%
\pgfpathmoveto{\pgfqpoint{3.745554in}{0.500000in}}%
\pgfpathlineto{\pgfqpoint{3.769833in}{0.500000in}}%
\pgfpathlineto{\pgfqpoint{3.769833in}{0.500000in}}%
\pgfpathlineto{\pgfqpoint{3.745554in}{0.500000in}}%
\pgfpathlineto{\pgfqpoint{3.745554in}{0.500000in}}%
\pgfpathclose%
\pgfusepath{fill}%
\end{pgfscope}%
\begin{pgfscope}%
\pgfpathrectangle{\pgfqpoint{0.750000in}{0.500000in}}{\pgfqpoint{4.650000in}{3.020000in}}%
\pgfusepath{clip}%
\pgfsetbuttcap%
\pgfsetmiterjoin%
\definecolor{currentfill}{rgb}{1.000000,0.000000,0.000000}%
\pgfsetfillcolor{currentfill}%
\pgfsetlinewidth{0.000000pt}%
\definecolor{currentstroke}{rgb}{0.000000,0.000000,0.000000}%
\pgfsetstrokecolor{currentstroke}%
\pgfsetstrokeopacity{0.000000}%
\pgfsetdash{}{0pt}%
\pgfpathmoveto{\pgfqpoint{3.769833in}{0.500000in}}%
\pgfpathlineto{\pgfqpoint{3.794111in}{0.500000in}}%
\pgfpathlineto{\pgfqpoint{3.794111in}{0.557403in}}%
\pgfpathlineto{\pgfqpoint{3.769833in}{0.557403in}}%
\pgfpathlineto{\pgfqpoint{3.769833in}{0.500000in}}%
\pgfpathclose%
\pgfusepath{fill}%
\end{pgfscope}%
\begin{pgfscope}%
\pgfpathrectangle{\pgfqpoint{0.750000in}{0.500000in}}{\pgfqpoint{4.650000in}{3.020000in}}%
\pgfusepath{clip}%
\pgfsetbuttcap%
\pgfsetmiterjoin%
\definecolor{currentfill}{rgb}{1.000000,0.000000,0.000000}%
\pgfsetfillcolor{currentfill}%
\pgfsetlinewidth{0.000000pt}%
\definecolor{currentstroke}{rgb}{0.000000,0.000000,0.000000}%
\pgfsetstrokecolor{currentstroke}%
\pgfsetstrokeopacity{0.000000}%
\pgfsetdash{}{0pt}%
\pgfpathmoveto{\pgfqpoint{3.794111in}{0.500000in}}%
\pgfpathlineto{\pgfqpoint{3.818390in}{0.500000in}}%
\pgfpathlineto{\pgfqpoint{3.818390in}{0.500000in}}%
\pgfpathlineto{\pgfqpoint{3.794111in}{0.500000in}}%
\pgfpathlineto{\pgfqpoint{3.794111in}{0.500000in}}%
\pgfpathclose%
\pgfusepath{fill}%
\end{pgfscope}%
\begin{pgfscope}%
\pgfpathrectangle{\pgfqpoint{0.750000in}{0.500000in}}{\pgfqpoint{4.650000in}{3.020000in}}%
\pgfusepath{clip}%
\pgfsetbuttcap%
\pgfsetmiterjoin%
\definecolor{currentfill}{rgb}{1.000000,0.000000,0.000000}%
\pgfsetfillcolor{currentfill}%
\pgfsetlinewidth{0.000000pt}%
\definecolor{currentstroke}{rgb}{0.000000,0.000000,0.000000}%
\pgfsetstrokecolor{currentstroke}%
\pgfsetstrokeopacity{0.000000}%
\pgfsetdash{}{0pt}%
\pgfpathmoveto{\pgfqpoint{3.818390in}{0.500000in}}%
\pgfpathlineto{\pgfqpoint{3.842668in}{0.500000in}}%
\pgfpathlineto{\pgfqpoint{3.842668in}{0.503021in}}%
\pgfpathlineto{\pgfqpoint{3.818390in}{0.503021in}}%
\pgfpathlineto{\pgfqpoint{3.818390in}{0.500000in}}%
\pgfpathclose%
\pgfusepath{fill}%
\end{pgfscope}%
\begin{pgfscope}%
\pgfpathrectangle{\pgfqpoint{0.750000in}{0.500000in}}{\pgfqpoint{4.650000in}{3.020000in}}%
\pgfusepath{clip}%
\pgfsetbuttcap%
\pgfsetmiterjoin%
\definecolor{currentfill}{rgb}{1.000000,0.000000,0.000000}%
\pgfsetfillcolor{currentfill}%
\pgfsetlinewidth{0.000000pt}%
\definecolor{currentstroke}{rgb}{0.000000,0.000000,0.000000}%
\pgfsetstrokecolor{currentstroke}%
\pgfsetstrokeopacity{0.000000}%
\pgfsetdash{}{0pt}%
\pgfpathmoveto{\pgfqpoint{3.842668in}{0.500000in}}%
\pgfpathlineto{\pgfqpoint{3.866947in}{0.500000in}}%
\pgfpathlineto{\pgfqpoint{3.866947in}{0.512085in}}%
\pgfpathlineto{\pgfqpoint{3.842668in}{0.512085in}}%
\pgfpathlineto{\pgfqpoint{3.842668in}{0.500000in}}%
\pgfpathclose%
\pgfusepath{fill}%
\end{pgfscope}%
\begin{pgfscope}%
\pgfpathrectangle{\pgfqpoint{0.750000in}{0.500000in}}{\pgfqpoint{4.650000in}{3.020000in}}%
\pgfusepath{clip}%
\pgfsetbuttcap%
\pgfsetmiterjoin%
\definecolor{currentfill}{rgb}{1.000000,0.000000,0.000000}%
\pgfsetfillcolor{currentfill}%
\pgfsetlinewidth{0.000000pt}%
\definecolor{currentstroke}{rgb}{0.000000,0.000000,0.000000}%
\pgfsetstrokecolor{currentstroke}%
\pgfsetstrokeopacity{0.000000}%
\pgfsetdash{}{0pt}%
\pgfpathmoveto{\pgfqpoint{3.866947in}{0.500000in}}%
\pgfpathlineto{\pgfqpoint{3.891226in}{0.500000in}}%
\pgfpathlineto{\pgfqpoint{3.891226in}{0.500000in}}%
\pgfpathlineto{\pgfqpoint{3.866947in}{0.500000in}}%
\pgfpathlineto{\pgfqpoint{3.866947in}{0.500000in}}%
\pgfpathclose%
\pgfusepath{fill}%
\end{pgfscope}%
\begin{pgfscope}%
\pgfpathrectangle{\pgfqpoint{0.750000in}{0.500000in}}{\pgfqpoint{4.650000in}{3.020000in}}%
\pgfusepath{clip}%
\pgfsetbuttcap%
\pgfsetmiterjoin%
\definecolor{currentfill}{rgb}{1.000000,0.000000,0.000000}%
\pgfsetfillcolor{currentfill}%
\pgfsetlinewidth{0.000000pt}%
\definecolor{currentstroke}{rgb}{0.000000,0.000000,0.000000}%
\pgfsetstrokecolor{currentstroke}%
\pgfsetstrokeopacity{0.000000}%
\pgfsetdash{}{0pt}%
\pgfpathmoveto{\pgfqpoint{3.891226in}{0.500000in}}%
\pgfpathlineto{\pgfqpoint{3.915504in}{0.500000in}}%
\pgfpathlineto{\pgfqpoint{3.915504in}{0.500000in}}%
\pgfpathlineto{\pgfqpoint{3.891226in}{0.500000in}}%
\pgfpathlineto{\pgfqpoint{3.891226in}{0.500000in}}%
\pgfpathclose%
\pgfusepath{fill}%
\end{pgfscope}%
\begin{pgfscope}%
\pgfpathrectangle{\pgfqpoint{0.750000in}{0.500000in}}{\pgfqpoint{4.650000in}{3.020000in}}%
\pgfusepath{clip}%
\pgfsetbuttcap%
\pgfsetmiterjoin%
\definecolor{currentfill}{rgb}{1.000000,0.000000,0.000000}%
\pgfsetfillcolor{currentfill}%
\pgfsetlinewidth{0.000000pt}%
\definecolor{currentstroke}{rgb}{0.000000,0.000000,0.000000}%
\pgfsetstrokecolor{currentstroke}%
\pgfsetstrokeopacity{0.000000}%
\pgfsetdash{}{0pt}%
\pgfpathmoveto{\pgfqpoint{3.915504in}{0.500000in}}%
\pgfpathlineto{\pgfqpoint{3.939783in}{0.500000in}}%
\pgfpathlineto{\pgfqpoint{3.939783in}{0.509064in}}%
\pgfpathlineto{\pgfqpoint{3.915504in}{0.509064in}}%
\pgfpathlineto{\pgfqpoint{3.915504in}{0.500000in}}%
\pgfpathclose%
\pgfusepath{fill}%
\end{pgfscope}%
\begin{pgfscope}%
\pgfpathrectangle{\pgfqpoint{0.750000in}{0.500000in}}{\pgfqpoint{4.650000in}{3.020000in}}%
\pgfusepath{clip}%
\pgfsetbuttcap%
\pgfsetmiterjoin%
\definecolor{currentfill}{rgb}{1.000000,0.000000,0.000000}%
\pgfsetfillcolor{currentfill}%
\pgfsetlinewidth{0.000000pt}%
\definecolor{currentstroke}{rgb}{0.000000,0.000000,0.000000}%
\pgfsetstrokecolor{currentstroke}%
\pgfsetstrokeopacity{0.000000}%
\pgfsetdash{}{0pt}%
\pgfpathmoveto{\pgfqpoint{3.939783in}{0.500000in}}%
\pgfpathlineto{\pgfqpoint{3.964061in}{0.500000in}}%
\pgfpathlineto{\pgfqpoint{3.964061in}{0.500000in}}%
\pgfpathlineto{\pgfqpoint{3.939783in}{0.500000in}}%
\pgfpathlineto{\pgfqpoint{3.939783in}{0.500000in}}%
\pgfpathclose%
\pgfusepath{fill}%
\end{pgfscope}%
\begin{pgfscope}%
\pgfpathrectangle{\pgfqpoint{0.750000in}{0.500000in}}{\pgfqpoint{4.650000in}{3.020000in}}%
\pgfusepath{clip}%
\pgfsetbuttcap%
\pgfsetmiterjoin%
\definecolor{currentfill}{rgb}{1.000000,0.000000,0.000000}%
\pgfsetfillcolor{currentfill}%
\pgfsetlinewidth{0.000000pt}%
\definecolor{currentstroke}{rgb}{0.000000,0.000000,0.000000}%
\pgfsetstrokecolor{currentstroke}%
\pgfsetstrokeopacity{0.000000}%
\pgfsetdash{}{0pt}%
\pgfpathmoveto{\pgfqpoint{3.964061in}{0.500000in}}%
\pgfpathlineto{\pgfqpoint{3.988340in}{0.500000in}}%
\pgfpathlineto{\pgfqpoint{3.988340in}{0.500000in}}%
\pgfpathlineto{\pgfqpoint{3.964061in}{0.500000in}}%
\pgfpathlineto{\pgfqpoint{3.964061in}{0.500000in}}%
\pgfpathclose%
\pgfusepath{fill}%
\end{pgfscope}%
\begin{pgfscope}%
\pgfpathrectangle{\pgfqpoint{0.750000in}{0.500000in}}{\pgfqpoint{4.650000in}{3.020000in}}%
\pgfusepath{clip}%
\pgfsetbuttcap%
\pgfsetmiterjoin%
\definecolor{currentfill}{rgb}{1.000000,0.000000,0.000000}%
\pgfsetfillcolor{currentfill}%
\pgfsetlinewidth{0.000000pt}%
\definecolor{currentstroke}{rgb}{0.000000,0.000000,0.000000}%
\pgfsetstrokecolor{currentstroke}%
\pgfsetstrokeopacity{0.000000}%
\pgfsetdash{}{0pt}%
\pgfpathmoveto{\pgfqpoint{3.988340in}{0.500000in}}%
\pgfpathlineto{\pgfqpoint{4.012618in}{0.500000in}}%
\pgfpathlineto{\pgfqpoint{4.012618in}{0.503021in}}%
\pgfpathlineto{\pgfqpoint{3.988340in}{0.503021in}}%
\pgfpathlineto{\pgfqpoint{3.988340in}{0.500000in}}%
\pgfpathclose%
\pgfusepath{fill}%
\end{pgfscope}%
\begin{pgfscope}%
\pgfpathrectangle{\pgfqpoint{0.750000in}{0.500000in}}{\pgfqpoint{4.650000in}{3.020000in}}%
\pgfusepath{clip}%
\pgfsetbuttcap%
\pgfsetmiterjoin%
\definecolor{currentfill}{rgb}{1.000000,0.000000,0.000000}%
\pgfsetfillcolor{currentfill}%
\pgfsetlinewidth{0.000000pt}%
\definecolor{currentstroke}{rgb}{0.000000,0.000000,0.000000}%
\pgfsetstrokecolor{currentstroke}%
\pgfsetstrokeopacity{0.000000}%
\pgfsetdash{}{0pt}%
\pgfpathmoveto{\pgfqpoint{4.012618in}{0.500000in}}%
\pgfpathlineto{\pgfqpoint{4.036897in}{0.500000in}}%
\pgfpathlineto{\pgfqpoint{4.036897in}{0.503021in}}%
\pgfpathlineto{\pgfqpoint{4.012618in}{0.503021in}}%
\pgfpathlineto{\pgfqpoint{4.012618in}{0.500000in}}%
\pgfpathclose%
\pgfusepath{fill}%
\end{pgfscope}%
\begin{pgfscope}%
\pgfpathrectangle{\pgfqpoint{0.750000in}{0.500000in}}{\pgfqpoint{4.650000in}{3.020000in}}%
\pgfusepath{clip}%
\pgfsetbuttcap%
\pgfsetmiterjoin%
\definecolor{currentfill}{rgb}{1.000000,0.000000,0.000000}%
\pgfsetfillcolor{currentfill}%
\pgfsetlinewidth{0.000000pt}%
\definecolor{currentstroke}{rgb}{0.000000,0.000000,0.000000}%
\pgfsetstrokecolor{currentstroke}%
\pgfsetstrokeopacity{0.000000}%
\pgfsetdash{}{0pt}%
\pgfpathmoveto{\pgfqpoint{4.036897in}{0.500000in}}%
\pgfpathlineto{\pgfqpoint{4.061176in}{0.500000in}}%
\pgfpathlineto{\pgfqpoint{4.061176in}{0.503021in}}%
\pgfpathlineto{\pgfqpoint{4.036897in}{0.503021in}}%
\pgfpathlineto{\pgfqpoint{4.036897in}{0.500000in}}%
\pgfpathclose%
\pgfusepath{fill}%
\end{pgfscope}%
\begin{pgfscope}%
\pgfpathrectangle{\pgfqpoint{0.750000in}{0.500000in}}{\pgfqpoint{4.650000in}{3.020000in}}%
\pgfusepath{clip}%
\pgfsetbuttcap%
\pgfsetmiterjoin%
\definecolor{currentfill}{rgb}{1.000000,0.000000,0.000000}%
\pgfsetfillcolor{currentfill}%
\pgfsetlinewidth{0.000000pt}%
\definecolor{currentstroke}{rgb}{0.000000,0.000000,0.000000}%
\pgfsetstrokecolor{currentstroke}%
\pgfsetstrokeopacity{0.000000}%
\pgfsetdash{}{0pt}%
\pgfpathmoveto{\pgfqpoint{4.061176in}{0.500000in}}%
\pgfpathlineto{\pgfqpoint{4.085454in}{0.500000in}}%
\pgfpathlineto{\pgfqpoint{4.085454in}{0.503021in}}%
\pgfpathlineto{\pgfqpoint{4.061176in}{0.503021in}}%
\pgfpathlineto{\pgfqpoint{4.061176in}{0.500000in}}%
\pgfpathclose%
\pgfusepath{fill}%
\end{pgfscope}%
\begin{pgfscope}%
\pgfpathrectangle{\pgfqpoint{0.750000in}{0.500000in}}{\pgfqpoint{4.650000in}{3.020000in}}%
\pgfusepath{clip}%
\pgfsetbuttcap%
\pgfsetmiterjoin%
\definecolor{currentfill}{rgb}{1.000000,0.000000,0.000000}%
\pgfsetfillcolor{currentfill}%
\pgfsetlinewidth{0.000000pt}%
\definecolor{currentstroke}{rgb}{0.000000,0.000000,0.000000}%
\pgfsetstrokecolor{currentstroke}%
\pgfsetstrokeopacity{0.000000}%
\pgfsetdash{}{0pt}%
\pgfpathmoveto{\pgfqpoint{4.085454in}{0.500000in}}%
\pgfpathlineto{\pgfqpoint{4.109733in}{0.500000in}}%
\pgfpathlineto{\pgfqpoint{4.109733in}{0.500000in}}%
\pgfpathlineto{\pgfqpoint{4.085454in}{0.500000in}}%
\pgfpathlineto{\pgfqpoint{4.085454in}{0.500000in}}%
\pgfpathclose%
\pgfusepath{fill}%
\end{pgfscope}%
\begin{pgfscope}%
\pgfpathrectangle{\pgfqpoint{0.750000in}{0.500000in}}{\pgfqpoint{4.650000in}{3.020000in}}%
\pgfusepath{clip}%
\pgfsetbuttcap%
\pgfsetmiterjoin%
\definecolor{currentfill}{rgb}{1.000000,0.000000,0.000000}%
\pgfsetfillcolor{currentfill}%
\pgfsetlinewidth{0.000000pt}%
\definecolor{currentstroke}{rgb}{0.000000,0.000000,0.000000}%
\pgfsetstrokecolor{currentstroke}%
\pgfsetstrokeopacity{0.000000}%
\pgfsetdash{}{0pt}%
\pgfpathmoveto{\pgfqpoint{4.109733in}{0.500000in}}%
\pgfpathlineto{\pgfqpoint{4.134011in}{0.500000in}}%
\pgfpathlineto{\pgfqpoint{4.134011in}{0.500000in}}%
\pgfpathlineto{\pgfqpoint{4.109733in}{0.500000in}}%
\pgfpathlineto{\pgfqpoint{4.109733in}{0.500000in}}%
\pgfpathclose%
\pgfusepath{fill}%
\end{pgfscope}%
\begin{pgfscope}%
\pgfpathrectangle{\pgfqpoint{0.750000in}{0.500000in}}{\pgfqpoint{4.650000in}{3.020000in}}%
\pgfusepath{clip}%
\pgfsetbuttcap%
\pgfsetmiterjoin%
\definecolor{currentfill}{rgb}{1.000000,0.000000,0.000000}%
\pgfsetfillcolor{currentfill}%
\pgfsetlinewidth{0.000000pt}%
\definecolor{currentstroke}{rgb}{0.000000,0.000000,0.000000}%
\pgfsetstrokecolor{currentstroke}%
\pgfsetstrokeopacity{0.000000}%
\pgfsetdash{}{0pt}%
\pgfpathmoveto{\pgfqpoint{4.134011in}{0.500000in}}%
\pgfpathlineto{\pgfqpoint{4.158290in}{0.500000in}}%
\pgfpathlineto{\pgfqpoint{4.158290in}{0.500000in}}%
\pgfpathlineto{\pgfqpoint{4.134011in}{0.500000in}}%
\pgfpathlineto{\pgfqpoint{4.134011in}{0.500000in}}%
\pgfpathclose%
\pgfusepath{fill}%
\end{pgfscope}%
\begin{pgfscope}%
\pgfpathrectangle{\pgfqpoint{0.750000in}{0.500000in}}{\pgfqpoint{4.650000in}{3.020000in}}%
\pgfusepath{clip}%
\pgfsetbuttcap%
\pgfsetmiterjoin%
\definecolor{currentfill}{rgb}{1.000000,0.000000,0.000000}%
\pgfsetfillcolor{currentfill}%
\pgfsetlinewidth{0.000000pt}%
\definecolor{currentstroke}{rgb}{0.000000,0.000000,0.000000}%
\pgfsetstrokecolor{currentstroke}%
\pgfsetstrokeopacity{0.000000}%
\pgfsetdash{}{0pt}%
\pgfpathmoveto{\pgfqpoint{4.158290in}{0.500000in}}%
\pgfpathlineto{\pgfqpoint{4.182569in}{0.500000in}}%
\pgfpathlineto{\pgfqpoint{4.182569in}{0.500000in}}%
\pgfpathlineto{\pgfqpoint{4.158290in}{0.500000in}}%
\pgfpathlineto{\pgfqpoint{4.158290in}{0.500000in}}%
\pgfpathclose%
\pgfusepath{fill}%
\end{pgfscope}%
\begin{pgfscope}%
\pgfpathrectangle{\pgfqpoint{0.750000in}{0.500000in}}{\pgfqpoint{4.650000in}{3.020000in}}%
\pgfusepath{clip}%
\pgfsetbuttcap%
\pgfsetmiterjoin%
\definecolor{currentfill}{rgb}{1.000000,0.000000,0.000000}%
\pgfsetfillcolor{currentfill}%
\pgfsetlinewidth{0.000000pt}%
\definecolor{currentstroke}{rgb}{0.000000,0.000000,0.000000}%
\pgfsetstrokecolor{currentstroke}%
\pgfsetstrokeopacity{0.000000}%
\pgfsetdash{}{0pt}%
\pgfpathmoveto{\pgfqpoint{4.182569in}{0.500000in}}%
\pgfpathlineto{\pgfqpoint{4.206847in}{0.500000in}}%
\pgfpathlineto{\pgfqpoint{4.206847in}{0.500000in}}%
\pgfpathlineto{\pgfqpoint{4.182569in}{0.500000in}}%
\pgfpathlineto{\pgfqpoint{4.182569in}{0.500000in}}%
\pgfpathclose%
\pgfusepath{fill}%
\end{pgfscope}%
\begin{pgfscope}%
\pgfpathrectangle{\pgfqpoint{0.750000in}{0.500000in}}{\pgfqpoint{4.650000in}{3.020000in}}%
\pgfusepath{clip}%
\pgfsetbuttcap%
\pgfsetmiterjoin%
\definecolor{currentfill}{rgb}{1.000000,0.000000,0.000000}%
\pgfsetfillcolor{currentfill}%
\pgfsetlinewidth{0.000000pt}%
\definecolor{currentstroke}{rgb}{0.000000,0.000000,0.000000}%
\pgfsetstrokecolor{currentstroke}%
\pgfsetstrokeopacity{0.000000}%
\pgfsetdash{}{0pt}%
\pgfpathmoveto{\pgfqpoint{4.206847in}{0.500000in}}%
\pgfpathlineto{\pgfqpoint{4.231126in}{0.500000in}}%
\pgfpathlineto{\pgfqpoint{4.231126in}{0.500000in}}%
\pgfpathlineto{\pgfqpoint{4.206847in}{0.500000in}}%
\pgfpathlineto{\pgfqpoint{4.206847in}{0.500000in}}%
\pgfpathclose%
\pgfusepath{fill}%
\end{pgfscope}%
\begin{pgfscope}%
\pgfpathrectangle{\pgfqpoint{0.750000in}{0.500000in}}{\pgfqpoint{4.650000in}{3.020000in}}%
\pgfusepath{clip}%
\pgfsetbuttcap%
\pgfsetmiterjoin%
\definecolor{currentfill}{rgb}{1.000000,0.000000,0.000000}%
\pgfsetfillcolor{currentfill}%
\pgfsetlinewidth{0.000000pt}%
\definecolor{currentstroke}{rgb}{0.000000,0.000000,0.000000}%
\pgfsetstrokecolor{currentstroke}%
\pgfsetstrokeopacity{0.000000}%
\pgfsetdash{}{0pt}%
\pgfpathmoveto{\pgfqpoint{4.231126in}{0.500000in}}%
\pgfpathlineto{\pgfqpoint{4.255404in}{0.500000in}}%
\pgfpathlineto{\pgfqpoint{4.255404in}{0.503021in}}%
\pgfpathlineto{\pgfqpoint{4.231126in}{0.503021in}}%
\pgfpathlineto{\pgfqpoint{4.231126in}{0.500000in}}%
\pgfpathclose%
\pgfusepath{fill}%
\end{pgfscope}%
\begin{pgfscope}%
\pgfpathrectangle{\pgfqpoint{0.750000in}{0.500000in}}{\pgfqpoint{4.650000in}{3.020000in}}%
\pgfusepath{clip}%
\pgfsetbuttcap%
\pgfsetmiterjoin%
\definecolor{currentfill}{rgb}{1.000000,0.000000,0.000000}%
\pgfsetfillcolor{currentfill}%
\pgfsetlinewidth{0.000000pt}%
\definecolor{currentstroke}{rgb}{0.000000,0.000000,0.000000}%
\pgfsetstrokecolor{currentstroke}%
\pgfsetstrokeopacity{0.000000}%
\pgfsetdash{}{0pt}%
\pgfpathmoveto{\pgfqpoint{4.255404in}{0.500000in}}%
\pgfpathlineto{\pgfqpoint{4.279683in}{0.500000in}}%
\pgfpathlineto{\pgfqpoint{4.279683in}{0.500000in}}%
\pgfpathlineto{\pgfqpoint{4.255404in}{0.500000in}}%
\pgfpathlineto{\pgfqpoint{4.255404in}{0.500000in}}%
\pgfpathclose%
\pgfusepath{fill}%
\end{pgfscope}%
\begin{pgfscope}%
\pgfpathrectangle{\pgfqpoint{0.750000in}{0.500000in}}{\pgfqpoint{4.650000in}{3.020000in}}%
\pgfusepath{clip}%
\pgfsetbuttcap%
\pgfsetmiterjoin%
\definecolor{currentfill}{rgb}{1.000000,0.000000,0.000000}%
\pgfsetfillcolor{currentfill}%
\pgfsetlinewidth{0.000000pt}%
\definecolor{currentstroke}{rgb}{0.000000,0.000000,0.000000}%
\pgfsetstrokecolor{currentstroke}%
\pgfsetstrokeopacity{0.000000}%
\pgfsetdash{}{0pt}%
\pgfpathmoveto{\pgfqpoint{4.279683in}{0.500000in}}%
\pgfpathlineto{\pgfqpoint{4.303961in}{0.500000in}}%
\pgfpathlineto{\pgfqpoint{4.303961in}{0.500000in}}%
\pgfpathlineto{\pgfqpoint{4.279683in}{0.500000in}}%
\pgfpathlineto{\pgfqpoint{4.279683in}{0.500000in}}%
\pgfpathclose%
\pgfusepath{fill}%
\end{pgfscope}%
\begin{pgfscope}%
\pgfpathrectangle{\pgfqpoint{0.750000in}{0.500000in}}{\pgfqpoint{4.650000in}{3.020000in}}%
\pgfusepath{clip}%
\pgfsetbuttcap%
\pgfsetmiterjoin%
\definecolor{currentfill}{rgb}{1.000000,0.000000,0.000000}%
\pgfsetfillcolor{currentfill}%
\pgfsetlinewidth{0.000000pt}%
\definecolor{currentstroke}{rgb}{0.000000,0.000000,0.000000}%
\pgfsetstrokecolor{currentstroke}%
\pgfsetstrokeopacity{0.000000}%
\pgfsetdash{}{0pt}%
\pgfpathmoveto{\pgfqpoint{4.303961in}{0.500000in}}%
\pgfpathlineto{\pgfqpoint{4.328240in}{0.500000in}}%
\pgfpathlineto{\pgfqpoint{4.328240in}{0.500000in}}%
\pgfpathlineto{\pgfqpoint{4.303961in}{0.500000in}}%
\pgfpathlineto{\pgfqpoint{4.303961in}{0.500000in}}%
\pgfpathclose%
\pgfusepath{fill}%
\end{pgfscope}%
\begin{pgfscope}%
\pgfpathrectangle{\pgfqpoint{0.750000in}{0.500000in}}{\pgfqpoint{4.650000in}{3.020000in}}%
\pgfusepath{clip}%
\pgfsetbuttcap%
\pgfsetmiterjoin%
\definecolor{currentfill}{rgb}{1.000000,0.000000,0.000000}%
\pgfsetfillcolor{currentfill}%
\pgfsetlinewidth{0.000000pt}%
\definecolor{currentstroke}{rgb}{0.000000,0.000000,0.000000}%
\pgfsetstrokecolor{currentstroke}%
\pgfsetstrokeopacity{0.000000}%
\pgfsetdash{}{0pt}%
\pgfpathmoveto{\pgfqpoint{4.328240in}{0.500000in}}%
\pgfpathlineto{\pgfqpoint{4.352519in}{0.500000in}}%
\pgfpathlineto{\pgfqpoint{4.352519in}{0.500000in}}%
\pgfpathlineto{\pgfqpoint{4.328240in}{0.500000in}}%
\pgfpathlineto{\pgfqpoint{4.328240in}{0.500000in}}%
\pgfpathclose%
\pgfusepath{fill}%
\end{pgfscope}%
\begin{pgfscope}%
\pgfpathrectangle{\pgfqpoint{0.750000in}{0.500000in}}{\pgfqpoint{4.650000in}{3.020000in}}%
\pgfusepath{clip}%
\pgfsetbuttcap%
\pgfsetmiterjoin%
\definecolor{currentfill}{rgb}{1.000000,0.000000,0.000000}%
\pgfsetfillcolor{currentfill}%
\pgfsetlinewidth{0.000000pt}%
\definecolor{currentstroke}{rgb}{0.000000,0.000000,0.000000}%
\pgfsetstrokecolor{currentstroke}%
\pgfsetstrokeopacity{0.000000}%
\pgfsetdash{}{0pt}%
\pgfpathmoveto{\pgfqpoint{4.352519in}{0.500000in}}%
\pgfpathlineto{\pgfqpoint{4.376797in}{0.500000in}}%
\pgfpathlineto{\pgfqpoint{4.376797in}{0.500000in}}%
\pgfpathlineto{\pgfqpoint{4.352519in}{0.500000in}}%
\pgfpathlineto{\pgfqpoint{4.352519in}{0.500000in}}%
\pgfpathclose%
\pgfusepath{fill}%
\end{pgfscope}%
\begin{pgfscope}%
\pgfpathrectangle{\pgfqpoint{0.750000in}{0.500000in}}{\pgfqpoint{4.650000in}{3.020000in}}%
\pgfusepath{clip}%
\pgfsetbuttcap%
\pgfsetmiterjoin%
\definecolor{currentfill}{rgb}{1.000000,0.000000,0.000000}%
\pgfsetfillcolor{currentfill}%
\pgfsetlinewidth{0.000000pt}%
\definecolor{currentstroke}{rgb}{0.000000,0.000000,0.000000}%
\pgfsetstrokecolor{currentstroke}%
\pgfsetstrokeopacity{0.000000}%
\pgfsetdash{}{0pt}%
\pgfpathmoveto{\pgfqpoint{4.376797in}{0.500000in}}%
\pgfpathlineto{\pgfqpoint{4.401076in}{0.500000in}}%
\pgfpathlineto{\pgfqpoint{4.401076in}{0.500000in}}%
\pgfpathlineto{\pgfqpoint{4.376797in}{0.500000in}}%
\pgfpathlineto{\pgfqpoint{4.376797in}{0.500000in}}%
\pgfpathclose%
\pgfusepath{fill}%
\end{pgfscope}%
\begin{pgfscope}%
\pgfpathrectangle{\pgfqpoint{0.750000in}{0.500000in}}{\pgfqpoint{4.650000in}{3.020000in}}%
\pgfusepath{clip}%
\pgfsetbuttcap%
\pgfsetmiterjoin%
\definecolor{currentfill}{rgb}{1.000000,0.000000,0.000000}%
\pgfsetfillcolor{currentfill}%
\pgfsetlinewidth{0.000000pt}%
\definecolor{currentstroke}{rgb}{0.000000,0.000000,0.000000}%
\pgfsetstrokecolor{currentstroke}%
\pgfsetstrokeopacity{0.000000}%
\pgfsetdash{}{0pt}%
\pgfpathmoveto{\pgfqpoint{4.401076in}{0.500000in}}%
\pgfpathlineto{\pgfqpoint{4.425354in}{0.500000in}}%
\pgfpathlineto{\pgfqpoint{4.425354in}{0.500000in}}%
\pgfpathlineto{\pgfqpoint{4.401076in}{0.500000in}}%
\pgfpathlineto{\pgfqpoint{4.401076in}{0.500000in}}%
\pgfpathclose%
\pgfusepath{fill}%
\end{pgfscope}%
\begin{pgfscope}%
\pgfpathrectangle{\pgfqpoint{0.750000in}{0.500000in}}{\pgfqpoint{4.650000in}{3.020000in}}%
\pgfusepath{clip}%
\pgfsetbuttcap%
\pgfsetmiterjoin%
\definecolor{currentfill}{rgb}{1.000000,0.000000,0.000000}%
\pgfsetfillcolor{currentfill}%
\pgfsetlinewidth{0.000000pt}%
\definecolor{currentstroke}{rgb}{0.000000,0.000000,0.000000}%
\pgfsetstrokecolor{currentstroke}%
\pgfsetstrokeopacity{0.000000}%
\pgfsetdash{}{0pt}%
\pgfpathmoveto{\pgfqpoint{4.425354in}{0.500000in}}%
\pgfpathlineto{\pgfqpoint{4.449633in}{0.500000in}}%
\pgfpathlineto{\pgfqpoint{4.449633in}{0.500000in}}%
\pgfpathlineto{\pgfqpoint{4.425354in}{0.500000in}}%
\pgfpathlineto{\pgfqpoint{4.425354in}{0.500000in}}%
\pgfpathclose%
\pgfusepath{fill}%
\end{pgfscope}%
\begin{pgfscope}%
\pgfpathrectangle{\pgfqpoint{0.750000in}{0.500000in}}{\pgfqpoint{4.650000in}{3.020000in}}%
\pgfusepath{clip}%
\pgfsetbuttcap%
\pgfsetmiterjoin%
\definecolor{currentfill}{rgb}{1.000000,0.000000,0.000000}%
\pgfsetfillcolor{currentfill}%
\pgfsetlinewidth{0.000000pt}%
\definecolor{currentstroke}{rgb}{0.000000,0.000000,0.000000}%
\pgfsetstrokecolor{currentstroke}%
\pgfsetstrokeopacity{0.000000}%
\pgfsetdash{}{0pt}%
\pgfpathmoveto{\pgfqpoint{4.449633in}{0.500000in}}%
\pgfpathlineto{\pgfqpoint{4.473912in}{0.500000in}}%
\pgfpathlineto{\pgfqpoint{4.473912in}{0.500000in}}%
\pgfpathlineto{\pgfqpoint{4.449633in}{0.500000in}}%
\pgfpathlineto{\pgfqpoint{4.449633in}{0.500000in}}%
\pgfpathclose%
\pgfusepath{fill}%
\end{pgfscope}%
\begin{pgfscope}%
\pgfpathrectangle{\pgfqpoint{0.750000in}{0.500000in}}{\pgfqpoint{4.650000in}{3.020000in}}%
\pgfusepath{clip}%
\pgfsetbuttcap%
\pgfsetmiterjoin%
\definecolor{currentfill}{rgb}{1.000000,0.000000,0.000000}%
\pgfsetfillcolor{currentfill}%
\pgfsetlinewidth{0.000000pt}%
\definecolor{currentstroke}{rgb}{0.000000,0.000000,0.000000}%
\pgfsetstrokecolor{currentstroke}%
\pgfsetstrokeopacity{0.000000}%
\pgfsetdash{}{0pt}%
\pgfpathmoveto{\pgfqpoint{4.473912in}{0.500000in}}%
\pgfpathlineto{\pgfqpoint{4.498190in}{0.500000in}}%
\pgfpathlineto{\pgfqpoint{4.498190in}{0.500000in}}%
\pgfpathlineto{\pgfqpoint{4.473912in}{0.500000in}}%
\pgfpathlineto{\pgfqpoint{4.473912in}{0.500000in}}%
\pgfpathclose%
\pgfusepath{fill}%
\end{pgfscope}%
\begin{pgfscope}%
\pgfpathrectangle{\pgfqpoint{0.750000in}{0.500000in}}{\pgfqpoint{4.650000in}{3.020000in}}%
\pgfusepath{clip}%
\pgfsetbuttcap%
\pgfsetmiterjoin%
\definecolor{currentfill}{rgb}{1.000000,0.000000,0.000000}%
\pgfsetfillcolor{currentfill}%
\pgfsetlinewidth{0.000000pt}%
\definecolor{currentstroke}{rgb}{0.000000,0.000000,0.000000}%
\pgfsetstrokecolor{currentstroke}%
\pgfsetstrokeopacity{0.000000}%
\pgfsetdash{}{0pt}%
\pgfpathmoveto{\pgfqpoint{4.498190in}{0.500000in}}%
\pgfpathlineto{\pgfqpoint{4.522469in}{0.500000in}}%
\pgfpathlineto{\pgfqpoint{4.522469in}{0.500000in}}%
\pgfpathlineto{\pgfqpoint{4.498190in}{0.500000in}}%
\pgfpathlineto{\pgfqpoint{4.498190in}{0.500000in}}%
\pgfpathclose%
\pgfusepath{fill}%
\end{pgfscope}%
\begin{pgfscope}%
\pgfpathrectangle{\pgfqpoint{0.750000in}{0.500000in}}{\pgfqpoint{4.650000in}{3.020000in}}%
\pgfusepath{clip}%
\pgfsetbuttcap%
\pgfsetmiterjoin%
\definecolor{currentfill}{rgb}{1.000000,0.000000,0.000000}%
\pgfsetfillcolor{currentfill}%
\pgfsetlinewidth{0.000000pt}%
\definecolor{currentstroke}{rgb}{0.000000,0.000000,0.000000}%
\pgfsetstrokecolor{currentstroke}%
\pgfsetstrokeopacity{0.000000}%
\pgfsetdash{}{0pt}%
\pgfpathmoveto{\pgfqpoint{4.522469in}{0.500000in}}%
\pgfpathlineto{\pgfqpoint{4.546747in}{0.500000in}}%
\pgfpathlineto{\pgfqpoint{4.546747in}{0.500000in}}%
\pgfpathlineto{\pgfqpoint{4.522469in}{0.500000in}}%
\pgfpathlineto{\pgfqpoint{4.522469in}{0.500000in}}%
\pgfpathclose%
\pgfusepath{fill}%
\end{pgfscope}%
\begin{pgfscope}%
\pgfpathrectangle{\pgfqpoint{0.750000in}{0.500000in}}{\pgfqpoint{4.650000in}{3.020000in}}%
\pgfusepath{clip}%
\pgfsetbuttcap%
\pgfsetmiterjoin%
\definecolor{currentfill}{rgb}{1.000000,0.000000,0.000000}%
\pgfsetfillcolor{currentfill}%
\pgfsetlinewidth{0.000000pt}%
\definecolor{currentstroke}{rgb}{0.000000,0.000000,0.000000}%
\pgfsetstrokecolor{currentstroke}%
\pgfsetstrokeopacity{0.000000}%
\pgfsetdash{}{0pt}%
\pgfpathmoveto{\pgfqpoint{4.546747in}{0.500000in}}%
\pgfpathlineto{\pgfqpoint{4.571026in}{0.500000in}}%
\pgfpathlineto{\pgfqpoint{4.571026in}{0.503021in}}%
\pgfpathlineto{\pgfqpoint{4.546747in}{0.503021in}}%
\pgfpathlineto{\pgfqpoint{4.546747in}{0.500000in}}%
\pgfpathclose%
\pgfusepath{fill}%
\end{pgfscope}%
\begin{pgfscope}%
\pgfpathrectangle{\pgfqpoint{0.750000in}{0.500000in}}{\pgfqpoint{4.650000in}{3.020000in}}%
\pgfusepath{clip}%
\pgfsetbuttcap%
\pgfsetmiterjoin%
\definecolor{currentfill}{rgb}{1.000000,0.000000,0.000000}%
\pgfsetfillcolor{currentfill}%
\pgfsetlinewidth{0.000000pt}%
\definecolor{currentstroke}{rgb}{0.000000,0.000000,0.000000}%
\pgfsetstrokecolor{currentstroke}%
\pgfsetstrokeopacity{0.000000}%
\pgfsetdash{}{0pt}%
\pgfpathmoveto{\pgfqpoint{4.571026in}{0.500000in}}%
\pgfpathlineto{\pgfqpoint{4.595304in}{0.500000in}}%
\pgfpathlineto{\pgfqpoint{4.595304in}{0.500000in}}%
\pgfpathlineto{\pgfqpoint{4.571026in}{0.500000in}}%
\pgfpathlineto{\pgfqpoint{4.571026in}{0.500000in}}%
\pgfpathclose%
\pgfusepath{fill}%
\end{pgfscope}%
\begin{pgfscope}%
\pgfpathrectangle{\pgfqpoint{0.750000in}{0.500000in}}{\pgfqpoint{4.650000in}{3.020000in}}%
\pgfusepath{clip}%
\pgfsetbuttcap%
\pgfsetmiterjoin%
\definecolor{currentfill}{rgb}{1.000000,0.000000,0.000000}%
\pgfsetfillcolor{currentfill}%
\pgfsetlinewidth{0.000000pt}%
\definecolor{currentstroke}{rgb}{0.000000,0.000000,0.000000}%
\pgfsetstrokecolor{currentstroke}%
\pgfsetstrokeopacity{0.000000}%
\pgfsetdash{}{0pt}%
\pgfpathmoveto{\pgfqpoint{4.595304in}{0.500000in}}%
\pgfpathlineto{\pgfqpoint{4.619583in}{0.500000in}}%
\pgfpathlineto{\pgfqpoint{4.619583in}{0.500000in}}%
\pgfpathlineto{\pgfqpoint{4.595304in}{0.500000in}}%
\pgfpathlineto{\pgfqpoint{4.595304in}{0.500000in}}%
\pgfpathclose%
\pgfusepath{fill}%
\end{pgfscope}%
\begin{pgfscope}%
\pgfpathrectangle{\pgfqpoint{0.750000in}{0.500000in}}{\pgfqpoint{4.650000in}{3.020000in}}%
\pgfusepath{clip}%
\pgfsetbuttcap%
\pgfsetmiterjoin%
\definecolor{currentfill}{rgb}{1.000000,0.000000,0.000000}%
\pgfsetfillcolor{currentfill}%
\pgfsetlinewidth{0.000000pt}%
\definecolor{currentstroke}{rgb}{0.000000,0.000000,0.000000}%
\pgfsetstrokecolor{currentstroke}%
\pgfsetstrokeopacity{0.000000}%
\pgfsetdash{}{0pt}%
\pgfpathmoveto{\pgfqpoint{4.619583in}{0.500000in}}%
\pgfpathlineto{\pgfqpoint{4.643862in}{0.500000in}}%
\pgfpathlineto{\pgfqpoint{4.643862in}{0.503021in}}%
\pgfpathlineto{\pgfqpoint{4.619583in}{0.503021in}}%
\pgfpathlineto{\pgfqpoint{4.619583in}{0.500000in}}%
\pgfpathclose%
\pgfusepath{fill}%
\end{pgfscope}%
\begin{pgfscope}%
\pgfpathrectangle{\pgfqpoint{0.750000in}{0.500000in}}{\pgfqpoint{4.650000in}{3.020000in}}%
\pgfusepath{clip}%
\pgfsetbuttcap%
\pgfsetmiterjoin%
\definecolor{currentfill}{rgb}{1.000000,0.000000,0.000000}%
\pgfsetfillcolor{currentfill}%
\pgfsetlinewidth{0.000000pt}%
\definecolor{currentstroke}{rgb}{0.000000,0.000000,0.000000}%
\pgfsetstrokecolor{currentstroke}%
\pgfsetstrokeopacity{0.000000}%
\pgfsetdash{}{0pt}%
\pgfpathmoveto{\pgfqpoint{4.643862in}{0.500000in}}%
\pgfpathlineto{\pgfqpoint{4.668140in}{0.500000in}}%
\pgfpathlineto{\pgfqpoint{4.668140in}{0.500000in}}%
\pgfpathlineto{\pgfqpoint{4.643862in}{0.500000in}}%
\pgfpathlineto{\pgfqpoint{4.643862in}{0.500000in}}%
\pgfpathclose%
\pgfusepath{fill}%
\end{pgfscope}%
\begin{pgfscope}%
\pgfpathrectangle{\pgfqpoint{0.750000in}{0.500000in}}{\pgfqpoint{4.650000in}{3.020000in}}%
\pgfusepath{clip}%
\pgfsetbuttcap%
\pgfsetmiterjoin%
\definecolor{currentfill}{rgb}{1.000000,0.000000,0.000000}%
\pgfsetfillcolor{currentfill}%
\pgfsetlinewidth{0.000000pt}%
\definecolor{currentstroke}{rgb}{0.000000,0.000000,0.000000}%
\pgfsetstrokecolor{currentstroke}%
\pgfsetstrokeopacity{0.000000}%
\pgfsetdash{}{0pt}%
\pgfpathmoveto{\pgfqpoint{4.668140in}{0.500000in}}%
\pgfpathlineto{\pgfqpoint{4.692419in}{0.500000in}}%
\pgfpathlineto{\pgfqpoint{4.692419in}{0.503021in}}%
\pgfpathlineto{\pgfqpoint{4.668140in}{0.503021in}}%
\pgfpathlineto{\pgfqpoint{4.668140in}{0.500000in}}%
\pgfpathclose%
\pgfusepath{fill}%
\end{pgfscope}%
\begin{pgfscope}%
\pgfpathrectangle{\pgfqpoint{0.750000in}{0.500000in}}{\pgfqpoint{4.650000in}{3.020000in}}%
\pgfusepath{clip}%
\pgfsetbuttcap%
\pgfsetmiterjoin%
\definecolor{currentfill}{rgb}{1.000000,0.000000,0.000000}%
\pgfsetfillcolor{currentfill}%
\pgfsetlinewidth{0.000000pt}%
\definecolor{currentstroke}{rgb}{0.000000,0.000000,0.000000}%
\pgfsetstrokecolor{currentstroke}%
\pgfsetstrokeopacity{0.000000}%
\pgfsetdash{}{0pt}%
\pgfpathmoveto{\pgfqpoint{4.692419in}{0.500000in}}%
\pgfpathlineto{\pgfqpoint{4.716697in}{0.500000in}}%
\pgfpathlineto{\pgfqpoint{4.716697in}{0.500000in}}%
\pgfpathlineto{\pgfqpoint{4.692419in}{0.500000in}}%
\pgfpathlineto{\pgfqpoint{4.692419in}{0.500000in}}%
\pgfpathclose%
\pgfusepath{fill}%
\end{pgfscope}%
\begin{pgfscope}%
\pgfpathrectangle{\pgfqpoint{0.750000in}{0.500000in}}{\pgfqpoint{4.650000in}{3.020000in}}%
\pgfusepath{clip}%
\pgfsetbuttcap%
\pgfsetmiterjoin%
\definecolor{currentfill}{rgb}{1.000000,0.000000,0.000000}%
\pgfsetfillcolor{currentfill}%
\pgfsetlinewidth{0.000000pt}%
\definecolor{currentstroke}{rgb}{0.000000,0.000000,0.000000}%
\pgfsetstrokecolor{currentstroke}%
\pgfsetstrokeopacity{0.000000}%
\pgfsetdash{}{0pt}%
\pgfpathmoveto{\pgfqpoint{4.716697in}{0.500000in}}%
\pgfpathlineto{\pgfqpoint{4.740976in}{0.500000in}}%
\pgfpathlineto{\pgfqpoint{4.740976in}{0.500000in}}%
\pgfpathlineto{\pgfqpoint{4.716697in}{0.500000in}}%
\pgfpathlineto{\pgfqpoint{4.716697in}{0.500000in}}%
\pgfpathclose%
\pgfusepath{fill}%
\end{pgfscope}%
\begin{pgfscope}%
\pgfpathrectangle{\pgfqpoint{0.750000in}{0.500000in}}{\pgfqpoint{4.650000in}{3.020000in}}%
\pgfusepath{clip}%
\pgfsetbuttcap%
\pgfsetmiterjoin%
\definecolor{currentfill}{rgb}{1.000000,0.000000,0.000000}%
\pgfsetfillcolor{currentfill}%
\pgfsetlinewidth{0.000000pt}%
\definecolor{currentstroke}{rgb}{0.000000,0.000000,0.000000}%
\pgfsetstrokecolor{currentstroke}%
\pgfsetstrokeopacity{0.000000}%
\pgfsetdash{}{0pt}%
\pgfpathmoveto{\pgfqpoint{4.740976in}{0.500000in}}%
\pgfpathlineto{\pgfqpoint{4.765255in}{0.500000in}}%
\pgfpathlineto{\pgfqpoint{4.765255in}{0.500000in}}%
\pgfpathlineto{\pgfqpoint{4.740976in}{0.500000in}}%
\pgfpathlineto{\pgfqpoint{4.740976in}{0.500000in}}%
\pgfpathclose%
\pgfusepath{fill}%
\end{pgfscope}%
\begin{pgfscope}%
\pgfpathrectangle{\pgfqpoint{0.750000in}{0.500000in}}{\pgfqpoint{4.650000in}{3.020000in}}%
\pgfusepath{clip}%
\pgfsetbuttcap%
\pgfsetmiterjoin%
\definecolor{currentfill}{rgb}{1.000000,0.000000,0.000000}%
\pgfsetfillcolor{currentfill}%
\pgfsetlinewidth{0.000000pt}%
\definecolor{currentstroke}{rgb}{0.000000,0.000000,0.000000}%
\pgfsetstrokecolor{currentstroke}%
\pgfsetstrokeopacity{0.000000}%
\pgfsetdash{}{0pt}%
\pgfpathmoveto{\pgfqpoint{4.765255in}{0.500000in}}%
\pgfpathlineto{\pgfqpoint{4.789533in}{0.500000in}}%
\pgfpathlineto{\pgfqpoint{4.789533in}{0.500000in}}%
\pgfpathlineto{\pgfqpoint{4.765255in}{0.500000in}}%
\pgfpathlineto{\pgfqpoint{4.765255in}{0.500000in}}%
\pgfpathclose%
\pgfusepath{fill}%
\end{pgfscope}%
\begin{pgfscope}%
\pgfpathrectangle{\pgfqpoint{0.750000in}{0.500000in}}{\pgfqpoint{4.650000in}{3.020000in}}%
\pgfusepath{clip}%
\pgfsetbuttcap%
\pgfsetmiterjoin%
\definecolor{currentfill}{rgb}{1.000000,0.000000,0.000000}%
\pgfsetfillcolor{currentfill}%
\pgfsetlinewidth{0.000000pt}%
\definecolor{currentstroke}{rgb}{0.000000,0.000000,0.000000}%
\pgfsetstrokecolor{currentstroke}%
\pgfsetstrokeopacity{0.000000}%
\pgfsetdash{}{0pt}%
\pgfpathmoveto{\pgfqpoint{4.789533in}{0.500000in}}%
\pgfpathlineto{\pgfqpoint{4.813812in}{0.500000in}}%
\pgfpathlineto{\pgfqpoint{4.813812in}{0.503021in}}%
\pgfpathlineto{\pgfqpoint{4.789533in}{0.503021in}}%
\pgfpathlineto{\pgfqpoint{4.789533in}{0.500000in}}%
\pgfpathclose%
\pgfusepath{fill}%
\end{pgfscope}%
\begin{pgfscope}%
\pgfpathrectangle{\pgfqpoint{0.750000in}{0.500000in}}{\pgfqpoint{4.650000in}{3.020000in}}%
\pgfusepath{clip}%
\pgfsetbuttcap%
\pgfsetmiterjoin%
\definecolor{currentfill}{rgb}{1.000000,0.000000,0.000000}%
\pgfsetfillcolor{currentfill}%
\pgfsetlinewidth{0.000000pt}%
\definecolor{currentstroke}{rgb}{0.000000,0.000000,0.000000}%
\pgfsetstrokecolor{currentstroke}%
\pgfsetstrokeopacity{0.000000}%
\pgfsetdash{}{0pt}%
\pgfpathmoveto{\pgfqpoint{4.813812in}{0.500000in}}%
\pgfpathlineto{\pgfqpoint{4.838090in}{0.500000in}}%
\pgfpathlineto{\pgfqpoint{4.838090in}{0.500000in}}%
\pgfpathlineto{\pgfqpoint{4.813812in}{0.500000in}}%
\pgfpathlineto{\pgfqpoint{4.813812in}{0.500000in}}%
\pgfpathclose%
\pgfusepath{fill}%
\end{pgfscope}%
\begin{pgfscope}%
\pgfpathrectangle{\pgfqpoint{0.750000in}{0.500000in}}{\pgfqpoint{4.650000in}{3.020000in}}%
\pgfusepath{clip}%
\pgfsetbuttcap%
\pgfsetmiterjoin%
\definecolor{currentfill}{rgb}{1.000000,0.000000,0.000000}%
\pgfsetfillcolor{currentfill}%
\pgfsetlinewidth{0.000000pt}%
\definecolor{currentstroke}{rgb}{0.000000,0.000000,0.000000}%
\pgfsetstrokecolor{currentstroke}%
\pgfsetstrokeopacity{0.000000}%
\pgfsetdash{}{0pt}%
\pgfpathmoveto{\pgfqpoint{4.838090in}{0.500000in}}%
\pgfpathlineto{\pgfqpoint{4.862369in}{0.500000in}}%
\pgfpathlineto{\pgfqpoint{4.862369in}{0.500000in}}%
\pgfpathlineto{\pgfqpoint{4.838090in}{0.500000in}}%
\pgfpathlineto{\pgfqpoint{4.838090in}{0.500000in}}%
\pgfpathclose%
\pgfusepath{fill}%
\end{pgfscope}%
\begin{pgfscope}%
\pgfpathrectangle{\pgfqpoint{0.750000in}{0.500000in}}{\pgfqpoint{4.650000in}{3.020000in}}%
\pgfusepath{clip}%
\pgfsetbuttcap%
\pgfsetmiterjoin%
\definecolor{currentfill}{rgb}{1.000000,0.000000,0.000000}%
\pgfsetfillcolor{currentfill}%
\pgfsetlinewidth{0.000000pt}%
\definecolor{currentstroke}{rgb}{0.000000,0.000000,0.000000}%
\pgfsetstrokecolor{currentstroke}%
\pgfsetstrokeopacity{0.000000}%
\pgfsetdash{}{0pt}%
\pgfpathmoveto{\pgfqpoint{4.862369in}{0.500000in}}%
\pgfpathlineto{\pgfqpoint{4.886647in}{0.500000in}}%
\pgfpathlineto{\pgfqpoint{4.886647in}{0.503021in}}%
\pgfpathlineto{\pgfqpoint{4.862369in}{0.503021in}}%
\pgfpathlineto{\pgfqpoint{4.862369in}{0.500000in}}%
\pgfpathclose%
\pgfusepath{fill}%
\end{pgfscope}%
\begin{pgfscope}%
\pgfpathrectangle{\pgfqpoint{0.750000in}{0.500000in}}{\pgfqpoint{4.650000in}{3.020000in}}%
\pgfusepath{clip}%
\pgfsetbuttcap%
\pgfsetmiterjoin%
\definecolor{currentfill}{rgb}{1.000000,0.000000,0.000000}%
\pgfsetfillcolor{currentfill}%
\pgfsetlinewidth{0.000000pt}%
\definecolor{currentstroke}{rgb}{0.000000,0.000000,0.000000}%
\pgfsetstrokecolor{currentstroke}%
\pgfsetstrokeopacity{0.000000}%
\pgfsetdash{}{0pt}%
\pgfpathmoveto{\pgfqpoint{4.886647in}{0.500000in}}%
\pgfpathlineto{\pgfqpoint{4.910926in}{0.500000in}}%
\pgfpathlineto{\pgfqpoint{4.910926in}{0.500000in}}%
\pgfpathlineto{\pgfqpoint{4.886647in}{0.500000in}}%
\pgfpathlineto{\pgfqpoint{4.886647in}{0.500000in}}%
\pgfpathclose%
\pgfusepath{fill}%
\end{pgfscope}%
\begin{pgfscope}%
\pgfpathrectangle{\pgfqpoint{0.750000in}{0.500000in}}{\pgfqpoint{4.650000in}{3.020000in}}%
\pgfusepath{clip}%
\pgfsetbuttcap%
\pgfsetmiterjoin%
\definecolor{currentfill}{rgb}{1.000000,0.000000,0.000000}%
\pgfsetfillcolor{currentfill}%
\pgfsetlinewidth{0.000000pt}%
\definecolor{currentstroke}{rgb}{0.000000,0.000000,0.000000}%
\pgfsetstrokecolor{currentstroke}%
\pgfsetstrokeopacity{0.000000}%
\pgfsetdash{}{0pt}%
\pgfpathmoveto{\pgfqpoint{4.910926in}{0.500000in}}%
\pgfpathlineto{\pgfqpoint{4.935205in}{0.500000in}}%
\pgfpathlineto{\pgfqpoint{4.935205in}{0.500000in}}%
\pgfpathlineto{\pgfqpoint{4.910926in}{0.500000in}}%
\pgfpathlineto{\pgfqpoint{4.910926in}{0.500000in}}%
\pgfpathclose%
\pgfusepath{fill}%
\end{pgfscope}%
\begin{pgfscope}%
\pgfpathrectangle{\pgfqpoint{0.750000in}{0.500000in}}{\pgfqpoint{4.650000in}{3.020000in}}%
\pgfusepath{clip}%
\pgfsetbuttcap%
\pgfsetmiterjoin%
\definecolor{currentfill}{rgb}{1.000000,0.000000,0.000000}%
\pgfsetfillcolor{currentfill}%
\pgfsetlinewidth{0.000000pt}%
\definecolor{currentstroke}{rgb}{0.000000,0.000000,0.000000}%
\pgfsetstrokecolor{currentstroke}%
\pgfsetstrokeopacity{0.000000}%
\pgfsetdash{}{0pt}%
\pgfpathmoveto{\pgfqpoint{4.935205in}{0.500000in}}%
\pgfpathlineto{\pgfqpoint{4.959483in}{0.500000in}}%
\pgfpathlineto{\pgfqpoint{4.959483in}{0.506042in}}%
\pgfpathlineto{\pgfqpoint{4.935205in}{0.506042in}}%
\pgfpathlineto{\pgfqpoint{4.935205in}{0.500000in}}%
\pgfpathclose%
\pgfusepath{fill}%
\end{pgfscope}%
\begin{pgfscope}%
\pgfpathrectangle{\pgfqpoint{0.750000in}{0.500000in}}{\pgfqpoint{4.650000in}{3.020000in}}%
\pgfusepath{clip}%
\pgfsetbuttcap%
\pgfsetmiterjoin%
\definecolor{currentfill}{rgb}{1.000000,0.000000,0.000000}%
\pgfsetfillcolor{currentfill}%
\pgfsetlinewidth{0.000000pt}%
\definecolor{currentstroke}{rgb}{0.000000,0.000000,0.000000}%
\pgfsetstrokecolor{currentstroke}%
\pgfsetstrokeopacity{0.000000}%
\pgfsetdash{}{0pt}%
\pgfpathmoveto{\pgfqpoint{4.959483in}{0.500000in}}%
\pgfpathlineto{\pgfqpoint{4.983762in}{0.500000in}}%
\pgfpathlineto{\pgfqpoint{4.983762in}{0.500000in}}%
\pgfpathlineto{\pgfqpoint{4.959483in}{0.500000in}}%
\pgfpathlineto{\pgfqpoint{4.959483in}{0.500000in}}%
\pgfpathclose%
\pgfusepath{fill}%
\end{pgfscope}%
\begin{pgfscope}%
\pgfpathrectangle{\pgfqpoint{0.750000in}{0.500000in}}{\pgfqpoint{4.650000in}{3.020000in}}%
\pgfusepath{clip}%
\pgfsetbuttcap%
\pgfsetmiterjoin%
\definecolor{currentfill}{rgb}{1.000000,0.000000,0.000000}%
\pgfsetfillcolor{currentfill}%
\pgfsetlinewidth{0.000000pt}%
\definecolor{currentstroke}{rgb}{0.000000,0.000000,0.000000}%
\pgfsetstrokecolor{currentstroke}%
\pgfsetstrokeopacity{0.000000}%
\pgfsetdash{}{0pt}%
\pgfpathmoveto{\pgfqpoint{4.983762in}{0.500000in}}%
\pgfpathlineto{\pgfqpoint{5.008040in}{0.500000in}}%
\pgfpathlineto{\pgfqpoint{5.008040in}{0.500000in}}%
\pgfpathlineto{\pgfqpoint{4.983762in}{0.500000in}}%
\pgfpathlineto{\pgfqpoint{4.983762in}{0.500000in}}%
\pgfpathclose%
\pgfusepath{fill}%
\end{pgfscope}%
\begin{pgfscope}%
\pgfpathrectangle{\pgfqpoint{0.750000in}{0.500000in}}{\pgfqpoint{4.650000in}{3.020000in}}%
\pgfusepath{clip}%
\pgfsetbuttcap%
\pgfsetmiterjoin%
\definecolor{currentfill}{rgb}{1.000000,0.000000,0.000000}%
\pgfsetfillcolor{currentfill}%
\pgfsetlinewidth{0.000000pt}%
\definecolor{currentstroke}{rgb}{0.000000,0.000000,0.000000}%
\pgfsetstrokecolor{currentstroke}%
\pgfsetstrokeopacity{0.000000}%
\pgfsetdash{}{0pt}%
\pgfpathmoveto{\pgfqpoint{5.008040in}{0.500000in}}%
\pgfpathlineto{\pgfqpoint{5.032319in}{0.500000in}}%
\pgfpathlineto{\pgfqpoint{5.032319in}{0.503021in}}%
\pgfpathlineto{\pgfqpoint{5.008040in}{0.503021in}}%
\pgfpathlineto{\pgfqpoint{5.008040in}{0.500000in}}%
\pgfpathclose%
\pgfusepath{fill}%
\end{pgfscope}%
\begin{pgfscope}%
\pgfpathrectangle{\pgfqpoint{0.750000in}{0.500000in}}{\pgfqpoint{4.650000in}{3.020000in}}%
\pgfusepath{clip}%
\pgfsetbuttcap%
\pgfsetmiterjoin%
\definecolor{currentfill}{rgb}{1.000000,0.000000,0.000000}%
\pgfsetfillcolor{currentfill}%
\pgfsetlinewidth{0.000000pt}%
\definecolor{currentstroke}{rgb}{0.000000,0.000000,0.000000}%
\pgfsetstrokecolor{currentstroke}%
\pgfsetstrokeopacity{0.000000}%
\pgfsetdash{}{0pt}%
\pgfpathmoveto{\pgfqpoint{5.032319in}{0.500000in}}%
\pgfpathlineto{\pgfqpoint{5.056597in}{0.500000in}}%
\pgfpathlineto{\pgfqpoint{5.056597in}{0.506042in}}%
\pgfpathlineto{\pgfqpoint{5.032319in}{0.506042in}}%
\pgfpathlineto{\pgfqpoint{5.032319in}{0.500000in}}%
\pgfpathclose%
\pgfusepath{fill}%
\end{pgfscope}%
\begin{pgfscope}%
\pgfpathrectangle{\pgfqpoint{0.750000in}{0.500000in}}{\pgfqpoint{4.650000in}{3.020000in}}%
\pgfusepath{clip}%
\pgfsetbuttcap%
\pgfsetmiterjoin%
\definecolor{currentfill}{rgb}{1.000000,0.000000,0.000000}%
\pgfsetfillcolor{currentfill}%
\pgfsetlinewidth{0.000000pt}%
\definecolor{currentstroke}{rgb}{0.000000,0.000000,0.000000}%
\pgfsetstrokecolor{currentstroke}%
\pgfsetstrokeopacity{0.000000}%
\pgfsetdash{}{0pt}%
\pgfpathmoveto{\pgfqpoint{5.056597in}{0.500000in}}%
\pgfpathlineto{\pgfqpoint{5.080876in}{0.500000in}}%
\pgfpathlineto{\pgfqpoint{5.080876in}{0.500000in}}%
\pgfpathlineto{\pgfqpoint{5.056597in}{0.500000in}}%
\pgfpathlineto{\pgfqpoint{5.056597in}{0.500000in}}%
\pgfpathclose%
\pgfusepath{fill}%
\end{pgfscope}%
\begin{pgfscope}%
\pgfpathrectangle{\pgfqpoint{0.750000in}{0.500000in}}{\pgfqpoint{4.650000in}{3.020000in}}%
\pgfusepath{clip}%
\pgfsetbuttcap%
\pgfsetmiterjoin%
\definecolor{currentfill}{rgb}{1.000000,0.000000,0.000000}%
\pgfsetfillcolor{currentfill}%
\pgfsetlinewidth{0.000000pt}%
\definecolor{currentstroke}{rgb}{0.000000,0.000000,0.000000}%
\pgfsetstrokecolor{currentstroke}%
\pgfsetstrokeopacity{0.000000}%
\pgfsetdash{}{0pt}%
\pgfpathmoveto{\pgfqpoint{5.080876in}{0.500000in}}%
\pgfpathlineto{\pgfqpoint{5.105155in}{0.500000in}}%
\pgfpathlineto{\pgfqpoint{5.105155in}{0.500000in}}%
\pgfpathlineto{\pgfqpoint{5.080876in}{0.500000in}}%
\pgfpathlineto{\pgfqpoint{5.080876in}{0.500000in}}%
\pgfpathclose%
\pgfusepath{fill}%
\end{pgfscope}%
\begin{pgfscope}%
\pgfpathrectangle{\pgfqpoint{0.750000in}{0.500000in}}{\pgfqpoint{4.650000in}{3.020000in}}%
\pgfusepath{clip}%
\pgfsetbuttcap%
\pgfsetmiterjoin%
\definecolor{currentfill}{rgb}{1.000000,0.000000,0.000000}%
\pgfsetfillcolor{currentfill}%
\pgfsetlinewidth{0.000000pt}%
\definecolor{currentstroke}{rgb}{0.000000,0.000000,0.000000}%
\pgfsetstrokecolor{currentstroke}%
\pgfsetstrokeopacity{0.000000}%
\pgfsetdash{}{0pt}%
\pgfpathmoveto{\pgfqpoint{5.105155in}{0.500000in}}%
\pgfpathlineto{\pgfqpoint{5.129433in}{0.500000in}}%
\pgfpathlineto{\pgfqpoint{5.129433in}{0.503021in}}%
\pgfpathlineto{\pgfqpoint{5.105155in}{0.503021in}}%
\pgfpathlineto{\pgfqpoint{5.105155in}{0.500000in}}%
\pgfpathclose%
\pgfusepath{fill}%
\end{pgfscope}%
\begin{pgfscope}%
\pgfpathrectangle{\pgfqpoint{0.750000in}{0.500000in}}{\pgfqpoint{4.650000in}{3.020000in}}%
\pgfusepath{clip}%
\pgfsetbuttcap%
\pgfsetmiterjoin%
\definecolor{currentfill}{rgb}{1.000000,0.000000,0.000000}%
\pgfsetfillcolor{currentfill}%
\pgfsetlinewidth{0.000000pt}%
\definecolor{currentstroke}{rgb}{0.000000,0.000000,0.000000}%
\pgfsetstrokecolor{currentstroke}%
\pgfsetstrokeopacity{0.000000}%
\pgfsetdash{}{0pt}%
\pgfpathmoveto{\pgfqpoint{5.129433in}{0.500000in}}%
\pgfpathlineto{\pgfqpoint{5.153712in}{0.500000in}}%
\pgfpathlineto{\pgfqpoint{5.153712in}{0.503021in}}%
\pgfpathlineto{\pgfqpoint{5.129433in}{0.503021in}}%
\pgfpathlineto{\pgfqpoint{5.129433in}{0.500000in}}%
\pgfpathclose%
\pgfusepath{fill}%
\end{pgfscope}%
\begin{pgfscope}%
\pgfpathrectangle{\pgfqpoint{0.750000in}{0.500000in}}{\pgfqpoint{4.650000in}{3.020000in}}%
\pgfusepath{clip}%
\pgfsetbuttcap%
\pgfsetmiterjoin%
\definecolor{currentfill}{rgb}{1.000000,0.000000,0.000000}%
\pgfsetfillcolor{currentfill}%
\pgfsetlinewidth{0.000000pt}%
\definecolor{currentstroke}{rgb}{0.000000,0.000000,0.000000}%
\pgfsetstrokecolor{currentstroke}%
\pgfsetstrokeopacity{0.000000}%
\pgfsetdash{}{0pt}%
\pgfpathmoveto{\pgfqpoint{5.153712in}{0.500000in}}%
\pgfpathlineto{\pgfqpoint{5.177990in}{0.500000in}}%
\pgfpathlineto{\pgfqpoint{5.177990in}{0.509064in}}%
\pgfpathlineto{\pgfqpoint{5.153712in}{0.509064in}}%
\pgfpathlineto{\pgfqpoint{5.153712in}{0.500000in}}%
\pgfpathclose%
\pgfusepath{fill}%
\end{pgfscope}%
\begin{pgfscope}%
\pgfpathrectangle{\pgfqpoint{0.750000in}{0.500000in}}{\pgfqpoint{4.650000in}{3.020000in}}%
\pgfusepath{clip}%
\pgfsetbuttcap%
\pgfsetmiterjoin%
\definecolor{currentfill}{rgb}{0.000000,0.500000,0.000000}%
\pgfsetfillcolor{currentfill}%
\pgfsetlinewidth{0.000000pt}%
\definecolor{currentstroke}{rgb}{0.000000,0.000000,0.000000}%
\pgfsetstrokecolor{currentstroke}%
\pgfsetstrokeopacity{0.000000}%
\pgfsetdash{}{0pt}%
\pgfpathmoveto{\pgfqpoint{0.961364in}{0.500000in}}%
\pgfpathlineto{\pgfqpoint{0.994389in}{0.500000in}}%
\pgfpathlineto{\pgfqpoint{0.994389in}{0.635954in}}%
\pgfpathlineto{\pgfqpoint{0.961364in}{0.635954in}}%
\pgfpathlineto{\pgfqpoint{0.961364in}{0.500000in}}%
\pgfpathclose%
\pgfusepath{fill}%
\end{pgfscope}%
\begin{pgfscope}%
\pgfpathrectangle{\pgfqpoint{0.750000in}{0.500000in}}{\pgfqpoint{4.650000in}{3.020000in}}%
\pgfusepath{clip}%
\pgfsetbuttcap%
\pgfsetmiterjoin%
\definecolor{currentfill}{rgb}{0.000000,0.500000,0.000000}%
\pgfsetfillcolor{currentfill}%
\pgfsetlinewidth{0.000000pt}%
\definecolor{currentstroke}{rgb}{0.000000,0.000000,0.000000}%
\pgfsetstrokecolor{currentstroke}%
\pgfsetstrokeopacity{0.000000}%
\pgfsetdash{}{0pt}%
\pgfpathmoveto{\pgfqpoint{0.994389in}{0.500000in}}%
\pgfpathlineto{\pgfqpoint{1.027415in}{0.500000in}}%
\pgfpathlineto{\pgfqpoint{1.027415in}{0.965266in}}%
\pgfpathlineto{\pgfqpoint{0.994389in}{0.965266in}}%
\pgfpathlineto{\pgfqpoint{0.994389in}{0.500000in}}%
\pgfpathclose%
\pgfusepath{fill}%
\end{pgfscope}%
\begin{pgfscope}%
\pgfpathrectangle{\pgfqpoint{0.750000in}{0.500000in}}{\pgfqpoint{4.650000in}{3.020000in}}%
\pgfusepath{clip}%
\pgfsetbuttcap%
\pgfsetmiterjoin%
\definecolor{currentfill}{rgb}{0.000000,0.500000,0.000000}%
\pgfsetfillcolor{currentfill}%
\pgfsetlinewidth{0.000000pt}%
\definecolor{currentstroke}{rgb}{0.000000,0.000000,0.000000}%
\pgfsetstrokecolor{currentstroke}%
\pgfsetstrokeopacity{0.000000}%
\pgfsetdash{}{0pt}%
\pgfpathmoveto{\pgfqpoint{1.027415in}{0.500000in}}%
\pgfpathlineto{\pgfqpoint{1.060440in}{0.500000in}}%
\pgfpathlineto{\pgfqpoint{1.060440in}{1.415426in}}%
\pgfpathlineto{\pgfqpoint{1.027415in}{1.415426in}}%
\pgfpathlineto{\pgfqpoint{1.027415in}{0.500000in}}%
\pgfpathclose%
\pgfusepath{fill}%
\end{pgfscope}%
\begin{pgfscope}%
\pgfpathrectangle{\pgfqpoint{0.750000in}{0.500000in}}{\pgfqpoint{4.650000in}{3.020000in}}%
\pgfusepath{clip}%
\pgfsetbuttcap%
\pgfsetmiterjoin%
\definecolor{currentfill}{rgb}{0.000000,0.500000,0.000000}%
\pgfsetfillcolor{currentfill}%
\pgfsetlinewidth{0.000000pt}%
\definecolor{currentstroke}{rgb}{0.000000,0.000000,0.000000}%
\pgfsetstrokecolor{currentstroke}%
\pgfsetstrokeopacity{0.000000}%
\pgfsetdash{}{0pt}%
\pgfpathmoveto{\pgfqpoint{1.060440in}{0.500000in}}%
\pgfpathlineto{\pgfqpoint{1.093466in}{0.500000in}}%
\pgfpathlineto{\pgfqpoint{1.093466in}{1.496999in}}%
\pgfpathlineto{\pgfqpoint{1.060440in}{1.496999in}}%
\pgfpathlineto{\pgfqpoint{1.060440in}{0.500000in}}%
\pgfpathclose%
\pgfusepath{fill}%
\end{pgfscope}%
\begin{pgfscope}%
\pgfpathrectangle{\pgfqpoint{0.750000in}{0.500000in}}{\pgfqpoint{4.650000in}{3.020000in}}%
\pgfusepath{clip}%
\pgfsetbuttcap%
\pgfsetmiterjoin%
\definecolor{currentfill}{rgb}{0.000000,0.500000,0.000000}%
\pgfsetfillcolor{currentfill}%
\pgfsetlinewidth{0.000000pt}%
\definecolor{currentstroke}{rgb}{0.000000,0.000000,0.000000}%
\pgfsetstrokecolor{currentstroke}%
\pgfsetstrokeopacity{0.000000}%
\pgfsetdash{}{0pt}%
\pgfpathmoveto{\pgfqpoint{1.093466in}{0.500000in}}%
\pgfpathlineto{\pgfqpoint{1.126491in}{0.500000in}}%
\pgfpathlineto{\pgfqpoint{1.126491in}{1.654102in}}%
\pgfpathlineto{\pgfqpoint{1.093466in}{1.654102in}}%
\pgfpathlineto{\pgfqpoint{1.093466in}{0.500000in}}%
\pgfpathclose%
\pgfusepath{fill}%
\end{pgfscope}%
\begin{pgfscope}%
\pgfpathrectangle{\pgfqpoint{0.750000in}{0.500000in}}{\pgfqpoint{4.650000in}{3.020000in}}%
\pgfusepath{clip}%
\pgfsetbuttcap%
\pgfsetmiterjoin%
\definecolor{currentfill}{rgb}{0.000000,0.500000,0.000000}%
\pgfsetfillcolor{currentfill}%
\pgfsetlinewidth{0.000000pt}%
\definecolor{currentstroke}{rgb}{0.000000,0.000000,0.000000}%
\pgfsetstrokecolor{currentstroke}%
\pgfsetstrokeopacity{0.000000}%
\pgfsetdash{}{0pt}%
\pgfpathmoveto{\pgfqpoint{1.126491in}{0.500000in}}%
\pgfpathlineto{\pgfqpoint{1.159517in}{0.500000in}}%
\pgfpathlineto{\pgfqpoint{1.159517in}{1.669208in}}%
\pgfpathlineto{\pgfqpoint{1.126491in}{1.669208in}}%
\pgfpathlineto{\pgfqpoint{1.126491in}{0.500000in}}%
\pgfpathclose%
\pgfusepath{fill}%
\end{pgfscope}%
\begin{pgfscope}%
\pgfpathrectangle{\pgfqpoint{0.750000in}{0.500000in}}{\pgfqpoint{4.650000in}{3.020000in}}%
\pgfusepath{clip}%
\pgfsetbuttcap%
\pgfsetmiterjoin%
\definecolor{currentfill}{rgb}{0.000000,0.500000,0.000000}%
\pgfsetfillcolor{currentfill}%
\pgfsetlinewidth{0.000000pt}%
\definecolor{currentstroke}{rgb}{0.000000,0.000000,0.000000}%
\pgfsetstrokecolor{currentstroke}%
\pgfsetstrokeopacity{0.000000}%
\pgfsetdash{}{0pt}%
\pgfpathmoveto{\pgfqpoint{1.159517in}{0.500000in}}%
\pgfpathlineto{\pgfqpoint{1.192543in}{0.500000in}}%
\pgfpathlineto{\pgfqpoint{1.192543in}{1.678271in}}%
\pgfpathlineto{\pgfqpoint{1.159517in}{1.678271in}}%
\pgfpathlineto{\pgfqpoint{1.159517in}{0.500000in}}%
\pgfpathclose%
\pgfusepath{fill}%
\end{pgfscope}%
\begin{pgfscope}%
\pgfpathrectangle{\pgfqpoint{0.750000in}{0.500000in}}{\pgfqpoint{4.650000in}{3.020000in}}%
\pgfusepath{clip}%
\pgfsetbuttcap%
\pgfsetmiterjoin%
\definecolor{currentfill}{rgb}{0.000000,0.500000,0.000000}%
\pgfsetfillcolor{currentfill}%
\pgfsetlinewidth{0.000000pt}%
\definecolor{currentstroke}{rgb}{0.000000,0.000000,0.000000}%
\pgfsetstrokecolor{currentstroke}%
\pgfsetstrokeopacity{0.000000}%
\pgfsetdash{}{0pt}%
\pgfpathmoveto{\pgfqpoint{1.192543in}{0.500000in}}%
\pgfpathlineto{\pgfqpoint{1.225568in}{0.500000in}}%
\pgfpathlineto{\pgfqpoint{1.225568in}{1.868607in}}%
\pgfpathlineto{\pgfqpoint{1.192543in}{1.868607in}}%
\pgfpathlineto{\pgfqpoint{1.192543in}{0.500000in}}%
\pgfpathclose%
\pgfusepath{fill}%
\end{pgfscope}%
\begin{pgfscope}%
\pgfpathrectangle{\pgfqpoint{0.750000in}{0.500000in}}{\pgfqpoint{4.650000in}{3.020000in}}%
\pgfusepath{clip}%
\pgfsetbuttcap%
\pgfsetmiterjoin%
\definecolor{currentfill}{rgb}{0.000000,0.500000,0.000000}%
\pgfsetfillcolor{currentfill}%
\pgfsetlinewidth{0.000000pt}%
\definecolor{currentstroke}{rgb}{0.000000,0.000000,0.000000}%
\pgfsetstrokecolor{currentstroke}%
\pgfsetstrokeopacity{0.000000}%
\pgfsetdash{}{0pt}%
\pgfpathmoveto{\pgfqpoint{1.225568in}{0.500000in}}%
\pgfpathlineto{\pgfqpoint{1.258594in}{0.500000in}}%
\pgfpathlineto{\pgfqpoint{1.258594in}{1.602741in}}%
\pgfpathlineto{\pgfqpoint{1.225568in}{1.602741in}}%
\pgfpathlineto{\pgfqpoint{1.225568in}{0.500000in}}%
\pgfpathclose%
\pgfusepath{fill}%
\end{pgfscope}%
\begin{pgfscope}%
\pgfpathrectangle{\pgfqpoint{0.750000in}{0.500000in}}{\pgfqpoint{4.650000in}{3.020000in}}%
\pgfusepath{clip}%
\pgfsetbuttcap%
\pgfsetmiterjoin%
\definecolor{currentfill}{rgb}{0.000000,0.500000,0.000000}%
\pgfsetfillcolor{currentfill}%
\pgfsetlinewidth{0.000000pt}%
\definecolor{currentstroke}{rgb}{0.000000,0.000000,0.000000}%
\pgfsetstrokecolor{currentstroke}%
\pgfsetstrokeopacity{0.000000}%
\pgfsetdash{}{0pt}%
\pgfpathmoveto{\pgfqpoint{1.258594in}{0.500000in}}%
\pgfpathlineto{\pgfqpoint{1.291619in}{0.500000in}}%
\pgfpathlineto{\pgfqpoint{1.291619in}{1.629932in}}%
\pgfpathlineto{\pgfqpoint{1.258594in}{1.629932in}}%
\pgfpathlineto{\pgfqpoint{1.258594in}{0.500000in}}%
\pgfpathclose%
\pgfusepath{fill}%
\end{pgfscope}%
\begin{pgfscope}%
\pgfpathrectangle{\pgfqpoint{0.750000in}{0.500000in}}{\pgfqpoint{4.650000in}{3.020000in}}%
\pgfusepath{clip}%
\pgfsetbuttcap%
\pgfsetmiterjoin%
\definecolor{currentfill}{rgb}{0.000000,0.500000,0.000000}%
\pgfsetfillcolor{currentfill}%
\pgfsetlinewidth{0.000000pt}%
\definecolor{currentstroke}{rgb}{0.000000,0.000000,0.000000}%
\pgfsetstrokecolor{currentstroke}%
\pgfsetstrokeopacity{0.000000}%
\pgfsetdash{}{0pt}%
\pgfpathmoveto{\pgfqpoint{1.291619in}{0.500000in}}%
\pgfpathlineto{\pgfqpoint{1.324645in}{0.500000in}}%
\pgfpathlineto{\pgfqpoint{1.324645in}{1.539296in}}%
\pgfpathlineto{\pgfqpoint{1.291619in}{1.539296in}}%
\pgfpathlineto{\pgfqpoint{1.291619in}{0.500000in}}%
\pgfpathclose%
\pgfusepath{fill}%
\end{pgfscope}%
\begin{pgfscope}%
\pgfpathrectangle{\pgfqpoint{0.750000in}{0.500000in}}{\pgfqpoint{4.650000in}{3.020000in}}%
\pgfusepath{clip}%
\pgfsetbuttcap%
\pgfsetmiterjoin%
\definecolor{currentfill}{rgb}{0.000000,0.500000,0.000000}%
\pgfsetfillcolor{currentfill}%
\pgfsetlinewidth{0.000000pt}%
\definecolor{currentstroke}{rgb}{0.000000,0.000000,0.000000}%
\pgfsetstrokecolor{currentstroke}%
\pgfsetstrokeopacity{0.000000}%
\pgfsetdash{}{0pt}%
\pgfpathmoveto{\pgfqpoint{1.324645in}{0.500000in}}%
\pgfpathlineto{\pgfqpoint{1.357670in}{0.500000in}}%
\pgfpathlineto{\pgfqpoint{1.357670in}{1.527211in}}%
\pgfpathlineto{\pgfqpoint{1.324645in}{1.527211in}}%
\pgfpathlineto{\pgfqpoint{1.324645in}{0.500000in}}%
\pgfpathclose%
\pgfusepath{fill}%
\end{pgfscope}%
\begin{pgfscope}%
\pgfpathrectangle{\pgfqpoint{0.750000in}{0.500000in}}{\pgfqpoint{4.650000in}{3.020000in}}%
\pgfusepath{clip}%
\pgfsetbuttcap%
\pgfsetmiterjoin%
\definecolor{currentfill}{rgb}{0.000000,0.500000,0.000000}%
\pgfsetfillcolor{currentfill}%
\pgfsetlinewidth{0.000000pt}%
\definecolor{currentstroke}{rgb}{0.000000,0.000000,0.000000}%
\pgfsetstrokecolor{currentstroke}%
\pgfsetstrokeopacity{0.000000}%
\pgfsetdash{}{0pt}%
\pgfpathmoveto{\pgfqpoint{1.357670in}{0.500000in}}%
\pgfpathlineto{\pgfqpoint{1.390696in}{0.500000in}}%
\pgfpathlineto{\pgfqpoint{1.390696in}{1.336875in}}%
\pgfpathlineto{\pgfqpoint{1.357670in}{1.336875in}}%
\pgfpathlineto{\pgfqpoint{1.357670in}{0.500000in}}%
\pgfpathclose%
\pgfusepath{fill}%
\end{pgfscope}%
\begin{pgfscope}%
\pgfpathrectangle{\pgfqpoint{0.750000in}{0.500000in}}{\pgfqpoint{4.650000in}{3.020000in}}%
\pgfusepath{clip}%
\pgfsetbuttcap%
\pgfsetmiterjoin%
\definecolor{currentfill}{rgb}{0.000000,0.500000,0.000000}%
\pgfsetfillcolor{currentfill}%
\pgfsetlinewidth{0.000000pt}%
\definecolor{currentstroke}{rgb}{0.000000,0.000000,0.000000}%
\pgfsetstrokecolor{currentstroke}%
\pgfsetstrokeopacity{0.000000}%
\pgfsetdash{}{0pt}%
\pgfpathmoveto{\pgfqpoint{1.390696in}{0.500000in}}%
\pgfpathlineto{\pgfqpoint{1.423722in}{0.500000in}}%
\pgfpathlineto{\pgfqpoint{1.423722in}{1.351981in}}%
\pgfpathlineto{\pgfqpoint{1.390696in}{1.351981in}}%
\pgfpathlineto{\pgfqpoint{1.390696in}{0.500000in}}%
\pgfpathclose%
\pgfusepath{fill}%
\end{pgfscope}%
\begin{pgfscope}%
\pgfpathrectangle{\pgfqpoint{0.750000in}{0.500000in}}{\pgfqpoint{4.650000in}{3.020000in}}%
\pgfusepath{clip}%
\pgfsetbuttcap%
\pgfsetmiterjoin%
\definecolor{currentfill}{rgb}{0.000000,0.500000,0.000000}%
\pgfsetfillcolor{currentfill}%
\pgfsetlinewidth{0.000000pt}%
\definecolor{currentstroke}{rgb}{0.000000,0.000000,0.000000}%
\pgfsetstrokecolor{currentstroke}%
\pgfsetstrokeopacity{0.000000}%
\pgfsetdash{}{0pt}%
\pgfpathmoveto{\pgfqpoint{1.423722in}{0.500000in}}%
\pgfpathlineto{\pgfqpoint{1.456747in}{0.500000in}}%
\pgfpathlineto{\pgfqpoint{1.456747in}{1.255302in}}%
\pgfpathlineto{\pgfqpoint{1.423722in}{1.255302in}}%
\pgfpathlineto{\pgfqpoint{1.423722in}{0.500000in}}%
\pgfpathclose%
\pgfusepath{fill}%
\end{pgfscope}%
\begin{pgfscope}%
\pgfpathrectangle{\pgfqpoint{0.750000in}{0.500000in}}{\pgfqpoint{4.650000in}{3.020000in}}%
\pgfusepath{clip}%
\pgfsetbuttcap%
\pgfsetmiterjoin%
\definecolor{currentfill}{rgb}{0.000000,0.500000,0.000000}%
\pgfsetfillcolor{currentfill}%
\pgfsetlinewidth{0.000000pt}%
\definecolor{currentstroke}{rgb}{0.000000,0.000000,0.000000}%
\pgfsetstrokecolor{currentstroke}%
\pgfsetstrokeopacity{0.000000}%
\pgfsetdash{}{0pt}%
\pgfpathmoveto{\pgfqpoint{1.456747in}{0.500000in}}%
\pgfpathlineto{\pgfqpoint{1.489773in}{0.500000in}}%
\pgfpathlineto{\pgfqpoint{1.489773in}{1.209984in}}%
\pgfpathlineto{\pgfqpoint{1.456747in}{1.209984in}}%
\pgfpathlineto{\pgfqpoint{1.456747in}{0.500000in}}%
\pgfpathclose%
\pgfusepath{fill}%
\end{pgfscope}%
\begin{pgfscope}%
\pgfpathrectangle{\pgfqpoint{0.750000in}{0.500000in}}{\pgfqpoint{4.650000in}{3.020000in}}%
\pgfusepath{clip}%
\pgfsetbuttcap%
\pgfsetmiterjoin%
\definecolor{currentfill}{rgb}{0.000000,0.500000,0.000000}%
\pgfsetfillcolor{currentfill}%
\pgfsetlinewidth{0.000000pt}%
\definecolor{currentstroke}{rgb}{0.000000,0.000000,0.000000}%
\pgfsetstrokecolor{currentstroke}%
\pgfsetstrokeopacity{0.000000}%
\pgfsetdash{}{0pt}%
\pgfpathmoveto{\pgfqpoint{1.489773in}{0.500000in}}%
\pgfpathlineto{\pgfqpoint{1.522798in}{0.500000in}}%
\pgfpathlineto{\pgfqpoint{1.522798in}{1.222069in}}%
\pgfpathlineto{\pgfqpoint{1.489773in}{1.222069in}}%
\pgfpathlineto{\pgfqpoint{1.489773in}{0.500000in}}%
\pgfpathclose%
\pgfusepath{fill}%
\end{pgfscope}%
\begin{pgfscope}%
\pgfpathrectangle{\pgfqpoint{0.750000in}{0.500000in}}{\pgfqpoint{4.650000in}{3.020000in}}%
\pgfusepath{clip}%
\pgfsetbuttcap%
\pgfsetmiterjoin%
\definecolor{currentfill}{rgb}{0.000000,0.500000,0.000000}%
\pgfsetfillcolor{currentfill}%
\pgfsetlinewidth{0.000000pt}%
\definecolor{currentstroke}{rgb}{0.000000,0.000000,0.000000}%
\pgfsetstrokecolor{currentstroke}%
\pgfsetstrokeopacity{0.000000}%
\pgfsetdash{}{0pt}%
\pgfpathmoveto{\pgfqpoint{1.522798in}{0.500000in}}%
\pgfpathlineto{\pgfqpoint{1.555824in}{0.500000in}}%
\pgfpathlineto{\pgfqpoint{1.555824in}{1.152581in}}%
\pgfpathlineto{\pgfqpoint{1.522798in}{1.152581in}}%
\pgfpathlineto{\pgfqpoint{1.522798in}{0.500000in}}%
\pgfpathclose%
\pgfusepath{fill}%
\end{pgfscope}%
\begin{pgfscope}%
\pgfpathrectangle{\pgfqpoint{0.750000in}{0.500000in}}{\pgfqpoint{4.650000in}{3.020000in}}%
\pgfusepath{clip}%
\pgfsetbuttcap%
\pgfsetmiterjoin%
\definecolor{currentfill}{rgb}{0.000000,0.500000,0.000000}%
\pgfsetfillcolor{currentfill}%
\pgfsetlinewidth{0.000000pt}%
\definecolor{currentstroke}{rgb}{0.000000,0.000000,0.000000}%
\pgfsetstrokecolor{currentstroke}%
\pgfsetstrokeopacity{0.000000}%
\pgfsetdash{}{0pt}%
\pgfpathmoveto{\pgfqpoint{1.555824in}{0.500000in}}%
\pgfpathlineto{\pgfqpoint{1.588849in}{0.500000in}}%
\pgfpathlineto{\pgfqpoint{1.588849in}{1.149560in}}%
\pgfpathlineto{\pgfqpoint{1.555824in}{1.149560in}}%
\pgfpathlineto{\pgfqpoint{1.555824in}{0.500000in}}%
\pgfpathclose%
\pgfusepath{fill}%
\end{pgfscope}%
\begin{pgfscope}%
\pgfpathrectangle{\pgfqpoint{0.750000in}{0.500000in}}{\pgfqpoint{4.650000in}{3.020000in}}%
\pgfusepath{clip}%
\pgfsetbuttcap%
\pgfsetmiterjoin%
\definecolor{currentfill}{rgb}{0.000000,0.500000,0.000000}%
\pgfsetfillcolor{currentfill}%
\pgfsetlinewidth{0.000000pt}%
\definecolor{currentstroke}{rgb}{0.000000,0.000000,0.000000}%
\pgfsetstrokecolor{currentstroke}%
\pgfsetstrokeopacity{0.000000}%
\pgfsetdash{}{0pt}%
\pgfpathmoveto{\pgfqpoint{1.588849in}{0.500000in}}%
\pgfpathlineto{\pgfqpoint{1.621875in}{0.500000in}}%
\pgfpathlineto{\pgfqpoint{1.621875in}{0.995478in}}%
\pgfpathlineto{\pgfqpoint{1.588849in}{0.995478in}}%
\pgfpathlineto{\pgfqpoint{1.588849in}{0.500000in}}%
\pgfpathclose%
\pgfusepath{fill}%
\end{pgfscope}%
\begin{pgfscope}%
\pgfpathrectangle{\pgfqpoint{0.750000in}{0.500000in}}{\pgfqpoint{4.650000in}{3.020000in}}%
\pgfusepath{clip}%
\pgfsetbuttcap%
\pgfsetmiterjoin%
\definecolor{currentfill}{rgb}{0.000000,0.500000,0.000000}%
\pgfsetfillcolor{currentfill}%
\pgfsetlinewidth{0.000000pt}%
\definecolor{currentstroke}{rgb}{0.000000,0.000000,0.000000}%
\pgfsetstrokecolor{currentstroke}%
\pgfsetstrokeopacity{0.000000}%
\pgfsetdash{}{0pt}%
\pgfpathmoveto{\pgfqpoint{1.621875in}{0.500000in}}%
\pgfpathlineto{\pgfqpoint{1.654901in}{0.500000in}}%
\pgfpathlineto{\pgfqpoint{1.654901in}{1.061945in}}%
\pgfpathlineto{\pgfqpoint{1.621875in}{1.061945in}}%
\pgfpathlineto{\pgfqpoint{1.621875in}{0.500000in}}%
\pgfpathclose%
\pgfusepath{fill}%
\end{pgfscope}%
\begin{pgfscope}%
\pgfpathrectangle{\pgfqpoint{0.750000in}{0.500000in}}{\pgfqpoint{4.650000in}{3.020000in}}%
\pgfusepath{clip}%
\pgfsetbuttcap%
\pgfsetmiterjoin%
\definecolor{currentfill}{rgb}{0.000000,0.500000,0.000000}%
\pgfsetfillcolor{currentfill}%
\pgfsetlinewidth{0.000000pt}%
\definecolor{currentstroke}{rgb}{0.000000,0.000000,0.000000}%
\pgfsetstrokecolor{currentstroke}%
\pgfsetstrokeopacity{0.000000}%
\pgfsetdash{}{0pt}%
\pgfpathmoveto{\pgfqpoint{1.654901in}{0.500000in}}%
\pgfpathlineto{\pgfqpoint{1.687926in}{0.500000in}}%
\pgfpathlineto{\pgfqpoint{1.687926in}{1.089136in}}%
\pgfpathlineto{\pgfqpoint{1.654901in}{1.089136in}}%
\pgfpathlineto{\pgfqpoint{1.654901in}{0.500000in}}%
\pgfpathclose%
\pgfusepath{fill}%
\end{pgfscope}%
\begin{pgfscope}%
\pgfpathrectangle{\pgfqpoint{0.750000in}{0.500000in}}{\pgfqpoint{4.650000in}{3.020000in}}%
\pgfusepath{clip}%
\pgfsetbuttcap%
\pgfsetmiterjoin%
\definecolor{currentfill}{rgb}{0.000000,0.500000,0.000000}%
\pgfsetfillcolor{currentfill}%
\pgfsetlinewidth{0.000000pt}%
\definecolor{currentstroke}{rgb}{0.000000,0.000000,0.000000}%
\pgfsetstrokecolor{currentstroke}%
\pgfsetstrokeopacity{0.000000}%
\pgfsetdash{}{0pt}%
\pgfpathmoveto{\pgfqpoint{1.687926in}{0.500000in}}%
\pgfpathlineto{\pgfqpoint{1.720952in}{0.500000in}}%
\pgfpathlineto{\pgfqpoint{1.720952in}{0.953181in}}%
\pgfpathlineto{\pgfqpoint{1.687926in}{0.953181in}}%
\pgfpathlineto{\pgfqpoint{1.687926in}{0.500000in}}%
\pgfpathclose%
\pgfusepath{fill}%
\end{pgfscope}%
\begin{pgfscope}%
\pgfpathrectangle{\pgfqpoint{0.750000in}{0.500000in}}{\pgfqpoint{4.650000in}{3.020000in}}%
\pgfusepath{clip}%
\pgfsetbuttcap%
\pgfsetmiterjoin%
\definecolor{currentfill}{rgb}{0.000000,0.500000,0.000000}%
\pgfsetfillcolor{currentfill}%
\pgfsetlinewidth{0.000000pt}%
\definecolor{currentstroke}{rgb}{0.000000,0.000000,0.000000}%
\pgfsetstrokecolor{currentstroke}%
\pgfsetstrokeopacity{0.000000}%
\pgfsetdash{}{0pt}%
\pgfpathmoveto{\pgfqpoint{1.720952in}{0.500000in}}%
\pgfpathlineto{\pgfqpoint{1.753977in}{0.500000in}}%
\pgfpathlineto{\pgfqpoint{1.753977in}{1.004542in}}%
\pgfpathlineto{\pgfqpoint{1.720952in}{1.004542in}}%
\pgfpathlineto{\pgfqpoint{1.720952in}{0.500000in}}%
\pgfpathclose%
\pgfusepath{fill}%
\end{pgfscope}%
\begin{pgfscope}%
\pgfpathrectangle{\pgfqpoint{0.750000in}{0.500000in}}{\pgfqpoint{4.650000in}{3.020000in}}%
\pgfusepath{clip}%
\pgfsetbuttcap%
\pgfsetmiterjoin%
\definecolor{currentfill}{rgb}{0.000000,0.500000,0.000000}%
\pgfsetfillcolor{currentfill}%
\pgfsetlinewidth{0.000000pt}%
\definecolor{currentstroke}{rgb}{0.000000,0.000000,0.000000}%
\pgfsetstrokecolor{currentstroke}%
\pgfsetstrokeopacity{0.000000}%
\pgfsetdash{}{0pt}%
\pgfpathmoveto{\pgfqpoint{1.753977in}{0.500000in}}%
\pgfpathlineto{\pgfqpoint{1.787003in}{0.500000in}}%
\pgfpathlineto{\pgfqpoint{1.787003in}{0.995478in}}%
\pgfpathlineto{\pgfqpoint{1.753977in}{0.995478in}}%
\pgfpathlineto{\pgfqpoint{1.753977in}{0.500000in}}%
\pgfpathclose%
\pgfusepath{fill}%
\end{pgfscope}%
\begin{pgfscope}%
\pgfpathrectangle{\pgfqpoint{0.750000in}{0.500000in}}{\pgfqpoint{4.650000in}{3.020000in}}%
\pgfusepath{clip}%
\pgfsetbuttcap%
\pgfsetmiterjoin%
\definecolor{currentfill}{rgb}{0.000000,0.500000,0.000000}%
\pgfsetfillcolor{currentfill}%
\pgfsetlinewidth{0.000000pt}%
\definecolor{currentstroke}{rgb}{0.000000,0.000000,0.000000}%
\pgfsetstrokecolor{currentstroke}%
\pgfsetstrokeopacity{0.000000}%
\pgfsetdash{}{0pt}%
\pgfpathmoveto{\pgfqpoint{1.787003in}{0.500000in}}%
\pgfpathlineto{\pgfqpoint{1.820028in}{0.500000in}}%
\pgfpathlineto{\pgfqpoint{1.820028in}{0.901821in}}%
\pgfpathlineto{\pgfqpoint{1.787003in}{0.901821in}}%
\pgfpathlineto{\pgfqpoint{1.787003in}{0.500000in}}%
\pgfpathclose%
\pgfusepath{fill}%
\end{pgfscope}%
\begin{pgfscope}%
\pgfpathrectangle{\pgfqpoint{0.750000in}{0.500000in}}{\pgfqpoint{4.650000in}{3.020000in}}%
\pgfusepath{clip}%
\pgfsetbuttcap%
\pgfsetmiterjoin%
\definecolor{currentfill}{rgb}{0.000000,0.500000,0.000000}%
\pgfsetfillcolor{currentfill}%
\pgfsetlinewidth{0.000000pt}%
\definecolor{currentstroke}{rgb}{0.000000,0.000000,0.000000}%
\pgfsetstrokecolor{currentstroke}%
\pgfsetstrokeopacity{0.000000}%
\pgfsetdash{}{0pt}%
\pgfpathmoveto{\pgfqpoint{1.820028in}{0.500000in}}%
\pgfpathlineto{\pgfqpoint{1.853054in}{0.500000in}}%
\pgfpathlineto{\pgfqpoint{1.853054in}{0.880672in}}%
\pgfpathlineto{\pgfqpoint{1.820028in}{0.880672in}}%
\pgfpathlineto{\pgfqpoint{1.820028in}{0.500000in}}%
\pgfpathclose%
\pgfusepath{fill}%
\end{pgfscope}%
\begin{pgfscope}%
\pgfpathrectangle{\pgfqpoint{0.750000in}{0.500000in}}{\pgfqpoint{4.650000in}{3.020000in}}%
\pgfusepath{clip}%
\pgfsetbuttcap%
\pgfsetmiterjoin%
\definecolor{currentfill}{rgb}{0.000000,0.500000,0.000000}%
\pgfsetfillcolor{currentfill}%
\pgfsetlinewidth{0.000000pt}%
\definecolor{currentstroke}{rgb}{0.000000,0.000000,0.000000}%
\pgfsetstrokecolor{currentstroke}%
\pgfsetstrokeopacity{0.000000}%
\pgfsetdash{}{0pt}%
\pgfpathmoveto{\pgfqpoint{1.853054in}{0.500000in}}%
\pgfpathlineto{\pgfqpoint{1.886080in}{0.500000in}}%
\pgfpathlineto{\pgfqpoint{1.886080in}{0.907863in}}%
\pgfpathlineto{\pgfqpoint{1.853054in}{0.907863in}}%
\pgfpathlineto{\pgfqpoint{1.853054in}{0.500000in}}%
\pgfpathclose%
\pgfusepath{fill}%
\end{pgfscope}%
\begin{pgfscope}%
\pgfpathrectangle{\pgfqpoint{0.750000in}{0.500000in}}{\pgfqpoint{4.650000in}{3.020000in}}%
\pgfusepath{clip}%
\pgfsetbuttcap%
\pgfsetmiterjoin%
\definecolor{currentfill}{rgb}{0.000000,0.500000,0.000000}%
\pgfsetfillcolor{currentfill}%
\pgfsetlinewidth{0.000000pt}%
\definecolor{currentstroke}{rgb}{0.000000,0.000000,0.000000}%
\pgfsetstrokecolor{currentstroke}%
\pgfsetstrokeopacity{0.000000}%
\pgfsetdash{}{0pt}%
\pgfpathmoveto{\pgfqpoint{1.886080in}{0.500000in}}%
\pgfpathlineto{\pgfqpoint{1.919105in}{0.500000in}}%
\pgfpathlineto{\pgfqpoint{1.919105in}{0.841397in}}%
\pgfpathlineto{\pgfqpoint{1.886080in}{0.841397in}}%
\pgfpathlineto{\pgfqpoint{1.886080in}{0.500000in}}%
\pgfpathclose%
\pgfusepath{fill}%
\end{pgfscope}%
\begin{pgfscope}%
\pgfpathrectangle{\pgfqpoint{0.750000in}{0.500000in}}{\pgfqpoint{4.650000in}{3.020000in}}%
\pgfusepath{clip}%
\pgfsetbuttcap%
\pgfsetmiterjoin%
\definecolor{currentfill}{rgb}{0.000000,0.500000,0.000000}%
\pgfsetfillcolor{currentfill}%
\pgfsetlinewidth{0.000000pt}%
\definecolor{currentstroke}{rgb}{0.000000,0.000000,0.000000}%
\pgfsetstrokecolor{currentstroke}%
\pgfsetstrokeopacity{0.000000}%
\pgfsetdash{}{0pt}%
\pgfpathmoveto{\pgfqpoint{1.919105in}{0.500000in}}%
\pgfpathlineto{\pgfqpoint{1.952131in}{0.500000in}}%
\pgfpathlineto{\pgfqpoint{1.952131in}{0.889736in}}%
\pgfpathlineto{\pgfqpoint{1.919105in}{0.889736in}}%
\pgfpathlineto{\pgfqpoint{1.919105in}{0.500000in}}%
\pgfpathclose%
\pgfusepath{fill}%
\end{pgfscope}%
\begin{pgfscope}%
\pgfpathrectangle{\pgfqpoint{0.750000in}{0.500000in}}{\pgfqpoint{4.650000in}{3.020000in}}%
\pgfusepath{clip}%
\pgfsetbuttcap%
\pgfsetmiterjoin%
\definecolor{currentfill}{rgb}{0.000000,0.500000,0.000000}%
\pgfsetfillcolor{currentfill}%
\pgfsetlinewidth{0.000000pt}%
\definecolor{currentstroke}{rgb}{0.000000,0.000000,0.000000}%
\pgfsetstrokecolor{currentstroke}%
\pgfsetstrokeopacity{0.000000}%
\pgfsetdash{}{0pt}%
\pgfpathmoveto{\pgfqpoint{1.952131in}{0.500000in}}%
\pgfpathlineto{\pgfqpoint{1.985156in}{0.500000in}}%
\pgfpathlineto{\pgfqpoint{1.985156in}{0.808163in}}%
\pgfpathlineto{\pgfqpoint{1.952131in}{0.808163in}}%
\pgfpathlineto{\pgfqpoint{1.952131in}{0.500000in}}%
\pgfpathclose%
\pgfusepath{fill}%
\end{pgfscope}%
\begin{pgfscope}%
\pgfpathrectangle{\pgfqpoint{0.750000in}{0.500000in}}{\pgfqpoint{4.650000in}{3.020000in}}%
\pgfusepath{clip}%
\pgfsetbuttcap%
\pgfsetmiterjoin%
\definecolor{currentfill}{rgb}{0.000000,0.500000,0.000000}%
\pgfsetfillcolor{currentfill}%
\pgfsetlinewidth{0.000000pt}%
\definecolor{currentstroke}{rgb}{0.000000,0.000000,0.000000}%
\pgfsetstrokecolor{currentstroke}%
\pgfsetstrokeopacity{0.000000}%
\pgfsetdash{}{0pt}%
\pgfpathmoveto{\pgfqpoint{1.985156in}{0.500000in}}%
\pgfpathlineto{\pgfqpoint{2.018182in}{0.500000in}}%
\pgfpathlineto{\pgfqpoint{2.018182in}{0.823269in}}%
\pgfpathlineto{\pgfqpoint{1.985156in}{0.823269in}}%
\pgfpathlineto{\pgfqpoint{1.985156in}{0.500000in}}%
\pgfpathclose%
\pgfusepath{fill}%
\end{pgfscope}%
\begin{pgfscope}%
\pgfpathrectangle{\pgfqpoint{0.750000in}{0.500000in}}{\pgfqpoint{4.650000in}{3.020000in}}%
\pgfusepath{clip}%
\pgfsetbuttcap%
\pgfsetmiterjoin%
\definecolor{currentfill}{rgb}{0.000000,0.500000,0.000000}%
\pgfsetfillcolor{currentfill}%
\pgfsetlinewidth{0.000000pt}%
\definecolor{currentstroke}{rgb}{0.000000,0.000000,0.000000}%
\pgfsetstrokecolor{currentstroke}%
\pgfsetstrokeopacity{0.000000}%
\pgfsetdash{}{0pt}%
\pgfpathmoveto{\pgfqpoint{2.018182in}{0.500000in}}%
\pgfpathlineto{\pgfqpoint{2.051207in}{0.500000in}}%
\pgfpathlineto{\pgfqpoint{2.051207in}{0.811184in}}%
\pgfpathlineto{\pgfqpoint{2.018182in}{0.811184in}}%
\pgfpathlineto{\pgfqpoint{2.018182in}{0.500000in}}%
\pgfpathclose%
\pgfusepath{fill}%
\end{pgfscope}%
\begin{pgfscope}%
\pgfpathrectangle{\pgfqpoint{0.750000in}{0.500000in}}{\pgfqpoint{4.650000in}{3.020000in}}%
\pgfusepath{clip}%
\pgfsetbuttcap%
\pgfsetmiterjoin%
\definecolor{currentfill}{rgb}{0.000000,0.500000,0.000000}%
\pgfsetfillcolor{currentfill}%
\pgfsetlinewidth{0.000000pt}%
\definecolor{currentstroke}{rgb}{0.000000,0.000000,0.000000}%
\pgfsetstrokecolor{currentstroke}%
\pgfsetstrokeopacity{0.000000}%
\pgfsetdash{}{0pt}%
\pgfpathmoveto{\pgfqpoint{2.051207in}{0.500000in}}%
\pgfpathlineto{\pgfqpoint{2.084233in}{0.500000in}}%
\pgfpathlineto{\pgfqpoint{2.084233in}{0.787015in}}%
\pgfpathlineto{\pgfqpoint{2.051207in}{0.787015in}}%
\pgfpathlineto{\pgfqpoint{2.051207in}{0.500000in}}%
\pgfpathclose%
\pgfusepath{fill}%
\end{pgfscope}%
\begin{pgfscope}%
\pgfpathrectangle{\pgfqpoint{0.750000in}{0.500000in}}{\pgfqpoint{4.650000in}{3.020000in}}%
\pgfusepath{clip}%
\pgfsetbuttcap%
\pgfsetmiterjoin%
\definecolor{currentfill}{rgb}{0.000000,0.500000,0.000000}%
\pgfsetfillcolor{currentfill}%
\pgfsetlinewidth{0.000000pt}%
\definecolor{currentstroke}{rgb}{0.000000,0.000000,0.000000}%
\pgfsetstrokecolor{currentstroke}%
\pgfsetstrokeopacity{0.000000}%
\pgfsetdash{}{0pt}%
\pgfpathmoveto{\pgfqpoint{2.084233in}{0.500000in}}%
\pgfpathlineto{\pgfqpoint{2.117259in}{0.500000in}}%
\pgfpathlineto{\pgfqpoint{2.117259in}{0.723569in}}%
\pgfpathlineto{\pgfqpoint{2.084233in}{0.723569in}}%
\pgfpathlineto{\pgfqpoint{2.084233in}{0.500000in}}%
\pgfpathclose%
\pgfusepath{fill}%
\end{pgfscope}%
\begin{pgfscope}%
\pgfpathrectangle{\pgfqpoint{0.750000in}{0.500000in}}{\pgfqpoint{4.650000in}{3.020000in}}%
\pgfusepath{clip}%
\pgfsetbuttcap%
\pgfsetmiterjoin%
\definecolor{currentfill}{rgb}{0.000000,0.500000,0.000000}%
\pgfsetfillcolor{currentfill}%
\pgfsetlinewidth{0.000000pt}%
\definecolor{currentstroke}{rgb}{0.000000,0.000000,0.000000}%
\pgfsetstrokecolor{currentstroke}%
\pgfsetstrokeopacity{0.000000}%
\pgfsetdash{}{0pt}%
\pgfpathmoveto{\pgfqpoint{2.117259in}{0.500000in}}%
\pgfpathlineto{\pgfqpoint{2.150284in}{0.500000in}}%
\pgfpathlineto{\pgfqpoint{2.150284in}{0.796078in}}%
\pgfpathlineto{\pgfqpoint{2.117259in}{0.796078in}}%
\pgfpathlineto{\pgfqpoint{2.117259in}{0.500000in}}%
\pgfpathclose%
\pgfusepath{fill}%
\end{pgfscope}%
\begin{pgfscope}%
\pgfpathrectangle{\pgfqpoint{0.750000in}{0.500000in}}{\pgfqpoint{4.650000in}{3.020000in}}%
\pgfusepath{clip}%
\pgfsetbuttcap%
\pgfsetmiterjoin%
\definecolor{currentfill}{rgb}{0.000000,0.500000,0.000000}%
\pgfsetfillcolor{currentfill}%
\pgfsetlinewidth{0.000000pt}%
\definecolor{currentstroke}{rgb}{0.000000,0.000000,0.000000}%
\pgfsetstrokecolor{currentstroke}%
\pgfsetstrokeopacity{0.000000}%
\pgfsetdash{}{0pt}%
\pgfpathmoveto{\pgfqpoint{2.150284in}{0.500000in}}%
\pgfpathlineto{\pgfqpoint{2.183310in}{0.500000in}}%
\pgfpathlineto{\pgfqpoint{2.183310in}{0.735654in}}%
\pgfpathlineto{\pgfqpoint{2.150284in}{0.735654in}}%
\pgfpathlineto{\pgfqpoint{2.150284in}{0.500000in}}%
\pgfpathclose%
\pgfusepath{fill}%
\end{pgfscope}%
\begin{pgfscope}%
\pgfpathrectangle{\pgfqpoint{0.750000in}{0.500000in}}{\pgfqpoint{4.650000in}{3.020000in}}%
\pgfusepath{clip}%
\pgfsetbuttcap%
\pgfsetmiterjoin%
\definecolor{currentfill}{rgb}{0.000000,0.500000,0.000000}%
\pgfsetfillcolor{currentfill}%
\pgfsetlinewidth{0.000000pt}%
\definecolor{currentstroke}{rgb}{0.000000,0.000000,0.000000}%
\pgfsetstrokecolor{currentstroke}%
\pgfsetstrokeopacity{0.000000}%
\pgfsetdash{}{0pt}%
\pgfpathmoveto{\pgfqpoint{2.183310in}{0.500000in}}%
\pgfpathlineto{\pgfqpoint{2.216335in}{0.500000in}}%
\pgfpathlineto{\pgfqpoint{2.216335in}{0.765866in}}%
\pgfpathlineto{\pgfqpoint{2.183310in}{0.765866in}}%
\pgfpathlineto{\pgfqpoint{2.183310in}{0.500000in}}%
\pgfpathclose%
\pgfusepath{fill}%
\end{pgfscope}%
\begin{pgfscope}%
\pgfpathrectangle{\pgfqpoint{0.750000in}{0.500000in}}{\pgfqpoint{4.650000in}{3.020000in}}%
\pgfusepath{clip}%
\pgfsetbuttcap%
\pgfsetmiterjoin%
\definecolor{currentfill}{rgb}{0.000000,0.500000,0.000000}%
\pgfsetfillcolor{currentfill}%
\pgfsetlinewidth{0.000000pt}%
\definecolor{currentstroke}{rgb}{0.000000,0.000000,0.000000}%
\pgfsetstrokecolor{currentstroke}%
\pgfsetstrokeopacity{0.000000}%
\pgfsetdash{}{0pt}%
\pgfpathmoveto{\pgfqpoint{2.216335in}{0.500000in}}%
\pgfpathlineto{\pgfqpoint{2.249361in}{0.500000in}}%
\pgfpathlineto{\pgfqpoint{2.249361in}{0.741697in}}%
\pgfpathlineto{\pgfqpoint{2.216335in}{0.741697in}}%
\pgfpathlineto{\pgfqpoint{2.216335in}{0.500000in}}%
\pgfpathclose%
\pgfusepath{fill}%
\end{pgfscope}%
\begin{pgfscope}%
\pgfpathrectangle{\pgfqpoint{0.750000in}{0.500000in}}{\pgfqpoint{4.650000in}{3.020000in}}%
\pgfusepath{clip}%
\pgfsetbuttcap%
\pgfsetmiterjoin%
\definecolor{currentfill}{rgb}{0.000000,0.500000,0.000000}%
\pgfsetfillcolor{currentfill}%
\pgfsetlinewidth{0.000000pt}%
\definecolor{currentstroke}{rgb}{0.000000,0.000000,0.000000}%
\pgfsetstrokecolor{currentstroke}%
\pgfsetstrokeopacity{0.000000}%
\pgfsetdash{}{0pt}%
\pgfpathmoveto{\pgfqpoint{2.249361in}{0.500000in}}%
\pgfpathlineto{\pgfqpoint{2.282386in}{0.500000in}}%
\pgfpathlineto{\pgfqpoint{2.282386in}{0.711485in}}%
\pgfpathlineto{\pgfqpoint{2.249361in}{0.711485in}}%
\pgfpathlineto{\pgfqpoint{2.249361in}{0.500000in}}%
\pgfpathclose%
\pgfusepath{fill}%
\end{pgfscope}%
\begin{pgfscope}%
\pgfpathrectangle{\pgfqpoint{0.750000in}{0.500000in}}{\pgfqpoint{4.650000in}{3.020000in}}%
\pgfusepath{clip}%
\pgfsetbuttcap%
\pgfsetmiterjoin%
\definecolor{currentfill}{rgb}{0.000000,0.500000,0.000000}%
\pgfsetfillcolor{currentfill}%
\pgfsetlinewidth{0.000000pt}%
\definecolor{currentstroke}{rgb}{0.000000,0.000000,0.000000}%
\pgfsetstrokecolor{currentstroke}%
\pgfsetstrokeopacity{0.000000}%
\pgfsetdash{}{0pt}%
\pgfpathmoveto{\pgfqpoint{2.282386in}{0.500000in}}%
\pgfpathlineto{\pgfqpoint{2.315412in}{0.500000in}}%
\pgfpathlineto{\pgfqpoint{2.315412in}{0.684294in}}%
\pgfpathlineto{\pgfqpoint{2.282386in}{0.684294in}}%
\pgfpathlineto{\pgfqpoint{2.282386in}{0.500000in}}%
\pgfpathclose%
\pgfusepath{fill}%
\end{pgfscope}%
\begin{pgfscope}%
\pgfpathrectangle{\pgfqpoint{0.750000in}{0.500000in}}{\pgfqpoint{4.650000in}{3.020000in}}%
\pgfusepath{clip}%
\pgfsetbuttcap%
\pgfsetmiterjoin%
\definecolor{currentfill}{rgb}{0.000000,0.500000,0.000000}%
\pgfsetfillcolor{currentfill}%
\pgfsetlinewidth{0.000000pt}%
\definecolor{currentstroke}{rgb}{0.000000,0.000000,0.000000}%
\pgfsetstrokecolor{currentstroke}%
\pgfsetstrokeopacity{0.000000}%
\pgfsetdash{}{0pt}%
\pgfpathmoveto{\pgfqpoint{2.315412in}{0.500000in}}%
\pgfpathlineto{\pgfqpoint{2.348437in}{0.500000in}}%
\pgfpathlineto{\pgfqpoint{2.348437in}{0.690336in}}%
\pgfpathlineto{\pgfqpoint{2.315412in}{0.690336in}}%
\pgfpathlineto{\pgfqpoint{2.315412in}{0.500000in}}%
\pgfpathclose%
\pgfusepath{fill}%
\end{pgfscope}%
\begin{pgfscope}%
\pgfpathrectangle{\pgfqpoint{0.750000in}{0.500000in}}{\pgfqpoint{4.650000in}{3.020000in}}%
\pgfusepath{clip}%
\pgfsetbuttcap%
\pgfsetmiterjoin%
\definecolor{currentfill}{rgb}{0.000000,0.500000,0.000000}%
\pgfsetfillcolor{currentfill}%
\pgfsetlinewidth{0.000000pt}%
\definecolor{currentstroke}{rgb}{0.000000,0.000000,0.000000}%
\pgfsetstrokecolor{currentstroke}%
\pgfsetstrokeopacity{0.000000}%
\pgfsetdash{}{0pt}%
\pgfpathmoveto{\pgfqpoint{2.348437in}{0.500000in}}%
\pgfpathlineto{\pgfqpoint{2.381463in}{0.500000in}}%
\pgfpathlineto{\pgfqpoint{2.381463in}{0.678251in}}%
\pgfpathlineto{\pgfqpoint{2.348437in}{0.678251in}}%
\pgfpathlineto{\pgfqpoint{2.348437in}{0.500000in}}%
\pgfpathclose%
\pgfusepath{fill}%
\end{pgfscope}%
\begin{pgfscope}%
\pgfpathrectangle{\pgfqpoint{0.750000in}{0.500000in}}{\pgfqpoint{4.650000in}{3.020000in}}%
\pgfusepath{clip}%
\pgfsetbuttcap%
\pgfsetmiterjoin%
\definecolor{currentfill}{rgb}{0.000000,0.500000,0.000000}%
\pgfsetfillcolor{currentfill}%
\pgfsetlinewidth{0.000000pt}%
\definecolor{currentstroke}{rgb}{0.000000,0.000000,0.000000}%
\pgfsetstrokecolor{currentstroke}%
\pgfsetstrokeopacity{0.000000}%
\pgfsetdash{}{0pt}%
\pgfpathmoveto{\pgfqpoint{2.381463in}{0.500000in}}%
\pgfpathlineto{\pgfqpoint{2.414489in}{0.500000in}}%
\pgfpathlineto{\pgfqpoint{2.414489in}{0.696379in}}%
\pgfpathlineto{\pgfqpoint{2.381463in}{0.696379in}}%
\pgfpathlineto{\pgfqpoint{2.381463in}{0.500000in}}%
\pgfpathclose%
\pgfusepath{fill}%
\end{pgfscope}%
\begin{pgfscope}%
\pgfpathrectangle{\pgfqpoint{0.750000in}{0.500000in}}{\pgfqpoint{4.650000in}{3.020000in}}%
\pgfusepath{clip}%
\pgfsetbuttcap%
\pgfsetmiterjoin%
\definecolor{currentfill}{rgb}{0.000000,0.500000,0.000000}%
\pgfsetfillcolor{currentfill}%
\pgfsetlinewidth{0.000000pt}%
\definecolor{currentstroke}{rgb}{0.000000,0.000000,0.000000}%
\pgfsetstrokecolor{currentstroke}%
\pgfsetstrokeopacity{0.000000}%
\pgfsetdash{}{0pt}%
\pgfpathmoveto{\pgfqpoint{2.414489in}{0.500000in}}%
\pgfpathlineto{\pgfqpoint{2.447514in}{0.500000in}}%
\pgfpathlineto{\pgfqpoint{2.447514in}{0.708463in}}%
\pgfpathlineto{\pgfqpoint{2.414489in}{0.708463in}}%
\pgfpathlineto{\pgfqpoint{2.414489in}{0.500000in}}%
\pgfpathclose%
\pgfusepath{fill}%
\end{pgfscope}%
\begin{pgfscope}%
\pgfpathrectangle{\pgfqpoint{0.750000in}{0.500000in}}{\pgfqpoint{4.650000in}{3.020000in}}%
\pgfusepath{clip}%
\pgfsetbuttcap%
\pgfsetmiterjoin%
\definecolor{currentfill}{rgb}{0.000000,0.500000,0.000000}%
\pgfsetfillcolor{currentfill}%
\pgfsetlinewidth{0.000000pt}%
\definecolor{currentstroke}{rgb}{0.000000,0.000000,0.000000}%
\pgfsetstrokecolor{currentstroke}%
\pgfsetstrokeopacity{0.000000}%
\pgfsetdash{}{0pt}%
\pgfpathmoveto{\pgfqpoint{2.447514in}{0.500000in}}%
\pgfpathlineto{\pgfqpoint{2.480540in}{0.500000in}}%
\pgfpathlineto{\pgfqpoint{2.480540in}{0.684294in}}%
\pgfpathlineto{\pgfqpoint{2.447514in}{0.684294in}}%
\pgfpathlineto{\pgfqpoint{2.447514in}{0.500000in}}%
\pgfpathclose%
\pgfusepath{fill}%
\end{pgfscope}%
\begin{pgfscope}%
\pgfpathrectangle{\pgfqpoint{0.750000in}{0.500000in}}{\pgfqpoint{4.650000in}{3.020000in}}%
\pgfusepath{clip}%
\pgfsetbuttcap%
\pgfsetmiterjoin%
\definecolor{currentfill}{rgb}{0.000000,0.500000,0.000000}%
\pgfsetfillcolor{currentfill}%
\pgfsetlinewidth{0.000000pt}%
\definecolor{currentstroke}{rgb}{0.000000,0.000000,0.000000}%
\pgfsetstrokecolor{currentstroke}%
\pgfsetstrokeopacity{0.000000}%
\pgfsetdash{}{0pt}%
\pgfpathmoveto{\pgfqpoint{2.480540in}{0.500000in}}%
\pgfpathlineto{\pgfqpoint{2.513565in}{0.500000in}}%
\pgfpathlineto{\pgfqpoint{2.513565in}{0.651060in}}%
\pgfpathlineto{\pgfqpoint{2.480540in}{0.651060in}}%
\pgfpathlineto{\pgfqpoint{2.480540in}{0.500000in}}%
\pgfpathclose%
\pgfusepath{fill}%
\end{pgfscope}%
\begin{pgfscope}%
\pgfpathrectangle{\pgfqpoint{0.750000in}{0.500000in}}{\pgfqpoint{4.650000in}{3.020000in}}%
\pgfusepath{clip}%
\pgfsetbuttcap%
\pgfsetmiterjoin%
\definecolor{currentfill}{rgb}{0.000000,0.500000,0.000000}%
\pgfsetfillcolor{currentfill}%
\pgfsetlinewidth{0.000000pt}%
\definecolor{currentstroke}{rgb}{0.000000,0.000000,0.000000}%
\pgfsetstrokecolor{currentstroke}%
\pgfsetstrokeopacity{0.000000}%
\pgfsetdash{}{0pt}%
\pgfpathmoveto{\pgfqpoint{2.513565in}{0.500000in}}%
\pgfpathlineto{\pgfqpoint{2.546591in}{0.500000in}}%
\pgfpathlineto{\pgfqpoint{2.546591in}{0.690336in}}%
\pgfpathlineto{\pgfqpoint{2.513565in}{0.690336in}}%
\pgfpathlineto{\pgfqpoint{2.513565in}{0.500000in}}%
\pgfpathclose%
\pgfusepath{fill}%
\end{pgfscope}%
\begin{pgfscope}%
\pgfpathrectangle{\pgfqpoint{0.750000in}{0.500000in}}{\pgfqpoint{4.650000in}{3.020000in}}%
\pgfusepath{clip}%
\pgfsetbuttcap%
\pgfsetmiterjoin%
\definecolor{currentfill}{rgb}{0.000000,0.500000,0.000000}%
\pgfsetfillcolor{currentfill}%
\pgfsetlinewidth{0.000000pt}%
\definecolor{currentstroke}{rgb}{0.000000,0.000000,0.000000}%
\pgfsetstrokecolor{currentstroke}%
\pgfsetstrokeopacity{0.000000}%
\pgfsetdash{}{0pt}%
\pgfpathmoveto{\pgfqpoint{2.546591in}{0.500000in}}%
\pgfpathlineto{\pgfqpoint{2.579616in}{0.500000in}}%
\pgfpathlineto{\pgfqpoint{2.579616in}{0.666166in}}%
\pgfpathlineto{\pgfqpoint{2.546591in}{0.666166in}}%
\pgfpathlineto{\pgfqpoint{2.546591in}{0.500000in}}%
\pgfpathclose%
\pgfusepath{fill}%
\end{pgfscope}%
\begin{pgfscope}%
\pgfpathrectangle{\pgfqpoint{0.750000in}{0.500000in}}{\pgfqpoint{4.650000in}{3.020000in}}%
\pgfusepath{clip}%
\pgfsetbuttcap%
\pgfsetmiterjoin%
\definecolor{currentfill}{rgb}{0.000000,0.500000,0.000000}%
\pgfsetfillcolor{currentfill}%
\pgfsetlinewidth{0.000000pt}%
\definecolor{currentstroke}{rgb}{0.000000,0.000000,0.000000}%
\pgfsetstrokecolor{currentstroke}%
\pgfsetstrokeopacity{0.000000}%
\pgfsetdash{}{0pt}%
\pgfpathmoveto{\pgfqpoint{2.579616in}{0.500000in}}%
\pgfpathlineto{\pgfqpoint{2.612642in}{0.500000in}}%
\pgfpathlineto{\pgfqpoint{2.612642in}{0.669188in}}%
\pgfpathlineto{\pgfqpoint{2.579616in}{0.669188in}}%
\pgfpathlineto{\pgfqpoint{2.579616in}{0.500000in}}%
\pgfpathclose%
\pgfusepath{fill}%
\end{pgfscope}%
\begin{pgfscope}%
\pgfpathrectangle{\pgfqpoint{0.750000in}{0.500000in}}{\pgfqpoint{4.650000in}{3.020000in}}%
\pgfusepath{clip}%
\pgfsetbuttcap%
\pgfsetmiterjoin%
\definecolor{currentfill}{rgb}{0.000000,0.500000,0.000000}%
\pgfsetfillcolor{currentfill}%
\pgfsetlinewidth{0.000000pt}%
\definecolor{currentstroke}{rgb}{0.000000,0.000000,0.000000}%
\pgfsetstrokecolor{currentstroke}%
\pgfsetstrokeopacity{0.000000}%
\pgfsetdash{}{0pt}%
\pgfpathmoveto{\pgfqpoint{2.612642in}{0.500000in}}%
\pgfpathlineto{\pgfqpoint{2.645668in}{0.500000in}}%
\pgfpathlineto{\pgfqpoint{2.645668in}{0.654082in}}%
\pgfpathlineto{\pgfqpoint{2.612642in}{0.654082in}}%
\pgfpathlineto{\pgfqpoint{2.612642in}{0.500000in}}%
\pgfpathclose%
\pgfusepath{fill}%
\end{pgfscope}%
\begin{pgfscope}%
\pgfpathrectangle{\pgfqpoint{0.750000in}{0.500000in}}{\pgfqpoint{4.650000in}{3.020000in}}%
\pgfusepath{clip}%
\pgfsetbuttcap%
\pgfsetmiterjoin%
\definecolor{currentfill}{rgb}{0.000000,0.500000,0.000000}%
\pgfsetfillcolor{currentfill}%
\pgfsetlinewidth{0.000000pt}%
\definecolor{currentstroke}{rgb}{0.000000,0.000000,0.000000}%
\pgfsetstrokecolor{currentstroke}%
\pgfsetstrokeopacity{0.000000}%
\pgfsetdash{}{0pt}%
\pgfpathmoveto{\pgfqpoint{2.645668in}{0.500000in}}%
\pgfpathlineto{\pgfqpoint{2.678693in}{0.500000in}}%
\pgfpathlineto{\pgfqpoint{2.678693in}{0.723569in}}%
\pgfpathlineto{\pgfqpoint{2.645668in}{0.723569in}}%
\pgfpathlineto{\pgfqpoint{2.645668in}{0.500000in}}%
\pgfpathclose%
\pgfusepath{fill}%
\end{pgfscope}%
\begin{pgfscope}%
\pgfpathrectangle{\pgfqpoint{0.750000in}{0.500000in}}{\pgfqpoint{4.650000in}{3.020000in}}%
\pgfusepath{clip}%
\pgfsetbuttcap%
\pgfsetmiterjoin%
\definecolor{currentfill}{rgb}{0.000000,0.500000,0.000000}%
\pgfsetfillcolor{currentfill}%
\pgfsetlinewidth{0.000000pt}%
\definecolor{currentstroke}{rgb}{0.000000,0.000000,0.000000}%
\pgfsetstrokecolor{currentstroke}%
\pgfsetstrokeopacity{0.000000}%
\pgfsetdash{}{0pt}%
\pgfpathmoveto{\pgfqpoint{2.678693in}{0.500000in}}%
\pgfpathlineto{\pgfqpoint{2.711719in}{0.500000in}}%
\pgfpathlineto{\pgfqpoint{2.711719in}{0.669188in}}%
\pgfpathlineto{\pgfqpoint{2.678693in}{0.669188in}}%
\pgfpathlineto{\pgfqpoint{2.678693in}{0.500000in}}%
\pgfpathclose%
\pgfusepath{fill}%
\end{pgfscope}%
\begin{pgfscope}%
\pgfpathrectangle{\pgfqpoint{0.750000in}{0.500000in}}{\pgfqpoint{4.650000in}{3.020000in}}%
\pgfusepath{clip}%
\pgfsetbuttcap%
\pgfsetmiterjoin%
\definecolor{currentfill}{rgb}{0.000000,0.500000,0.000000}%
\pgfsetfillcolor{currentfill}%
\pgfsetlinewidth{0.000000pt}%
\definecolor{currentstroke}{rgb}{0.000000,0.000000,0.000000}%
\pgfsetstrokecolor{currentstroke}%
\pgfsetstrokeopacity{0.000000}%
\pgfsetdash{}{0pt}%
\pgfpathmoveto{\pgfqpoint{2.711719in}{0.500000in}}%
\pgfpathlineto{\pgfqpoint{2.744744in}{0.500000in}}%
\pgfpathlineto{\pgfqpoint{2.744744in}{0.648039in}}%
\pgfpathlineto{\pgfqpoint{2.711719in}{0.648039in}}%
\pgfpathlineto{\pgfqpoint{2.711719in}{0.500000in}}%
\pgfpathclose%
\pgfusepath{fill}%
\end{pgfscope}%
\begin{pgfscope}%
\pgfpathrectangle{\pgfqpoint{0.750000in}{0.500000in}}{\pgfqpoint{4.650000in}{3.020000in}}%
\pgfusepath{clip}%
\pgfsetbuttcap%
\pgfsetmiterjoin%
\definecolor{currentfill}{rgb}{0.000000,0.500000,0.000000}%
\pgfsetfillcolor{currentfill}%
\pgfsetlinewidth{0.000000pt}%
\definecolor{currentstroke}{rgb}{0.000000,0.000000,0.000000}%
\pgfsetstrokecolor{currentstroke}%
\pgfsetstrokeopacity{0.000000}%
\pgfsetdash{}{0pt}%
\pgfpathmoveto{\pgfqpoint{2.744744in}{0.500000in}}%
\pgfpathlineto{\pgfqpoint{2.777770in}{0.500000in}}%
\pgfpathlineto{\pgfqpoint{2.777770in}{0.702421in}}%
\pgfpathlineto{\pgfqpoint{2.744744in}{0.702421in}}%
\pgfpathlineto{\pgfqpoint{2.744744in}{0.500000in}}%
\pgfpathclose%
\pgfusepath{fill}%
\end{pgfscope}%
\begin{pgfscope}%
\pgfpathrectangle{\pgfqpoint{0.750000in}{0.500000in}}{\pgfqpoint{4.650000in}{3.020000in}}%
\pgfusepath{clip}%
\pgfsetbuttcap%
\pgfsetmiterjoin%
\definecolor{currentfill}{rgb}{0.000000,0.500000,0.000000}%
\pgfsetfillcolor{currentfill}%
\pgfsetlinewidth{0.000000pt}%
\definecolor{currentstroke}{rgb}{0.000000,0.000000,0.000000}%
\pgfsetstrokecolor{currentstroke}%
\pgfsetstrokeopacity{0.000000}%
\pgfsetdash{}{0pt}%
\pgfpathmoveto{\pgfqpoint{2.777770in}{0.500000in}}%
\pgfpathlineto{\pgfqpoint{2.810795in}{0.500000in}}%
\pgfpathlineto{\pgfqpoint{2.810795in}{0.672209in}}%
\pgfpathlineto{\pgfqpoint{2.777770in}{0.672209in}}%
\pgfpathlineto{\pgfqpoint{2.777770in}{0.500000in}}%
\pgfpathclose%
\pgfusepath{fill}%
\end{pgfscope}%
\begin{pgfscope}%
\pgfpathrectangle{\pgfqpoint{0.750000in}{0.500000in}}{\pgfqpoint{4.650000in}{3.020000in}}%
\pgfusepath{clip}%
\pgfsetbuttcap%
\pgfsetmiterjoin%
\definecolor{currentfill}{rgb}{0.000000,0.500000,0.000000}%
\pgfsetfillcolor{currentfill}%
\pgfsetlinewidth{0.000000pt}%
\definecolor{currentstroke}{rgb}{0.000000,0.000000,0.000000}%
\pgfsetstrokecolor{currentstroke}%
\pgfsetstrokeopacity{0.000000}%
\pgfsetdash{}{0pt}%
\pgfpathmoveto{\pgfqpoint{2.810795in}{0.500000in}}%
\pgfpathlineto{\pgfqpoint{2.843821in}{0.500000in}}%
\pgfpathlineto{\pgfqpoint{2.843821in}{0.714506in}}%
\pgfpathlineto{\pgfqpoint{2.810795in}{0.714506in}}%
\pgfpathlineto{\pgfqpoint{2.810795in}{0.500000in}}%
\pgfpathclose%
\pgfusepath{fill}%
\end{pgfscope}%
\begin{pgfscope}%
\pgfpathrectangle{\pgfqpoint{0.750000in}{0.500000in}}{\pgfqpoint{4.650000in}{3.020000in}}%
\pgfusepath{clip}%
\pgfsetbuttcap%
\pgfsetmiterjoin%
\definecolor{currentfill}{rgb}{0.000000,0.500000,0.000000}%
\pgfsetfillcolor{currentfill}%
\pgfsetlinewidth{0.000000pt}%
\definecolor{currentstroke}{rgb}{0.000000,0.000000,0.000000}%
\pgfsetstrokecolor{currentstroke}%
\pgfsetstrokeopacity{0.000000}%
\pgfsetdash{}{0pt}%
\pgfpathmoveto{\pgfqpoint{2.843821in}{0.500000in}}%
\pgfpathlineto{\pgfqpoint{2.876847in}{0.500000in}}%
\pgfpathlineto{\pgfqpoint{2.876847in}{0.666166in}}%
\pgfpathlineto{\pgfqpoint{2.843821in}{0.666166in}}%
\pgfpathlineto{\pgfqpoint{2.843821in}{0.500000in}}%
\pgfpathclose%
\pgfusepath{fill}%
\end{pgfscope}%
\begin{pgfscope}%
\pgfpathrectangle{\pgfqpoint{0.750000in}{0.500000in}}{\pgfqpoint{4.650000in}{3.020000in}}%
\pgfusepath{clip}%
\pgfsetbuttcap%
\pgfsetmiterjoin%
\definecolor{currentfill}{rgb}{0.000000,0.500000,0.000000}%
\pgfsetfillcolor{currentfill}%
\pgfsetlinewidth{0.000000pt}%
\definecolor{currentstroke}{rgb}{0.000000,0.000000,0.000000}%
\pgfsetstrokecolor{currentstroke}%
\pgfsetstrokeopacity{0.000000}%
\pgfsetdash{}{0pt}%
\pgfpathmoveto{\pgfqpoint{2.876847in}{0.500000in}}%
\pgfpathlineto{\pgfqpoint{2.909872in}{0.500000in}}%
\pgfpathlineto{\pgfqpoint{2.909872in}{0.732633in}}%
\pgfpathlineto{\pgfqpoint{2.876847in}{0.732633in}}%
\pgfpathlineto{\pgfqpoint{2.876847in}{0.500000in}}%
\pgfpathclose%
\pgfusepath{fill}%
\end{pgfscope}%
\begin{pgfscope}%
\pgfpathrectangle{\pgfqpoint{0.750000in}{0.500000in}}{\pgfqpoint{4.650000in}{3.020000in}}%
\pgfusepath{clip}%
\pgfsetbuttcap%
\pgfsetmiterjoin%
\definecolor{currentfill}{rgb}{0.000000,0.500000,0.000000}%
\pgfsetfillcolor{currentfill}%
\pgfsetlinewidth{0.000000pt}%
\definecolor{currentstroke}{rgb}{0.000000,0.000000,0.000000}%
\pgfsetstrokecolor{currentstroke}%
\pgfsetstrokeopacity{0.000000}%
\pgfsetdash{}{0pt}%
\pgfpathmoveto{\pgfqpoint{2.909872in}{0.500000in}}%
\pgfpathlineto{\pgfqpoint{2.942898in}{0.500000in}}%
\pgfpathlineto{\pgfqpoint{2.942898in}{0.717527in}}%
\pgfpathlineto{\pgfqpoint{2.909872in}{0.717527in}}%
\pgfpathlineto{\pgfqpoint{2.909872in}{0.500000in}}%
\pgfpathclose%
\pgfusepath{fill}%
\end{pgfscope}%
\begin{pgfscope}%
\pgfpathrectangle{\pgfqpoint{0.750000in}{0.500000in}}{\pgfqpoint{4.650000in}{3.020000in}}%
\pgfusepath{clip}%
\pgfsetbuttcap%
\pgfsetmiterjoin%
\definecolor{currentfill}{rgb}{0.000000,0.500000,0.000000}%
\pgfsetfillcolor{currentfill}%
\pgfsetlinewidth{0.000000pt}%
\definecolor{currentstroke}{rgb}{0.000000,0.000000,0.000000}%
\pgfsetstrokecolor{currentstroke}%
\pgfsetstrokeopacity{0.000000}%
\pgfsetdash{}{0pt}%
\pgfpathmoveto{\pgfqpoint{2.942898in}{0.500000in}}%
\pgfpathlineto{\pgfqpoint{2.975923in}{0.500000in}}%
\pgfpathlineto{\pgfqpoint{2.975923in}{0.699400in}}%
\pgfpathlineto{\pgfqpoint{2.942898in}{0.699400in}}%
\pgfpathlineto{\pgfqpoint{2.942898in}{0.500000in}}%
\pgfpathclose%
\pgfusepath{fill}%
\end{pgfscope}%
\begin{pgfscope}%
\pgfpathrectangle{\pgfqpoint{0.750000in}{0.500000in}}{\pgfqpoint{4.650000in}{3.020000in}}%
\pgfusepath{clip}%
\pgfsetbuttcap%
\pgfsetmiterjoin%
\definecolor{currentfill}{rgb}{0.000000,0.500000,0.000000}%
\pgfsetfillcolor{currentfill}%
\pgfsetlinewidth{0.000000pt}%
\definecolor{currentstroke}{rgb}{0.000000,0.000000,0.000000}%
\pgfsetstrokecolor{currentstroke}%
\pgfsetstrokeopacity{0.000000}%
\pgfsetdash{}{0pt}%
\pgfpathmoveto{\pgfqpoint{2.975923in}{0.500000in}}%
\pgfpathlineto{\pgfqpoint{3.008949in}{0.500000in}}%
\pgfpathlineto{\pgfqpoint{3.008949in}{0.714506in}}%
\pgfpathlineto{\pgfqpoint{2.975923in}{0.714506in}}%
\pgfpathlineto{\pgfqpoint{2.975923in}{0.500000in}}%
\pgfpathclose%
\pgfusepath{fill}%
\end{pgfscope}%
\begin{pgfscope}%
\pgfpathrectangle{\pgfqpoint{0.750000in}{0.500000in}}{\pgfqpoint{4.650000in}{3.020000in}}%
\pgfusepath{clip}%
\pgfsetbuttcap%
\pgfsetmiterjoin%
\definecolor{currentfill}{rgb}{0.000000,0.500000,0.000000}%
\pgfsetfillcolor{currentfill}%
\pgfsetlinewidth{0.000000pt}%
\definecolor{currentstroke}{rgb}{0.000000,0.000000,0.000000}%
\pgfsetstrokecolor{currentstroke}%
\pgfsetstrokeopacity{0.000000}%
\pgfsetdash{}{0pt}%
\pgfpathmoveto{\pgfqpoint{3.008949in}{0.500000in}}%
\pgfpathlineto{\pgfqpoint{3.041974in}{0.500000in}}%
\pgfpathlineto{\pgfqpoint{3.041974in}{0.699400in}}%
\pgfpathlineto{\pgfqpoint{3.008949in}{0.699400in}}%
\pgfpathlineto{\pgfqpoint{3.008949in}{0.500000in}}%
\pgfpathclose%
\pgfusepath{fill}%
\end{pgfscope}%
\begin{pgfscope}%
\pgfpathrectangle{\pgfqpoint{0.750000in}{0.500000in}}{\pgfqpoint{4.650000in}{3.020000in}}%
\pgfusepath{clip}%
\pgfsetbuttcap%
\pgfsetmiterjoin%
\definecolor{currentfill}{rgb}{0.000000,0.500000,0.000000}%
\pgfsetfillcolor{currentfill}%
\pgfsetlinewidth{0.000000pt}%
\definecolor{currentstroke}{rgb}{0.000000,0.000000,0.000000}%
\pgfsetstrokecolor{currentstroke}%
\pgfsetstrokeopacity{0.000000}%
\pgfsetdash{}{0pt}%
\pgfpathmoveto{\pgfqpoint{3.041974in}{0.500000in}}%
\pgfpathlineto{\pgfqpoint{3.075000in}{0.500000in}}%
\pgfpathlineto{\pgfqpoint{3.075000in}{0.660124in}}%
\pgfpathlineto{\pgfqpoint{3.041974in}{0.660124in}}%
\pgfpathlineto{\pgfqpoint{3.041974in}{0.500000in}}%
\pgfpathclose%
\pgfusepath{fill}%
\end{pgfscope}%
\begin{pgfscope}%
\pgfpathrectangle{\pgfqpoint{0.750000in}{0.500000in}}{\pgfqpoint{4.650000in}{3.020000in}}%
\pgfusepath{clip}%
\pgfsetbuttcap%
\pgfsetmiterjoin%
\definecolor{currentfill}{rgb}{0.000000,0.500000,0.000000}%
\pgfsetfillcolor{currentfill}%
\pgfsetlinewidth{0.000000pt}%
\definecolor{currentstroke}{rgb}{0.000000,0.000000,0.000000}%
\pgfsetstrokecolor{currentstroke}%
\pgfsetstrokeopacity{0.000000}%
\pgfsetdash{}{0pt}%
\pgfpathmoveto{\pgfqpoint{3.075000in}{0.500000in}}%
\pgfpathlineto{\pgfqpoint{3.108026in}{0.500000in}}%
\pgfpathlineto{\pgfqpoint{3.108026in}{0.684294in}}%
\pgfpathlineto{\pgfqpoint{3.075000in}{0.684294in}}%
\pgfpathlineto{\pgfqpoint{3.075000in}{0.500000in}}%
\pgfpathclose%
\pgfusepath{fill}%
\end{pgfscope}%
\begin{pgfscope}%
\pgfpathrectangle{\pgfqpoint{0.750000in}{0.500000in}}{\pgfqpoint{4.650000in}{3.020000in}}%
\pgfusepath{clip}%
\pgfsetbuttcap%
\pgfsetmiterjoin%
\definecolor{currentfill}{rgb}{0.000000,0.500000,0.000000}%
\pgfsetfillcolor{currentfill}%
\pgfsetlinewidth{0.000000pt}%
\definecolor{currentstroke}{rgb}{0.000000,0.000000,0.000000}%
\pgfsetstrokecolor{currentstroke}%
\pgfsetstrokeopacity{0.000000}%
\pgfsetdash{}{0pt}%
\pgfpathmoveto{\pgfqpoint{3.108026in}{0.500000in}}%
\pgfpathlineto{\pgfqpoint{3.141051in}{0.500000in}}%
\pgfpathlineto{\pgfqpoint{3.141051in}{0.654082in}}%
\pgfpathlineto{\pgfqpoint{3.108026in}{0.654082in}}%
\pgfpathlineto{\pgfqpoint{3.108026in}{0.500000in}}%
\pgfpathclose%
\pgfusepath{fill}%
\end{pgfscope}%
\begin{pgfscope}%
\pgfpathrectangle{\pgfqpoint{0.750000in}{0.500000in}}{\pgfqpoint{4.650000in}{3.020000in}}%
\pgfusepath{clip}%
\pgfsetbuttcap%
\pgfsetmiterjoin%
\definecolor{currentfill}{rgb}{0.000000,0.500000,0.000000}%
\pgfsetfillcolor{currentfill}%
\pgfsetlinewidth{0.000000pt}%
\definecolor{currentstroke}{rgb}{0.000000,0.000000,0.000000}%
\pgfsetstrokecolor{currentstroke}%
\pgfsetstrokeopacity{0.000000}%
\pgfsetdash{}{0pt}%
\pgfpathmoveto{\pgfqpoint{3.141051in}{0.500000in}}%
\pgfpathlineto{\pgfqpoint{3.174077in}{0.500000in}}%
\pgfpathlineto{\pgfqpoint{3.174077in}{0.645018in}}%
\pgfpathlineto{\pgfqpoint{3.141051in}{0.645018in}}%
\pgfpathlineto{\pgfqpoint{3.141051in}{0.500000in}}%
\pgfpathclose%
\pgfusepath{fill}%
\end{pgfscope}%
\begin{pgfscope}%
\pgfpathrectangle{\pgfqpoint{0.750000in}{0.500000in}}{\pgfqpoint{4.650000in}{3.020000in}}%
\pgfusepath{clip}%
\pgfsetbuttcap%
\pgfsetmiterjoin%
\definecolor{currentfill}{rgb}{0.000000,0.500000,0.000000}%
\pgfsetfillcolor{currentfill}%
\pgfsetlinewidth{0.000000pt}%
\definecolor{currentstroke}{rgb}{0.000000,0.000000,0.000000}%
\pgfsetstrokecolor{currentstroke}%
\pgfsetstrokeopacity{0.000000}%
\pgfsetdash{}{0pt}%
\pgfpathmoveto{\pgfqpoint{3.174077in}{0.500000in}}%
\pgfpathlineto{\pgfqpoint{3.207102in}{0.500000in}}%
\pgfpathlineto{\pgfqpoint{3.207102in}{0.645018in}}%
\pgfpathlineto{\pgfqpoint{3.174077in}{0.645018in}}%
\pgfpathlineto{\pgfqpoint{3.174077in}{0.500000in}}%
\pgfpathclose%
\pgfusepath{fill}%
\end{pgfscope}%
\begin{pgfscope}%
\pgfpathrectangle{\pgfqpoint{0.750000in}{0.500000in}}{\pgfqpoint{4.650000in}{3.020000in}}%
\pgfusepath{clip}%
\pgfsetbuttcap%
\pgfsetmiterjoin%
\definecolor{currentfill}{rgb}{0.000000,0.500000,0.000000}%
\pgfsetfillcolor{currentfill}%
\pgfsetlinewidth{0.000000pt}%
\definecolor{currentstroke}{rgb}{0.000000,0.000000,0.000000}%
\pgfsetstrokecolor{currentstroke}%
\pgfsetstrokeopacity{0.000000}%
\pgfsetdash{}{0pt}%
\pgfpathmoveto{\pgfqpoint{3.207102in}{0.500000in}}%
\pgfpathlineto{\pgfqpoint{3.240128in}{0.500000in}}%
\pgfpathlineto{\pgfqpoint{3.240128in}{0.593657in}}%
\pgfpathlineto{\pgfqpoint{3.207102in}{0.593657in}}%
\pgfpathlineto{\pgfqpoint{3.207102in}{0.500000in}}%
\pgfpathclose%
\pgfusepath{fill}%
\end{pgfscope}%
\begin{pgfscope}%
\pgfpathrectangle{\pgfqpoint{0.750000in}{0.500000in}}{\pgfqpoint{4.650000in}{3.020000in}}%
\pgfusepath{clip}%
\pgfsetbuttcap%
\pgfsetmiterjoin%
\definecolor{currentfill}{rgb}{0.000000,0.500000,0.000000}%
\pgfsetfillcolor{currentfill}%
\pgfsetlinewidth{0.000000pt}%
\definecolor{currentstroke}{rgb}{0.000000,0.000000,0.000000}%
\pgfsetstrokecolor{currentstroke}%
\pgfsetstrokeopacity{0.000000}%
\pgfsetdash{}{0pt}%
\pgfpathmoveto{\pgfqpoint{3.240128in}{0.500000in}}%
\pgfpathlineto{\pgfqpoint{3.273153in}{0.500000in}}%
\pgfpathlineto{\pgfqpoint{3.273153in}{0.587615in}}%
\pgfpathlineto{\pgfqpoint{3.240128in}{0.587615in}}%
\pgfpathlineto{\pgfqpoint{3.240128in}{0.500000in}}%
\pgfpathclose%
\pgfusepath{fill}%
\end{pgfscope}%
\begin{pgfscope}%
\pgfpathrectangle{\pgfqpoint{0.750000in}{0.500000in}}{\pgfqpoint{4.650000in}{3.020000in}}%
\pgfusepath{clip}%
\pgfsetbuttcap%
\pgfsetmiterjoin%
\definecolor{currentfill}{rgb}{0.000000,0.500000,0.000000}%
\pgfsetfillcolor{currentfill}%
\pgfsetlinewidth{0.000000pt}%
\definecolor{currentstroke}{rgb}{0.000000,0.000000,0.000000}%
\pgfsetstrokecolor{currentstroke}%
\pgfsetstrokeopacity{0.000000}%
\pgfsetdash{}{0pt}%
\pgfpathmoveto{\pgfqpoint{3.273153in}{0.500000in}}%
\pgfpathlineto{\pgfqpoint{3.306179in}{0.500000in}}%
\pgfpathlineto{\pgfqpoint{3.306179in}{0.551361in}}%
\pgfpathlineto{\pgfqpoint{3.273153in}{0.551361in}}%
\pgfpathlineto{\pgfqpoint{3.273153in}{0.500000in}}%
\pgfpathclose%
\pgfusepath{fill}%
\end{pgfscope}%
\begin{pgfscope}%
\pgfpathrectangle{\pgfqpoint{0.750000in}{0.500000in}}{\pgfqpoint{4.650000in}{3.020000in}}%
\pgfusepath{clip}%
\pgfsetbuttcap%
\pgfsetmiterjoin%
\definecolor{currentfill}{rgb}{0.000000,0.500000,0.000000}%
\pgfsetfillcolor{currentfill}%
\pgfsetlinewidth{0.000000pt}%
\definecolor{currentstroke}{rgb}{0.000000,0.000000,0.000000}%
\pgfsetstrokecolor{currentstroke}%
\pgfsetstrokeopacity{0.000000}%
\pgfsetdash{}{0pt}%
\pgfpathmoveto{\pgfqpoint{3.306179in}{0.500000in}}%
\pgfpathlineto{\pgfqpoint{3.339205in}{0.500000in}}%
\pgfpathlineto{\pgfqpoint{3.339205in}{0.533233in}}%
\pgfpathlineto{\pgfqpoint{3.306179in}{0.533233in}}%
\pgfpathlineto{\pgfqpoint{3.306179in}{0.500000in}}%
\pgfpathclose%
\pgfusepath{fill}%
\end{pgfscope}%
\begin{pgfscope}%
\pgfpathrectangle{\pgfqpoint{0.750000in}{0.500000in}}{\pgfqpoint{4.650000in}{3.020000in}}%
\pgfusepath{clip}%
\pgfsetbuttcap%
\pgfsetmiterjoin%
\definecolor{currentfill}{rgb}{0.000000,0.500000,0.000000}%
\pgfsetfillcolor{currentfill}%
\pgfsetlinewidth{0.000000pt}%
\definecolor{currentstroke}{rgb}{0.000000,0.000000,0.000000}%
\pgfsetstrokecolor{currentstroke}%
\pgfsetstrokeopacity{0.000000}%
\pgfsetdash{}{0pt}%
\pgfpathmoveto{\pgfqpoint{3.339205in}{0.500000in}}%
\pgfpathlineto{\pgfqpoint{3.372230in}{0.500000in}}%
\pgfpathlineto{\pgfqpoint{3.372230in}{0.539276in}}%
\pgfpathlineto{\pgfqpoint{3.339205in}{0.539276in}}%
\pgfpathlineto{\pgfqpoint{3.339205in}{0.500000in}}%
\pgfpathclose%
\pgfusepath{fill}%
\end{pgfscope}%
\begin{pgfscope}%
\pgfpathrectangle{\pgfqpoint{0.750000in}{0.500000in}}{\pgfqpoint{4.650000in}{3.020000in}}%
\pgfusepath{clip}%
\pgfsetbuttcap%
\pgfsetmiterjoin%
\definecolor{currentfill}{rgb}{0.000000,0.500000,0.000000}%
\pgfsetfillcolor{currentfill}%
\pgfsetlinewidth{0.000000pt}%
\definecolor{currentstroke}{rgb}{0.000000,0.000000,0.000000}%
\pgfsetstrokecolor{currentstroke}%
\pgfsetstrokeopacity{0.000000}%
\pgfsetdash{}{0pt}%
\pgfpathmoveto{\pgfqpoint{3.372230in}{0.500000in}}%
\pgfpathlineto{\pgfqpoint{3.405256in}{0.500000in}}%
\pgfpathlineto{\pgfqpoint{3.405256in}{0.515106in}}%
\pgfpathlineto{\pgfqpoint{3.372230in}{0.515106in}}%
\pgfpathlineto{\pgfqpoint{3.372230in}{0.500000in}}%
\pgfpathclose%
\pgfusepath{fill}%
\end{pgfscope}%
\begin{pgfscope}%
\pgfpathrectangle{\pgfqpoint{0.750000in}{0.500000in}}{\pgfqpoint{4.650000in}{3.020000in}}%
\pgfusepath{clip}%
\pgfsetbuttcap%
\pgfsetmiterjoin%
\definecolor{currentfill}{rgb}{0.000000,0.500000,0.000000}%
\pgfsetfillcolor{currentfill}%
\pgfsetlinewidth{0.000000pt}%
\definecolor{currentstroke}{rgb}{0.000000,0.000000,0.000000}%
\pgfsetstrokecolor{currentstroke}%
\pgfsetstrokeopacity{0.000000}%
\pgfsetdash{}{0pt}%
\pgfpathmoveto{\pgfqpoint{3.405256in}{0.500000in}}%
\pgfpathlineto{\pgfqpoint{3.438281in}{0.500000in}}%
\pgfpathlineto{\pgfqpoint{3.438281in}{0.527191in}}%
\pgfpathlineto{\pgfqpoint{3.405256in}{0.527191in}}%
\pgfpathlineto{\pgfqpoint{3.405256in}{0.500000in}}%
\pgfpathclose%
\pgfusepath{fill}%
\end{pgfscope}%
\begin{pgfscope}%
\pgfpathrectangle{\pgfqpoint{0.750000in}{0.500000in}}{\pgfqpoint{4.650000in}{3.020000in}}%
\pgfusepath{clip}%
\pgfsetbuttcap%
\pgfsetmiterjoin%
\definecolor{currentfill}{rgb}{0.000000,0.500000,0.000000}%
\pgfsetfillcolor{currentfill}%
\pgfsetlinewidth{0.000000pt}%
\definecolor{currentstroke}{rgb}{0.000000,0.000000,0.000000}%
\pgfsetstrokecolor{currentstroke}%
\pgfsetstrokeopacity{0.000000}%
\pgfsetdash{}{0pt}%
\pgfpathmoveto{\pgfqpoint{3.438281in}{0.500000in}}%
\pgfpathlineto{\pgfqpoint{3.471307in}{0.500000in}}%
\pgfpathlineto{\pgfqpoint{3.471307in}{0.506042in}}%
\pgfpathlineto{\pgfqpoint{3.438281in}{0.506042in}}%
\pgfpathlineto{\pgfqpoint{3.438281in}{0.500000in}}%
\pgfpathclose%
\pgfusepath{fill}%
\end{pgfscope}%
\begin{pgfscope}%
\pgfpathrectangle{\pgfqpoint{0.750000in}{0.500000in}}{\pgfqpoint{4.650000in}{3.020000in}}%
\pgfusepath{clip}%
\pgfsetbuttcap%
\pgfsetmiterjoin%
\definecolor{currentfill}{rgb}{0.000000,0.500000,0.000000}%
\pgfsetfillcolor{currentfill}%
\pgfsetlinewidth{0.000000pt}%
\definecolor{currentstroke}{rgb}{0.000000,0.000000,0.000000}%
\pgfsetstrokecolor{currentstroke}%
\pgfsetstrokeopacity{0.000000}%
\pgfsetdash{}{0pt}%
\pgfpathmoveto{\pgfqpoint{3.471307in}{0.500000in}}%
\pgfpathlineto{\pgfqpoint{3.504332in}{0.500000in}}%
\pgfpathlineto{\pgfqpoint{3.504332in}{0.503021in}}%
\pgfpathlineto{\pgfqpoint{3.471307in}{0.503021in}}%
\pgfpathlineto{\pgfqpoint{3.471307in}{0.500000in}}%
\pgfpathclose%
\pgfusepath{fill}%
\end{pgfscope}%
\begin{pgfscope}%
\pgfpathrectangle{\pgfqpoint{0.750000in}{0.500000in}}{\pgfqpoint{4.650000in}{3.020000in}}%
\pgfusepath{clip}%
\pgfsetbuttcap%
\pgfsetmiterjoin%
\definecolor{currentfill}{rgb}{0.000000,0.500000,0.000000}%
\pgfsetfillcolor{currentfill}%
\pgfsetlinewidth{0.000000pt}%
\definecolor{currentstroke}{rgb}{0.000000,0.000000,0.000000}%
\pgfsetstrokecolor{currentstroke}%
\pgfsetstrokeopacity{0.000000}%
\pgfsetdash{}{0pt}%
\pgfpathmoveto{\pgfqpoint{3.504332in}{0.500000in}}%
\pgfpathlineto{\pgfqpoint{3.537358in}{0.500000in}}%
\pgfpathlineto{\pgfqpoint{3.537358in}{0.524170in}}%
\pgfpathlineto{\pgfqpoint{3.504332in}{0.524170in}}%
\pgfpathlineto{\pgfqpoint{3.504332in}{0.500000in}}%
\pgfpathclose%
\pgfusepath{fill}%
\end{pgfscope}%
\begin{pgfscope}%
\pgfpathrectangle{\pgfqpoint{0.750000in}{0.500000in}}{\pgfqpoint{4.650000in}{3.020000in}}%
\pgfusepath{clip}%
\pgfsetbuttcap%
\pgfsetmiterjoin%
\definecolor{currentfill}{rgb}{0.000000,0.500000,0.000000}%
\pgfsetfillcolor{currentfill}%
\pgfsetlinewidth{0.000000pt}%
\definecolor{currentstroke}{rgb}{0.000000,0.000000,0.000000}%
\pgfsetstrokecolor{currentstroke}%
\pgfsetstrokeopacity{0.000000}%
\pgfsetdash{}{0pt}%
\pgfpathmoveto{\pgfqpoint{3.537358in}{0.500000in}}%
\pgfpathlineto{\pgfqpoint{3.570384in}{0.500000in}}%
\pgfpathlineto{\pgfqpoint{3.570384in}{0.509064in}}%
\pgfpathlineto{\pgfqpoint{3.537358in}{0.509064in}}%
\pgfpathlineto{\pgfqpoint{3.537358in}{0.500000in}}%
\pgfpathclose%
\pgfusepath{fill}%
\end{pgfscope}%
\begin{pgfscope}%
\pgfpathrectangle{\pgfqpoint{0.750000in}{0.500000in}}{\pgfqpoint{4.650000in}{3.020000in}}%
\pgfusepath{clip}%
\pgfsetbuttcap%
\pgfsetmiterjoin%
\definecolor{currentfill}{rgb}{0.000000,0.500000,0.000000}%
\pgfsetfillcolor{currentfill}%
\pgfsetlinewidth{0.000000pt}%
\definecolor{currentstroke}{rgb}{0.000000,0.000000,0.000000}%
\pgfsetstrokecolor{currentstroke}%
\pgfsetstrokeopacity{0.000000}%
\pgfsetdash{}{0pt}%
\pgfpathmoveto{\pgfqpoint{3.570384in}{0.500000in}}%
\pgfpathlineto{\pgfqpoint{3.603409in}{0.500000in}}%
\pgfpathlineto{\pgfqpoint{3.603409in}{0.503021in}}%
\pgfpathlineto{\pgfqpoint{3.570384in}{0.503021in}}%
\pgfpathlineto{\pgfqpoint{3.570384in}{0.500000in}}%
\pgfpathclose%
\pgfusepath{fill}%
\end{pgfscope}%
\begin{pgfscope}%
\pgfpathrectangle{\pgfqpoint{0.750000in}{0.500000in}}{\pgfqpoint{4.650000in}{3.020000in}}%
\pgfusepath{clip}%
\pgfsetbuttcap%
\pgfsetmiterjoin%
\definecolor{currentfill}{rgb}{0.000000,0.500000,0.000000}%
\pgfsetfillcolor{currentfill}%
\pgfsetlinewidth{0.000000pt}%
\definecolor{currentstroke}{rgb}{0.000000,0.000000,0.000000}%
\pgfsetstrokecolor{currentstroke}%
\pgfsetstrokeopacity{0.000000}%
\pgfsetdash{}{0pt}%
\pgfpathmoveto{\pgfqpoint{3.603409in}{0.500000in}}%
\pgfpathlineto{\pgfqpoint{3.636435in}{0.500000in}}%
\pgfpathlineto{\pgfqpoint{3.636435in}{0.500000in}}%
\pgfpathlineto{\pgfqpoint{3.603409in}{0.500000in}}%
\pgfpathlineto{\pgfqpoint{3.603409in}{0.500000in}}%
\pgfpathclose%
\pgfusepath{fill}%
\end{pgfscope}%
\begin{pgfscope}%
\pgfpathrectangle{\pgfqpoint{0.750000in}{0.500000in}}{\pgfqpoint{4.650000in}{3.020000in}}%
\pgfusepath{clip}%
\pgfsetbuttcap%
\pgfsetmiterjoin%
\definecolor{currentfill}{rgb}{0.000000,0.500000,0.000000}%
\pgfsetfillcolor{currentfill}%
\pgfsetlinewidth{0.000000pt}%
\definecolor{currentstroke}{rgb}{0.000000,0.000000,0.000000}%
\pgfsetstrokecolor{currentstroke}%
\pgfsetstrokeopacity{0.000000}%
\pgfsetdash{}{0pt}%
\pgfpathmoveto{\pgfqpoint{3.636435in}{0.500000in}}%
\pgfpathlineto{\pgfqpoint{3.669460in}{0.500000in}}%
\pgfpathlineto{\pgfqpoint{3.669460in}{0.503021in}}%
\pgfpathlineto{\pgfqpoint{3.636435in}{0.503021in}}%
\pgfpathlineto{\pgfqpoint{3.636435in}{0.500000in}}%
\pgfpathclose%
\pgfusepath{fill}%
\end{pgfscope}%
\begin{pgfscope}%
\pgfpathrectangle{\pgfqpoint{0.750000in}{0.500000in}}{\pgfqpoint{4.650000in}{3.020000in}}%
\pgfusepath{clip}%
\pgfsetbuttcap%
\pgfsetmiterjoin%
\definecolor{currentfill}{rgb}{0.000000,0.500000,0.000000}%
\pgfsetfillcolor{currentfill}%
\pgfsetlinewidth{0.000000pt}%
\definecolor{currentstroke}{rgb}{0.000000,0.000000,0.000000}%
\pgfsetstrokecolor{currentstroke}%
\pgfsetstrokeopacity{0.000000}%
\pgfsetdash{}{0pt}%
\pgfpathmoveto{\pgfqpoint{3.669460in}{0.500000in}}%
\pgfpathlineto{\pgfqpoint{3.702486in}{0.500000in}}%
\pgfpathlineto{\pgfqpoint{3.702486in}{0.503021in}}%
\pgfpathlineto{\pgfqpoint{3.669460in}{0.503021in}}%
\pgfpathlineto{\pgfqpoint{3.669460in}{0.500000in}}%
\pgfpathclose%
\pgfusepath{fill}%
\end{pgfscope}%
\begin{pgfscope}%
\pgfpathrectangle{\pgfqpoint{0.750000in}{0.500000in}}{\pgfqpoint{4.650000in}{3.020000in}}%
\pgfusepath{clip}%
\pgfsetbuttcap%
\pgfsetmiterjoin%
\definecolor{currentfill}{rgb}{0.000000,0.500000,0.000000}%
\pgfsetfillcolor{currentfill}%
\pgfsetlinewidth{0.000000pt}%
\definecolor{currentstroke}{rgb}{0.000000,0.000000,0.000000}%
\pgfsetstrokecolor{currentstroke}%
\pgfsetstrokeopacity{0.000000}%
\pgfsetdash{}{0pt}%
\pgfpathmoveto{\pgfqpoint{3.702486in}{0.500000in}}%
\pgfpathlineto{\pgfqpoint{3.735511in}{0.500000in}}%
\pgfpathlineto{\pgfqpoint{3.735511in}{0.500000in}}%
\pgfpathlineto{\pgfqpoint{3.702486in}{0.500000in}}%
\pgfpathlineto{\pgfqpoint{3.702486in}{0.500000in}}%
\pgfpathclose%
\pgfusepath{fill}%
\end{pgfscope}%
\begin{pgfscope}%
\pgfpathrectangle{\pgfqpoint{0.750000in}{0.500000in}}{\pgfqpoint{4.650000in}{3.020000in}}%
\pgfusepath{clip}%
\pgfsetbuttcap%
\pgfsetmiterjoin%
\definecolor{currentfill}{rgb}{0.000000,0.500000,0.000000}%
\pgfsetfillcolor{currentfill}%
\pgfsetlinewidth{0.000000pt}%
\definecolor{currentstroke}{rgb}{0.000000,0.000000,0.000000}%
\pgfsetstrokecolor{currentstroke}%
\pgfsetstrokeopacity{0.000000}%
\pgfsetdash{}{0pt}%
\pgfpathmoveto{\pgfqpoint{3.735511in}{0.500000in}}%
\pgfpathlineto{\pgfqpoint{3.768537in}{0.500000in}}%
\pgfpathlineto{\pgfqpoint{3.768537in}{0.500000in}}%
\pgfpathlineto{\pgfqpoint{3.735511in}{0.500000in}}%
\pgfpathlineto{\pgfqpoint{3.735511in}{0.500000in}}%
\pgfpathclose%
\pgfusepath{fill}%
\end{pgfscope}%
\begin{pgfscope}%
\pgfpathrectangle{\pgfqpoint{0.750000in}{0.500000in}}{\pgfqpoint{4.650000in}{3.020000in}}%
\pgfusepath{clip}%
\pgfsetbuttcap%
\pgfsetmiterjoin%
\definecolor{currentfill}{rgb}{0.000000,0.500000,0.000000}%
\pgfsetfillcolor{currentfill}%
\pgfsetlinewidth{0.000000pt}%
\definecolor{currentstroke}{rgb}{0.000000,0.000000,0.000000}%
\pgfsetstrokecolor{currentstroke}%
\pgfsetstrokeopacity{0.000000}%
\pgfsetdash{}{0pt}%
\pgfpathmoveto{\pgfqpoint{3.768537in}{0.500000in}}%
\pgfpathlineto{\pgfqpoint{3.801562in}{0.500000in}}%
\pgfpathlineto{\pgfqpoint{3.801562in}{0.503021in}}%
\pgfpathlineto{\pgfqpoint{3.768537in}{0.503021in}}%
\pgfpathlineto{\pgfqpoint{3.768537in}{0.500000in}}%
\pgfpathclose%
\pgfusepath{fill}%
\end{pgfscope}%
\begin{pgfscope}%
\pgfpathrectangle{\pgfqpoint{0.750000in}{0.500000in}}{\pgfqpoint{4.650000in}{3.020000in}}%
\pgfusepath{clip}%
\pgfsetbuttcap%
\pgfsetmiterjoin%
\definecolor{currentfill}{rgb}{0.000000,0.500000,0.000000}%
\pgfsetfillcolor{currentfill}%
\pgfsetlinewidth{0.000000pt}%
\definecolor{currentstroke}{rgb}{0.000000,0.000000,0.000000}%
\pgfsetstrokecolor{currentstroke}%
\pgfsetstrokeopacity{0.000000}%
\pgfsetdash{}{0pt}%
\pgfpathmoveto{\pgfqpoint{3.801562in}{0.500000in}}%
\pgfpathlineto{\pgfqpoint{3.834588in}{0.500000in}}%
\pgfpathlineto{\pgfqpoint{3.834588in}{0.503021in}}%
\pgfpathlineto{\pgfqpoint{3.801562in}{0.503021in}}%
\pgfpathlineto{\pgfqpoint{3.801562in}{0.500000in}}%
\pgfpathclose%
\pgfusepath{fill}%
\end{pgfscope}%
\begin{pgfscope}%
\pgfpathrectangle{\pgfqpoint{0.750000in}{0.500000in}}{\pgfqpoint{4.650000in}{3.020000in}}%
\pgfusepath{clip}%
\pgfsetbuttcap%
\pgfsetmiterjoin%
\definecolor{currentfill}{rgb}{0.000000,0.500000,0.000000}%
\pgfsetfillcolor{currentfill}%
\pgfsetlinewidth{0.000000pt}%
\definecolor{currentstroke}{rgb}{0.000000,0.000000,0.000000}%
\pgfsetstrokecolor{currentstroke}%
\pgfsetstrokeopacity{0.000000}%
\pgfsetdash{}{0pt}%
\pgfpathmoveto{\pgfqpoint{3.834588in}{0.500000in}}%
\pgfpathlineto{\pgfqpoint{3.867614in}{0.500000in}}%
\pgfpathlineto{\pgfqpoint{3.867614in}{0.500000in}}%
\pgfpathlineto{\pgfqpoint{3.834588in}{0.500000in}}%
\pgfpathlineto{\pgfqpoint{3.834588in}{0.500000in}}%
\pgfpathclose%
\pgfusepath{fill}%
\end{pgfscope}%
\begin{pgfscope}%
\pgfpathrectangle{\pgfqpoint{0.750000in}{0.500000in}}{\pgfqpoint{4.650000in}{3.020000in}}%
\pgfusepath{clip}%
\pgfsetbuttcap%
\pgfsetmiterjoin%
\definecolor{currentfill}{rgb}{0.000000,0.500000,0.000000}%
\pgfsetfillcolor{currentfill}%
\pgfsetlinewidth{0.000000pt}%
\definecolor{currentstroke}{rgb}{0.000000,0.000000,0.000000}%
\pgfsetstrokecolor{currentstroke}%
\pgfsetstrokeopacity{0.000000}%
\pgfsetdash{}{0pt}%
\pgfpathmoveto{\pgfqpoint{3.867614in}{0.500000in}}%
\pgfpathlineto{\pgfqpoint{3.900639in}{0.500000in}}%
\pgfpathlineto{\pgfqpoint{3.900639in}{0.500000in}}%
\pgfpathlineto{\pgfqpoint{3.867614in}{0.500000in}}%
\pgfpathlineto{\pgfqpoint{3.867614in}{0.500000in}}%
\pgfpathclose%
\pgfusepath{fill}%
\end{pgfscope}%
\begin{pgfscope}%
\pgfpathrectangle{\pgfqpoint{0.750000in}{0.500000in}}{\pgfqpoint{4.650000in}{3.020000in}}%
\pgfusepath{clip}%
\pgfsetbuttcap%
\pgfsetmiterjoin%
\definecolor{currentfill}{rgb}{0.000000,0.500000,0.000000}%
\pgfsetfillcolor{currentfill}%
\pgfsetlinewidth{0.000000pt}%
\definecolor{currentstroke}{rgb}{0.000000,0.000000,0.000000}%
\pgfsetstrokecolor{currentstroke}%
\pgfsetstrokeopacity{0.000000}%
\pgfsetdash{}{0pt}%
\pgfpathmoveto{\pgfqpoint{3.900639in}{0.500000in}}%
\pgfpathlineto{\pgfqpoint{3.933665in}{0.500000in}}%
\pgfpathlineto{\pgfqpoint{3.933665in}{0.503021in}}%
\pgfpathlineto{\pgfqpoint{3.900639in}{0.503021in}}%
\pgfpathlineto{\pgfqpoint{3.900639in}{0.500000in}}%
\pgfpathclose%
\pgfusepath{fill}%
\end{pgfscope}%
\begin{pgfscope}%
\pgfpathrectangle{\pgfqpoint{0.750000in}{0.500000in}}{\pgfqpoint{4.650000in}{3.020000in}}%
\pgfusepath{clip}%
\pgfsetbuttcap%
\pgfsetmiterjoin%
\definecolor{currentfill}{rgb}{0.000000,0.500000,0.000000}%
\pgfsetfillcolor{currentfill}%
\pgfsetlinewidth{0.000000pt}%
\definecolor{currentstroke}{rgb}{0.000000,0.000000,0.000000}%
\pgfsetstrokecolor{currentstroke}%
\pgfsetstrokeopacity{0.000000}%
\pgfsetdash{}{0pt}%
\pgfpathmoveto{\pgfqpoint{3.933665in}{0.500000in}}%
\pgfpathlineto{\pgfqpoint{3.966690in}{0.500000in}}%
\pgfpathlineto{\pgfqpoint{3.966690in}{0.500000in}}%
\pgfpathlineto{\pgfqpoint{3.933665in}{0.500000in}}%
\pgfpathlineto{\pgfqpoint{3.933665in}{0.500000in}}%
\pgfpathclose%
\pgfusepath{fill}%
\end{pgfscope}%
\begin{pgfscope}%
\pgfpathrectangle{\pgfqpoint{0.750000in}{0.500000in}}{\pgfqpoint{4.650000in}{3.020000in}}%
\pgfusepath{clip}%
\pgfsetbuttcap%
\pgfsetmiterjoin%
\definecolor{currentfill}{rgb}{0.000000,0.500000,0.000000}%
\pgfsetfillcolor{currentfill}%
\pgfsetlinewidth{0.000000pt}%
\definecolor{currentstroke}{rgb}{0.000000,0.000000,0.000000}%
\pgfsetstrokecolor{currentstroke}%
\pgfsetstrokeopacity{0.000000}%
\pgfsetdash{}{0pt}%
\pgfpathmoveto{\pgfqpoint{3.966690in}{0.500000in}}%
\pgfpathlineto{\pgfqpoint{3.999716in}{0.500000in}}%
\pgfpathlineto{\pgfqpoint{3.999716in}{0.500000in}}%
\pgfpathlineto{\pgfqpoint{3.966690in}{0.500000in}}%
\pgfpathlineto{\pgfqpoint{3.966690in}{0.500000in}}%
\pgfpathclose%
\pgfusepath{fill}%
\end{pgfscope}%
\begin{pgfscope}%
\pgfpathrectangle{\pgfqpoint{0.750000in}{0.500000in}}{\pgfqpoint{4.650000in}{3.020000in}}%
\pgfusepath{clip}%
\pgfsetbuttcap%
\pgfsetmiterjoin%
\definecolor{currentfill}{rgb}{0.000000,0.500000,0.000000}%
\pgfsetfillcolor{currentfill}%
\pgfsetlinewidth{0.000000pt}%
\definecolor{currentstroke}{rgb}{0.000000,0.000000,0.000000}%
\pgfsetstrokecolor{currentstroke}%
\pgfsetstrokeopacity{0.000000}%
\pgfsetdash{}{0pt}%
\pgfpathmoveto{\pgfqpoint{3.999716in}{0.500000in}}%
\pgfpathlineto{\pgfqpoint{4.032741in}{0.500000in}}%
\pgfpathlineto{\pgfqpoint{4.032741in}{0.503021in}}%
\pgfpathlineto{\pgfqpoint{3.999716in}{0.503021in}}%
\pgfpathlineto{\pgfqpoint{3.999716in}{0.500000in}}%
\pgfpathclose%
\pgfusepath{fill}%
\end{pgfscope}%
\begin{pgfscope}%
\pgfpathrectangle{\pgfqpoint{0.750000in}{0.500000in}}{\pgfqpoint{4.650000in}{3.020000in}}%
\pgfusepath{clip}%
\pgfsetbuttcap%
\pgfsetmiterjoin%
\definecolor{currentfill}{rgb}{0.000000,0.500000,0.000000}%
\pgfsetfillcolor{currentfill}%
\pgfsetlinewidth{0.000000pt}%
\definecolor{currentstroke}{rgb}{0.000000,0.000000,0.000000}%
\pgfsetstrokecolor{currentstroke}%
\pgfsetstrokeopacity{0.000000}%
\pgfsetdash{}{0pt}%
\pgfpathmoveto{\pgfqpoint{4.032741in}{0.500000in}}%
\pgfpathlineto{\pgfqpoint{4.065767in}{0.500000in}}%
\pgfpathlineto{\pgfqpoint{4.065767in}{0.503021in}}%
\pgfpathlineto{\pgfqpoint{4.032741in}{0.503021in}}%
\pgfpathlineto{\pgfqpoint{4.032741in}{0.500000in}}%
\pgfpathclose%
\pgfusepath{fill}%
\end{pgfscope}%
\begin{pgfscope}%
\pgfpathrectangle{\pgfqpoint{0.750000in}{0.500000in}}{\pgfqpoint{4.650000in}{3.020000in}}%
\pgfusepath{clip}%
\pgfsetbuttcap%
\pgfsetmiterjoin%
\definecolor{currentfill}{rgb}{0.000000,0.500000,0.000000}%
\pgfsetfillcolor{currentfill}%
\pgfsetlinewidth{0.000000pt}%
\definecolor{currentstroke}{rgb}{0.000000,0.000000,0.000000}%
\pgfsetstrokecolor{currentstroke}%
\pgfsetstrokeopacity{0.000000}%
\pgfsetdash{}{0pt}%
\pgfpathmoveto{\pgfqpoint{4.065767in}{0.500000in}}%
\pgfpathlineto{\pgfqpoint{4.098793in}{0.500000in}}%
\pgfpathlineto{\pgfqpoint{4.098793in}{0.500000in}}%
\pgfpathlineto{\pgfqpoint{4.065767in}{0.500000in}}%
\pgfpathlineto{\pgfqpoint{4.065767in}{0.500000in}}%
\pgfpathclose%
\pgfusepath{fill}%
\end{pgfscope}%
\begin{pgfscope}%
\pgfpathrectangle{\pgfqpoint{0.750000in}{0.500000in}}{\pgfqpoint{4.650000in}{3.020000in}}%
\pgfusepath{clip}%
\pgfsetbuttcap%
\pgfsetmiterjoin%
\definecolor{currentfill}{rgb}{0.000000,0.500000,0.000000}%
\pgfsetfillcolor{currentfill}%
\pgfsetlinewidth{0.000000pt}%
\definecolor{currentstroke}{rgb}{0.000000,0.000000,0.000000}%
\pgfsetstrokecolor{currentstroke}%
\pgfsetstrokeopacity{0.000000}%
\pgfsetdash{}{0pt}%
\pgfpathmoveto{\pgfqpoint{4.098793in}{0.500000in}}%
\pgfpathlineto{\pgfqpoint{4.131818in}{0.500000in}}%
\pgfpathlineto{\pgfqpoint{4.131818in}{0.500000in}}%
\pgfpathlineto{\pgfqpoint{4.098793in}{0.500000in}}%
\pgfpathlineto{\pgfqpoint{4.098793in}{0.500000in}}%
\pgfpathclose%
\pgfusepath{fill}%
\end{pgfscope}%
\begin{pgfscope}%
\pgfpathrectangle{\pgfqpoint{0.750000in}{0.500000in}}{\pgfqpoint{4.650000in}{3.020000in}}%
\pgfusepath{clip}%
\pgfsetbuttcap%
\pgfsetmiterjoin%
\definecolor{currentfill}{rgb}{0.000000,0.500000,0.000000}%
\pgfsetfillcolor{currentfill}%
\pgfsetlinewidth{0.000000pt}%
\definecolor{currentstroke}{rgb}{0.000000,0.000000,0.000000}%
\pgfsetstrokecolor{currentstroke}%
\pgfsetstrokeopacity{0.000000}%
\pgfsetdash{}{0pt}%
\pgfpathmoveto{\pgfqpoint{4.131818in}{0.500000in}}%
\pgfpathlineto{\pgfqpoint{4.164844in}{0.500000in}}%
\pgfpathlineto{\pgfqpoint{4.164844in}{0.503021in}}%
\pgfpathlineto{\pgfqpoint{4.131818in}{0.503021in}}%
\pgfpathlineto{\pgfqpoint{4.131818in}{0.500000in}}%
\pgfpathclose%
\pgfusepath{fill}%
\end{pgfscope}%
\begin{pgfscope}%
\pgfpathrectangle{\pgfqpoint{0.750000in}{0.500000in}}{\pgfqpoint{4.650000in}{3.020000in}}%
\pgfusepath{clip}%
\pgfsetbuttcap%
\pgfsetmiterjoin%
\definecolor{currentfill}{rgb}{0.000000,0.500000,0.000000}%
\pgfsetfillcolor{currentfill}%
\pgfsetlinewidth{0.000000pt}%
\definecolor{currentstroke}{rgb}{0.000000,0.000000,0.000000}%
\pgfsetstrokecolor{currentstroke}%
\pgfsetstrokeopacity{0.000000}%
\pgfsetdash{}{0pt}%
\pgfpathmoveto{\pgfqpoint{4.164844in}{0.500000in}}%
\pgfpathlineto{\pgfqpoint{4.197869in}{0.500000in}}%
\pgfpathlineto{\pgfqpoint{4.197869in}{0.500000in}}%
\pgfpathlineto{\pgfqpoint{4.164844in}{0.500000in}}%
\pgfpathlineto{\pgfqpoint{4.164844in}{0.500000in}}%
\pgfpathclose%
\pgfusepath{fill}%
\end{pgfscope}%
\begin{pgfscope}%
\pgfpathrectangle{\pgfqpoint{0.750000in}{0.500000in}}{\pgfqpoint{4.650000in}{3.020000in}}%
\pgfusepath{clip}%
\pgfsetbuttcap%
\pgfsetmiterjoin%
\definecolor{currentfill}{rgb}{0.000000,0.500000,0.000000}%
\pgfsetfillcolor{currentfill}%
\pgfsetlinewidth{0.000000pt}%
\definecolor{currentstroke}{rgb}{0.000000,0.000000,0.000000}%
\pgfsetstrokecolor{currentstroke}%
\pgfsetstrokeopacity{0.000000}%
\pgfsetdash{}{0pt}%
\pgfpathmoveto{\pgfqpoint{4.197869in}{0.500000in}}%
\pgfpathlineto{\pgfqpoint{4.230895in}{0.500000in}}%
\pgfpathlineto{\pgfqpoint{4.230895in}{0.500000in}}%
\pgfpathlineto{\pgfqpoint{4.197869in}{0.500000in}}%
\pgfpathlineto{\pgfqpoint{4.197869in}{0.500000in}}%
\pgfpathclose%
\pgfusepath{fill}%
\end{pgfscope}%
\begin{pgfscope}%
\pgfpathrectangle{\pgfqpoint{0.750000in}{0.500000in}}{\pgfqpoint{4.650000in}{3.020000in}}%
\pgfusepath{clip}%
\pgfsetbuttcap%
\pgfsetmiterjoin%
\definecolor{currentfill}{rgb}{0.000000,0.500000,0.000000}%
\pgfsetfillcolor{currentfill}%
\pgfsetlinewidth{0.000000pt}%
\definecolor{currentstroke}{rgb}{0.000000,0.000000,0.000000}%
\pgfsetstrokecolor{currentstroke}%
\pgfsetstrokeopacity{0.000000}%
\pgfsetdash{}{0pt}%
\pgfpathmoveto{\pgfqpoint{4.230895in}{0.500000in}}%
\pgfpathlineto{\pgfqpoint{4.263920in}{0.500000in}}%
\pgfpathlineto{\pgfqpoint{4.263920in}{0.500000in}}%
\pgfpathlineto{\pgfqpoint{4.230895in}{0.500000in}}%
\pgfpathlineto{\pgfqpoint{4.230895in}{0.500000in}}%
\pgfpathclose%
\pgfusepath{fill}%
\end{pgfscope}%
\begin{pgfscope}%
\pgfpathrectangle{\pgfqpoint{0.750000in}{0.500000in}}{\pgfqpoint{4.650000in}{3.020000in}}%
\pgfusepath{clip}%
\pgfsetbuttcap%
\pgfsetmiterjoin%
\definecolor{currentfill}{rgb}{0.000000,0.500000,0.000000}%
\pgfsetfillcolor{currentfill}%
\pgfsetlinewidth{0.000000pt}%
\definecolor{currentstroke}{rgb}{0.000000,0.000000,0.000000}%
\pgfsetstrokecolor{currentstroke}%
\pgfsetstrokeopacity{0.000000}%
\pgfsetdash{}{0pt}%
\pgfpathmoveto{\pgfqpoint{4.263920in}{0.500000in}}%
\pgfpathlineto{\pgfqpoint{4.296946in}{0.500000in}}%
\pgfpathlineto{\pgfqpoint{4.296946in}{0.503021in}}%
\pgfpathlineto{\pgfqpoint{4.263920in}{0.503021in}}%
\pgfpathlineto{\pgfqpoint{4.263920in}{0.500000in}}%
\pgfpathclose%
\pgfusepath{fill}%
\end{pgfscope}%
\begin{pgfscope}%
\pgfpathrectangle{\pgfqpoint{0.750000in}{0.500000in}}{\pgfqpoint{4.650000in}{3.020000in}}%
\pgfusepath{clip}%
\pgfsetbuttcap%
\pgfsetmiterjoin%
\definecolor{currentfill}{rgb}{0.000000,0.500000,0.000000}%
\pgfsetfillcolor{currentfill}%
\pgfsetlinewidth{0.000000pt}%
\definecolor{currentstroke}{rgb}{0.000000,0.000000,0.000000}%
\pgfsetstrokecolor{currentstroke}%
\pgfsetstrokeopacity{0.000000}%
\pgfsetdash{}{0pt}%
\pgfpathmoveto{\pgfqpoint{4.296946in}{0.500000in}}%
\pgfpathlineto{\pgfqpoint{4.329972in}{0.500000in}}%
\pgfpathlineto{\pgfqpoint{4.329972in}{0.500000in}}%
\pgfpathlineto{\pgfqpoint{4.296946in}{0.500000in}}%
\pgfpathlineto{\pgfqpoint{4.296946in}{0.500000in}}%
\pgfpathclose%
\pgfusepath{fill}%
\end{pgfscope}%
\begin{pgfscope}%
\pgfpathrectangle{\pgfqpoint{0.750000in}{0.500000in}}{\pgfqpoint{4.650000in}{3.020000in}}%
\pgfusepath{clip}%
\pgfsetbuttcap%
\pgfsetmiterjoin%
\definecolor{currentfill}{rgb}{0.000000,0.500000,0.000000}%
\pgfsetfillcolor{currentfill}%
\pgfsetlinewidth{0.000000pt}%
\definecolor{currentstroke}{rgb}{0.000000,0.000000,0.000000}%
\pgfsetstrokecolor{currentstroke}%
\pgfsetstrokeopacity{0.000000}%
\pgfsetdash{}{0pt}%
\pgfpathmoveto{\pgfqpoint{4.329972in}{0.500000in}}%
\pgfpathlineto{\pgfqpoint{4.362997in}{0.500000in}}%
\pgfpathlineto{\pgfqpoint{4.362997in}{0.500000in}}%
\pgfpathlineto{\pgfqpoint{4.329972in}{0.500000in}}%
\pgfpathlineto{\pgfqpoint{4.329972in}{0.500000in}}%
\pgfpathclose%
\pgfusepath{fill}%
\end{pgfscope}%
\begin{pgfscope}%
\pgfpathrectangle{\pgfqpoint{0.750000in}{0.500000in}}{\pgfqpoint{4.650000in}{3.020000in}}%
\pgfusepath{clip}%
\pgfsetbuttcap%
\pgfsetmiterjoin%
\definecolor{currentfill}{rgb}{0.000000,0.500000,0.000000}%
\pgfsetfillcolor{currentfill}%
\pgfsetlinewidth{0.000000pt}%
\definecolor{currentstroke}{rgb}{0.000000,0.000000,0.000000}%
\pgfsetstrokecolor{currentstroke}%
\pgfsetstrokeopacity{0.000000}%
\pgfsetdash{}{0pt}%
\pgfpathmoveto{\pgfqpoint{4.362997in}{0.500000in}}%
\pgfpathlineto{\pgfqpoint{4.396023in}{0.500000in}}%
\pgfpathlineto{\pgfqpoint{4.396023in}{0.503021in}}%
\pgfpathlineto{\pgfqpoint{4.362997in}{0.503021in}}%
\pgfpathlineto{\pgfqpoint{4.362997in}{0.500000in}}%
\pgfpathclose%
\pgfusepath{fill}%
\end{pgfscope}%
\begin{pgfscope}%
\pgfpathrectangle{\pgfqpoint{0.750000in}{0.500000in}}{\pgfqpoint{4.650000in}{3.020000in}}%
\pgfusepath{clip}%
\pgfsetbuttcap%
\pgfsetmiterjoin%
\definecolor{currentfill}{rgb}{0.000000,0.500000,0.000000}%
\pgfsetfillcolor{currentfill}%
\pgfsetlinewidth{0.000000pt}%
\definecolor{currentstroke}{rgb}{0.000000,0.000000,0.000000}%
\pgfsetstrokecolor{currentstroke}%
\pgfsetstrokeopacity{0.000000}%
\pgfsetdash{}{0pt}%
\pgfpathmoveto{\pgfqpoint{4.396023in}{0.500000in}}%
\pgfpathlineto{\pgfqpoint{4.429048in}{0.500000in}}%
\pgfpathlineto{\pgfqpoint{4.429048in}{0.500000in}}%
\pgfpathlineto{\pgfqpoint{4.396023in}{0.500000in}}%
\pgfpathlineto{\pgfqpoint{4.396023in}{0.500000in}}%
\pgfpathclose%
\pgfusepath{fill}%
\end{pgfscope}%
\begin{pgfscope}%
\pgfpathrectangle{\pgfqpoint{0.750000in}{0.500000in}}{\pgfqpoint{4.650000in}{3.020000in}}%
\pgfusepath{clip}%
\pgfsetbuttcap%
\pgfsetmiterjoin%
\definecolor{currentfill}{rgb}{0.000000,0.500000,0.000000}%
\pgfsetfillcolor{currentfill}%
\pgfsetlinewidth{0.000000pt}%
\definecolor{currentstroke}{rgb}{0.000000,0.000000,0.000000}%
\pgfsetstrokecolor{currentstroke}%
\pgfsetstrokeopacity{0.000000}%
\pgfsetdash{}{0pt}%
\pgfpathmoveto{\pgfqpoint{4.429048in}{0.500000in}}%
\pgfpathlineto{\pgfqpoint{4.462074in}{0.500000in}}%
\pgfpathlineto{\pgfqpoint{4.462074in}{0.500000in}}%
\pgfpathlineto{\pgfqpoint{4.429048in}{0.500000in}}%
\pgfpathlineto{\pgfqpoint{4.429048in}{0.500000in}}%
\pgfpathclose%
\pgfusepath{fill}%
\end{pgfscope}%
\begin{pgfscope}%
\pgfpathrectangle{\pgfqpoint{0.750000in}{0.500000in}}{\pgfqpoint{4.650000in}{3.020000in}}%
\pgfusepath{clip}%
\pgfsetbuttcap%
\pgfsetmiterjoin%
\definecolor{currentfill}{rgb}{0.000000,0.500000,0.000000}%
\pgfsetfillcolor{currentfill}%
\pgfsetlinewidth{0.000000pt}%
\definecolor{currentstroke}{rgb}{0.000000,0.000000,0.000000}%
\pgfsetstrokecolor{currentstroke}%
\pgfsetstrokeopacity{0.000000}%
\pgfsetdash{}{0pt}%
\pgfpathmoveto{\pgfqpoint{4.462074in}{0.500000in}}%
\pgfpathlineto{\pgfqpoint{4.495099in}{0.500000in}}%
\pgfpathlineto{\pgfqpoint{4.495099in}{0.500000in}}%
\pgfpathlineto{\pgfqpoint{4.462074in}{0.500000in}}%
\pgfpathlineto{\pgfqpoint{4.462074in}{0.500000in}}%
\pgfpathclose%
\pgfusepath{fill}%
\end{pgfscope}%
\begin{pgfscope}%
\pgfpathrectangle{\pgfqpoint{0.750000in}{0.500000in}}{\pgfqpoint{4.650000in}{3.020000in}}%
\pgfusepath{clip}%
\pgfsetbuttcap%
\pgfsetmiterjoin%
\definecolor{currentfill}{rgb}{0.000000,0.500000,0.000000}%
\pgfsetfillcolor{currentfill}%
\pgfsetlinewidth{0.000000pt}%
\definecolor{currentstroke}{rgb}{0.000000,0.000000,0.000000}%
\pgfsetstrokecolor{currentstroke}%
\pgfsetstrokeopacity{0.000000}%
\pgfsetdash{}{0pt}%
\pgfpathmoveto{\pgfqpoint{4.495099in}{0.500000in}}%
\pgfpathlineto{\pgfqpoint{4.528125in}{0.500000in}}%
\pgfpathlineto{\pgfqpoint{4.528125in}{0.503021in}}%
\pgfpathlineto{\pgfqpoint{4.495099in}{0.503021in}}%
\pgfpathlineto{\pgfqpoint{4.495099in}{0.500000in}}%
\pgfpathclose%
\pgfusepath{fill}%
\end{pgfscope}%
\begin{pgfscope}%
\pgfpathrectangle{\pgfqpoint{0.750000in}{0.500000in}}{\pgfqpoint{4.650000in}{3.020000in}}%
\pgfusepath{clip}%
\pgfsetbuttcap%
\pgfsetmiterjoin%
\definecolor{currentfill}{rgb}{0.000000,0.500000,0.000000}%
\pgfsetfillcolor{currentfill}%
\pgfsetlinewidth{0.000000pt}%
\definecolor{currentstroke}{rgb}{0.000000,0.000000,0.000000}%
\pgfsetstrokecolor{currentstroke}%
\pgfsetstrokeopacity{0.000000}%
\pgfsetdash{}{0pt}%
\pgfpathmoveto{\pgfqpoint{4.528125in}{0.500000in}}%
\pgfpathlineto{\pgfqpoint{4.561151in}{0.500000in}}%
\pgfpathlineto{\pgfqpoint{4.561151in}{0.503021in}}%
\pgfpathlineto{\pgfqpoint{4.528125in}{0.503021in}}%
\pgfpathlineto{\pgfqpoint{4.528125in}{0.500000in}}%
\pgfpathclose%
\pgfusepath{fill}%
\end{pgfscope}%
\begin{pgfscope}%
\pgfpathrectangle{\pgfqpoint{0.750000in}{0.500000in}}{\pgfqpoint{4.650000in}{3.020000in}}%
\pgfusepath{clip}%
\pgfsetbuttcap%
\pgfsetmiterjoin%
\definecolor{currentfill}{rgb}{0.000000,0.500000,0.000000}%
\pgfsetfillcolor{currentfill}%
\pgfsetlinewidth{0.000000pt}%
\definecolor{currentstroke}{rgb}{0.000000,0.000000,0.000000}%
\pgfsetstrokecolor{currentstroke}%
\pgfsetstrokeopacity{0.000000}%
\pgfsetdash{}{0pt}%
\pgfpathmoveto{\pgfqpoint{4.561151in}{0.500000in}}%
\pgfpathlineto{\pgfqpoint{4.594176in}{0.500000in}}%
\pgfpathlineto{\pgfqpoint{4.594176in}{0.503021in}}%
\pgfpathlineto{\pgfqpoint{4.561151in}{0.503021in}}%
\pgfpathlineto{\pgfqpoint{4.561151in}{0.500000in}}%
\pgfpathclose%
\pgfusepath{fill}%
\end{pgfscope}%
\begin{pgfscope}%
\pgfpathrectangle{\pgfqpoint{0.750000in}{0.500000in}}{\pgfqpoint{4.650000in}{3.020000in}}%
\pgfusepath{clip}%
\pgfsetbuttcap%
\pgfsetmiterjoin%
\definecolor{currentfill}{rgb}{0.000000,0.500000,0.000000}%
\pgfsetfillcolor{currentfill}%
\pgfsetlinewidth{0.000000pt}%
\definecolor{currentstroke}{rgb}{0.000000,0.000000,0.000000}%
\pgfsetstrokecolor{currentstroke}%
\pgfsetstrokeopacity{0.000000}%
\pgfsetdash{}{0pt}%
\pgfpathmoveto{\pgfqpoint{4.594176in}{0.500000in}}%
\pgfpathlineto{\pgfqpoint{4.627202in}{0.500000in}}%
\pgfpathlineto{\pgfqpoint{4.627202in}{0.500000in}}%
\pgfpathlineto{\pgfqpoint{4.594176in}{0.500000in}}%
\pgfpathlineto{\pgfqpoint{4.594176in}{0.500000in}}%
\pgfpathclose%
\pgfusepath{fill}%
\end{pgfscope}%
\begin{pgfscope}%
\pgfpathrectangle{\pgfqpoint{0.750000in}{0.500000in}}{\pgfqpoint{4.650000in}{3.020000in}}%
\pgfusepath{clip}%
\pgfsetbuttcap%
\pgfsetmiterjoin%
\definecolor{currentfill}{rgb}{0.000000,0.500000,0.000000}%
\pgfsetfillcolor{currentfill}%
\pgfsetlinewidth{0.000000pt}%
\definecolor{currentstroke}{rgb}{0.000000,0.000000,0.000000}%
\pgfsetstrokecolor{currentstroke}%
\pgfsetstrokeopacity{0.000000}%
\pgfsetdash{}{0pt}%
\pgfpathmoveto{\pgfqpoint{4.627202in}{0.500000in}}%
\pgfpathlineto{\pgfqpoint{4.660227in}{0.500000in}}%
\pgfpathlineto{\pgfqpoint{4.660227in}{0.503021in}}%
\pgfpathlineto{\pgfqpoint{4.627202in}{0.503021in}}%
\pgfpathlineto{\pgfqpoint{4.627202in}{0.500000in}}%
\pgfpathclose%
\pgfusepath{fill}%
\end{pgfscope}%
\begin{pgfscope}%
\pgfpathrectangle{\pgfqpoint{0.750000in}{0.500000in}}{\pgfqpoint{4.650000in}{3.020000in}}%
\pgfusepath{clip}%
\pgfsetbuttcap%
\pgfsetmiterjoin%
\definecolor{currentfill}{rgb}{0.000000,0.500000,0.000000}%
\pgfsetfillcolor{currentfill}%
\pgfsetlinewidth{0.000000pt}%
\definecolor{currentstroke}{rgb}{0.000000,0.000000,0.000000}%
\pgfsetstrokecolor{currentstroke}%
\pgfsetstrokeopacity{0.000000}%
\pgfsetdash{}{0pt}%
\pgfpathmoveto{\pgfqpoint{4.660227in}{0.500000in}}%
\pgfpathlineto{\pgfqpoint{4.693253in}{0.500000in}}%
\pgfpathlineto{\pgfqpoint{4.693253in}{0.503021in}}%
\pgfpathlineto{\pgfqpoint{4.660227in}{0.503021in}}%
\pgfpathlineto{\pgfqpoint{4.660227in}{0.500000in}}%
\pgfpathclose%
\pgfusepath{fill}%
\end{pgfscope}%
\begin{pgfscope}%
\pgfpathrectangle{\pgfqpoint{0.750000in}{0.500000in}}{\pgfqpoint{4.650000in}{3.020000in}}%
\pgfusepath{clip}%
\pgfsetbuttcap%
\pgfsetmiterjoin%
\definecolor{currentfill}{rgb}{0.000000,0.500000,0.000000}%
\pgfsetfillcolor{currentfill}%
\pgfsetlinewidth{0.000000pt}%
\definecolor{currentstroke}{rgb}{0.000000,0.000000,0.000000}%
\pgfsetstrokecolor{currentstroke}%
\pgfsetstrokeopacity{0.000000}%
\pgfsetdash{}{0pt}%
\pgfpathmoveto{\pgfqpoint{4.693253in}{0.500000in}}%
\pgfpathlineto{\pgfqpoint{4.726278in}{0.500000in}}%
\pgfpathlineto{\pgfqpoint{4.726278in}{0.503021in}}%
\pgfpathlineto{\pgfqpoint{4.693253in}{0.503021in}}%
\pgfpathlineto{\pgfqpoint{4.693253in}{0.500000in}}%
\pgfpathclose%
\pgfusepath{fill}%
\end{pgfscope}%
\begin{pgfscope}%
\pgfpathrectangle{\pgfqpoint{0.750000in}{0.500000in}}{\pgfqpoint{4.650000in}{3.020000in}}%
\pgfusepath{clip}%
\pgfsetbuttcap%
\pgfsetmiterjoin%
\definecolor{currentfill}{rgb}{0.000000,0.500000,0.000000}%
\pgfsetfillcolor{currentfill}%
\pgfsetlinewidth{0.000000pt}%
\definecolor{currentstroke}{rgb}{0.000000,0.000000,0.000000}%
\pgfsetstrokecolor{currentstroke}%
\pgfsetstrokeopacity{0.000000}%
\pgfsetdash{}{0pt}%
\pgfpathmoveto{\pgfqpoint{4.726278in}{0.500000in}}%
\pgfpathlineto{\pgfqpoint{4.759304in}{0.500000in}}%
\pgfpathlineto{\pgfqpoint{4.759304in}{0.500000in}}%
\pgfpathlineto{\pgfqpoint{4.726278in}{0.500000in}}%
\pgfpathlineto{\pgfqpoint{4.726278in}{0.500000in}}%
\pgfpathclose%
\pgfusepath{fill}%
\end{pgfscope}%
\begin{pgfscope}%
\pgfpathrectangle{\pgfqpoint{0.750000in}{0.500000in}}{\pgfqpoint{4.650000in}{3.020000in}}%
\pgfusepath{clip}%
\pgfsetbuttcap%
\pgfsetmiterjoin%
\definecolor{currentfill}{rgb}{0.000000,0.500000,0.000000}%
\pgfsetfillcolor{currentfill}%
\pgfsetlinewidth{0.000000pt}%
\definecolor{currentstroke}{rgb}{0.000000,0.000000,0.000000}%
\pgfsetstrokecolor{currentstroke}%
\pgfsetstrokeopacity{0.000000}%
\pgfsetdash{}{0pt}%
\pgfpathmoveto{\pgfqpoint{4.759304in}{0.500000in}}%
\pgfpathlineto{\pgfqpoint{4.792330in}{0.500000in}}%
\pgfpathlineto{\pgfqpoint{4.792330in}{0.500000in}}%
\pgfpathlineto{\pgfqpoint{4.759304in}{0.500000in}}%
\pgfpathlineto{\pgfqpoint{4.759304in}{0.500000in}}%
\pgfpathclose%
\pgfusepath{fill}%
\end{pgfscope}%
\begin{pgfscope}%
\pgfpathrectangle{\pgfqpoint{0.750000in}{0.500000in}}{\pgfqpoint{4.650000in}{3.020000in}}%
\pgfusepath{clip}%
\pgfsetbuttcap%
\pgfsetmiterjoin%
\definecolor{currentfill}{rgb}{0.000000,0.500000,0.000000}%
\pgfsetfillcolor{currentfill}%
\pgfsetlinewidth{0.000000pt}%
\definecolor{currentstroke}{rgb}{0.000000,0.000000,0.000000}%
\pgfsetstrokecolor{currentstroke}%
\pgfsetstrokeopacity{0.000000}%
\pgfsetdash{}{0pt}%
\pgfpathmoveto{\pgfqpoint{4.792330in}{0.500000in}}%
\pgfpathlineto{\pgfqpoint{4.825355in}{0.500000in}}%
\pgfpathlineto{\pgfqpoint{4.825355in}{0.506042in}}%
\pgfpathlineto{\pgfqpoint{4.792330in}{0.506042in}}%
\pgfpathlineto{\pgfqpoint{4.792330in}{0.500000in}}%
\pgfpathclose%
\pgfusepath{fill}%
\end{pgfscope}%
\begin{pgfscope}%
\pgfpathrectangle{\pgfqpoint{0.750000in}{0.500000in}}{\pgfqpoint{4.650000in}{3.020000in}}%
\pgfusepath{clip}%
\pgfsetbuttcap%
\pgfsetmiterjoin%
\definecolor{currentfill}{rgb}{0.000000,0.500000,0.000000}%
\pgfsetfillcolor{currentfill}%
\pgfsetlinewidth{0.000000pt}%
\definecolor{currentstroke}{rgb}{0.000000,0.000000,0.000000}%
\pgfsetstrokecolor{currentstroke}%
\pgfsetstrokeopacity{0.000000}%
\pgfsetdash{}{0pt}%
\pgfpathmoveto{\pgfqpoint{4.825355in}{0.500000in}}%
\pgfpathlineto{\pgfqpoint{4.858381in}{0.500000in}}%
\pgfpathlineto{\pgfqpoint{4.858381in}{0.500000in}}%
\pgfpathlineto{\pgfqpoint{4.825355in}{0.500000in}}%
\pgfpathlineto{\pgfqpoint{4.825355in}{0.500000in}}%
\pgfpathclose%
\pgfusepath{fill}%
\end{pgfscope}%
\begin{pgfscope}%
\pgfpathrectangle{\pgfqpoint{0.750000in}{0.500000in}}{\pgfqpoint{4.650000in}{3.020000in}}%
\pgfusepath{clip}%
\pgfsetbuttcap%
\pgfsetmiterjoin%
\definecolor{currentfill}{rgb}{0.000000,0.500000,0.000000}%
\pgfsetfillcolor{currentfill}%
\pgfsetlinewidth{0.000000pt}%
\definecolor{currentstroke}{rgb}{0.000000,0.000000,0.000000}%
\pgfsetstrokecolor{currentstroke}%
\pgfsetstrokeopacity{0.000000}%
\pgfsetdash{}{0pt}%
\pgfpathmoveto{\pgfqpoint{4.858381in}{0.500000in}}%
\pgfpathlineto{\pgfqpoint{4.891406in}{0.500000in}}%
\pgfpathlineto{\pgfqpoint{4.891406in}{0.506042in}}%
\pgfpathlineto{\pgfqpoint{4.858381in}{0.506042in}}%
\pgfpathlineto{\pgfqpoint{4.858381in}{0.500000in}}%
\pgfpathclose%
\pgfusepath{fill}%
\end{pgfscope}%
\begin{pgfscope}%
\pgfpathrectangle{\pgfqpoint{0.750000in}{0.500000in}}{\pgfqpoint{4.650000in}{3.020000in}}%
\pgfusepath{clip}%
\pgfsetbuttcap%
\pgfsetmiterjoin%
\definecolor{currentfill}{rgb}{0.000000,0.500000,0.000000}%
\pgfsetfillcolor{currentfill}%
\pgfsetlinewidth{0.000000pt}%
\definecolor{currentstroke}{rgb}{0.000000,0.000000,0.000000}%
\pgfsetstrokecolor{currentstroke}%
\pgfsetstrokeopacity{0.000000}%
\pgfsetdash{}{0pt}%
\pgfpathmoveto{\pgfqpoint{4.891406in}{0.500000in}}%
\pgfpathlineto{\pgfqpoint{4.924432in}{0.500000in}}%
\pgfpathlineto{\pgfqpoint{4.924432in}{0.503021in}}%
\pgfpathlineto{\pgfqpoint{4.891406in}{0.503021in}}%
\pgfpathlineto{\pgfqpoint{4.891406in}{0.500000in}}%
\pgfpathclose%
\pgfusepath{fill}%
\end{pgfscope}%
\begin{pgfscope}%
\pgfpathrectangle{\pgfqpoint{0.750000in}{0.500000in}}{\pgfqpoint{4.650000in}{3.020000in}}%
\pgfusepath{clip}%
\pgfsetbuttcap%
\pgfsetmiterjoin%
\definecolor{currentfill}{rgb}{0.000000,0.500000,0.000000}%
\pgfsetfillcolor{currentfill}%
\pgfsetlinewidth{0.000000pt}%
\definecolor{currentstroke}{rgb}{0.000000,0.000000,0.000000}%
\pgfsetstrokecolor{currentstroke}%
\pgfsetstrokeopacity{0.000000}%
\pgfsetdash{}{0pt}%
\pgfpathmoveto{\pgfqpoint{4.924432in}{0.500000in}}%
\pgfpathlineto{\pgfqpoint{4.957457in}{0.500000in}}%
\pgfpathlineto{\pgfqpoint{4.957457in}{0.500000in}}%
\pgfpathlineto{\pgfqpoint{4.924432in}{0.500000in}}%
\pgfpathlineto{\pgfqpoint{4.924432in}{0.500000in}}%
\pgfpathclose%
\pgfusepath{fill}%
\end{pgfscope}%
\begin{pgfscope}%
\pgfpathrectangle{\pgfqpoint{0.750000in}{0.500000in}}{\pgfqpoint{4.650000in}{3.020000in}}%
\pgfusepath{clip}%
\pgfsetbuttcap%
\pgfsetmiterjoin%
\definecolor{currentfill}{rgb}{0.000000,0.500000,0.000000}%
\pgfsetfillcolor{currentfill}%
\pgfsetlinewidth{0.000000pt}%
\definecolor{currentstroke}{rgb}{0.000000,0.000000,0.000000}%
\pgfsetstrokecolor{currentstroke}%
\pgfsetstrokeopacity{0.000000}%
\pgfsetdash{}{0pt}%
\pgfpathmoveto{\pgfqpoint{4.957457in}{0.500000in}}%
\pgfpathlineto{\pgfqpoint{4.990483in}{0.500000in}}%
\pgfpathlineto{\pgfqpoint{4.990483in}{0.506042in}}%
\pgfpathlineto{\pgfqpoint{4.957457in}{0.506042in}}%
\pgfpathlineto{\pgfqpoint{4.957457in}{0.500000in}}%
\pgfpathclose%
\pgfusepath{fill}%
\end{pgfscope}%
\begin{pgfscope}%
\pgfpathrectangle{\pgfqpoint{0.750000in}{0.500000in}}{\pgfqpoint{4.650000in}{3.020000in}}%
\pgfusepath{clip}%
\pgfsetbuttcap%
\pgfsetmiterjoin%
\definecolor{currentfill}{rgb}{0.000000,0.500000,0.000000}%
\pgfsetfillcolor{currentfill}%
\pgfsetlinewidth{0.000000pt}%
\definecolor{currentstroke}{rgb}{0.000000,0.000000,0.000000}%
\pgfsetstrokecolor{currentstroke}%
\pgfsetstrokeopacity{0.000000}%
\pgfsetdash{}{0pt}%
\pgfpathmoveto{\pgfqpoint{4.990483in}{0.500000in}}%
\pgfpathlineto{\pgfqpoint{5.023509in}{0.500000in}}%
\pgfpathlineto{\pgfqpoint{5.023509in}{0.500000in}}%
\pgfpathlineto{\pgfqpoint{4.990483in}{0.500000in}}%
\pgfpathlineto{\pgfqpoint{4.990483in}{0.500000in}}%
\pgfpathclose%
\pgfusepath{fill}%
\end{pgfscope}%
\begin{pgfscope}%
\pgfpathrectangle{\pgfqpoint{0.750000in}{0.500000in}}{\pgfqpoint{4.650000in}{3.020000in}}%
\pgfusepath{clip}%
\pgfsetbuttcap%
\pgfsetmiterjoin%
\definecolor{currentfill}{rgb}{0.000000,0.500000,0.000000}%
\pgfsetfillcolor{currentfill}%
\pgfsetlinewidth{0.000000pt}%
\definecolor{currentstroke}{rgb}{0.000000,0.000000,0.000000}%
\pgfsetstrokecolor{currentstroke}%
\pgfsetstrokeopacity{0.000000}%
\pgfsetdash{}{0pt}%
\pgfpathmoveto{\pgfqpoint{5.023509in}{0.500000in}}%
\pgfpathlineto{\pgfqpoint{5.056534in}{0.500000in}}%
\pgfpathlineto{\pgfqpoint{5.056534in}{0.509064in}}%
\pgfpathlineto{\pgfqpoint{5.023509in}{0.509064in}}%
\pgfpathlineto{\pgfqpoint{5.023509in}{0.500000in}}%
\pgfpathclose%
\pgfusepath{fill}%
\end{pgfscope}%
\begin{pgfscope}%
\pgfpathrectangle{\pgfqpoint{0.750000in}{0.500000in}}{\pgfqpoint{4.650000in}{3.020000in}}%
\pgfusepath{clip}%
\pgfsetbuttcap%
\pgfsetmiterjoin%
\definecolor{currentfill}{rgb}{0.000000,0.500000,0.000000}%
\pgfsetfillcolor{currentfill}%
\pgfsetlinewidth{0.000000pt}%
\definecolor{currentstroke}{rgb}{0.000000,0.000000,0.000000}%
\pgfsetstrokecolor{currentstroke}%
\pgfsetstrokeopacity{0.000000}%
\pgfsetdash{}{0pt}%
\pgfpathmoveto{\pgfqpoint{5.056534in}{0.500000in}}%
\pgfpathlineto{\pgfqpoint{5.089560in}{0.500000in}}%
\pgfpathlineto{\pgfqpoint{5.089560in}{0.509064in}}%
\pgfpathlineto{\pgfqpoint{5.056534in}{0.509064in}}%
\pgfpathlineto{\pgfqpoint{5.056534in}{0.500000in}}%
\pgfpathclose%
\pgfusepath{fill}%
\end{pgfscope}%
\begin{pgfscope}%
\pgfpathrectangle{\pgfqpoint{0.750000in}{0.500000in}}{\pgfqpoint{4.650000in}{3.020000in}}%
\pgfusepath{clip}%
\pgfsetbuttcap%
\pgfsetmiterjoin%
\definecolor{currentfill}{rgb}{0.000000,0.500000,0.000000}%
\pgfsetfillcolor{currentfill}%
\pgfsetlinewidth{0.000000pt}%
\definecolor{currentstroke}{rgb}{0.000000,0.000000,0.000000}%
\pgfsetstrokecolor{currentstroke}%
\pgfsetstrokeopacity{0.000000}%
\pgfsetdash{}{0pt}%
\pgfpathmoveto{\pgfqpoint{5.089560in}{0.500000in}}%
\pgfpathlineto{\pgfqpoint{5.122585in}{0.500000in}}%
\pgfpathlineto{\pgfqpoint{5.122585in}{0.503021in}}%
\pgfpathlineto{\pgfqpoint{5.089560in}{0.503021in}}%
\pgfpathlineto{\pgfqpoint{5.089560in}{0.500000in}}%
\pgfpathclose%
\pgfusepath{fill}%
\end{pgfscope}%
\begin{pgfscope}%
\pgfpathrectangle{\pgfqpoint{0.750000in}{0.500000in}}{\pgfqpoint{4.650000in}{3.020000in}}%
\pgfusepath{clip}%
\pgfsetbuttcap%
\pgfsetmiterjoin%
\definecolor{currentfill}{rgb}{0.000000,0.500000,0.000000}%
\pgfsetfillcolor{currentfill}%
\pgfsetlinewidth{0.000000pt}%
\definecolor{currentstroke}{rgb}{0.000000,0.000000,0.000000}%
\pgfsetstrokecolor{currentstroke}%
\pgfsetstrokeopacity{0.000000}%
\pgfsetdash{}{0pt}%
\pgfpathmoveto{\pgfqpoint{5.122585in}{0.500000in}}%
\pgfpathlineto{\pgfqpoint{5.155611in}{0.500000in}}%
\pgfpathlineto{\pgfqpoint{5.155611in}{0.506042in}}%
\pgfpathlineto{\pgfqpoint{5.122585in}{0.506042in}}%
\pgfpathlineto{\pgfqpoint{5.122585in}{0.500000in}}%
\pgfpathclose%
\pgfusepath{fill}%
\end{pgfscope}%
\begin{pgfscope}%
\pgfpathrectangle{\pgfqpoint{0.750000in}{0.500000in}}{\pgfqpoint{4.650000in}{3.020000in}}%
\pgfusepath{clip}%
\pgfsetbuttcap%
\pgfsetmiterjoin%
\definecolor{currentfill}{rgb}{0.000000,0.500000,0.000000}%
\pgfsetfillcolor{currentfill}%
\pgfsetlinewidth{0.000000pt}%
\definecolor{currentstroke}{rgb}{0.000000,0.000000,0.000000}%
\pgfsetstrokecolor{currentstroke}%
\pgfsetstrokeopacity{0.000000}%
\pgfsetdash{}{0pt}%
\pgfpathmoveto{\pgfqpoint{5.155611in}{0.500000in}}%
\pgfpathlineto{\pgfqpoint{5.188636in}{0.500000in}}%
\pgfpathlineto{\pgfqpoint{5.188636in}{0.515106in}}%
\pgfpathlineto{\pgfqpoint{5.155611in}{0.515106in}}%
\pgfpathlineto{\pgfqpoint{5.155611in}{0.500000in}}%
\pgfpathclose%
\pgfusepath{fill}%
\end{pgfscope}%
\begin{pgfscope}%
\pgfsetbuttcap%
\pgfsetroundjoin%
\definecolor{currentfill}{rgb}{0.000000,0.000000,0.000000}%
\pgfsetfillcolor{currentfill}%
\pgfsetlinewidth{0.803000pt}%
\definecolor{currentstroke}{rgb}{0.000000,0.000000,0.000000}%
\pgfsetstrokecolor{currentstroke}%
\pgfsetdash{}{0pt}%
\pgfsys@defobject{currentmarker}{\pgfqpoint{0.000000in}{-0.048611in}}{\pgfqpoint{0.000000in}{0.000000in}}{%
\pgfpathmoveto{\pgfqpoint{0.000000in}{0.000000in}}%
\pgfpathlineto{\pgfqpoint{0.000000in}{-0.048611in}}%
\pgfusepath{stroke,fill}%
}%
\begin{pgfscope}%
\pgfsys@transformshift{0.945286in}{0.500000in}%
\pgfsys@useobject{currentmarker}{}%
\end{pgfscope}%
\end{pgfscope}%
\begin{pgfscope}%
\definecolor{textcolor}{rgb}{0.000000,0.000000,0.000000}%
\pgfsetstrokecolor{textcolor}%
\pgfsetfillcolor{textcolor}%
\pgftext[x=0.945286in,y=0.402778in,,top]{\color{textcolor}\rmfamily\fontsize{13.000000}{15.600000}\selectfont \(\displaystyle {0.0}\)}%
\end{pgfscope}%
\begin{pgfscope}%
\pgfsetbuttcap%
\pgfsetroundjoin%
\definecolor{currentfill}{rgb}{0.000000,0.000000,0.000000}%
\pgfsetfillcolor{currentfill}%
\pgfsetlinewidth{0.803000pt}%
\definecolor{currentstroke}{rgb}{0.000000,0.000000,0.000000}%
\pgfsetstrokecolor{currentstroke}%
\pgfsetdash{}{0pt}%
\pgfsys@defobject{currentmarker}{\pgfqpoint{0.000000in}{-0.048611in}}{\pgfqpoint{0.000000in}{0.000000in}}{%
\pgfpathmoveto{\pgfqpoint{0.000000in}{0.000000in}}%
\pgfpathlineto{\pgfqpoint{0.000000in}{-0.048611in}}%
\pgfusepath{stroke,fill}%
}%
\begin{pgfscope}%
\pgfsys@transformshift{1.792379in}{0.500000in}%
\pgfsys@useobject{currentmarker}{}%
\end{pgfscope}%
\end{pgfscope}%
\begin{pgfscope}%
\definecolor{textcolor}{rgb}{0.000000,0.000000,0.000000}%
\pgfsetstrokecolor{textcolor}%
\pgfsetfillcolor{textcolor}%
\pgftext[x=1.792379in,y=0.402778in,,top]{\color{textcolor}\rmfamily\fontsize{13.000000}{15.600000}\selectfont \(\displaystyle {0.1}\)}%
\end{pgfscope}%
\begin{pgfscope}%
\pgfsetbuttcap%
\pgfsetroundjoin%
\definecolor{currentfill}{rgb}{0.000000,0.000000,0.000000}%
\pgfsetfillcolor{currentfill}%
\pgfsetlinewidth{0.803000pt}%
\definecolor{currentstroke}{rgb}{0.000000,0.000000,0.000000}%
\pgfsetstrokecolor{currentstroke}%
\pgfsetdash{}{0pt}%
\pgfsys@defobject{currentmarker}{\pgfqpoint{0.000000in}{-0.048611in}}{\pgfqpoint{0.000000in}{0.000000in}}{%
\pgfpathmoveto{\pgfqpoint{0.000000in}{0.000000in}}%
\pgfpathlineto{\pgfqpoint{0.000000in}{-0.048611in}}%
\pgfusepath{stroke,fill}%
}%
\begin{pgfscope}%
\pgfsys@transformshift{2.639473in}{0.500000in}%
\pgfsys@useobject{currentmarker}{}%
\end{pgfscope}%
\end{pgfscope}%
\begin{pgfscope}%
\definecolor{textcolor}{rgb}{0.000000,0.000000,0.000000}%
\pgfsetstrokecolor{textcolor}%
\pgfsetfillcolor{textcolor}%
\pgftext[x=2.639473in,y=0.402778in,,top]{\color{textcolor}\rmfamily\fontsize{13.000000}{15.600000}\selectfont \(\displaystyle {0.2}\)}%
\end{pgfscope}%
\begin{pgfscope}%
\pgfsetbuttcap%
\pgfsetroundjoin%
\definecolor{currentfill}{rgb}{0.000000,0.000000,0.000000}%
\pgfsetfillcolor{currentfill}%
\pgfsetlinewidth{0.803000pt}%
\definecolor{currentstroke}{rgb}{0.000000,0.000000,0.000000}%
\pgfsetstrokecolor{currentstroke}%
\pgfsetdash{}{0pt}%
\pgfsys@defobject{currentmarker}{\pgfqpoint{0.000000in}{-0.048611in}}{\pgfqpoint{0.000000in}{0.000000in}}{%
\pgfpathmoveto{\pgfqpoint{0.000000in}{0.000000in}}%
\pgfpathlineto{\pgfqpoint{0.000000in}{-0.048611in}}%
\pgfusepath{stroke,fill}%
}%
\begin{pgfscope}%
\pgfsys@transformshift{3.486567in}{0.500000in}%
\pgfsys@useobject{currentmarker}{}%
\end{pgfscope}%
\end{pgfscope}%
\begin{pgfscope}%
\definecolor{textcolor}{rgb}{0.000000,0.000000,0.000000}%
\pgfsetstrokecolor{textcolor}%
\pgfsetfillcolor{textcolor}%
\pgftext[x=3.486567in,y=0.402778in,,top]{\color{textcolor}\rmfamily\fontsize{13.000000}{15.600000}\selectfont \(\displaystyle {0.3}\)}%
\end{pgfscope}%
\begin{pgfscope}%
\pgfsetbuttcap%
\pgfsetroundjoin%
\definecolor{currentfill}{rgb}{0.000000,0.000000,0.000000}%
\pgfsetfillcolor{currentfill}%
\pgfsetlinewidth{0.803000pt}%
\definecolor{currentstroke}{rgb}{0.000000,0.000000,0.000000}%
\pgfsetstrokecolor{currentstroke}%
\pgfsetdash{}{0pt}%
\pgfsys@defobject{currentmarker}{\pgfqpoint{0.000000in}{-0.048611in}}{\pgfqpoint{0.000000in}{0.000000in}}{%
\pgfpathmoveto{\pgfqpoint{0.000000in}{0.000000in}}%
\pgfpathlineto{\pgfqpoint{0.000000in}{-0.048611in}}%
\pgfusepath{stroke,fill}%
}%
\begin{pgfscope}%
\pgfsys@transformshift{4.333661in}{0.500000in}%
\pgfsys@useobject{currentmarker}{}%
\end{pgfscope}%
\end{pgfscope}%
\begin{pgfscope}%
\definecolor{textcolor}{rgb}{0.000000,0.000000,0.000000}%
\pgfsetstrokecolor{textcolor}%
\pgfsetfillcolor{textcolor}%
\pgftext[x=4.333661in,y=0.402778in,,top]{\color{textcolor}\rmfamily\fontsize{13.000000}{15.600000}\selectfont \(\displaystyle {0.4}\)}%
\end{pgfscope}%
\begin{pgfscope}%
\pgfsetbuttcap%
\pgfsetroundjoin%
\definecolor{currentfill}{rgb}{0.000000,0.000000,0.000000}%
\pgfsetfillcolor{currentfill}%
\pgfsetlinewidth{0.803000pt}%
\definecolor{currentstroke}{rgb}{0.000000,0.000000,0.000000}%
\pgfsetstrokecolor{currentstroke}%
\pgfsetdash{}{0pt}%
\pgfsys@defobject{currentmarker}{\pgfqpoint{0.000000in}{-0.048611in}}{\pgfqpoint{0.000000in}{0.000000in}}{%
\pgfpathmoveto{\pgfqpoint{0.000000in}{0.000000in}}%
\pgfpathlineto{\pgfqpoint{0.000000in}{-0.048611in}}%
\pgfusepath{stroke,fill}%
}%
\begin{pgfscope}%
\pgfsys@transformshift{5.180754in}{0.500000in}%
\pgfsys@useobject{currentmarker}{}%
\end{pgfscope}%
\end{pgfscope}%
\begin{pgfscope}%
\definecolor{textcolor}{rgb}{0.000000,0.000000,0.000000}%
\pgfsetstrokecolor{textcolor}%
\pgfsetfillcolor{textcolor}%
\pgftext[x=5.180754in,y=0.402778in,,top]{\color{textcolor}\rmfamily\fontsize{13.000000}{15.600000}\selectfont \(\displaystyle {0.5}\)}%
\end{pgfscope}%
\begin{pgfscope}%
\definecolor{textcolor}{rgb}{0.000000,0.000000,0.000000}%
\pgfsetstrokecolor{textcolor}%
\pgfsetfillcolor{textcolor}%
\pgftext[x=3.075000in,y=0.199075in,,top]{\color{textcolor}\rmfamily\fontsize{13.000000}{15.600000}\selectfont Loss}%
\end{pgfscope}%
\begin{pgfscope}%
\pgfsetbuttcap%
\pgfsetroundjoin%
\definecolor{currentfill}{rgb}{0.000000,0.000000,0.000000}%
\pgfsetfillcolor{currentfill}%
\pgfsetlinewidth{0.803000pt}%
\definecolor{currentstroke}{rgb}{0.000000,0.000000,0.000000}%
\pgfsetstrokecolor{currentstroke}%
\pgfsetdash{}{0pt}%
\pgfsys@defobject{currentmarker}{\pgfqpoint{-0.048611in}{0.000000in}}{\pgfqpoint{-0.000000in}{0.000000in}}{%
\pgfpathmoveto{\pgfqpoint{-0.000000in}{0.000000in}}%
\pgfpathlineto{\pgfqpoint{-0.048611in}{0.000000in}}%
\pgfusepath{stroke,fill}%
}%
\begin{pgfscope}%
\pgfsys@transformshift{0.750000in}{0.500000in}%
\pgfsys@useobject{currentmarker}{}%
\end{pgfscope}%
\end{pgfscope}%
\begin{pgfscope}%
\definecolor{textcolor}{rgb}{0.000000,0.000000,0.000000}%
\pgfsetstrokecolor{textcolor}%
\pgfsetfillcolor{textcolor}%
\pgftext[x=0.571181in, y=0.442130in, left, base]{\color{textcolor}\rmfamily\fontsize{13.000000}{15.600000}\selectfont \(\displaystyle {0}\)}%
\end{pgfscope}%
\begin{pgfscope}%
\pgfsetbuttcap%
\pgfsetroundjoin%
\definecolor{currentfill}{rgb}{0.000000,0.000000,0.000000}%
\pgfsetfillcolor{currentfill}%
\pgfsetlinewidth{0.803000pt}%
\definecolor{currentstroke}{rgb}{0.000000,0.000000,0.000000}%
\pgfsetstrokecolor{currentstroke}%
\pgfsetdash{}{0pt}%
\pgfsys@defobject{currentmarker}{\pgfqpoint{-0.048611in}{0.000000in}}{\pgfqpoint{-0.000000in}{0.000000in}}{%
\pgfpathmoveto{\pgfqpoint{-0.000000in}{0.000000in}}%
\pgfpathlineto{\pgfqpoint{-0.048611in}{0.000000in}}%
\pgfusepath{stroke,fill}%
}%
\begin{pgfscope}%
\pgfsys@transformshift{0.750000in}{1.104242in}%
\pgfsys@useobject{currentmarker}{}%
\end{pgfscope}%
\end{pgfscope}%
\begin{pgfscope}%
\definecolor{textcolor}{rgb}{0.000000,0.000000,0.000000}%
\pgfsetstrokecolor{textcolor}%
\pgfsetfillcolor{textcolor}%
\pgftext[x=0.407989in, y=1.046371in, left, base]{\color{textcolor}\rmfamily\fontsize{13.000000}{15.600000}\selectfont \(\displaystyle {200}\)}%
\end{pgfscope}%
\begin{pgfscope}%
\pgfsetbuttcap%
\pgfsetroundjoin%
\definecolor{currentfill}{rgb}{0.000000,0.000000,0.000000}%
\pgfsetfillcolor{currentfill}%
\pgfsetlinewidth{0.803000pt}%
\definecolor{currentstroke}{rgb}{0.000000,0.000000,0.000000}%
\pgfsetstrokecolor{currentstroke}%
\pgfsetdash{}{0pt}%
\pgfsys@defobject{currentmarker}{\pgfqpoint{-0.048611in}{0.000000in}}{\pgfqpoint{-0.000000in}{0.000000in}}{%
\pgfpathmoveto{\pgfqpoint{-0.000000in}{0.000000in}}%
\pgfpathlineto{\pgfqpoint{-0.048611in}{0.000000in}}%
\pgfusepath{stroke,fill}%
}%
\begin{pgfscope}%
\pgfsys@transformshift{0.750000in}{1.708483in}%
\pgfsys@useobject{currentmarker}{}%
\end{pgfscope}%
\end{pgfscope}%
\begin{pgfscope}%
\definecolor{textcolor}{rgb}{0.000000,0.000000,0.000000}%
\pgfsetstrokecolor{textcolor}%
\pgfsetfillcolor{textcolor}%
\pgftext[x=0.407989in, y=1.650613in, left, base]{\color{textcolor}\rmfamily\fontsize{13.000000}{15.600000}\selectfont \(\displaystyle {400}\)}%
\end{pgfscope}%
\begin{pgfscope}%
\pgfsetbuttcap%
\pgfsetroundjoin%
\definecolor{currentfill}{rgb}{0.000000,0.000000,0.000000}%
\pgfsetfillcolor{currentfill}%
\pgfsetlinewidth{0.803000pt}%
\definecolor{currentstroke}{rgb}{0.000000,0.000000,0.000000}%
\pgfsetstrokecolor{currentstroke}%
\pgfsetdash{}{0pt}%
\pgfsys@defobject{currentmarker}{\pgfqpoint{-0.048611in}{0.000000in}}{\pgfqpoint{-0.000000in}{0.000000in}}{%
\pgfpathmoveto{\pgfqpoint{-0.000000in}{0.000000in}}%
\pgfpathlineto{\pgfqpoint{-0.048611in}{0.000000in}}%
\pgfusepath{stroke,fill}%
}%
\begin{pgfscope}%
\pgfsys@transformshift{0.750000in}{2.312725in}%
\pgfsys@useobject{currentmarker}{}%
\end{pgfscope}%
\end{pgfscope}%
\begin{pgfscope}%
\definecolor{textcolor}{rgb}{0.000000,0.000000,0.000000}%
\pgfsetstrokecolor{textcolor}%
\pgfsetfillcolor{textcolor}%
\pgftext[x=0.407989in, y=2.254855in, left, base]{\color{textcolor}\rmfamily\fontsize{13.000000}{15.600000}\selectfont \(\displaystyle {600}\)}%
\end{pgfscope}%
\begin{pgfscope}%
\pgfsetbuttcap%
\pgfsetroundjoin%
\definecolor{currentfill}{rgb}{0.000000,0.000000,0.000000}%
\pgfsetfillcolor{currentfill}%
\pgfsetlinewidth{0.803000pt}%
\definecolor{currentstroke}{rgb}{0.000000,0.000000,0.000000}%
\pgfsetstrokecolor{currentstroke}%
\pgfsetdash{}{0pt}%
\pgfsys@defobject{currentmarker}{\pgfqpoint{-0.048611in}{0.000000in}}{\pgfqpoint{-0.000000in}{0.000000in}}{%
\pgfpathmoveto{\pgfqpoint{-0.000000in}{0.000000in}}%
\pgfpathlineto{\pgfqpoint{-0.048611in}{0.000000in}}%
\pgfusepath{stroke,fill}%
}%
\begin{pgfscope}%
\pgfsys@transformshift{0.750000in}{2.916967in}%
\pgfsys@useobject{currentmarker}{}%
\end{pgfscope}%
\end{pgfscope}%
\begin{pgfscope}%
\definecolor{textcolor}{rgb}{0.000000,0.000000,0.000000}%
\pgfsetstrokecolor{textcolor}%
\pgfsetfillcolor{textcolor}%
\pgftext[x=0.407989in, y=2.859097in, left, base]{\color{textcolor}\rmfamily\fontsize{13.000000}{15.600000}\selectfont \(\displaystyle {800}\)}%
\end{pgfscope}%
\begin{pgfscope}%
\definecolor{textcolor}{rgb}{0.000000,0.000000,0.000000}%
\pgfsetstrokecolor{textcolor}%
\pgfsetfillcolor{textcolor}%
\pgftext[x=0.352433in,y=2.010000in,,bottom,rotate=90.000000]{\color{textcolor}\rmfamily\fontsize{13.000000}{15.600000}\selectfont Count}%
\end{pgfscope}%
\begin{pgfscope}%
\pgfsetrectcap%
\pgfsetmiterjoin%
\pgfsetlinewidth{0.803000pt}%
\definecolor{currentstroke}{rgb}{0.000000,0.000000,0.000000}%
\pgfsetstrokecolor{currentstroke}%
\pgfsetdash{}{0pt}%
\pgfpathmoveto{\pgfqpoint{0.750000in}{0.500000in}}%
\pgfpathlineto{\pgfqpoint{0.750000in}{3.520000in}}%
\pgfusepath{stroke}%
\end{pgfscope}%
\begin{pgfscope}%
\pgfsetrectcap%
\pgfsetmiterjoin%
\pgfsetlinewidth{0.803000pt}%
\definecolor{currentstroke}{rgb}{0.000000,0.000000,0.000000}%
\pgfsetstrokecolor{currentstroke}%
\pgfsetdash{}{0pt}%
\pgfpathmoveto{\pgfqpoint{5.400000in}{0.500000in}}%
\pgfpathlineto{\pgfqpoint{5.400000in}{3.520000in}}%
\pgfusepath{stroke}%
\end{pgfscope}%
\begin{pgfscope}%
\pgfsetrectcap%
\pgfsetmiterjoin%
\pgfsetlinewidth{0.803000pt}%
\definecolor{currentstroke}{rgb}{0.000000,0.000000,0.000000}%
\pgfsetstrokecolor{currentstroke}%
\pgfsetdash{}{0pt}%
\pgfpathmoveto{\pgfqpoint{0.750000in}{0.500000in}}%
\pgfpathlineto{\pgfqpoint{5.400000in}{0.500000in}}%
\pgfusepath{stroke}%
\end{pgfscope}%
\begin{pgfscope}%
\pgfsetrectcap%
\pgfsetmiterjoin%
\pgfsetlinewidth{0.803000pt}%
\definecolor{currentstroke}{rgb}{0.000000,0.000000,0.000000}%
\pgfsetstrokecolor{currentstroke}%
\pgfsetdash{}{0pt}%
\pgfpathmoveto{\pgfqpoint{0.750000in}{3.520000in}}%
\pgfpathlineto{\pgfqpoint{5.400000in}{3.520000in}}%
\pgfusepath{stroke}%
\end{pgfscope}%
\begin{pgfscope}%
\definecolor{textcolor}{rgb}{0.000000,0.000000,0.000000}%
\pgfsetstrokecolor{textcolor}%
\pgfsetfillcolor{textcolor}%
\pgftext[x=3.075000in,y=3.603333in,,base]{\color{textcolor}\rmfamily\fontsize{13.000000}{15.600000}\selectfont Loss Histogram for \(\displaystyle f(x)=2x\)}%
\end{pgfscope}%
\begin{pgfscope}%
\pgfsetbuttcap%
\pgfsetmiterjoin%
\definecolor{currentfill}{rgb}{1.000000,1.000000,1.000000}%
\pgfsetfillcolor{currentfill}%
\pgfsetfillopacity{0.800000}%
\pgfsetlinewidth{1.003750pt}%
\definecolor{currentstroke}{rgb}{0.800000,0.800000,0.800000}%
\pgfsetstrokecolor{currentstroke}%
\pgfsetstrokeopacity{0.800000}%
\pgfsetdash{}{0pt}%
\pgfpathmoveto{\pgfqpoint{4.360497in}{2.877408in}}%
\pgfpathlineto{\pgfqpoint{5.273611in}{2.877408in}}%
\pgfpathquadraticcurveto{\pgfqpoint{5.309722in}{2.877408in}}{\pgfqpoint{5.309722in}{2.913519in}}%
\pgfpathlineto{\pgfqpoint{5.309722in}{3.393611in}}%
\pgfpathquadraticcurveto{\pgfqpoint{5.309722in}{3.429722in}}{\pgfqpoint{5.273611in}{3.429722in}}%
\pgfpathlineto{\pgfqpoint{4.360497in}{3.429722in}}%
\pgfpathquadraticcurveto{\pgfqpoint{4.324386in}{3.429722in}}{\pgfqpoint{4.324386in}{3.393611in}}%
\pgfpathlineto{\pgfqpoint{4.324386in}{2.913519in}}%
\pgfpathquadraticcurveto{\pgfqpoint{4.324386in}{2.877408in}}{\pgfqpoint{4.360497in}{2.877408in}}%
\pgfpathlineto{\pgfqpoint{4.360497in}{2.877408in}}%
\pgfpathclose%
\pgfusepath{stroke,fill}%
\end{pgfscope}%
\begin{pgfscope}%
\pgfsetbuttcap%
\pgfsetmiterjoin%
\definecolor{currentfill}{rgb}{1.000000,0.000000,0.000000}%
\pgfsetfillcolor{currentfill}%
\pgfsetlinewidth{0.000000pt}%
\definecolor{currentstroke}{rgb}{0.000000,0.000000,0.000000}%
\pgfsetstrokecolor{currentstroke}%
\pgfsetstrokeopacity{0.000000}%
\pgfsetdash{}{0pt}%
\pgfpathmoveto{\pgfqpoint{4.396608in}{3.231111in}}%
\pgfpathlineto{\pgfqpoint{4.757719in}{3.231111in}}%
\pgfpathlineto{\pgfqpoint{4.757719in}{3.357500in}}%
\pgfpathlineto{\pgfqpoint{4.396608in}{3.357500in}}%
\pgfpathlineto{\pgfqpoint{4.396608in}{3.231111in}}%
\pgfpathclose%
\pgfusepath{fill}%
\end{pgfscope}%
\begin{pgfscope}%
\definecolor{textcolor}{rgb}{0.000000,0.000000,0.000000}%
\pgfsetstrokecolor{textcolor}%
\pgfsetfillcolor{textcolor}%
\pgftext[x=4.902164in,y=3.231111in,left,base]{\color{textcolor}\rmfamily\fontsize{13.000000}{15.600000}\selectfont SNN}%
\end{pgfscope}%
\begin{pgfscope}%
\pgfsetbuttcap%
\pgfsetmiterjoin%
\definecolor{currentfill}{rgb}{0.000000,0.500000,0.000000}%
\pgfsetfillcolor{currentfill}%
\pgfsetlinewidth{0.000000pt}%
\definecolor{currentstroke}{rgb}{0.000000,0.000000,0.000000}%
\pgfsetstrokecolor{currentstroke}%
\pgfsetstrokeopacity{0.000000}%
\pgfsetdash{}{0pt}%
\pgfpathmoveto{\pgfqpoint{4.396608in}{2.982037in}}%
\pgfpathlineto{\pgfqpoint{4.757719in}{2.982037in}}%
\pgfpathlineto{\pgfqpoint{4.757719in}{3.108426in}}%
\pgfpathlineto{\pgfqpoint{4.396608in}{3.108426in}}%
\pgfpathlineto{\pgfqpoint{4.396608in}{2.982037in}}%
\pgfpathclose%
\pgfusepath{fill}%
\end{pgfscope}%
\begin{pgfscope}%
\definecolor{textcolor}{rgb}{0.000000,0.000000,0.000000}%
\pgfsetstrokecolor{textcolor}%
\pgfsetfillcolor{textcolor}%
\pgftext[x=4.902164in,y=2.982037in,left,base]{\color{textcolor}\rmfamily\fontsize{13.000000}{15.600000}\selectfont NN}%
\end{pgfscope}%
\end{pgfpicture}%
\makeatother%
\endgroup%

    \caption{Caption}
    \label{fig:my_label}
\end{figure}

\begin{figure}
%% Creator: Matplotlib, PGF backend
%%
%% To include the figure in your LaTeX document, write
%%   \input{<filename>.pgf}
%%
%% Make sure the required packages are loaded in your preamble
%%   \usepackage{pgf}
%%
%% Also ensure that all the required font packages are loaded; for instance,
%% the lmodern package is sometimes necessary when using math font.
%%   \usepackage{lmodern}
%%
%% Figures using additional raster images can only be included by \input if
%% they are in the same directory as the main LaTeX file. For loading figures
%% from other directories you can use the `import` package
%%   \usepackage{import}
%%
%% and then include the figures with
%%   \import{<path to file>}{<filename>.pgf}
%%
%% Matplotlib used the following preamble
%%
\begingroup%
\makeatletter%
\begin{pgfpicture}%
\pgfpathrectangle{\pgfpointorigin}{\pgfqpoint{6.000000in}{4.000000in}}%
\pgfusepath{use as bounding box, clip}%
\begin{pgfscope}%
\pgfsetbuttcap%
\pgfsetmiterjoin%
\pgfsetlinewidth{0.000000pt}%
\definecolor{currentstroke}{rgb}{1.000000,1.000000,1.000000}%
\pgfsetstrokecolor{currentstroke}%
\pgfsetstrokeopacity{0.000000}%
\pgfsetdash{}{0pt}%
\pgfpathmoveto{\pgfqpoint{0.000000in}{0.000000in}}%
\pgfpathlineto{\pgfqpoint{6.000000in}{0.000000in}}%
\pgfpathlineto{\pgfqpoint{6.000000in}{4.000000in}}%
\pgfpathlineto{\pgfqpoint{0.000000in}{4.000000in}}%
\pgfpathlineto{\pgfqpoint{0.000000in}{0.000000in}}%
\pgfpathclose%
\pgfusepath{}%
\end{pgfscope}%
\begin{pgfscope}%
\pgfsetbuttcap%
\pgfsetmiterjoin%
\definecolor{currentfill}{rgb}{1.000000,1.000000,1.000000}%
\pgfsetfillcolor{currentfill}%
\pgfsetlinewidth{0.000000pt}%
\definecolor{currentstroke}{rgb}{0.000000,0.000000,0.000000}%
\pgfsetstrokecolor{currentstroke}%
\pgfsetstrokeopacity{0.000000}%
\pgfsetdash{}{0pt}%
\pgfpathmoveto{\pgfqpoint{0.750000in}{0.500000in}}%
\pgfpathlineto{\pgfqpoint{5.400000in}{0.500000in}}%
\pgfpathlineto{\pgfqpoint{5.400000in}{3.520000in}}%
\pgfpathlineto{\pgfqpoint{0.750000in}{3.520000in}}%
\pgfpathlineto{\pgfqpoint{0.750000in}{0.500000in}}%
\pgfpathclose%
\pgfusepath{fill}%
\end{pgfscope}%
\begin{pgfscope}%
\pgfpathrectangle{\pgfqpoint{0.750000in}{0.500000in}}{\pgfqpoint{4.650000in}{3.020000in}}%
\pgfusepath{clip}%
\pgfsetbuttcap%
\pgfsetmiterjoin%
\definecolor{currentfill}{rgb}{0.121569,0.466667,0.705882}%
\pgfsetfillcolor{currentfill}%
\pgfsetlinewidth{0.000000pt}%
\definecolor{currentstroke}{rgb}{0.000000,0.000000,0.000000}%
\pgfsetstrokecolor{currentstroke}%
\pgfsetstrokeopacity{0.000000}%
\pgfsetdash{}{0pt}%
\pgfpathmoveto{\pgfqpoint{0.961364in}{0.500000in}}%
\pgfpathlineto{\pgfqpoint{0.987702in}{0.500000in}}%
\pgfpathlineto{\pgfqpoint{0.987702in}{0.500000in}}%
\pgfpathlineto{\pgfqpoint{0.961364in}{0.500000in}}%
\pgfpathlineto{\pgfqpoint{0.961364in}{0.500000in}}%
\pgfpathclose%
\pgfusepath{fill}%
\end{pgfscope}%
\begin{pgfscope}%
\pgfpathrectangle{\pgfqpoint{0.750000in}{0.500000in}}{\pgfqpoint{4.650000in}{3.020000in}}%
\pgfusepath{clip}%
\pgfsetbuttcap%
\pgfsetmiterjoin%
\definecolor{currentfill}{rgb}{0.121569,0.466667,0.705882}%
\pgfsetfillcolor{currentfill}%
\pgfsetlinewidth{0.000000pt}%
\definecolor{currentstroke}{rgb}{0.000000,0.000000,0.000000}%
\pgfsetstrokecolor{currentstroke}%
\pgfsetstrokeopacity{0.000000}%
\pgfsetdash{}{0pt}%
\pgfpathmoveto{\pgfqpoint{0.994286in}{0.500000in}}%
\pgfpathlineto{\pgfqpoint{1.020624in}{0.500000in}}%
\pgfpathlineto{\pgfqpoint{1.020624in}{0.500000in}}%
\pgfpathlineto{\pgfqpoint{0.994286in}{0.500000in}}%
\pgfpathlineto{\pgfqpoint{0.994286in}{0.500000in}}%
\pgfpathclose%
\pgfusepath{fill}%
\end{pgfscope}%
\begin{pgfscope}%
\pgfpathrectangle{\pgfqpoint{0.750000in}{0.500000in}}{\pgfqpoint{4.650000in}{3.020000in}}%
\pgfusepath{clip}%
\pgfsetbuttcap%
\pgfsetmiterjoin%
\definecolor{currentfill}{rgb}{0.121569,0.466667,0.705882}%
\pgfsetfillcolor{currentfill}%
\pgfsetlinewidth{0.000000pt}%
\definecolor{currentstroke}{rgb}{0.000000,0.000000,0.000000}%
\pgfsetstrokecolor{currentstroke}%
\pgfsetstrokeopacity{0.000000}%
\pgfsetdash{}{0pt}%
\pgfpathmoveto{\pgfqpoint{1.027209in}{0.500000in}}%
\pgfpathlineto{\pgfqpoint{1.053547in}{0.500000in}}%
\pgfpathlineto{\pgfqpoint{1.053547in}{0.500000in}}%
\pgfpathlineto{\pgfqpoint{1.027209in}{0.500000in}}%
\pgfpathlineto{\pgfqpoint{1.027209in}{0.500000in}}%
\pgfpathclose%
\pgfusepath{fill}%
\end{pgfscope}%
\begin{pgfscope}%
\pgfpathrectangle{\pgfqpoint{0.750000in}{0.500000in}}{\pgfqpoint{4.650000in}{3.020000in}}%
\pgfusepath{clip}%
\pgfsetbuttcap%
\pgfsetmiterjoin%
\definecolor{currentfill}{rgb}{0.121569,0.466667,0.705882}%
\pgfsetfillcolor{currentfill}%
\pgfsetlinewidth{0.000000pt}%
\definecolor{currentstroke}{rgb}{0.000000,0.000000,0.000000}%
\pgfsetstrokecolor{currentstroke}%
\pgfsetstrokeopacity{0.000000}%
\pgfsetdash{}{0pt}%
\pgfpathmoveto{\pgfqpoint{1.060132in}{0.500000in}}%
\pgfpathlineto{\pgfqpoint{1.086470in}{0.500000in}}%
\pgfpathlineto{\pgfqpoint{1.086470in}{0.500000in}}%
\pgfpathlineto{\pgfqpoint{1.060132in}{0.500000in}}%
\pgfpathlineto{\pgfqpoint{1.060132in}{0.500000in}}%
\pgfpathclose%
\pgfusepath{fill}%
\end{pgfscope}%
\begin{pgfscope}%
\pgfpathrectangle{\pgfqpoint{0.750000in}{0.500000in}}{\pgfqpoint{4.650000in}{3.020000in}}%
\pgfusepath{clip}%
\pgfsetbuttcap%
\pgfsetmiterjoin%
\definecolor{currentfill}{rgb}{0.121569,0.466667,0.705882}%
\pgfsetfillcolor{currentfill}%
\pgfsetlinewidth{0.000000pt}%
\definecolor{currentstroke}{rgb}{0.000000,0.000000,0.000000}%
\pgfsetstrokecolor{currentstroke}%
\pgfsetstrokeopacity{0.000000}%
\pgfsetdash{}{0pt}%
\pgfpathmoveto{\pgfqpoint{1.093054in}{0.500000in}}%
\pgfpathlineto{\pgfqpoint{1.119393in}{0.500000in}}%
\pgfpathlineto{\pgfqpoint{1.119393in}{0.500000in}}%
\pgfpathlineto{\pgfqpoint{1.093054in}{0.500000in}}%
\pgfpathlineto{\pgfqpoint{1.093054in}{0.500000in}}%
\pgfpathclose%
\pgfusepath{fill}%
\end{pgfscope}%
\begin{pgfscope}%
\pgfpathrectangle{\pgfqpoint{0.750000in}{0.500000in}}{\pgfqpoint{4.650000in}{3.020000in}}%
\pgfusepath{clip}%
\pgfsetbuttcap%
\pgfsetmiterjoin%
\definecolor{currentfill}{rgb}{0.121569,0.466667,0.705882}%
\pgfsetfillcolor{currentfill}%
\pgfsetlinewidth{0.000000pt}%
\definecolor{currentstroke}{rgb}{0.000000,0.000000,0.000000}%
\pgfsetstrokecolor{currentstroke}%
\pgfsetstrokeopacity{0.000000}%
\pgfsetdash{}{0pt}%
\pgfpathmoveto{\pgfqpoint{1.125977in}{0.500000in}}%
\pgfpathlineto{\pgfqpoint{1.152315in}{0.500000in}}%
\pgfpathlineto{\pgfqpoint{1.152315in}{0.500000in}}%
\pgfpathlineto{\pgfqpoint{1.125977in}{0.500000in}}%
\pgfpathlineto{\pgfqpoint{1.125977in}{0.500000in}}%
\pgfpathclose%
\pgfusepath{fill}%
\end{pgfscope}%
\begin{pgfscope}%
\pgfpathrectangle{\pgfqpoint{0.750000in}{0.500000in}}{\pgfqpoint{4.650000in}{3.020000in}}%
\pgfusepath{clip}%
\pgfsetbuttcap%
\pgfsetmiterjoin%
\definecolor{currentfill}{rgb}{0.121569,0.466667,0.705882}%
\pgfsetfillcolor{currentfill}%
\pgfsetlinewidth{0.000000pt}%
\definecolor{currentstroke}{rgb}{0.000000,0.000000,0.000000}%
\pgfsetstrokecolor{currentstroke}%
\pgfsetstrokeopacity{0.000000}%
\pgfsetdash{}{0pt}%
\pgfpathmoveto{\pgfqpoint{1.158900in}{0.500000in}}%
\pgfpathlineto{\pgfqpoint{1.185238in}{0.500000in}}%
\pgfpathlineto{\pgfqpoint{1.185238in}{0.500000in}}%
\pgfpathlineto{\pgfqpoint{1.158900in}{0.500000in}}%
\pgfpathlineto{\pgfqpoint{1.158900in}{0.500000in}}%
\pgfpathclose%
\pgfusepath{fill}%
\end{pgfscope}%
\begin{pgfscope}%
\pgfpathrectangle{\pgfqpoint{0.750000in}{0.500000in}}{\pgfqpoint{4.650000in}{3.020000in}}%
\pgfusepath{clip}%
\pgfsetbuttcap%
\pgfsetmiterjoin%
\definecolor{currentfill}{rgb}{0.121569,0.466667,0.705882}%
\pgfsetfillcolor{currentfill}%
\pgfsetlinewidth{0.000000pt}%
\definecolor{currentstroke}{rgb}{0.000000,0.000000,0.000000}%
\pgfsetstrokecolor{currentstroke}%
\pgfsetstrokeopacity{0.000000}%
\pgfsetdash{}{0pt}%
\pgfpathmoveto{\pgfqpoint{1.191822in}{0.500000in}}%
\pgfpathlineto{\pgfqpoint{1.218161in}{0.500000in}}%
\pgfpathlineto{\pgfqpoint{1.218161in}{0.500000in}}%
\pgfpathlineto{\pgfqpoint{1.191822in}{0.500000in}}%
\pgfpathlineto{\pgfqpoint{1.191822in}{0.500000in}}%
\pgfpathclose%
\pgfusepath{fill}%
\end{pgfscope}%
\begin{pgfscope}%
\pgfpathrectangle{\pgfqpoint{0.750000in}{0.500000in}}{\pgfqpoint{4.650000in}{3.020000in}}%
\pgfusepath{clip}%
\pgfsetbuttcap%
\pgfsetmiterjoin%
\definecolor{currentfill}{rgb}{0.121569,0.466667,0.705882}%
\pgfsetfillcolor{currentfill}%
\pgfsetlinewidth{0.000000pt}%
\definecolor{currentstroke}{rgb}{0.000000,0.000000,0.000000}%
\pgfsetstrokecolor{currentstroke}%
\pgfsetstrokeopacity{0.000000}%
\pgfsetdash{}{0pt}%
\pgfpathmoveto{\pgfqpoint{1.224745in}{0.500000in}}%
\pgfpathlineto{\pgfqpoint{1.251083in}{0.500000in}}%
\pgfpathlineto{\pgfqpoint{1.251083in}{0.500000in}}%
\pgfpathlineto{\pgfqpoint{1.224745in}{0.500000in}}%
\pgfpathlineto{\pgfqpoint{1.224745in}{0.500000in}}%
\pgfpathclose%
\pgfusepath{fill}%
\end{pgfscope}%
\begin{pgfscope}%
\pgfpathrectangle{\pgfqpoint{0.750000in}{0.500000in}}{\pgfqpoint{4.650000in}{3.020000in}}%
\pgfusepath{clip}%
\pgfsetbuttcap%
\pgfsetmiterjoin%
\definecolor{currentfill}{rgb}{0.121569,0.466667,0.705882}%
\pgfsetfillcolor{currentfill}%
\pgfsetlinewidth{0.000000pt}%
\definecolor{currentstroke}{rgb}{0.000000,0.000000,0.000000}%
\pgfsetstrokecolor{currentstroke}%
\pgfsetstrokeopacity{0.000000}%
\pgfsetdash{}{0pt}%
\pgfpathmoveto{\pgfqpoint{1.257668in}{0.500000in}}%
\pgfpathlineto{\pgfqpoint{1.284006in}{0.500000in}}%
\pgfpathlineto{\pgfqpoint{1.284006in}{0.500000in}}%
\pgfpathlineto{\pgfqpoint{1.257668in}{0.500000in}}%
\pgfpathlineto{\pgfqpoint{1.257668in}{0.500000in}}%
\pgfpathclose%
\pgfusepath{fill}%
\end{pgfscope}%
\begin{pgfscope}%
\pgfpathrectangle{\pgfqpoint{0.750000in}{0.500000in}}{\pgfqpoint{4.650000in}{3.020000in}}%
\pgfusepath{clip}%
\pgfsetbuttcap%
\pgfsetmiterjoin%
\definecolor{currentfill}{rgb}{0.121569,0.466667,0.705882}%
\pgfsetfillcolor{currentfill}%
\pgfsetlinewidth{0.000000pt}%
\definecolor{currentstroke}{rgb}{0.000000,0.000000,0.000000}%
\pgfsetstrokecolor{currentstroke}%
\pgfsetstrokeopacity{0.000000}%
\pgfsetdash{}{0pt}%
\pgfpathmoveto{\pgfqpoint{1.290590in}{0.500000in}}%
\pgfpathlineto{\pgfqpoint{1.316929in}{0.500000in}}%
\pgfpathlineto{\pgfqpoint{1.316929in}{0.500000in}}%
\pgfpathlineto{\pgfqpoint{1.290590in}{0.500000in}}%
\pgfpathlineto{\pgfqpoint{1.290590in}{0.500000in}}%
\pgfpathclose%
\pgfusepath{fill}%
\end{pgfscope}%
\begin{pgfscope}%
\pgfpathrectangle{\pgfqpoint{0.750000in}{0.500000in}}{\pgfqpoint{4.650000in}{3.020000in}}%
\pgfusepath{clip}%
\pgfsetbuttcap%
\pgfsetmiterjoin%
\definecolor{currentfill}{rgb}{0.121569,0.466667,0.705882}%
\pgfsetfillcolor{currentfill}%
\pgfsetlinewidth{0.000000pt}%
\definecolor{currentstroke}{rgb}{0.000000,0.000000,0.000000}%
\pgfsetstrokecolor{currentstroke}%
\pgfsetstrokeopacity{0.000000}%
\pgfsetdash{}{0pt}%
\pgfpathmoveto{\pgfqpoint{1.323513in}{0.500000in}}%
\pgfpathlineto{\pgfqpoint{1.349851in}{0.500000in}}%
\pgfpathlineto{\pgfqpoint{1.349851in}{0.500000in}}%
\pgfpathlineto{\pgfqpoint{1.323513in}{0.500000in}}%
\pgfpathlineto{\pgfqpoint{1.323513in}{0.500000in}}%
\pgfpathclose%
\pgfusepath{fill}%
\end{pgfscope}%
\begin{pgfscope}%
\pgfpathrectangle{\pgfqpoint{0.750000in}{0.500000in}}{\pgfqpoint{4.650000in}{3.020000in}}%
\pgfusepath{clip}%
\pgfsetbuttcap%
\pgfsetmiterjoin%
\definecolor{currentfill}{rgb}{0.121569,0.466667,0.705882}%
\pgfsetfillcolor{currentfill}%
\pgfsetlinewidth{0.000000pt}%
\definecolor{currentstroke}{rgb}{0.000000,0.000000,0.000000}%
\pgfsetstrokecolor{currentstroke}%
\pgfsetstrokeopacity{0.000000}%
\pgfsetdash{}{0pt}%
\pgfpathmoveto{\pgfqpoint{1.356436in}{0.500000in}}%
\pgfpathlineto{\pgfqpoint{1.382774in}{0.500000in}}%
\pgfpathlineto{\pgfqpoint{1.382774in}{0.500000in}}%
\pgfpathlineto{\pgfqpoint{1.356436in}{0.500000in}}%
\pgfpathlineto{\pgfqpoint{1.356436in}{0.500000in}}%
\pgfpathclose%
\pgfusepath{fill}%
\end{pgfscope}%
\begin{pgfscope}%
\pgfpathrectangle{\pgfqpoint{0.750000in}{0.500000in}}{\pgfqpoint{4.650000in}{3.020000in}}%
\pgfusepath{clip}%
\pgfsetbuttcap%
\pgfsetmiterjoin%
\definecolor{currentfill}{rgb}{0.121569,0.466667,0.705882}%
\pgfsetfillcolor{currentfill}%
\pgfsetlinewidth{0.000000pt}%
\definecolor{currentstroke}{rgb}{0.000000,0.000000,0.000000}%
\pgfsetstrokecolor{currentstroke}%
\pgfsetstrokeopacity{0.000000}%
\pgfsetdash{}{0pt}%
\pgfpathmoveto{\pgfqpoint{1.389359in}{0.500000in}}%
\pgfpathlineto{\pgfqpoint{1.415697in}{0.500000in}}%
\pgfpathlineto{\pgfqpoint{1.415697in}{0.500000in}}%
\pgfpathlineto{\pgfqpoint{1.389359in}{0.500000in}}%
\pgfpathlineto{\pgfqpoint{1.389359in}{0.500000in}}%
\pgfpathclose%
\pgfusepath{fill}%
\end{pgfscope}%
\begin{pgfscope}%
\pgfpathrectangle{\pgfqpoint{0.750000in}{0.500000in}}{\pgfqpoint{4.650000in}{3.020000in}}%
\pgfusepath{clip}%
\pgfsetbuttcap%
\pgfsetmiterjoin%
\definecolor{currentfill}{rgb}{0.121569,0.466667,0.705882}%
\pgfsetfillcolor{currentfill}%
\pgfsetlinewidth{0.000000pt}%
\definecolor{currentstroke}{rgb}{0.000000,0.000000,0.000000}%
\pgfsetstrokecolor{currentstroke}%
\pgfsetstrokeopacity{0.000000}%
\pgfsetdash{}{0pt}%
\pgfpathmoveto{\pgfqpoint{1.422281in}{0.500000in}}%
\pgfpathlineto{\pgfqpoint{1.448619in}{0.500000in}}%
\pgfpathlineto{\pgfqpoint{1.448619in}{0.500000in}}%
\pgfpathlineto{\pgfqpoint{1.422281in}{0.500000in}}%
\pgfpathlineto{\pgfqpoint{1.422281in}{0.500000in}}%
\pgfpathclose%
\pgfusepath{fill}%
\end{pgfscope}%
\begin{pgfscope}%
\pgfpathrectangle{\pgfqpoint{0.750000in}{0.500000in}}{\pgfqpoint{4.650000in}{3.020000in}}%
\pgfusepath{clip}%
\pgfsetbuttcap%
\pgfsetmiterjoin%
\definecolor{currentfill}{rgb}{0.121569,0.466667,0.705882}%
\pgfsetfillcolor{currentfill}%
\pgfsetlinewidth{0.000000pt}%
\definecolor{currentstroke}{rgb}{0.000000,0.000000,0.000000}%
\pgfsetstrokecolor{currentstroke}%
\pgfsetstrokeopacity{0.000000}%
\pgfsetdash{}{0pt}%
\pgfpathmoveto{\pgfqpoint{1.455204in}{0.500000in}}%
\pgfpathlineto{\pgfqpoint{1.481542in}{0.500000in}}%
\pgfpathlineto{\pgfqpoint{1.481542in}{0.500000in}}%
\pgfpathlineto{\pgfqpoint{1.455204in}{0.500000in}}%
\pgfpathlineto{\pgfqpoint{1.455204in}{0.500000in}}%
\pgfpathclose%
\pgfusepath{fill}%
\end{pgfscope}%
\begin{pgfscope}%
\pgfpathrectangle{\pgfqpoint{0.750000in}{0.500000in}}{\pgfqpoint{4.650000in}{3.020000in}}%
\pgfusepath{clip}%
\pgfsetbuttcap%
\pgfsetmiterjoin%
\definecolor{currentfill}{rgb}{0.121569,0.466667,0.705882}%
\pgfsetfillcolor{currentfill}%
\pgfsetlinewidth{0.000000pt}%
\definecolor{currentstroke}{rgb}{0.000000,0.000000,0.000000}%
\pgfsetstrokecolor{currentstroke}%
\pgfsetstrokeopacity{0.000000}%
\pgfsetdash{}{0pt}%
\pgfpathmoveto{\pgfqpoint{1.488127in}{0.500000in}}%
\pgfpathlineto{\pgfqpoint{1.514465in}{0.500000in}}%
\pgfpathlineto{\pgfqpoint{1.514465in}{0.500000in}}%
\pgfpathlineto{\pgfqpoint{1.488127in}{0.500000in}}%
\pgfpathlineto{\pgfqpoint{1.488127in}{0.500000in}}%
\pgfpathclose%
\pgfusepath{fill}%
\end{pgfscope}%
\begin{pgfscope}%
\pgfpathrectangle{\pgfqpoint{0.750000in}{0.500000in}}{\pgfqpoint{4.650000in}{3.020000in}}%
\pgfusepath{clip}%
\pgfsetbuttcap%
\pgfsetmiterjoin%
\definecolor{currentfill}{rgb}{0.121569,0.466667,0.705882}%
\pgfsetfillcolor{currentfill}%
\pgfsetlinewidth{0.000000pt}%
\definecolor{currentstroke}{rgb}{0.000000,0.000000,0.000000}%
\pgfsetstrokecolor{currentstroke}%
\pgfsetstrokeopacity{0.000000}%
\pgfsetdash{}{0pt}%
\pgfpathmoveto{\pgfqpoint{1.521049in}{0.500000in}}%
\pgfpathlineto{\pgfqpoint{1.547387in}{0.500000in}}%
\pgfpathlineto{\pgfqpoint{1.547387in}{0.500000in}}%
\pgfpathlineto{\pgfqpoint{1.521049in}{0.500000in}}%
\pgfpathlineto{\pgfqpoint{1.521049in}{0.500000in}}%
\pgfpathclose%
\pgfusepath{fill}%
\end{pgfscope}%
\begin{pgfscope}%
\pgfpathrectangle{\pgfqpoint{0.750000in}{0.500000in}}{\pgfqpoint{4.650000in}{3.020000in}}%
\pgfusepath{clip}%
\pgfsetbuttcap%
\pgfsetmiterjoin%
\definecolor{currentfill}{rgb}{0.121569,0.466667,0.705882}%
\pgfsetfillcolor{currentfill}%
\pgfsetlinewidth{0.000000pt}%
\definecolor{currentstroke}{rgb}{0.000000,0.000000,0.000000}%
\pgfsetstrokecolor{currentstroke}%
\pgfsetstrokeopacity{0.000000}%
\pgfsetdash{}{0pt}%
\pgfpathmoveto{\pgfqpoint{1.553972in}{0.500000in}}%
\pgfpathlineto{\pgfqpoint{1.580310in}{0.500000in}}%
\pgfpathlineto{\pgfqpoint{1.580310in}{0.500000in}}%
\pgfpathlineto{\pgfqpoint{1.553972in}{0.500000in}}%
\pgfpathlineto{\pgfqpoint{1.553972in}{0.500000in}}%
\pgfpathclose%
\pgfusepath{fill}%
\end{pgfscope}%
\begin{pgfscope}%
\pgfpathrectangle{\pgfqpoint{0.750000in}{0.500000in}}{\pgfqpoint{4.650000in}{3.020000in}}%
\pgfusepath{clip}%
\pgfsetbuttcap%
\pgfsetmiterjoin%
\definecolor{currentfill}{rgb}{0.121569,0.466667,0.705882}%
\pgfsetfillcolor{currentfill}%
\pgfsetlinewidth{0.000000pt}%
\definecolor{currentstroke}{rgb}{0.000000,0.000000,0.000000}%
\pgfsetstrokecolor{currentstroke}%
\pgfsetstrokeopacity{0.000000}%
\pgfsetdash{}{0pt}%
\pgfpathmoveto{\pgfqpoint{1.586895in}{0.500000in}}%
\pgfpathlineto{\pgfqpoint{1.613233in}{0.500000in}}%
\pgfpathlineto{\pgfqpoint{1.613233in}{0.500000in}}%
\pgfpathlineto{\pgfqpoint{1.586895in}{0.500000in}}%
\pgfpathlineto{\pgfqpoint{1.586895in}{0.500000in}}%
\pgfpathclose%
\pgfusepath{fill}%
\end{pgfscope}%
\begin{pgfscope}%
\pgfpathrectangle{\pgfqpoint{0.750000in}{0.500000in}}{\pgfqpoint{4.650000in}{3.020000in}}%
\pgfusepath{clip}%
\pgfsetbuttcap%
\pgfsetmiterjoin%
\definecolor{currentfill}{rgb}{0.121569,0.466667,0.705882}%
\pgfsetfillcolor{currentfill}%
\pgfsetlinewidth{0.000000pt}%
\definecolor{currentstroke}{rgb}{0.000000,0.000000,0.000000}%
\pgfsetstrokecolor{currentstroke}%
\pgfsetstrokeopacity{0.000000}%
\pgfsetdash{}{0pt}%
\pgfpathmoveto{\pgfqpoint{1.619817in}{0.500000in}}%
\pgfpathlineto{\pgfqpoint{1.646155in}{0.500000in}}%
\pgfpathlineto{\pgfqpoint{1.646155in}{0.500000in}}%
\pgfpathlineto{\pgfqpoint{1.619817in}{0.500000in}}%
\pgfpathlineto{\pgfqpoint{1.619817in}{0.500000in}}%
\pgfpathclose%
\pgfusepath{fill}%
\end{pgfscope}%
\begin{pgfscope}%
\pgfpathrectangle{\pgfqpoint{0.750000in}{0.500000in}}{\pgfqpoint{4.650000in}{3.020000in}}%
\pgfusepath{clip}%
\pgfsetbuttcap%
\pgfsetmiterjoin%
\definecolor{currentfill}{rgb}{0.121569,0.466667,0.705882}%
\pgfsetfillcolor{currentfill}%
\pgfsetlinewidth{0.000000pt}%
\definecolor{currentstroke}{rgb}{0.000000,0.000000,0.000000}%
\pgfsetstrokecolor{currentstroke}%
\pgfsetstrokeopacity{0.000000}%
\pgfsetdash{}{0pt}%
\pgfpathmoveto{\pgfqpoint{1.652740in}{0.500000in}}%
\pgfpathlineto{\pgfqpoint{1.679078in}{0.500000in}}%
\pgfpathlineto{\pgfqpoint{1.679078in}{0.500000in}}%
\pgfpathlineto{\pgfqpoint{1.652740in}{0.500000in}}%
\pgfpathlineto{\pgfqpoint{1.652740in}{0.500000in}}%
\pgfpathclose%
\pgfusepath{fill}%
\end{pgfscope}%
\begin{pgfscope}%
\pgfpathrectangle{\pgfqpoint{0.750000in}{0.500000in}}{\pgfqpoint{4.650000in}{3.020000in}}%
\pgfusepath{clip}%
\pgfsetbuttcap%
\pgfsetmiterjoin%
\definecolor{currentfill}{rgb}{0.121569,0.466667,0.705882}%
\pgfsetfillcolor{currentfill}%
\pgfsetlinewidth{0.000000pt}%
\definecolor{currentstroke}{rgb}{0.000000,0.000000,0.000000}%
\pgfsetstrokecolor{currentstroke}%
\pgfsetstrokeopacity{0.000000}%
\pgfsetdash{}{0pt}%
\pgfpathmoveto{\pgfqpoint{1.685663in}{0.500000in}}%
\pgfpathlineto{\pgfqpoint{1.712001in}{0.500000in}}%
\pgfpathlineto{\pgfqpoint{1.712001in}{0.500000in}}%
\pgfpathlineto{\pgfqpoint{1.685663in}{0.500000in}}%
\pgfpathlineto{\pgfqpoint{1.685663in}{0.500000in}}%
\pgfpathclose%
\pgfusepath{fill}%
\end{pgfscope}%
\begin{pgfscope}%
\pgfpathrectangle{\pgfqpoint{0.750000in}{0.500000in}}{\pgfqpoint{4.650000in}{3.020000in}}%
\pgfusepath{clip}%
\pgfsetbuttcap%
\pgfsetmiterjoin%
\definecolor{currentfill}{rgb}{0.121569,0.466667,0.705882}%
\pgfsetfillcolor{currentfill}%
\pgfsetlinewidth{0.000000pt}%
\definecolor{currentstroke}{rgb}{0.000000,0.000000,0.000000}%
\pgfsetstrokecolor{currentstroke}%
\pgfsetstrokeopacity{0.000000}%
\pgfsetdash{}{0pt}%
\pgfpathmoveto{\pgfqpoint{1.718585in}{0.500000in}}%
\pgfpathlineto{\pgfqpoint{1.744924in}{0.500000in}}%
\pgfpathlineto{\pgfqpoint{1.744924in}{0.500000in}}%
\pgfpathlineto{\pgfqpoint{1.718585in}{0.500000in}}%
\pgfpathlineto{\pgfqpoint{1.718585in}{0.500000in}}%
\pgfpathclose%
\pgfusepath{fill}%
\end{pgfscope}%
\begin{pgfscope}%
\pgfpathrectangle{\pgfqpoint{0.750000in}{0.500000in}}{\pgfqpoint{4.650000in}{3.020000in}}%
\pgfusepath{clip}%
\pgfsetbuttcap%
\pgfsetmiterjoin%
\definecolor{currentfill}{rgb}{0.121569,0.466667,0.705882}%
\pgfsetfillcolor{currentfill}%
\pgfsetlinewidth{0.000000pt}%
\definecolor{currentstroke}{rgb}{0.000000,0.000000,0.000000}%
\pgfsetstrokecolor{currentstroke}%
\pgfsetstrokeopacity{0.000000}%
\pgfsetdash{}{0pt}%
\pgfpathmoveto{\pgfqpoint{1.751508in}{0.500000in}}%
\pgfpathlineto{\pgfqpoint{1.777846in}{0.500000in}}%
\pgfpathlineto{\pgfqpoint{1.777846in}{0.500000in}}%
\pgfpathlineto{\pgfqpoint{1.751508in}{0.500000in}}%
\pgfpathlineto{\pgfqpoint{1.751508in}{0.500000in}}%
\pgfpathclose%
\pgfusepath{fill}%
\end{pgfscope}%
\begin{pgfscope}%
\pgfpathrectangle{\pgfqpoint{0.750000in}{0.500000in}}{\pgfqpoint{4.650000in}{3.020000in}}%
\pgfusepath{clip}%
\pgfsetbuttcap%
\pgfsetmiterjoin%
\definecolor{currentfill}{rgb}{0.121569,0.466667,0.705882}%
\pgfsetfillcolor{currentfill}%
\pgfsetlinewidth{0.000000pt}%
\definecolor{currentstroke}{rgb}{0.000000,0.000000,0.000000}%
\pgfsetstrokecolor{currentstroke}%
\pgfsetstrokeopacity{0.000000}%
\pgfsetdash{}{0pt}%
\pgfpathmoveto{\pgfqpoint{1.784431in}{0.500000in}}%
\pgfpathlineto{\pgfqpoint{1.810769in}{0.500000in}}%
\pgfpathlineto{\pgfqpoint{1.810769in}{0.500000in}}%
\pgfpathlineto{\pgfqpoint{1.784431in}{0.500000in}}%
\pgfpathlineto{\pgfqpoint{1.784431in}{0.500000in}}%
\pgfpathclose%
\pgfusepath{fill}%
\end{pgfscope}%
\begin{pgfscope}%
\pgfpathrectangle{\pgfqpoint{0.750000in}{0.500000in}}{\pgfqpoint{4.650000in}{3.020000in}}%
\pgfusepath{clip}%
\pgfsetbuttcap%
\pgfsetmiterjoin%
\definecolor{currentfill}{rgb}{0.121569,0.466667,0.705882}%
\pgfsetfillcolor{currentfill}%
\pgfsetlinewidth{0.000000pt}%
\definecolor{currentstroke}{rgb}{0.000000,0.000000,0.000000}%
\pgfsetstrokecolor{currentstroke}%
\pgfsetstrokeopacity{0.000000}%
\pgfsetdash{}{0pt}%
\pgfpathmoveto{\pgfqpoint{1.817353in}{0.500000in}}%
\pgfpathlineto{\pgfqpoint{1.843692in}{0.500000in}}%
\pgfpathlineto{\pgfqpoint{1.843692in}{0.500000in}}%
\pgfpathlineto{\pgfqpoint{1.817353in}{0.500000in}}%
\pgfpathlineto{\pgfqpoint{1.817353in}{0.500000in}}%
\pgfpathclose%
\pgfusepath{fill}%
\end{pgfscope}%
\begin{pgfscope}%
\pgfpathrectangle{\pgfqpoint{0.750000in}{0.500000in}}{\pgfqpoint{4.650000in}{3.020000in}}%
\pgfusepath{clip}%
\pgfsetbuttcap%
\pgfsetmiterjoin%
\definecolor{currentfill}{rgb}{0.121569,0.466667,0.705882}%
\pgfsetfillcolor{currentfill}%
\pgfsetlinewidth{0.000000pt}%
\definecolor{currentstroke}{rgb}{0.000000,0.000000,0.000000}%
\pgfsetstrokecolor{currentstroke}%
\pgfsetstrokeopacity{0.000000}%
\pgfsetdash{}{0pt}%
\pgfpathmoveto{\pgfqpoint{1.850276in}{0.500000in}}%
\pgfpathlineto{\pgfqpoint{1.876614in}{0.500000in}}%
\pgfpathlineto{\pgfqpoint{1.876614in}{0.500000in}}%
\pgfpathlineto{\pgfqpoint{1.850276in}{0.500000in}}%
\pgfpathlineto{\pgfqpoint{1.850276in}{0.500000in}}%
\pgfpathclose%
\pgfusepath{fill}%
\end{pgfscope}%
\begin{pgfscope}%
\pgfpathrectangle{\pgfqpoint{0.750000in}{0.500000in}}{\pgfqpoint{4.650000in}{3.020000in}}%
\pgfusepath{clip}%
\pgfsetbuttcap%
\pgfsetmiterjoin%
\definecolor{currentfill}{rgb}{0.121569,0.466667,0.705882}%
\pgfsetfillcolor{currentfill}%
\pgfsetlinewidth{0.000000pt}%
\definecolor{currentstroke}{rgb}{0.000000,0.000000,0.000000}%
\pgfsetstrokecolor{currentstroke}%
\pgfsetstrokeopacity{0.000000}%
\pgfsetdash{}{0pt}%
\pgfpathmoveto{\pgfqpoint{1.883199in}{0.500000in}}%
\pgfpathlineto{\pgfqpoint{1.909537in}{0.500000in}}%
\pgfpathlineto{\pgfqpoint{1.909537in}{0.500000in}}%
\pgfpathlineto{\pgfqpoint{1.883199in}{0.500000in}}%
\pgfpathlineto{\pgfqpoint{1.883199in}{0.500000in}}%
\pgfpathclose%
\pgfusepath{fill}%
\end{pgfscope}%
\begin{pgfscope}%
\pgfpathrectangle{\pgfqpoint{0.750000in}{0.500000in}}{\pgfqpoint{4.650000in}{3.020000in}}%
\pgfusepath{clip}%
\pgfsetbuttcap%
\pgfsetmiterjoin%
\definecolor{currentfill}{rgb}{0.121569,0.466667,0.705882}%
\pgfsetfillcolor{currentfill}%
\pgfsetlinewidth{0.000000pt}%
\definecolor{currentstroke}{rgb}{0.000000,0.000000,0.000000}%
\pgfsetstrokecolor{currentstroke}%
\pgfsetstrokeopacity{0.000000}%
\pgfsetdash{}{0pt}%
\pgfpathmoveto{\pgfqpoint{1.916121in}{0.500000in}}%
\pgfpathlineto{\pgfqpoint{1.942460in}{0.500000in}}%
\pgfpathlineto{\pgfqpoint{1.942460in}{0.500000in}}%
\pgfpathlineto{\pgfqpoint{1.916121in}{0.500000in}}%
\pgfpathlineto{\pgfqpoint{1.916121in}{0.500000in}}%
\pgfpathclose%
\pgfusepath{fill}%
\end{pgfscope}%
\begin{pgfscope}%
\pgfpathrectangle{\pgfqpoint{0.750000in}{0.500000in}}{\pgfqpoint{4.650000in}{3.020000in}}%
\pgfusepath{clip}%
\pgfsetbuttcap%
\pgfsetmiterjoin%
\definecolor{currentfill}{rgb}{0.121569,0.466667,0.705882}%
\pgfsetfillcolor{currentfill}%
\pgfsetlinewidth{0.000000pt}%
\definecolor{currentstroke}{rgb}{0.000000,0.000000,0.000000}%
\pgfsetstrokecolor{currentstroke}%
\pgfsetstrokeopacity{0.000000}%
\pgfsetdash{}{0pt}%
\pgfpathmoveto{\pgfqpoint{1.949044in}{0.500000in}}%
\pgfpathlineto{\pgfqpoint{1.975382in}{0.500000in}}%
\pgfpathlineto{\pgfqpoint{1.975382in}{0.500000in}}%
\pgfpathlineto{\pgfqpoint{1.949044in}{0.500000in}}%
\pgfpathlineto{\pgfqpoint{1.949044in}{0.500000in}}%
\pgfpathclose%
\pgfusepath{fill}%
\end{pgfscope}%
\begin{pgfscope}%
\pgfpathrectangle{\pgfqpoint{0.750000in}{0.500000in}}{\pgfqpoint{4.650000in}{3.020000in}}%
\pgfusepath{clip}%
\pgfsetbuttcap%
\pgfsetmiterjoin%
\definecolor{currentfill}{rgb}{0.121569,0.466667,0.705882}%
\pgfsetfillcolor{currentfill}%
\pgfsetlinewidth{0.000000pt}%
\definecolor{currentstroke}{rgb}{0.000000,0.000000,0.000000}%
\pgfsetstrokecolor{currentstroke}%
\pgfsetstrokeopacity{0.000000}%
\pgfsetdash{}{0pt}%
\pgfpathmoveto{\pgfqpoint{1.981967in}{0.500000in}}%
\pgfpathlineto{\pgfqpoint{2.008305in}{0.500000in}}%
\pgfpathlineto{\pgfqpoint{2.008305in}{0.500000in}}%
\pgfpathlineto{\pgfqpoint{1.981967in}{0.500000in}}%
\pgfpathlineto{\pgfqpoint{1.981967in}{0.500000in}}%
\pgfpathclose%
\pgfusepath{fill}%
\end{pgfscope}%
\begin{pgfscope}%
\pgfpathrectangle{\pgfqpoint{0.750000in}{0.500000in}}{\pgfqpoint{4.650000in}{3.020000in}}%
\pgfusepath{clip}%
\pgfsetbuttcap%
\pgfsetmiterjoin%
\definecolor{currentfill}{rgb}{0.121569,0.466667,0.705882}%
\pgfsetfillcolor{currentfill}%
\pgfsetlinewidth{0.000000pt}%
\definecolor{currentstroke}{rgb}{0.000000,0.000000,0.000000}%
\pgfsetstrokecolor{currentstroke}%
\pgfsetstrokeopacity{0.000000}%
\pgfsetdash{}{0pt}%
\pgfpathmoveto{\pgfqpoint{2.014890in}{0.500000in}}%
\pgfpathlineto{\pgfqpoint{2.041228in}{0.500000in}}%
\pgfpathlineto{\pgfqpoint{2.041228in}{0.500000in}}%
\pgfpathlineto{\pgfqpoint{2.014890in}{0.500000in}}%
\pgfpathlineto{\pgfqpoint{2.014890in}{0.500000in}}%
\pgfpathclose%
\pgfusepath{fill}%
\end{pgfscope}%
\begin{pgfscope}%
\pgfpathrectangle{\pgfqpoint{0.750000in}{0.500000in}}{\pgfqpoint{4.650000in}{3.020000in}}%
\pgfusepath{clip}%
\pgfsetbuttcap%
\pgfsetmiterjoin%
\definecolor{currentfill}{rgb}{0.121569,0.466667,0.705882}%
\pgfsetfillcolor{currentfill}%
\pgfsetlinewidth{0.000000pt}%
\definecolor{currentstroke}{rgb}{0.000000,0.000000,0.000000}%
\pgfsetstrokecolor{currentstroke}%
\pgfsetstrokeopacity{0.000000}%
\pgfsetdash{}{0pt}%
\pgfpathmoveto{\pgfqpoint{2.047812in}{0.500000in}}%
\pgfpathlineto{\pgfqpoint{2.074150in}{0.500000in}}%
\pgfpathlineto{\pgfqpoint{2.074150in}{0.500001in}}%
\pgfpathlineto{\pgfqpoint{2.047812in}{0.500001in}}%
\pgfpathlineto{\pgfqpoint{2.047812in}{0.500000in}}%
\pgfpathclose%
\pgfusepath{fill}%
\end{pgfscope}%
\begin{pgfscope}%
\pgfpathrectangle{\pgfqpoint{0.750000in}{0.500000in}}{\pgfqpoint{4.650000in}{3.020000in}}%
\pgfusepath{clip}%
\pgfsetbuttcap%
\pgfsetmiterjoin%
\definecolor{currentfill}{rgb}{0.121569,0.466667,0.705882}%
\pgfsetfillcolor{currentfill}%
\pgfsetlinewidth{0.000000pt}%
\definecolor{currentstroke}{rgb}{0.000000,0.000000,0.000000}%
\pgfsetstrokecolor{currentstroke}%
\pgfsetstrokeopacity{0.000000}%
\pgfsetdash{}{0pt}%
\pgfpathmoveto{\pgfqpoint{2.080735in}{0.500000in}}%
\pgfpathlineto{\pgfqpoint{2.107073in}{0.500000in}}%
\pgfpathlineto{\pgfqpoint{2.107073in}{0.500001in}}%
\pgfpathlineto{\pgfqpoint{2.080735in}{0.500001in}}%
\pgfpathlineto{\pgfqpoint{2.080735in}{0.500000in}}%
\pgfpathclose%
\pgfusepath{fill}%
\end{pgfscope}%
\begin{pgfscope}%
\pgfpathrectangle{\pgfqpoint{0.750000in}{0.500000in}}{\pgfqpoint{4.650000in}{3.020000in}}%
\pgfusepath{clip}%
\pgfsetbuttcap%
\pgfsetmiterjoin%
\definecolor{currentfill}{rgb}{0.121569,0.466667,0.705882}%
\pgfsetfillcolor{currentfill}%
\pgfsetlinewidth{0.000000pt}%
\definecolor{currentstroke}{rgb}{0.000000,0.000000,0.000000}%
\pgfsetstrokecolor{currentstroke}%
\pgfsetstrokeopacity{0.000000}%
\pgfsetdash{}{0pt}%
\pgfpathmoveto{\pgfqpoint{2.113658in}{0.500000in}}%
\pgfpathlineto{\pgfqpoint{2.139996in}{0.500000in}}%
\pgfpathlineto{\pgfqpoint{2.139996in}{0.500004in}}%
\pgfpathlineto{\pgfqpoint{2.113658in}{0.500004in}}%
\pgfpathlineto{\pgfqpoint{2.113658in}{0.500000in}}%
\pgfpathclose%
\pgfusepath{fill}%
\end{pgfscope}%
\begin{pgfscope}%
\pgfpathrectangle{\pgfqpoint{0.750000in}{0.500000in}}{\pgfqpoint{4.650000in}{3.020000in}}%
\pgfusepath{clip}%
\pgfsetbuttcap%
\pgfsetmiterjoin%
\definecolor{currentfill}{rgb}{0.121569,0.466667,0.705882}%
\pgfsetfillcolor{currentfill}%
\pgfsetlinewidth{0.000000pt}%
\definecolor{currentstroke}{rgb}{0.000000,0.000000,0.000000}%
\pgfsetstrokecolor{currentstroke}%
\pgfsetstrokeopacity{0.000000}%
\pgfsetdash{}{0pt}%
\pgfpathmoveto{\pgfqpoint{2.146580in}{0.500000in}}%
\pgfpathlineto{\pgfqpoint{2.172918in}{0.500000in}}%
\pgfpathlineto{\pgfqpoint{2.172918in}{0.500010in}}%
\pgfpathlineto{\pgfqpoint{2.146580in}{0.500010in}}%
\pgfpathlineto{\pgfqpoint{2.146580in}{0.500000in}}%
\pgfpathclose%
\pgfusepath{fill}%
\end{pgfscope}%
\begin{pgfscope}%
\pgfpathrectangle{\pgfqpoint{0.750000in}{0.500000in}}{\pgfqpoint{4.650000in}{3.020000in}}%
\pgfusepath{clip}%
\pgfsetbuttcap%
\pgfsetmiterjoin%
\definecolor{currentfill}{rgb}{0.121569,0.466667,0.705882}%
\pgfsetfillcolor{currentfill}%
\pgfsetlinewidth{0.000000pt}%
\definecolor{currentstroke}{rgb}{0.000000,0.000000,0.000000}%
\pgfsetstrokecolor{currentstroke}%
\pgfsetstrokeopacity{0.000000}%
\pgfsetdash{}{0pt}%
\pgfpathmoveto{\pgfqpoint{2.179503in}{0.500000in}}%
\pgfpathlineto{\pgfqpoint{2.205841in}{0.500000in}}%
\pgfpathlineto{\pgfqpoint{2.205841in}{0.500025in}}%
\pgfpathlineto{\pgfqpoint{2.179503in}{0.500025in}}%
\pgfpathlineto{\pgfqpoint{2.179503in}{0.500000in}}%
\pgfpathclose%
\pgfusepath{fill}%
\end{pgfscope}%
\begin{pgfscope}%
\pgfpathrectangle{\pgfqpoint{0.750000in}{0.500000in}}{\pgfqpoint{4.650000in}{3.020000in}}%
\pgfusepath{clip}%
\pgfsetbuttcap%
\pgfsetmiterjoin%
\definecolor{currentfill}{rgb}{0.121569,0.466667,0.705882}%
\pgfsetfillcolor{currentfill}%
\pgfsetlinewidth{0.000000pt}%
\definecolor{currentstroke}{rgb}{0.000000,0.000000,0.000000}%
\pgfsetstrokecolor{currentstroke}%
\pgfsetstrokeopacity{0.000000}%
\pgfsetdash{}{0pt}%
\pgfpathmoveto{\pgfqpoint{2.212426in}{0.500000in}}%
\pgfpathlineto{\pgfqpoint{2.238764in}{0.500000in}}%
\pgfpathlineto{\pgfqpoint{2.238764in}{0.500060in}}%
\pgfpathlineto{\pgfqpoint{2.212426in}{0.500060in}}%
\pgfpathlineto{\pgfqpoint{2.212426in}{0.500000in}}%
\pgfpathclose%
\pgfusepath{fill}%
\end{pgfscope}%
\begin{pgfscope}%
\pgfpathrectangle{\pgfqpoint{0.750000in}{0.500000in}}{\pgfqpoint{4.650000in}{3.020000in}}%
\pgfusepath{clip}%
\pgfsetbuttcap%
\pgfsetmiterjoin%
\definecolor{currentfill}{rgb}{0.121569,0.466667,0.705882}%
\pgfsetfillcolor{currentfill}%
\pgfsetlinewidth{0.000000pt}%
\definecolor{currentstroke}{rgb}{0.000000,0.000000,0.000000}%
\pgfsetstrokecolor{currentstroke}%
\pgfsetstrokeopacity{0.000000}%
\pgfsetdash{}{0pt}%
\pgfpathmoveto{\pgfqpoint{2.245348in}{0.500000in}}%
\pgfpathlineto{\pgfqpoint{2.271686in}{0.500000in}}%
\pgfpathlineto{\pgfqpoint{2.271686in}{0.500138in}}%
\pgfpathlineto{\pgfqpoint{2.245348in}{0.500138in}}%
\pgfpathlineto{\pgfqpoint{2.245348in}{0.500000in}}%
\pgfpathclose%
\pgfusepath{fill}%
\end{pgfscope}%
\begin{pgfscope}%
\pgfpathrectangle{\pgfqpoint{0.750000in}{0.500000in}}{\pgfqpoint{4.650000in}{3.020000in}}%
\pgfusepath{clip}%
\pgfsetbuttcap%
\pgfsetmiterjoin%
\definecolor{currentfill}{rgb}{0.121569,0.466667,0.705882}%
\pgfsetfillcolor{currentfill}%
\pgfsetlinewidth{0.000000pt}%
\definecolor{currentstroke}{rgb}{0.000000,0.000000,0.000000}%
\pgfsetstrokecolor{currentstroke}%
\pgfsetstrokeopacity{0.000000}%
\pgfsetdash{}{0pt}%
\pgfpathmoveto{\pgfqpoint{2.278271in}{0.500000in}}%
\pgfpathlineto{\pgfqpoint{2.304609in}{0.500000in}}%
\pgfpathlineto{\pgfqpoint{2.304609in}{0.500306in}}%
\pgfpathlineto{\pgfqpoint{2.278271in}{0.500306in}}%
\pgfpathlineto{\pgfqpoint{2.278271in}{0.500000in}}%
\pgfpathclose%
\pgfusepath{fill}%
\end{pgfscope}%
\begin{pgfscope}%
\pgfpathrectangle{\pgfqpoint{0.750000in}{0.500000in}}{\pgfqpoint{4.650000in}{3.020000in}}%
\pgfusepath{clip}%
\pgfsetbuttcap%
\pgfsetmiterjoin%
\definecolor{currentfill}{rgb}{0.121569,0.466667,0.705882}%
\pgfsetfillcolor{currentfill}%
\pgfsetlinewidth{0.000000pt}%
\definecolor{currentstroke}{rgb}{0.000000,0.000000,0.000000}%
\pgfsetstrokecolor{currentstroke}%
\pgfsetstrokeopacity{0.000000}%
\pgfsetdash{}{0pt}%
\pgfpathmoveto{\pgfqpoint{2.311194in}{0.500000in}}%
\pgfpathlineto{\pgfqpoint{2.337532in}{0.500000in}}%
\pgfpathlineto{\pgfqpoint{2.337532in}{0.500657in}}%
\pgfpathlineto{\pgfqpoint{2.311194in}{0.500657in}}%
\pgfpathlineto{\pgfqpoint{2.311194in}{0.500000in}}%
\pgfpathclose%
\pgfusepath{fill}%
\end{pgfscope}%
\begin{pgfscope}%
\pgfpathrectangle{\pgfqpoint{0.750000in}{0.500000in}}{\pgfqpoint{4.650000in}{3.020000in}}%
\pgfusepath{clip}%
\pgfsetbuttcap%
\pgfsetmiterjoin%
\definecolor{currentfill}{rgb}{0.121569,0.466667,0.705882}%
\pgfsetfillcolor{currentfill}%
\pgfsetlinewidth{0.000000pt}%
\definecolor{currentstroke}{rgb}{0.000000,0.000000,0.000000}%
\pgfsetstrokecolor{currentstroke}%
\pgfsetstrokeopacity{0.000000}%
\pgfsetdash{}{0pt}%
\pgfpathmoveto{\pgfqpoint{2.344116in}{0.500000in}}%
\pgfpathlineto{\pgfqpoint{2.370455in}{0.500000in}}%
\pgfpathlineto{\pgfqpoint{2.370455in}{0.501360in}}%
\pgfpathlineto{\pgfqpoint{2.344116in}{0.501360in}}%
\pgfpathlineto{\pgfqpoint{2.344116in}{0.500000in}}%
\pgfpathclose%
\pgfusepath{fill}%
\end{pgfscope}%
\begin{pgfscope}%
\pgfpathrectangle{\pgfqpoint{0.750000in}{0.500000in}}{\pgfqpoint{4.650000in}{3.020000in}}%
\pgfusepath{clip}%
\pgfsetbuttcap%
\pgfsetmiterjoin%
\definecolor{currentfill}{rgb}{0.121569,0.466667,0.705882}%
\pgfsetfillcolor{currentfill}%
\pgfsetlinewidth{0.000000pt}%
\definecolor{currentstroke}{rgb}{0.000000,0.000000,0.000000}%
\pgfsetstrokecolor{currentstroke}%
\pgfsetstrokeopacity{0.000000}%
\pgfsetdash{}{0pt}%
\pgfpathmoveto{\pgfqpoint{2.377039in}{0.500000in}}%
\pgfpathlineto{\pgfqpoint{2.403377in}{0.500000in}}%
\pgfpathlineto{\pgfqpoint{2.403377in}{0.502721in}}%
\pgfpathlineto{\pgfqpoint{2.377039in}{0.502721in}}%
\pgfpathlineto{\pgfqpoint{2.377039in}{0.500000in}}%
\pgfpathclose%
\pgfusepath{fill}%
\end{pgfscope}%
\begin{pgfscope}%
\pgfpathrectangle{\pgfqpoint{0.750000in}{0.500000in}}{\pgfqpoint{4.650000in}{3.020000in}}%
\pgfusepath{clip}%
\pgfsetbuttcap%
\pgfsetmiterjoin%
\definecolor{currentfill}{rgb}{0.121569,0.466667,0.705882}%
\pgfsetfillcolor{currentfill}%
\pgfsetlinewidth{0.000000pt}%
\definecolor{currentstroke}{rgb}{0.000000,0.000000,0.000000}%
\pgfsetstrokecolor{currentstroke}%
\pgfsetstrokeopacity{0.000000}%
\pgfsetdash{}{0pt}%
\pgfpathmoveto{\pgfqpoint{2.409962in}{0.500000in}}%
\pgfpathlineto{\pgfqpoint{2.436300in}{0.500000in}}%
\pgfpathlineto{\pgfqpoint{2.436300in}{0.505256in}}%
\pgfpathlineto{\pgfqpoint{2.409962in}{0.505256in}}%
\pgfpathlineto{\pgfqpoint{2.409962in}{0.500000in}}%
\pgfpathclose%
\pgfusepath{fill}%
\end{pgfscope}%
\begin{pgfscope}%
\pgfpathrectangle{\pgfqpoint{0.750000in}{0.500000in}}{\pgfqpoint{4.650000in}{3.020000in}}%
\pgfusepath{clip}%
\pgfsetbuttcap%
\pgfsetmiterjoin%
\definecolor{currentfill}{rgb}{0.121569,0.466667,0.705882}%
\pgfsetfillcolor{currentfill}%
\pgfsetlinewidth{0.000000pt}%
\definecolor{currentstroke}{rgb}{0.000000,0.000000,0.000000}%
\pgfsetstrokecolor{currentstroke}%
\pgfsetstrokeopacity{0.000000}%
\pgfsetdash{}{0pt}%
\pgfpathmoveto{\pgfqpoint{2.442884in}{0.500000in}}%
\pgfpathlineto{\pgfqpoint{2.469223in}{0.500000in}}%
\pgfpathlineto{\pgfqpoint{2.469223in}{0.509811in}}%
\pgfpathlineto{\pgfqpoint{2.442884in}{0.509811in}}%
\pgfpathlineto{\pgfqpoint{2.442884in}{0.500000in}}%
\pgfpathclose%
\pgfusepath{fill}%
\end{pgfscope}%
\begin{pgfscope}%
\pgfpathrectangle{\pgfqpoint{0.750000in}{0.500000in}}{\pgfqpoint{4.650000in}{3.020000in}}%
\pgfusepath{clip}%
\pgfsetbuttcap%
\pgfsetmiterjoin%
\definecolor{currentfill}{rgb}{0.121569,0.466667,0.705882}%
\pgfsetfillcolor{currentfill}%
\pgfsetlinewidth{0.000000pt}%
\definecolor{currentstroke}{rgb}{0.000000,0.000000,0.000000}%
\pgfsetstrokecolor{currentstroke}%
\pgfsetstrokeopacity{0.000000}%
\pgfsetdash{}{0pt}%
\pgfpathmoveto{\pgfqpoint{2.475807in}{0.500000in}}%
\pgfpathlineto{\pgfqpoint{2.502145in}{0.500000in}}%
\pgfpathlineto{\pgfqpoint{2.502145in}{0.517703in}}%
\pgfpathlineto{\pgfqpoint{2.475807in}{0.517703in}}%
\pgfpathlineto{\pgfqpoint{2.475807in}{0.500000in}}%
\pgfpathclose%
\pgfusepath{fill}%
\end{pgfscope}%
\begin{pgfscope}%
\pgfpathrectangle{\pgfqpoint{0.750000in}{0.500000in}}{\pgfqpoint{4.650000in}{3.020000in}}%
\pgfusepath{clip}%
\pgfsetbuttcap%
\pgfsetmiterjoin%
\definecolor{currentfill}{rgb}{0.121569,0.466667,0.705882}%
\pgfsetfillcolor{currentfill}%
\pgfsetlinewidth{0.000000pt}%
\definecolor{currentstroke}{rgb}{0.000000,0.000000,0.000000}%
\pgfsetstrokecolor{currentstroke}%
\pgfsetstrokeopacity{0.000000}%
\pgfsetdash{}{0pt}%
\pgfpathmoveto{\pgfqpoint{2.508730in}{0.500000in}}%
\pgfpathlineto{\pgfqpoint{2.535068in}{0.500000in}}%
\pgfpathlineto{\pgfqpoint{2.535068in}{0.530887in}}%
\pgfpathlineto{\pgfqpoint{2.508730in}{0.530887in}}%
\pgfpathlineto{\pgfqpoint{2.508730in}{0.500000in}}%
\pgfpathclose%
\pgfusepath{fill}%
\end{pgfscope}%
\begin{pgfscope}%
\pgfpathrectangle{\pgfqpoint{0.750000in}{0.500000in}}{\pgfqpoint{4.650000in}{3.020000in}}%
\pgfusepath{clip}%
\pgfsetbuttcap%
\pgfsetmiterjoin%
\definecolor{currentfill}{rgb}{0.121569,0.466667,0.705882}%
\pgfsetfillcolor{currentfill}%
\pgfsetlinewidth{0.000000pt}%
\definecolor{currentstroke}{rgb}{0.000000,0.000000,0.000000}%
\pgfsetstrokecolor{currentstroke}%
\pgfsetstrokeopacity{0.000000}%
\pgfsetdash{}{0pt}%
\pgfpathmoveto{\pgfqpoint{2.541653in}{0.500000in}}%
\pgfpathlineto{\pgfqpoint{2.567991in}{0.500000in}}%
\pgfpathlineto{\pgfqpoint{2.567991in}{0.552121in}}%
\pgfpathlineto{\pgfqpoint{2.541653in}{0.552121in}}%
\pgfpathlineto{\pgfqpoint{2.541653in}{0.500000in}}%
\pgfpathclose%
\pgfusepath{fill}%
\end{pgfscope}%
\begin{pgfscope}%
\pgfpathrectangle{\pgfqpoint{0.750000in}{0.500000in}}{\pgfqpoint{4.650000in}{3.020000in}}%
\pgfusepath{clip}%
\pgfsetbuttcap%
\pgfsetmiterjoin%
\definecolor{currentfill}{rgb}{0.121569,0.466667,0.705882}%
\pgfsetfillcolor{currentfill}%
\pgfsetlinewidth{0.000000pt}%
\definecolor{currentstroke}{rgb}{0.000000,0.000000,0.000000}%
\pgfsetstrokecolor{currentstroke}%
\pgfsetstrokeopacity{0.000000}%
\pgfsetdash{}{0pt}%
\pgfpathmoveto{\pgfqpoint{2.574575in}{0.500000in}}%
\pgfpathlineto{\pgfqpoint{2.600913in}{0.500000in}}%
\pgfpathlineto{\pgfqpoint{2.600913in}{0.585096in}}%
\pgfpathlineto{\pgfqpoint{2.574575in}{0.585096in}}%
\pgfpathlineto{\pgfqpoint{2.574575in}{0.500000in}}%
\pgfpathclose%
\pgfusepath{fill}%
\end{pgfscope}%
\begin{pgfscope}%
\pgfpathrectangle{\pgfqpoint{0.750000in}{0.500000in}}{\pgfqpoint{4.650000in}{3.020000in}}%
\pgfusepath{clip}%
\pgfsetbuttcap%
\pgfsetmiterjoin%
\definecolor{currentfill}{rgb}{0.121569,0.466667,0.705882}%
\pgfsetfillcolor{currentfill}%
\pgfsetlinewidth{0.000000pt}%
\definecolor{currentstroke}{rgb}{0.000000,0.000000,0.000000}%
\pgfsetstrokecolor{currentstroke}%
\pgfsetstrokeopacity{0.000000}%
\pgfsetdash{}{0pt}%
\pgfpathmoveto{\pgfqpoint{2.607498in}{0.500000in}}%
\pgfpathlineto{\pgfqpoint{2.633836in}{0.500000in}}%
\pgfpathlineto{\pgfqpoint{2.633836in}{0.634452in}}%
\pgfpathlineto{\pgfqpoint{2.607498in}{0.634452in}}%
\pgfpathlineto{\pgfqpoint{2.607498in}{0.500000in}}%
\pgfpathclose%
\pgfusepath{fill}%
\end{pgfscope}%
\begin{pgfscope}%
\pgfpathrectangle{\pgfqpoint{0.750000in}{0.500000in}}{\pgfqpoint{4.650000in}{3.020000in}}%
\pgfusepath{clip}%
\pgfsetbuttcap%
\pgfsetmiterjoin%
\definecolor{currentfill}{rgb}{0.121569,0.466667,0.705882}%
\pgfsetfillcolor{currentfill}%
\pgfsetlinewidth{0.000000pt}%
\definecolor{currentstroke}{rgb}{0.000000,0.000000,0.000000}%
\pgfsetstrokecolor{currentstroke}%
\pgfsetstrokeopacity{0.000000}%
\pgfsetdash{}{0pt}%
\pgfpathmoveto{\pgfqpoint{2.640421in}{0.500000in}}%
\pgfpathlineto{\pgfqpoint{2.666759in}{0.500000in}}%
\pgfpathlineto{\pgfqpoint{2.666759in}{0.705632in}}%
\pgfpathlineto{\pgfqpoint{2.640421in}{0.705632in}}%
\pgfpathlineto{\pgfqpoint{2.640421in}{0.500000in}}%
\pgfpathclose%
\pgfusepath{fill}%
\end{pgfscope}%
\begin{pgfscope}%
\pgfpathrectangle{\pgfqpoint{0.750000in}{0.500000in}}{\pgfqpoint{4.650000in}{3.020000in}}%
\pgfusepath{clip}%
\pgfsetbuttcap%
\pgfsetmiterjoin%
\definecolor{currentfill}{rgb}{0.121569,0.466667,0.705882}%
\pgfsetfillcolor{currentfill}%
\pgfsetlinewidth{0.000000pt}%
\definecolor{currentstroke}{rgb}{0.000000,0.000000,0.000000}%
\pgfsetstrokecolor{currentstroke}%
\pgfsetstrokeopacity{0.000000}%
\pgfsetdash{}{0pt}%
\pgfpathmoveto{\pgfqpoint{2.673343in}{0.500000in}}%
\pgfpathlineto{\pgfqpoint{2.699681in}{0.500000in}}%
\pgfpathlineto{\pgfqpoint{2.699681in}{0.804493in}}%
\pgfpathlineto{\pgfqpoint{2.673343in}{0.804493in}}%
\pgfpathlineto{\pgfqpoint{2.673343in}{0.500000in}}%
\pgfpathclose%
\pgfusepath{fill}%
\end{pgfscope}%
\begin{pgfscope}%
\pgfpathrectangle{\pgfqpoint{0.750000in}{0.500000in}}{\pgfqpoint{4.650000in}{3.020000in}}%
\pgfusepath{clip}%
\pgfsetbuttcap%
\pgfsetmiterjoin%
\definecolor{currentfill}{rgb}{0.121569,0.466667,0.705882}%
\pgfsetfillcolor{currentfill}%
\pgfsetlinewidth{0.000000pt}%
\definecolor{currentstroke}{rgb}{0.000000,0.000000,0.000000}%
\pgfsetstrokecolor{currentstroke}%
\pgfsetstrokeopacity{0.000000}%
\pgfsetdash{}{0pt}%
\pgfpathmoveto{\pgfqpoint{2.706266in}{0.500000in}}%
\pgfpathlineto{\pgfqpoint{2.732604in}{0.500000in}}%
\pgfpathlineto{\pgfqpoint{2.732604in}{0.936632in}}%
\pgfpathlineto{\pgfqpoint{2.706266in}{0.936632in}}%
\pgfpathlineto{\pgfqpoint{2.706266in}{0.500000in}}%
\pgfpathclose%
\pgfusepath{fill}%
\end{pgfscope}%
\begin{pgfscope}%
\pgfpathrectangle{\pgfqpoint{0.750000in}{0.500000in}}{\pgfqpoint{4.650000in}{3.020000in}}%
\pgfusepath{clip}%
\pgfsetbuttcap%
\pgfsetmiterjoin%
\definecolor{currentfill}{rgb}{0.121569,0.466667,0.705882}%
\pgfsetfillcolor{currentfill}%
\pgfsetlinewidth{0.000000pt}%
\definecolor{currentstroke}{rgb}{0.000000,0.000000,0.000000}%
\pgfsetstrokecolor{currentstroke}%
\pgfsetstrokeopacity{0.000000}%
\pgfsetdash{}{0pt}%
\pgfpathmoveto{\pgfqpoint{2.739189in}{0.500000in}}%
\pgfpathlineto{\pgfqpoint{2.765527in}{0.500000in}}%
\pgfpathlineto{\pgfqpoint{2.765527in}{1.106433in}}%
\pgfpathlineto{\pgfqpoint{2.739189in}{1.106433in}}%
\pgfpathlineto{\pgfqpoint{2.739189in}{0.500000in}}%
\pgfpathclose%
\pgfusepath{fill}%
\end{pgfscope}%
\begin{pgfscope}%
\pgfpathrectangle{\pgfqpoint{0.750000in}{0.500000in}}{\pgfqpoint{4.650000in}{3.020000in}}%
\pgfusepath{clip}%
\pgfsetbuttcap%
\pgfsetmiterjoin%
\definecolor{currentfill}{rgb}{0.121569,0.466667,0.705882}%
\pgfsetfillcolor{currentfill}%
\pgfsetlinewidth{0.000000pt}%
\definecolor{currentstroke}{rgb}{0.000000,0.000000,0.000000}%
\pgfsetstrokecolor{currentstroke}%
\pgfsetstrokeopacity{0.000000}%
\pgfsetdash{}{0pt}%
\pgfpathmoveto{\pgfqpoint{2.772111in}{0.500000in}}%
\pgfpathlineto{\pgfqpoint{2.798449in}{0.500000in}}%
\pgfpathlineto{\pgfqpoint{2.798449in}{1.315928in}}%
\pgfpathlineto{\pgfqpoint{2.772111in}{1.315928in}}%
\pgfpathlineto{\pgfqpoint{2.772111in}{0.500000in}}%
\pgfpathclose%
\pgfusepath{fill}%
\end{pgfscope}%
\begin{pgfscope}%
\pgfpathrectangle{\pgfqpoint{0.750000in}{0.500000in}}{\pgfqpoint{4.650000in}{3.020000in}}%
\pgfusepath{clip}%
\pgfsetbuttcap%
\pgfsetmiterjoin%
\definecolor{currentfill}{rgb}{0.121569,0.466667,0.705882}%
\pgfsetfillcolor{currentfill}%
\pgfsetlinewidth{0.000000pt}%
\definecolor{currentstroke}{rgb}{0.000000,0.000000,0.000000}%
\pgfsetstrokecolor{currentstroke}%
\pgfsetstrokeopacity{0.000000}%
\pgfsetdash{}{0pt}%
\pgfpathmoveto{\pgfqpoint{2.805034in}{0.500000in}}%
\pgfpathlineto{\pgfqpoint{2.831372in}{0.500000in}}%
\pgfpathlineto{\pgfqpoint{2.831372in}{1.563621in}}%
\pgfpathlineto{\pgfqpoint{2.805034in}{1.563621in}}%
\pgfpathlineto{\pgfqpoint{2.805034in}{0.500000in}}%
\pgfpathclose%
\pgfusepath{fill}%
\end{pgfscope}%
\begin{pgfscope}%
\pgfpathrectangle{\pgfqpoint{0.750000in}{0.500000in}}{\pgfqpoint{4.650000in}{3.020000in}}%
\pgfusepath{clip}%
\pgfsetbuttcap%
\pgfsetmiterjoin%
\definecolor{currentfill}{rgb}{0.121569,0.466667,0.705882}%
\pgfsetfillcolor{currentfill}%
\pgfsetlinewidth{0.000000pt}%
\definecolor{currentstroke}{rgb}{0.000000,0.000000,0.000000}%
\pgfsetstrokecolor{currentstroke}%
\pgfsetstrokeopacity{0.000000}%
\pgfsetdash{}{0pt}%
\pgfpathmoveto{\pgfqpoint{2.837957in}{0.500000in}}%
\pgfpathlineto{\pgfqpoint{2.864295in}{0.500000in}}%
\pgfpathlineto{\pgfqpoint{2.864295in}{1.843521in}}%
\pgfpathlineto{\pgfqpoint{2.837957in}{1.843521in}}%
\pgfpathlineto{\pgfqpoint{2.837957in}{0.500000in}}%
\pgfpathclose%
\pgfusepath{fill}%
\end{pgfscope}%
\begin{pgfscope}%
\pgfpathrectangle{\pgfqpoint{0.750000in}{0.500000in}}{\pgfqpoint{4.650000in}{3.020000in}}%
\pgfusepath{clip}%
\pgfsetbuttcap%
\pgfsetmiterjoin%
\definecolor{currentfill}{rgb}{0.121569,0.466667,0.705882}%
\pgfsetfillcolor{currentfill}%
\pgfsetlinewidth{0.000000pt}%
\definecolor{currentstroke}{rgb}{0.000000,0.000000,0.000000}%
\pgfsetstrokecolor{currentstroke}%
\pgfsetstrokeopacity{0.000000}%
\pgfsetdash{}{0pt}%
\pgfpathmoveto{\pgfqpoint{2.870879in}{0.500000in}}%
\pgfpathlineto{\pgfqpoint{2.897218in}{0.500000in}}%
\pgfpathlineto{\pgfqpoint{2.897218in}{2.144655in}}%
\pgfpathlineto{\pgfqpoint{2.870879in}{2.144655in}}%
\pgfpathlineto{\pgfqpoint{2.870879in}{0.500000in}}%
\pgfpathclose%
\pgfusepath{fill}%
\end{pgfscope}%
\begin{pgfscope}%
\pgfpathrectangle{\pgfqpoint{0.750000in}{0.500000in}}{\pgfqpoint{4.650000in}{3.020000in}}%
\pgfusepath{clip}%
\pgfsetbuttcap%
\pgfsetmiterjoin%
\definecolor{currentfill}{rgb}{0.121569,0.466667,0.705882}%
\pgfsetfillcolor{currentfill}%
\pgfsetlinewidth{0.000000pt}%
\definecolor{currentstroke}{rgb}{0.000000,0.000000,0.000000}%
\pgfsetstrokecolor{currentstroke}%
\pgfsetstrokeopacity{0.000000}%
\pgfsetdash{}{0pt}%
\pgfpathmoveto{\pgfqpoint{2.903802in}{0.500000in}}%
\pgfpathlineto{\pgfqpoint{2.930140in}{0.500000in}}%
\pgfpathlineto{\pgfqpoint{2.930140in}{2.451286in}}%
\pgfpathlineto{\pgfqpoint{2.903802in}{2.451286in}}%
\pgfpathlineto{\pgfqpoint{2.903802in}{0.500000in}}%
\pgfpathclose%
\pgfusepath{fill}%
\end{pgfscope}%
\begin{pgfscope}%
\pgfpathrectangle{\pgfqpoint{0.750000in}{0.500000in}}{\pgfqpoint{4.650000in}{3.020000in}}%
\pgfusepath{clip}%
\pgfsetbuttcap%
\pgfsetmiterjoin%
\definecolor{currentfill}{rgb}{0.121569,0.466667,0.705882}%
\pgfsetfillcolor{currentfill}%
\pgfsetlinewidth{0.000000pt}%
\definecolor{currentstroke}{rgb}{0.000000,0.000000,0.000000}%
\pgfsetstrokecolor{currentstroke}%
\pgfsetstrokeopacity{0.000000}%
\pgfsetdash{}{0pt}%
\pgfpathmoveto{\pgfqpoint{2.936725in}{0.500000in}}%
\pgfpathlineto{\pgfqpoint{2.963063in}{0.500000in}}%
\pgfpathlineto{\pgfqpoint{2.963063in}{2.743978in}}%
\pgfpathlineto{\pgfqpoint{2.936725in}{2.743978in}}%
\pgfpathlineto{\pgfqpoint{2.936725in}{0.500000in}}%
\pgfpathclose%
\pgfusepath{fill}%
\end{pgfscope}%
\begin{pgfscope}%
\pgfpathrectangle{\pgfqpoint{0.750000in}{0.500000in}}{\pgfqpoint{4.650000in}{3.020000in}}%
\pgfusepath{clip}%
\pgfsetbuttcap%
\pgfsetmiterjoin%
\definecolor{currentfill}{rgb}{0.121569,0.466667,0.705882}%
\pgfsetfillcolor{currentfill}%
\pgfsetlinewidth{0.000000pt}%
\definecolor{currentstroke}{rgb}{0.000000,0.000000,0.000000}%
\pgfsetstrokecolor{currentstroke}%
\pgfsetstrokeopacity{0.000000}%
\pgfsetdash{}{0pt}%
\pgfpathmoveto{\pgfqpoint{2.969647in}{0.500000in}}%
\pgfpathlineto{\pgfqpoint{2.995986in}{0.500000in}}%
\pgfpathlineto{\pgfqpoint{2.995986in}{3.001484in}}%
\pgfpathlineto{\pgfqpoint{2.969647in}{3.001484in}}%
\pgfpathlineto{\pgfqpoint{2.969647in}{0.500000in}}%
\pgfpathclose%
\pgfusepath{fill}%
\end{pgfscope}%
\begin{pgfscope}%
\pgfpathrectangle{\pgfqpoint{0.750000in}{0.500000in}}{\pgfqpoint{4.650000in}{3.020000in}}%
\pgfusepath{clip}%
\pgfsetbuttcap%
\pgfsetmiterjoin%
\definecolor{currentfill}{rgb}{0.121569,0.466667,0.705882}%
\pgfsetfillcolor{currentfill}%
\pgfsetlinewidth{0.000000pt}%
\definecolor{currentstroke}{rgb}{0.000000,0.000000,0.000000}%
\pgfsetstrokecolor{currentstroke}%
\pgfsetstrokeopacity{0.000000}%
\pgfsetdash{}{0pt}%
\pgfpathmoveto{\pgfqpoint{3.002570in}{0.500000in}}%
\pgfpathlineto{\pgfqpoint{3.028908in}{0.500000in}}%
\pgfpathlineto{\pgfqpoint{3.028908in}{3.203217in}}%
\pgfpathlineto{\pgfqpoint{3.002570in}{3.203217in}}%
\pgfpathlineto{\pgfqpoint{3.002570in}{0.500000in}}%
\pgfpathclose%
\pgfusepath{fill}%
\end{pgfscope}%
\begin{pgfscope}%
\pgfpathrectangle{\pgfqpoint{0.750000in}{0.500000in}}{\pgfqpoint{4.650000in}{3.020000in}}%
\pgfusepath{clip}%
\pgfsetbuttcap%
\pgfsetmiterjoin%
\definecolor{currentfill}{rgb}{0.121569,0.466667,0.705882}%
\pgfsetfillcolor{currentfill}%
\pgfsetlinewidth{0.000000pt}%
\definecolor{currentstroke}{rgb}{0.000000,0.000000,0.000000}%
\pgfsetstrokecolor{currentstroke}%
\pgfsetstrokeopacity{0.000000}%
\pgfsetdash{}{0pt}%
\pgfpathmoveto{\pgfqpoint{3.035493in}{0.500000in}}%
\pgfpathlineto{\pgfqpoint{3.061831in}{0.500000in}}%
\pgfpathlineto{\pgfqpoint{3.061831in}{3.331941in}}%
\pgfpathlineto{\pgfqpoint{3.035493in}{3.331941in}}%
\pgfpathlineto{\pgfqpoint{3.035493in}{0.500000in}}%
\pgfpathclose%
\pgfusepath{fill}%
\end{pgfscope}%
\begin{pgfscope}%
\pgfpathrectangle{\pgfqpoint{0.750000in}{0.500000in}}{\pgfqpoint{4.650000in}{3.020000in}}%
\pgfusepath{clip}%
\pgfsetbuttcap%
\pgfsetmiterjoin%
\definecolor{currentfill}{rgb}{0.121569,0.466667,0.705882}%
\pgfsetfillcolor{currentfill}%
\pgfsetlinewidth{0.000000pt}%
\definecolor{currentstroke}{rgb}{0.000000,0.000000,0.000000}%
\pgfsetstrokecolor{currentstroke}%
\pgfsetstrokeopacity{0.000000}%
\pgfsetdash{}{0pt}%
\pgfpathmoveto{\pgfqpoint{3.068415in}{0.500000in}}%
\pgfpathlineto{\pgfqpoint{3.094754in}{0.500000in}}%
\pgfpathlineto{\pgfqpoint{3.094754in}{3.376190in}}%
\pgfpathlineto{\pgfqpoint{3.068415in}{3.376190in}}%
\pgfpathlineto{\pgfqpoint{3.068415in}{0.500000in}}%
\pgfpathclose%
\pgfusepath{fill}%
\end{pgfscope}%
\begin{pgfscope}%
\pgfpathrectangle{\pgfqpoint{0.750000in}{0.500000in}}{\pgfqpoint{4.650000in}{3.020000in}}%
\pgfusepath{clip}%
\pgfsetbuttcap%
\pgfsetmiterjoin%
\definecolor{currentfill}{rgb}{0.121569,0.466667,0.705882}%
\pgfsetfillcolor{currentfill}%
\pgfsetlinewidth{0.000000pt}%
\definecolor{currentstroke}{rgb}{0.000000,0.000000,0.000000}%
\pgfsetstrokecolor{currentstroke}%
\pgfsetstrokeopacity{0.000000}%
\pgfsetdash{}{0pt}%
\pgfpathmoveto{\pgfqpoint{3.101338in}{0.500000in}}%
\pgfpathlineto{\pgfqpoint{3.127676in}{0.500000in}}%
\pgfpathlineto{\pgfqpoint{3.127676in}{3.331941in}}%
\pgfpathlineto{\pgfqpoint{3.101338in}{3.331941in}}%
\pgfpathlineto{\pgfqpoint{3.101338in}{0.500000in}}%
\pgfpathclose%
\pgfusepath{fill}%
\end{pgfscope}%
\begin{pgfscope}%
\pgfpathrectangle{\pgfqpoint{0.750000in}{0.500000in}}{\pgfqpoint{4.650000in}{3.020000in}}%
\pgfusepath{clip}%
\pgfsetbuttcap%
\pgfsetmiterjoin%
\definecolor{currentfill}{rgb}{0.121569,0.466667,0.705882}%
\pgfsetfillcolor{currentfill}%
\pgfsetlinewidth{0.000000pt}%
\definecolor{currentstroke}{rgb}{0.000000,0.000000,0.000000}%
\pgfsetstrokecolor{currentstroke}%
\pgfsetstrokeopacity{0.000000}%
\pgfsetdash{}{0pt}%
\pgfpathmoveto{\pgfqpoint{3.134261in}{0.500000in}}%
\pgfpathlineto{\pgfqpoint{3.160599in}{0.500000in}}%
\pgfpathlineto{\pgfqpoint{3.160599in}{3.203217in}}%
\pgfpathlineto{\pgfqpoint{3.134261in}{3.203217in}}%
\pgfpathlineto{\pgfqpoint{3.134261in}{0.500000in}}%
\pgfpathclose%
\pgfusepath{fill}%
\end{pgfscope}%
\begin{pgfscope}%
\pgfpathrectangle{\pgfqpoint{0.750000in}{0.500000in}}{\pgfqpoint{4.650000in}{3.020000in}}%
\pgfusepath{clip}%
\pgfsetbuttcap%
\pgfsetmiterjoin%
\definecolor{currentfill}{rgb}{0.121569,0.466667,0.705882}%
\pgfsetfillcolor{currentfill}%
\pgfsetlinewidth{0.000000pt}%
\definecolor{currentstroke}{rgb}{0.000000,0.000000,0.000000}%
\pgfsetstrokecolor{currentstroke}%
\pgfsetstrokeopacity{0.000000}%
\pgfsetdash{}{0pt}%
\pgfpathmoveto{\pgfqpoint{3.167184in}{0.500000in}}%
\pgfpathlineto{\pgfqpoint{3.193522in}{0.500000in}}%
\pgfpathlineto{\pgfqpoint{3.193522in}{3.001484in}}%
\pgfpathlineto{\pgfqpoint{3.167184in}{3.001484in}}%
\pgfpathlineto{\pgfqpoint{3.167184in}{0.500000in}}%
\pgfpathclose%
\pgfusepath{fill}%
\end{pgfscope}%
\begin{pgfscope}%
\pgfpathrectangle{\pgfqpoint{0.750000in}{0.500000in}}{\pgfqpoint{4.650000in}{3.020000in}}%
\pgfusepath{clip}%
\pgfsetbuttcap%
\pgfsetmiterjoin%
\definecolor{currentfill}{rgb}{0.121569,0.466667,0.705882}%
\pgfsetfillcolor{currentfill}%
\pgfsetlinewidth{0.000000pt}%
\definecolor{currentstroke}{rgb}{0.000000,0.000000,0.000000}%
\pgfsetstrokecolor{currentstroke}%
\pgfsetstrokeopacity{0.000000}%
\pgfsetdash{}{0pt}%
\pgfpathmoveto{\pgfqpoint{3.200106in}{0.500000in}}%
\pgfpathlineto{\pgfqpoint{3.226444in}{0.500000in}}%
\pgfpathlineto{\pgfqpoint{3.226444in}{2.743978in}}%
\pgfpathlineto{\pgfqpoint{3.200106in}{2.743978in}}%
\pgfpathlineto{\pgfqpoint{3.200106in}{0.500000in}}%
\pgfpathclose%
\pgfusepath{fill}%
\end{pgfscope}%
\begin{pgfscope}%
\pgfpathrectangle{\pgfqpoint{0.750000in}{0.500000in}}{\pgfqpoint{4.650000in}{3.020000in}}%
\pgfusepath{clip}%
\pgfsetbuttcap%
\pgfsetmiterjoin%
\definecolor{currentfill}{rgb}{0.121569,0.466667,0.705882}%
\pgfsetfillcolor{currentfill}%
\pgfsetlinewidth{0.000000pt}%
\definecolor{currentstroke}{rgb}{0.000000,0.000000,0.000000}%
\pgfsetstrokecolor{currentstroke}%
\pgfsetstrokeopacity{0.000000}%
\pgfsetdash{}{0pt}%
\pgfpathmoveto{\pgfqpoint{3.233029in}{0.500000in}}%
\pgfpathlineto{\pgfqpoint{3.259367in}{0.500000in}}%
\pgfpathlineto{\pgfqpoint{3.259367in}{2.451286in}}%
\pgfpathlineto{\pgfqpoint{3.233029in}{2.451286in}}%
\pgfpathlineto{\pgfqpoint{3.233029in}{0.500000in}}%
\pgfpathclose%
\pgfusepath{fill}%
\end{pgfscope}%
\begin{pgfscope}%
\pgfpathrectangle{\pgfqpoint{0.750000in}{0.500000in}}{\pgfqpoint{4.650000in}{3.020000in}}%
\pgfusepath{clip}%
\pgfsetbuttcap%
\pgfsetmiterjoin%
\definecolor{currentfill}{rgb}{0.121569,0.466667,0.705882}%
\pgfsetfillcolor{currentfill}%
\pgfsetlinewidth{0.000000pt}%
\definecolor{currentstroke}{rgb}{0.000000,0.000000,0.000000}%
\pgfsetstrokecolor{currentstroke}%
\pgfsetstrokeopacity{0.000000}%
\pgfsetdash{}{0pt}%
\pgfpathmoveto{\pgfqpoint{3.265952in}{0.500000in}}%
\pgfpathlineto{\pgfqpoint{3.292290in}{0.500000in}}%
\pgfpathlineto{\pgfqpoint{3.292290in}{2.144655in}}%
\pgfpathlineto{\pgfqpoint{3.265952in}{2.144655in}}%
\pgfpathlineto{\pgfqpoint{3.265952in}{0.500000in}}%
\pgfpathclose%
\pgfusepath{fill}%
\end{pgfscope}%
\begin{pgfscope}%
\pgfpathrectangle{\pgfqpoint{0.750000in}{0.500000in}}{\pgfqpoint{4.650000in}{3.020000in}}%
\pgfusepath{clip}%
\pgfsetbuttcap%
\pgfsetmiterjoin%
\definecolor{currentfill}{rgb}{0.121569,0.466667,0.705882}%
\pgfsetfillcolor{currentfill}%
\pgfsetlinewidth{0.000000pt}%
\definecolor{currentstroke}{rgb}{0.000000,0.000000,0.000000}%
\pgfsetstrokecolor{currentstroke}%
\pgfsetstrokeopacity{0.000000}%
\pgfsetdash{}{0pt}%
\pgfpathmoveto{\pgfqpoint{3.298874in}{0.500000in}}%
\pgfpathlineto{\pgfqpoint{3.325212in}{0.500000in}}%
\pgfpathlineto{\pgfqpoint{3.325212in}{1.843521in}}%
\pgfpathlineto{\pgfqpoint{3.298874in}{1.843521in}}%
\pgfpathlineto{\pgfqpoint{3.298874in}{0.500000in}}%
\pgfpathclose%
\pgfusepath{fill}%
\end{pgfscope}%
\begin{pgfscope}%
\pgfpathrectangle{\pgfqpoint{0.750000in}{0.500000in}}{\pgfqpoint{4.650000in}{3.020000in}}%
\pgfusepath{clip}%
\pgfsetbuttcap%
\pgfsetmiterjoin%
\definecolor{currentfill}{rgb}{0.121569,0.466667,0.705882}%
\pgfsetfillcolor{currentfill}%
\pgfsetlinewidth{0.000000pt}%
\definecolor{currentstroke}{rgb}{0.000000,0.000000,0.000000}%
\pgfsetstrokecolor{currentstroke}%
\pgfsetstrokeopacity{0.000000}%
\pgfsetdash{}{0pt}%
\pgfpathmoveto{\pgfqpoint{3.331797in}{0.500000in}}%
\pgfpathlineto{\pgfqpoint{3.358135in}{0.500000in}}%
\pgfpathlineto{\pgfqpoint{3.358135in}{1.563621in}}%
\pgfpathlineto{\pgfqpoint{3.331797in}{1.563621in}}%
\pgfpathlineto{\pgfqpoint{3.331797in}{0.500000in}}%
\pgfpathclose%
\pgfusepath{fill}%
\end{pgfscope}%
\begin{pgfscope}%
\pgfpathrectangle{\pgfqpoint{0.750000in}{0.500000in}}{\pgfqpoint{4.650000in}{3.020000in}}%
\pgfusepath{clip}%
\pgfsetbuttcap%
\pgfsetmiterjoin%
\definecolor{currentfill}{rgb}{0.121569,0.466667,0.705882}%
\pgfsetfillcolor{currentfill}%
\pgfsetlinewidth{0.000000pt}%
\definecolor{currentstroke}{rgb}{0.000000,0.000000,0.000000}%
\pgfsetstrokecolor{currentstroke}%
\pgfsetstrokeopacity{0.000000}%
\pgfsetdash{}{0pt}%
\pgfpathmoveto{\pgfqpoint{3.364720in}{0.500000in}}%
\pgfpathlineto{\pgfqpoint{3.391058in}{0.500000in}}%
\pgfpathlineto{\pgfqpoint{3.391058in}{1.315928in}}%
\pgfpathlineto{\pgfqpoint{3.364720in}{1.315928in}}%
\pgfpathlineto{\pgfqpoint{3.364720in}{0.500000in}}%
\pgfpathclose%
\pgfusepath{fill}%
\end{pgfscope}%
\begin{pgfscope}%
\pgfpathrectangle{\pgfqpoint{0.750000in}{0.500000in}}{\pgfqpoint{4.650000in}{3.020000in}}%
\pgfusepath{clip}%
\pgfsetbuttcap%
\pgfsetmiterjoin%
\definecolor{currentfill}{rgb}{0.121569,0.466667,0.705882}%
\pgfsetfillcolor{currentfill}%
\pgfsetlinewidth{0.000000pt}%
\definecolor{currentstroke}{rgb}{0.000000,0.000000,0.000000}%
\pgfsetstrokecolor{currentstroke}%
\pgfsetstrokeopacity{0.000000}%
\pgfsetdash{}{0pt}%
\pgfpathmoveto{\pgfqpoint{3.397642in}{0.500000in}}%
\pgfpathlineto{\pgfqpoint{3.423980in}{0.500000in}}%
\pgfpathlineto{\pgfqpoint{3.423980in}{1.106433in}}%
\pgfpathlineto{\pgfqpoint{3.397642in}{1.106433in}}%
\pgfpathlineto{\pgfqpoint{3.397642in}{0.500000in}}%
\pgfpathclose%
\pgfusepath{fill}%
\end{pgfscope}%
\begin{pgfscope}%
\pgfpathrectangle{\pgfqpoint{0.750000in}{0.500000in}}{\pgfqpoint{4.650000in}{3.020000in}}%
\pgfusepath{clip}%
\pgfsetbuttcap%
\pgfsetmiterjoin%
\definecolor{currentfill}{rgb}{0.121569,0.466667,0.705882}%
\pgfsetfillcolor{currentfill}%
\pgfsetlinewidth{0.000000pt}%
\definecolor{currentstroke}{rgb}{0.000000,0.000000,0.000000}%
\pgfsetstrokecolor{currentstroke}%
\pgfsetstrokeopacity{0.000000}%
\pgfsetdash{}{0pt}%
\pgfpathmoveto{\pgfqpoint{3.430565in}{0.500000in}}%
\pgfpathlineto{\pgfqpoint{3.456903in}{0.500000in}}%
\pgfpathlineto{\pgfqpoint{3.456903in}{0.936632in}}%
\pgfpathlineto{\pgfqpoint{3.430565in}{0.936632in}}%
\pgfpathlineto{\pgfqpoint{3.430565in}{0.500000in}}%
\pgfpathclose%
\pgfusepath{fill}%
\end{pgfscope}%
\begin{pgfscope}%
\pgfpathrectangle{\pgfqpoint{0.750000in}{0.500000in}}{\pgfqpoint{4.650000in}{3.020000in}}%
\pgfusepath{clip}%
\pgfsetbuttcap%
\pgfsetmiterjoin%
\definecolor{currentfill}{rgb}{0.121569,0.466667,0.705882}%
\pgfsetfillcolor{currentfill}%
\pgfsetlinewidth{0.000000pt}%
\definecolor{currentstroke}{rgb}{0.000000,0.000000,0.000000}%
\pgfsetstrokecolor{currentstroke}%
\pgfsetstrokeopacity{0.000000}%
\pgfsetdash{}{0pt}%
\pgfpathmoveto{\pgfqpoint{3.463488in}{0.500000in}}%
\pgfpathlineto{\pgfqpoint{3.489826in}{0.500000in}}%
\pgfpathlineto{\pgfqpoint{3.489826in}{0.804493in}}%
\pgfpathlineto{\pgfqpoint{3.463488in}{0.804493in}}%
\pgfpathlineto{\pgfqpoint{3.463488in}{0.500000in}}%
\pgfpathclose%
\pgfusepath{fill}%
\end{pgfscope}%
\begin{pgfscope}%
\pgfpathrectangle{\pgfqpoint{0.750000in}{0.500000in}}{\pgfqpoint{4.650000in}{3.020000in}}%
\pgfusepath{clip}%
\pgfsetbuttcap%
\pgfsetmiterjoin%
\definecolor{currentfill}{rgb}{0.121569,0.466667,0.705882}%
\pgfsetfillcolor{currentfill}%
\pgfsetlinewidth{0.000000pt}%
\definecolor{currentstroke}{rgb}{0.000000,0.000000,0.000000}%
\pgfsetstrokecolor{currentstroke}%
\pgfsetstrokeopacity{0.000000}%
\pgfsetdash{}{0pt}%
\pgfpathmoveto{\pgfqpoint{3.496410in}{0.500000in}}%
\pgfpathlineto{\pgfqpoint{3.522749in}{0.500000in}}%
\pgfpathlineto{\pgfqpoint{3.522749in}{0.705632in}}%
\pgfpathlineto{\pgfqpoint{3.496410in}{0.705632in}}%
\pgfpathlineto{\pgfqpoint{3.496410in}{0.500000in}}%
\pgfpathclose%
\pgfusepath{fill}%
\end{pgfscope}%
\begin{pgfscope}%
\pgfpathrectangle{\pgfqpoint{0.750000in}{0.500000in}}{\pgfqpoint{4.650000in}{3.020000in}}%
\pgfusepath{clip}%
\pgfsetbuttcap%
\pgfsetmiterjoin%
\definecolor{currentfill}{rgb}{0.121569,0.466667,0.705882}%
\pgfsetfillcolor{currentfill}%
\pgfsetlinewidth{0.000000pt}%
\definecolor{currentstroke}{rgb}{0.000000,0.000000,0.000000}%
\pgfsetstrokecolor{currentstroke}%
\pgfsetstrokeopacity{0.000000}%
\pgfsetdash{}{0pt}%
\pgfpathmoveto{\pgfqpoint{3.529333in}{0.500000in}}%
\pgfpathlineto{\pgfqpoint{3.555671in}{0.500000in}}%
\pgfpathlineto{\pgfqpoint{3.555671in}{0.634452in}}%
\pgfpathlineto{\pgfqpoint{3.529333in}{0.634452in}}%
\pgfpathlineto{\pgfqpoint{3.529333in}{0.500000in}}%
\pgfpathclose%
\pgfusepath{fill}%
\end{pgfscope}%
\begin{pgfscope}%
\pgfpathrectangle{\pgfqpoint{0.750000in}{0.500000in}}{\pgfqpoint{4.650000in}{3.020000in}}%
\pgfusepath{clip}%
\pgfsetbuttcap%
\pgfsetmiterjoin%
\definecolor{currentfill}{rgb}{0.121569,0.466667,0.705882}%
\pgfsetfillcolor{currentfill}%
\pgfsetlinewidth{0.000000pt}%
\definecolor{currentstroke}{rgb}{0.000000,0.000000,0.000000}%
\pgfsetstrokecolor{currentstroke}%
\pgfsetstrokeopacity{0.000000}%
\pgfsetdash{}{0pt}%
\pgfpathmoveto{\pgfqpoint{3.562256in}{0.500000in}}%
\pgfpathlineto{\pgfqpoint{3.588594in}{0.500000in}}%
\pgfpathlineto{\pgfqpoint{3.588594in}{0.585096in}}%
\pgfpathlineto{\pgfqpoint{3.562256in}{0.585096in}}%
\pgfpathlineto{\pgfqpoint{3.562256in}{0.500000in}}%
\pgfpathclose%
\pgfusepath{fill}%
\end{pgfscope}%
\begin{pgfscope}%
\pgfpathrectangle{\pgfqpoint{0.750000in}{0.500000in}}{\pgfqpoint{4.650000in}{3.020000in}}%
\pgfusepath{clip}%
\pgfsetbuttcap%
\pgfsetmiterjoin%
\definecolor{currentfill}{rgb}{0.121569,0.466667,0.705882}%
\pgfsetfillcolor{currentfill}%
\pgfsetlinewidth{0.000000pt}%
\definecolor{currentstroke}{rgb}{0.000000,0.000000,0.000000}%
\pgfsetstrokecolor{currentstroke}%
\pgfsetstrokeopacity{0.000000}%
\pgfsetdash{}{0pt}%
\pgfpathmoveto{\pgfqpoint{3.595178in}{0.500000in}}%
\pgfpathlineto{\pgfqpoint{3.621517in}{0.500000in}}%
\pgfpathlineto{\pgfqpoint{3.621517in}{0.552121in}}%
\pgfpathlineto{\pgfqpoint{3.595178in}{0.552121in}}%
\pgfpathlineto{\pgfqpoint{3.595178in}{0.500000in}}%
\pgfpathclose%
\pgfusepath{fill}%
\end{pgfscope}%
\begin{pgfscope}%
\pgfpathrectangle{\pgfqpoint{0.750000in}{0.500000in}}{\pgfqpoint{4.650000in}{3.020000in}}%
\pgfusepath{clip}%
\pgfsetbuttcap%
\pgfsetmiterjoin%
\definecolor{currentfill}{rgb}{0.121569,0.466667,0.705882}%
\pgfsetfillcolor{currentfill}%
\pgfsetlinewidth{0.000000pt}%
\definecolor{currentstroke}{rgb}{0.000000,0.000000,0.000000}%
\pgfsetstrokecolor{currentstroke}%
\pgfsetstrokeopacity{0.000000}%
\pgfsetdash{}{0pt}%
\pgfpathmoveto{\pgfqpoint{3.628101in}{0.500000in}}%
\pgfpathlineto{\pgfqpoint{3.654439in}{0.500000in}}%
\pgfpathlineto{\pgfqpoint{3.654439in}{0.530887in}}%
\pgfpathlineto{\pgfqpoint{3.628101in}{0.530887in}}%
\pgfpathlineto{\pgfqpoint{3.628101in}{0.500000in}}%
\pgfpathclose%
\pgfusepath{fill}%
\end{pgfscope}%
\begin{pgfscope}%
\pgfpathrectangle{\pgfqpoint{0.750000in}{0.500000in}}{\pgfqpoint{4.650000in}{3.020000in}}%
\pgfusepath{clip}%
\pgfsetbuttcap%
\pgfsetmiterjoin%
\definecolor{currentfill}{rgb}{0.121569,0.466667,0.705882}%
\pgfsetfillcolor{currentfill}%
\pgfsetlinewidth{0.000000pt}%
\definecolor{currentstroke}{rgb}{0.000000,0.000000,0.000000}%
\pgfsetstrokecolor{currentstroke}%
\pgfsetstrokeopacity{0.000000}%
\pgfsetdash{}{0pt}%
\pgfpathmoveto{\pgfqpoint{3.661024in}{0.500000in}}%
\pgfpathlineto{\pgfqpoint{3.687362in}{0.500000in}}%
\pgfpathlineto{\pgfqpoint{3.687362in}{0.517703in}}%
\pgfpathlineto{\pgfqpoint{3.661024in}{0.517703in}}%
\pgfpathlineto{\pgfqpoint{3.661024in}{0.500000in}}%
\pgfpathclose%
\pgfusepath{fill}%
\end{pgfscope}%
\begin{pgfscope}%
\pgfpathrectangle{\pgfqpoint{0.750000in}{0.500000in}}{\pgfqpoint{4.650000in}{3.020000in}}%
\pgfusepath{clip}%
\pgfsetbuttcap%
\pgfsetmiterjoin%
\definecolor{currentfill}{rgb}{0.121569,0.466667,0.705882}%
\pgfsetfillcolor{currentfill}%
\pgfsetlinewidth{0.000000pt}%
\definecolor{currentstroke}{rgb}{0.000000,0.000000,0.000000}%
\pgfsetstrokecolor{currentstroke}%
\pgfsetstrokeopacity{0.000000}%
\pgfsetdash{}{0pt}%
\pgfpathmoveto{\pgfqpoint{3.693946in}{0.500000in}}%
\pgfpathlineto{\pgfqpoint{3.720285in}{0.500000in}}%
\pgfpathlineto{\pgfqpoint{3.720285in}{0.509811in}}%
\pgfpathlineto{\pgfqpoint{3.693946in}{0.509811in}}%
\pgfpathlineto{\pgfqpoint{3.693946in}{0.500000in}}%
\pgfpathclose%
\pgfusepath{fill}%
\end{pgfscope}%
\begin{pgfscope}%
\pgfpathrectangle{\pgfqpoint{0.750000in}{0.500000in}}{\pgfqpoint{4.650000in}{3.020000in}}%
\pgfusepath{clip}%
\pgfsetbuttcap%
\pgfsetmiterjoin%
\definecolor{currentfill}{rgb}{0.121569,0.466667,0.705882}%
\pgfsetfillcolor{currentfill}%
\pgfsetlinewidth{0.000000pt}%
\definecolor{currentstroke}{rgb}{0.000000,0.000000,0.000000}%
\pgfsetstrokecolor{currentstroke}%
\pgfsetstrokeopacity{0.000000}%
\pgfsetdash{}{0pt}%
\pgfpathmoveto{\pgfqpoint{3.726869in}{0.500000in}}%
\pgfpathlineto{\pgfqpoint{3.753207in}{0.500000in}}%
\pgfpathlineto{\pgfqpoint{3.753207in}{0.505256in}}%
\pgfpathlineto{\pgfqpoint{3.726869in}{0.505256in}}%
\pgfpathlineto{\pgfqpoint{3.726869in}{0.500000in}}%
\pgfpathclose%
\pgfusepath{fill}%
\end{pgfscope}%
\begin{pgfscope}%
\pgfpathrectangle{\pgfqpoint{0.750000in}{0.500000in}}{\pgfqpoint{4.650000in}{3.020000in}}%
\pgfusepath{clip}%
\pgfsetbuttcap%
\pgfsetmiterjoin%
\definecolor{currentfill}{rgb}{0.121569,0.466667,0.705882}%
\pgfsetfillcolor{currentfill}%
\pgfsetlinewidth{0.000000pt}%
\definecolor{currentstroke}{rgb}{0.000000,0.000000,0.000000}%
\pgfsetstrokecolor{currentstroke}%
\pgfsetstrokeopacity{0.000000}%
\pgfsetdash{}{0pt}%
\pgfpathmoveto{\pgfqpoint{3.759792in}{0.500000in}}%
\pgfpathlineto{\pgfqpoint{3.786130in}{0.500000in}}%
\pgfpathlineto{\pgfqpoint{3.786130in}{0.502721in}}%
\pgfpathlineto{\pgfqpoint{3.759792in}{0.502721in}}%
\pgfpathlineto{\pgfqpoint{3.759792in}{0.500000in}}%
\pgfpathclose%
\pgfusepath{fill}%
\end{pgfscope}%
\begin{pgfscope}%
\pgfpathrectangle{\pgfqpoint{0.750000in}{0.500000in}}{\pgfqpoint{4.650000in}{3.020000in}}%
\pgfusepath{clip}%
\pgfsetbuttcap%
\pgfsetmiterjoin%
\definecolor{currentfill}{rgb}{0.121569,0.466667,0.705882}%
\pgfsetfillcolor{currentfill}%
\pgfsetlinewidth{0.000000pt}%
\definecolor{currentstroke}{rgb}{0.000000,0.000000,0.000000}%
\pgfsetstrokecolor{currentstroke}%
\pgfsetstrokeopacity{0.000000}%
\pgfsetdash{}{0pt}%
\pgfpathmoveto{\pgfqpoint{3.792715in}{0.500000in}}%
\pgfpathlineto{\pgfqpoint{3.819053in}{0.500000in}}%
\pgfpathlineto{\pgfqpoint{3.819053in}{0.501360in}}%
\pgfpathlineto{\pgfqpoint{3.792715in}{0.501360in}}%
\pgfpathlineto{\pgfqpoint{3.792715in}{0.500000in}}%
\pgfpathclose%
\pgfusepath{fill}%
\end{pgfscope}%
\begin{pgfscope}%
\pgfpathrectangle{\pgfqpoint{0.750000in}{0.500000in}}{\pgfqpoint{4.650000in}{3.020000in}}%
\pgfusepath{clip}%
\pgfsetbuttcap%
\pgfsetmiterjoin%
\definecolor{currentfill}{rgb}{0.121569,0.466667,0.705882}%
\pgfsetfillcolor{currentfill}%
\pgfsetlinewidth{0.000000pt}%
\definecolor{currentstroke}{rgb}{0.000000,0.000000,0.000000}%
\pgfsetstrokecolor{currentstroke}%
\pgfsetstrokeopacity{0.000000}%
\pgfsetdash{}{0pt}%
\pgfpathmoveto{\pgfqpoint{3.825637in}{0.500000in}}%
\pgfpathlineto{\pgfqpoint{3.851975in}{0.500000in}}%
\pgfpathlineto{\pgfqpoint{3.851975in}{0.500657in}}%
\pgfpathlineto{\pgfqpoint{3.825637in}{0.500657in}}%
\pgfpathlineto{\pgfqpoint{3.825637in}{0.500000in}}%
\pgfpathclose%
\pgfusepath{fill}%
\end{pgfscope}%
\begin{pgfscope}%
\pgfpathrectangle{\pgfqpoint{0.750000in}{0.500000in}}{\pgfqpoint{4.650000in}{3.020000in}}%
\pgfusepath{clip}%
\pgfsetbuttcap%
\pgfsetmiterjoin%
\definecolor{currentfill}{rgb}{0.121569,0.466667,0.705882}%
\pgfsetfillcolor{currentfill}%
\pgfsetlinewidth{0.000000pt}%
\definecolor{currentstroke}{rgb}{0.000000,0.000000,0.000000}%
\pgfsetstrokecolor{currentstroke}%
\pgfsetstrokeopacity{0.000000}%
\pgfsetdash{}{0pt}%
\pgfpathmoveto{\pgfqpoint{3.858560in}{0.500000in}}%
\pgfpathlineto{\pgfqpoint{3.884898in}{0.500000in}}%
\pgfpathlineto{\pgfqpoint{3.884898in}{0.500306in}}%
\pgfpathlineto{\pgfqpoint{3.858560in}{0.500306in}}%
\pgfpathlineto{\pgfqpoint{3.858560in}{0.500000in}}%
\pgfpathclose%
\pgfusepath{fill}%
\end{pgfscope}%
\begin{pgfscope}%
\pgfpathrectangle{\pgfqpoint{0.750000in}{0.500000in}}{\pgfqpoint{4.650000in}{3.020000in}}%
\pgfusepath{clip}%
\pgfsetbuttcap%
\pgfsetmiterjoin%
\definecolor{currentfill}{rgb}{0.121569,0.466667,0.705882}%
\pgfsetfillcolor{currentfill}%
\pgfsetlinewidth{0.000000pt}%
\definecolor{currentstroke}{rgb}{0.000000,0.000000,0.000000}%
\pgfsetstrokecolor{currentstroke}%
\pgfsetstrokeopacity{0.000000}%
\pgfsetdash{}{0pt}%
\pgfpathmoveto{\pgfqpoint{3.891483in}{0.500000in}}%
\pgfpathlineto{\pgfqpoint{3.917821in}{0.500000in}}%
\pgfpathlineto{\pgfqpoint{3.917821in}{0.500138in}}%
\pgfpathlineto{\pgfqpoint{3.891483in}{0.500138in}}%
\pgfpathlineto{\pgfqpoint{3.891483in}{0.500000in}}%
\pgfpathclose%
\pgfusepath{fill}%
\end{pgfscope}%
\begin{pgfscope}%
\pgfpathrectangle{\pgfqpoint{0.750000in}{0.500000in}}{\pgfqpoint{4.650000in}{3.020000in}}%
\pgfusepath{clip}%
\pgfsetbuttcap%
\pgfsetmiterjoin%
\definecolor{currentfill}{rgb}{0.121569,0.466667,0.705882}%
\pgfsetfillcolor{currentfill}%
\pgfsetlinewidth{0.000000pt}%
\definecolor{currentstroke}{rgb}{0.000000,0.000000,0.000000}%
\pgfsetstrokecolor{currentstroke}%
\pgfsetstrokeopacity{0.000000}%
\pgfsetdash{}{0pt}%
\pgfpathmoveto{\pgfqpoint{3.924405in}{0.500000in}}%
\pgfpathlineto{\pgfqpoint{3.950743in}{0.500000in}}%
\pgfpathlineto{\pgfqpoint{3.950743in}{0.500060in}}%
\pgfpathlineto{\pgfqpoint{3.924405in}{0.500060in}}%
\pgfpathlineto{\pgfqpoint{3.924405in}{0.500000in}}%
\pgfpathclose%
\pgfusepath{fill}%
\end{pgfscope}%
\begin{pgfscope}%
\pgfpathrectangle{\pgfqpoint{0.750000in}{0.500000in}}{\pgfqpoint{4.650000in}{3.020000in}}%
\pgfusepath{clip}%
\pgfsetbuttcap%
\pgfsetmiterjoin%
\definecolor{currentfill}{rgb}{0.121569,0.466667,0.705882}%
\pgfsetfillcolor{currentfill}%
\pgfsetlinewidth{0.000000pt}%
\definecolor{currentstroke}{rgb}{0.000000,0.000000,0.000000}%
\pgfsetstrokecolor{currentstroke}%
\pgfsetstrokeopacity{0.000000}%
\pgfsetdash{}{0pt}%
\pgfpathmoveto{\pgfqpoint{3.957328in}{0.500000in}}%
\pgfpathlineto{\pgfqpoint{3.983666in}{0.500000in}}%
\pgfpathlineto{\pgfqpoint{3.983666in}{0.500025in}}%
\pgfpathlineto{\pgfqpoint{3.957328in}{0.500025in}}%
\pgfpathlineto{\pgfqpoint{3.957328in}{0.500000in}}%
\pgfpathclose%
\pgfusepath{fill}%
\end{pgfscope}%
\begin{pgfscope}%
\pgfpathrectangle{\pgfqpoint{0.750000in}{0.500000in}}{\pgfqpoint{4.650000in}{3.020000in}}%
\pgfusepath{clip}%
\pgfsetbuttcap%
\pgfsetmiterjoin%
\definecolor{currentfill}{rgb}{0.121569,0.466667,0.705882}%
\pgfsetfillcolor{currentfill}%
\pgfsetlinewidth{0.000000pt}%
\definecolor{currentstroke}{rgb}{0.000000,0.000000,0.000000}%
\pgfsetstrokecolor{currentstroke}%
\pgfsetstrokeopacity{0.000000}%
\pgfsetdash{}{0pt}%
\pgfpathmoveto{\pgfqpoint{3.990251in}{0.500000in}}%
\pgfpathlineto{\pgfqpoint{4.016589in}{0.500000in}}%
\pgfpathlineto{\pgfqpoint{4.016589in}{0.500010in}}%
\pgfpathlineto{\pgfqpoint{3.990251in}{0.500010in}}%
\pgfpathlineto{\pgfqpoint{3.990251in}{0.500000in}}%
\pgfpathclose%
\pgfusepath{fill}%
\end{pgfscope}%
\begin{pgfscope}%
\pgfpathrectangle{\pgfqpoint{0.750000in}{0.500000in}}{\pgfqpoint{4.650000in}{3.020000in}}%
\pgfusepath{clip}%
\pgfsetbuttcap%
\pgfsetmiterjoin%
\definecolor{currentfill}{rgb}{0.121569,0.466667,0.705882}%
\pgfsetfillcolor{currentfill}%
\pgfsetlinewidth{0.000000pt}%
\definecolor{currentstroke}{rgb}{0.000000,0.000000,0.000000}%
\pgfsetstrokecolor{currentstroke}%
\pgfsetstrokeopacity{0.000000}%
\pgfsetdash{}{0pt}%
\pgfpathmoveto{\pgfqpoint{4.023173in}{0.500000in}}%
\pgfpathlineto{\pgfqpoint{4.049511in}{0.500000in}}%
\pgfpathlineto{\pgfqpoint{4.049511in}{0.500004in}}%
\pgfpathlineto{\pgfqpoint{4.023173in}{0.500004in}}%
\pgfpathlineto{\pgfqpoint{4.023173in}{0.500000in}}%
\pgfpathclose%
\pgfusepath{fill}%
\end{pgfscope}%
\begin{pgfscope}%
\pgfpathrectangle{\pgfqpoint{0.750000in}{0.500000in}}{\pgfqpoint{4.650000in}{3.020000in}}%
\pgfusepath{clip}%
\pgfsetbuttcap%
\pgfsetmiterjoin%
\definecolor{currentfill}{rgb}{0.121569,0.466667,0.705882}%
\pgfsetfillcolor{currentfill}%
\pgfsetlinewidth{0.000000pt}%
\definecolor{currentstroke}{rgb}{0.000000,0.000000,0.000000}%
\pgfsetstrokecolor{currentstroke}%
\pgfsetstrokeopacity{0.000000}%
\pgfsetdash{}{0pt}%
\pgfpathmoveto{\pgfqpoint{4.056096in}{0.500000in}}%
\pgfpathlineto{\pgfqpoint{4.082434in}{0.500000in}}%
\pgfpathlineto{\pgfqpoint{4.082434in}{0.500001in}}%
\pgfpathlineto{\pgfqpoint{4.056096in}{0.500001in}}%
\pgfpathlineto{\pgfqpoint{4.056096in}{0.500000in}}%
\pgfpathclose%
\pgfusepath{fill}%
\end{pgfscope}%
\begin{pgfscope}%
\pgfpathrectangle{\pgfqpoint{0.750000in}{0.500000in}}{\pgfqpoint{4.650000in}{3.020000in}}%
\pgfusepath{clip}%
\pgfsetbuttcap%
\pgfsetmiterjoin%
\definecolor{currentfill}{rgb}{0.121569,0.466667,0.705882}%
\pgfsetfillcolor{currentfill}%
\pgfsetlinewidth{0.000000pt}%
\definecolor{currentstroke}{rgb}{0.000000,0.000000,0.000000}%
\pgfsetstrokecolor{currentstroke}%
\pgfsetstrokeopacity{0.000000}%
\pgfsetdash{}{0pt}%
\pgfpathmoveto{\pgfqpoint{4.089019in}{0.500000in}}%
\pgfpathlineto{\pgfqpoint{4.115357in}{0.500000in}}%
\pgfpathlineto{\pgfqpoint{4.115357in}{0.500001in}}%
\pgfpathlineto{\pgfqpoint{4.089019in}{0.500001in}}%
\pgfpathlineto{\pgfqpoint{4.089019in}{0.500000in}}%
\pgfpathclose%
\pgfusepath{fill}%
\end{pgfscope}%
\begin{pgfscope}%
\pgfpathrectangle{\pgfqpoint{0.750000in}{0.500000in}}{\pgfqpoint{4.650000in}{3.020000in}}%
\pgfusepath{clip}%
\pgfsetbuttcap%
\pgfsetmiterjoin%
\definecolor{currentfill}{rgb}{0.121569,0.466667,0.705882}%
\pgfsetfillcolor{currentfill}%
\pgfsetlinewidth{0.000000pt}%
\definecolor{currentstroke}{rgb}{0.000000,0.000000,0.000000}%
\pgfsetstrokecolor{currentstroke}%
\pgfsetstrokeopacity{0.000000}%
\pgfsetdash{}{0pt}%
\pgfpathmoveto{\pgfqpoint{4.121941in}{0.500000in}}%
\pgfpathlineto{\pgfqpoint{4.148280in}{0.500000in}}%
\pgfpathlineto{\pgfqpoint{4.148280in}{0.500000in}}%
\pgfpathlineto{\pgfqpoint{4.121941in}{0.500000in}}%
\pgfpathlineto{\pgfqpoint{4.121941in}{0.500000in}}%
\pgfpathclose%
\pgfusepath{fill}%
\end{pgfscope}%
\begin{pgfscope}%
\pgfpathrectangle{\pgfqpoint{0.750000in}{0.500000in}}{\pgfqpoint{4.650000in}{3.020000in}}%
\pgfusepath{clip}%
\pgfsetbuttcap%
\pgfsetmiterjoin%
\definecolor{currentfill}{rgb}{0.121569,0.466667,0.705882}%
\pgfsetfillcolor{currentfill}%
\pgfsetlinewidth{0.000000pt}%
\definecolor{currentstroke}{rgb}{0.000000,0.000000,0.000000}%
\pgfsetstrokecolor{currentstroke}%
\pgfsetstrokeopacity{0.000000}%
\pgfsetdash{}{0pt}%
\pgfpathmoveto{\pgfqpoint{4.154864in}{0.500000in}}%
\pgfpathlineto{\pgfqpoint{4.181202in}{0.500000in}}%
\pgfpathlineto{\pgfqpoint{4.181202in}{0.500000in}}%
\pgfpathlineto{\pgfqpoint{4.154864in}{0.500000in}}%
\pgfpathlineto{\pgfqpoint{4.154864in}{0.500000in}}%
\pgfpathclose%
\pgfusepath{fill}%
\end{pgfscope}%
\begin{pgfscope}%
\pgfpathrectangle{\pgfqpoint{0.750000in}{0.500000in}}{\pgfqpoint{4.650000in}{3.020000in}}%
\pgfusepath{clip}%
\pgfsetbuttcap%
\pgfsetmiterjoin%
\definecolor{currentfill}{rgb}{0.121569,0.466667,0.705882}%
\pgfsetfillcolor{currentfill}%
\pgfsetlinewidth{0.000000pt}%
\definecolor{currentstroke}{rgb}{0.000000,0.000000,0.000000}%
\pgfsetstrokecolor{currentstroke}%
\pgfsetstrokeopacity{0.000000}%
\pgfsetdash{}{0pt}%
\pgfpathmoveto{\pgfqpoint{4.187787in}{0.500000in}}%
\pgfpathlineto{\pgfqpoint{4.214125in}{0.500000in}}%
\pgfpathlineto{\pgfqpoint{4.214125in}{0.500000in}}%
\pgfpathlineto{\pgfqpoint{4.187787in}{0.500000in}}%
\pgfpathlineto{\pgfqpoint{4.187787in}{0.500000in}}%
\pgfpathclose%
\pgfusepath{fill}%
\end{pgfscope}%
\begin{pgfscope}%
\pgfpathrectangle{\pgfqpoint{0.750000in}{0.500000in}}{\pgfqpoint{4.650000in}{3.020000in}}%
\pgfusepath{clip}%
\pgfsetbuttcap%
\pgfsetmiterjoin%
\definecolor{currentfill}{rgb}{0.121569,0.466667,0.705882}%
\pgfsetfillcolor{currentfill}%
\pgfsetlinewidth{0.000000pt}%
\definecolor{currentstroke}{rgb}{0.000000,0.000000,0.000000}%
\pgfsetstrokecolor{currentstroke}%
\pgfsetstrokeopacity{0.000000}%
\pgfsetdash{}{0pt}%
\pgfpathmoveto{\pgfqpoint{4.220709in}{0.500000in}}%
\pgfpathlineto{\pgfqpoint{4.247048in}{0.500000in}}%
\pgfpathlineto{\pgfqpoint{4.247048in}{0.500000in}}%
\pgfpathlineto{\pgfqpoint{4.220709in}{0.500000in}}%
\pgfpathlineto{\pgfqpoint{4.220709in}{0.500000in}}%
\pgfpathclose%
\pgfusepath{fill}%
\end{pgfscope}%
\begin{pgfscope}%
\pgfpathrectangle{\pgfqpoint{0.750000in}{0.500000in}}{\pgfqpoint{4.650000in}{3.020000in}}%
\pgfusepath{clip}%
\pgfsetbuttcap%
\pgfsetmiterjoin%
\definecolor{currentfill}{rgb}{0.121569,0.466667,0.705882}%
\pgfsetfillcolor{currentfill}%
\pgfsetlinewidth{0.000000pt}%
\definecolor{currentstroke}{rgb}{0.000000,0.000000,0.000000}%
\pgfsetstrokecolor{currentstroke}%
\pgfsetstrokeopacity{0.000000}%
\pgfsetdash{}{0pt}%
\pgfpathmoveto{\pgfqpoint{4.253632in}{0.500000in}}%
\pgfpathlineto{\pgfqpoint{4.279970in}{0.500000in}}%
\pgfpathlineto{\pgfqpoint{4.279970in}{0.500000in}}%
\pgfpathlineto{\pgfqpoint{4.253632in}{0.500000in}}%
\pgfpathlineto{\pgfqpoint{4.253632in}{0.500000in}}%
\pgfpathclose%
\pgfusepath{fill}%
\end{pgfscope}%
\begin{pgfscope}%
\pgfpathrectangle{\pgfqpoint{0.750000in}{0.500000in}}{\pgfqpoint{4.650000in}{3.020000in}}%
\pgfusepath{clip}%
\pgfsetbuttcap%
\pgfsetmiterjoin%
\definecolor{currentfill}{rgb}{0.121569,0.466667,0.705882}%
\pgfsetfillcolor{currentfill}%
\pgfsetlinewidth{0.000000pt}%
\definecolor{currentstroke}{rgb}{0.000000,0.000000,0.000000}%
\pgfsetstrokecolor{currentstroke}%
\pgfsetstrokeopacity{0.000000}%
\pgfsetdash{}{0pt}%
\pgfpathmoveto{\pgfqpoint{4.286555in}{0.500000in}}%
\pgfpathlineto{\pgfqpoint{4.312893in}{0.500000in}}%
\pgfpathlineto{\pgfqpoint{4.312893in}{0.500000in}}%
\pgfpathlineto{\pgfqpoint{4.286555in}{0.500000in}}%
\pgfpathlineto{\pgfqpoint{4.286555in}{0.500000in}}%
\pgfpathclose%
\pgfusepath{fill}%
\end{pgfscope}%
\begin{pgfscope}%
\pgfpathrectangle{\pgfqpoint{0.750000in}{0.500000in}}{\pgfqpoint{4.650000in}{3.020000in}}%
\pgfusepath{clip}%
\pgfsetbuttcap%
\pgfsetmiterjoin%
\definecolor{currentfill}{rgb}{0.121569,0.466667,0.705882}%
\pgfsetfillcolor{currentfill}%
\pgfsetlinewidth{0.000000pt}%
\definecolor{currentstroke}{rgb}{0.000000,0.000000,0.000000}%
\pgfsetstrokecolor{currentstroke}%
\pgfsetstrokeopacity{0.000000}%
\pgfsetdash{}{0pt}%
\pgfpathmoveto{\pgfqpoint{4.319477in}{0.500000in}}%
\pgfpathlineto{\pgfqpoint{4.345816in}{0.500000in}}%
\pgfpathlineto{\pgfqpoint{4.345816in}{0.500000in}}%
\pgfpathlineto{\pgfqpoint{4.319477in}{0.500000in}}%
\pgfpathlineto{\pgfqpoint{4.319477in}{0.500000in}}%
\pgfpathclose%
\pgfusepath{fill}%
\end{pgfscope}%
\begin{pgfscope}%
\pgfpathrectangle{\pgfqpoint{0.750000in}{0.500000in}}{\pgfqpoint{4.650000in}{3.020000in}}%
\pgfusepath{clip}%
\pgfsetbuttcap%
\pgfsetmiterjoin%
\definecolor{currentfill}{rgb}{0.121569,0.466667,0.705882}%
\pgfsetfillcolor{currentfill}%
\pgfsetlinewidth{0.000000pt}%
\definecolor{currentstroke}{rgb}{0.000000,0.000000,0.000000}%
\pgfsetstrokecolor{currentstroke}%
\pgfsetstrokeopacity{0.000000}%
\pgfsetdash{}{0pt}%
\pgfpathmoveto{\pgfqpoint{4.352400in}{0.500000in}}%
\pgfpathlineto{\pgfqpoint{4.378738in}{0.500000in}}%
\pgfpathlineto{\pgfqpoint{4.378738in}{0.500000in}}%
\pgfpathlineto{\pgfqpoint{4.352400in}{0.500000in}}%
\pgfpathlineto{\pgfqpoint{4.352400in}{0.500000in}}%
\pgfpathclose%
\pgfusepath{fill}%
\end{pgfscope}%
\begin{pgfscope}%
\pgfpathrectangle{\pgfqpoint{0.750000in}{0.500000in}}{\pgfqpoint{4.650000in}{3.020000in}}%
\pgfusepath{clip}%
\pgfsetbuttcap%
\pgfsetmiterjoin%
\definecolor{currentfill}{rgb}{0.121569,0.466667,0.705882}%
\pgfsetfillcolor{currentfill}%
\pgfsetlinewidth{0.000000pt}%
\definecolor{currentstroke}{rgb}{0.000000,0.000000,0.000000}%
\pgfsetstrokecolor{currentstroke}%
\pgfsetstrokeopacity{0.000000}%
\pgfsetdash{}{0pt}%
\pgfpathmoveto{\pgfqpoint{4.385323in}{0.500000in}}%
\pgfpathlineto{\pgfqpoint{4.411661in}{0.500000in}}%
\pgfpathlineto{\pgfqpoint{4.411661in}{0.500000in}}%
\pgfpathlineto{\pgfqpoint{4.385323in}{0.500000in}}%
\pgfpathlineto{\pgfqpoint{4.385323in}{0.500000in}}%
\pgfpathclose%
\pgfusepath{fill}%
\end{pgfscope}%
\begin{pgfscope}%
\pgfpathrectangle{\pgfqpoint{0.750000in}{0.500000in}}{\pgfqpoint{4.650000in}{3.020000in}}%
\pgfusepath{clip}%
\pgfsetbuttcap%
\pgfsetmiterjoin%
\definecolor{currentfill}{rgb}{0.121569,0.466667,0.705882}%
\pgfsetfillcolor{currentfill}%
\pgfsetlinewidth{0.000000pt}%
\definecolor{currentstroke}{rgb}{0.000000,0.000000,0.000000}%
\pgfsetstrokecolor{currentstroke}%
\pgfsetstrokeopacity{0.000000}%
\pgfsetdash{}{0pt}%
\pgfpathmoveto{\pgfqpoint{4.418246in}{0.500000in}}%
\pgfpathlineto{\pgfqpoint{4.444584in}{0.500000in}}%
\pgfpathlineto{\pgfqpoint{4.444584in}{0.500000in}}%
\pgfpathlineto{\pgfqpoint{4.418246in}{0.500000in}}%
\pgfpathlineto{\pgfqpoint{4.418246in}{0.500000in}}%
\pgfpathclose%
\pgfusepath{fill}%
\end{pgfscope}%
\begin{pgfscope}%
\pgfpathrectangle{\pgfqpoint{0.750000in}{0.500000in}}{\pgfqpoint{4.650000in}{3.020000in}}%
\pgfusepath{clip}%
\pgfsetbuttcap%
\pgfsetmiterjoin%
\definecolor{currentfill}{rgb}{0.121569,0.466667,0.705882}%
\pgfsetfillcolor{currentfill}%
\pgfsetlinewidth{0.000000pt}%
\definecolor{currentstroke}{rgb}{0.000000,0.000000,0.000000}%
\pgfsetstrokecolor{currentstroke}%
\pgfsetstrokeopacity{0.000000}%
\pgfsetdash{}{0pt}%
\pgfpathmoveto{\pgfqpoint{4.451168in}{0.500000in}}%
\pgfpathlineto{\pgfqpoint{4.477506in}{0.500000in}}%
\pgfpathlineto{\pgfqpoint{4.477506in}{0.500000in}}%
\pgfpathlineto{\pgfqpoint{4.451168in}{0.500000in}}%
\pgfpathlineto{\pgfqpoint{4.451168in}{0.500000in}}%
\pgfpathclose%
\pgfusepath{fill}%
\end{pgfscope}%
\begin{pgfscope}%
\pgfpathrectangle{\pgfqpoint{0.750000in}{0.500000in}}{\pgfqpoint{4.650000in}{3.020000in}}%
\pgfusepath{clip}%
\pgfsetbuttcap%
\pgfsetmiterjoin%
\definecolor{currentfill}{rgb}{0.121569,0.466667,0.705882}%
\pgfsetfillcolor{currentfill}%
\pgfsetlinewidth{0.000000pt}%
\definecolor{currentstroke}{rgb}{0.000000,0.000000,0.000000}%
\pgfsetstrokecolor{currentstroke}%
\pgfsetstrokeopacity{0.000000}%
\pgfsetdash{}{0pt}%
\pgfpathmoveto{\pgfqpoint{4.484091in}{0.500000in}}%
\pgfpathlineto{\pgfqpoint{4.510429in}{0.500000in}}%
\pgfpathlineto{\pgfqpoint{4.510429in}{0.500000in}}%
\pgfpathlineto{\pgfqpoint{4.484091in}{0.500000in}}%
\pgfpathlineto{\pgfqpoint{4.484091in}{0.500000in}}%
\pgfpathclose%
\pgfusepath{fill}%
\end{pgfscope}%
\begin{pgfscope}%
\pgfpathrectangle{\pgfqpoint{0.750000in}{0.500000in}}{\pgfqpoint{4.650000in}{3.020000in}}%
\pgfusepath{clip}%
\pgfsetbuttcap%
\pgfsetmiterjoin%
\definecolor{currentfill}{rgb}{0.121569,0.466667,0.705882}%
\pgfsetfillcolor{currentfill}%
\pgfsetlinewidth{0.000000pt}%
\definecolor{currentstroke}{rgb}{0.000000,0.000000,0.000000}%
\pgfsetstrokecolor{currentstroke}%
\pgfsetstrokeopacity{0.000000}%
\pgfsetdash{}{0pt}%
\pgfpathmoveto{\pgfqpoint{4.517014in}{0.500000in}}%
\pgfpathlineto{\pgfqpoint{4.543352in}{0.500000in}}%
\pgfpathlineto{\pgfqpoint{4.543352in}{0.500000in}}%
\pgfpathlineto{\pgfqpoint{4.517014in}{0.500000in}}%
\pgfpathlineto{\pgfqpoint{4.517014in}{0.500000in}}%
\pgfpathclose%
\pgfusepath{fill}%
\end{pgfscope}%
\begin{pgfscope}%
\pgfpathrectangle{\pgfqpoint{0.750000in}{0.500000in}}{\pgfqpoint{4.650000in}{3.020000in}}%
\pgfusepath{clip}%
\pgfsetbuttcap%
\pgfsetmiterjoin%
\definecolor{currentfill}{rgb}{0.121569,0.466667,0.705882}%
\pgfsetfillcolor{currentfill}%
\pgfsetlinewidth{0.000000pt}%
\definecolor{currentstroke}{rgb}{0.000000,0.000000,0.000000}%
\pgfsetstrokecolor{currentstroke}%
\pgfsetstrokeopacity{0.000000}%
\pgfsetdash{}{0pt}%
\pgfpathmoveto{\pgfqpoint{4.549936in}{0.500000in}}%
\pgfpathlineto{\pgfqpoint{4.576274in}{0.500000in}}%
\pgfpathlineto{\pgfqpoint{4.576274in}{0.500000in}}%
\pgfpathlineto{\pgfqpoint{4.549936in}{0.500000in}}%
\pgfpathlineto{\pgfqpoint{4.549936in}{0.500000in}}%
\pgfpathclose%
\pgfusepath{fill}%
\end{pgfscope}%
\begin{pgfscope}%
\pgfpathrectangle{\pgfqpoint{0.750000in}{0.500000in}}{\pgfqpoint{4.650000in}{3.020000in}}%
\pgfusepath{clip}%
\pgfsetbuttcap%
\pgfsetmiterjoin%
\definecolor{currentfill}{rgb}{0.121569,0.466667,0.705882}%
\pgfsetfillcolor{currentfill}%
\pgfsetlinewidth{0.000000pt}%
\definecolor{currentstroke}{rgb}{0.000000,0.000000,0.000000}%
\pgfsetstrokecolor{currentstroke}%
\pgfsetstrokeopacity{0.000000}%
\pgfsetdash{}{0pt}%
\pgfpathmoveto{\pgfqpoint{4.582859in}{0.500000in}}%
\pgfpathlineto{\pgfqpoint{4.609197in}{0.500000in}}%
\pgfpathlineto{\pgfqpoint{4.609197in}{0.500000in}}%
\pgfpathlineto{\pgfqpoint{4.582859in}{0.500000in}}%
\pgfpathlineto{\pgfqpoint{4.582859in}{0.500000in}}%
\pgfpathclose%
\pgfusepath{fill}%
\end{pgfscope}%
\begin{pgfscope}%
\pgfpathrectangle{\pgfqpoint{0.750000in}{0.500000in}}{\pgfqpoint{4.650000in}{3.020000in}}%
\pgfusepath{clip}%
\pgfsetbuttcap%
\pgfsetmiterjoin%
\definecolor{currentfill}{rgb}{0.121569,0.466667,0.705882}%
\pgfsetfillcolor{currentfill}%
\pgfsetlinewidth{0.000000pt}%
\definecolor{currentstroke}{rgb}{0.000000,0.000000,0.000000}%
\pgfsetstrokecolor{currentstroke}%
\pgfsetstrokeopacity{0.000000}%
\pgfsetdash{}{0pt}%
\pgfpathmoveto{\pgfqpoint{4.615782in}{0.500000in}}%
\pgfpathlineto{\pgfqpoint{4.642120in}{0.500000in}}%
\pgfpathlineto{\pgfqpoint{4.642120in}{0.500000in}}%
\pgfpathlineto{\pgfqpoint{4.615782in}{0.500000in}}%
\pgfpathlineto{\pgfqpoint{4.615782in}{0.500000in}}%
\pgfpathclose%
\pgfusepath{fill}%
\end{pgfscope}%
\begin{pgfscope}%
\pgfpathrectangle{\pgfqpoint{0.750000in}{0.500000in}}{\pgfqpoint{4.650000in}{3.020000in}}%
\pgfusepath{clip}%
\pgfsetbuttcap%
\pgfsetmiterjoin%
\definecolor{currentfill}{rgb}{0.121569,0.466667,0.705882}%
\pgfsetfillcolor{currentfill}%
\pgfsetlinewidth{0.000000pt}%
\definecolor{currentstroke}{rgb}{0.000000,0.000000,0.000000}%
\pgfsetstrokecolor{currentstroke}%
\pgfsetstrokeopacity{0.000000}%
\pgfsetdash{}{0pt}%
\pgfpathmoveto{\pgfqpoint{4.648704in}{0.500000in}}%
\pgfpathlineto{\pgfqpoint{4.675042in}{0.500000in}}%
\pgfpathlineto{\pgfqpoint{4.675042in}{0.500000in}}%
\pgfpathlineto{\pgfqpoint{4.648704in}{0.500000in}}%
\pgfpathlineto{\pgfqpoint{4.648704in}{0.500000in}}%
\pgfpathclose%
\pgfusepath{fill}%
\end{pgfscope}%
\begin{pgfscope}%
\pgfpathrectangle{\pgfqpoint{0.750000in}{0.500000in}}{\pgfqpoint{4.650000in}{3.020000in}}%
\pgfusepath{clip}%
\pgfsetbuttcap%
\pgfsetmiterjoin%
\definecolor{currentfill}{rgb}{0.121569,0.466667,0.705882}%
\pgfsetfillcolor{currentfill}%
\pgfsetlinewidth{0.000000pt}%
\definecolor{currentstroke}{rgb}{0.000000,0.000000,0.000000}%
\pgfsetstrokecolor{currentstroke}%
\pgfsetstrokeopacity{0.000000}%
\pgfsetdash{}{0pt}%
\pgfpathmoveto{\pgfqpoint{4.681627in}{0.500000in}}%
\pgfpathlineto{\pgfqpoint{4.707965in}{0.500000in}}%
\pgfpathlineto{\pgfqpoint{4.707965in}{0.500000in}}%
\pgfpathlineto{\pgfqpoint{4.681627in}{0.500000in}}%
\pgfpathlineto{\pgfqpoint{4.681627in}{0.500000in}}%
\pgfpathclose%
\pgfusepath{fill}%
\end{pgfscope}%
\begin{pgfscope}%
\pgfpathrectangle{\pgfqpoint{0.750000in}{0.500000in}}{\pgfqpoint{4.650000in}{3.020000in}}%
\pgfusepath{clip}%
\pgfsetbuttcap%
\pgfsetmiterjoin%
\definecolor{currentfill}{rgb}{0.121569,0.466667,0.705882}%
\pgfsetfillcolor{currentfill}%
\pgfsetlinewidth{0.000000pt}%
\definecolor{currentstroke}{rgb}{0.000000,0.000000,0.000000}%
\pgfsetstrokecolor{currentstroke}%
\pgfsetstrokeopacity{0.000000}%
\pgfsetdash{}{0pt}%
\pgfpathmoveto{\pgfqpoint{4.714550in}{0.500000in}}%
\pgfpathlineto{\pgfqpoint{4.740888in}{0.500000in}}%
\pgfpathlineto{\pgfqpoint{4.740888in}{0.500000in}}%
\pgfpathlineto{\pgfqpoint{4.714550in}{0.500000in}}%
\pgfpathlineto{\pgfqpoint{4.714550in}{0.500000in}}%
\pgfpathclose%
\pgfusepath{fill}%
\end{pgfscope}%
\begin{pgfscope}%
\pgfpathrectangle{\pgfqpoint{0.750000in}{0.500000in}}{\pgfqpoint{4.650000in}{3.020000in}}%
\pgfusepath{clip}%
\pgfsetbuttcap%
\pgfsetmiterjoin%
\definecolor{currentfill}{rgb}{0.121569,0.466667,0.705882}%
\pgfsetfillcolor{currentfill}%
\pgfsetlinewidth{0.000000pt}%
\definecolor{currentstroke}{rgb}{0.000000,0.000000,0.000000}%
\pgfsetstrokecolor{currentstroke}%
\pgfsetstrokeopacity{0.000000}%
\pgfsetdash{}{0pt}%
\pgfpathmoveto{\pgfqpoint{4.747472in}{0.500000in}}%
\pgfpathlineto{\pgfqpoint{4.773811in}{0.500000in}}%
\pgfpathlineto{\pgfqpoint{4.773811in}{0.500000in}}%
\pgfpathlineto{\pgfqpoint{4.747472in}{0.500000in}}%
\pgfpathlineto{\pgfqpoint{4.747472in}{0.500000in}}%
\pgfpathclose%
\pgfusepath{fill}%
\end{pgfscope}%
\begin{pgfscope}%
\pgfpathrectangle{\pgfqpoint{0.750000in}{0.500000in}}{\pgfqpoint{4.650000in}{3.020000in}}%
\pgfusepath{clip}%
\pgfsetbuttcap%
\pgfsetmiterjoin%
\definecolor{currentfill}{rgb}{0.121569,0.466667,0.705882}%
\pgfsetfillcolor{currentfill}%
\pgfsetlinewidth{0.000000pt}%
\definecolor{currentstroke}{rgb}{0.000000,0.000000,0.000000}%
\pgfsetstrokecolor{currentstroke}%
\pgfsetstrokeopacity{0.000000}%
\pgfsetdash{}{0pt}%
\pgfpathmoveto{\pgfqpoint{4.780395in}{0.500000in}}%
\pgfpathlineto{\pgfqpoint{4.806733in}{0.500000in}}%
\pgfpathlineto{\pgfqpoint{4.806733in}{0.500000in}}%
\pgfpathlineto{\pgfqpoint{4.780395in}{0.500000in}}%
\pgfpathlineto{\pgfqpoint{4.780395in}{0.500000in}}%
\pgfpathclose%
\pgfusepath{fill}%
\end{pgfscope}%
\begin{pgfscope}%
\pgfpathrectangle{\pgfqpoint{0.750000in}{0.500000in}}{\pgfqpoint{4.650000in}{3.020000in}}%
\pgfusepath{clip}%
\pgfsetbuttcap%
\pgfsetmiterjoin%
\definecolor{currentfill}{rgb}{0.121569,0.466667,0.705882}%
\pgfsetfillcolor{currentfill}%
\pgfsetlinewidth{0.000000pt}%
\definecolor{currentstroke}{rgb}{0.000000,0.000000,0.000000}%
\pgfsetstrokecolor{currentstroke}%
\pgfsetstrokeopacity{0.000000}%
\pgfsetdash{}{0pt}%
\pgfpathmoveto{\pgfqpoint{4.813318in}{0.500000in}}%
\pgfpathlineto{\pgfqpoint{4.839656in}{0.500000in}}%
\pgfpathlineto{\pgfqpoint{4.839656in}{0.500000in}}%
\pgfpathlineto{\pgfqpoint{4.813318in}{0.500000in}}%
\pgfpathlineto{\pgfqpoint{4.813318in}{0.500000in}}%
\pgfpathclose%
\pgfusepath{fill}%
\end{pgfscope}%
\begin{pgfscope}%
\pgfpathrectangle{\pgfqpoint{0.750000in}{0.500000in}}{\pgfqpoint{4.650000in}{3.020000in}}%
\pgfusepath{clip}%
\pgfsetbuttcap%
\pgfsetmiterjoin%
\definecolor{currentfill}{rgb}{0.121569,0.466667,0.705882}%
\pgfsetfillcolor{currentfill}%
\pgfsetlinewidth{0.000000pt}%
\definecolor{currentstroke}{rgb}{0.000000,0.000000,0.000000}%
\pgfsetstrokecolor{currentstroke}%
\pgfsetstrokeopacity{0.000000}%
\pgfsetdash{}{0pt}%
\pgfpathmoveto{\pgfqpoint{4.846240in}{0.500000in}}%
\pgfpathlineto{\pgfqpoint{4.872579in}{0.500000in}}%
\pgfpathlineto{\pgfqpoint{4.872579in}{0.500000in}}%
\pgfpathlineto{\pgfqpoint{4.846240in}{0.500000in}}%
\pgfpathlineto{\pgfqpoint{4.846240in}{0.500000in}}%
\pgfpathclose%
\pgfusepath{fill}%
\end{pgfscope}%
\begin{pgfscope}%
\pgfpathrectangle{\pgfqpoint{0.750000in}{0.500000in}}{\pgfqpoint{4.650000in}{3.020000in}}%
\pgfusepath{clip}%
\pgfsetbuttcap%
\pgfsetmiterjoin%
\definecolor{currentfill}{rgb}{0.121569,0.466667,0.705882}%
\pgfsetfillcolor{currentfill}%
\pgfsetlinewidth{0.000000pt}%
\definecolor{currentstroke}{rgb}{0.000000,0.000000,0.000000}%
\pgfsetstrokecolor{currentstroke}%
\pgfsetstrokeopacity{0.000000}%
\pgfsetdash{}{0pt}%
\pgfpathmoveto{\pgfqpoint{4.879163in}{0.500000in}}%
\pgfpathlineto{\pgfqpoint{4.905501in}{0.500000in}}%
\pgfpathlineto{\pgfqpoint{4.905501in}{0.500000in}}%
\pgfpathlineto{\pgfqpoint{4.879163in}{0.500000in}}%
\pgfpathlineto{\pgfqpoint{4.879163in}{0.500000in}}%
\pgfpathclose%
\pgfusepath{fill}%
\end{pgfscope}%
\begin{pgfscope}%
\pgfpathrectangle{\pgfqpoint{0.750000in}{0.500000in}}{\pgfqpoint{4.650000in}{3.020000in}}%
\pgfusepath{clip}%
\pgfsetbuttcap%
\pgfsetmiterjoin%
\definecolor{currentfill}{rgb}{0.121569,0.466667,0.705882}%
\pgfsetfillcolor{currentfill}%
\pgfsetlinewidth{0.000000pt}%
\definecolor{currentstroke}{rgb}{0.000000,0.000000,0.000000}%
\pgfsetstrokecolor{currentstroke}%
\pgfsetstrokeopacity{0.000000}%
\pgfsetdash{}{0pt}%
\pgfpathmoveto{\pgfqpoint{4.912086in}{0.500000in}}%
\pgfpathlineto{\pgfqpoint{4.938424in}{0.500000in}}%
\pgfpathlineto{\pgfqpoint{4.938424in}{0.500000in}}%
\pgfpathlineto{\pgfqpoint{4.912086in}{0.500000in}}%
\pgfpathlineto{\pgfqpoint{4.912086in}{0.500000in}}%
\pgfpathclose%
\pgfusepath{fill}%
\end{pgfscope}%
\begin{pgfscope}%
\pgfpathrectangle{\pgfqpoint{0.750000in}{0.500000in}}{\pgfqpoint{4.650000in}{3.020000in}}%
\pgfusepath{clip}%
\pgfsetbuttcap%
\pgfsetmiterjoin%
\definecolor{currentfill}{rgb}{0.121569,0.466667,0.705882}%
\pgfsetfillcolor{currentfill}%
\pgfsetlinewidth{0.000000pt}%
\definecolor{currentstroke}{rgb}{0.000000,0.000000,0.000000}%
\pgfsetstrokecolor{currentstroke}%
\pgfsetstrokeopacity{0.000000}%
\pgfsetdash{}{0pt}%
\pgfpathmoveto{\pgfqpoint{4.945008in}{0.500000in}}%
\pgfpathlineto{\pgfqpoint{4.971347in}{0.500000in}}%
\pgfpathlineto{\pgfqpoint{4.971347in}{0.500000in}}%
\pgfpathlineto{\pgfqpoint{4.945008in}{0.500000in}}%
\pgfpathlineto{\pgfqpoint{4.945008in}{0.500000in}}%
\pgfpathclose%
\pgfusepath{fill}%
\end{pgfscope}%
\begin{pgfscope}%
\pgfpathrectangle{\pgfqpoint{0.750000in}{0.500000in}}{\pgfqpoint{4.650000in}{3.020000in}}%
\pgfusepath{clip}%
\pgfsetbuttcap%
\pgfsetmiterjoin%
\definecolor{currentfill}{rgb}{0.121569,0.466667,0.705882}%
\pgfsetfillcolor{currentfill}%
\pgfsetlinewidth{0.000000pt}%
\definecolor{currentstroke}{rgb}{0.000000,0.000000,0.000000}%
\pgfsetstrokecolor{currentstroke}%
\pgfsetstrokeopacity{0.000000}%
\pgfsetdash{}{0pt}%
\pgfpathmoveto{\pgfqpoint{4.977931in}{0.500000in}}%
\pgfpathlineto{\pgfqpoint{5.004269in}{0.500000in}}%
\pgfpathlineto{\pgfqpoint{5.004269in}{0.500000in}}%
\pgfpathlineto{\pgfqpoint{4.977931in}{0.500000in}}%
\pgfpathlineto{\pgfqpoint{4.977931in}{0.500000in}}%
\pgfpathclose%
\pgfusepath{fill}%
\end{pgfscope}%
\begin{pgfscope}%
\pgfpathrectangle{\pgfqpoint{0.750000in}{0.500000in}}{\pgfqpoint{4.650000in}{3.020000in}}%
\pgfusepath{clip}%
\pgfsetbuttcap%
\pgfsetmiterjoin%
\definecolor{currentfill}{rgb}{0.121569,0.466667,0.705882}%
\pgfsetfillcolor{currentfill}%
\pgfsetlinewidth{0.000000pt}%
\definecolor{currentstroke}{rgb}{0.000000,0.000000,0.000000}%
\pgfsetstrokecolor{currentstroke}%
\pgfsetstrokeopacity{0.000000}%
\pgfsetdash{}{0pt}%
\pgfpathmoveto{\pgfqpoint{5.010854in}{0.500000in}}%
\pgfpathlineto{\pgfqpoint{5.037192in}{0.500000in}}%
\pgfpathlineto{\pgfqpoint{5.037192in}{0.500000in}}%
\pgfpathlineto{\pgfqpoint{5.010854in}{0.500000in}}%
\pgfpathlineto{\pgfqpoint{5.010854in}{0.500000in}}%
\pgfpathclose%
\pgfusepath{fill}%
\end{pgfscope}%
\begin{pgfscope}%
\pgfpathrectangle{\pgfqpoint{0.750000in}{0.500000in}}{\pgfqpoint{4.650000in}{3.020000in}}%
\pgfusepath{clip}%
\pgfsetbuttcap%
\pgfsetmiterjoin%
\definecolor{currentfill}{rgb}{0.121569,0.466667,0.705882}%
\pgfsetfillcolor{currentfill}%
\pgfsetlinewidth{0.000000pt}%
\definecolor{currentstroke}{rgb}{0.000000,0.000000,0.000000}%
\pgfsetstrokecolor{currentstroke}%
\pgfsetstrokeopacity{0.000000}%
\pgfsetdash{}{0pt}%
\pgfpathmoveto{\pgfqpoint{5.043777in}{0.500000in}}%
\pgfpathlineto{\pgfqpoint{5.070115in}{0.500000in}}%
\pgfpathlineto{\pgfqpoint{5.070115in}{0.500000in}}%
\pgfpathlineto{\pgfqpoint{5.043777in}{0.500000in}}%
\pgfpathlineto{\pgfqpoint{5.043777in}{0.500000in}}%
\pgfpathclose%
\pgfusepath{fill}%
\end{pgfscope}%
\begin{pgfscope}%
\pgfpathrectangle{\pgfqpoint{0.750000in}{0.500000in}}{\pgfqpoint{4.650000in}{3.020000in}}%
\pgfusepath{clip}%
\pgfsetbuttcap%
\pgfsetmiterjoin%
\definecolor{currentfill}{rgb}{0.121569,0.466667,0.705882}%
\pgfsetfillcolor{currentfill}%
\pgfsetlinewidth{0.000000pt}%
\definecolor{currentstroke}{rgb}{0.000000,0.000000,0.000000}%
\pgfsetstrokecolor{currentstroke}%
\pgfsetstrokeopacity{0.000000}%
\pgfsetdash{}{0pt}%
\pgfpathmoveto{\pgfqpoint{5.076699in}{0.500000in}}%
\pgfpathlineto{\pgfqpoint{5.103037in}{0.500000in}}%
\pgfpathlineto{\pgfqpoint{5.103037in}{0.500000in}}%
\pgfpathlineto{\pgfqpoint{5.076699in}{0.500000in}}%
\pgfpathlineto{\pgfqpoint{5.076699in}{0.500000in}}%
\pgfpathclose%
\pgfusepath{fill}%
\end{pgfscope}%
\begin{pgfscope}%
\pgfpathrectangle{\pgfqpoint{0.750000in}{0.500000in}}{\pgfqpoint{4.650000in}{3.020000in}}%
\pgfusepath{clip}%
\pgfsetbuttcap%
\pgfsetmiterjoin%
\definecolor{currentfill}{rgb}{0.121569,0.466667,0.705882}%
\pgfsetfillcolor{currentfill}%
\pgfsetlinewidth{0.000000pt}%
\definecolor{currentstroke}{rgb}{0.000000,0.000000,0.000000}%
\pgfsetstrokecolor{currentstroke}%
\pgfsetstrokeopacity{0.000000}%
\pgfsetdash{}{0pt}%
\pgfpathmoveto{\pgfqpoint{5.109622in}{0.500000in}}%
\pgfpathlineto{\pgfqpoint{5.135960in}{0.500000in}}%
\pgfpathlineto{\pgfqpoint{5.135960in}{0.500000in}}%
\pgfpathlineto{\pgfqpoint{5.109622in}{0.500000in}}%
\pgfpathlineto{\pgfqpoint{5.109622in}{0.500000in}}%
\pgfpathclose%
\pgfusepath{fill}%
\end{pgfscope}%
\begin{pgfscope}%
\pgfpathrectangle{\pgfqpoint{0.750000in}{0.500000in}}{\pgfqpoint{4.650000in}{3.020000in}}%
\pgfusepath{clip}%
\pgfsetbuttcap%
\pgfsetmiterjoin%
\definecolor{currentfill}{rgb}{0.121569,0.466667,0.705882}%
\pgfsetfillcolor{currentfill}%
\pgfsetlinewidth{0.000000pt}%
\definecolor{currentstroke}{rgb}{0.000000,0.000000,0.000000}%
\pgfsetstrokecolor{currentstroke}%
\pgfsetstrokeopacity{0.000000}%
\pgfsetdash{}{0pt}%
\pgfpathmoveto{\pgfqpoint{5.142545in}{0.500000in}}%
\pgfpathlineto{\pgfqpoint{5.168883in}{0.500000in}}%
\pgfpathlineto{\pgfqpoint{5.168883in}{0.500000in}}%
\pgfpathlineto{\pgfqpoint{5.142545in}{0.500000in}}%
\pgfpathlineto{\pgfqpoint{5.142545in}{0.500000in}}%
\pgfpathclose%
\pgfusepath{fill}%
\end{pgfscope}%
\begin{pgfscope}%
\pgfpathrectangle{\pgfqpoint{0.750000in}{0.500000in}}{\pgfqpoint{4.650000in}{3.020000in}}%
\pgfusepath{clip}%
\pgfsetbuttcap%
\pgfsetroundjoin%
\definecolor{currentfill}{rgb}{1.000000,0.000000,0.000000}%
\pgfsetfillcolor{currentfill}%
\pgfsetlinewidth{1.003750pt}%
\definecolor{currentstroke}{rgb}{1.000000,0.000000,0.000000}%
\pgfsetstrokecolor{currentstroke}%
\pgfsetdash{}{0pt}%
\pgfsys@defobject{currentmarker}{\pgfqpoint{-0.041667in}{-0.041667in}}{\pgfqpoint{0.041667in}{0.041667in}}{%
\pgfpathmoveto{\pgfqpoint{0.000000in}{-0.041667in}}%
\pgfpathcurveto{\pgfqpoint{0.011050in}{-0.041667in}}{\pgfqpoint{0.021649in}{-0.037276in}}{\pgfqpoint{0.029463in}{-0.029463in}}%
\pgfpathcurveto{\pgfqpoint{0.037276in}{-0.021649in}}{\pgfqpoint{0.041667in}{-0.011050in}}{\pgfqpoint{0.041667in}{0.000000in}}%
\pgfpathcurveto{\pgfqpoint{0.041667in}{0.011050in}}{\pgfqpoint{0.037276in}{0.021649in}}{\pgfqpoint{0.029463in}{0.029463in}}%
\pgfpathcurveto{\pgfqpoint{0.021649in}{0.037276in}}{\pgfqpoint{0.011050in}{0.041667in}}{\pgfqpoint{0.000000in}{0.041667in}}%
\pgfpathcurveto{\pgfqpoint{-0.011050in}{0.041667in}}{\pgfqpoint{-0.021649in}{0.037276in}}{\pgfqpoint{-0.029463in}{0.029463in}}%
\pgfpathcurveto{\pgfqpoint{-0.037276in}{0.021649in}}{\pgfqpoint{-0.041667in}{0.011050in}}{\pgfqpoint{-0.041667in}{0.000000in}}%
\pgfpathcurveto{\pgfqpoint{-0.041667in}{-0.011050in}}{\pgfqpoint{-0.037276in}{-0.021649in}}{\pgfqpoint{-0.029463in}{-0.029463in}}%
\pgfpathcurveto{\pgfqpoint{-0.021649in}{-0.037276in}}{\pgfqpoint{-0.011050in}{-0.041667in}}{\pgfqpoint{0.000000in}{-0.041667in}}%
\pgfpathlineto{\pgfqpoint{0.000000in}{-0.041667in}}%
\pgfpathclose%
\pgfusepath{stroke,fill}%
}%
\begin{pgfscope}%
\pgfsys@transformshift{4.135110in}{0.500000in}%
\pgfsys@useobject{currentmarker}{}%
\end{pgfscope}%
\end{pgfscope}%
\begin{pgfscope}%
\pgfpathrectangle{\pgfqpoint{0.750000in}{0.500000in}}{\pgfqpoint{4.650000in}{3.020000in}}%
\pgfusepath{clip}%
\pgfsetbuttcap%
\pgfsetroundjoin%
\definecolor{currentfill}{rgb}{0.000000,0.500000,0.000000}%
\pgfsetfillcolor{currentfill}%
\pgfsetlinewidth{1.003750pt}%
\definecolor{currentstroke}{rgb}{0.000000,0.500000,0.000000}%
\pgfsetstrokecolor{currentstroke}%
\pgfsetdash{}{0pt}%
\pgfsys@defobject{currentmarker}{\pgfqpoint{-0.041667in}{-0.041667in}}{\pgfqpoint{0.041667in}{0.041667in}}{%
\pgfpathmoveto{\pgfqpoint{0.000000in}{-0.041667in}}%
\pgfpathcurveto{\pgfqpoint{0.011050in}{-0.041667in}}{\pgfqpoint{0.021649in}{-0.037276in}}{\pgfqpoint{0.029463in}{-0.029463in}}%
\pgfpathcurveto{\pgfqpoint{0.037276in}{-0.021649in}}{\pgfqpoint{0.041667in}{-0.011050in}}{\pgfqpoint{0.041667in}{0.000000in}}%
\pgfpathcurveto{\pgfqpoint{0.041667in}{0.011050in}}{\pgfqpoint{0.037276in}{0.021649in}}{\pgfqpoint{0.029463in}{0.029463in}}%
\pgfpathcurveto{\pgfqpoint{0.021649in}{0.037276in}}{\pgfqpoint{0.011050in}{0.041667in}}{\pgfqpoint{0.000000in}{0.041667in}}%
\pgfpathcurveto{\pgfqpoint{-0.011050in}{0.041667in}}{\pgfqpoint{-0.021649in}{0.037276in}}{\pgfqpoint{-0.029463in}{0.029463in}}%
\pgfpathcurveto{\pgfqpoint{-0.037276in}{0.021649in}}{\pgfqpoint{-0.041667in}{0.011050in}}{\pgfqpoint{-0.041667in}{0.000000in}}%
\pgfpathcurveto{\pgfqpoint{-0.041667in}{-0.011050in}}{\pgfqpoint{-0.037276in}{-0.021649in}}{\pgfqpoint{-0.029463in}{-0.029463in}}%
\pgfpathcurveto{\pgfqpoint{-0.021649in}{-0.037276in}}{\pgfqpoint{-0.011050in}{-0.041667in}}{\pgfqpoint{0.000000in}{-0.041667in}}%
\pgfpathlineto{\pgfqpoint{0.000000in}{-0.041667in}}%
\pgfpathclose%
\pgfusepath{stroke,fill}%
}%
\begin{pgfscope}%
\pgfsys@transformshift{5.188636in}{0.500000in}%
\pgfsys@useobject{currentmarker}{}%
\end{pgfscope}%
\end{pgfscope}%
\begin{pgfscope}%
\pgfsetbuttcap%
\pgfsetroundjoin%
\definecolor{currentfill}{rgb}{0.000000,0.000000,0.000000}%
\pgfsetfillcolor{currentfill}%
\pgfsetlinewidth{0.803000pt}%
\definecolor{currentstroke}{rgb}{0.000000,0.000000,0.000000}%
\pgfsetstrokecolor{currentstroke}%
\pgfsetdash{}{0pt}%
\pgfsys@defobject{currentmarker}{\pgfqpoint{0.000000in}{-0.048611in}}{\pgfqpoint{0.000000in}{0.000000in}}{%
\pgfpathmoveto{\pgfqpoint{0.000000in}{0.000000in}}%
\pgfpathlineto{\pgfqpoint{0.000000in}{-0.048611in}}%
\pgfusepath{stroke,fill}%
}%
\begin{pgfscope}%
\pgfsys@transformshift{0.974533in}{0.500000in}%
\pgfsys@useobject{currentmarker}{}%
\end{pgfscope}%
\end{pgfscope}%
\begin{pgfscope}%
\definecolor{textcolor}{rgb}{0.000000,0.000000,0.000000}%
\pgfsetstrokecolor{textcolor}%
\pgfsetfillcolor{textcolor}%
\pgftext[x=0.974533in,y=0.402778in,,top]{\color{textcolor}\rmfamily\fontsize{13.000000}{15.600000}\selectfont \(\displaystyle {0}\)}%
\end{pgfscope}%
\begin{pgfscope}%
\pgfsetbuttcap%
\pgfsetroundjoin%
\definecolor{currentfill}{rgb}{0.000000,0.000000,0.000000}%
\pgfsetfillcolor{currentfill}%
\pgfsetlinewidth{0.803000pt}%
\definecolor{currentstroke}{rgb}{0.000000,0.000000,0.000000}%
\pgfsetstrokecolor{currentstroke}%
\pgfsetdash{}{0pt}%
\pgfsys@defobject{currentmarker}{\pgfqpoint{0.000000in}{-0.048611in}}{\pgfqpoint{0.000000in}{0.000000in}}{%
\pgfpathmoveto{\pgfqpoint{0.000000in}{0.000000in}}%
\pgfpathlineto{\pgfqpoint{0.000000in}{-0.048611in}}%
\pgfusepath{stroke,fill}%
}%
\begin{pgfscope}%
\pgfsys@transformshift{1.632986in}{0.500000in}%
\pgfsys@useobject{currentmarker}{}%
\end{pgfscope}%
\end{pgfscope}%
\begin{pgfscope}%
\definecolor{textcolor}{rgb}{0.000000,0.000000,0.000000}%
\pgfsetstrokecolor{textcolor}%
\pgfsetfillcolor{textcolor}%
\pgftext[x=1.632986in,y=0.402778in,,top]{\color{textcolor}\rmfamily\fontsize{13.000000}{15.600000}\selectfont \(\displaystyle {20}\)}%
\end{pgfscope}%
\begin{pgfscope}%
\pgfsetbuttcap%
\pgfsetroundjoin%
\definecolor{currentfill}{rgb}{0.000000,0.000000,0.000000}%
\pgfsetfillcolor{currentfill}%
\pgfsetlinewidth{0.803000pt}%
\definecolor{currentstroke}{rgb}{0.000000,0.000000,0.000000}%
\pgfsetstrokecolor{currentstroke}%
\pgfsetdash{}{0pt}%
\pgfsys@defobject{currentmarker}{\pgfqpoint{0.000000in}{-0.048611in}}{\pgfqpoint{0.000000in}{0.000000in}}{%
\pgfpathmoveto{\pgfqpoint{0.000000in}{0.000000in}}%
\pgfpathlineto{\pgfqpoint{0.000000in}{-0.048611in}}%
\pgfusepath{stroke,fill}%
}%
\begin{pgfscope}%
\pgfsys@transformshift{2.291440in}{0.500000in}%
\pgfsys@useobject{currentmarker}{}%
\end{pgfscope}%
\end{pgfscope}%
\begin{pgfscope}%
\definecolor{textcolor}{rgb}{0.000000,0.000000,0.000000}%
\pgfsetstrokecolor{textcolor}%
\pgfsetfillcolor{textcolor}%
\pgftext[x=2.291440in,y=0.402778in,,top]{\color{textcolor}\rmfamily\fontsize{13.000000}{15.600000}\selectfont \(\displaystyle {40}\)}%
\end{pgfscope}%
\begin{pgfscope}%
\pgfsetbuttcap%
\pgfsetroundjoin%
\definecolor{currentfill}{rgb}{0.000000,0.000000,0.000000}%
\pgfsetfillcolor{currentfill}%
\pgfsetlinewidth{0.803000pt}%
\definecolor{currentstroke}{rgb}{0.000000,0.000000,0.000000}%
\pgfsetstrokecolor{currentstroke}%
\pgfsetdash{}{0pt}%
\pgfsys@defobject{currentmarker}{\pgfqpoint{0.000000in}{-0.048611in}}{\pgfqpoint{0.000000in}{0.000000in}}{%
\pgfpathmoveto{\pgfqpoint{0.000000in}{0.000000in}}%
\pgfpathlineto{\pgfqpoint{0.000000in}{-0.048611in}}%
\pgfusepath{stroke,fill}%
}%
\begin{pgfscope}%
\pgfsys@transformshift{2.949894in}{0.500000in}%
\pgfsys@useobject{currentmarker}{}%
\end{pgfscope}%
\end{pgfscope}%
\begin{pgfscope}%
\definecolor{textcolor}{rgb}{0.000000,0.000000,0.000000}%
\pgfsetstrokecolor{textcolor}%
\pgfsetfillcolor{textcolor}%
\pgftext[x=2.949894in,y=0.402778in,,top]{\color{textcolor}\rmfamily\fontsize{13.000000}{15.600000}\selectfont \(\displaystyle {60}\)}%
\end{pgfscope}%
\begin{pgfscope}%
\pgfsetbuttcap%
\pgfsetroundjoin%
\definecolor{currentfill}{rgb}{0.000000,0.000000,0.000000}%
\pgfsetfillcolor{currentfill}%
\pgfsetlinewidth{0.803000pt}%
\definecolor{currentstroke}{rgb}{0.000000,0.000000,0.000000}%
\pgfsetstrokecolor{currentstroke}%
\pgfsetdash{}{0pt}%
\pgfsys@defobject{currentmarker}{\pgfqpoint{0.000000in}{-0.048611in}}{\pgfqpoint{0.000000in}{0.000000in}}{%
\pgfpathmoveto{\pgfqpoint{0.000000in}{0.000000in}}%
\pgfpathlineto{\pgfqpoint{0.000000in}{-0.048611in}}%
\pgfusepath{stroke,fill}%
}%
\begin{pgfscope}%
\pgfsys@transformshift{3.608347in}{0.500000in}%
\pgfsys@useobject{currentmarker}{}%
\end{pgfscope}%
\end{pgfscope}%
\begin{pgfscope}%
\definecolor{textcolor}{rgb}{0.000000,0.000000,0.000000}%
\pgfsetstrokecolor{textcolor}%
\pgfsetfillcolor{textcolor}%
\pgftext[x=3.608347in,y=0.402778in,,top]{\color{textcolor}\rmfamily\fontsize{13.000000}{15.600000}\selectfont \(\displaystyle {80}\)}%
\end{pgfscope}%
\begin{pgfscope}%
\pgfsetbuttcap%
\pgfsetroundjoin%
\definecolor{currentfill}{rgb}{0.000000,0.000000,0.000000}%
\pgfsetfillcolor{currentfill}%
\pgfsetlinewidth{0.803000pt}%
\definecolor{currentstroke}{rgb}{0.000000,0.000000,0.000000}%
\pgfsetstrokecolor{currentstroke}%
\pgfsetdash{}{0pt}%
\pgfsys@defobject{currentmarker}{\pgfqpoint{0.000000in}{-0.048611in}}{\pgfqpoint{0.000000in}{0.000000in}}{%
\pgfpathmoveto{\pgfqpoint{0.000000in}{0.000000in}}%
\pgfpathlineto{\pgfqpoint{0.000000in}{-0.048611in}}%
\pgfusepath{stroke,fill}%
}%
\begin{pgfscope}%
\pgfsys@transformshift{4.266801in}{0.500000in}%
\pgfsys@useobject{currentmarker}{}%
\end{pgfscope}%
\end{pgfscope}%
\begin{pgfscope}%
\definecolor{textcolor}{rgb}{0.000000,0.000000,0.000000}%
\pgfsetstrokecolor{textcolor}%
\pgfsetfillcolor{textcolor}%
\pgftext[x=4.266801in,y=0.402778in,,top]{\color{textcolor}\rmfamily\fontsize{13.000000}{15.600000}\selectfont \(\displaystyle {100}\)}%
\end{pgfscope}%
\begin{pgfscope}%
\pgfsetbuttcap%
\pgfsetroundjoin%
\definecolor{currentfill}{rgb}{0.000000,0.000000,0.000000}%
\pgfsetfillcolor{currentfill}%
\pgfsetlinewidth{0.803000pt}%
\definecolor{currentstroke}{rgb}{0.000000,0.000000,0.000000}%
\pgfsetstrokecolor{currentstroke}%
\pgfsetdash{}{0pt}%
\pgfsys@defobject{currentmarker}{\pgfqpoint{0.000000in}{-0.048611in}}{\pgfqpoint{0.000000in}{0.000000in}}{%
\pgfpathmoveto{\pgfqpoint{0.000000in}{0.000000in}}%
\pgfpathlineto{\pgfqpoint{0.000000in}{-0.048611in}}%
\pgfusepath{stroke,fill}%
}%
\begin{pgfscope}%
\pgfsys@transformshift{4.925255in}{0.500000in}%
\pgfsys@useobject{currentmarker}{}%
\end{pgfscope}%
\end{pgfscope}%
\begin{pgfscope}%
\definecolor{textcolor}{rgb}{0.000000,0.000000,0.000000}%
\pgfsetstrokecolor{textcolor}%
\pgfsetfillcolor{textcolor}%
\pgftext[x=4.925255in,y=0.402778in,,top]{\color{textcolor}\rmfamily\fontsize{13.000000}{15.600000}\selectfont \(\displaystyle {120}\)}%
\end{pgfscope}%
\begin{pgfscope}%
\definecolor{textcolor}{rgb}{0.000000,0.000000,0.000000}%
\pgfsetstrokecolor{textcolor}%
\pgfsetfillcolor{textcolor}%
\pgftext[x=3.075000in,y=0.199075in,,top]{\color{textcolor}\rmfamily\fontsize{13.000000}{15.600000}\selectfont Number of correct bits}%
\end{pgfscope}%
\begin{pgfscope}%
\pgfsetbuttcap%
\pgfsetroundjoin%
\definecolor{currentfill}{rgb}{0.000000,0.000000,0.000000}%
\pgfsetfillcolor{currentfill}%
\pgfsetlinewidth{0.803000pt}%
\definecolor{currentstroke}{rgb}{0.000000,0.000000,0.000000}%
\pgfsetstrokecolor{currentstroke}%
\pgfsetdash{}{0pt}%
\pgfsys@defobject{currentmarker}{\pgfqpoint{-0.048611in}{0.000000in}}{\pgfqpoint{-0.000000in}{0.000000in}}{%
\pgfpathmoveto{\pgfqpoint{-0.000000in}{0.000000in}}%
\pgfpathlineto{\pgfqpoint{-0.048611in}{0.000000in}}%
\pgfusepath{stroke,fill}%
}%
\begin{pgfscope}%
\pgfsys@transformshift{0.750000in}{0.500000in}%
\pgfsys@useobject{currentmarker}{}%
\end{pgfscope}%
\end{pgfscope}%
\begin{pgfscope}%
\definecolor{textcolor}{rgb}{0.000000,0.000000,0.000000}%
\pgfsetstrokecolor{textcolor}%
\pgfsetfillcolor{textcolor}%
\pgftext[x=0.362657in, y=0.442130in, left, base]{\color{textcolor}\rmfamily\fontsize{13.000000}{15.600000}\selectfont \(\displaystyle {0.00}\)}%
\end{pgfscope}%
\begin{pgfscope}%
\pgfsetbuttcap%
\pgfsetroundjoin%
\definecolor{currentfill}{rgb}{0.000000,0.000000,0.000000}%
\pgfsetfillcolor{currentfill}%
\pgfsetlinewidth{0.803000pt}%
\definecolor{currentstroke}{rgb}{0.000000,0.000000,0.000000}%
\pgfsetstrokecolor{currentstroke}%
\pgfsetdash{}{0pt}%
\pgfsys@defobject{currentmarker}{\pgfqpoint{-0.048611in}{0.000000in}}{\pgfqpoint{-0.000000in}{0.000000in}}{%
\pgfpathmoveto{\pgfqpoint{-0.000000in}{0.000000in}}%
\pgfpathlineto{\pgfqpoint{-0.048611in}{0.000000in}}%
\pgfusepath{stroke,fill}%
}%
\begin{pgfscope}%
\pgfsys@transformshift{0.750000in}{0.908631in}%
\pgfsys@useobject{currentmarker}{}%
\end{pgfscope}%
\end{pgfscope}%
\begin{pgfscope}%
\definecolor{textcolor}{rgb}{0.000000,0.000000,0.000000}%
\pgfsetstrokecolor{textcolor}%
\pgfsetfillcolor{textcolor}%
\pgftext[x=0.362657in, y=0.850760in, left, base]{\color{textcolor}\rmfamily\fontsize{13.000000}{15.600000}\selectfont \(\displaystyle {0.01}\)}%
\end{pgfscope}%
\begin{pgfscope}%
\pgfsetbuttcap%
\pgfsetroundjoin%
\definecolor{currentfill}{rgb}{0.000000,0.000000,0.000000}%
\pgfsetfillcolor{currentfill}%
\pgfsetlinewidth{0.803000pt}%
\definecolor{currentstroke}{rgb}{0.000000,0.000000,0.000000}%
\pgfsetstrokecolor{currentstroke}%
\pgfsetdash{}{0pt}%
\pgfsys@defobject{currentmarker}{\pgfqpoint{-0.048611in}{0.000000in}}{\pgfqpoint{-0.000000in}{0.000000in}}{%
\pgfpathmoveto{\pgfqpoint{-0.000000in}{0.000000in}}%
\pgfpathlineto{\pgfqpoint{-0.048611in}{0.000000in}}%
\pgfusepath{stroke,fill}%
}%
\begin{pgfscope}%
\pgfsys@transformshift{0.750000in}{1.317261in}%
\pgfsys@useobject{currentmarker}{}%
\end{pgfscope}%
\end{pgfscope}%
\begin{pgfscope}%
\definecolor{textcolor}{rgb}{0.000000,0.000000,0.000000}%
\pgfsetstrokecolor{textcolor}%
\pgfsetfillcolor{textcolor}%
\pgftext[x=0.362657in, y=1.259391in, left, base]{\color{textcolor}\rmfamily\fontsize{13.000000}{15.600000}\selectfont \(\displaystyle {0.02}\)}%
\end{pgfscope}%
\begin{pgfscope}%
\pgfsetbuttcap%
\pgfsetroundjoin%
\definecolor{currentfill}{rgb}{0.000000,0.000000,0.000000}%
\pgfsetfillcolor{currentfill}%
\pgfsetlinewidth{0.803000pt}%
\definecolor{currentstroke}{rgb}{0.000000,0.000000,0.000000}%
\pgfsetstrokecolor{currentstroke}%
\pgfsetdash{}{0pt}%
\pgfsys@defobject{currentmarker}{\pgfqpoint{-0.048611in}{0.000000in}}{\pgfqpoint{-0.000000in}{0.000000in}}{%
\pgfpathmoveto{\pgfqpoint{-0.000000in}{0.000000in}}%
\pgfpathlineto{\pgfqpoint{-0.048611in}{0.000000in}}%
\pgfusepath{stroke,fill}%
}%
\begin{pgfscope}%
\pgfsys@transformshift{0.750000in}{1.725892in}%
\pgfsys@useobject{currentmarker}{}%
\end{pgfscope}%
\end{pgfscope}%
\begin{pgfscope}%
\definecolor{textcolor}{rgb}{0.000000,0.000000,0.000000}%
\pgfsetstrokecolor{textcolor}%
\pgfsetfillcolor{textcolor}%
\pgftext[x=0.362657in, y=1.668021in, left, base]{\color{textcolor}\rmfamily\fontsize{13.000000}{15.600000}\selectfont \(\displaystyle {0.03}\)}%
\end{pgfscope}%
\begin{pgfscope}%
\pgfsetbuttcap%
\pgfsetroundjoin%
\definecolor{currentfill}{rgb}{0.000000,0.000000,0.000000}%
\pgfsetfillcolor{currentfill}%
\pgfsetlinewidth{0.803000pt}%
\definecolor{currentstroke}{rgb}{0.000000,0.000000,0.000000}%
\pgfsetstrokecolor{currentstroke}%
\pgfsetdash{}{0pt}%
\pgfsys@defobject{currentmarker}{\pgfqpoint{-0.048611in}{0.000000in}}{\pgfqpoint{-0.000000in}{0.000000in}}{%
\pgfpathmoveto{\pgfqpoint{-0.000000in}{0.000000in}}%
\pgfpathlineto{\pgfqpoint{-0.048611in}{0.000000in}}%
\pgfusepath{stroke,fill}%
}%
\begin{pgfscope}%
\pgfsys@transformshift{0.750000in}{2.134522in}%
\pgfsys@useobject{currentmarker}{}%
\end{pgfscope}%
\end{pgfscope}%
\begin{pgfscope}%
\definecolor{textcolor}{rgb}{0.000000,0.000000,0.000000}%
\pgfsetstrokecolor{textcolor}%
\pgfsetfillcolor{textcolor}%
\pgftext[x=0.362657in, y=2.076652in, left, base]{\color{textcolor}\rmfamily\fontsize{13.000000}{15.600000}\selectfont \(\displaystyle {0.04}\)}%
\end{pgfscope}%
\begin{pgfscope}%
\pgfsetbuttcap%
\pgfsetroundjoin%
\definecolor{currentfill}{rgb}{0.000000,0.000000,0.000000}%
\pgfsetfillcolor{currentfill}%
\pgfsetlinewidth{0.803000pt}%
\definecolor{currentstroke}{rgb}{0.000000,0.000000,0.000000}%
\pgfsetstrokecolor{currentstroke}%
\pgfsetdash{}{0pt}%
\pgfsys@defobject{currentmarker}{\pgfqpoint{-0.048611in}{0.000000in}}{\pgfqpoint{-0.000000in}{0.000000in}}{%
\pgfpathmoveto{\pgfqpoint{-0.000000in}{0.000000in}}%
\pgfpathlineto{\pgfqpoint{-0.048611in}{0.000000in}}%
\pgfusepath{stroke,fill}%
}%
\begin{pgfscope}%
\pgfsys@transformshift{0.750000in}{2.543153in}%
\pgfsys@useobject{currentmarker}{}%
\end{pgfscope}%
\end{pgfscope}%
\begin{pgfscope}%
\definecolor{textcolor}{rgb}{0.000000,0.000000,0.000000}%
\pgfsetstrokecolor{textcolor}%
\pgfsetfillcolor{textcolor}%
\pgftext[x=0.362657in, y=2.485282in, left, base]{\color{textcolor}\rmfamily\fontsize{13.000000}{15.600000}\selectfont \(\displaystyle {0.05}\)}%
\end{pgfscope}%
\begin{pgfscope}%
\pgfsetbuttcap%
\pgfsetroundjoin%
\definecolor{currentfill}{rgb}{0.000000,0.000000,0.000000}%
\pgfsetfillcolor{currentfill}%
\pgfsetlinewidth{0.803000pt}%
\definecolor{currentstroke}{rgb}{0.000000,0.000000,0.000000}%
\pgfsetstrokecolor{currentstroke}%
\pgfsetdash{}{0pt}%
\pgfsys@defobject{currentmarker}{\pgfqpoint{-0.048611in}{0.000000in}}{\pgfqpoint{-0.000000in}{0.000000in}}{%
\pgfpathmoveto{\pgfqpoint{-0.000000in}{0.000000in}}%
\pgfpathlineto{\pgfqpoint{-0.048611in}{0.000000in}}%
\pgfusepath{stroke,fill}%
}%
\begin{pgfscope}%
\pgfsys@transformshift{0.750000in}{2.951783in}%
\pgfsys@useobject{currentmarker}{}%
\end{pgfscope}%
\end{pgfscope}%
\begin{pgfscope}%
\definecolor{textcolor}{rgb}{0.000000,0.000000,0.000000}%
\pgfsetstrokecolor{textcolor}%
\pgfsetfillcolor{textcolor}%
\pgftext[x=0.362657in, y=2.893913in, left, base]{\color{textcolor}\rmfamily\fontsize{13.000000}{15.600000}\selectfont \(\displaystyle {0.06}\)}%
\end{pgfscope}%
\begin{pgfscope}%
\pgfsetbuttcap%
\pgfsetroundjoin%
\definecolor{currentfill}{rgb}{0.000000,0.000000,0.000000}%
\pgfsetfillcolor{currentfill}%
\pgfsetlinewidth{0.803000pt}%
\definecolor{currentstroke}{rgb}{0.000000,0.000000,0.000000}%
\pgfsetstrokecolor{currentstroke}%
\pgfsetdash{}{0pt}%
\pgfsys@defobject{currentmarker}{\pgfqpoint{-0.048611in}{0.000000in}}{\pgfqpoint{-0.000000in}{0.000000in}}{%
\pgfpathmoveto{\pgfqpoint{-0.000000in}{0.000000in}}%
\pgfpathlineto{\pgfqpoint{-0.048611in}{0.000000in}}%
\pgfusepath{stroke,fill}%
}%
\begin{pgfscope}%
\pgfsys@transformshift{0.750000in}{3.360414in}%
\pgfsys@useobject{currentmarker}{}%
\end{pgfscope}%
\end{pgfscope}%
\begin{pgfscope}%
\definecolor{textcolor}{rgb}{0.000000,0.000000,0.000000}%
\pgfsetstrokecolor{textcolor}%
\pgfsetfillcolor{textcolor}%
\pgftext[x=0.362657in, y=3.302543in, left, base]{\color{textcolor}\rmfamily\fontsize{13.000000}{15.600000}\selectfont \(\displaystyle {0.07}\)}%
\end{pgfscope}%
\begin{pgfscope}%
\definecolor{textcolor}{rgb}{0.000000,0.000000,0.000000}%
\pgfsetstrokecolor{textcolor}%
\pgfsetfillcolor{textcolor}%
\pgftext[x=0.307102in,y=2.010000in,,bottom,rotate=90.000000]{\color{textcolor}\rmfamily\fontsize{13.000000}{15.600000}\selectfont Probability}%
\end{pgfscope}%
\begin{pgfscope}%
\pgfsetrectcap%
\pgfsetmiterjoin%
\pgfsetlinewidth{0.803000pt}%
\definecolor{currentstroke}{rgb}{0.000000,0.000000,0.000000}%
\pgfsetstrokecolor{currentstroke}%
\pgfsetdash{}{0pt}%
\pgfpathmoveto{\pgfqpoint{0.750000in}{0.500000in}}%
\pgfpathlineto{\pgfqpoint{0.750000in}{3.520000in}}%
\pgfusepath{stroke}%
\end{pgfscope}%
\begin{pgfscope}%
\pgfsetrectcap%
\pgfsetmiterjoin%
\pgfsetlinewidth{0.803000pt}%
\definecolor{currentstroke}{rgb}{0.000000,0.000000,0.000000}%
\pgfsetstrokecolor{currentstroke}%
\pgfsetdash{}{0pt}%
\pgfpathmoveto{\pgfqpoint{5.400000in}{0.500000in}}%
\pgfpathlineto{\pgfqpoint{5.400000in}{3.520000in}}%
\pgfusepath{stroke}%
\end{pgfscope}%
\begin{pgfscope}%
\pgfsetrectcap%
\pgfsetmiterjoin%
\pgfsetlinewidth{0.803000pt}%
\definecolor{currentstroke}{rgb}{0.000000,0.000000,0.000000}%
\pgfsetstrokecolor{currentstroke}%
\pgfsetdash{}{0pt}%
\pgfpathmoveto{\pgfqpoint{0.750000in}{0.500000in}}%
\pgfpathlineto{\pgfqpoint{5.400000in}{0.500000in}}%
\pgfusepath{stroke}%
\end{pgfscope}%
\begin{pgfscope}%
\pgfsetrectcap%
\pgfsetmiterjoin%
\pgfsetlinewidth{0.803000pt}%
\definecolor{currentstroke}{rgb}{0.000000,0.000000,0.000000}%
\pgfsetstrokecolor{currentstroke}%
\pgfsetdash{}{0pt}%
\pgfpathmoveto{\pgfqpoint{0.750000in}{3.520000in}}%
\pgfpathlineto{\pgfqpoint{5.400000in}{3.520000in}}%
\pgfusepath{stroke}%
\end{pgfscope}%
\begin{pgfscope}%
\definecolor{textcolor}{rgb}{0.000000,0.000000,0.000000}%
\pgfsetstrokecolor{textcolor}%
\pgfsetfillcolor{textcolor}%
\pgftext[x=3.075000in,y=3.603333in,,base]{\color{textcolor}\rmfamily\fontsize{13.000000}{15.600000}\selectfont PDF for random bits \(\displaystyle 2x\)}%
\end{pgfscope}%
\begin{pgfscope}%
\pgfsetbuttcap%
\pgfsetmiterjoin%
\definecolor{currentfill}{rgb}{1.000000,1.000000,1.000000}%
\pgfsetfillcolor{currentfill}%
\pgfsetfillopacity{0.800000}%
\pgfsetlinewidth{1.003750pt}%
\definecolor{currentstroke}{rgb}{0.800000,0.800000,0.800000}%
\pgfsetstrokecolor{currentstroke}%
\pgfsetstrokeopacity{0.800000}%
\pgfsetdash{}{0pt}%
\pgfpathmoveto{\pgfqpoint{3.210355in}{2.628334in}}%
\pgfpathlineto{\pgfqpoint{5.273611in}{2.628334in}}%
\pgfpathquadraticcurveto{\pgfqpoint{5.309722in}{2.628334in}}{\pgfqpoint{5.309722in}{2.664445in}}%
\pgfpathlineto{\pgfqpoint{5.309722in}{3.393611in}}%
\pgfpathquadraticcurveto{\pgfqpoint{5.309722in}{3.429722in}}{\pgfqpoint{5.273611in}{3.429722in}}%
\pgfpathlineto{\pgfqpoint{3.210355in}{3.429722in}}%
\pgfpathquadraticcurveto{\pgfqpoint{3.174244in}{3.429722in}}{\pgfqpoint{3.174244in}{3.393611in}}%
\pgfpathlineto{\pgfqpoint{3.174244in}{2.664445in}}%
\pgfpathquadraticcurveto{\pgfqpoint{3.174244in}{2.628334in}}{\pgfqpoint{3.210355in}{2.628334in}}%
\pgfpathlineto{\pgfqpoint{3.210355in}{2.628334in}}%
\pgfpathclose%
\pgfusepath{stroke,fill}%
\end{pgfscope}%
\begin{pgfscope}%
\pgfsetbuttcap%
\pgfsetroundjoin%
\definecolor{currentfill}{rgb}{1.000000,0.000000,0.000000}%
\pgfsetfillcolor{currentfill}%
\pgfsetlinewidth{1.003750pt}%
\definecolor{currentstroke}{rgb}{1.000000,0.000000,0.000000}%
\pgfsetstrokecolor{currentstroke}%
\pgfsetdash{}{0pt}%
\pgfsys@defobject{currentmarker}{\pgfqpoint{-0.041667in}{-0.041667in}}{\pgfqpoint{0.041667in}{0.041667in}}{%
\pgfpathmoveto{\pgfqpoint{0.000000in}{-0.041667in}}%
\pgfpathcurveto{\pgfqpoint{0.011050in}{-0.041667in}}{\pgfqpoint{0.021649in}{-0.037276in}}{\pgfqpoint{0.029463in}{-0.029463in}}%
\pgfpathcurveto{\pgfqpoint{0.037276in}{-0.021649in}}{\pgfqpoint{0.041667in}{-0.011050in}}{\pgfqpoint{0.041667in}{0.000000in}}%
\pgfpathcurveto{\pgfqpoint{0.041667in}{0.011050in}}{\pgfqpoint{0.037276in}{0.021649in}}{\pgfqpoint{0.029463in}{0.029463in}}%
\pgfpathcurveto{\pgfqpoint{0.021649in}{0.037276in}}{\pgfqpoint{0.011050in}{0.041667in}}{\pgfqpoint{0.000000in}{0.041667in}}%
\pgfpathcurveto{\pgfqpoint{-0.011050in}{0.041667in}}{\pgfqpoint{-0.021649in}{0.037276in}}{\pgfqpoint{-0.029463in}{0.029463in}}%
\pgfpathcurveto{\pgfqpoint{-0.037276in}{0.021649in}}{\pgfqpoint{-0.041667in}{0.011050in}}{\pgfqpoint{-0.041667in}{0.000000in}}%
\pgfpathcurveto{\pgfqpoint{-0.041667in}{-0.011050in}}{\pgfqpoint{-0.037276in}{-0.021649in}}{\pgfqpoint{-0.029463in}{-0.029463in}}%
\pgfpathcurveto{\pgfqpoint{-0.021649in}{-0.037276in}}{\pgfqpoint{-0.011050in}{-0.041667in}}{\pgfqpoint{0.000000in}{-0.041667in}}%
\pgfpathlineto{\pgfqpoint{0.000000in}{-0.041667in}}%
\pgfpathclose%
\pgfusepath{stroke,fill}%
}%
\begin{pgfscope}%
\pgfsys@transformshift{3.427022in}{3.278507in}%
\pgfsys@useobject{currentmarker}{}%
\end{pgfscope}%
\end{pgfscope}%
\begin{pgfscope}%
\definecolor{textcolor}{rgb}{0.000000,0.000000,0.000000}%
\pgfsetstrokecolor{textcolor}%
\pgfsetfillcolor{textcolor}%
\pgftext[x=3.752022in,y=3.231111in,left,base]{\color{textcolor}\rmfamily\fontsize{13.000000}{15.600000}\selectfont SNN}%
\end{pgfscope}%
\begin{pgfscope}%
\pgfsetbuttcap%
\pgfsetroundjoin%
\definecolor{currentfill}{rgb}{0.000000,0.500000,0.000000}%
\pgfsetfillcolor{currentfill}%
\pgfsetlinewidth{1.003750pt}%
\definecolor{currentstroke}{rgb}{0.000000,0.500000,0.000000}%
\pgfsetstrokecolor{currentstroke}%
\pgfsetdash{}{0pt}%
\pgfsys@defobject{currentmarker}{\pgfqpoint{-0.041667in}{-0.041667in}}{\pgfqpoint{0.041667in}{0.041667in}}{%
\pgfpathmoveto{\pgfqpoint{0.000000in}{-0.041667in}}%
\pgfpathcurveto{\pgfqpoint{0.011050in}{-0.041667in}}{\pgfqpoint{0.021649in}{-0.037276in}}{\pgfqpoint{0.029463in}{-0.029463in}}%
\pgfpathcurveto{\pgfqpoint{0.037276in}{-0.021649in}}{\pgfqpoint{0.041667in}{-0.011050in}}{\pgfqpoint{0.041667in}{0.000000in}}%
\pgfpathcurveto{\pgfqpoint{0.041667in}{0.011050in}}{\pgfqpoint{0.037276in}{0.021649in}}{\pgfqpoint{0.029463in}{0.029463in}}%
\pgfpathcurveto{\pgfqpoint{0.021649in}{0.037276in}}{\pgfqpoint{0.011050in}{0.041667in}}{\pgfqpoint{0.000000in}{0.041667in}}%
\pgfpathcurveto{\pgfqpoint{-0.011050in}{0.041667in}}{\pgfqpoint{-0.021649in}{0.037276in}}{\pgfqpoint{-0.029463in}{0.029463in}}%
\pgfpathcurveto{\pgfqpoint{-0.037276in}{0.021649in}}{\pgfqpoint{-0.041667in}{0.011050in}}{\pgfqpoint{-0.041667in}{0.000000in}}%
\pgfpathcurveto{\pgfqpoint{-0.041667in}{-0.011050in}}{\pgfqpoint{-0.037276in}{-0.021649in}}{\pgfqpoint{-0.029463in}{-0.029463in}}%
\pgfpathcurveto{\pgfqpoint{-0.021649in}{-0.037276in}}{\pgfqpoint{-0.011050in}{-0.041667in}}{\pgfqpoint{0.000000in}{-0.041667in}}%
\pgfpathlineto{\pgfqpoint{0.000000in}{-0.041667in}}%
\pgfpathclose%
\pgfusepath{stroke,fill}%
}%
\begin{pgfscope}%
\pgfsys@transformshift{3.427022in}{3.029433in}%
\pgfsys@useobject{currentmarker}{}%
\end{pgfscope}%
\end{pgfscope}%
\begin{pgfscope}%
\definecolor{textcolor}{rgb}{0.000000,0.000000,0.000000}%
\pgfsetstrokecolor{textcolor}%
\pgfsetfillcolor{textcolor}%
\pgftext[x=3.752022in,y=2.982037in,left,base]{\color{textcolor}\rmfamily\fontsize{13.000000}{15.600000}\selectfont NN}%
\end{pgfscope}%
\begin{pgfscope}%
\pgfsetbuttcap%
\pgfsetmiterjoin%
\definecolor{currentfill}{rgb}{0.121569,0.466667,0.705882}%
\pgfsetfillcolor{currentfill}%
\pgfsetlinewidth{0.000000pt}%
\definecolor{currentstroke}{rgb}{0.000000,0.000000,0.000000}%
\pgfsetstrokecolor{currentstroke}%
\pgfsetstrokeopacity{0.000000}%
\pgfsetdash{}{0pt}%
\pgfpathmoveto{\pgfqpoint{3.246466in}{2.732964in}}%
\pgfpathlineto{\pgfqpoint{3.607577in}{2.732964in}}%
\pgfpathlineto{\pgfqpoint{3.607577in}{2.859352in}}%
\pgfpathlineto{\pgfqpoint{3.246466in}{2.859352in}}%
\pgfpathlineto{\pgfqpoint{3.246466in}{2.732964in}}%
\pgfpathclose%
\pgfusepath{fill}%
\end{pgfscope}%
\begin{pgfscope}%
\definecolor{textcolor}{rgb}{0.000000,0.000000,0.000000}%
\pgfsetstrokecolor{textcolor}%
\pgfsetfillcolor{textcolor}%
\pgftext[x=3.752022in,y=2.732964in,left,base]{\color{textcolor}\rmfamily\fontsize{13.000000}{15.600000}\selectfont Binomial distributon}%
\end{pgfscope}%
\end{pgfpicture}%
\makeatother%
\endgroup%

    \caption{Caption}
    \label{fig:my_label}
\end{figure}


\begin{figure}
%% Creator: Matplotlib, PGF backend
%%
%% To include the figure in your LaTeX document, write
%%   \input{<filename>.pgf}
%%
%% Make sure the required packages are loaded in your preamble
%%   \usepackage{pgf}
%%
%% Also ensure that all the required font packages are loaded; for instance,
%% the lmodern package is sometimes necessary when using math font.
%%   \usepackage{lmodern}
%%
%% Figures using additional raster images can only be included by \input if
%% they are in the same directory as the main LaTeX file. For loading figures
%% from other directories you can use the `import` package
%%   \usepackage{import}
%%
%% and then include the figures with
%%   \import{<path to file>}{<filename>.pgf}
%%
%% Matplotlib used the following preamble
%%   \usepackage{fontspec}
%%   \setmainfont{DejaVuSerif.ttf}[Path=\detokenize{C:/I/python38/Lib/site-packages/matplotlib/mpl-data/fonts/ttf/}]
%%   \setsansfont{DejaVuSans.ttf}[Path=\detokenize{C:/I/python38/Lib/site-packages/matplotlib/mpl-data/fonts/ttf/}]
%%   \setmonofont{DejaVuSansMono.ttf}[Path=\detokenize{C:/I/python38/Lib/site-packages/matplotlib/mpl-data/fonts/ttf/}]
%%
\begingroup%
\makeatletter%
\begin{pgfpicture}%
\pgfpathrectangle{\pgfpointorigin}{\pgfqpoint{6.000000in}{4.000000in}}%
\pgfusepath{use as bounding box, clip}%
\begin{pgfscope}%
\pgfsetbuttcap%
\pgfsetmiterjoin%
\pgfsetlinewidth{0.000000pt}%
\definecolor{currentstroke}{rgb}{1.000000,1.000000,1.000000}%
\pgfsetstrokecolor{currentstroke}%
\pgfsetstrokeopacity{0.000000}%
\pgfsetdash{}{0pt}%
\pgfpathmoveto{\pgfqpoint{0.000000in}{0.000000in}}%
\pgfpathlineto{\pgfqpoint{6.000000in}{0.000000in}}%
\pgfpathlineto{\pgfqpoint{6.000000in}{4.000000in}}%
\pgfpathlineto{\pgfqpoint{0.000000in}{4.000000in}}%
\pgfpathlineto{\pgfqpoint{0.000000in}{0.000000in}}%
\pgfpathclose%
\pgfusepath{}%
\end{pgfscope}%
\begin{pgfscope}%
\pgfsetbuttcap%
\pgfsetmiterjoin%
\definecolor{currentfill}{rgb}{1.000000,1.000000,1.000000}%
\pgfsetfillcolor{currentfill}%
\pgfsetlinewidth{0.000000pt}%
\definecolor{currentstroke}{rgb}{0.000000,0.000000,0.000000}%
\pgfsetstrokecolor{currentstroke}%
\pgfsetstrokeopacity{0.000000}%
\pgfsetdash{}{0pt}%
\pgfpathmoveto{\pgfqpoint{0.750000in}{0.500000in}}%
\pgfpathlineto{\pgfqpoint{5.400000in}{0.500000in}}%
\pgfpathlineto{\pgfqpoint{5.400000in}{3.520000in}}%
\pgfpathlineto{\pgfqpoint{0.750000in}{3.520000in}}%
\pgfpathlineto{\pgfqpoint{0.750000in}{0.500000in}}%
\pgfpathclose%
\pgfusepath{fill}%
\end{pgfscope}%
\begin{pgfscope}%
\pgfsetbuttcap%
\pgfsetroundjoin%
\definecolor{currentfill}{rgb}{0.000000,0.000000,0.000000}%
\pgfsetfillcolor{currentfill}%
\pgfsetlinewidth{0.803000pt}%
\definecolor{currentstroke}{rgb}{0.000000,0.000000,0.000000}%
\pgfsetstrokecolor{currentstroke}%
\pgfsetdash{}{0pt}%
\pgfsys@defobject{currentmarker}{\pgfqpoint{0.000000in}{-0.048611in}}{\pgfqpoint{0.000000in}{0.000000in}}{%
\pgfpathmoveto{\pgfqpoint{0.000000in}{0.000000in}}%
\pgfpathlineto{\pgfqpoint{0.000000in}{-0.048611in}}%
\pgfusepath{stroke,fill}%
}%
\begin{pgfscope}%
\pgfsys@transformshift{0.961364in}{0.500000in}%
\pgfsys@useobject{currentmarker}{}%
\end{pgfscope}%
\end{pgfscope}%
\begin{pgfscope}%
\definecolor{textcolor}{rgb}{0.000000,0.000000,0.000000}%
\pgfsetstrokecolor{textcolor}%
\pgfsetfillcolor{textcolor}%
\pgftext[x=0.961364in,y=0.402778in,,top]{\color{textcolor}\sffamily\fontsize{10.000000}{12.000000}\selectfont 0}%
\end{pgfscope}%
\begin{pgfscope}%
\pgfsetbuttcap%
\pgfsetroundjoin%
\definecolor{currentfill}{rgb}{0.000000,0.000000,0.000000}%
\pgfsetfillcolor{currentfill}%
\pgfsetlinewidth{0.803000pt}%
\definecolor{currentstroke}{rgb}{0.000000,0.000000,0.000000}%
\pgfsetstrokecolor{currentstroke}%
\pgfsetdash{}{0pt}%
\pgfsys@defobject{currentmarker}{\pgfqpoint{0.000000in}{-0.048611in}}{\pgfqpoint{0.000000in}{0.000000in}}{%
\pgfpathmoveto{\pgfqpoint{0.000000in}{0.000000in}}%
\pgfpathlineto{\pgfqpoint{0.000000in}{-0.048611in}}%
\pgfusepath{stroke,fill}%
}%
\begin{pgfscope}%
\pgfsys@transformshift{1.806987in}{0.500000in}%
\pgfsys@useobject{currentmarker}{}%
\end{pgfscope}%
\end{pgfscope}%
\begin{pgfscope}%
\definecolor{textcolor}{rgb}{0.000000,0.000000,0.000000}%
\pgfsetstrokecolor{textcolor}%
\pgfsetfillcolor{textcolor}%
\pgftext[x=1.806987in,y=0.402778in,,top]{\color{textcolor}\sffamily\fontsize{10.000000}{12.000000}\selectfont 1000}%
\end{pgfscope}%
\begin{pgfscope}%
\pgfsetbuttcap%
\pgfsetroundjoin%
\definecolor{currentfill}{rgb}{0.000000,0.000000,0.000000}%
\pgfsetfillcolor{currentfill}%
\pgfsetlinewidth{0.803000pt}%
\definecolor{currentstroke}{rgb}{0.000000,0.000000,0.000000}%
\pgfsetstrokecolor{currentstroke}%
\pgfsetdash{}{0pt}%
\pgfsys@defobject{currentmarker}{\pgfqpoint{0.000000in}{-0.048611in}}{\pgfqpoint{0.000000in}{0.000000in}}{%
\pgfpathmoveto{\pgfqpoint{0.000000in}{0.000000in}}%
\pgfpathlineto{\pgfqpoint{0.000000in}{-0.048611in}}%
\pgfusepath{stroke,fill}%
}%
\begin{pgfscope}%
\pgfsys@transformshift{2.652611in}{0.500000in}%
\pgfsys@useobject{currentmarker}{}%
\end{pgfscope}%
\end{pgfscope}%
\begin{pgfscope}%
\definecolor{textcolor}{rgb}{0.000000,0.000000,0.000000}%
\pgfsetstrokecolor{textcolor}%
\pgfsetfillcolor{textcolor}%
\pgftext[x=2.652611in,y=0.402778in,,top]{\color{textcolor}\sffamily\fontsize{10.000000}{12.000000}\selectfont 2000}%
\end{pgfscope}%
\begin{pgfscope}%
\pgfsetbuttcap%
\pgfsetroundjoin%
\definecolor{currentfill}{rgb}{0.000000,0.000000,0.000000}%
\pgfsetfillcolor{currentfill}%
\pgfsetlinewidth{0.803000pt}%
\definecolor{currentstroke}{rgb}{0.000000,0.000000,0.000000}%
\pgfsetstrokecolor{currentstroke}%
\pgfsetdash{}{0pt}%
\pgfsys@defobject{currentmarker}{\pgfqpoint{0.000000in}{-0.048611in}}{\pgfqpoint{0.000000in}{0.000000in}}{%
\pgfpathmoveto{\pgfqpoint{0.000000in}{0.000000in}}%
\pgfpathlineto{\pgfqpoint{0.000000in}{-0.048611in}}%
\pgfusepath{stroke,fill}%
}%
\begin{pgfscope}%
\pgfsys@transformshift{3.498235in}{0.500000in}%
\pgfsys@useobject{currentmarker}{}%
\end{pgfscope}%
\end{pgfscope}%
\begin{pgfscope}%
\definecolor{textcolor}{rgb}{0.000000,0.000000,0.000000}%
\pgfsetstrokecolor{textcolor}%
\pgfsetfillcolor{textcolor}%
\pgftext[x=3.498235in,y=0.402778in,,top]{\color{textcolor}\sffamily\fontsize{10.000000}{12.000000}\selectfont 3000}%
\end{pgfscope}%
\begin{pgfscope}%
\pgfsetbuttcap%
\pgfsetroundjoin%
\definecolor{currentfill}{rgb}{0.000000,0.000000,0.000000}%
\pgfsetfillcolor{currentfill}%
\pgfsetlinewidth{0.803000pt}%
\definecolor{currentstroke}{rgb}{0.000000,0.000000,0.000000}%
\pgfsetstrokecolor{currentstroke}%
\pgfsetdash{}{0pt}%
\pgfsys@defobject{currentmarker}{\pgfqpoint{0.000000in}{-0.048611in}}{\pgfqpoint{0.000000in}{0.000000in}}{%
\pgfpathmoveto{\pgfqpoint{0.000000in}{0.000000in}}%
\pgfpathlineto{\pgfqpoint{0.000000in}{-0.048611in}}%
\pgfusepath{stroke,fill}%
}%
\begin{pgfscope}%
\pgfsys@transformshift{4.343858in}{0.500000in}%
\pgfsys@useobject{currentmarker}{}%
\end{pgfscope}%
\end{pgfscope}%
\begin{pgfscope}%
\definecolor{textcolor}{rgb}{0.000000,0.000000,0.000000}%
\pgfsetstrokecolor{textcolor}%
\pgfsetfillcolor{textcolor}%
\pgftext[x=4.343858in,y=0.402778in,,top]{\color{textcolor}\sffamily\fontsize{10.000000}{12.000000}\selectfont 4000}%
\end{pgfscope}%
\begin{pgfscope}%
\pgfsetbuttcap%
\pgfsetroundjoin%
\definecolor{currentfill}{rgb}{0.000000,0.000000,0.000000}%
\pgfsetfillcolor{currentfill}%
\pgfsetlinewidth{0.803000pt}%
\definecolor{currentstroke}{rgb}{0.000000,0.000000,0.000000}%
\pgfsetstrokecolor{currentstroke}%
\pgfsetdash{}{0pt}%
\pgfsys@defobject{currentmarker}{\pgfqpoint{0.000000in}{-0.048611in}}{\pgfqpoint{0.000000in}{0.000000in}}{%
\pgfpathmoveto{\pgfqpoint{0.000000in}{0.000000in}}%
\pgfpathlineto{\pgfqpoint{0.000000in}{-0.048611in}}%
\pgfusepath{stroke,fill}%
}%
\begin{pgfscope}%
\pgfsys@transformshift{5.189482in}{0.500000in}%
\pgfsys@useobject{currentmarker}{}%
\end{pgfscope}%
\end{pgfscope}%
\begin{pgfscope}%
\definecolor{textcolor}{rgb}{0.000000,0.000000,0.000000}%
\pgfsetstrokecolor{textcolor}%
\pgfsetfillcolor{textcolor}%
\pgftext[x=5.189482in,y=0.402778in,,top]{\color{textcolor}\sffamily\fontsize{10.000000}{12.000000}\selectfont 5000}%
\end{pgfscope}%
\begin{pgfscope}%
\definecolor{textcolor}{rgb}{0.000000,0.000000,0.000000}%
\pgfsetstrokecolor{textcolor}%
\pgfsetfillcolor{textcolor}%
\pgftext[x=3.075000in,y=0.212809in,,top]{\color{textcolor}\sffamily\fontsize{10.000000}{12.000000}\selectfont iteration}%
\end{pgfscope}%
\begin{pgfscope}%
\pgfsetbuttcap%
\pgfsetroundjoin%
\definecolor{currentfill}{rgb}{0.000000,0.000000,0.000000}%
\pgfsetfillcolor{currentfill}%
\pgfsetlinewidth{0.803000pt}%
\definecolor{currentstroke}{rgb}{0.000000,0.000000,0.000000}%
\pgfsetstrokecolor{currentstroke}%
\pgfsetdash{}{0pt}%
\pgfsys@defobject{currentmarker}{\pgfqpoint{-0.048611in}{0.000000in}}{\pgfqpoint{-0.000000in}{0.000000in}}{%
\pgfpathmoveto{\pgfqpoint{-0.000000in}{0.000000in}}%
\pgfpathlineto{\pgfqpoint{-0.048611in}{0.000000in}}%
\pgfusepath{stroke,fill}%
}%
\begin{pgfscope}%
\pgfsys@transformshift{0.750000in}{0.654724in}%
\pgfsys@useobject{currentmarker}{}%
\end{pgfscope}%
\end{pgfscope}%
\begin{pgfscope}%
\definecolor{textcolor}{rgb}{0.000000,0.000000,0.000000}%
\pgfsetstrokecolor{textcolor}%
\pgfsetfillcolor{textcolor}%
\pgftext[x=0.255168in, y=0.601963in, left, base]{\color{textcolor}\sffamily\fontsize{10.000000}{12.000000}\selectfont 0.400}%
\end{pgfscope}%
\begin{pgfscope}%
\pgfsetbuttcap%
\pgfsetroundjoin%
\definecolor{currentfill}{rgb}{0.000000,0.000000,0.000000}%
\pgfsetfillcolor{currentfill}%
\pgfsetlinewidth{0.803000pt}%
\definecolor{currentstroke}{rgb}{0.000000,0.000000,0.000000}%
\pgfsetstrokecolor{currentstroke}%
\pgfsetdash{}{0pt}%
\pgfsys@defobject{currentmarker}{\pgfqpoint{-0.048611in}{0.000000in}}{\pgfqpoint{-0.000000in}{0.000000in}}{%
\pgfpathmoveto{\pgfqpoint{-0.000000in}{0.000000in}}%
\pgfpathlineto{\pgfqpoint{-0.048611in}{0.000000in}}%
\pgfusepath{stroke,fill}%
}%
\begin{pgfscope}%
\pgfsys@transformshift{0.750000in}{0.993082in}%
\pgfsys@useobject{currentmarker}{}%
\end{pgfscope}%
\end{pgfscope}%
\begin{pgfscope}%
\definecolor{textcolor}{rgb}{0.000000,0.000000,0.000000}%
\pgfsetstrokecolor{textcolor}%
\pgfsetfillcolor{textcolor}%
\pgftext[x=0.255168in, y=0.940320in, left, base]{\color{textcolor}\sffamily\fontsize{10.000000}{12.000000}\selectfont 0.425}%
\end{pgfscope}%
\begin{pgfscope}%
\pgfsetbuttcap%
\pgfsetroundjoin%
\definecolor{currentfill}{rgb}{0.000000,0.000000,0.000000}%
\pgfsetfillcolor{currentfill}%
\pgfsetlinewidth{0.803000pt}%
\definecolor{currentstroke}{rgb}{0.000000,0.000000,0.000000}%
\pgfsetstrokecolor{currentstroke}%
\pgfsetdash{}{0pt}%
\pgfsys@defobject{currentmarker}{\pgfqpoint{-0.048611in}{0.000000in}}{\pgfqpoint{-0.000000in}{0.000000in}}{%
\pgfpathmoveto{\pgfqpoint{-0.000000in}{0.000000in}}%
\pgfpathlineto{\pgfqpoint{-0.048611in}{0.000000in}}%
\pgfusepath{stroke,fill}%
}%
\begin{pgfscope}%
\pgfsys@transformshift{0.750000in}{1.331440in}%
\pgfsys@useobject{currentmarker}{}%
\end{pgfscope}%
\end{pgfscope}%
\begin{pgfscope}%
\definecolor{textcolor}{rgb}{0.000000,0.000000,0.000000}%
\pgfsetstrokecolor{textcolor}%
\pgfsetfillcolor{textcolor}%
\pgftext[x=0.255168in, y=1.278678in, left, base]{\color{textcolor}\sffamily\fontsize{10.000000}{12.000000}\selectfont 0.450}%
\end{pgfscope}%
\begin{pgfscope}%
\pgfsetbuttcap%
\pgfsetroundjoin%
\definecolor{currentfill}{rgb}{0.000000,0.000000,0.000000}%
\pgfsetfillcolor{currentfill}%
\pgfsetlinewidth{0.803000pt}%
\definecolor{currentstroke}{rgb}{0.000000,0.000000,0.000000}%
\pgfsetstrokecolor{currentstroke}%
\pgfsetdash{}{0pt}%
\pgfsys@defobject{currentmarker}{\pgfqpoint{-0.048611in}{0.000000in}}{\pgfqpoint{-0.000000in}{0.000000in}}{%
\pgfpathmoveto{\pgfqpoint{-0.000000in}{0.000000in}}%
\pgfpathlineto{\pgfqpoint{-0.048611in}{0.000000in}}%
\pgfusepath{stroke,fill}%
}%
\begin{pgfscope}%
\pgfsys@transformshift{0.750000in}{1.669797in}%
\pgfsys@useobject{currentmarker}{}%
\end{pgfscope}%
\end{pgfscope}%
\begin{pgfscope}%
\definecolor{textcolor}{rgb}{0.000000,0.000000,0.000000}%
\pgfsetstrokecolor{textcolor}%
\pgfsetfillcolor{textcolor}%
\pgftext[x=0.255168in, y=1.617036in, left, base]{\color{textcolor}\sffamily\fontsize{10.000000}{12.000000}\selectfont 0.475}%
\end{pgfscope}%
\begin{pgfscope}%
\pgfsetbuttcap%
\pgfsetroundjoin%
\definecolor{currentfill}{rgb}{0.000000,0.000000,0.000000}%
\pgfsetfillcolor{currentfill}%
\pgfsetlinewidth{0.803000pt}%
\definecolor{currentstroke}{rgb}{0.000000,0.000000,0.000000}%
\pgfsetstrokecolor{currentstroke}%
\pgfsetdash{}{0pt}%
\pgfsys@defobject{currentmarker}{\pgfqpoint{-0.048611in}{0.000000in}}{\pgfqpoint{-0.000000in}{0.000000in}}{%
\pgfpathmoveto{\pgfqpoint{-0.000000in}{0.000000in}}%
\pgfpathlineto{\pgfqpoint{-0.048611in}{0.000000in}}%
\pgfusepath{stroke,fill}%
}%
\begin{pgfscope}%
\pgfsys@transformshift{0.750000in}{2.008155in}%
\pgfsys@useobject{currentmarker}{}%
\end{pgfscope}%
\end{pgfscope}%
\begin{pgfscope}%
\definecolor{textcolor}{rgb}{0.000000,0.000000,0.000000}%
\pgfsetstrokecolor{textcolor}%
\pgfsetfillcolor{textcolor}%
\pgftext[x=0.255168in, y=1.955393in, left, base]{\color{textcolor}\sffamily\fontsize{10.000000}{12.000000}\selectfont 0.500}%
\end{pgfscope}%
\begin{pgfscope}%
\pgfsetbuttcap%
\pgfsetroundjoin%
\definecolor{currentfill}{rgb}{0.000000,0.000000,0.000000}%
\pgfsetfillcolor{currentfill}%
\pgfsetlinewidth{0.803000pt}%
\definecolor{currentstroke}{rgb}{0.000000,0.000000,0.000000}%
\pgfsetstrokecolor{currentstroke}%
\pgfsetdash{}{0pt}%
\pgfsys@defobject{currentmarker}{\pgfqpoint{-0.048611in}{0.000000in}}{\pgfqpoint{-0.000000in}{0.000000in}}{%
\pgfpathmoveto{\pgfqpoint{-0.000000in}{0.000000in}}%
\pgfpathlineto{\pgfqpoint{-0.048611in}{0.000000in}}%
\pgfusepath{stroke,fill}%
}%
\begin{pgfscope}%
\pgfsys@transformshift{0.750000in}{2.346513in}%
\pgfsys@useobject{currentmarker}{}%
\end{pgfscope}%
\end{pgfscope}%
\begin{pgfscope}%
\definecolor{textcolor}{rgb}{0.000000,0.000000,0.000000}%
\pgfsetstrokecolor{textcolor}%
\pgfsetfillcolor{textcolor}%
\pgftext[x=0.255168in, y=2.293751in, left, base]{\color{textcolor}\sffamily\fontsize{10.000000}{12.000000}\selectfont 0.525}%
\end{pgfscope}%
\begin{pgfscope}%
\pgfsetbuttcap%
\pgfsetroundjoin%
\definecolor{currentfill}{rgb}{0.000000,0.000000,0.000000}%
\pgfsetfillcolor{currentfill}%
\pgfsetlinewidth{0.803000pt}%
\definecolor{currentstroke}{rgb}{0.000000,0.000000,0.000000}%
\pgfsetstrokecolor{currentstroke}%
\pgfsetdash{}{0pt}%
\pgfsys@defobject{currentmarker}{\pgfqpoint{-0.048611in}{0.000000in}}{\pgfqpoint{-0.000000in}{0.000000in}}{%
\pgfpathmoveto{\pgfqpoint{-0.000000in}{0.000000in}}%
\pgfpathlineto{\pgfqpoint{-0.048611in}{0.000000in}}%
\pgfusepath{stroke,fill}%
}%
\begin{pgfscope}%
\pgfsys@transformshift{0.750000in}{2.684870in}%
\pgfsys@useobject{currentmarker}{}%
\end{pgfscope}%
\end{pgfscope}%
\begin{pgfscope}%
\definecolor{textcolor}{rgb}{0.000000,0.000000,0.000000}%
\pgfsetstrokecolor{textcolor}%
\pgfsetfillcolor{textcolor}%
\pgftext[x=0.255168in, y=2.632109in, left, base]{\color{textcolor}\sffamily\fontsize{10.000000}{12.000000}\selectfont 0.550}%
\end{pgfscope}%
\begin{pgfscope}%
\pgfsetbuttcap%
\pgfsetroundjoin%
\definecolor{currentfill}{rgb}{0.000000,0.000000,0.000000}%
\pgfsetfillcolor{currentfill}%
\pgfsetlinewidth{0.803000pt}%
\definecolor{currentstroke}{rgb}{0.000000,0.000000,0.000000}%
\pgfsetstrokecolor{currentstroke}%
\pgfsetdash{}{0pt}%
\pgfsys@defobject{currentmarker}{\pgfqpoint{-0.048611in}{0.000000in}}{\pgfqpoint{-0.000000in}{0.000000in}}{%
\pgfpathmoveto{\pgfqpoint{-0.000000in}{0.000000in}}%
\pgfpathlineto{\pgfqpoint{-0.048611in}{0.000000in}}%
\pgfusepath{stroke,fill}%
}%
\begin{pgfscope}%
\pgfsys@transformshift{0.750000in}{3.023228in}%
\pgfsys@useobject{currentmarker}{}%
\end{pgfscope}%
\end{pgfscope}%
\begin{pgfscope}%
\definecolor{textcolor}{rgb}{0.000000,0.000000,0.000000}%
\pgfsetstrokecolor{textcolor}%
\pgfsetfillcolor{textcolor}%
\pgftext[x=0.255168in, y=2.970466in, left, base]{\color{textcolor}\sffamily\fontsize{10.000000}{12.000000}\selectfont 0.575}%
\end{pgfscope}%
\begin{pgfscope}%
\pgfsetbuttcap%
\pgfsetroundjoin%
\definecolor{currentfill}{rgb}{0.000000,0.000000,0.000000}%
\pgfsetfillcolor{currentfill}%
\pgfsetlinewidth{0.803000pt}%
\definecolor{currentstroke}{rgb}{0.000000,0.000000,0.000000}%
\pgfsetstrokecolor{currentstroke}%
\pgfsetdash{}{0pt}%
\pgfsys@defobject{currentmarker}{\pgfqpoint{-0.048611in}{0.000000in}}{\pgfqpoint{-0.000000in}{0.000000in}}{%
\pgfpathmoveto{\pgfqpoint{-0.000000in}{0.000000in}}%
\pgfpathlineto{\pgfqpoint{-0.048611in}{0.000000in}}%
\pgfusepath{stroke,fill}%
}%
\begin{pgfscope}%
\pgfsys@transformshift{0.750000in}{3.361586in}%
\pgfsys@useobject{currentmarker}{}%
\end{pgfscope}%
\end{pgfscope}%
\begin{pgfscope}%
\definecolor{textcolor}{rgb}{0.000000,0.000000,0.000000}%
\pgfsetstrokecolor{textcolor}%
\pgfsetfillcolor{textcolor}%
\pgftext[x=0.255168in, y=3.308824in, left, base]{\color{textcolor}\sffamily\fontsize{10.000000}{12.000000}\selectfont 0.600}%
\end{pgfscope}%
\begin{pgfscope}%
\definecolor{textcolor}{rgb}{0.000000,0.000000,0.000000}%
\pgfsetstrokecolor{textcolor}%
\pgfsetfillcolor{textcolor}%
\pgftext[x=0.199612in,y=2.010000in,,bottom,rotate=90.000000]{\color{textcolor}\sffamily\fontsize{10.000000}{12.000000}\selectfont loss function value}%
\end{pgfscope}%
\begin{pgfscope}%
\pgfpathrectangle{\pgfqpoint{0.750000in}{0.500000in}}{\pgfqpoint{4.650000in}{3.020000in}}%
\pgfusepath{clip}%
\pgfsetrectcap%
\pgfsetroundjoin%
\pgfsetlinewidth{1.505625pt}%
\definecolor{currentstroke}{rgb}{1.000000,0.000000,0.000000}%
\pgfsetstrokecolor{currentstroke}%
\pgfsetdash{}{0pt}%
\pgfpathmoveto{\pgfqpoint{0.961364in}{1.998738in}}%
\pgfpathlineto{\pgfqpoint{0.962209in}{1.896009in}}%
\pgfpathlineto{\pgfqpoint{0.964746in}{2.135645in}}%
\pgfpathlineto{\pgfqpoint{0.966437in}{1.922162in}}%
\pgfpathlineto{\pgfqpoint{0.967283in}{1.992333in}}%
\pgfpathlineto{\pgfqpoint{0.968129in}{2.069418in}}%
\pgfpathlineto{\pgfqpoint{0.968974in}{1.932684in}}%
\pgfpathlineto{\pgfqpoint{0.969820in}{2.065238in}}%
\pgfpathlineto{\pgfqpoint{0.972357in}{1.814257in}}%
\pgfpathlineto{\pgfqpoint{0.973202in}{2.023452in}}%
\pgfpathlineto{\pgfqpoint{0.974048in}{1.941734in}}%
\pgfpathlineto{\pgfqpoint{0.974894in}{1.978247in}}%
\pgfpathlineto{\pgfqpoint{0.975739in}{2.091924in}}%
\pgfpathlineto{\pgfqpoint{0.976585in}{1.977986in}}%
\pgfpathlineto{\pgfqpoint{0.977430in}{2.116023in}}%
\pgfpathlineto{\pgfqpoint{0.979122in}{1.852091in}}%
\pgfpathlineto{\pgfqpoint{0.981659in}{2.119340in}}%
\pgfpathlineto{\pgfqpoint{0.982504in}{2.116779in}}%
\pgfpathlineto{\pgfqpoint{0.986732in}{1.703060in}}%
\pgfpathlineto{\pgfqpoint{0.987578in}{2.109536in}}%
\pgfpathlineto{\pgfqpoint{0.988424in}{1.938143in}}%
\pgfpathlineto{\pgfqpoint{0.989269in}{2.124529in}}%
\pgfpathlineto{\pgfqpoint{0.990115in}{1.983912in}}%
\pgfpathlineto{\pgfqpoint{0.991806in}{1.751972in}}%
\pgfpathlineto{\pgfqpoint{0.992652in}{1.804184in}}%
\pgfpathlineto{\pgfqpoint{0.993497in}{2.104781in}}%
\pgfpathlineto{\pgfqpoint{0.994343in}{1.950025in}}%
\pgfpathlineto{\pgfqpoint{0.995189in}{1.832706in}}%
\pgfpathlineto{\pgfqpoint{0.997725in}{2.099532in}}%
\pgfpathlineto{\pgfqpoint{0.999417in}{1.954804in}}%
\pgfpathlineto{\pgfqpoint{1.000262in}{2.092712in}}%
\pgfpathlineto{\pgfqpoint{1.001108in}{1.704959in}}%
\pgfpathlineto{\pgfqpoint{1.001954in}{1.956234in}}%
\pgfpathlineto{\pgfqpoint{1.002799in}{1.971059in}}%
\pgfpathlineto{\pgfqpoint{1.003645in}{2.322265in}}%
\pgfpathlineto{\pgfqpoint{1.004490in}{2.165932in}}%
\pgfpathlineto{\pgfqpoint{1.005336in}{2.108426in}}%
\pgfpathlineto{\pgfqpoint{1.007873in}{2.220509in}}%
\pgfpathlineto{\pgfqpoint{1.009564in}{1.633563in}}%
\pgfpathlineto{\pgfqpoint{1.010410in}{2.095404in}}%
\pgfpathlineto{\pgfqpoint{1.011255in}{1.926032in}}%
\pgfpathlineto{\pgfqpoint{1.012101in}{1.903864in}}%
\pgfpathlineto{\pgfqpoint{1.012947in}{1.939510in}}%
\pgfpathlineto{\pgfqpoint{1.013792in}{2.120114in}}%
\pgfpathlineto{\pgfqpoint{1.014638in}{1.835080in}}%
\pgfpathlineto{\pgfqpoint{1.015484in}{2.211136in}}%
\pgfpathlineto{\pgfqpoint{1.016329in}{1.993361in}}%
\pgfpathlineto{\pgfqpoint{1.017175in}{2.100432in}}%
\pgfpathlineto{\pgfqpoint{1.018020in}{2.046964in}}%
\pgfpathlineto{\pgfqpoint{1.018866in}{2.037039in}}%
\pgfpathlineto{\pgfqpoint{1.019712in}{2.322844in}}%
\pgfpathlineto{\pgfqpoint{1.020557in}{1.842630in}}%
\pgfpathlineto{\pgfqpoint{1.021403in}{1.916450in}}%
\pgfpathlineto{\pgfqpoint{1.022249in}{1.961396in}}%
\pgfpathlineto{\pgfqpoint{1.023094in}{1.809592in}}%
\pgfpathlineto{\pgfqpoint{1.023940in}{1.305767in}}%
\pgfpathlineto{\pgfqpoint{1.024785in}{2.009310in}}%
\pgfpathlineto{\pgfqpoint{1.025631in}{2.000121in}}%
\pgfpathlineto{\pgfqpoint{1.026477in}{1.637840in}}%
\pgfpathlineto{\pgfqpoint{1.027322in}{2.073734in}}%
\pgfpathlineto{\pgfqpoint{1.028168in}{1.858438in}}%
\pgfpathlineto{\pgfqpoint{1.029859in}{2.091790in}}%
\pgfpathlineto{\pgfqpoint{1.030705in}{1.918735in}}%
\pgfpathlineto{\pgfqpoint{1.031550in}{1.366878in}}%
\pgfpathlineto{\pgfqpoint{1.032396in}{1.661058in}}%
\pgfpathlineto{\pgfqpoint{1.034087in}{2.270354in}}%
\pgfpathlineto{\pgfqpoint{1.034933in}{1.844788in}}%
\pgfpathlineto{\pgfqpoint{1.035779in}{2.111575in}}%
\pgfpathlineto{\pgfqpoint{1.036624in}{2.001296in}}%
\pgfpathlineto{\pgfqpoint{1.037470in}{2.279255in}}%
\pgfpathlineto{\pgfqpoint{1.040007in}{1.868699in}}%
\pgfpathlineto{\pgfqpoint{1.040852in}{2.476686in}}%
\pgfpathlineto{\pgfqpoint{1.041698in}{1.799101in}}%
\pgfpathlineto{\pgfqpoint{1.042544in}{1.989229in}}%
\pgfpathlineto{\pgfqpoint{1.043389in}{1.968134in}}%
\pgfpathlineto{\pgfqpoint{1.044235in}{1.682813in}}%
\pgfpathlineto{\pgfqpoint{1.045080in}{2.285085in}}%
\pgfpathlineto{\pgfqpoint{1.045926in}{1.962119in}}%
\pgfpathlineto{\pgfqpoint{1.046772in}{2.086339in}}%
\pgfpathlineto{\pgfqpoint{1.047617in}{1.642924in}}%
\pgfpathlineto{\pgfqpoint{1.048463in}{1.927207in}}%
\pgfpathlineto{\pgfqpoint{1.049308in}{1.981984in}}%
\pgfpathlineto{\pgfqpoint{1.050154in}{1.860049in}}%
\pgfpathlineto{\pgfqpoint{1.051845in}{2.201899in}}%
\pgfpathlineto{\pgfqpoint{1.052691in}{1.700329in}}%
\pgfpathlineto{\pgfqpoint{1.053537in}{1.916041in}}%
\pgfpathlineto{\pgfqpoint{1.054382in}{2.014840in}}%
\pgfpathlineto{\pgfqpoint{1.055228in}{1.966834in}}%
\pgfpathlineto{\pgfqpoint{1.056073in}{2.027208in}}%
\pgfpathlineto{\pgfqpoint{1.056919in}{1.648725in}}%
\pgfpathlineto{\pgfqpoint{1.057765in}{2.136418in}}%
\pgfpathlineto{\pgfqpoint{1.058610in}{1.668614in}}%
\pgfpathlineto{\pgfqpoint{1.059456in}{1.949979in}}%
\pgfpathlineto{\pgfqpoint{1.060302in}{2.228609in}}%
\pgfpathlineto{\pgfqpoint{1.061147in}{2.110427in}}%
\pgfpathlineto{\pgfqpoint{1.061993in}{2.107874in}}%
\pgfpathlineto{\pgfqpoint{1.064530in}{1.722277in}}%
\pgfpathlineto{\pgfqpoint{1.065375in}{1.396352in}}%
\pgfpathlineto{\pgfqpoint{1.067067in}{2.540468in}}%
\pgfpathlineto{\pgfqpoint{1.067912in}{1.784347in}}%
\pgfpathlineto{\pgfqpoint{1.068758in}{2.245907in}}%
\pgfpathlineto{\pgfqpoint{1.069603in}{2.402995in}}%
\pgfpathlineto{\pgfqpoint{1.070449in}{1.951885in}}%
\pgfpathlineto{\pgfqpoint{1.071295in}{2.037929in}}%
\pgfpathlineto{\pgfqpoint{1.072140in}{2.259962in}}%
\pgfpathlineto{\pgfqpoint{1.074677in}{1.795252in}}%
\pgfpathlineto{\pgfqpoint{1.075523in}{2.188384in}}%
\pgfpathlineto{\pgfqpoint{1.076368in}{1.431713in}}%
\pgfpathlineto{\pgfqpoint{1.077214in}{1.918346in}}%
\pgfpathlineto{\pgfqpoint{1.078060in}{1.722136in}}%
\pgfpathlineto{\pgfqpoint{1.078905in}{2.383396in}}%
\pgfpathlineto{\pgfqpoint{1.079751in}{2.154621in}}%
\pgfpathlineto{\pgfqpoint{1.080597in}{1.839574in}}%
\pgfpathlineto{\pgfqpoint{1.081442in}{2.430248in}}%
\pgfpathlineto{\pgfqpoint{1.082288in}{2.279696in}}%
\pgfpathlineto{\pgfqpoint{1.083979in}{1.598597in}}%
\pgfpathlineto{\pgfqpoint{1.084825in}{2.343886in}}%
\pgfpathlineto{\pgfqpoint{1.085670in}{1.981128in}}%
\pgfpathlineto{\pgfqpoint{1.086516in}{2.233065in}}%
\pgfpathlineto{\pgfqpoint{1.087362in}{2.101221in}}%
\pgfpathlineto{\pgfqpoint{1.089053in}{2.570315in}}%
\pgfpathlineto{\pgfqpoint{1.089898in}{1.813765in}}%
\pgfpathlineto{\pgfqpoint{1.090744in}{2.377929in}}%
\pgfpathlineto{\pgfqpoint{1.093281in}{1.640285in}}%
\pgfpathlineto{\pgfqpoint{1.094972in}{2.166645in}}%
\pgfpathlineto{\pgfqpoint{1.095818in}{1.614377in}}%
\pgfpathlineto{\pgfqpoint{1.096663in}{1.618410in}}%
\pgfpathlineto{\pgfqpoint{1.097509in}{2.382652in}}%
\pgfpathlineto{\pgfqpoint{1.098355in}{2.179195in}}%
\pgfpathlineto{\pgfqpoint{1.099200in}{1.998378in}}%
\pgfpathlineto{\pgfqpoint{1.100046in}{2.021107in}}%
\pgfpathlineto{\pgfqpoint{1.100892in}{2.077674in}}%
\pgfpathlineto{\pgfqpoint{1.101737in}{1.608781in}}%
\pgfpathlineto{\pgfqpoint{1.102583in}{2.043350in}}%
\pgfpathlineto{\pgfqpoint{1.103428in}{2.023599in}}%
\pgfpathlineto{\pgfqpoint{1.104274in}{1.863471in}}%
\pgfpathlineto{\pgfqpoint{1.105120in}{2.372103in}}%
\pgfpathlineto{\pgfqpoint{1.106811in}{1.776268in}}%
\pgfpathlineto{\pgfqpoint{1.107657in}{2.063877in}}%
\pgfpathlineto{\pgfqpoint{1.109348in}{1.539923in}}%
\pgfpathlineto{\pgfqpoint{1.111885in}{2.334477in}}%
\pgfpathlineto{\pgfqpoint{1.114422in}{1.629849in}}%
\pgfpathlineto{\pgfqpoint{1.115267in}{1.808941in}}%
\pgfpathlineto{\pgfqpoint{1.116958in}{1.964803in}}%
\pgfpathlineto{\pgfqpoint{1.117804in}{1.804639in}}%
\pgfpathlineto{\pgfqpoint{1.118650in}{2.227136in}}%
\pgfpathlineto{\pgfqpoint{1.119495in}{1.966966in}}%
\pgfpathlineto{\pgfqpoint{1.120341in}{1.980375in}}%
\pgfpathlineto{\pgfqpoint{1.121187in}{2.152243in}}%
\pgfpathlineto{\pgfqpoint{1.122032in}{1.992285in}}%
\pgfpathlineto{\pgfqpoint{1.123723in}{2.249233in}}%
\pgfpathlineto{\pgfqpoint{1.124569in}{1.531081in}}%
\pgfpathlineto{\pgfqpoint{1.125415in}{2.167783in}}%
\pgfpathlineto{\pgfqpoint{1.126260in}{1.777641in}}%
\pgfpathlineto{\pgfqpoint{1.127106in}{2.210544in}}%
\pgfpathlineto{\pgfqpoint{1.127951in}{1.859765in}}%
\pgfpathlineto{\pgfqpoint{1.128797in}{1.828759in}}%
\pgfpathlineto{\pgfqpoint{1.129643in}{2.353898in}}%
\pgfpathlineto{\pgfqpoint{1.130488in}{2.242087in}}%
\pgfpathlineto{\pgfqpoint{1.133025in}{1.916422in}}%
\pgfpathlineto{\pgfqpoint{1.133871in}{2.298720in}}%
\pgfpathlineto{\pgfqpoint{1.134716in}{1.427030in}}%
\pgfpathlineto{\pgfqpoint{1.135562in}{1.724686in}}%
\pgfpathlineto{\pgfqpoint{1.137253in}{2.276566in}}%
\pgfpathlineto{\pgfqpoint{1.138945in}{2.058359in}}%
\pgfpathlineto{\pgfqpoint{1.139790in}{2.025279in}}%
\pgfpathlineto{\pgfqpoint{1.140636in}{2.289437in}}%
\pgfpathlineto{\pgfqpoint{1.141481in}{1.749393in}}%
\pgfpathlineto{\pgfqpoint{1.142327in}{2.143833in}}%
\pgfpathlineto{\pgfqpoint{1.144018in}{1.573627in}}%
\pgfpathlineto{\pgfqpoint{1.144864in}{2.267510in}}%
\pgfpathlineto{\pgfqpoint{1.145710in}{1.748481in}}%
\pgfpathlineto{\pgfqpoint{1.148246in}{2.427373in}}%
\pgfpathlineto{\pgfqpoint{1.149938in}{1.762427in}}%
\pgfpathlineto{\pgfqpoint{1.150783in}{1.998703in}}%
\pgfpathlineto{\pgfqpoint{1.151629in}{1.882966in}}%
\pgfpathlineto{\pgfqpoint{1.152475in}{1.772472in}}%
\pgfpathlineto{\pgfqpoint{1.153320in}{2.514351in}}%
\pgfpathlineto{\pgfqpoint{1.155011in}{1.747323in}}%
\pgfpathlineto{\pgfqpoint{1.155857in}{1.860653in}}%
\pgfpathlineto{\pgfqpoint{1.156703in}{1.518622in}}%
\pgfpathlineto{\pgfqpoint{1.157548in}{1.864770in}}%
\pgfpathlineto{\pgfqpoint{1.158394in}{1.352614in}}%
\pgfpathlineto{\pgfqpoint{1.159240in}{2.690194in}}%
\pgfpathlineto{\pgfqpoint{1.160085in}{2.014721in}}%
\pgfpathlineto{\pgfqpoint{1.160931in}{2.088529in}}%
\pgfpathlineto{\pgfqpoint{1.162622in}{1.670047in}}%
\pgfpathlineto{\pgfqpoint{1.165159in}{2.144100in}}%
\pgfpathlineto{\pgfqpoint{1.166005in}{2.254162in}}%
\pgfpathlineto{\pgfqpoint{1.166850in}{1.878161in}}%
\pgfpathlineto{\pgfqpoint{1.167696in}{2.119611in}}%
\pgfpathlineto{\pgfqpoint{1.168541in}{2.359394in}}%
\pgfpathlineto{\pgfqpoint{1.170233in}{1.733447in}}%
\pgfpathlineto{\pgfqpoint{1.172770in}{1.947781in}}%
\pgfpathlineto{\pgfqpoint{1.173615in}{1.954381in}}%
\pgfpathlineto{\pgfqpoint{1.174461in}{1.552668in}}%
\pgfpathlineto{\pgfqpoint{1.175306in}{2.542144in}}%
\pgfpathlineto{\pgfqpoint{1.176152in}{2.155874in}}%
\pgfpathlineto{\pgfqpoint{1.176998in}{1.763702in}}%
\pgfpathlineto{\pgfqpoint{1.177843in}{2.454225in}}%
\pgfpathlineto{\pgfqpoint{1.178689in}{1.951905in}}%
\pgfpathlineto{\pgfqpoint{1.179535in}{1.827958in}}%
\pgfpathlineto{\pgfqpoint{1.180380in}{2.477647in}}%
\pgfpathlineto{\pgfqpoint{1.181226in}{2.095754in}}%
\pgfpathlineto{\pgfqpoint{1.182071in}{2.202037in}}%
\pgfpathlineto{\pgfqpoint{1.182917in}{2.107114in}}%
\pgfpathlineto{\pgfqpoint{1.183763in}{1.382272in}}%
\pgfpathlineto{\pgfqpoint{1.186300in}{2.372524in}}%
\pgfpathlineto{\pgfqpoint{1.187145in}{1.963052in}}%
\pgfpathlineto{\pgfqpoint{1.187991in}{2.124000in}}%
\pgfpathlineto{\pgfqpoint{1.189682in}{1.576537in}}%
\pgfpathlineto{\pgfqpoint{1.191373in}{2.362295in}}%
\pgfpathlineto{\pgfqpoint{1.193910in}{1.732893in}}%
\pgfpathlineto{\pgfqpoint{1.194756in}{2.231804in}}%
\pgfpathlineto{\pgfqpoint{1.195601in}{1.927061in}}%
\pgfpathlineto{\pgfqpoint{1.196447in}{2.172426in}}%
\pgfpathlineto{\pgfqpoint{1.198138in}{1.504345in}}%
\pgfpathlineto{\pgfqpoint{1.199830in}{2.185515in}}%
\pgfpathlineto{\pgfqpoint{1.200675in}{2.175916in}}%
\pgfpathlineto{\pgfqpoint{1.201521in}{2.130258in}}%
\pgfpathlineto{\pgfqpoint{1.202366in}{2.432098in}}%
\pgfpathlineto{\pgfqpoint{1.204903in}{1.788091in}}%
\pgfpathlineto{\pgfqpoint{1.207440in}{2.336344in}}%
\pgfpathlineto{\pgfqpoint{1.209131in}{1.645598in}}%
\pgfpathlineto{\pgfqpoint{1.209977in}{2.216594in}}%
\pgfpathlineto{\pgfqpoint{1.210823in}{1.763954in}}%
\pgfpathlineto{\pgfqpoint{1.211668in}{2.300470in}}%
\pgfpathlineto{\pgfqpoint{1.212514in}{2.232119in}}%
\pgfpathlineto{\pgfqpoint{1.213359in}{1.355519in}}%
\pgfpathlineto{\pgfqpoint{1.214205in}{1.872638in}}%
\pgfpathlineto{\pgfqpoint{1.215051in}{1.802356in}}%
\pgfpathlineto{\pgfqpoint{1.217588in}{2.263988in}}%
\pgfpathlineto{\pgfqpoint{1.219279in}{2.060373in}}%
\pgfpathlineto{\pgfqpoint{1.220124in}{2.140753in}}%
\pgfpathlineto{\pgfqpoint{1.220970in}{2.025510in}}%
\pgfpathlineto{\pgfqpoint{1.221816in}{1.589183in}}%
\pgfpathlineto{\pgfqpoint{1.222661in}{1.875274in}}%
\pgfpathlineto{\pgfqpoint{1.223507in}{2.261367in}}%
\pgfpathlineto{\pgfqpoint{1.226044in}{1.590633in}}%
\pgfpathlineto{\pgfqpoint{1.226889in}{2.077415in}}%
\pgfpathlineto{\pgfqpoint{1.227735in}{1.670505in}}%
\pgfpathlineto{\pgfqpoint{1.228581in}{2.212681in}}%
\pgfpathlineto{\pgfqpoint{1.229426in}{1.611198in}}%
\pgfpathlineto{\pgfqpoint{1.230272in}{2.010389in}}%
\pgfpathlineto{\pgfqpoint{1.231118in}{2.170308in}}%
\pgfpathlineto{\pgfqpoint{1.231963in}{1.933641in}}%
\pgfpathlineto{\pgfqpoint{1.232809in}{2.027967in}}%
\pgfpathlineto{\pgfqpoint{1.233654in}{2.035072in}}%
\pgfpathlineto{\pgfqpoint{1.234500in}{2.290419in}}%
\pgfpathlineto{\pgfqpoint{1.237037in}{1.583733in}}%
\pgfpathlineto{\pgfqpoint{1.237883in}{2.313386in}}%
\pgfpathlineto{\pgfqpoint{1.238728in}{1.728940in}}%
\pgfpathlineto{\pgfqpoint{1.239574in}{1.946867in}}%
\pgfpathlineto{\pgfqpoint{1.240419in}{1.471086in}}%
\pgfpathlineto{\pgfqpoint{1.241265in}{2.152610in}}%
\pgfpathlineto{\pgfqpoint{1.242111in}{1.692846in}}%
\pgfpathlineto{\pgfqpoint{1.242956in}{1.478562in}}%
\pgfpathlineto{\pgfqpoint{1.243802in}{1.959588in}}%
\pgfpathlineto{\pgfqpoint{1.244648in}{1.442141in}}%
\pgfpathlineto{\pgfqpoint{1.245493in}{1.549712in}}%
\pgfpathlineto{\pgfqpoint{1.247184in}{2.073597in}}%
\pgfpathlineto{\pgfqpoint{1.248030in}{1.916307in}}%
\pgfpathlineto{\pgfqpoint{1.248876in}{2.181316in}}%
\pgfpathlineto{\pgfqpoint{1.251413in}{1.510368in}}%
\pgfpathlineto{\pgfqpoint{1.252258in}{2.649604in}}%
\pgfpathlineto{\pgfqpoint{1.253104in}{2.532415in}}%
\pgfpathlineto{\pgfqpoint{1.255641in}{1.594443in}}%
\pgfpathlineto{\pgfqpoint{1.256486in}{2.487604in}}%
\pgfpathlineto{\pgfqpoint{1.257332in}{1.778613in}}%
\pgfpathlineto{\pgfqpoint{1.258178in}{2.495433in}}%
\pgfpathlineto{\pgfqpoint{1.259023in}{2.350853in}}%
\pgfpathlineto{\pgfqpoint{1.259869in}{2.187627in}}%
\pgfpathlineto{\pgfqpoint{1.260714in}{1.641821in}}%
\pgfpathlineto{\pgfqpoint{1.261560in}{2.375521in}}%
\pgfpathlineto{\pgfqpoint{1.262406in}{1.969542in}}%
\pgfpathlineto{\pgfqpoint{1.263251in}{1.859701in}}%
\pgfpathlineto{\pgfqpoint{1.264097in}{1.868744in}}%
\pgfpathlineto{\pgfqpoint{1.265788in}{2.162044in}}%
\pgfpathlineto{\pgfqpoint{1.268325in}{1.827218in}}%
\pgfpathlineto{\pgfqpoint{1.270862in}{2.571962in}}%
\pgfpathlineto{\pgfqpoint{1.271708in}{1.868510in}}%
\pgfpathlineto{\pgfqpoint{1.272553in}{1.906010in}}%
\pgfpathlineto{\pgfqpoint{1.274244in}{2.102064in}}%
\pgfpathlineto{\pgfqpoint{1.275936in}{1.685983in}}%
\pgfpathlineto{\pgfqpoint{1.277627in}{2.379731in}}%
\pgfpathlineto{\pgfqpoint{1.280164in}{1.754484in}}%
\pgfpathlineto{\pgfqpoint{1.281009in}{2.008725in}}%
\pgfpathlineto{\pgfqpoint{1.281855in}{1.740390in}}%
\pgfpathlineto{\pgfqpoint{1.282701in}{2.056133in}}%
\pgfpathlineto{\pgfqpoint{1.283546in}{2.051131in}}%
\pgfpathlineto{\pgfqpoint{1.284392in}{2.077704in}}%
\pgfpathlineto{\pgfqpoint{1.285238in}{1.908144in}}%
\pgfpathlineto{\pgfqpoint{1.286083in}{2.212016in}}%
\pgfpathlineto{\pgfqpoint{1.287774in}{1.495042in}}%
\pgfpathlineto{\pgfqpoint{1.288620in}{2.450203in}}%
\pgfpathlineto{\pgfqpoint{1.289466in}{2.086100in}}%
\pgfpathlineto{\pgfqpoint{1.290311in}{2.113116in}}%
\pgfpathlineto{\pgfqpoint{1.291157in}{2.651708in}}%
\pgfpathlineto{\pgfqpoint{1.292848in}{1.721513in}}%
\pgfpathlineto{\pgfqpoint{1.294539in}{2.334020in}}%
\pgfpathlineto{\pgfqpoint{1.295385in}{1.778167in}}%
\pgfpathlineto{\pgfqpoint{1.296231in}{2.085888in}}%
\pgfpathlineto{\pgfqpoint{1.297076in}{1.983727in}}%
\pgfpathlineto{\pgfqpoint{1.297922in}{2.330610in}}%
\pgfpathlineto{\pgfqpoint{1.298767in}{2.285808in}}%
\pgfpathlineto{\pgfqpoint{1.299613in}{2.322892in}}%
\pgfpathlineto{\pgfqpoint{1.300459in}{2.799537in}}%
\pgfpathlineto{\pgfqpoint{1.302996in}{1.948600in}}%
\pgfpathlineto{\pgfqpoint{1.304687in}{2.072010in}}%
\pgfpathlineto{\pgfqpoint{1.305532in}{2.354196in}}%
\pgfpathlineto{\pgfqpoint{1.306378in}{2.134451in}}%
\pgfpathlineto{\pgfqpoint{1.307224in}{2.139703in}}%
\pgfpathlineto{\pgfqpoint{1.308069in}{1.744145in}}%
\pgfpathlineto{\pgfqpoint{1.308915in}{1.849142in}}%
\pgfpathlineto{\pgfqpoint{1.310606in}{2.398959in}}%
\pgfpathlineto{\pgfqpoint{1.312297in}{1.665193in}}%
\pgfpathlineto{\pgfqpoint{1.313989in}{2.229661in}}%
\pgfpathlineto{\pgfqpoint{1.314834in}{1.742019in}}%
\pgfpathlineto{\pgfqpoint{1.315680in}{2.009947in}}%
\pgfpathlineto{\pgfqpoint{1.316526in}{1.977091in}}%
\pgfpathlineto{\pgfqpoint{1.317371in}{1.589410in}}%
\pgfpathlineto{\pgfqpoint{1.319908in}{2.338092in}}%
\pgfpathlineto{\pgfqpoint{1.321599in}{1.725522in}}%
\pgfpathlineto{\pgfqpoint{1.322445in}{2.236419in}}%
\pgfpathlineto{\pgfqpoint{1.323291in}{1.899226in}}%
\pgfpathlineto{\pgfqpoint{1.324136in}{2.247255in}}%
\pgfpathlineto{\pgfqpoint{1.326673in}{1.482688in}}%
\pgfpathlineto{\pgfqpoint{1.327519in}{2.319867in}}%
\pgfpathlineto{\pgfqpoint{1.328364in}{1.922228in}}%
\pgfpathlineto{\pgfqpoint{1.329210in}{1.811756in}}%
\pgfpathlineto{\pgfqpoint{1.330056in}{2.214324in}}%
\pgfpathlineto{\pgfqpoint{1.330901in}{1.894283in}}%
\pgfpathlineto{\pgfqpoint{1.331747in}{1.920102in}}%
\pgfpathlineto{\pgfqpoint{1.333438in}{1.384574in}}%
\pgfpathlineto{\pgfqpoint{1.334284in}{1.491713in}}%
\pgfpathlineto{\pgfqpoint{1.336821in}{2.211303in}}%
\pgfpathlineto{\pgfqpoint{1.337666in}{1.580891in}}%
\pgfpathlineto{\pgfqpoint{1.339357in}{2.223009in}}%
\pgfpathlineto{\pgfqpoint{1.340203in}{2.209983in}}%
\pgfpathlineto{\pgfqpoint{1.341894in}{2.634263in}}%
\pgfpathlineto{\pgfqpoint{1.342740in}{1.594078in}}%
\pgfpathlineto{\pgfqpoint{1.343586in}{2.122338in}}%
\pgfpathlineto{\pgfqpoint{1.346122in}{1.811363in}}%
\pgfpathlineto{\pgfqpoint{1.346968in}{2.124656in}}%
\pgfpathlineto{\pgfqpoint{1.347814in}{1.906599in}}%
\pgfpathlineto{\pgfqpoint{1.349505in}{2.312715in}}%
\pgfpathlineto{\pgfqpoint{1.350351in}{2.525550in}}%
\pgfpathlineto{\pgfqpoint{1.352887in}{1.685242in}}%
\pgfpathlineto{\pgfqpoint{1.353733in}{1.704949in}}%
\pgfpathlineto{\pgfqpoint{1.354579in}{2.633527in}}%
\pgfpathlineto{\pgfqpoint{1.355424in}{1.933531in}}%
\pgfpathlineto{\pgfqpoint{1.357116in}{2.097748in}}%
\pgfpathlineto{\pgfqpoint{1.357961in}{2.016113in}}%
\pgfpathlineto{\pgfqpoint{1.358807in}{1.634579in}}%
\pgfpathlineto{\pgfqpoint{1.359652in}{2.016144in}}%
\pgfpathlineto{\pgfqpoint{1.360498in}{1.821276in}}%
\pgfpathlineto{\pgfqpoint{1.363035in}{2.323200in}}%
\pgfpathlineto{\pgfqpoint{1.363881in}{1.930904in}}%
\pgfpathlineto{\pgfqpoint{1.364726in}{2.190379in}}%
\pgfpathlineto{\pgfqpoint{1.366417in}{1.508478in}}%
\pgfpathlineto{\pgfqpoint{1.368109in}{2.306537in}}%
\pgfpathlineto{\pgfqpoint{1.368954in}{2.190501in}}%
\pgfpathlineto{\pgfqpoint{1.371491in}{1.609034in}}%
\pgfpathlineto{\pgfqpoint{1.372337in}{2.219173in}}%
\pgfpathlineto{\pgfqpoint{1.373182in}{1.899959in}}%
\pgfpathlineto{\pgfqpoint{1.374874in}{2.080406in}}%
\pgfpathlineto{\pgfqpoint{1.376565in}{2.381218in}}%
\pgfpathlineto{\pgfqpoint{1.377410in}{2.104844in}}%
\pgfpathlineto{\pgfqpoint{1.378256in}{2.130181in}}%
\pgfpathlineto{\pgfqpoint{1.379102in}{2.370106in}}%
\pgfpathlineto{\pgfqpoint{1.379947in}{2.039611in}}%
\pgfpathlineto{\pgfqpoint{1.380793in}{2.567891in}}%
\pgfpathlineto{\pgfqpoint{1.382484in}{1.567042in}}%
\pgfpathlineto{\pgfqpoint{1.383330in}{1.652714in}}%
\pgfpathlineto{\pgfqpoint{1.384175in}{2.123493in}}%
\pgfpathlineto{\pgfqpoint{1.385021in}{1.965263in}}%
\pgfpathlineto{\pgfqpoint{1.385867in}{1.934757in}}%
\pgfpathlineto{\pgfqpoint{1.387558in}{2.203780in}}%
\pgfpathlineto{\pgfqpoint{1.389249in}{1.616993in}}%
\pgfpathlineto{\pgfqpoint{1.390095in}{2.090963in}}%
\pgfpathlineto{\pgfqpoint{1.390940in}{1.571769in}}%
\pgfpathlineto{\pgfqpoint{1.391786in}{1.884254in}}%
\pgfpathlineto{\pgfqpoint{1.392632in}{2.295537in}}%
\pgfpathlineto{\pgfqpoint{1.394323in}{1.451997in}}%
\pgfpathlineto{\pgfqpoint{1.395169in}{1.997336in}}%
\pgfpathlineto{\pgfqpoint{1.396014in}{1.556928in}}%
\pgfpathlineto{\pgfqpoint{1.396860in}{1.997268in}}%
\pgfpathlineto{\pgfqpoint{1.397705in}{1.979480in}}%
\pgfpathlineto{\pgfqpoint{1.399397in}{2.047797in}}%
\pgfpathlineto{\pgfqpoint{1.400242in}{1.963008in}}%
\pgfpathlineto{\pgfqpoint{1.401934in}{1.609777in}}%
\pgfpathlineto{\pgfqpoint{1.403625in}{2.542939in}}%
\pgfpathlineto{\pgfqpoint{1.404470in}{2.031657in}}%
\pgfpathlineto{\pgfqpoint{1.405316in}{2.130705in}}%
\pgfpathlineto{\pgfqpoint{1.407853in}{1.621654in}}%
\pgfpathlineto{\pgfqpoint{1.408699in}{1.932508in}}%
\pgfpathlineto{\pgfqpoint{1.409544in}{1.497239in}}%
\pgfpathlineto{\pgfqpoint{1.411235in}{2.261912in}}%
\pgfpathlineto{\pgfqpoint{1.412081in}{2.116238in}}%
\pgfpathlineto{\pgfqpoint{1.412927in}{2.309366in}}%
\pgfpathlineto{\pgfqpoint{1.413772in}{2.270060in}}%
\pgfpathlineto{\pgfqpoint{1.414618in}{2.290413in}}%
\pgfpathlineto{\pgfqpoint{1.417155in}{1.727116in}}%
\pgfpathlineto{\pgfqpoint{1.418000in}{2.238421in}}%
\pgfpathlineto{\pgfqpoint{1.418846in}{1.604011in}}%
\pgfpathlineto{\pgfqpoint{1.419692in}{1.761299in}}%
\pgfpathlineto{\pgfqpoint{1.420537in}{2.224897in}}%
\pgfpathlineto{\pgfqpoint{1.421383in}{1.653865in}}%
\pgfpathlineto{\pgfqpoint{1.422229in}{1.779671in}}%
\pgfpathlineto{\pgfqpoint{1.423920in}{1.954758in}}%
\pgfpathlineto{\pgfqpoint{1.424765in}{1.887177in}}%
\pgfpathlineto{\pgfqpoint{1.425611in}{1.405099in}}%
\pgfpathlineto{\pgfqpoint{1.426457in}{1.670490in}}%
\pgfpathlineto{\pgfqpoint{1.428148in}{2.530522in}}%
\pgfpathlineto{\pgfqpoint{1.428994in}{1.453356in}}%
\pgfpathlineto{\pgfqpoint{1.429839in}{2.084385in}}%
\pgfpathlineto{\pgfqpoint{1.432376in}{1.549345in}}%
\pgfpathlineto{\pgfqpoint{1.433222in}{2.036599in}}%
\pgfpathlineto{\pgfqpoint{1.434067in}{1.423837in}}%
\pgfpathlineto{\pgfqpoint{1.435759in}{2.388141in}}%
\pgfpathlineto{\pgfqpoint{1.436604in}{1.957200in}}%
\pgfpathlineto{\pgfqpoint{1.437450in}{2.273657in}}%
\pgfpathlineto{\pgfqpoint{1.440832in}{1.704897in}}%
\pgfpathlineto{\pgfqpoint{1.441678in}{1.744451in}}%
\pgfpathlineto{\pgfqpoint{1.445060in}{2.504480in}}%
\pgfpathlineto{\pgfqpoint{1.446752in}{1.986108in}}%
\pgfpathlineto{\pgfqpoint{1.447597in}{2.139802in}}%
\pgfpathlineto{\pgfqpoint{1.449288in}{1.621011in}}%
\pgfpathlineto{\pgfqpoint{1.450134in}{1.696777in}}%
\pgfpathlineto{\pgfqpoint{1.450980in}{2.209127in}}%
\pgfpathlineto{\pgfqpoint{1.451825in}{1.696602in}}%
\pgfpathlineto{\pgfqpoint{1.452671in}{1.802529in}}%
\pgfpathlineto{\pgfqpoint{1.453517in}{1.947770in}}%
\pgfpathlineto{\pgfqpoint{1.454362in}{1.696323in}}%
\pgfpathlineto{\pgfqpoint{1.455208in}{2.214449in}}%
\pgfpathlineto{\pgfqpoint{1.456053in}{1.903745in}}%
\pgfpathlineto{\pgfqpoint{1.458590in}{2.633691in}}%
\pgfpathlineto{\pgfqpoint{1.459436in}{1.878420in}}%
\pgfpathlineto{\pgfqpoint{1.460282in}{2.745264in}}%
\pgfpathlineto{\pgfqpoint{1.461127in}{1.692020in}}%
\pgfpathlineto{\pgfqpoint{1.461973in}{2.011903in}}%
\pgfpathlineto{\pgfqpoint{1.462818in}{2.099598in}}%
\pgfpathlineto{\pgfqpoint{1.463664in}{2.650394in}}%
\pgfpathlineto{\pgfqpoint{1.464510in}{1.918171in}}%
\pgfpathlineto{\pgfqpoint{1.465355in}{2.503568in}}%
\pgfpathlineto{\pgfqpoint{1.466201in}{2.076539in}}%
\pgfpathlineto{\pgfqpoint{1.467047in}{2.504424in}}%
\pgfpathlineto{\pgfqpoint{1.467892in}{2.007916in}}%
\pgfpathlineto{\pgfqpoint{1.468738in}{2.660612in}}%
\pgfpathlineto{\pgfqpoint{1.469583in}{1.975713in}}%
\pgfpathlineto{\pgfqpoint{1.470429in}{2.395147in}}%
\pgfpathlineto{\pgfqpoint{1.472120in}{2.026360in}}%
\pgfpathlineto{\pgfqpoint{1.472966in}{2.129140in}}%
\pgfpathlineto{\pgfqpoint{1.475503in}{1.586857in}}%
\pgfpathlineto{\pgfqpoint{1.478040in}{2.334200in}}%
\pgfpathlineto{\pgfqpoint{1.478885in}{2.110958in}}%
\pgfpathlineto{\pgfqpoint{1.479731in}{2.256708in}}%
\pgfpathlineto{\pgfqpoint{1.482268in}{1.777309in}}%
\pgfpathlineto{\pgfqpoint{1.483113in}{2.187041in}}%
\pgfpathlineto{\pgfqpoint{1.483959in}{1.866182in}}%
\pgfpathlineto{\pgfqpoint{1.484805in}{1.829536in}}%
\pgfpathlineto{\pgfqpoint{1.485650in}{2.396658in}}%
\pgfpathlineto{\pgfqpoint{1.486496in}{2.295141in}}%
\pgfpathlineto{\pgfqpoint{1.487342in}{1.687547in}}%
\pgfpathlineto{\pgfqpoint{1.488187in}{2.100520in}}%
\pgfpathlineto{\pgfqpoint{1.489033in}{2.166871in}}%
\pgfpathlineto{\pgfqpoint{1.489878in}{1.656256in}}%
\pgfpathlineto{\pgfqpoint{1.490724in}{2.381436in}}%
\pgfpathlineto{\pgfqpoint{1.491570in}{2.172559in}}%
\pgfpathlineto{\pgfqpoint{1.492415in}{2.067680in}}%
\pgfpathlineto{\pgfqpoint{1.493261in}{2.683613in}}%
\pgfpathlineto{\pgfqpoint{1.494107in}{2.057062in}}%
\pgfpathlineto{\pgfqpoint{1.494952in}{2.164125in}}%
\pgfpathlineto{\pgfqpoint{1.495798in}{2.481699in}}%
\pgfpathlineto{\pgfqpoint{1.496643in}{2.273468in}}%
\pgfpathlineto{\pgfqpoint{1.499180in}{1.528814in}}%
\pgfpathlineto{\pgfqpoint{1.501717in}{2.901808in}}%
\pgfpathlineto{\pgfqpoint{1.503408in}{1.654716in}}%
\pgfpathlineto{\pgfqpoint{1.505100in}{2.383785in}}%
\pgfpathlineto{\pgfqpoint{1.505945in}{2.060466in}}%
\pgfpathlineto{\pgfqpoint{1.506791in}{2.046996in}}%
\pgfpathlineto{\pgfqpoint{1.509328in}{1.824702in}}%
\pgfpathlineto{\pgfqpoint{1.510173in}{2.098551in}}%
\pgfpathlineto{\pgfqpoint{1.511865in}{1.645533in}}%
\pgfpathlineto{\pgfqpoint{1.512710in}{2.083591in}}%
\pgfpathlineto{\pgfqpoint{1.513556in}{2.036852in}}%
\pgfpathlineto{\pgfqpoint{1.514402in}{1.724813in}}%
\pgfpathlineto{\pgfqpoint{1.515247in}{1.748948in}}%
\pgfpathlineto{\pgfqpoint{1.516093in}{2.660419in}}%
\pgfpathlineto{\pgfqpoint{1.516938in}{2.240224in}}%
\pgfpathlineto{\pgfqpoint{1.517784in}{2.218434in}}%
\pgfpathlineto{\pgfqpoint{1.518630in}{2.261481in}}%
\pgfpathlineto{\pgfqpoint{1.519475in}{1.996372in}}%
\pgfpathlineto{\pgfqpoint{1.520321in}{2.763454in}}%
\pgfpathlineto{\pgfqpoint{1.522012in}{1.708172in}}%
\pgfpathlineto{\pgfqpoint{1.522858in}{2.111161in}}%
\pgfpathlineto{\pgfqpoint{1.523703in}{2.002331in}}%
\pgfpathlineto{\pgfqpoint{1.524549in}{1.268819in}}%
\pgfpathlineto{\pgfqpoint{1.525395in}{2.318093in}}%
\pgfpathlineto{\pgfqpoint{1.526240in}{1.986226in}}%
\pgfpathlineto{\pgfqpoint{1.527086in}{1.789723in}}%
\pgfpathlineto{\pgfqpoint{1.527931in}{2.357464in}}%
\pgfpathlineto{\pgfqpoint{1.528777in}{1.726248in}}%
\pgfpathlineto{\pgfqpoint{1.529623in}{2.299723in}}%
\pgfpathlineto{\pgfqpoint{1.530468in}{1.329070in}}%
\pgfpathlineto{\pgfqpoint{1.531314in}{1.970600in}}%
\pgfpathlineto{\pgfqpoint{1.532160in}{2.392904in}}%
\pgfpathlineto{\pgfqpoint{1.533005in}{1.954443in}}%
\pgfpathlineto{\pgfqpoint{1.533851in}{2.256762in}}%
\pgfpathlineto{\pgfqpoint{1.534696in}{2.155560in}}%
\pgfpathlineto{\pgfqpoint{1.535542in}{2.475555in}}%
\pgfpathlineto{\pgfqpoint{1.537233in}{2.053987in}}%
\pgfpathlineto{\pgfqpoint{1.538079in}{2.138383in}}%
\pgfpathlineto{\pgfqpoint{1.539770in}{1.632745in}}%
\pgfpathlineto{\pgfqpoint{1.540616in}{2.546354in}}%
\pgfpathlineto{\pgfqpoint{1.541461in}{1.759990in}}%
\pgfpathlineto{\pgfqpoint{1.542307in}{2.001013in}}%
\pgfpathlineto{\pgfqpoint{1.543153in}{2.748973in}}%
\pgfpathlineto{\pgfqpoint{1.543998in}{2.685538in}}%
\pgfpathlineto{\pgfqpoint{1.546535in}{1.885627in}}%
\pgfpathlineto{\pgfqpoint{1.547381in}{2.418866in}}%
\pgfpathlineto{\pgfqpoint{1.548226in}{2.237395in}}%
\pgfpathlineto{\pgfqpoint{1.551609in}{1.705850in}}%
\pgfpathlineto{\pgfqpoint{1.552455in}{2.103953in}}%
\pgfpathlineto{\pgfqpoint{1.553300in}{1.916749in}}%
\pgfpathlineto{\pgfqpoint{1.554146in}{1.863736in}}%
\pgfpathlineto{\pgfqpoint{1.554991in}{2.098663in}}%
\pgfpathlineto{\pgfqpoint{1.555837in}{2.037605in}}%
\pgfpathlineto{\pgfqpoint{1.556683in}{1.544329in}}%
\pgfpathlineto{\pgfqpoint{1.557528in}{1.617507in}}%
\pgfpathlineto{\pgfqpoint{1.558374in}{2.280123in}}%
\pgfpathlineto{\pgfqpoint{1.559220in}{2.041552in}}%
\pgfpathlineto{\pgfqpoint{1.560065in}{2.353701in}}%
\pgfpathlineto{\pgfqpoint{1.560911in}{2.203128in}}%
\pgfpathlineto{\pgfqpoint{1.561756in}{1.869728in}}%
\pgfpathlineto{\pgfqpoint{1.562602in}{2.876890in}}%
\pgfpathlineto{\pgfqpoint{1.563448in}{2.187483in}}%
\pgfpathlineto{\pgfqpoint{1.564293in}{2.030846in}}%
\pgfpathlineto{\pgfqpoint{1.565139in}{1.450408in}}%
\pgfpathlineto{\pgfqpoint{1.566830in}{2.689939in}}%
\pgfpathlineto{\pgfqpoint{1.568521in}{1.946150in}}%
\pgfpathlineto{\pgfqpoint{1.569367in}{2.320743in}}%
\pgfpathlineto{\pgfqpoint{1.570213in}{1.650143in}}%
\pgfpathlineto{\pgfqpoint{1.571058in}{1.741038in}}%
\pgfpathlineto{\pgfqpoint{1.572750in}{2.605441in}}%
\pgfpathlineto{\pgfqpoint{1.573595in}{2.119777in}}%
\pgfpathlineto{\pgfqpoint{1.574441in}{1.904162in}}%
\pgfpathlineto{\pgfqpoint{1.576132in}{2.452929in}}%
\pgfpathlineto{\pgfqpoint{1.576978in}{1.656049in}}%
\pgfpathlineto{\pgfqpoint{1.577823in}{2.254314in}}%
\pgfpathlineto{\pgfqpoint{1.579515in}{1.817058in}}%
\pgfpathlineto{\pgfqpoint{1.581206in}{1.192568in}}%
\pgfpathlineto{\pgfqpoint{1.582051in}{2.352514in}}%
\pgfpathlineto{\pgfqpoint{1.582897in}{1.596723in}}%
\pgfpathlineto{\pgfqpoint{1.583743in}{2.201189in}}%
\pgfpathlineto{\pgfqpoint{1.584588in}{1.048594in}}%
\pgfpathlineto{\pgfqpoint{1.585434in}{1.804184in}}%
\pgfpathlineto{\pgfqpoint{1.587125in}{2.360013in}}%
\pgfpathlineto{\pgfqpoint{1.587971in}{2.140536in}}%
\pgfpathlineto{\pgfqpoint{1.590508in}{1.603848in}}%
\pgfpathlineto{\pgfqpoint{1.591353in}{1.619135in}}%
\pgfpathlineto{\pgfqpoint{1.593045in}{2.180473in}}%
\pgfpathlineto{\pgfqpoint{1.593890in}{1.817634in}}%
\pgfpathlineto{\pgfqpoint{1.594736in}{2.301942in}}%
\pgfpathlineto{\pgfqpoint{1.595581in}{1.723738in}}%
\pgfpathlineto{\pgfqpoint{1.596427in}{1.947787in}}%
\pgfpathlineto{\pgfqpoint{1.597273in}{2.004264in}}%
\pgfpathlineto{\pgfqpoint{1.598118in}{2.608069in}}%
\pgfpathlineto{\pgfqpoint{1.598964in}{1.241124in}}%
\pgfpathlineto{\pgfqpoint{1.599810in}{2.067082in}}%
\pgfpathlineto{\pgfqpoint{1.600655in}{1.746738in}}%
\pgfpathlineto{\pgfqpoint{1.601501in}{1.759884in}}%
\pgfpathlineto{\pgfqpoint{1.604038in}{2.764205in}}%
\pgfpathlineto{\pgfqpoint{1.604883in}{1.357532in}}%
\pgfpathlineto{\pgfqpoint{1.605729in}{2.266006in}}%
\pgfpathlineto{\pgfqpoint{1.606574in}{2.255811in}}%
\pgfpathlineto{\pgfqpoint{1.607420in}{1.747008in}}%
\pgfpathlineto{\pgfqpoint{1.608266in}{2.070216in}}%
\pgfpathlineto{\pgfqpoint{1.609111in}{1.816108in}}%
\pgfpathlineto{\pgfqpoint{1.609957in}{2.010412in}}%
\pgfpathlineto{\pgfqpoint{1.610803in}{2.454021in}}%
\pgfpathlineto{\pgfqpoint{1.611648in}{2.130131in}}%
\pgfpathlineto{\pgfqpoint{1.612494in}{1.466913in}}%
\pgfpathlineto{\pgfqpoint{1.613339in}{2.259184in}}%
\pgfpathlineto{\pgfqpoint{1.614185in}{1.490573in}}%
\pgfpathlineto{\pgfqpoint{1.615876in}{2.334198in}}%
\pgfpathlineto{\pgfqpoint{1.616722in}{1.582520in}}%
\pgfpathlineto{\pgfqpoint{1.617568in}{2.007348in}}%
\pgfpathlineto{\pgfqpoint{1.618413in}{2.089471in}}%
\pgfpathlineto{\pgfqpoint{1.619259in}{2.393548in}}%
\pgfpathlineto{\pgfqpoint{1.620104in}{1.509654in}}%
\pgfpathlineto{\pgfqpoint{1.620950in}{2.126838in}}%
\pgfpathlineto{\pgfqpoint{1.622641in}{2.576648in}}%
\pgfpathlineto{\pgfqpoint{1.623487in}{1.970926in}}%
\pgfpathlineto{\pgfqpoint{1.624333in}{2.185061in}}%
\pgfpathlineto{\pgfqpoint{1.625178in}{1.905148in}}%
\pgfpathlineto{\pgfqpoint{1.626024in}{1.945779in}}%
\pgfpathlineto{\pgfqpoint{1.627715in}{2.169705in}}%
\pgfpathlineto{\pgfqpoint{1.628561in}{2.673255in}}%
\pgfpathlineto{\pgfqpoint{1.630252in}{1.655343in}}%
\pgfpathlineto{\pgfqpoint{1.631098in}{2.195634in}}%
\pgfpathlineto{\pgfqpoint{1.631943in}{1.749190in}}%
\pgfpathlineto{\pgfqpoint{1.632789in}{1.937902in}}%
\pgfpathlineto{\pgfqpoint{1.633634in}{1.956796in}}%
\pgfpathlineto{\pgfqpoint{1.634480in}{1.877103in}}%
\pgfpathlineto{\pgfqpoint{1.635326in}{2.575276in}}%
\pgfpathlineto{\pgfqpoint{1.636171in}{2.145546in}}%
\pgfpathlineto{\pgfqpoint{1.637017in}{1.634519in}}%
\pgfpathlineto{\pgfqpoint{1.637863in}{1.861051in}}%
\pgfpathlineto{\pgfqpoint{1.638708in}{1.758382in}}%
\pgfpathlineto{\pgfqpoint{1.639554in}{2.779954in}}%
\pgfpathlineto{\pgfqpoint{1.641245in}{1.834111in}}%
\pgfpathlineto{\pgfqpoint{1.642091in}{1.857478in}}%
\pgfpathlineto{\pgfqpoint{1.642936in}{2.171174in}}%
\pgfpathlineto{\pgfqpoint{1.643782in}{2.170164in}}%
\pgfpathlineto{\pgfqpoint{1.645473in}{1.566717in}}%
\pgfpathlineto{\pgfqpoint{1.646319in}{1.952162in}}%
\pgfpathlineto{\pgfqpoint{1.648010in}{1.791054in}}%
\pgfpathlineto{\pgfqpoint{1.649701in}{2.240130in}}%
\pgfpathlineto{\pgfqpoint{1.651393in}{1.627036in}}%
\pgfpathlineto{\pgfqpoint{1.653084in}{2.072625in}}%
\pgfpathlineto{\pgfqpoint{1.653929in}{1.864814in}}%
\pgfpathlineto{\pgfqpoint{1.654775in}{1.894840in}}%
\pgfpathlineto{\pgfqpoint{1.655621in}{2.175058in}}%
\pgfpathlineto{\pgfqpoint{1.656466in}{1.736991in}}%
\pgfpathlineto{\pgfqpoint{1.657312in}{1.782722in}}%
\pgfpathlineto{\pgfqpoint{1.658158in}{2.014477in}}%
\pgfpathlineto{\pgfqpoint{1.659003in}{1.838320in}}%
\pgfpathlineto{\pgfqpoint{1.661540in}{1.511678in}}%
\pgfpathlineto{\pgfqpoint{1.664077in}{2.741139in}}%
\pgfpathlineto{\pgfqpoint{1.664923in}{2.110614in}}%
\pgfpathlineto{\pgfqpoint{1.665768in}{2.111232in}}%
\pgfpathlineto{\pgfqpoint{1.666614in}{2.011993in}}%
\pgfpathlineto{\pgfqpoint{1.668305in}{1.473167in}}%
\pgfpathlineto{\pgfqpoint{1.669151in}{1.899814in}}%
\pgfpathlineto{\pgfqpoint{1.669996in}{1.483802in}}%
\pgfpathlineto{\pgfqpoint{1.672533in}{2.219142in}}%
\pgfpathlineto{\pgfqpoint{1.675916in}{1.806050in}}%
\pgfpathlineto{\pgfqpoint{1.676761in}{2.308035in}}%
\pgfpathlineto{\pgfqpoint{1.677607in}{1.778302in}}%
\pgfpathlineto{\pgfqpoint{1.678453in}{2.225893in}}%
\pgfpathlineto{\pgfqpoint{1.679298in}{1.686014in}}%
\pgfpathlineto{\pgfqpoint{1.680144in}{1.788721in}}%
\pgfpathlineto{\pgfqpoint{1.680989in}{1.617861in}}%
\pgfpathlineto{\pgfqpoint{1.681835in}{2.017086in}}%
\pgfpathlineto{\pgfqpoint{1.682681in}{1.807315in}}%
\pgfpathlineto{\pgfqpoint{1.683526in}{1.708405in}}%
\pgfpathlineto{\pgfqpoint{1.684372in}{1.799356in}}%
\pgfpathlineto{\pgfqpoint{1.685217in}{2.318032in}}%
\pgfpathlineto{\pgfqpoint{1.686909in}{1.501831in}}%
\pgfpathlineto{\pgfqpoint{1.687754in}{2.188449in}}%
\pgfpathlineto{\pgfqpoint{1.688600in}{1.801560in}}%
\pgfpathlineto{\pgfqpoint{1.689446in}{2.107795in}}%
\pgfpathlineto{\pgfqpoint{1.691137in}{1.593457in}}%
\pgfpathlineto{\pgfqpoint{1.693674in}{2.487217in}}%
\pgfpathlineto{\pgfqpoint{1.696211in}{1.811296in}}%
\pgfpathlineto{\pgfqpoint{1.697056in}{1.965238in}}%
\pgfpathlineto{\pgfqpoint{1.697902in}{2.504114in}}%
\pgfpathlineto{\pgfqpoint{1.700439in}{1.497580in}}%
\pgfpathlineto{\pgfqpoint{1.702130in}{1.907872in}}%
\pgfpathlineto{\pgfqpoint{1.702976in}{1.924497in}}%
\pgfpathlineto{\pgfqpoint{1.704667in}{2.116881in}}%
\pgfpathlineto{\pgfqpoint{1.705512in}{2.012593in}}%
\pgfpathlineto{\pgfqpoint{1.706358in}{1.588942in}}%
\pgfpathlineto{\pgfqpoint{1.707204in}{2.218567in}}%
\pgfpathlineto{\pgfqpoint{1.708049in}{2.110260in}}%
\pgfpathlineto{\pgfqpoint{1.708895in}{1.586014in}}%
\pgfpathlineto{\pgfqpoint{1.709741in}{2.009229in}}%
\pgfpathlineto{\pgfqpoint{1.711432in}{1.687644in}}%
\pgfpathlineto{\pgfqpoint{1.712277in}{2.007862in}}%
\pgfpathlineto{\pgfqpoint{1.713123in}{1.692881in}}%
\pgfpathlineto{\pgfqpoint{1.713969in}{2.121953in}}%
\pgfpathlineto{\pgfqpoint{1.714814in}{1.804066in}}%
\pgfpathlineto{\pgfqpoint{1.718197in}{2.422848in}}%
\pgfpathlineto{\pgfqpoint{1.719042in}{1.796954in}}%
\pgfpathlineto{\pgfqpoint{1.719888in}{2.324494in}}%
\pgfpathlineto{\pgfqpoint{1.720734in}{1.692994in}}%
\pgfpathlineto{\pgfqpoint{1.723271in}{2.535909in}}%
\pgfpathlineto{\pgfqpoint{1.724116in}{1.794891in}}%
\pgfpathlineto{\pgfqpoint{1.724962in}{2.318949in}}%
\pgfpathlineto{\pgfqpoint{1.725807in}{2.431099in}}%
\pgfpathlineto{\pgfqpoint{1.726653in}{1.690145in}}%
\pgfpathlineto{\pgfqpoint{1.727499in}{2.637755in}}%
\pgfpathlineto{\pgfqpoint{1.728344in}{2.320763in}}%
\pgfpathlineto{\pgfqpoint{1.729190in}{1.694542in}}%
\pgfpathlineto{\pgfqpoint{1.730036in}{2.539572in}}%
\pgfpathlineto{\pgfqpoint{1.730881in}{1.789952in}}%
\pgfpathlineto{\pgfqpoint{1.731727in}{2.108530in}}%
\pgfpathlineto{\pgfqpoint{1.732572in}{2.115176in}}%
\pgfpathlineto{\pgfqpoint{1.733418in}{2.221327in}}%
\pgfpathlineto{\pgfqpoint{1.734264in}{2.009867in}}%
\pgfpathlineto{\pgfqpoint{1.735109in}{2.225070in}}%
\pgfpathlineto{\pgfqpoint{1.735955in}{1.898585in}}%
\pgfpathlineto{\pgfqpoint{1.738492in}{2.745929in}}%
\pgfpathlineto{\pgfqpoint{1.739337in}{2.425690in}}%
\pgfpathlineto{\pgfqpoint{1.740183in}{1.483780in}}%
\pgfpathlineto{\pgfqpoint{1.741029in}{1.692612in}}%
\pgfpathlineto{\pgfqpoint{1.741874in}{1.599337in}}%
\pgfpathlineto{\pgfqpoint{1.744411in}{2.107651in}}%
\pgfpathlineto{\pgfqpoint{1.745257in}{2.225256in}}%
\pgfpathlineto{\pgfqpoint{1.746102in}{1.896084in}}%
\pgfpathlineto{\pgfqpoint{1.746948in}{2.196473in}}%
\pgfpathlineto{\pgfqpoint{1.748639in}{1.485390in}}%
\pgfpathlineto{\pgfqpoint{1.749485in}{2.744065in}}%
\pgfpathlineto{\pgfqpoint{1.750331in}{1.790051in}}%
\pgfpathlineto{\pgfqpoint{1.751176in}{1.581837in}}%
\pgfpathlineto{\pgfqpoint{1.753713in}{2.220446in}}%
\pgfpathlineto{\pgfqpoint{1.754559in}{1.803006in}}%
\pgfpathlineto{\pgfqpoint{1.755404in}{1.886674in}}%
\pgfpathlineto{\pgfqpoint{1.757096in}{2.273563in}}%
\pgfpathlineto{\pgfqpoint{1.757941in}{2.094521in}}%
\pgfpathlineto{\pgfqpoint{1.758787in}{2.095000in}}%
\pgfpathlineto{\pgfqpoint{1.759632in}{1.451872in}}%
\pgfpathlineto{\pgfqpoint{1.761324in}{2.565190in}}%
\pgfpathlineto{\pgfqpoint{1.762169in}{2.395781in}}%
\pgfpathlineto{\pgfqpoint{1.763015in}{1.856330in}}%
\pgfpathlineto{\pgfqpoint{1.763860in}{2.268837in}}%
\pgfpathlineto{\pgfqpoint{1.764706in}{1.848426in}}%
\pgfpathlineto{\pgfqpoint{1.765552in}{2.377493in}}%
\pgfpathlineto{\pgfqpoint{1.767243in}{1.571670in}}%
\pgfpathlineto{\pgfqpoint{1.769780in}{2.261465in}}%
\pgfpathlineto{\pgfqpoint{1.770625in}{2.051717in}}%
\pgfpathlineto{\pgfqpoint{1.771471in}{2.890306in}}%
\pgfpathlineto{\pgfqpoint{1.773162in}{1.624901in}}%
\pgfpathlineto{\pgfqpoint{1.774008in}{1.643865in}}%
\pgfpathlineto{\pgfqpoint{1.774854in}{1.442106in}}%
\pgfpathlineto{\pgfqpoint{1.775699in}{2.275841in}}%
\pgfpathlineto{\pgfqpoint{1.776545in}{2.152339in}}%
\pgfpathlineto{\pgfqpoint{1.777390in}{2.033175in}}%
\pgfpathlineto{\pgfqpoint{1.778236in}{1.539713in}}%
\pgfpathlineto{\pgfqpoint{1.779082in}{2.281111in}}%
\pgfpathlineto{\pgfqpoint{1.779927in}{2.070055in}}%
\pgfpathlineto{\pgfqpoint{1.782464in}{1.761155in}}%
\pgfpathlineto{\pgfqpoint{1.784155in}{2.154633in}}%
\pgfpathlineto{\pgfqpoint{1.785001in}{1.655286in}}%
\pgfpathlineto{\pgfqpoint{1.785847in}{2.364334in}}%
\pgfpathlineto{\pgfqpoint{1.786692in}{1.768420in}}%
\pgfpathlineto{\pgfqpoint{1.787538in}{2.482990in}}%
\pgfpathlineto{\pgfqpoint{1.790075in}{1.532127in}}%
\pgfpathlineto{\pgfqpoint{1.790920in}{1.961541in}}%
\pgfpathlineto{\pgfqpoint{1.791766in}{1.647993in}}%
\pgfpathlineto{\pgfqpoint{1.792612in}{1.635540in}}%
\pgfpathlineto{\pgfqpoint{1.795149in}{2.267899in}}%
\pgfpathlineto{\pgfqpoint{1.795994in}{1.405667in}}%
\pgfpathlineto{\pgfqpoint{1.796840in}{2.484598in}}%
\pgfpathlineto{\pgfqpoint{1.797685in}{2.378806in}}%
\pgfpathlineto{\pgfqpoint{1.798531in}{1.743878in}}%
\pgfpathlineto{\pgfqpoint{1.799377in}{2.377984in}}%
\pgfpathlineto{\pgfqpoint{1.800222in}{1.952880in}}%
\pgfpathlineto{\pgfqpoint{1.801068in}{1.557526in}}%
\pgfpathlineto{\pgfqpoint{1.801914in}{1.836339in}}%
\pgfpathlineto{\pgfqpoint{1.802759in}{1.526223in}}%
\pgfpathlineto{\pgfqpoint{1.804450in}{2.086351in}}%
\pgfpathlineto{\pgfqpoint{1.805296in}{2.067354in}}%
\pgfpathlineto{\pgfqpoint{1.806142in}{1.996638in}}%
\pgfpathlineto{\pgfqpoint{1.806987in}{2.054423in}}%
\pgfpathlineto{\pgfqpoint{1.807833in}{2.264973in}}%
\pgfpathlineto{\pgfqpoint{1.809524in}{1.534020in}}%
\pgfpathlineto{\pgfqpoint{1.811215in}{2.060566in}}%
\pgfpathlineto{\pgfqpoint{1.813752in}{1.450278in}}%
\pgfpathlineto{\pgfqpoint{1.814598in}{2.168611in}}%
\pgfpathlineto{\pgfqpoint{1.815444in}{1.646483in}}%
\pgfpathlineto{\pgfqpoint{1.816289in}{2.276125in}}%
\pgfpathlineto{\pgfqpoint{1.817135in}{1.923461in}}%
\pgfpathlineto{\pgfqpoint{1.817980in}{1.743571in}}%
\pgfpathlineto{\pgfqpoint{1.819672in}{2.306166in}}%
\pgfpathlineto{\pgfqpoint{1.822209in}{0.980930in}}%
\pgfpathlineto{\pgfqpoint{1.823054in}{1.606607in}}%
\pgfpathlineto{\pgfqpoint{1.824745in}{2.215738in}}%
\pgfpathlineto{\pgfqpoint{1.825591in}{1.595703in}}%
\pgfpathlineto{\pgfqpoint{1.826437in}{1.919077in}}%
\pgfpathlineto{\pgfqpoint{1.828974in}{2.342271in}}%
\pgfpathlineto{\pgfqpoint{1.829819in}{2.540540in}}%
\pgfpathlineto{\pgfqpoint{1.830665in}{1.563902in}}%
\pgfpathlineto{\pgfqpoint{1.831510in}{1.892379in}}%
\pgfpathlineto{\pgfqpoint{1.832356in}{1.584291in}}%
\pgfpathlineto{\pgfqpoint{1.834893in}{2.223081in}}%
\pgfpathlineto{\pgfqpoint{1.835739in}{1.578035in}}%
\pgfpathlineto{\pgfqpoint{1.836584in}{2.201273in}}%
\pgfpathlineto{\pgfqpoint{1.837430in}{2.076502in}}%
\pgfpathlineto{\pgfqpoint{1.838275in}{2.163704in}}%
\pgfpathlineto{\pgfqpoint{1.839121in}{1.233530in}}%
\pgfpathlineto{\pgfqpoint{1.840812in}{2.294043in}}%
\pgfpathlineto{\pgfqpoint{1.843349in}{1.085356in}}%
\pgfpathlineto{\pgfqpoint{1.844195in}{2.337450in}}%
\pgfpathlineto{\pgfqpoint{1.845040in}{1.934381in}}%
\pgfpathlineto{\pgfqpoint{1.845886in}{1.497686in}}%
\pgfpathlineto{\pgfqpoint{1.846732in}{1.768707in}}%
\pgfpathlineto{\pgfqpoint{1.847577in}{2.434461in}}%
\pgfpathlineto{\pgfqpoint{1.848423in}{2.201306in}}%
\pgfpathlineto{\pgfqpoint{1.849268in}{2.323603in}}%
\pgfpathlineto{\pgfqpoint{1.850960in}{1.687967in}}%
\pgfpathlineto{\pgfqpoint{1.851805in}{2.364904in}}%
\pgfpathlineto{\pgfqpoint{1.852651in}{1.902508in}}%
\pgfpathlineto{\pgfqpoint{1.853497in}{2.011268in}}%
\pgfpathlineto{\pgfqpoint{1.854342in}{2.643176in}}%
\pgfpathlineto{\pgfqpoint{1.855188in}{2.007482in}}%
\pgfpathlineto{\pgfqpoint{1.856033in}{2.322408in}}%
\pgfpathlineto{\pgfqpoint{1.856879in}{2.747463in}}%
\pgfpathlineto{\pgfqpoint{1.858570in}{1.585340in}}%
\pgfpathlineto{\pgfqpoint{1.859416in}{2.006869in}}%
\pgfpathlineto{\pgfqpoint{1.860262in}{1.797315in}}%
\pgfpathlineto{\pgfqpoint{1.862798in}{2.640151in}}%
\pgfpathlineto{\pgfqpoint{1.863644in}{1.389372in}}%
\pgfpathlineto{\pgfqpoint{1.864490in}{2.432182in}}%
\pgfpathlineto{\pgfqpoint{1.865335in}{1.582136in}}%
\pgfpathlineto{\pgfqpoint{1.867027in}{2.112799in}}%
\pgfpathlineto{\pgfqpoint{1.867872in}{1.798767in}}%
\pgfpathlineto{\pgfqpoint{1.868718in}{1.894426in}}%
\pgfpathlineto{\pgfqpoint{1.869563in}{1.899178in}}%
\pgfpathlineto{\pgfqpoint{1.870409in}{2.108970in}}%
\pgfpathlineto{\pgfqpoint{1.871255in}{1.575356in}}%
\pgfpathlineto{\pgfqpoint{1.873792in}{2.475458in}}%
\pgfpathlineto{\pgfqpoint{1.875483in}{1.852742in}}%
\pgfpathlineto{\pgfqpoint{1.876328in}{2.235563in}}%
\pgfpathlineto{\pgfqpoint{1.877174in}{1.636195in}}%
\pgfpathlineto{\pgfqpoint{1.879711in}{2.370828in}}%
\pgfpathlineto{\pgfqpoint{1.881402in}{1.640998in}}%
\pgfpathlineto{\pgfqpoint{1.882248in}{1.976091in}}%
\pgfpathlineto{\pgfqpoint{1.883093in}{1.745676in}}%
\pgfpathlineto{\pgfqpoint{1.883939in}{2.060309in}}%
\pgfpathlineto{\pgfqpoint{1.884785in}{1.852374in}}%
\pgfpathlineto{\pgfqpoint{1.885630in}{1.533429in}}%
\pgfpathlineto{\pgfqpoint{1.886476in}{2.346068in}}%
\pgfpathlineto{\pgfqpoint{1.887322in}{2.298963in}}%
\pgfpathlineto{\pgfqpoint{1.889013in}{1.392618in}}%
\pgfpathlineto{\pgfqpoint{1.889858in}{2.456889in}}%
\pgfpathlineto{\pgfqpoint{1.890704in}{1.852102in}}%
\pgfpathlineto{\pgfqpoint{1.891550in}{2.169310in}}%
\pgfpathlineto{\pgfqpoint{1.892395in}{1.712389in}}%
\pgfpathlineto{\pgfqpoint{1.894087in}{2.183854in}}%
\pgfpathlineto{\pgfqpoint{1.894932in}{1.705283in}}%
\pgfpathlineto{\pgfqpoint{1.895778in}{2.371556in}}%
\pgfpathlineto{\pgfqpoint{1.896623in}{1.825112in}}%
\pgfpathlineto{\pgfqpoint{1.897469in}{1.702777in}}%
\pgfpathlineto{\pgfqpoint{1.899160in}{2.537791in}}%
\pgfpathlineto{\pgfqpoint{1.900006in}{1.903638in}}%
\pgfpathlineto{\pgfqpoint{1.900852in}{2.536072in}}%
\pgfpathlineto{\pgfqpoint{1.901697in}{2.324903in}}%
\pgfpathlineto{\pgfqpoint{1.902543in}{1.374332in}}%
\pgfpathlineto{\pgfqpoint{1.905080in}{2.536694in}}%
\pgfpathlineto{\pgfqpoint{1.905925in}{1.796557in}}%
\pgfpathlineto{\pgfqpoint{1.906771in}{1.796633in}}%
\pgfpathlineto{\pgfqpoint{1.907617in}{2.327756in}}%
\pgfpathlineto{\pgfqpoint{1.908462in}{2.114999in}}%
\pgfpathlineto{\pgfqpoint{1.910153in}{1.479824in}}%
\pgfpathlineto{\pgfqpoint{1.910999in}{2.220045in}}%
\pgfpathlineto{\pgfqpoint{1.911845in}{2.121198in}}%
\pgfpathlineto{\pgfqpoint{1.912690in}{1.585405in}}%
\pgfpathlineto{\pgfqpoint{1.915227in}{2.429897in}}%
\pgfpathlineto{\pgfqpoint{1.916073in}{1.796064in}}%
\pgfpathlineto{\pgfqpoint{1.916918in}{2.536716in}}%
\pgfpathlineto{\pgfqpoint{1.917764in}{2.007340in}}%
\pgfpathlineto{\pgfqpoint{1.918610in}{1.689757in}}%
\pgfpathlineto{\pgfqpoint{1.919455in}{2.219474in}}%
\pgfpathlineto{\pgfqpoint{1.920301in}{1.479424in}}%
\pgfpathlineto{\pgfqpoint{1.921147in}{2.008227in}}%
\pgfpathlineto{\pgfqpoint{1.921992in}{2.113602in}}%
\pgfpathlineto{\pgfqpoint{1.923683in}{1.901048in}}%
\pgfpathlineto{\pgfqpoint{1.924529in}{2.111213in}}%
\pgfpathlineto{\pgfqpoint{1.925375in}{1.796161in}}%
\pgfpathlineto{\pgfqpoint{1.927911in}{2.432698in}}%
\pgfpathlineto{\pgfqpoint{1.930448in}{1.585647in}}%
\pgfpathlineto{\pgfqpoint{1.932140in}{1.690850in}}%
\pgfpathlineto{\pgfqpoint{1.933831in}{2.011327in}}%
\pgfpathlineto{\pgfqpoint{1.934676in}{1.997826in}}%
\pgfpathlineto{\pgfqpoint{1.935522in}{1.902656in}}%
\pgfpathlineto{\pgfqpoint{1.936368in}{1.588687in}}%
\pgfpathlineto{\pgfqpoint{1.938905in}{2.343321in}}%
\pgfpathlineto{\pgfqpoint{1.939750in}{1.558720in}}%
\pgfpathlineto{\pgfqpoint{1.940596in}{2.905215in}}%
\pgfpathlineto{\pgfqpoint{1.941441in}{2.052667in}}%
\pgfpathlineto{\pgfqpoint{1.943133in}{1.747014in}}%
\pgfpathlineto{\pgfqpoint{1.943978in}{2.239650in}}%
\pgfpathlineto{\pgfqpoint{1.944824in}{1.526760in}}%
\pgfpathlineto{\pgfqpoint{1.945670in}{1.953679in}}%
\pgfpathlineto{\pgfqpoint{1.946515in}{2.588282in}}%
\pgfpathlineto{\pgfqpoint{1.947361in}{1.744038in}}%
\pgfpathlineto{\pgfqpoint{1.949052in}{2.789867in}}%
\pgfpathlineto{\pgfqpoint{1.949898in}{1.121253in}}%
\pgfpathlineto{\pgfqpoint{1.950743in}{2.007629in}}%
\pgfpathlineto{\pgfqpoint{1.951589in}{2.406024in}}%
\pgfpathlineto{\pgfqpoint{1.952435in}{2.218439in}}%
\pgfpathlineto{\pgfqpoint{1.953280in}{1.958542in}}%
\pgfpathlineto{\pgfqpoint{1.954126in}{2.371873in}}%
\pgfpathlineto{\pgfqpoint{1.955817in}{1.752149in}}%
\pgfpathlineto{\pgfqpoint{1.956663in}{2.150641in}}%
\pgfpathlineto{\pgfqpoint{1.957508in}{1.641981in}}%
\pgfpathlineto{\pgfqpoint{1.958354in}{2.272265in}}%
\pgfpathlineto{\pgfqpoint{1.959200in}{1.849816in}}%
\pgfpathlineto{\pgfqpoint{1.961736in}{2.199927in}}%
\pgfpathlineto{\pgfqpoint{1.962582in}{1.637763in}}%
\pgfpathlineto{\pgfqpoint{1.963428in}{2.376065in}}%
\pgfpathlineto{\pgfqpoint{1.964273in}{2.164733in}}%
\pgfpathlineto{\pgfqpoint{1.965119in}{2.167196in}}%
\pgfpathlineto{\pgfqpoint{1.965965in}{2.590159in}}%
\pgfpathlineto{\pgfqpoint{1.968501in}{1.641490in}}%
\pgfpathlineto{\pgfqpoint{1.969347in}{2.228228in}}%
\pgfpathlineto{\pgfqpoint{1.970193in}{1.426726in}}%
\pgfpathlineto{\pgfqpoint{1.971038in}{1.639863in}}%
\pgfpathlineto{\pgfqpoint{1.971884in}{2.589938in}}%
\pgfpathlineto{\pgfqpoint{1.972730in}{2.165368in}}%
\pgfpathlineto{\pgfqpoint{1.973575in}{2.162237in}}%
\pgfpathlineto{\pgfqpoint{1.974421in}{2.166917in}}%
\pgfpathlineto{\pgfqpoint{1.975266in}{2.063019in}}%
\pgfpathlineto{\pgfqpoint{1.976112in}{1.640918in}}%
\pgfpathlineto{\pgfqpoint{1.976958in}{1.744905in}}%
\pgfpathlineto{\pgfqpoint{1.978649in}{2.271958in}}%
\pgfpathlineto{\pgfqpoint{1.980340in}{1.428913in}}%
\pgfpathlineto{\pgfqpoint{1.981186in}{2.695057in}}%
\pgfpathlineto{\pgfqpoint{1.982031in}{2.485588in}}%
\pgfpathlineto{\pgfqpoint{1.982877in}{1.955101in}}%
\pgfpathlineto{\pgfqpoint{1.983723in}{2.167264in}}%
\pgfpathlineto{\pgfqpoint{1.984568in}{2.060438in}}%
\pgfpathlineto{\pgfqpoint{1.985414in}{1.424708in}}%
\pgfpathlineto{\pgfqpoint{1.986260in}{2.270868in}}%
\pgfpathlineto{\pgfqpoint{1.987105in}{1.636272in}}%
\pgfpathlineto{\pgfqpoint{1.987951in}{1.215899in}}%
\pgfpathlineto{\pgfqpoint{1.988796in}{2.159413in}}%
\pgfpathlineto{\pgfqpoint{1.989642in}{1.855383in}}%
\pgfpathlineto{\pgfqpoint{1.990488in}{1.642601in}}%
\pgfpathlineto{\pgfqpoint{1.991333in}{1.742729in}}%
\pgfpathlineto{\pgfqpoint{1.992179in}{2.155715in}}%
\pgfpathlineto{\pgfqpoint{1.993025in}{1.959418in}}%
\pgfpathlineto{\pgfqpoint{1.993870in}{1.534197in}}%
\pgfpathlineto{\pgfqpoint{1.994716in}{1.638903in}}%
\pgfpathlineto{\pgfqpoint{1.995561in}{1.772009in}}%
\pgfpathlineto{\pgfqpoint{1.996407in}{1.428044in}}%
\pgfpathlineto{\pgfqpoint{1.997253in}{2.375199in}}%
\pgfpathlineto{\pgfqpoint{1.998098in}{1.956070in}}%
\pgfpathlineto{\pgfqpoint{1.999790in}{2.694064in}}%
\pgfpathlineto{\pgfqpoint{2.001481in}{1.955764in}}%
\pgfpathlineto{\pgfqpoint{2.002326in}{2.060154in}}%
\pgfpathlineto{\pgfqpoint{2.003172in}{1.639431in}}%
\pgfpathlineto{\pgfqpoint{2.004018in}{1.744479in}}%
\pgfpathlineto{\pgfqpoint{2.005709in}{2.167148in}}%
\pgfpathlineto{\pgfqpoint{2.006554in}{2.057354in}}%
\pgfpathlineto{\pgfqpoint{2.007400in}{2.696127in}}%
\pgfpathlineto{\pgfqpoint{2.008246in}{1.742535in}}%
\pgfpathlineto{\pgfqpoint{2.009091in}{2.061783in}}%
\pgfpathlineto{\pgfqpoint{2.009937in}{2.274237in}}%
\pgfpathlineto{\pgfqpoint{2.010783in}{1.642495in}}%
\pgfpathlineto{\pgfqpoint{2.011628in}{2.159772in}}%
\pgfpathlineto{\pgfqpoint{2.014165in}{1.637898in}}%
\pgfpathlineto{\pgfqpoint{2.015856in}{2.166113in}}%
\pgfpathlineto{\pgfqpoint{2.016702in}{1.630005in}}%
\pgfpathlineto{\pgfqpoint{2.017548in}{2.271393in}}%
\pgfpathlineto{\pgfqpoint{2.018393in}{2.162360in}}%
\pgfpathlineto{\pgfqpoint{2.019239in}{1.850868in}}%
\pgfpathlineto{\pgfqpoint{2.020084in}{2.114706in}}%
\pgfpathlineto{\pgfqpoint{2.020930in}{1.694250in}}%
\pgfpathlineto{\pgfqpoint{2.021776in}{2.379286in}}%
\pgfpathlineto{\pgfqpoint{2.022621in}{2.060068in}}%
\pgfpathlineto{\pgfqpoint{2.023467in}{1.323163in}}%
\pgfpathlineto{\pgfqpoint{2.025158in}{2.176257in}}%
\pgfpathlineto{\pgfqpoint{2.026004in}{2.175239in}}%
\pgfpathlineto{\pgfqpoint{2.026849in}{1.637200in}}%
\pgfpathlineto{\pgfqpoint{2.027695in}{2.798998in}}%
\pgfpathlineto{\pgfqpoint{2.028541in}{1.746853in}}%
\pgfpathlineto{\pgfqpoint{2.029386in}{2.167368in}}%
\pgfpathlineto{\pgfqpoint{2.031078in}{1.849811in}}%
\pgfpathlineto{\pgfqpoint{2.031923in}{2.485308in}}%
\pgfpathlineto{\pgfqpoint{2.032769in}{1.744110in}}%
\pgfpathlineto{\pgfqpoint{2.033614in}{2.165566in}}%
\pgfpathlineto{\pgfqpoint{2.034460in}{2.486278in}}%
\pgfpathlineto{\pgfqpoint{2.035306in}{1.424863in}}%
\pgfpathlineto{\pgfqpoint{2.036151in}{1.635217in}}%
\pgfpathlineto{\pgfqpoint{2.036997in}{2.694861in}}%
\pgfpathlineto{\pgfqpoint{2.037843in}{2.378451in}}%
\pgfpathlineto{\pgfqpoint{2.039534in}{1.738867in}}%
\pgfpathlineto{\pgfqpoint{2.040379in}{2.589857in}}%
\pgfpathlineto{\pgfqpoint{2.041225in}{1.954817in}}%
\pgfpathlineto{\pgfqpoint{2.042071in}{1.957573in}}%
\pgfpathlineto{\pgfqpoint{2.042916in}{2.166503in}}%
\pgfpathlineto{\pgfqpoint{2.044608in}{1.321723in}}%
\pgfpathlineto{\pgfqpoint{2.045453in}{1.955435in}}%
\pgfpathlineto{\pgfqpoint{2.046299in}{1.533545in}}%
\pgfpathlineto{\pgfqpoint{2.047144in}{1.637532in}}%
\pgfpathlineto{\pgfqpoint{2.047990in}{2.165193in}}%
\pgfpathlineto{\pgfqpoint{2.048836in}{1.637252in}}%
\pgfpathlineto{\pgfqpoint{2.049681in}{2.690826in}}%
\pgfpathlineto{\pgfqpoint{2.050527in}{1.850412in}}%
\pgfpathlineto{\pgfqpoint{2.051373in}{2.484137in}}%
\pgfpathlineto{\pgfqpoint{2.052218in}{2.166234in}}%
\pgfpathlineto{\pgfqpoint{2.053064in}{2.061841in}}%
\pgfpathlineto{\pgfqpoint{2.053909in}{1.750249in}}%
\pgfpathlineto{\pgfqpoint{2.056446in}{2.588399in}}%
\pgfpathlineto{\pgfqpoint{2.058138in}{1.734554in}}%
\pgfpathlineto{\pgfqpoint{2.058983in}{1.744736in}}%
\pgfpathlineto{\pgfqpoint{2.059829in}{2.270740in}}%
\pgfpathlineto{\pgfqpoint{2.060674in}{2.166283in}}%
\pgfpathlineto{\pgfqpoint{2.063211in}{1.746028in}}%
\pgfpathlineto{\pgfqpoint{2.064057in}{1.855024in}}%
\pgfpathlineto{\pgfqpoint{2.064903in}{1.839814in}}%
\pgfpathlineto{\pgfqpoint{2.065748in}{1.854722in}}%
\pgfpathlineto{\pgfqpoint{2.067439in}{2.169871in}}%
\pgfpathlineto{\pgfqpoint{2.068285in}{2.172163in}}%
\pgfpathlineto{\pgfqpoint{2.069131in}{1.221218in}}%
\pgfpathlineto{\pgfqpoint{2.069976in}{1.426791in}}%
\pgfpathlineto{\pgfqpoint{2.073359in}{2.692206in}}%
\pgfpathlineto{\pgfqpoint{2.075050in}{1.953291in}}%
\pgfpathlineto{\pgfqpoint{2.075896in}{2.431044in}}%
\pgfpathlineto{\pgfqpoint{2.076741in}{1.373113in}}%
\pgfpathlineto{\pgfqpoint{2.077587in}{2.129645in}}%
\pgfpathlineto{\pgfqpoint{2.080124in}{1.585473in}}%
\pgfpathlineto{\pgfqpoint{2.083506in}{2.010631in}}%
\pgfpathlineto{\pgfqpoint{2.084352in}{1.797361in}}%
\pgfpathlineto{\pgfqpoint{2.086889in}{2.431192in}}%
\pgfpathlineto{\pgfqpoint{2.087734in}{1.797407in}}%
\pgfpathlineto{\pgfqpoint{2.088580in}{1.901805in}}%
\pgfpathlineto{\pgfqpoint{2.089426in}{2.114347in}}%
\pgfpathlineto{\pgfqpoint{2.091962in}{1.796651in}}%
\pgfpathlineto{\pgfqpoint{2.092808in}{2.219931in}}%
\pgfpathlineto{\pgfqpoint{2.095345in}{1.688199in}}%
\pgfpathlineto{\pgfqpoint{2.097036in}{2.120897in}}%
\pgfpathlineto{\pgfqpoint{2.097882in}{1.801756in}}%
\pgfpathlineto{\pgfqpoint{2.098727in}{2.324381in}}%
\pgfpathlineto{\pgfqpoint{2.099573in}{2.005895in}}%
\pgfpathlineto{\pgfqpoint{2.101264in}{2.218919in}}%
\pgfpathlineto{\pgfqpoint{2.102956in}{1.799984in}}%
\pgfpathlineto{\pgfqpoint{2.104647in}{2.432882in}}%
\pgfpathlineto{\pgfqpoint{2.105492in}{2.216151in}}%
\pgfpathlineto{\pgfqpoint{2.106338in}{2.640268in}}%
\pgfpathlineto{\pgfqpoint{2.107184in}{1.691900in}}%
\pgfpathlineto{\pgfqpoint{2.108029in}{1.797052in}}%
\pgfpathlineto{\pgfqpoint{2.108875in}{2.012542in}}%
\pgfpathlineto{\pgfqpoint{2.109721in}{1.586626in}}%
\pgfpathlineto{\pgfqpoint{2.110566in}{2.324802in}}%
\pgfpathlineto{\pgfqpoint{2.111412in}{2.114437in}}%
\pgfpathlineto{\pgfqpoint{2.112257in}{2.008779in}}%
\pgfpathlineto{\pgfqpoint{2.113103in}{2.220884in}}%
\pgfpathlineto{\pgfqpoint{2.113949in}{2.219718in}}%
\pgfpathlineto{\pgfqpoint{2.117331in}{1.261353in}}%
\pgfpathlineto{\pgfqpoint{2.119868in}{2.113903in}}%
\pgfpathlineto{\pgfqpoint{2.122405in}{1.373049in}}%
\pgfpathlineto{\pgfqpoint{2.123251in}{1.902978in}}%
\pgfpathlineto{\pgfqpoint{2.124096in}{1.901951in}}%
\pgfpathlineto{\pgfqpoint{2.125787in}{1.796141in}}%
\pgfpathlineto{\pgfqpoint{2.126633in}{2.318749in}}%
\pgfpathlineto{\pgfqpoint{2.127479in}{1.598764in}}%
\pgfpathlineto{\pgfqpoint{2.128324in}{2.117671in}}%
\pgfpathlineto{\pgfqpoint{2.129170in}{2.112623in}}%
\pgfpathlineto{\pgfqpoint{2.130016in}{1.796161in}}%
\pgfpathlineto{\pgfqpoint{2.130861in}{2.535428in}}%
\pgfpathlineto{\pgfqpoint{2.131707in}{2.115541in}}%
\pgfpathlineto{\pgfqpoint{2.132552in}{2.113312in}}%
\pgfpathlineto{\pgfqpoint{2.133398in}{2.218778in}}%
\pgfpathlineto{\pgfqpoint{2.135935in}{1.691952in}}%
\pgfpathlineto{\pgfqpoint{2.136781in}{1.902549in}}%
\pgfpathlineto{\pgfqpoint{2.137626in}{1.695178in}}%
\pgfpathlineto{\pgfqpoint{2.138472in}{2.220667in}}%
\pgfpathlineto{\pgfqpoint{2.139317in}{1.372557in}}%
\pgfpathlineto{\pgfqpoint{2.140163in}{2.011892in}}%
\pgfpathlineto{\pgfqpoint{2.141009in}{2.325147in}}%
\pgfpathlineto{\pgfqpoint{2.141854in}{1.694455in}}%
\pgfpathlineto{\pgfqpoint{2.142700in}{2.326407in}}%
\pgfpathlineto{\pgfqpoint{2.143546in}{1.900042in}}%
\pgfpathlineto{\pgfqpoint{2.145237in}{1.797190in}}%
\pgfpathlineto{\pgfqpoint{2.146082in}{1.903303in}}%
\pgfpathlineto{\pgfqpoint{2.146928in}{1.902734in}}%
\pgfpathlineto{\pgfqpoint{2.148619in}{2.535121in}}%
\pgfpathlineto{\pgfqpoint{2.149465in}{1.690690in}}%
\pgfpathlineto{\pgfqpoint{2.150311in}{2.430903in}}%
\pgfpathlineto{\pgfqpoint{2.151156in}{2.006953in}}%
\pgfpathlineto{\pgfqpoint{2.152002in}{2.431118in}}%
\pgfpathlineto{\pgfqpoint{2.152847in}{1.373001in}}%
\pgfpathlineto{\pgfqpoint{2.153693in}{1.902982in}}%
\pgfpathlineto{\pgfqpoint{2.154539in}{1.480167in}}%
\pgfpathlineto{\pgfqpoint{2.155384in}{2.746142in}}%
\pgfpathlineto{\pgfqpoint{2.156230in}{1.902337in}}%
\pgfpathlineto{\pgfqpoint{2.157076in}{1.798510in}}%
\pgfpathlineto{\pgfqpoint{2.158767in}{2.430653in}}%
\pgfpathlineto{\pgfqpoint{2.160458in}{2.006779in}}%
\pgfpathlineto{\pgfqpoint{2.162149in}{2.642851in}}%
\pgfpathlineto{\pgfqpoint{2.164686in}{1.496939in}}%
\pgfpathlineto{\pgfqpoint{2.166377in}{2.748612in}}%
\pgfpathlineto{\pgfqpoint{2.168914in}{1.057079in}}%
\pgfpathlineto{\pgfqpoint{2.169760in}{2.218693in}}%
\pgfpathlineto{\pgfqpoint{2.170605in}{2.008870in}}%
\pgfpathlineto{\pgfqpoint{2.171451in}{2.220038in}}%
\pgfpathlineto{\pgfqpoint{2.172297in}{1.481159in}}%
\pgfpathlineto{\pgfqpoint{2.173988in}{2.430754in}}%
\pgfpathlineto{\pgfqpoint{2.177370in}{1.268789in}}%
\pgfpathlineto{\pgfqpoint{2.178216in}{2.220111in}}%
\pgfpathlineto{\pgfqpoint{2.179062in}{1.479789in}}%
\pgfpathlineto{\pgfqpoint{2.180753in}{1.902293in}}%
\pgfpathlineto{\pgfqpoint{2.182444in}{1.585271in}}%
\pgfpathlineto{\pgfqpoint{2.184135in}{2.325366in}}%
\pgfpathlineto{\pgfqpoint{2.184981in}{1.903013in}}%
\pgfpathlineto{\pgfqpoint{2.185827in}{2.642770in}}%
\pgfpathlineto{\pgfqpoint{2.186672in}{1.902321in}}%
\pgfpathlineto{\pgfqpoint{2.187518in}{2.009065in}}%
\pgfpathlineto{\pgfqpoint{2.188364in}{2.114074in}}%
\pgfpathlineto{\pgfqpoint{2.189209in}{2.853429in}}%
\pgfpathlineto{\pgfqpoint{2.190055in}{2.325340in}}%
\pgfpathlineto{\pgfqpoint{2.190900in}{1.902267in}}%
\pgfpathlineto{\pgfqpoint{2.191746in}{2.539343in}}%
\pgfpathlineto{\pgfqpoint{2.192592in}{2.114734in}}%
\pgfpathlineto{\pgfqpoint{2.193437in}{2.008260in}}%
\pgfpathlineto{\pgfqpoint{2.194283in}{2.113906in}}%
\pgfpathlineto{\pgfqpoint{2.195129in}{2.113885in}}%
\pgfpathlineto{\pgfqpoint{2.195974in}{2.957713in}}%
\pgfpathlineto{\pgfqpoint{2.196820in}{2.008061in}}%
\pgfpathlineto{\pgfqpoint{2.197665in}{2.431115in}}%
\pgfpathlineto{\pgfqpoint{2.199357in}{1.796793in}}%
\pgfpathlineto{\pgfqpoint{2.200202in}{2.008507in}}%
\pgfpathlineto{\pgfqpoint{2.201894in}{1.268084in}}%
\pgfpathlineto{\pgfqpoint{2.202739in}{1.585088in}}%
\pgfpathlineto{\pgfqpoint{2.203585in}{2.536886in}}%
\pgfpathlineto{\pgfqpoint{2.204430in}{1.585351in}}%
\pgfpathlineto{\pgfqpoint{2.205276in}{2.114644in}}%
\pgfpathlineto{\pgfqpoint{2.207813in}{1.796628in}}%
\pgfpathlineto{\pgfqpoint{2.209504in}{1.690696in}}%
\pgfpathlineto{\pgfqpoint{2.210350in}{1.056404in}}%
\pgfpathlineto{\pgfqpoint{2.212887in}{2.325321in}}%
\pgfpathlineto{\pgfqpoint{2.213732in}{2.326349in}}%
\pgfpathlineto{\pgfqpoint{2.217115in}{1.373912in}}%
\pgfpathlineto{\pgfqpoint{2.219652in}{1.902572in}}%
\pgfpathlineto{\pgfqpoint{2.221343in}{1.902092in}}%
\pgfpathlineto{\pgfqpoint{2.222189in}{2.325176in}}%
\pgfpathlineto{\pgfqpoint{2.223034in}{1.796746in}}%
\pgfpathlineto{\pgfqpoint{2.223880in}{2.219506in}}%
\pgfpathlineto{\pgfqpoint{2.226417in}{1.694950in}}%
\pgfpathlineto{\pgfqpoint{2.227262in}{1.901959in}}%
\pgfpathlineto{\pgfqpoint{2.228108in}{1.691353in}}%
\pgfpathlineto{\pgfqpoint{2.228954in}{2.959801in}}%
\pgfpathlineto{\pgfqpoint{2.229799in}{2.096693in}}%
\pgfpathlineto{\pgfqpoint{2.230645in}{1.162726in}}%
\pgfpathlineto{\pgfqpoint{2.231490in}{2.431011in}}%
\pgfpathlineto{\pgfqpoint{2.232336in}{2.113875in}}%
\pgfpathlineto{\pgfqpoint{2.233182in}{1.902583in}}%
\pgfpathlineto{\pgfqpoint{2.234027in}{2.001516in}}%
\pgfpathlineto{\pgfqpoint{2.234873in}{1.902208in}}%
\pgfpathlineto{\pgfqpoint{2.235719in}{2.176933in}}%
\pgfpathlineto{\pgfqpoint{2.236564in}{2.054757in}}%
\pgfpathlineto{\pgfqpoint{2.237410in}{1.639654in}}%
\pgfpathlineto{\pgfqpoint{2.238255in}{1.831047in}}%
\pgfpathlineto{\pgfqpoint{2.239947in}{2.374897in}}%
\pgfpathlineto{\pgfqpoint{2.240792in}{1.541145in}}%
\pgfpathlineto{\pgfqpoint{2.241638in}{2.062989in}}%
\pgfpathlineto{\pgfqpoint{2.242483in}{2.379479in}}%
\pgfpathlineto{\pgfqpoint{2.243329in}{1.956538in}}%
\pgfpathlineto{\pgfqpoint{2.244175in}{2.488129in}}%
\pgfpathlineto{\pgfqpoint{2.245020in}{2.103750in}}%
\pgfpathlineto{\pgfqpoint{2.245866in}{1.426870in}}%
\pgfpathlineto{\pgfqpoint{2.246712in}{2.483224in}}%
\pgfpathlineto{\pgfqpoint{2.247557in}{2.278295in}}%
\pgfpathlineto{\pgfqpoint{2.248403in}{1.638370in}}%
\pgfpathlineto{\pgfqpoint{2.249248in}{2.265325in}}%
\pgfpathlineto{\pgfqpoint{2.250094in}{1.740902in}}%
\pgfpathlineto{\pgfqpoint{2.250940in}{3.117618in}}%
\pgfpathlineto{\pgfqpoint{2.251785in}{2.480355in}}%
\pgfpathlineto{\pgfqpoint{2.252631in}{2.274862in}}%
\pgfpathlineto{\pgfqpoint{2.253477in}{1.321592in}}%
\pgfpathlineto{\pgfqpoint{2.256013in}{2.377687in}}%
\pgfpathlineto{\pgfqpoint{2.256859in}{1.430392in}}%
\pgfpathlineto{\pgfqpoint{2.257705in}{2.169137in}}%
\pgfpathlineto{\pgfqpoint{2.258550in}{2.065605in}}%
\pgfpathlineto{\pgfqpoint{2.259396in}{2.165486in}}%
\pgfpathlineto{\pgfqpoint{2.261087in}{2.060592in}}%
\pgfpathlineto{\pgfqpoint{2.261933in}{2.062580in}}%
\pgfpathlineto{\pgfqpoint{2.263624in}{2.800608in}}%
\pgfpathlineto{\pgfqpoint{2.264470in}{1.532859in}}%
\pgfpathlineto{\pgfqpoint{2.265315in}{2.166184in}}%
\pgfpathlineto{\pgfqpoint{2.266161in}{2.378198in}}%
\pgfpathlineto{\pgfqpoint{2.267007in}{1.426714in}}%
\pgfpathlineto{\pgfqpoint{2.267852in}{2.435561in}}%
\pgfpathlineto{\pgfqpoint{2.268698in}{1.815962in}}%
\pgfpathlineto{\pgfqpoint{2.271235in}{2.166783in}}%
\pgfpathlineto{\pgfqpoint{2.272080in}{1.520759in}}%
\pgfpathlineto{\pgfqpoint{2.272926in}{2.694987in}}%
\pgfpathlineto{\pgfqpoint{2.273772in}{1.579763in}}%
\pgfpathlineto{\pgfqpoint{2.274617in}{1.991205in}}%
\pgfpathlineto{\pgfqpoint{2.275463in}{2.198982in}}%
\pgfpathlineto{\pgfqpoint{2.278845in}{1.043650in}}%
\pgfpathlineto{\pgfqpoint{2.279691in}{2.418896in}}%
\pgfpathlineto{\pgfqpoint{2.280537in}{1.954491in}}%
\pgfpathlineto{\pgfqpoint{2.281382in}{2.482160in}}%
\pgfpathlineto{\pgfqpoint{2.282228in}{2.417514in}}%
\pgfpathlineto{\pgfqpoint{2.283073in}{1.392948in}}%
\pgfpathlineto{\pgfqpoint{2.283919in}{2.026387in}}%
\pgfpathlineto{\pgfqpoint{2.284765in}{2.333435in}}%
\pgfpathlineto{\pgfqpoint{2.285610in}{1.973550in}}%
\pgfpathlineto{\pgfqpoint{2.286456in}{2.313675in}}%
\pgfpathlineto{\pgfqpoint{2.288993in}{1.505103in}}%
\pgfpathlineto{\pgfqpoint{2.291530in}{2.289247in}}%
\pgfpathlineto{\pgfqpoint{2.292375in}{2.018863in}}%
\pgfpathlineto{\pgfqpoint{2.293221in}{2.736390in}}%
\pgfpathlineto{\pgfqpoint{2.294067in}{1.459620in}}%
\pgfpathlineto{\pgfqpoint{2.294912in}{1.602642in}}%
\pgfpathlineto{\pgfqpoint{2.298295in}{2.900465in}}%
\pgfpathlineto{\pgfqpoint{2.299986in}{1.745772in}}%
\pgfpathlineto{\pgfqpoint{2.300832in}{1.858139in}}%
\pgfpathlineto{\pgfqpoint{2.301677in}{1.561212in}}%
\pgfpathlineto{\pgfqpoint{2.302523in}{2.376415in}}%
\pgfpathlineto{\pgfqpoint{2.303368in}{1.849913in}}%
\pgfpathlineto{\pgfqpoint{2.304214in}{2.482189in}}%
\pgfpathlineto{\pgfqpoint{2.305060in}{1.738690in}}%
\pgfpathlineto{\pgfqpoint{2.305905in}{2.482551in}}%
\pgfpathlineto{\pgfqpoint{2.306751in}{2.259548in}}%
\pgfpathlineto{\pgfqpoint{2.307597in}{1.741844in}}%
\pgfpathlineto{\pgfqpoint{2.308442in}{2.163739in}}%
\pgfpathlineto{\pgfqpoint{2.310133in}{1.734871in}}%
\pgfpathlineto{\pgfqpoint{2.310979in}{1.954126in}}%
\pgfpathlineto{\pgfqpoint{2.311825in}{1.940269in}}%
\pgfpathlineto{\pgfqpoint{2.313516in}{1.708019in}}%
\pgfpathlineto{\pgfqpoint{2.314362in}{1.837824in}}%
\pgfpathlineto{\pgfqpoint{2.315207in}{2.061937in}}%
\pgfpathlineto{\pgfqpoint{2.316898in}{1.259098in}}%
\pgfpathlineto{\pgfqpoint{2.317744in}{2.031764in}}%
\pgfpathlineto{\pgfqpoint{2.318590in}{1.164127in}}%
\pgfpathlineto{\pgfqpoint{2.319435in}{1.750792in}}%
\pgfpathlineto{\pgfqpoint{2.320281in}{1.967631in}}%
\pgfpathlineto{\pgfqpoint{2.321126in}{1.861779in}}%
\pgfpathlineto{\pgfqpoint{2.321972in}{1.957344in}}%
\pgfpathlineto{\pgfqpoint{2.322818in}{2.251557in}}%
\pgfpathlineto{\pgfqpoint{2.323663in}{1.849740in}}%
\pgfpathlineto{\pgfqpoint{2.324509in}{2.589109in}}%
\pgfpathlineto{\pgfqpoint{2.325355in}{2.171452in}}%
\pgfpathlineto{\pgfqpoint{2.326200in}{0.898842in}}%
\pgfpathlineto{\pgfqpoint{2.327046in}{1.633605in}}%
\pgfpathlineto{\pgfqpoint{2.327891in}{1.859463in}}%
\pgfpathlineto{\pgfqpoint{2.328737in}{1.641625in}}%
\pgfpathlineto{\pgfqpoint{2.329583in}{1.749013in}}%
\pgfpathlineto{\pgfqpoint{2.330428in}{1.849184in}}%
\pgfpathlineto{\pgfqpoint{2.331274in}{2.166059in}}%
\pgfpathlineto{\pgfqpoint{2.332120in}{1.743287in}}%
\pgfpathlineto{\pgfqpoint{2.333811in}{2.588425in}}%
\pgfpathlineto{\pgfqpoint{2.334656in}{1.638050in}}%
\pgfpathlineto{\pgfqpoint{2.335502in}{1.849527in}}%
\pgfpathlineto{\pgfqpoint{2.336348in}{2.168378in}}%
\pgfpathlineto{\pgfqpoint{2.337193in}{1.955383in}}%
\pgfpathlineto{\pgfqpoint{2.338039in}{1.744326in}}%
\pgfpathlineto{\pgfqpoint{2.338885in}{1.850141in}}%
\pgfpathlineto{\pgfqpoint{2.339730in}{2.272513in}}%
\pgfpathlineto{\pgfqpoint{2.340576in}{2.061004in}}%
\pgfpathlineto{\pgfqpoint{2.342267in}{1.638102in}}%
\pgfpathlineto{\pgfqpoint{2.344804in}{2.589700in}}%
\pgfpathlineto{\pgfqpoint{2.346495in}{1.639197in}}%
\pgfpathlineto{\pgfqpoint{2.347341in}{1.849751in}}%
\pgfpathlineto{\pgfqpoint{2.348186in}{2.800468in}}%
\pgfpathlineto{\pgfqpoint{2.349032in}{2.166914in}}%
\pgfpathlineto{\pgfqpoint{2.349878in}{2.061031in}}%
\pgfpathlineto{\pgfqpoint{2.350723in}{1.743919in}}%
\pgfpathlineto{\pgfqpoint{2.351569in}{2.272552in}}%
\pgfpathlineto{\pgfqpoint{2.352415in}{2.061006in}}%
\pgfpathlineto{\pgfqpoint{2.353260in}{2.062820in}}%
\pgfpathlineto{\pgfqpoint{2.354106in}{2.483874in}}%
\pgfpathlineto{\pgfqpoint{2.354951in}{1.298662in}}%
\pgfpathlineto{\pgfqpoint{2.355797in}{1.743925in}}%
\pgfpathlineto{\pgfqpoint{2.356643in}{1.853692in}}%
\pgfpathlineto{\pgfqpoint{2.357488in}{1.320946in}}%
\pgfpathlineto{\pgfqpoint{2.359180in}{2.483905in}}%
\pgfpathlineto{\pgfqpoint{2.361716in}{1.532230in}}%
\pgfpathlineto{\pgfqpoint{2.362562in}{1.743347in}}%
\pgfpathlineto{\pgfqpoint{2.364253in}{1.426597in}}%
\pgfpathlineto{\pgfqpoint{2.365099in}{1.215390in}}%
\pgfpathlineto{\pgfqpoint{2.365945in}{1.878149in}}%
\pgfpathlineto{\pgfqpoint{2.366790in}{1.849623in}}%
\pgfpathlineto{\pgfqpoint{2.367636in}{1.849609in}}%
\pgfpathlineto{\pgfqpoint{2.368481in}{2.378259in}}%
\pgfpathlineto{\pgfqpoint{2.369327in}{1.955290in}}%
\pgfpathlineto{\pgfqpoint{2.370173in}{2.272519in}}%
\pgfpathlineto{\pgfqpoint{2.371018in}{1.532504in}}%
\pgfpathlineto{\pgfqpoint{2.372710in}{2.906819in}}%
\pgfpathlineto{\pgfqpoint{2.375246in}{1.743975in}}%
\pgfpathlineto{\pgfqpoint{2.376092in}{1.637958in}}%
\pgfpathlineto{\pgfqpoint{2.377783in}{2.166601in}}%
\pgfpathlineto{\pgfqpoint{2.379475in}{1.638112in}}%
\pgfpathlineto{\pgfqpoint{2.380320in}{1.744447in}}%
\pgfpathlineto{\pgfqpoint{2.381166in}{2.378225in}}%
\pgfpathlineto{\pgfqpoint{2.382011in}{2.378106in}}%
\pgfpathlineto{\pgfqpoint{2.383703in}{2.060835in}}%
\pgfpathlineto{\pgfqpoint{2.384548in}{1.322541in}}%
\pgfpathlineto{\pgfqpoint{2.385394in}{2.484119in}}%
\pgfpathlineto{\pgfqpoint{2.386240in}{2.289134in}}%
\pgfpathlineto{\pgfqpoint{2.387085in}{2.061108in}}%
\pgfpathlineto{\pgfqpoint{2.387931in}{2.378067in}}%
\pgfpathlineto{\pgfqpoint{2.388776in}{2.166647in}}%
\pgfpathlineto{\pgfqpoint{2.389622in}{2.061032in}}%
\pgfpathlineto{\pgfqpoint{2.390468in}{2.801206in}}%
\pgfpathlineto{\pgfqpoint{2.391313in}{1.532730in}}%
\pgfpathlineto{\pgfqpoint{2.392159in}{1.743836in}}%
\pgfpathlineto{\pgfqpoint{2.393850in}{2.589491in}}%
\pgfpathlineto{\pgfqpoint{2.395541in}{1.532447in}}%
\pgfpathlineto{\pgfqpoint{2.397233in}{2.272447in}}%
\pgfpathlineto{\pgfqpoint{2.398078in}{1.955292in}}%
\pgfpathlineto{\pgfqpoint{2.398924in}{2.483865in}}%
\pgfpathlineto{\pgfqpoint{2.399769in}{1.320908in}}%
\pgfpathlineto{\pgfqpoint{2.400615in}{2.272579in}}%
\pgfpathlineto{\pgfqpoint{2.403152in}{1.955146in}}%
\pgfpathlineto{\pgfqpoint{2.403998in}{1.532424in}}%
\pgfpathlineto{\pgfqpoint{2.405689in}{2.272194in}}%
\pgfpathlineto{\pgfqpoint{2.406534in}{2.166603in}}%
\pgfpathlineto{\pgfqpoint{2.407380in}{1.638552in}}%
\pgfpathlineto{\pgfqpoint{2.408226in}{2.483794in}}%
\pgfpathlineto{\pgfqpoint{2.409071in}{1.955432in}}%
\pgfpathlineto{\pgfqpoint{2.409917in}{2.061024in}}%
\pgfpathlineto{\pgfqpoint{2.410763in}{2.589110in}}%
\pgfpathlineto{\pgfqpoint{2.411608in}{1.517369in}}%
\pgfpathlineto{\pgfqpoint{2.412454in}{1.638112in}}%
\pgfpathlineto{\pgfqpoint{2.414991in}{2.053205in}}%
\pgfpathlineto{\pgfqpoint{2.415836in}{1.565585in}}%
\pgfpathlineto{\pgfqpoint{2.416682in}{1.737474in}}%
\pgfpathlineto{\pgfqpoint{2.417528in}{2.032741in}}%
\pgfpathlineto{\pgfqpoint{2.418373in}{1.366351in}}%
\pgfpathlineto{\pgfqpoint{2.419219in}{1.743867in}}%
\pgfpathlineto{\pgfqpoint{2.420910in}{2.642958in}}%
\pgfpathlineto{\pgfqpoint{2.421756in}{2.325283in}}%
\pgfpathlineto{\pgfqpoint{2.423447in}{1.796536in}}%
\pgfpathlineto{\pgfqpoint{2.424293in}{2.537103in}}%
\pgfpathlineto{\pgfqpoint{2.425138in}{1.585135in}}%
\pgfpathlineto{\pgfqpoint{2.425984in}{2.431108in}}%
\pgfpathlineto{\pgfqpoint{2.426829in}{2.219582in}}%
\pgfpathlineto{\pgfqpoint{2.427675in}{2.008210in}}%
\pgfpathlineto{\pgfqpoint{2.428521in}{2.325301in}}%
\pgfpathlineto{\pgfqpoint{2.429366in}{1.162407in}}%
\pgfpathlineto{\pgfqpoint{2.430212in}{1.690976in}}%
\pgfpathlineto{\pgfqpoint{2.432749in}{2.325228in}}%
\pgfpathlineto{\pgfqpoint{2.433594in}{1.691553in}}%
\pgfpathlineto{\pgfqpoint{2.434440in}{2.045744in}}%
\pgfpathlineto{\pgfqpoint{2.435286in}{2.219651in}}%
\pgfpathlineto{\pgfqpoint{2.436977in}{1.476160in}}%
\pgfpathlineto{\pgfqpoint{2.438668in}{2.233826in}}%
\pgfpathlineto{\pgfqpoint{2.439514in}{2.216311in}}%
\pgfpathlineto{\pgfqpoint{2.440359in}{2.217904in}}%
\pgfpathlineto{\pgfqpoint{2.441205in}{1.530148in}}%
\pgfpathlineto{\pgfqpoint{2.442051in}{2.914980in}}%
\pgfpathlineto{\pgfqpoint{2.442896in}{1.863901in}}%
\pgfpathlineto{\pgfqpoint{2.445433in}{2.421254in}}%
\pgfpathlineto{\pgfqpoint{2.446279in}{2.503498in}}%
\pgfpathlineto{\pgfqpoint{2.447124in}{1.706810in}}%
\pgfpathlineto{\pgfqpoint{2.447970in}{2.078388in}}%
\pgfpathlineto{\pgfqpoint{2.448816in}{2.002005in}}%
\pgfpathlineto{\pgfqpoint{2.449661in}{2.008005in}}%
\pgfpathlineto{\pgfqpoint{2.450507in}{2.063979in}}%
\pgfpathlineto{\pgfqpoint{2.451353in}{1.746276in}}%
\pgfpathlineto{\pgfqpoint{2.452198in}{2.270928in}}%
\pgfpathlineto{\pgfqpoint{2.453044in}{1.637197in}}%
\pgfpathlineto{\pgfqpoint{2.453889in}{1.850323in}}%
\pgfpathlineto{\pgfqpoint{2.455581in}{2.197472in}}%
\pgfpathlineto{\pgfqpoint{2.456426in}{2.060104in}}%
\pgfpathlineto{\pgfqpoint{2.457272in}{2.060584in}}%
\pgfpathlineto{\pgfqpoint{2.458118in}{2.493941in}}%
\pgfpathlineto{\pgfqpoint{2.458963in}{1.531977in}}%
\pgfpathlineto{\pgfqpoint{2.459809in}{2.166839in}}%
\pgfpathlineto{\pgfqpoint{2.461500in}{1.320839in}}%
\pgfpathlineto{\pgfqpoint{2.463191in}{2.061081in}}%
\pgfpathlineto{\pgfqpoint{2.464037in}{1.849360in}}%
\pgfpathlineto{\pgfqpoint{2.467419in}{2.377670in}}%
\pgfpathlineto{\pgfqpoint{2.469111in}{1.637101in}}%
\pgfpathlineto{\pgfqpoint{2.469956in}{1.638406in}}%
\pgfpathlineto{\pgfqpoint{2.470802in}{2.273900in}}%
\pgfpathlineto{\pgfqpoint{2.471648in}{1.639460in}}%
\pgfpathlineto{\pgfqpoint{2.472493in}{2.483936in}}%
\pgfpathlineto{\pgfqpoint{2.473339in}{1.638451in}}%
\pgfpathlineto{\pgfqpoint{2.474184in}{1.958616in}}%
\pgfpathlineto{\pgfqpoint{2.475030in}{2.375344in}}%
\pgfpathlineto{\pgfqpoint{2.475876in}{1.849279in}}%
\pgfpathlineto{\pgfqpoint{2.476721in}{2.273115in}}%
\pgfpathlineto{\pgfqpoint{2.477567in}{2.484544in}}%
\pgfpathlineto{\pgfqpoint{2.480104in}{1.744240in}}%
\pgfpathlineto{\pgfqpoint{2.480949in}{2.166439in}}%
\pgfpathlineto{\pgfqpoint{2.481795in}{1.959083in}}%
\pgfpathlineto{\pgfqpoint{2.482641in}{1.745252in}}%
\pgfpathlineto{\pgfqpoint{2.484332in}{2.166828in}}%
\pgfpathlineto{\pgfqpoint{2.485177in}{2.076412in}}%
\pgfpathlineto{\pgfqpoint{2.486023in}{2.272449in}}%
\pgfpathlineto{\pgfqpoint{2.487714in}{1.744353in}}%
\pgfpathlineto{\pgfqpoint{2.488560in}{2.374985in}}%
\pgfpathlineto{\pgfqpoint{2.489406in}{2.166904in}}%
\pgfpathlineto{\pgfqpoint{2.491097in}{2.800926in}}%
\pgfpathlineto{\pgfqpoint{2.491942in}{1.532389in}}%
\pgfpathlineto{\pgfqpoint{2.492788in}{1.955455in}}%
\pgfpathlineto{\pgfqpoint{2.493634in}{1.841922in}}%
\pgfpathlineto{\pgfqpoint{2.494479in}{2.586001in}}%
\pgfpathlineto{\pgfqpoint{2.495325in}{2.166594in}}%
\pgfpathlineto{\pgfqpoint{2.496171in}{2.166915in}}%
\pgfpathlineto{\pgfqpoint{2.497016in}{1.715215in}}%
\pgfpathlineto{\pgfqpoint{2.497862in}{2.422847in}}%
\pgfpathlineto{\pgfqpoint{2.498707in}{2.322048in}}%
\pgfpathlineto{\pgfqpoint{2.499553in}{1.793926in}}%
\pgfpathlineto{\pgfqpoint{2.500399in}{2.747139in}}%
\pgfpathlineto{\pgfqpoint{2.501244in}{2.107673in}}%
\pgfpathlineto{\pgfqpoint{2.502090in}{2.113483in}}%
\pgfpathlineto{\pgfqpoint{2.502936in}{2.641989in}}%
\pgfpathlineto{\pgfqpoint{2.504627in}{1.690323in}}%
\pgfpathlineto{\pgfqpoint{2.505472in}{1.796291in}}%
\pgfpathlineto{\pgfqpoint{2.506318in}{2.323399in}}%
\pgfpathlineto{\pgfqpoint{2.507164in}{1.375097in}}%
\pgfpathlineto{\pgfqpoint{2.508009in}{2.114188in}}%
\pgfpathlineto{\pgfqpoint{2.508855in}{2.008399in}}%
\pgfpathlineto{\pgfqpoint{2.509701in}{2.113830in}}%
\pgfpathlineto{\pgfqpoint{2.510546in}{1.904699in}}%
\pgfpathlineto{\pgfqpoint{2.513083in}{2.221418in}}%
\pgfpathlineto{\pgfqpoint{2.515620in}{1.690880in}}%
\pgfpathlineto{\pgfqpoint{2.516466in}{2.325594in}}%
\pgfpathlineto{\pgfqpoint{2.519002in}{1.480121in}}%
\pgfpathlineto{\pgfqpoint{2.520694in}{2.747852in}}%
\pgfpathlineto{\pgfqpoint{2.521539in}{2.536939in}}%
\pgfpathlineto{\pgfqpoint{2.522385in}{1.584144in}}%
\pgfpathlineto{\pgfqpoint{2.523231in}{1.796322in}}%
\pgfpathlineto{\pgfqpoint{2.524076in}{1.587122in}}%
\pgfpathlineto{\pgfqpoint{2.525767in}{2.328345in}}%
\pgfpathlineto{\pgfqpoint{2.527459in}{1.902318in}}%
\pgfpathlineto{\pgfqpoint{2.528304in}{1.902670in}}%
\pgfpathlineto{\pgfqpoint{2.529150in}{2.536494in}}%
\pgfpathlineto{\pgfqpoint{2.530841in}{1.902621in}}%
\pgfpathlineto{\pgfqpoint{2.531687in}{2.218822in}}%
\pgfpathlineto{\pgfqpoint{2.532532in}{1.797156in}}%
\pgfpathlineto{\pgfqpoint{2.533378in}{2.220179in}}%
\pgfpathlineto{\pgfqpoint{2.534224in}{2.043385in}}%
\pgfpathlineto{\pgfqpoint{2.535069in}{1.690782in}}%
\pgfpathlineto{\pgfqpoint{2.536761in}{2.536387in}}%
\pgfpathlineto{\pgfqpoint{2.539297in}{1.481246in}}%
\pgfpathlineto{\pgfqpoint{2.540143in}{2.220873in}}%
\pgfpathlineto{\pgfqpoint{2.540989in}{1.691516in}}%
\pgfpathlineto{\pgfqpoint{2.542680in}{2.747983in}}%
\pgfpathlineto{\pgfqpoint{2.543526in}{2.112827in}}%
\pgfpathlineto{\pgfqpoint{2.544371in}{2.119844in}}%
\pgfpathlineto{\pgfqpoint{2.545217in}{2.220094in}}%
\pgfpathlineto{\pgfqpoint{2.546908in}{2.007751in}}%
\pgfpathlineto{\pgfqpoint{2.547754in}{2.212645in}}%
\pgfpathlineto{\pgfqpoint{2.548599in}{2.114530in}}%
\pgfpathlineto{\pgfqpoint{2.550291in}{1.902313in}}%
\pgfpathlineto{\pgfqpoint{2.552827in}{2.219187in}}%
\pgfpathlineto{\pgfqpoint{2.555364in}{1.802946in}}%
\pgfpathlineto{\pgfqpoint{2.556210in}{1.895253in}}%
\pgfpathlineto{\pgfqpoint{2.557056in}{1.696437in}}%
\pgfpathlineto{\pgfqpoint{2.557901in}{2.527687in}}%
\pgfpathlineto{\pgfqpoint{2.558747in}{2.105170in}}%
\pgfpathlineto{\pgfqpoint{2.559592in}{1.688577in}}%
\pgfpathlineto{\pgfqpoint{2.560438in}{1.736522in}}%
\pgfpathlineto{\pgfqpoint{2.561284in}{2.154441in}}%
\pgfpathlineto{\pgfqpoint{2.562129in}{1.841361in}}%
\pgfpathlineto{\pgfqpoint{2.562975in}{2.169384in}}%
\pgfpathlineto{\pgfqpoint{2.563820in}{2.063364in}}%
\pgfpathlineto{\pgfqpoint{2.564666in}{1.956241in}}%
\pgfpathlineto{\pgfqpoint{2.565512in}{2.378160in}}%
\pgfpathlineto{\pgfqpoint{2.566357in}{1.427248in}}%
\pgfpathlineto{\pgfqpoint{2.567203in}{1.956776in}}%
\pgfpathlineto{\pgfqpoint{2.568049in}{2.370602in}}%
\pgfpathlineto{\pgfqpoint{2.568894in}{2.076739in}}%
\pgfpathlineto{\pgfqpoint{2.570585in}{1.311371in}}%
\pgfpathlineto{\pgfqpoint{2.572277in}{2.027214in}}%
\pgfpathlineto{\pgfqpoint{2.573122in}{1.532655in}}%
\pgfpathlineto{\pgfqpoint{2.573968in}{2.063977in}}%
\pgfpathlineto{\pgfqpoint{2.574814in}{1.220715in}}%
\pgfpathlineto{\pgfqpoint{2.577350in}{2.272205in}}%
\pgfpathlineto{\pgfqpoint{2.578196in}{2.272021in}}%
\pgfpathlineto{\pgfqpoint{2.579042in}{2.482943in}}%
\pgfpathlineto{\pgfqpoint{2.579887in}{1.849094in}}%
\pgfpathlineto{\pgfqpoint{2.581579in}{2.483894in}}%
\pgfpathlineto{\pgfqpoint{2.583270in}{1.426316in}}%
\pgfpathlineto{\pgfqpoint{2.584961in}{2.166925in}}%
\pgfpathlineto{\pgfqpoint{2.586652in}{1.850667in}}%
\pgfpathlineto{\pgfqpoint{2.587498in}{1.532632in}}%
\pgfpathlineto{\pgfqpoint{2.588344in}{1.955319in}}%
\pgfpathlineto{\pgfqpoint{2.589189in}{1.534410in}}%
\pgfpathlineto{\pgfqpoint{2.590035in}{2.272986in}}%
\pgfpathlineto{\pgfqpoint{2.590880in}{2.166428in}}%
\pgfpathlineto{\pgfqpoint{2.591726in}{1.956327in}}%
\pgfpathlineto{\pgfqpoint{2.592572in}{2.378145in}}%
\pgfpathlineto{\pgfqpoint{2.593417in}{1.637615in}}%
\pgfpathlineto{\pgfqpoint{2.594263in}{1.954674in}}%
\pgfpathlineto{\pgfqpoint{2.595109in}{1.851038in}}%
\pgfpathlineto{\pgfqpoint{2.596800in}{2.063540in}}%
\pgfpathlineto{\pgfqpoint{2.597645in}{2.478450in}}%
\pgfpathlineto{\pgfqpoint{2.598491in}{2.171591in}}%
\pgfpathlineto{\pgfqpoint{2.599337in}{2.167055in}}%
\pgfpathlineto{\pgfqpoint{2.601028in}{1.745014in}}%
\pgfpathlineto{\pgfqpoint{2.601874in}{1.849366in}}%
\pgfpathlineto{\pgfqpoint{2.602719in}{1.749505in}}%
\pgfpathlineto{\pgfqpoint{2.603565in}{2.274236in}}%
\pgfpathlineto{\pgfqpoint{2.605256in}{1.744946in}}%
\pgfpathlineto{\pgfqpoint{2.606102in}{1.955733in}}%
\pgfpathlineto{\pgfqpoint{2.607793in}{1.636840in}}%
\pgfpathlineto{\pgfqpoint{2.608639in}{1.849481in}}%
\pgfpathlineto{\pgfqpoint{2.609484in}{1.638791in}}%
\pgfpathlineto{\pgfqpoint{2.612021in}{2.272347in}}%
\pgfpathlineto{\pgfqpoint{2.612867in}{2.272530in}}%
\pgfpathlineto{\pgfqpoint{2.613712in}{1.638074in}}%
\pgfpathlineto{\pgfqpoint{2.614558in}{2.378075in}}%
\pgfpathlineto{\pgfqpoint{2.615404in}{1.955416in}}%
\pgfpathlineto{\pgfqpoint{2.617095in}{2.061149in}}%
\pgfpathlineto{\pgfqpoint{2.617940in}{1.955316in}}%
\pgfpathlineto{\pgfqpoint{2.618786in}{1.637694in}}%
\pgfpathlineto{\pgfqpoint{2.621323in}{2.589614in}}%
\pgfpathlineto{\pgfqpoint{2.623860in}{1.426941in}}%
\pgfpathlineto{\pgfqpoint{2.626397in}{2.695386in}}%
\pgfpathlineto{\pgfqpoint{2.628088in}{1.849537in}}%
\pgfpathlineto{\pgfqpoint{2.630625in}{2.287333in}}%
\pgfpathlineto{\pgfqpoint{2.631470in}{1.290151in}}%
\pgfpathlineto{\pgfqpoint{2.632316in}{1.763085in}}%
\pgfpathlineto{\pgfqpoint{2.634853in}{2.590162in}}%
\pgfpathlineto{\pgfqpoint{2.636544in}{1.255678in}}%
\pgfpathlineto{\pgfqpoint{2.639081in}{2.958178in}}%
\pgfpathlineto{\pgfqpoint{2.639927in}{1.579150in}}%
\pgfpathlineto{\pgfqpoint{2.640772in}{1.687260in}}%
\pgfpathlineto{\pgfqpoint{2.641618in}{2.644126in}}%
\pgfpathlineto{\pgfqpoint{2.642463in}{2.208218in}}%
\pgfpathlineto{\pgfqpoint{2.644155in}{1.215266in}}%
\pgfpathlineto{\pgfqpoint{2.645000in}{2.091250in}}%
\pgfpathlineto{\pgfqpoint{2.645846in}{2.037079in}}%
\pgfpathlineto{\pgfqpoint{2.646692in}{2.006605in}}%
\pgfpathlineto{\pgfqpoint{2.648383in}{2.464279in}}%
\pgfpathlineto{\pgfqpoint{2.650074in}{1.373370in}}%
\pgfpathlineto{\pgfqpoint{2.650920in}{2.323888in}}%
\pgfpathlineto{\pgfqpoint{2.651765in}{2.009735in}}%
\pgfpathlineto{\pgfqpoint{2.653457in}{1.585309in}}%
\pgfpathlineto{\pgfqpoint{2.654302in}{2.325568in}}%
\pgfpathlineto{\pgfqpoint{2.655148in}{2.114310in}}%
\pgfpathlineto{\pgfqpoint{2.655993in}{2.325159in}}%
\pgfpathlineto{\pgfqpoint{2.656839in}{1.691061in}}%
\pgfpathlineto{\pgfqpoint{2.657685in}{2.114081in}}%
\pgfpathlineto{\pgfqpoint{2.658530in}{2.325447in}}%
\pgfpathlineto{\pgfqpoint{2.661913in}{1.587294in}}%
\pgfpathlineto{\pgfqpoint{2.662758in}{2.007012in}}%
\pgfpathlineto{\pgfqpoint{2.663604in}{1.373373in}}%
\pgfpathlineto{\pgfqpoint{2.665295in}{2.642077in}}%
\pgfpathlineto{\pgfqpoint{2.667832in}{1.479534in}}%
\pgfpathlineto{\pgfqpoint{2.668678in}{1.585067in}}%
\pgfpathlineto{\pgfqpoint{2.671215in}{2.642909in}}%
\pgfpathlineto{\pgfqpoint{2.674597in}{1.162379in}}%
\pgfpathlineto{\pgfqpoint{2.675443in}{2.113977in}}%
\pgfpathlineto{\pgfqpoint{2.676288in}{2.009441in}}%
\pgfpathlineto{\pgfqpoint{2.677134in}{2.113767in}}%
\pgfpathlineto{\pgfqpoint{2.677980in}{1.162341in}}%
\pgfpathlineto{\pgfqpoint{2.679671in}{3.276880in}}%
\pgfpathlineto{\pgfqpoint{2.681362in}{1.585241in}}%
\pgfpathlineto{\pgfqpoint{2.682208in}{1.901507in}}%
\pgfpathlineto{\pgfqpoint{2.683053in}{1.056818in}}%
\pgfpathlineto{\pgfqpoint{2.683899in}{1.374597in}}%
\pgfpathlineto{\pgfqpoint{2.684745in}{2.642411in}}%
\pgfpathlineto{\pgfqpoint{2.685590in}{2.325288in}}%
\pgfpathlineto{\pgfqpoint{2.686436in}{2.219785in}}%
\pgfpathlineto{\pgfqpoint{2.687282in}{1.585431in}}%
\pgfpathlineto{\pgfqpoint{2.688127in}{2.113896in}}%
\pgfpathlineto{\pgfqpoint{2.688973in}{1.690907in}}%
\pgfpathlineto{\pgfqpoint{2.690664in}{2.008227in}}%
\pgfpathlineto{\pgfqpoint{2.691510in}{1.690944in}}%
\pgfpathlineto{\pgfqpoint{2.692355in}{2.430991in}}%
\pgfpathlineto{\pgfqpoint{2.693201in}{2.008096in}}%
\pgfpathlineto{\pgfqpoint{2.694047in}{1.902419in}}%
\pgfpathlineto{\pgfqpoint{2.694892in}{1.585195in}}%
\pgfpathlineto{\pgfqpoint{2.695738in}{2.219674in}}%
\pgfpathlineto{\pgfqpoint{2.696583in}{1.690915in}}%
\pgfpathlineto{\pgfqpoint{2.697429in}{2.959674in}}%
\pgfpathlineto{\pgfqpoint{2.699120in}{1.480414in}}%
\pgfpathlineto{\pgfqpoint{2.700812in}{2.536295in}}%
\pgfpathlineto{\pgfqpoint{2.702503in}{1.690806in}}%
\pgfpathlineto{\pgfqpoint{2.703348in}{2.536905in}}%
\pgfpathlineto{\pgfqpoint{2.704194in}{2.219526in}}%
\pgfpathlineto{\pgfqpoint{2.706731in}{1.585271in}}%
\pgfpathlineto{\pgfqpoint{2.707577in}{2.642539in}}%
\pgfpathlineto{\pgfqpoint{2.708422in}{2.324933in}}%
\pgfpathlineto{\pgfqpoint{2.710113in}{1.373725in}}%
\pgfpathlineto{\pgfqpoint{2.712650in}{2.007986in}}%
\pgfpathlineto{\pgfqpoint{2.713496in}{2.112819in}}%
\pgfpathlineto{\pgfqpoint{2.715187in}{2.642563in}}%
\pgfpathlineto{\pgfqpoint{2.716878in}{2.217166in}}%
\pgfpathlineto{\pgfqpoint{2.717724in}{2.748336in}}%
\pgfpathlineto{\pgfqpoint{2.719415in}{1.479489in}}%
\pgfpathlineto{\pgfqpoint{2.722798in}{2.429380in}}%
\pgfpathlineto{\pgfqpoint{2.723643in}{1.481111in}}%
\pgfpathlineto{\pgfqpoint{2.724489in}{1.690695in}}%
\pgfpathlineto{\pgfqpoint{2.725335in}{1.585552in}}%
\pgfpathlineto{\pgfqpoint{2.726180in}{2.219256in}}%
\pgfpathlineto{\pgfqpoint{2.727026in}{2.008661in}}%
\pgfpathlineto{\pgfqpoint{2.727871in}{1.903703in}}%
\pgfpathlineto{\pgfqpoint{2.729563in}{2.324776in}}%
\pgfpathlineto{\pgfqpoint{2.731254in}{1.821823in}}%
\pgfpathlineto{\pgfqpoint{2.732100in}{1.994020in}}%
\pgfpathlineto{\pgfqpoint{2.732945in}{1.796807in}}%
\pgfpathlineto{\pgfqpoint{2.733791in}{2.642050in}}%
\pgfpathlineto{\pgfqpoint{2.734636in}{2.113421in}}%
\pgfpathlineto{\pgfqpoint{2.735482in}{2.536714in}}%
\pgfpathlineto{\pgfqpoint{2.736328in}{1.585286in}}%
\pgfpathlineto{\pgfqpoint{2.737173in}{1.909374in}}%
\pgfpathlineto{\pgfqpoint{2.738019in}{2.215111in}}%
\pgfpathlineto{\pgfqpoint{2.738865in}{1.748839in}}%
\pgfpathlineto{\pgfqpoint{2.739710in}{2.693787in}}%
\pgfpathlineto{\pgfqpoint{2.741401in}{1.555301in}}%
\pgfpathlineto{\pgfqpoint{2.743093in}{2.061630in}}%
\pgfpathlineto{\pgfqpoint{2.743938in}{1.638548in}}%
\pgfpathlineto{\pgfqpoint{2.744784in}{2.269577in}}%
\pgfpathlineto{\pgfqpoint{2.745630in}{1.637335in}}%
\pgfpathlineto{\pgfqpoint{2.746475in}{1.850087in}}%
\pgfpathlineto{\pgfqpoint{2.747321in}{1.849673in}}%
\pgfpathlineto{\pgfqpoint{2.748166in}{2.586934in}}%
\pgfpathlineto{\pgfqpoint{2.749012in}{1.525185in}}%
\pgfpathlineto{\pgfqpoint{2.749858in}{1.849513in}}%
\pgfpathlineto{\pgfqpoint{2.750703in}{2.166759in}}%
\pgfpathlineto{\pgfqpoint{2.751549in}{1.492566in}}%
\pgfpathlineto{\pgfqpoint{2.752395in}{2.449976in}}%
\pgfpathlineto{\pgfqpoint{2.753240in}{2.284913in}}%
\pgfpathlineto{\pgfqpoint{2.754086in}{1.589418in}}%
\pgfpathlineto{\pgfqpoint{2.754931in}{1.932559in}}%
\pgfpathlineto{\pgfqpoint{2.755777in}{2.113130in}}%
\pgfpathlineto{\pgfqpoint{2.756623in}{2.747389in}}%
\pgfpathlineto{\pgfqpoint{2.757468in}{2.324620in}}%
\pgfpathlineto{\pgfqpoint{2.758314in}{2.536658in}}%
\pgfpathlineto{\pgfqpoint{2.760005in}{2.113812in}}%
\pgfpathlineto{\pgfqpoint{2.760851in}{2.323419in}}%
\pgfpathlineto{\pgfqpoint{2.761696in}{1.162345in}}%
\pgfpathlineto{\pgfqpoint{2.764233in}{2.853854in}}%
\pgfpathlineto{\pgfqpoint{2.765079in}{1.690959in}}%
\pgfpathlineto{\pgfqpoint{2.765925in}{1.902491in}}%
\pgfpathlineto{\pgfqpoint{2.766770in}{2.008338in}}%
\pgfpathlineto{\pgfqpoint{2.767616in}{1.691090in}}%
\pgfpathlineto{\pgfqpoint{2.768461in}{1.796757in}}%
\pgfpathlineto{\pgfqpoint{2.770998in}{2.113949in}}%
\pgfpathlineto{\pgfqpoint{2.771844in}{1.796400in}}%
\pgfpathlineto{\pgfqpoint{2.772690in}{2.111197in}}%
\pgfpathlineto{\pgfqpoint{2.773535in}{1.585197in}}%
\pgfpathlineto{\pgfqpoint{2.776072in}{2.536950in}}%
\pgfpathlineto{\pgfqpoint{2.776918in}{1.690993in}}%
\pgfpathlineto{\pgfqpoint{2.777763in}{2.325323in}}%
\pgfpathlineto{\pgfqpoint{2.778609in}{2.642267in}}%
\pgfpathlineto{\pgfqpoint{2.781146in}{1.373727in}}%
\pgfpathlineto{\pgfqpoint{2.783683in}{2.114052in}}%
\pgfpathlineto{\pgfqpoint{2.784528in}{2.219621in}}%
\pgfpathlineto{\pgfqpoint{2.785374in}{1.585242in}}%
\pgfpathlineto{\pgfqpoint{2.786220in}{1.691002in}}%
\pgfpathlineto{\pgfqpoint{2.787065in}{1.690858in}}%
\pgfpathlineto{\pgfqpoint{2.787911in}{1.585205in}}%
\pgfpathlineto{\pgfqpoint{2.788756in}{2.219143in}}%
\pgfpathlineto{\pgfqpoint{2.789602in}{1.902438in}}%
\pgfpathlineto{\pgfqpoint{2.790448in}{1.585209in}}%
\pgfpathlineto{\pgfqpoint{2.793830in}{2.536474in}}%
\pgfpathlineto{\pgfqpoint{2.794676in}{2.429669in}}%
\pgfpathlineto{\pgfqpoint{2.797213in}{1.692709in}}%
\pgfpathlineto{\pgfqpoint{2.798058in}{2.748357in}}%
\pgfpathlineto{\pgfqpoint{2.798904in}{1.585220in}}%
\pgfpathlineto{\pgfqpoint{2.799749in}{2.325470in}}%
\pgfpathlineto{\pgfqpoint{2.802286in}{1.902633in}}%
\pgfpathlineto{\pgfqpoint{2.804823in}{2.114585in}}%
\pgfpathlineto{\pgfqpoint{2.806514in}{1.056579in}}%
\pgfpathlineto{\pgfqpoint{2.807360in}{1.584923in}}%
\pgfpathlineto{\pgfqpoint{2.809897in}{2.431379in}}%
\pgfpathlineto{\pgfqpoint{2.810743in}{2.113711in}}%
\pgfpathlineto{\pgfqpoint{2.811588in}{2.114058in}}%
\pgfpathlineto{\pgfqpoint{2.812434in}{2.219570in}}%
\pgfpathlineto{\pgfqpoint{2.813279in}{2.854117in}}%
\pgfpathlineto{\pgfqpoint{2.814125in}{1.769838in}}%
\pgfpathlineto{\pgfqpoint{2.814971in}{1.796750in}}%
\pgfpathlineto{\pgfqpoint{2.815816in}{1.903368in}}%
\pgfpathlineto{\pgfqpoint{2.818353in}{2.431872in}}%
\pgfpathlineto{\pgfqpoint{2.819199in}{1.796923in}}%
\pgfpathlineto{\pgfqpoint{2.820044in}{2.007941in}}%
\pgfpathlineto{\pgfqpoint{2.820890in}{2.325543in}}%
\pgfpathlineto{\pgfqpoint{2.822581in}{1.479435in}}%
\pgfpathlineto{\pgfqpoint{2.823427in}{1.795948in}}%
\pgfpathlineto{\pgfqpoint{2.824273in}{1.268580in}}%
\pgfpathlineto{\pgfqpoint{2.825118in}{2.219632in}}%
\pgfpathlineto{\pgfqpoint{2.825964in}{1.691036in}}%
\pgfpathlineto{\pgfqpoint{2.826809in}{2.008529in}}%
\pgfpathlineto{\pgfqpoint{2.827655in}{1.689360in}}%
\pgfpathlineto{\pgfqpoint{2.828501in}{1.820949in}}%
\pgfpathlineto{\pgfqpoint{2.830192in}{2.736092in}}%
\pgfpathlineto{\pgfqpoint{2.831883in}{1.796283in}}%
\pgfpathlineto{\pgfqpoint{2.832729in}{2.191970in}}%
\pgfpathlineto{\pgfqpoint{2.833574in}{1.811980in}}%
\pgfpathlineto{\pgfqpoint{2.835266in}{2.810230in}}%
\pgfpathlineto{\pgfqpoint{2.837803in}{1.638062in}}%
\pgfpathlineto{\pgfqpoint{2.838648in}{1.638344in}}%
\pgfpathlineto{\pgfqpoint{2.841185in}{2.378161in}}%
\pgfpathlineto{\pgfqpoint{2.842031in}{1.849564in}}%
\pgfpathlineto{\pgfqpoint{2.842876in}{2.166713in}}%
\pgfpathlineto{\pgfqpoint{2.844568in}{1.215123in}}%
\pgfpathlineto{\pgfqpoint{2.845413in}{2.483641in}}%
\pgfpathlineto{\pgfqpoint{2.846259in}{1.849938in}}%
\pgfpathlineto{\pgfqpoint{2.847104in}{1.849545in}}%
\pgfpathlineto{\pgfqpoint{2.847950in}{1.743843in}}%
\pgfpathlineto{\pgfqpoint{2.848796in}{2.166815in}}%
\pgfpathlineto{\pgfqpoint{2.849641in}{2.060931in}}%
\pgfpathlineto{\pgfqpoint{2.850487in}{1.743896in}}%
\pgfpathlineto{\pgfqpoint{2.851333in}{2.061817in}}%
\pgfpathlineto{\pgfqpoint{2.852178in}{1.955461in}}%
\pgfpathlineto{\pgfqpoint{2.853024in}{1.213885in}}%
\pgfpathlineto{\pgfqpoint{2.853869in}{1.638149in}}%
\pgfpathlineto{\pgfqpoint{2.854715in}{2.378042in}}%
\pgfpathlineto{\pgfqpoint{2.855561in}{1.532267in}}%
\pgfpathlineto{\pgfqpoint{2.856406in}{2.166629in}}%
\pgfpathlineto{\pgfqpoint{2.857252in}{2.905906in}}%
\pgfpathlineto{\pgfqpoint{2.858098in}{1.427287in}}%
\pgfpathlineto{\pgfqpoint{2.858943in}{2.379699in}}%
\pgfpathlineto{\pgfqpoint{2.859789in}{2.166722in}}%
\pgfpathlineto{\pgfqpoint{2.860634in}{1.003554in}}%
\pgfpathlineto{\pgfqpoint{2.861480in}{2.167494in}}%
\pgfpathlineto{\pgfqpoint{2.862326in}{1.743971in}}%
\pgfpathlineto{\pgfqpoint{2.863171in}{1.211870in}}%
\pgfpathlineto{\pgfqpoint{2.864017in}{2.695187in}}%
\pgfpathlineto{\pgfqpoint{2.864863in}{1.848140in}}%
\pgfpathlineto{\pgfqpoint{2.865708in}{2.060838in}}%
\pgfpathlineto{\pgfqpoint{2.866554in}{1.639738in}}%
\pgfpathlineto{\pgfqpoint{2.867399in}{2.273109in}}%
\pgfpathlineto{\pgfqpoint{2.868245in}{1.743645in}}%
\pgfpathlineto{\pgfqpoint{2.869091in}{2.587843in}}%
\pgfpathlineto{\pgfqpoint{2.869936in}{1.638351in}}%
\pgfpathlineto{\pgfqpoint{2.870782in}{2.272638in}}%
\pgfpathlineto{\pgfqpoint{2.871628in}{2.693981in}}%
\pgfpathlineto{\pgfqpoint{2.874164in}{1.638348in}}%
\pgfpathlineto{\pgfqpoint{2.875010in}{1.527814in}}%
\pgfpathlineto{\pgfqpoint{2.875856in}{2.378443in}}%
\pgfpathlineto{\pgfqpoint{2.876701in}{2.061526in}}%
\pgfpathlineto{\pgfqpoint{2.877547in}{2.167528in}}%
\pgfpathlineto{\pgfqpoint{2.879238in}{1.955189in}}%
\pgfpathlineto{\pgfqpoint{2.880084in}{1.533524in}}%
\pgfpathlineto{\pgfqpoint{2.880929in}{2.173103in}}%
\pgfpathlineto{\pgfqpoint{2.881775in}{2.055928in}}%
\pgfpathlineto{\pgfqpoint{2.882621in}{1.847030in}}%
\pgfpathlineto{\pgfqpoint{2.883466in}{2.585584in}}%
\pgfpathlineto{\pgfqpoint{2.884312in}{2.242733in}}%
\pgfpathlineto{\pgfqpoint{2.886849in}{1.585485in}}%
\pgfpathlineto{\pgfqpoint{2.888540in}{2.431542in}}%
\pgfpathlineto{\pgfqpoint{2.889386in}{1.478788in}}%
\pgfpathlineto{\pgfqpoint{2.890231in}{2.009145in}}%
\pgfpathlineto{\pgfqpoint{2.891077in}{1.795596in}}%
\pgfpathlineto{\pgfqpoint{2.891922in}{2.540756in}}%
\pgfpathlineto{\pgfqpoint{2.892768in}{1.797456in}}%
\pgfpathlineto{\pgfqpoint{2.893614in}{1.797581in}}%
\pgfpathlineto{\pgfqpoint{2.894459in}{1.791318in}}%
\pgfpathlineto{\pgfqpoint{2.895305in}{1.692773in}}%
\pgfpathlineto{\pgfqpoint{2.896151in}{1.358046in}}%
\pgfpathlineto{\pgfqpoint{2.897842in}{2.334237in}}%
\pgfpathlineto{\pgfqpoint{2.899533in}{1.374689in}}%
\pgfpathlineto{\pgfqpoint{2.900379in}{2.005777in}}%
\pgfpathlineto{\pgfqpoint{2.901224in}{1.792040in}}%
\pgfpathlineto{\pgfqpoint{2.902916in}{2.336144in}}%
\pgfpathlineto{\pgfqpoint{2.903761in}{2.310723in}}%
\pgfpathlineto{\pgfqpoint{2.905452in}{1.547639in}}%
\pgfpathlineto{\pgfqpoint{2.907989in}{2.426321in}}%
\pgfpathlineto{\pgfqpoint{2.909681in}{1.574819in}}%
\pgfpathlineto{\pgfqpoint{2.910526in}{2.188131in}}%
\pgfpathlineto{\pgfqpoint{2.911372in}{1.888347in}}%
\pgfpathlineto{\pgfqpoint{2.912217in}{2.027973in}}%
\pgfpathlineto{\pgfqpoint{2.913909in}{1.320922in}}%
\pgfpathlineto{\pgfqpoint{2.916446in}{2.483407in}}%
\pgfpathlineto{\pgfqpoint{2.917291in}{1.532500in}}%
\pgfpathlineto{\pgfqpoint{2.918982in}{2.378172in}}%
\pgfpathlineto{\pgfqpoint{2.921519in}{0.792457in}}%
\pgfpathlineto{\pgfqpoint{2.922365in}{2.060830in}}%
\pgfpathlineto{\pgfqpoint{2.923211in}{1.955084in}}%
\pgfpathlineto{\pgfqpoint{2.924056in}{2.166486in}}%
\pgfpathlineto{\pgfqpoint{2.924902in}{1.744906in}}%
\pgfpathlineto{\pgfqpoint{2.925747in}{2.069078in}}%
\pgfpathlineto{\pgfqpoint{2.926593in}{1.744116in}}%
\pgfpathlineto{\pgfqpoint{2.928284in}{2.378236in}}%
\pgfpathlineto{\pgfqpoint{2.929130in}{1.427504in}}%
\pgfpathlineto{\pgfqpoint{2.929976in}{2.271233in}}%
\pgfpathlineto{\pgfqpoint{2.930821in}{1.743811in}}%
\pgfpathlineto{\pgfqpoint{2.932512in}{2.117754in}}%
\pgfpathlineto{\pgfqpoint{2.933358in}{1.638167in}}%
\pgfpathlineto{\pgfqpoint{2.934204in}{2.275285in}}%
\pgfpathlineto{\pgfqpoint{2.935049in}{2.133068in}}%
\pgfpathlineto{\pgfqpoint{2.935895in}{2.166715in}}%
\pgfpathlineto{\pgfqpoint{2.936741in}{1.694342in}}%
\pgfpathlineto{\pgfqpoint{2.937586in}{1.718639in}}%
\pgfpathlineto{\pgfqpoint{2.939277in}{2.292284in}}%
\pgfpathlineto{\pgfqpoint{2.941814in}{1.526053in}}%
\pgfpathlineto{\pgfqpoint{2.944351in}{2.272420in}}%
\pgfpathlineto{\pgfqpoint{2.946042in}{1.687055in}}%
\pgfpathlineto{\pgfqpoint{2.946888in}{2.685014in}}%
\pgfpathlineto{\pgfqpoint{2.947734in}{2.166677in}}%
\pgfpathlineto{\pgfqpoint{2.948579in}{2.073333in}}%
\pgfpathlineto{\pgfqpoint{2.949425in}{2.165183in}}%
\pgfpathlineto{\pgfqpoint{2.950271in}{1.849585in}}%
\pgfpathlineto{\pgfqpoint{2.951116in}{2.483356in}}%
\pgfpathlineto{\pgfqpoint{2.952807in}{1.215177in}}%
\pgfpathlineto{\pgfqpoint{2.954499in}{2.483450in}}%
\pgfpathlineto{\pgfqpoint{2.956190in}{2.061124in}}%
\pgfpathlineto{\pgfqpoint{2.957035in}{2.060492in}}%
\pgfpathlineto{\pgfqpoint{2.958727in}{2.695497in}}%
\pgfpathlineto{\pgfqpoint{2.959572in}{1.743826in}}%
\pgfpathlineto{\pgfqpoint{2.960418in}{2.695325in}}%
\pgfpathlineto{\pgfqpoint{2.962109in}{1.744136in}}%
\pgfpathlineto{\pgfqpoint{2.962955in}{2.375579in}}%
\pgfpathlineto{\pgfqpoint{2.963800in}{1.743879in}}%
\pgfpathlineto{\pgfqpoint{2.964646in}{1.959791in}}%
\pgfpathlineto{\pgfqpoint{2.965492in}{2.483673in}}%
\pgfpathlineto{\pgfqpoint{2.966337in}{1.321061in}}%
\pgfpathlineto{\pgfqpoint{2.967183in}{1.743642in}}%
\pgfpathlineto{\pgfqpoint{2.968874in}{2.483807in}}%
\pgfpathlineto{\pgfqpoint{2.970565in}{1.850110in}}%
\pgfpathlineto{\pgfqpoint{2.971411in}{2.056504in}}%
\pgfpathlineto{\pgfqpoint{2.972257in}{2.062980in}}%
\pgfpathlineto{\pgfqpoint{2.973102in}{1.975827in}}%
\pgfpathlineto{\pgfqpoint{2.973948in}{2.272715in}}%
\pgfpathlineto{\pgfqpoint{2.974794in}{2.061060in}}%
\pgfpathlineto{\pgfqpoint{2.975639in}{1.985260in}}%
\pgfpathlineto{\pgfqpoint{2.976485in}{1.743772in}}%
\pgfpathlineto{\pgfqpoint{2.978176in}{2.069041in}}%
\pgfpathlineto{\pgfqpoint{2.979022in}{1.743692in}}%
\pgfpathlineto{\pgfqpoint{2.979867in}{2.165050in}}%
\pgfpathlineto{\pgfqpoint{2.980713in}{1.742068in}}%
\pgfpathlineto{\pgfqpoint{2.981559in}{2.378147in}}%
\pgfpathlineto{\pgfqpoint{2.982404in}{1.953372in}}%
\pgfpathlineto{\pgfqpoint{2.984095in}{1.531673in}}%
\pgfpathlineto{\pgfqpoint{2.984941in}{2.306215in}}%
\pgfpathlineto{\pgfqpoint{2.985787in}{1.849935in}}%
\pgfpathlineto{\pgfqpoint{2.987478in}{2.803932in}}%
\pgfpathlineto{\pgfqpoint{2.988324in}{1.261706in}}%
\pgfpathlineto{\pgfqpoint{2.989169in}{1.827779in}}%
\pgfpathlineto{\pgfqpoint{2.990015in}{1.910275in}}%
\pgfpathlineto{\pgfqpoint{2.991706in}{2.744032in}}%
\pgfpathlineto{\pgfqpoint{2.992552in}{2.536863in}}%
\pgfpathlineto{\pgfqpoint{2.994243in}{1.798749in}}%
\pgfpathlineto{\pgfqpoint{2.996780in}{2.221841in}}%
\pgfpathlineto{\pgfqpoint{2.998471in}{1.269523in}}%
\pgfpathlineto{\pgfqpoint{2.999317in}{1.478704in}}%
\pgfpathlineto{\pgfqpoint{3.000162in}{2.112197in}}%
\pgfpathlineto{\pgfqpoint{3.001008in}{1.479747in}}%
\pgfpathlineto{\pgfqpoint{3.001854in}{2.641961in}}%
\pgfpathlineto{\pgfqpoint{3.002699in}{2.325531in}}%
\pgfpathlineto{\pgfqpoint{3.003545in}{2.536105in}}%
\pgfpathlineto{\pgfqpoint{3.006082in}{1.268782in}}%
\pgfpathlineto{\pgfqpoint{3.006927in}{2.536260in}}%
\pgfpathlineto{\pgfqpoint{3.007773in}{1.903126in}}%
\pgfpathlineto{\pgfqpoint{3.008619in}{1.902470in}}%
\pgfpathlineto{\pgfqpoint{3.009464in}{2.446097in}}%
\pgfpathlineto{\pgfqpoint{3.010310in}{1.697878in}}%
\pgfpathlineto{\pgfqpoint{3.012001in}{2.851913in}}%
\pgfpathlineto{\pgfqpoint{3.014538in}{1.684257in}}%
\pgfpathlineto{\pgfqpoint{3.015384in}{1.796907in}}%
\pgfpathlineto{\pgfqpoint{3.016229in}{2.009705in}}%
\pgfpathlineto{\pgfqpoint{3.017075in}{1.904654in}}%
\pgfpathlineto{\pgfqpoint{3.017920in}{1.903422in}}%
\pgfpathlineto{\pgfqpoint{3.020457in}{2.219624in}}%
\pgfpathlineto{\pgfqpoint{3.021303in}{1.373931in}}%
\pgfpathlineto{\pgfqpoint{3.022994in}{2.748725in}}%
\pgfpathlineto{\pgfqpoint{3.023840in}{1.902116in}}%
\pgfpathlineto{\pgfqpoint{3.024685in}{2.007999in}}%
\pgfpathlineto{\pgfqpoint{3.025531in}{1.902475in}}%
\pgfpathlineto{\pgfqpoint{3.026377in}{2.113713in}}%
\pgfpathlineto{\pgfqpoint{3.027222in}{1.374635in}}%
\pgfpathlineto{\pgfqpoint{3.028068in}{1.691251in}}%
\pgfpathlineto{\pgfqpoint{3.028914in}{3.065477in}}%
\pgfpathlineto{\pgfqpoint{3.030605in}{1.584635in}}%
\pgfpathlineto{\pgfqpoint{3.031450in}{2.537027in}}%
\pgfpathlineto{\pgfqpoint{3.032296in}{1.902608in}}%
\pgfpathlineto{\pgfqpoint{3.033142in}{1.585586in}}%
\pgfpathlineto{\pgfqpoint{3.033987in}{2.009768in}}%
\pgfpathlineto{\pgfqpoint{3.034833in}{1.267873in}}%
\pgfpathlineto{\pgfqpoint{3.035678in}{2.430928in}}%
\pgfpathlineto{\pgfqpoint{3.036524in}{1.902255in}}%
\pgfpathlineto{\pgfqpoint{3.038215in}{2.325330in}}%
\pgfpathlineto{\pgfqpoint{3.039061in}{1.795267in}}%
\pgfpathlineto{\pgfqpoint{3.039907in}{2.008075in}}%
\pgfpathlineto{\pgfqpoint{3.040752in}{1.902556in}}%
\pgfpathlineto{\pgfqpoint{3.043289in}{2.536894in}}%
\pgfpathlineto{\pgfqpoint{3.044135in}{2.008715in}}%
\pgfpathlineto{\pgfqpoint{3.044980in}{2.748758in}}%
\pgfpathlineto{\pgfqpoint{3.045826in}{1.902274in}}%
\pgfpathlineto{\pgfqpoint{3.046672in}{2.536613in}}%
\pgfpathlineto{\pgfqpoint{3.047517in}{2.113614in}}%
\pgfpathlineto{\pgfqpoint{3.048363in}{2.114172in}}%
\pgfpathlineto{\pgfqpoint{3.049208in}{2.112648in}}%
\pgfpathlineto{\pgfqpoint{3.050054in}{2.324247in}}%
\pgfpathlineto{\pgfqpoint{3.052591in}{1.479864in}}%
\pgfpathlineto{\pgfqpoint{3.055128in}{2.218461in}}%
\pgfpathlineto{\pgfqpoint{3.055973in}{2.114940in}}%
\pgfpathlineto{\pgfqpoint{3.057665in}{2.111843in}}%
\pgfpathlineto{\pgfqpoint{3.059356in}{1.267646in}}%
\pgfpathlineto{\pgfqpoint{3.061893in}{2.432326in}}%
\pgfpathlineto{\pgfqpoint{3.063584in}{2.111271in}}%
\pgfpathlineto{\pgfqpoint{3.064430in}{2.429501in}}%
\pgfpathlineto{\pgfqpoint{3.065275in}{1.373807in}}%
\pgfpathlineto{\pgfqpoint{3.066121in}{1.691096in}}%
\pgfpathlineto{\pgfqpoint{3.069503in}{2.854026in}}%
\pgfpathlineto{\pgfqpoint{3.070349in}{2.008062in}}%
\pgfpathlineto{\pgfqpoint{3.071195in}{2.325243in}}%
\pgfpathlineto{\pgfqpoint{3.072886in}{1.796833in}}%
\pgfpathlineto{\pgfqpoint{3.073732in}{2.219645in}}%
\pgfpathlineto{\pgfqpoint{3.074577in}{2.008172in}}%
\pgfpathlineto{\pgfqpoint{3.075423in}{2.113879in}}%
\pgfpathlineto{\pgfqpoint{3.076268in}{2.426766in}}%
\pgfpathlineto{\pgfqpoint{3.077114in}{2.219607in}}%
\pgfpathlineto{\pgfqpoint{3.077960in}{1.690943in}}%
\pgfpathlineto{\pgfqpoint{3.078805in}{2.112418in}}%
\pgfpathlineto{\pgfqpoint{3.079651in}{2.113697in}}%
\pgfpathlineto{\pgfqpoint{3.080497in}{1.586896in}}%
\pgfpathlineto{\pgfqpoint{3.081342in}{1.902456in}}%
\pgfpathlineto{\pgfqpoint{3.083033in}{2.113866in}}%
\pgfpathlineto{\pgfqpoint{3.083879in}{2.431026in}}%
\pgfpathlineto{\pgfqpoint{3.084725in}{1.269907in}}%
\pgfpathlineto{\pgfqpoint{3.085570in}{1.905967in}}%
\pgfpathlineto{\pgfqpoint{3.086416in}{1.902372in}}%
\pgfpathlineto{\pgfqpoint{3.087262in}{2.320916in}}%
\pgfpathlineto{\pgfqpoint{3.089798in}{1.586772in}}%
\pgfpathlineto{\pgfqpoint{3.090644in}{2.219315in}}%
\pgfpathlineto{\pgfqpoint{3.091490in}{1.796088in}}%
\pgfpathlineto{\pgfqpoint{3.092335in}{1.603996in}}%
\pgfpathlineto{\pgfqpoint{3.093181in}{2.430667in}}%
\pgfpathlineto{\pgfqpoint{3.094027in}{1.801846in}}%
\pgfpathlineto{\pgfqpoint{3.094872in}{1.691221in}}%
\pgfpathlineto{\pgfqpoint{3.097409in}{2.536717in}}%
\pgfpathlineto{\pgfqpoint{3.098255in}{1.373868in}}%
\pgfpathlineto{\pgfqpoint{3.099100in}{1.796724in}}%
\pgfpathlineto{\pgfqpoint{3.100792in}{2.325588in}}%
\pgfpathlineto{\pgfqpoint{3.101637in}{1.373760in}}%
\pgfpathlineto{\pgfqpoint{3.102483in}{2.113670in}}%
\pgfpathlineto{\pgfqpoint{3.105020in}{1.796915in}}%
\pgfpathlineto{\pgfqpoint{3.106711in}{2.431655in}}%
\pgfpathlineto{\pgfqpoint{3.108402in}{1.691157in}}%
\pgfpathlineto{\pgfqpoint{3.110093in}{2.219814in}}%
\pgfpathlineto{\pgfqpoint{3.111785in}{2.214052in}}%
\pgfpathlineto{\pgfqpoint{3.112630in}{2.325497in}}%
\pgfpathlineto{\pgfqpoint{3.113476in}{1.902552in}}%
\pgfpathlineto{\pgfqpoint{3.114322in}{2.113780in}}%
\pgfpathlineto{\pgfqpoint{3.115167in}{2.007146in}}%
\pgfpathlineto{\pgfqpoint{3.116013in}{2.325238in}}%
\pgfpathlineto{\pgfqpoint{3.119395in}{1.269464in}}%
\pgfpathlineto{\pgfqpoint{3.120241in}{2.427298in}}%
\pgfpathlineto{\pgfqpoint{3.121086in}{2.425005in}}%
\pgfpathlineto{\pgfqpoint{3.121932in}{1.744412in}}%
\pgfpathlineto{\pgfqpoint{3.122778in}{2.390709in}}%
\pgfpathlineto{\pgfqpoint{3.123623in}{1.538252in}}%
\pgfpathlineto{\pgfqpoint{3.124469in}{2.484093in}}%
\pgfpathlineto{\pgfqpoint{3.125315in}{1.532345in}}%
\pgfpathlineto{\pgfqpoint{3.126160in}{2.801060in}}%
\pgfpathlineto{\pgfqpoint{3.127006in}{1.955277in}}%
\pgfpathlineto{\pgfqpoint{3.127851in}{1.743837in}}%
\pgfpathlineto{\pgfqpoint{3.130388in}{2.272555in}}%
\pgfpathlineto{\pgfqpoint{3.131234in}{2.378080in}}%
\pgfpathlineto{\pgfqpoint{3.132080in}{1.743865in}}%
\pgfpathlineto{\pgfqpoint{3.132925in}{1.955332in}}%
\pgfpathlineto{\pgfqpoint{3.133771in}{1.955226in}}%
\pgfpathlineto{\pgfqpoint{3.135462in}{1.443392in}}%
\pgfpathlineto{\pgfqpoint{3.136308in}{2.106443in}}%
\pgfpathlineto{\pgfqpoint{3.137153in}{1.788962in}}%
\pgfpathlineto{\pgfqpoint{3.137999in}{2.182922in}}%
\pgfpathlineto{\pgfqpoint{3.138845in}{1.955272in}}%
\pgfpathlineto{\pgfqpoint{3.140536in}{1.638116in}}%
\pgfpathlineto{\pgfqpoint{3.141381in}{2.378813in}}%
\pgfpathlineto{\pgfqpoint{3.142227in}{1.849016in}}%
\pgfpathlineto{\pgfqpoint{3.144764in}{2.272492in}}%
\pgfpathlineto{\pgfqpoint{3.145610in}{1.954963in}}%
\pgfpathlineto{\pgfqpoint{3.147301in}{2.378234in}}%
\pgfpathlineto{\pgfqpoint{3.148146in}{1.427638in}}%
\pgfpathlineto{\pgfqpoint{3.150683in}{2.589863in}}%
\pgfpathlineto{\pgfqpoint{3.153220in}{1.443748in}}%
\pgfpathlineto{\pgfqpoint{3.154911in}{2.272861in}}%
\pgfpathlineto{\pgfqpoint{3.155757in}{2.378247in}}%
\pgfpathlineto{\pgfqpoint{3.156603in}{1.426696in}}%
\pgfpathlineto{\pgfqpoint{3.157448in}{2.484093in}}%
\pgfpathlineto{\pgfqpoint{3.158294in}{1.955305in}}%
\pgfpathlineto{\pgfqpoint{3.159140in}{1.638049in}}%
\pgfpathlineto{\pgfqpoint{3.160831in}{2.483920in}}%
\pgfpathlineto{\pgfqpoint{3.161676in}{2.377536in}}%
\pgfpathlineto{\pgfqpoint{3.162522in}{2.272577in}}%
\pgfpathlineto{\pgfqpoint{3.163368in}{2.483377in}}%
\pgfpathlineto{\pgfqpoint{3.164213in}{1.955243in}}%
\pgfpathlineto{\pgfqpoint{3.165059in}{1.955702in}}%
\pgfpathlineto{\pgfqpoint{3.165905in}{2.244247in}}%
\pgfpathlineto{\pgfqpoint{3.166750in}{2.120354in}}%
\pgfpathlineto{\pgfqpoint{3.168441in}{1.320865in}}%
\pgfpathlineto{\pgfqpoint{3.170133in}{2.166820in}}%
\pgfpathlineto{\pgfqpoint{3.171824in}{1.533183in}}%
\pgfpathlineto{\pgfqpoint{3.172670in}{2.483889in}}%
\pgfpathlineto{\pgfqpoint{3.173515in}{1.638313in}}%
\pgfpathlineto{\pgfqpoint{3.174361in}{1.956640in}}%
\pgfpathlineto{\pgfqpoint{3.175206in}{1.639339in}}%
\pgfpathlineto{\pgfqpoint{3.176052in}{2.589344in}}%
\pgfpathlineto{\pgfqpoint{3.176898in}{2.272327in}}%
\pgfpathlineto{\pgfqpoint{3.179435in}{1.638019in}}%
\pgfpathlineto{\pgfqpoint{3.180280in}{1.531864in}}%
\pgfpathlineto{\pgfqpoint{3.181971in}{1.956002in}}%
\pgfpathlineto{\pgfqpoint{3.182817in}{1.926664in}}%
\pgfpathlineto{\pgfqpoint{3.186200in}{2.377529in}}%
\pgfpathlineto{\pgfqpoint{3.188736in}{1.533506in}}%
\pgfpathlineto{\pgfqpoint{3.190428in}{2.591030in}}%
\pgfpathlineto{\pgfqpoint{3.191273in}{1.955625in}}%
\pgfpathlineto{\pgfqpoint{3.192119in}{2.166801in}}%
\pgfpathlineto{\pgfqpoint{3.192965in}{2.162865in}}%
\pgfpathlineto{\pgfqpoint{3.193810in}{1.532611in}}%
\pgfpathlineto{\pgfqpoint{3.194656in}{1.849448in}}%
\pgfpathlineto{\pgfqpoint{3.195501in}{1.743921in}}%
\pgfpathlineto{\pgfqpoint{3.199729in}{2.589564in}}%
\pgfpathlineto{\pgfqpoint{3.201421in}{1.743960in}}%
\pgfpathlineto{\pgfqpoint{3.202266in}{2.377975in}}%
\pgfpathlineto{\pgfqpoint{3.203112in}{2.060918in}}%
\pgfpathlineto{\pgfqpoint{3.203958in}{1.638368in}}%
\pgfpathlineto{\pgfqpoint{3.204803in}{1.852270in}}%
\pgfpathlineto{\pgfqpoint{3.205649in}{2.061106in}}%
\pgfpathlineto{\pgfqpoint{3.206494in}{2.695147in}}%
\pgfpathlineto{\pgfqpoint{3.207340in}{2.374932in}}%
\pgfpathlineto{\pgfqpoint{3.209031in}{1.743783in}}%
\pgfpathlineto{\pgfqpoint{3.209877in}{2.484083in}}%
\pgfpathlineto{\pgfqpoint{3.211568in}{1.318076in}}%
\pgfpathlineto{\pgfqpoint{3.213259in}{1.849716in}}%
\pgfpathlineto{\pgfqpoint{3.214105in}{1.744113in}}%
\pgfpathlineto{\pgfqpoint{3.214951in}{2.348189in}}%
\pgfpathlineto{\pgfqpoint{3.215796in}{2.109260in}}%
\pgfpathlineto{\pgfqpoint{3.216642in}{1.781714in}}%
\pgfpathlineto{\pgfqpoint{3.217488in}{2.234925in}}%
\pgfpathlineto{\pgfqpoint{3.218333in}{2.089260in}}%
\pgfpathlineto{\pgfqpoint{3.219179in}{1.818163in}}%
\pgfpathlineto{\pgfqpoint{3.220870in}{2.646440in}}%
\pgfpathlineto{\pgfqpoint{3.223407in}{1.586222in}}%
\pgfpathlineto{\pgfqpoint{3.225098in}{2.220236in}}%
\pgfpathlineto{\pgfqpoint{3.225944in}{1.691046in}}%
\pgfpathlineto{\pgfqpoint{3.226789in}{2.539644in}}%
\pgfpathlineto{\pgfqpoint{3.229326in}{1.268342in}}%
\pgfpathlineto{\pgfqpoint{3.230172in}{1.902457in}}%
\pgfpathlineto{\pgfqpoint{3.231018in}{1.484025in}}%
\pgfpathlineto{\pgfqpoint{3.233554in}{2.446591in}}%
\pgfpathlineto{\pgfqpoint{3.234400in}{2.431293in}}%
\pgfpathlineto{\pgfqpoint{3.235246in}{2.008197in}}%
\pgfpathlineto{\pgfqpoint{3.236091in}{2.642546in}}%
\pgfpathlineto{\pgfqpoint{3.236937in}{1.162652in}}%
\pgfpathlineto{\pgfqpoint{3.237783in}{1.902449in}}%
\pgfpathlineto{\pgfqpoint{3.238628in}{2.642162in}}%
\pgfpathlineto{\pgfqpoint{3.239474in}{2.325243in}}%
\pgfpathlineto{\pgfqpoint{3.242011in}{1.691026in}}%
\pgfpathlineto{\pgfqpoint{3.242856in}{2.219491in}}%
\pgfpathlineto{\pgfqpoint{3.243702in}{2.113863in}}%
\pgfpathlineto{\pgfqpoint{3.244548in}{1.585217in}}%
\pgfpathlineto{\pgfqpoint{3.246239in}{2.329378in}}%
\pgfpathlineto{\pgfqpoint{3.247084in}{2.007833in}}%
\pgfpathlineto{\pgfqpoint{3.247930in}{2.431410in}}%
\pgfpathlineto{\pgfqpoint{3.248776in}{1.585195in}}%
\pgfpathlineto{\pgfqpoint{3.249621in}{2.113899in}}%
\pgfpathlineto{\pgfqpoint{3.251313in}{1.373677in}}%
\pgfpathlineto{\pgfqpoint{3.252158in}{2.431073in}}%
\pgfpathlineto{\pgfqpoint{3.253004in}{1.902501in}}%
\pgfpathlineto{\pgfqpoint{3.253849in}{1.691009in}}%
\pgfpathlineto{\pgfqpoint{3.254695in}{1.902343in}}%
\pgfpathlineto{\pgfqpoint{3.255541in}{1.585262in}}%
\pgfpathlineto{\pgfqpoint{3.257232in}{2.430969in}}%
\pgfpathlineto{\pgfqpoint{3.258923in}{2.219624in}}%
\pgfpathlineto{\pgfqpoint{3.261460in}{1.691032in}}%
\pgfpathlineto{\pgfqpoint{3.262306in}{1.902507in}}%
\pgfpathlineto{\pgfqpoint{3.263151in}{2.536948in}}%
\pgfpathlineto{\pgfqpoint{3.265688in}{1.162378in}}%
\pgfpathlineto{\pgfqpoint{3.266534in}{2.555913in}}%
\pgfpathlineto{\pgfqpoint{3.267379in}{1.796699in}}%
\pgfpathlineto{\pgfqpoint{3.268225in}{1.691032in}}%
\pgfpathlineto{\pgfqpoint{3.269071in}{2.113879in}}%
\pgfpathlineto{\pgfqpoint{3.269916in}{2.008379in}}%
\pgfpathlineto{\pgfqpoint{3.270762in}{1.736945in}}%
\pgfpathlineto{\pgfqpoint{3.271608in}{2.298645in}}%
\pgfpathlineto{\pgfqpoint{3.272453in}{2.166061in}}%
\pgfpathlineto{\pgfqpoint{3.274144in}{1.858547in}}%
\pgfpathlineto{\pgfqpoint{3.276681in}{2.589610in}}%
\pgfpathlineto{\pgfqpoint{3.279218in}{1.743747in}}%
\pgfpathlineto{\pgfqpoint{3.280064in}{1.849451in}}%
\pgfpathlineto{\pgfqpoint{3.280909in}{1.743957in}}%
\pgfpathlineto{\pgfqpoint{3.283446in}{3.012611in}}%
\pgfpathlineto{\pgfqpoint{3.284292in}{1.638103in}}%
\pgfpathlineto{\pgfqpoint{3.285137in}{1.849538in}}%
\pgfpathlineto{\pgfqpoint{3.286829in}{2.378393in}}%
\pgfpathlineto{\pgfqpoint{3.287674in}{1.955255in}}%
\pgfpathlineto{\pgfqpoint{3.288520in}{2.272428in}}%
\pgfpathlineto{\pgfqpoint{3.289366in}{2.061043in}}%
\pgfpathlineto{\pgfqpoint{3.290211in}{2.483966in}}%
\pgfpathlineto{\pgfqpoint{3.293594in}{0.897913in}}%
\pgfpathlineto{\pgfqpoint{3.296131in}{2.272489in}}%
\pgfpathlineto{\pgfqpoint{3.296976in}{2.589894in}}%
\pgfpathlineto{\pgfqpoint{3.299513in}{1.426594in}}%
\pgfpathlineto{\pgfqpoint{3.300359in}{2.589862in}}%
\pgfpathlineto{\pgfqpoint{3.301204in}{1.849514in}}%
\pgfpathlineto{\pgfqpoint{3.302050in}{2.272424in}}%
\pgfpathlineto{\pgfqpoint{3.302896in}{2.166832in}}%
\pgfpathlineto{\pgfqpoint{3.303741in}{2.060961in}}%
\pgfpathlineto{\pgfqpoint{3.305432in}{1.740818in}}%
\pgfpathlineto{\pgfqpoint{3.306278in}{1.743813in}}%
\pgfpathlineto{\pgfqpoint{3.307124in}{2.589622in}}%
\pgfpathlineto{\pgfqpoint{3.309661in}{1.215227in}}%
\pgfpathlineto{\pgfqpoint{3.310506in}{2.801111in}}%
\pgfpathlineto{\pgfqpoint{3.311352in}{1.532351in}}%
\pgfpathlineto{\pgfqpoint{3.312197in}{2.166682in}}%
\pgfpathlineto{\pgfqpoint{3.313043in}{1.426610in}}%
\pgfpathlineto{\pgfqpoint{3.313889in}{1.638279in}}%
\pgfpathlineto{\pgfqpoint{3.314734in}{1.638020in}}%
\pgfpathlineto{\pgfqpoint{3.315580in}{1.743965in}}%
\pgfpathlineto{\pgfqpoint{3.316426in}{2.483829in}}%
\pgfpathlineto{\pgfqpoint{3.317271in}{2.272515in}}%
\pgfpathlineto{\pgfqpoint{3.318962in}{2.060964in}}%
\pgfpathlineto{\pgfqpoint{3.321499in}{2.377582in}}%
\pgfpathlineto{\pgfqpoint{3.322345in}{1.744036in}}%
\pgfpathlineto{\pgfqpoint{3.323191in}{1.849529in}}%
\pgfpathlineto{\pgfqpoint{3.324036in}{1.849400in}}%
\pgfpathlineto{\pgfqpoint{3.324882in}{2.166959in}}%
\pgfpathlineto{\pgfqpoint{3.326573in}{1.638093in}}%
\pgfpathlineto{\pgfqpoint{3.328264in}{1.955403in}}%
\pgfpathlineto{\pgfqpoint{3.329110in}{1.427309in}}%
\pgfpathlineto{\pgfqpoint{3.329956in}{1.955305in}}%
\pgfpathlineto{\pgfqpoint{3.330801in}{1.637968in}}%
\pgfpathlineto{\pgfqpoint{3.331647in}{1.959614in}}%
\pgfpathlineto{\pgfqpoint{3.332492in}{1.952285in}}%
\pgfpathlineto{\pgfqpoint{3.333338in}{1.849670in}}%
\pgfpathlineto{\pgfqpoint{3.334184in}{2.272580in}}%
\pgfpathlineto{\pgfqpoint{3.335029in}{1.743139in}}%
\pgfpathlineto{\pgfqpoint{3.335875in}{1.954877in}}%
\pgfpathlineto{\pgfqpoint{3.336721in}{2.378169in}}%
\pgfpathlineto{\pgfqpoint{3.337566in}{1.215137in}}%
\pgfpathlineto{\pgfqpoint{3.338412in}{1.532228in}}%
\pgfpathlineto{\pgfqpoint{3.339257in}{2.483709in}}%
\pgfpathlineto{\pgfqpoint{3.340103in}{2.166725in}}%
\pgfpathlineto{\pgfqpoint{3.340949in}{1.320938in}}%
\pgfpathlineto{\pgfqpoint{3.341794in}{1.743264in}}%
\pgfpathlineto{\pgfqpoint{3.344331in}{2.589710in}}%
\pgfpathlineto{\pgfqpoint{3.345177in}{2.165285in}}%
\pgfpathlineto{\pgfqpoint{3.346022in}{2.166528in}}%
\pgfpathlineto{\pgfqpoint{3.346868in}{2.483430in}}%
\pgfpathlineto{\pgfqpoint{3.349405in}{1.644152in}}%
\pgfpathlineto{\pgfqpoint{3.350251in}{1.630283in}}%
\pgfpathlineto{\pgfqpoint{3.351096in}{2.068148in}}%
\pgfpathlineto{\pgfqpoint{3.351942in}{1.914102in}}%
\pgfpathlineto{\pgfqpoint{3.352787in}{1.763938in}}%
\pgfpathlineto{\pgfqpoint{3.353633in}{2.341987in}}%
\pgfpathlineto{\pgfqpoint{3.354479in}{2.027301in}}%
\pgfpathlineto{\pgfqpoint{3.355324in}{1.902425in}}%
\pgfpathlineto{\pgfqpoint{3.356170in}{2.008208in}}%
\pgfpathlineto{\pgfqpoint{3.357015in}{1.690852in}}%
\pgfpathlineto{\pgfqpoint{3.358707in}{2.325431in}}%
\pgfpathlineto{\pgfqpoint{3.359552in}{1.691072in}}%
\pgfpathlineto{\pgfqpoint{3.360398in}{1.796517in}}%
\pgfpathlineto{\pgfqpoint{3.361244in}{2.008295in}}%
\pgfpathlineto{\pgfqpoint{3.362089in}{1.902728in}}%
\pgfpathlineto{\pgfqpoint{3.362935in}{1.479309in}}%
\pgfpathlineto{\pgfqpoint{3.363780in}{2.748304in}}%
\pgfpathlineto{\pgfqpoint{3.364626in}{2.218095in}}%
\pgfpathlineto{\pgfqpoint{3.365472in}{2.642603in}}%
\pgfpathlineto{\pgfqpoint{3.366317in}{2.107633in}}%
\pgfpathlineto{\pgfqpoint{3.367163in}{2.430156in}}%
\pgfpathlineto{\pgfqpoint{3.368009in}{2.324721in}}%
\pgfpathlineto{\pgfqpoint{3.368854in}{0.637273in}}%
\pgfpathlineto{\pgfqpoint{3.369700in}{1.905236in}}%
\pgfpathlineto{\pgfqpoint{3.370545in}{1.688180in}}%
\pgfpathlineto{\pgfqpoint{3.371391in}{1.793006in}}%
\pgfpathlineto{\pgfqpoint{3.372237in}{1.804997in}}%
\pgfpathlineto{\pgfqpoint{3.373928in}{2.221185in}}%
\pgfpathlineto{\pgfqpoint{3.376465in}{1.201277in}}%
\pgfpathlineto{\pgfqpoint{3.377310in}{2.207163in}}%
\pgfpathlineto{\pgfqpoint{3.378156in}{2.198239in}}%
\pgfpathlineto{\pgfqpoint{3.379002in}{1.896299in}}%
\pgfpathlineto{\pgfqpoint{3.379847in}{2.217621in}}%
\pgfpathlineto{\pgfqpoint{3.380693in}{1.582914in}}%
\pgfpathlineto{\pgfqpoint{3.381539in}{2.639831in}}%
\pgfpathlineto{\pgfqpoint{3.382384in}{2.431678in}}%
\pgfpathlineto{\pgfqpoint{3.383230in}{2.218235in}}%
\pgfpathlineto{\pgfqpoint{3.384075in}{1.553256in}}%
\pgfpathlineto{\pgfqpoint{3.384921in}{1.793127in}}%
\pgfpathlineto{\pgfqpoint{3.385767in}{1.481014in}}%
\pgfpathlineto{\pgfqpoint{3.387458in}{2.216294in}}%
\pgfpathlineto{\pgfqpoint{3.388304in}{2.204057in}}%
\pgfpathlineto{\pgfqpoint{3.389149in}{2.219390in}}%
\pgfpathlineto{\pgfqpoint{3.391686in}{1.243616in}}%
\pgfpathlineto{\pgfqpoint{3.392532in}{1.555802in}}%
\pgfpathlineto{\pgfqpoint{3.393377in}{1.529254in}}%
\pgfpathlineto{\pgfqpoint{3.396760in}{2.352477in}}%
\pgfpathlineto{\pgfqpoint{3.397605in}{1.584735in}}%
\pgfpathlineto{\pgfqpoint{3.398451in}{2.110503in}}%
\pgfpathlineto{\pgfqpoint{3.399297in}{1.904321in}}%
\pgfpathlineto{\pgfqpoint{3.400142in}{1.989236in}}%
\pgfpathlineto{\pgfqpoint{3.400988in}{2.430389in}}%
\pgfpathlineto{\pgfqpoint{3.403525in}{1.771971in}}%
\pgfpathlineto{\pgfqpoint{3.404370in}{1.891594in}}%
\pgfpathlineto{\pgfqpoint{3.405216in}{1.884230in}}%
\pgfpathlineto{\pgfqpoint{3.406062in}{2.482786in}}%
\pgfpathlineto{\pgfqpoint{3.406907in}{1.231535in}}%
\pgfpathlineto{\pgfqpoint{3.407753in}{1.732589in}}%
\pgfpathlineto{\pgfqpoint{3.408599in}{1.214938in}}%
\pgfpathlineto{\pgfqpoint{3.409444in}{2.058366in}}%
\pgfpathlineto{\pgfqpoint{3.410290in}{1.530689in}}%
\pgfpathlineto{\pgfqpoint{3.411135in}{1.847107in}}%
\pgfpathlineto{\pgfqpoint{3.411981in}{1.528339in}}%
\pgfpathlineto{\pgfqpoint{3.412827in}{1.983792in}}%
\pgfpathlineto{\pgfqpoint{3.413672in}{1.725554in}}%
\pgfpathlineto{\pgfqpoint{3.415364in}{2.190328in}}%
\pgfpathlineto{\pgfqpoint{3.417055in}{1.580722in}}%
\pgfpathlineto{\pgfqpoint{3.417900in}{2.586806in}}%
\pgfpathlineto{\pgfqpoint{3.418746in}{1.978557in}}%
\pgfpathlineto{\pgfqpoint{3.420437in}{2.377835in}}%
\pgfpathlineto{\pgfqpoint{3.422129in}{1.846830in}}%
\pgfpathlineto{\pgfqpoint{3.422974in}{2.168807in}}%
\pgfpathlineto{\pgfqpoint{3.423820in}{1.922555in}}%
\pgfpathlineto{\pgfqpoint{3.424665in}{2.065042in}}%
\pgfpathlineto{\pgfqpoint{3.425511in}{1.463658in}}%
\pgfpathlineto{\pgfqpoint{3.426357in}{1.470813in}}%
\pgfpathlineto{\pgfqpoint{3.428048in}{2.695261in}}%
\pgfpathlineto{\pgfqpoint{3.428894in}{2.144893in}}%
\pgfpathlineto{\pgfqpoint{3.429739in}{1.506272in}}%
\pgfpathlineto{\pgfqpoint{3.430585in}{2.316019in}}%
\pgfpathlineto{\pgfqpoint{3.431430in}{1.861911in}}%
\pgfpathlineto{\pgfqpoint{3.432276in}{1.791834in}}%
\pgfpathlineto{\pgfqpoint{3.433967in}{2.534585in}}%
\pgfpathlineto{\pgfqpoint{3.436504in}{1.364634in}}%
\pgfpathlineto{\pgfqpoint{3.437350in}{2.342337in}}%
\pgfpathlineto{\pgfqpoint{3.438195in}{1.375282in}}%
\pgfpathlineto{\pgfqpoint{3.439041in}{2.300288in}}%
\pgfpathlineto{\pgfqpoint{3.439887in}{2.099652in}}%
\pgfpathlineto{\pgfqpoint{3.440732in}{2.174261in}}%
\pgfpathlineto{\pgfqpoint{3.441578in}{1.114574in}}%
\pgfpathlineto{\pgfqpoint{3.442423in}{1.633642in}}%
\pgfpathlineto{\pgfqpoint{3.444115in}{1.951829in}}%
\pgfpathlineto{\pgfqpoint{3.444960in}{1.630915in}}%
\pgfpathlineto{\pgfqpoint{3.445806in}{1.636905in}}%
\pgfpathlineto{\pgfqpoint{3.446652in}{1.953223in}}%
\pgfpathlineto{\pgfqpoint{3.447497in}{3.224801in}}%
\pgfpathlineto{\pgfqpoint{3.449188in}{1.487025in}}%
\pgfpathlineto{\pgfqpoint{3.451725in}{2.326850in}}%
\pgfpathlineto{\pgfqpoint{3.452571in}{1.978497in}}%
\pgfpathlineto{\pgfqpoint{3.453417in}{2.314179in}}%
\pgfpathlineto{\pgfqpoint{3.455108in}{1.760013in}}%
\pgfpathlineto{\pgfqpoint{3.456799in}{2.353481in}}%
\pgfpathlineto{\pgfqpoint{3.458490in}{1.809607in}}%
\pgfpathlineto{\pgfqpoint{3.459336in}{2.210006in}}%
\pgfpathlineto{\pgfqpoint{3.460182in}{1.795095in}}%
\pgfpathlineto{\pgfqpoint{3.461027in}{2.003999in}}%
\pgfpathlineto{\pgfqpoint{3.461873in}{2.228521in}}%
\pgfpathlineto{\pgfqpoint{3.462718in}{2.212755in}}%
\pgfpathlineto{\pgfqpoint{3.464410in}{1.900200in}}%
\pgfpathlineto{\pgfqpoint{3.465255in}{2.856418in}}%
\pgfpathlineto{\pgfqpoint{3.466101in}{1.915224in}}%
\pgfpathlineto{\pgfqpoint{3.466947in}{2.216207in}}%
\pgfpathlineto{\pgfqpoint{3.467792in}{2.009618in}}%
\pgfpathlineto{\pgfqpoint{3.468638in}{2.321790in}}%
\pgfpathlineto{\pgfqpoint{3.470329in}{1.697155in}}%
\pgfpathlineto{\pgfqpoint{3.471175in}{2.428266in}}%
\pgfpathlineto{\pgfqpoint{3.472020in}{2.325814in}}%
\pgfpathlineto{\pgfqpoint{3.472866in}{1.997693in}}%
\pgfpathlineto{\pgfqpoint{3.473712in}{2.214586in}}%
\pgfpathlineto{\pgfqpoint{3.474557in}{2.533337in}}%
\pgfpathlineto{\pgfqpoint{3.475403in}{2.317363in}}%
\pgfpathlineto{\pgfqpoint{3.476248in}{1.901876in}}%
\pgfpathlineto{\pgfqpoint{3.477940in}{2.433101in}}%
\pgfpathlineto{\pgfqpoint{3.478785in}{1.480244in}}%
\pgfpathlineto{\pgfqpoint{3.479631in}{2.185898in}}%
\pgfpathlineto{\pgfqpoint{3.480477in}{2.642140in}}%
\pgfpathlineto{\pgfqpoint{3.484705in}{1.268702in}}%
\pgfpathlineto{\pgfqpoint{3.488087in}{2.857056in}}%
\pgfpathlineto{\pgfqpoint{3.489778in}{1.896476in}}%
\pgfpathlineto{\pgfqpoint{3.490624in}{2.113995in}}%
\pgfpathlineto{\pgfqpoint{3.491470in}{1.796584in}}%
\pgfpathlineto{\pgfqpoint{3.492315in}{0.845177in}}%
\pgfpathlineto{\pgfqpoint{3.493161in}{2.748401in}}%
\pgfpathlineto{\pgfqpoint{3.494007in}{2.219666in}}%
\pgfpathlineto{\pgfqpoint{3.496543in}{1.479538in}}%
\pgfpathlineto{\pgfqpoint{3.497389in}{2.641724in}}%
\pgfpathlineto{\pgfqpoint{3.498235in}{2.008695in}}%
\pgfpathlineto{\pgfqpoint{3.499926in}{2.642651in}}%
\pgfpathlineto{\pgfqpoint{3.501617in}{1.373774in}}%
\pgfpathlineto{\pgfqpoint{3.502463in}{2.008048in}}%
\pgfpathlineto{\pgfqpoint{3.503308in}{1.584369in}}%
\pgfpathlineto{\pgfqpoint{3.504154in}{1.479574in}}%
\pgfpathlineto{\pgfqpoint{3.505000in}{2.219635in}}%
\pgfpathlineto{\pgfqpoint{3.505845in}{1.267507in}}%
\pgfpathlineto{\pgfqpoint{3.506691in}{2.536183in}}%
\pgfpathlineto{\pgfqpoint{3.507537in}{1.265402in}}%
\pgfpathlineto{\pgfqpoint{3.508382in}{1.694036in}}%
\pgfpathlineto{\pgfqpoint{3.509228in}{2.008353in}}%
\pgfpathlineto{\pgfqpoint{3.510073in}{1.902208in}}%
\pgfpathlineto{\pgfqpoint{3.510919in}{2.006643in}}%
\pgfpathlineto{\pgfqpoint{3.511765in}{1.267994in}}%
\pgfpathlineto{\pgfqpoint{3.512610in}{2.430500in}}%
\pgfpathlineto{\pgfqpoint{3.513456in}{1.691003in}}%
\pgfpathlineto{\pgfqpoint{3.515147in}{2.325325in}}%
\pgfpathlineto{\pgfqpoint{3.515993in}{1.479586in}}%
\pgfpathlineto{\pgfqpoint{3.516838in}{1.701581in}}%
\pgfpathlineto{\pgfqpoint{3.517684in}{2.108993in}}%
\pgfpathlineto{\pgfqpoint{3.518530in}{1.480277in}}%
\pgfpathlineto{\pgfqpoint{3.519375in}{1.796717in}}%
\pgfpathlineto{\pgfqpoint{3.521912in}{2.431422in}}%
\pgfpathlineto{\pgfqpoint{3.522758in}{2.217925in}}%
\pgfpathlineto{\pgfqpoint{3.523603in}{2.326920in}}%
\pgfpathlineto{\pgfqpoint{3.524449in}{1.906719in}}%
\pgfpathlineto{\pgfqpoint{3.525295in}{2.083735in}}%
\pgfpathlineto{\pgfqpoint{3.526140in}{2.317307in}}%
\pgfpathlineto{\pgfqpoint{3.526986in}{1.480351in}}%
\pgfpathlineto{\pgfqpoint{3.527831in}{2.153307in}}%
\pgfpathlineto{\pgfqpoint{3.528677in}{2.339099in}}%
\pgfpathlineto{\pgfqpoint{3.529523in}{2.326139in}}%
\pgfpathlineto{\pgfqpoint{3.530368in}{1.566638in}}%
\pgfpathlineto{\pgfqpoint{3.531214in}{2.217762in}}%
\pgfpathlineto{\pgfqpoint{3.532060in}{1.920062in}}%
\pgfpathlineto{\pgfqpoint{3.532905in}{1.124683in}}%
\pgfpathlineto{\pgfqpoint{3.533751in}{2.231561in}}%
\pgfpathlineto{\pgfqpoint{3.534596in}{2.029093in}}%
\pgfpathlineto{\pgfqpoint{3.535442in}{2.408887in}}%
\pgfpathlineto{\pgfqpoint{3.536288in}{2.078142in}}%
\pgfpathlineto{\pgfqpoint{3.537133in}{2.111742in}}%
\pgfpathlineto{\pgfqpoint{3.537979in}{2.236173in}}%
\pgfpathlineto{\pgfqpoint{3.538825in}{1.816900in}}%
\pgfpathlineto{\pgfqpoint{3.539670in}{2.009653in}}%
\pgfpathlineto{\pgfqpoint{3.540516in}{1.888223in}}%
\pgfpathlineto{\pgfqpoint{3.541361in}{2.008851in}}%
\pgfpathlineto{\pgfqpoint{3.543053in}{2.497322in}}%
\pgfpathlineto{\pgfqpoint{3.543898in}{1.559480in}}%
\pgfpathlineto{\pgfqpoint{3.544744in}{1.833097in}}%
\pgfpathlineto{\pgfqpoint{3.546435in}{1.462553in}}%
\pgfpathlineto{\pgfqpoint{3.548126in}{2.059048in}}%
\pgfpathlineto{\pgfqpoint{3.548972in}{2.038765in}}%
\pgfpathlineto{\pgfqpoint{3.549818in}{1.646604in}}%
\pgfpathlineto{\pgfqpoint{3.550663in}{2.045233in}}%
\pgfpathlineto{\pgfqpoint{3.551509in}{1.763639in}}%
\pgfpathlineto{\pgfqpoint{3.552355in}{2.078663in}}%
\pgfpathlineto{\pgfqpoint{3.553200in}{2.076604in}}%
\pgfpathlineto{\pgfqpoint{3.554046in}{1.259829in}}%
\pgfpathlineto{\pgfqpoint{3.555737in}{2.515851in}}%
\pgfpathlineto{\pgfqpoint{3.556583in}{1.633620in}}%
\pgfpathlineto{\pgfqpoint{3.557428in}{2.184041in}}%
\pgfpathlineto{\pgfqpoint{3.558274in}{2.695287in}}%
\pgfpathlineto{\pgfqpoint{3.559965in}{1.954791in}}%
\pgfpathlineto{\pgfqpoint{3.560811in}{1.961280in}}%
\pgfpathlineto{\pgfqpoint{3.561656in}{2.054578in}}%
\pgfpathlineto{\pgfqpoint{3.562502in}{1.652935in}}%
\pgfpathlineto{\pgfqpoint{3.563348in}{2.044906in}}%
\pgfpathlineto{\pgfqpoint{3.564193in}{1.837311in}}%
\pgfpathlineto{\pgfqpoint{3.565039in}{2.011515in}}%
\pgfpathlineto{\pgfqpoint{3.565885in}{1.695303in}}%
\pgfpathlineto{\pgfqpoint{3.566730in}{2.638195in}}%
\pgfpathlineto{\pgfqpoint{3.567576in}{1.373324in}}%
\pgfpathlineto{\pgfqpoint{3.568421in}{1.968049in}}%
\pgfpathlineto{\pgfqpoint{3.569267in}{2.024229in}}%
\pgfpathlineto{\pgfqpoint{3.570113in}{2.019627in}}%
\pgfpathlineto{\pgfqpoint{3.571804in}{1.476207in}}%
\pgfpathlineto{\pgfqpoint{3.572650in}{2.670949in}}%
\pgfpathlineto{\pgfqpoint{3.573495in}{1.996760in}}%
\pgfpathlineto{\pgfqpoint{3.574341in}{1.061539in}}%
\pgfpathlineto{\pgfqpoint{3.576032in}{2.636847in}}%
\pgfpathlineto{\pgfqpoint{3.576878in}{1.683786in}}%
\pgfpathlineto{\pgfqpoint{3.577723in}{1.855930in}}%
\pgfpathlineto{\pgfqpoint{3.578569in}{2.544351in}}%
\pgfpathlineto{\pgfqpoint{3.579415in}{1.896088in}}%
\pgfpathlineto{\pgfqpoint{3.580260in}{2.013221in}}%
\pgfpathlineto{\pgfqpoint{3.581106in}{2.061903in}}%
\pgfpathlineto{\pgfqpoint{3.581951in}{0.946668in}}%
\pgfpathlineto{\pgfqpoint{3.582797in}{1.650336in}}%
\pgfpathlineto{\pgfqpoint{3.583643in}{1.672007in}}%
\pgfpathlineto{\pgfqpoint{3.584488in}{2.314475in}}%
\pgfpathlineto{\pgfqpoint{3.585334in}{1.804903in}}%
\pgfpathlineto{\pgfqpoint{3.586180in}{1.752879in}}%
\pgfpathlineto{\pgfqpoint{3.588716in}{2.934675in}}%
\pgfpathlineto{\pgfqpoint{3.590408in}{1.800264in}}%
\pgfpathlineto{\pgfqpoint{3.591253in}{2.106447in}}%
\pgfpathlineto{\pgfqpoint{3.592099in}{1.809819in}}%
\pgfpathlineto{\pgfqpoint{3.592944in}{2.537669in}}%
\pgfpathlineto{\pgfqpoint{3.593790in}{2.286225in}}%
\pgfpathlineto{\pgfqpoint{3.595481in}{1.533390in}}%
\pgfpathlineto{\pgfqpoint{3.596327in}{2.688672in}}%
\pgfpathlineto{\pgfqpoint{3.597173in}{1.521342in}}%
\pgfpathlineto{\pgfqpoint{3.598018in}{1.968400in}}%
\pgfpathlineto{\pgfqpoint{3.598864in}{2.494177in}}%
\pgfpathlineto{\pgfqpoint{3.601401in}{1.459171in}}%
\pgfpathlineto{\pgfqpoint{3.602246in}{2.373284in}}%
\pgfpathlineto{\pgfqpoint{3.603092in}{1.648926in}}%
\pgfpathlineto{\pgfqpoint{3.603938in}{2.349994in}}%
\pgfpathlineto{\pgfqpoint{3.604783in}{2.199060in}}%
\pgfpathlineto{\pgfqpoint{3.605629in}{2.107869in}}%
\pgfpathlineto{\pgfqpoint{3.606474in}{2.118661in}}%
\pgfpathlineto{\pgfqpoint{3.608166in}{2.722960in}}%
\pgfpathlineto{\pgfqpoint{3.609011in}{1.381139in}}%
\pgfpathlineto{\pgfqpoint{3.609857in}{1.651229in}}%
\pgfpathlineto{\pgfqpoint{3.611548in}{2.952662in}}%
\pgfpathlineto{\pgfqpoint{3.612394in}{2.074220in}}%
\pgfpathlineto{\pgfqpoint{3.613239in}{2.622513in}}%
\pgfpathlineto{\pgfqpoint{3.614085in}{2.480743in}}%
\pgfpathlineto{\pgfqpoint{3.614931in}{1.618222in}}%
\pgfpathlineto{\pgfqpoint{3.615776in}{2.088467in}}%
\pgfpathlineto{\pgfqpoint{3.616622in}{1.823890in}}%
\pgfpathlineto{\pgfqpoint{3.617468in}{2.447336in}}%
\pgfpathlineto{\pgfqpoint{3.618313in}{1.957524in}}%
\pgfpathlineto{\pgfqpoint{3.619159in}{1.671477in}}%
\pgfpathlineto{\pgfqpoint{3.621696in}{2.111833in}}%
\pgfpathlineto{\pgfqpoint{3.623387in}{1.574444in}}%
\pgfpathlineto{\pgfqpoint{3.624233in}{2.025706in}}%
\pgfpathlineto{\pgfqpoint{3.625078in}{1.479153in}}%
\pgfpathlineto{\pgfqpoint{3.625924in}{2.476716in}}%
\pgfpathlineto{\pgfqpoint{3.626769in}{1.788526in}}%
\pgfpathlineto{\pgfqpoint{3.629306in}{2.618804in}}%
\pgfpathlineto{\pgfqpoint{3.631843in}{1.873861in}}%
\pgfpathlineto{\pgfqpoint{3.632689in}{2.374668in}}%
\pgfpathlineto{\pgfqpoint{3.633534in}{2.027498in}}%
\pgfpathlineto{\pgfqpoint{3.634380in}{1.664216in}}%
\pgfpathlineto{\pgfqpoint{3.635226in}{1.821897in}}%
\pgfpathlineto{\pgfqpoint{3.636071in}{2.357501in}}%
\pgfpathlineto{\pgfqpoint{3.636917in}{1.351148in}}%
\pgfpathlineto{\pgfqpoint{3.637763in}{1.827207in}}%
\pgfpathlineto{\pgfqpoint{3.638608in}{1.857785in}}%
\pgfpathlineto{\pgfqpoint{3.639454in}{2.314750in}}%
\pgfpathlineto{\pgfqpoint{3.640299in}{1.565281in}}%
\pgfpathlineto{\pgfqpoint{3.641145in}{1.857665in}}%
\pgfpathlineto{\pgfqpoint{3.641991in}{2.482831in}}%
\pgfpathlineto{\pgfqpoint{3.642836in}{2.061625in}}%
\pgfpathlineto{\pgfqpoint{3.643682in}{1.427914in}}%
\pgfpathlineto{\pgfqpoint{3.644528in}{1.691649in}}%
\pgfpathlineto{\pgfqpoint{3.647064in}{2.401230in}}%
\pgfpathlineto{\pgfqpoint{3.648756in}{1.977942in}}%
\pgfpathlineto{\pgfqpoint{3.649601in}{2.157737in}}%
\pgfpathlineto{\pgfqpoint{3.650447in}{1.955468in}}%
\pgfpathlineto{\pgfqpoint{3.651293in}{2.582619in}}%
\pgfpathlineto{\pgfqpoint{3.652984in}{1.799654in}}%
\pgfpathlineto{\pgfqpoint{3.653829in}{2.167296in}}%
\pgfpathlineto{\pgfqpoint{3.654675in}{1.955360in}}%
\pgfpathlineto{\pgfqpoint{3.656366in}{1.954312in}}%
\pgfpathlineto{\pgfqpoint{3.657212in}{2.427381in}}%
\pgfpathlineto{\pgfqpoint{3.658058in}{2.253644in}}%
\pgfpathlineto{\pgfqpoint{3.658903in}{2.006773in}}%
\pgfpathlineto{\pgfqpoint{3.659749in}{2.637897in}}%
\pgfpathlineto{\pgfqpoint{3.662286in}{1.346190in}}%
\pgfpathlineto{\pgfqpoint{3.663131in}{2.396539in}}%
\pgfpathlineto{\pgfqpoint{3.663977in}{1.955156in}}%
\pgfpathlineto{\pgfqpoint{3.664823in}{1.687882in}}%
\pgfpathlineto{\pgfqpoint{3.667359in}{2.180915in}}%
\pgfpathlineto{\pgfqpoint{3.668205in}{2.010759in}}%
\pgfpathlineto{\pgfqpoint{3.669051in}{1.386909in}}%
\pgfpathlineto{\pgfqpoint{3.669896in}{2.570991in}}%
\pgfpathlineto{\pgfqpoint{3.670742in}{2.011074in}}%
\pgfpathlineto{\pgfqpoint{3.671587in}{2.079809in}}%
\pgfpathlineto{\pgfqpoint{3.672433in}{2.390056in}}%
\pgfpathlineto{\pgfqpoint{3.673279in}{1.290841in}}%
\pgfpathlineto{\pgfqpoint{3.674124in}{2.227484in}}%
\pgfpathlineto{\pgfqpoint{3.674970in}{2.085850in}}%
\pgfpathlineto{\pgfqpoint{3.675816in}{2.166997in}}%
\pgfpathlineto{\pgfqpoint{3.676661in}{1.949742in}}%
\pgfpathlineto{\pgfqpoint{3.677507in}{2.472631in}}%
\pgfpathlineto{\pgfqpoint{3.678352in}{1.379831in}}%
\pgfpathlineto{\pgfqpoint{3.679198in}{1.954598in}}%
\pgfpathlineto{\pgfqpoint{3.680044in}{1.807156in}}%
\pgfpathlineto{\pgfqpoint{3.680889in}{1.821212in}}%
\pgfpathlineto{\pgfqpoint{3.683426in}{2.500360in}}%
\pgfpathlineto{\pgfqpoint{3.684272in}{1.529620in}}%
\pgfpathlineto{\pgfqpoint{3.685117in}{1.694777in}}%
\pgfpathlineto{\pgfqpoint{3.685963in}{1.680086in}}%
\pgfpathlineto{\pgfqpoint{3.688500in}{2.539848in}}%
\pgfpathlineto{\pgfqpoint{3.690191in}{1.905046in}}%
\pgfpathlineto{\pgfqpoint{3.691037in}{2.008286in}}%
\pgfpathlineto{\pgfqpoint{3.691882in}{2.318778in}}%
\pgfpathlineto{\pgfqpoint{3.694419in}{1.325558in}}%
\pgfpathlineto{\pgfqpoint{3.695265in}{1.710227in}}%
\pgfpathlineto{\pgfqpoint{3.696111in}{2.656217in}}%
\pgfpathlineto{\pgfqpoint{3.696956in}{2.390837in}}%
\pgfpathlineto{\pgfqpoint{3.697802in}{2.031913in}}%
\pgfpathlineto{\pgfqpoint{3.698647in}{2.334576in}}%
\pgfpathlineto{\pgfqpoint{3.700339in}{1.447201in}}%
\pgfpathlineto{\pgfqpoint{3.701184in}{2.538171in}}%
\pgfpathlineto{\pgfqpoint{3.702030in}{2.113532in}}%
\pgfpathlineto{\pgfqpoint{3.702876in}{2.223058in}}%
\pgfpathlineto{\pgfqpoint{3.703721in}{2.218233in}}%
\pgfpathlineto{\pgfqpoint{3.705412in}{1.268119in}}%
\pgfpathlineto{\pgfqpoint{3.706258in}{1.479842in}}%
\pgfpathlineto{\pgfqpoint{3.707104in}{2.218892in}}%
\pgfpathlineto{\pgfqpoint{3.707949in}{1.797127in}}%
\pgfpathlineto{\pgfqpoint{3.708795in}{1.373841in}}%
\pgfpathlineto{\pgfqpoint{3.709641in}{2.442932in}}%
\pgfpathlineto{\pgfqpoint{3.710486in}{1.711019in}}%
\pgfpathlineto{\pgfqpoint{3.711332in}{1.902315in}}%
\pgfpathlineto{\pgfqpoint{3.712177in}{1.796620in}}%
\pgfpathlineto{\pgfqpoint{3.713869in}{2.008143in}}%
\pgfpathlineto{\pgfqpoint{3.714714in}{1.585198in}}%
\pgfpathlineto{\pgfqpoint{3.715560in}{1.796684in}}%
\pgfpathlineto{\pgfqpoint{3.716406in}{2.219648in}}%
\pgfpathlineto{\pgfqpoint{3.717251in}{1.796435in}}%
\pgfpathlineto{\pgfqpoint{3.718097in}{2.735680in}}%
\pgfpathlineto{\pgfqpoint{3.718942in}{1.690738in}}%
\pgfpathlineto{\pgfqpoint{3.719788in}{2.430779in}}%
\pgfpathlineto{\pgfqpoint{3.720634in}{1.585245in}}%
\pgfpathlineto{\pgfqpoint{3.721479in}{1.824480in}}%
\pgfpathlineto{\pgfqpoint{3.723171in}{1.689309in}}%
\pgfpathlineto{\pgfqpoint{3.725707in}{2.519893in}}%
\pgfpathlineto{\pgfqpoint{3.728244in}{1.157178in}}%
\pgfpathlineto{\pgfqpoint{3.729090in}{2.538063in}}%
\pgfpathlineto{\pgfqpoint{3.729936in}{2.014092in}}%
\pgfpathlineto{\pgfqpoint{3.731627in}{2.327267in}}%
\pgfpathlineto{\pgfqpoint{3.732472in}{2.113703in}}%
\pgfpathlineto{\pgfqpoint{3.733318in}{1.479575in}}%
\pgfpathlineto{\pgfqpoint{3.734164in}{2.325193in}}%
\pgfpathlineto{\pgfqpoint{3.735009in}{2.323481in}}%
\pgfpathlineto{\pgfqpoint{3.735855in}{1.689878in}}%
\pgfpathlineto{\pgfqpoint{3.736701in}{2.323676in}}%
\pgfpathlineto{\pgfqpoint{3.737546in}{1.691125in}}%
\pgfpathlineto{\pgfqpoint{3.738392in}{3.171103in}}%
\pgfpathlineto{\pgfqpoint{3.739237in}{1.695547in}}%
\pgfpathlineto{\pgfqpoint{3.740083in}{2.113606in}}%
\pgfpathlineto{\pgfqpoint{3.740929in}{1.796133in}}%
\pgfpathlineto{\pgfqpoint{3.741774in}{1.903044in}}%
\pgfpathlineto{\pgfqpoint{3.742620in}{2.485534in}}%
\pgfpathlineto{\pgfqpoint{3.743466in}{2.325286in}}%
\pgfpathlineto{\pgfqpoint{3.745157in}{1.583738in}}%
\pgfpathlineto{\pgfqpoint{3.746002in}{1.609611in}}%
\pgfpathlineto{\pgfqpoint{3.748539in}{2.315247in}}%
\pgfpathlineto{\pgfqpoint{3.749385in}{1.477379in}}%
\pgfpathlineto{\pgfqpoint{3.750231in}{1.899116in}}%
\pgfpathlineto{\pgfqpoint{3.751076in}{1.901843in}}%
\pgfpathlineto{\pgfqpoint{3.751922in}{1.888283in}}%
\pgfpathlineto{\pgfqpoint{3.753613in}{2.642427in}}%
\pgfpathlineto{\pgfqpoint{3.756150in}{1.955287in}}%
\pgfpathlineto{\pgfqpoint{3.756995in}{1.849720in}}%
\pgfpathlineto{\pgfqpoint{3.757841in}{2.188829in}}%
\pgfpathlineto{\pgfqpoint{3.758687in}{1.957012in}}%
\pgfpathlineto{\pgfqpoint{3.759532in}{1.530426in}}%
\pgfpathlineto{\pgfqpoint{3.760378in}{1.690924in}}%
\pgfpathlineto{\pgfqpoint{3.762915in}{2.852745in}}%
\pgfpathlineto{\pgfqpoint{3.763760in}{1.794754in}}%
\pgfpathlineto{\pgfqpoint{3.764606in}{2.006292in}}%
\pgfpathlineto{\pgfqpoint{3.765452in}{2.007459in}}%
\pgfpathlineto{\pgfqpoint{3.766297in}{2.325230in}}%
\pgfpathlineto{\pgfqpoint{3.767143in}{1.855408in}}%
\pgfpathlineto{\pgfqpoint{3.767989in}{2.748052in}}%
\pgfpathlineto{\pgfqpoint{3.768834in}{2.005427in}}%
\pgfpathlineto{\pgfqpoint{3.769680in}{2.008123in}}%
\pgfpathlineto{\pgfqpoint{3.770525in}{2.113662in}}%
\pgfpathlineto{\pgfqpoint{3.771371in}{1.744373in}}%
\pgfpathlineto{\pgfqpoint{3.772217in}{2.325385in}}%
\pgfpathlineto{\pgfqpoint{3.773908in}{1.690343in}}%
\pgfpathlineto{\pgfqpoint{3.774754in}{2.642687in}}%
\pgfpathlineto{\pgfqpoint{3.775599in}{2.642394in}}%
\pgfpathlineto{\pgfqpoint{3.776445in}{2.533490in}}%
\pgfpathlineto{\pgfqpoint{3.778982in}{1.346553in}}%
\pgfpathlineto{\pgfqpoint{3.779827in}{1.912661in}}%
\pgfpathlineto{\pgfqpoint{3.780673in}{1.902544in}}%
\pgfpathlineto{\pgfqpoint{3.781519in}{1.982308in}}%
\pgfpathlineto{\pgfqpoint{3.782364in}{1.627991in}}%
\pgfpathlineto{\pgfqpoint{3.783210in}{2.397717in}}%
\pgfpathlineto{\pgfqpoint{3.784055in}{2.178037in}}%
\pgfpathlineto{\pgfqpoint{3.784901in}{1.849561in}}%
\pgfpathlineto{\pgfqpoint{3.785747in}{2.059982in}}%
\pgfpathlineto{\pgfqpoint{3.786592in}{2.378183in}}%
\pgfpathlineto{\pgfqpoint{3.788284in}{1.743822in}}%
\pgfpathlineto{\pgfqpoint{3.789129in}{1.849615in}}%
\pgfpathlineto{\pgfqpoint{3.789975in}{2.060967in}}%
\pgfpathlineto{\pgfqpoint{3.790820in}{2.694326in}}%
\pgfpathlineto{\pgfqpoint{3.791666in}{1.849383in}}%
\pgfpathlineto{\pgfqpoint{3.792512in}{1.955292in}}%
\pgfpathlineto{\pgfqpoint{3.793357in}{1.839619in}}%
\pgfpathlineto{\pgfqpoint{3.794203in}{2.907017in}}%
\pgfpathlineto{\pgfqpoint{3.795049in}{1.849880in}}%
\pgfpathlineto{\pgfqpoint{3.795894in}{2.169722in}}%
\pgfpathlineto{\pgfqpoint{3.797585in}{1.997979in}}%
\pgfpathlineto{\pgfqpoint{3.798431in}{2.165781in}}%
\pgfpathlineto{\pgfqpoint{3.800968in}{1.828827in}}%
\pgfpathlineto{\pgfqpoint{3.802659in}{2.113979in}}%
\pgfpathlineto{\pgfqpoint{3.803505in}{1.690813in}}%
\pgfpathlineto{\pgfqpoint{3.804350in}{2.531999in}}%
\pgfpathlineto{\pgfqpoint{3.805196in}{2.008461in}}%
\pgfpathlineto{\pgfqpoint{3.806042in}{2.007151in}}%
\pgfpathlineto{\pgfqpoint{3.806887in}{1.900144in}}%
\pgfpathlineto{\pgfqpoint{3.807733in}{1.907634in}}%
\pgfpathlineto{\pgfqpoint{3.809424in}{2.534248in}}%
\pgfpathlineto{\pgfqpoint{3.810270in}{1.743847in}}%
\pgfpathlineto{\pgfqpoint{3.811961in}{2.741277in}}%
\pgfpathlineto{\pgfqpoint{3.814498in}{1.300945in}}%
\pgfpathlineto{\pgfqpoint{3.817880in}{2.641450in}}%
\pgfpathlineto{\pgfqpoint{3.818726in}{1.826208in}}%
\pgfpathlineto{\pgfqpoint{3.819572in}{2.589899in}}%
\pgfpathlineto{\pgfqpoint{3.820417in}{2.305236in}}%
\pgfpathlineto{\pgfqpoint{3.822109in}{1.157064in}}%
\pgfpathlineto{\pgfqpoint{3.822954in}{2.325480in}}%
\pgfpathlineto{\pgfqpoint{3.823800in}{2.154382in}}%
\pgfpathlineto{\pgfqpoint{3.826337in}{1.900493in}}%
\pgfpathlineto{\pgfqpoint{3.828874in}{2.267006in}}%
\pgfpathlineto{\pgfqpoint{3.831410in}{1.316835in}}%
\pgfpathlineto{\pgfqpoint{3.832256in}{1.317455in}}%
\pgfpathlineto{\pgfqpoint{3.836484in}{2.430661in}}%
\pgfpathlineto{\pgfqpoint{3.838175in}{1.587053in}}%
\pgfpathlineto{\pgfqpoint{3.839021in}{1.690959in}}%
\pgfpathlineto{\pgfqpoint{3.839867in}{1.479975in}}%
\pgfpathlineto{\pgfqpoint{3.843249in}{2.748127in}}%
\pgfpathlineto{\pgfqpoint{3.844940in}{2.008164in}}%
\pgfpathlineto{\pgfqpoint{3.846632in}{2.641510in}}%
\pgfpathlineto{\pgfqpoint{3.849168in}{1.479488in}}%
\pgfpathlineto{\pgfqpoint{3.850014in}{1.690825in}}%
\pgfpathlineto{\pgfqpoint{3.850860in}{2.008256in}}%
\pgfpathlineto{\pgfqpoint{3.851705in}{2.008103in}}%
\pgfpathlineto{\pgfqpoint{3.852551in}{1.690978in}}%
\pgfpathlineto{\pgfqpoint{3.853397in}{2.430999in}}%
\pgfpathlineto{\pgfqpoint{3.854242in}{2.008175in}}%
\pgfpathlineto{\pgfqpoint{3.855088in}{2.219640in}}%
\pgfpathlineto{\pgfqpoint{3.855933in}{2.853672in}}%
\pgfpathlineto{\pgfqpoint{3.856779in}{2.431069in}}%
\pgfpathlineto{\pgfqpoint{3.857625in}{2.325325in}}%
\pgfpathlineto{\pgfqpoint{3.858470in}{1.796728in}}%
\pgfpathlineto{\pgfqpoint{3.859316in}{2.113845in}}%
\pgfpathlineto{\pgfqpoint{3.860162in}{2.325397in}}%
\pgfpathlineto{\pgfqpoint{3.861007in}{1.373925in}}%
\pgfpathlineto{\pgfqpoint{3.862698in}{2.853815in}}%
\pgfpathlineto{\pgfqpoint{3.864390in}{1.796632in}}%
\pgfpathlineto{\pgfqpoint{3.865235in}{2.629312in}}%
\pgfpathlineto{\pgfqpoint{3.866081in}{2.430135in}}%
\pgfpathlineto{\pgfqpoint{3.867772in}{1.683898in}}%
\pgfpathlineto{\pgfqpoint{3.868618in}{1.849551in}}%
\pgfpathlineto{\pgfqpoint{3.870309in}{2.272485in}}%
\pgfpathlineto{\pgfqpoint{3.871155in}{1.849621in}}%
\pgfpathlineto{\pgfqpoint{3.872000in}{2.165790in}}%
\pgfpathlineto{\pgfqpoint{3.873692in}{2.482402in}}%
\pgfpathlineto{\pgfqpoint{3.875383in}{1.743803in}}%
\pgfpathlineto{\pgfqpoint{3.876228in}{1.840694in}}%
\pgfpathlineto{\pgfqpoint{3.878765in}{2.272441in}}%
\pgfpathlineto{\pgfqpoint{3.879611in}{1.955644in}}%
\pgfpathlineto{\pgfqpoint{3.880457in}{2.272475in}}%
\pgfpathlineto{\pgfqpoint{3.881302in}{2.003943in}}%
\pgfpathlineto{\pgfqpoint{3.882148in}{2.182776in}}%
\pgfpathlineto{\pgfqpoint{3.882993in}{2.306136in}}%
\pgfpathlineto{\pgfqpoint{3.883839in}{1.743150in}}%
\pgfpathlineto{\pgfqpoint{3.884685in}{2.378964in}}%
\pgfpathlineto{\pgfqpoint{3.885530in}{1.109057in}}%
\pgfpathlineto{\pgfqpoint{3.886376in}{2.061314in}}%
\pgfpathlineto{\pgfqpoint{3.887222in}{2.166483in}}%
\pgfpathlineto{\pgfqpoint{3.888067in}{1.849707in}}%
\pgfpathlineto{\pgfqpoint{3.888913in}{2.377925in}}%
\pgfpathlineto{\pgfqpoint{3.889758in}{2.272539in}}%
\pgfpathlineto{\pgfqpoint{3.891450in}{1.318249in}}%
\pgfpathlineto{\pgfqpoint{3.893141in}{2.695485in}}%
\pgfpathlineto{\pgfqpoint{3.893987in}{1.437460in}}%
\pgfpathlineto{\pgfqpoint{3.894832in}{2.060477in}}%
\pgfpathlineto{\pgfqpoint{3.895678in}{2.060807in}}%
\pgfpathlineto{\pgfqpoint{3.896523in}{1.819485in}}%
\pgfpathlineto{\pgfqpoint{3.897369in}{1.955297in}}%
\pgfpathlineto{\pgfqpoint{3.899060in}{2.272519in}}%
\pgfpathlineto{\pgfqpoint{3.902443in}{1.532349in}}%
\pgfpathlineto{\pgfqpoint{3.903288in}{1.638037in}}%
\pgfpathlineto{\pgfqpoint{3.904134in}{2.589716in}}%
\pgfpathlineto{\pgfqpoint{3.906671in}{1.003672in}}%
\pgfpathlineto{\pgfqpoint{3.907517in}{0.892045in}}%
\pgfpathlineto{\pgfqpoint{3.908362in}{2.378225in}}%
\pgfpathlineto{\pgfqpoint{3.909208in}{1.638020in}}%
\pgfpathlineto{\pgfqpoint{3.910053in}{1.743819in}}%
\pgfpathlineto{\pgfqpoint{3.910899in}{2.378503in}}%
\pgfpathlineto{\pgfqpoint{3.911745in}{2.166626in}}%
\pgfpathlineto{\pgfqpoint{3.912590in}{1.851513in}}%
\pgfpathlineto{\pgfqpoint{3.913436in}{2.484043in}}%
\pgfpathlineto{\pgfqpoint{3.914281in}{1.426859in}}%
\pgfpathlineto{\pgfqpoint{3.915127in}{2.801178in}}%
\pgfpathlineto{\pgfqpoint{3.915973in}{2.152059in}}%
\pgfpathlineto{\pgfqpoint{3.917664in}{2.377700in}}%
\pgfpathlineto{\pgfqpoint{3.918510in}{1.849517in}}%
\pgfpathlineto{\pgfqpoint{3.919355in}{1.849555in}}%
\pgfpathlineto{\pgfqpoint{3.920201in}{2.483971in}}%
\pgfpathlineto{\pgfqpoint{3.921046in}{1.955310in}}%
\pgfpathlineto{\pgfqpoint{3.921892in}{2.378240in}}%
\pgfpathlineto{\pgfqpoint{3.922738in}{2.272490in}}%
\pgfpathlineto{\pgfqpoint{3.923583in}{1.447905in}}%
\pgfpathlineto{\pgfqpoint{3.924429in}{2.061072in}}%
\pgfpathlineto{\pgfqpoint{3.925275in}{2.061031in}}%
\pgfpathlineto{\pgfqpoint{3.926966in}{2.483961in}}%
\pgfpathlineto{\pgfqpoint{3.929503in}{1.849561in}}%
\pgfpathlineto{\pgfqpoint{3.930348in}{2.585213in}}%
\pgfpathlineto{\pgfqpoint{3.931194in}{1.955314in}}%
\pgfpathlineto{\pgfqpoint{3.932040in}{2.466096in}}%
\pgfpathlineto{\pgfqpoint{3.933731in}{1.955195in}}%
\pgfpathlineto{\pgfqpoint{3.935422in}{2.390834in}}%
\pgfpathlineto{\pgfqpoint{3.937113in}{2.045723in}}%
\pgfpathlineto{\pgfqpoint{3.937959in}{2.165735in}}%
\pgfpathlineto{\pgfqpoint{3.938805in}{2.061021in}}%
\pgfpathlineto{\pgfqpoint{3.939650in}{1.725394in}}%
\pgfpathlineto{\pgfqpoint{3.940496in}{2.597069in}}%
\pgfpathlineto{\pgfqpoint{3.941341in}{1.479419in}}%
\pgfpathlineto{\pgfqpoint{3.942187in}{2.248974in}}%
\pgfpathlineto{\pgfqpoint{3.943878in}{1.412917in}}%
\pgfpathlineto{\pgfqpoint{3.944724in}{2.662322in}}%
\pgfpathlineto{\pgfqpoint{3.945570in}{2.060739in}}%
\pgfpathlineto{\pgfqpoint{3.946415in}{1.636539in}}%
\pgfpathlineto{\pgfqpoint{3.947261in}{1.720846in}}%
\pgfpathlineto{\pgfqpoint{3.948952in}{2.164534in}}%
\pgfpathlineto{\pgfqpoint{3.949798in}{1.796674in}}%
\pgfpathlineto{\pgfqpoint{3.950643in}{2.219661in}}%
\pgfpathlineto{\pgfqpoint{3.951489in}{1.585100in}}%
\pgfpathlineto{\pgfqpoint{3.952335in}{2.536667in}}%
\pgfpathlineto{\pgfqpoint{3.953180in}{1.796914in}}%
\pgfpathlineto{\pgfqpoint{3.954026in}{1.796692in}}%
\pgfpathlineto{\pgfqpoint{3.955717in}{2.642577in}}%
\pgfpathlineto{\pgfqpoint{3.956563in}{1.479961in}}%
\pgfpathlineto{\pgfqpoint{3.957408in}{1.690774in}}%
\pgfpathlineto{\pgfqpoint{3.958254in}{2.431008in}}%
\pgfpathlineto{\pgfqpoint{3.959100in}{2.429192in}}%
\pgfpathlineto{\pgfqpoint{3.959945in}{1.796605in}}%
\pgfpathlineto{\pgfqpoint{3.960791in}{2.959773in}}%
\pgfpathlineto{\pgfqpoint{3.961636in}{2.220199in}}%
\pgfpathlineto{\pgfqpoint{3.962482in}{2.748056in}}%
\pgfpathlineto{\pgfqpoint{3.964173in}{1.478789in}}%
\pgfpathlineto{\pgfqpoint{3.965019in}{1.796740in}}%
\pgfpathlineto{\pgfqpoint{3.965865in}{1.690134in}}%
\pgfpathlineto{\pgfqpoint{3.966710in}{1.797303in}}%
\pgfpathlineto{\pgfqpoint{3.969247in}{3.276891in}}%
\pgfpathlineto{\pgfqpoint{3.970938in}{1.373900in}}%
\pgfpathlineto{\pgfqpoint{3.973475in}{2.219372in}}%
\pgfpathlineto{\pgfqpoint{3.974321in}{2.219651in}}%
\pgfpathlineto{\pgfqpoint{3.975166in}{2.328615in}}%
\pgfpathlineto{\pgfqpoint{3.976012in}{2.958922in}}%
\pgfpathlineto{\pgfqpoint{3.976858in}{1.796683in}}%
\pgfpathlineto{\pgfqpoint{3.977703in}{2.113951in}}%
\pgfpathlineto{\pgfqpoint{3.978549in}{2.008216in}}%
\pgfpathlineto{\pgfqpoint{3.980240in}{1.585044in}}%
\pgfpathlineto{\pgfqpoint{3.982777in}{2.645462in}}%
\pgfpathlineto{\pgfqpoint{3.985314in}{1.585081in}}%
\pgfpathlineto{\pgfqpoint{3.986160in}{2.219631in}}%
\pgfpathlineto{\pgfqpoint{3.987005in}{2.113844in}}%
\pgfpathlineto{\pgfqpoint{3.987851in}{1.373687in}}%
\pgfpathlineto{\pgfqpoint{3.988696in}{1.479714in}}%
\pgfpathlineto{\pgfqpoint{3.991233in}{2.219674in}}%
\pgfpathlineto{\pgfqpoint{3.992079in}{1.058749in}}%
\pgfpathlineto{\pgfqpoint{3.992924in}{2.113868in}}%
\pgfpathlineto{\pgfqpoint{3.993770in}{2.007995in}}%
\pgfpathlineto{\pgfqpoint{3.994616in}{2.114056in}}%
\pgfpathlineto{\pgfqpoint{3.995461in}{2.642591in}}%
\pgfpathlineto{\pgfqpoint{3.997998in}{1.796688in}}%
\pgfpathlineto{\pgfqpoint{3.998844in}{1.268072in}}%
\pgfpathlineto{\pgfqpoint{3.999689in}{1.584552in}}%
\pgfpathlineto{\pgfqpoint{4.000535in}{1.902358in}}%
\pgfpathlineto{\pgfqpoint{4.001381in}{1.373996in}}%
\pgfpathlineto{\pgfqpoint{4.002226in}{2.748308in}}%
\pgfpathlineto{\pgfqpoint{4.003072in}{1.495022in}}%
\pgfpathlineto{\pgfqpoint{4.003918in}{2.219144in}}%
\pgfpathlineto{\pgfqpoint{4.004763in}{2.113932in}}%
\pgfpathlineto{\pgfqpoint{4.005609in}{2.219325in}}%
\pgfpathlineto{\pgfqpoint{4.006454in}{1.692196in}}%
\pgfpathlineto{\pgfqpoint{4.007300in}{2.431125in}}%
\pgfpathlineto{\pgfqpoint{4.008146in}{1.479480in}}%
\pgfpathlineto{\pgfqpoint{4.008991in}{2.008094in}}%
\pgfpathlineto{\pgfqpoint{4.011528in}{2.431094in}}%
\pgfpathlineto{\pgfqpoint{4.012374in}{1.690941in}}%
\pgfpathlineto{\pgfqpoint{4.013219in}{2.008135in}}%
\pgfpathlineto{\pgfqpoint{4.014065in}{2.291867in}}%
\pgfpathlineto{\pgfqpoint{4.014911in}{1.796661in}}%
\pgfpathlineto{\pgfqpoint{4.015756in}{2.166078in}}%
\pgfpathlineto{\pgfqpoint{4.017448in}{1.745669in}}%
\pgfpathlineto{\pgfqpoint{4.018293in}{2.378229in}}%
\pgfpathlineto{\pgfqpoint{4.019139in}{1.638359in}}%
\pgfpathlineto{\pgfqpoint{4.019984in}{2.272268in}}%
\pgfpathlineto{\pgfqpoint{4.020830in}{2.060894in}}%
\pgfpathlineto{\pgfqpoint{4.021676in}{1.744119in}}%
\pgfpathlineto{\pgfqpoint{4.023367in}{2.695473in}}%
\pgfpathlineto{\pgfqpoint{4.025058in}{1.320600in}}%
\pgfpathlineto{\pgfqpoint{4.025904in}{2.061035in}}%
\pgfpathlineto{\pgfqpoint{4.026749in}{1.532998in}}%
\pgfpathlineto{\pgfqpoint{4.030132in}{2.272566in}}%
\pgfpathlineto{\pgfqpoint{4.030978in}{2.166841in}}%
\pgfpathlineto{\pgfqpoint{4.031823in}{2.376389in}}%
\pgfpathlineto{\pgfqpoint{4.034360in}{1.743720in}}%
\pgfpathlineto{\pgfqpoint{4.036897in}{2.273041in}}%
\pgfpathlineto{\pgfqpoint{4.038588in}{1.532249in}}%
\pgfpathlineto{\pgfqpoint{4.041125in}{2.608329in}}%
\pgfpathlineto{\pgfqpoint{4.041971in}{1.320373in}}%
\pgfpathlineto{\pgfqpoint{4.042816in}{2.168512in}}%
\pgfpathlineto{\pgfqpoint{4.043662in}{2.589868in}}%
\pgfpathlineto{\pgfqpoint{4.044508in}{2.272615in}}%
\pgfpathlineto{\pgfqpoint{4.045353in}{2.163743in}}%
\pgfpathlineto{\pgfqpoint{4.046199in}{1.639816in}}%
\pgfpathlineto{\pgfqpoint{4.047044in}{2.167063in}}%
\pgfpathlineto{\pgfqpoint{4.047890in}{1.427136in}}%
\pgfpathlineto{\pgfqpoint{4.048736in}{1.958510in}}%
\pgfpathlineto{\pgfqpoint{4.049581in}{1.746173in}}%
\pgfpathlineto{\pgfqpoint{4.050427in}{2.166366in}}%
\pgfpathlineto{\pgfqpoint{4.051273in}{2.074722in}}%
\pgfpathlineto{\pgfqpoint{4.052118in}{1.638042in}}%
\pgfpathlineto{\pgfqpoint{4.052964in}{2.589667in}}%
\pgfpathlineto{\pgfqpoint{4.053809in}{2.272895in}}%
\pgfpathlineto{\pgfqpoint{4.055501in}{1.638579in}}%
\pgfpathlineto{\pgfqpoint{4.056346in}{1.849647in}}%
\pgfpathlineto{\pgfqpoint{4.058038in}{2.061045in}}%
\pgfpathlineto{\pgfqpoint{4.059729in}{2.694881in}}%
\pgfpathlineto{\pgfqpoint{4.062266in}{1.320880in}}%
\pgfpathlineto{\pgfqpoint{4.063111in}{1.109396in}}%
\pgfpathlineto{\pgfqpoint{4.063957in}{2.272414in}}%
\pgfpathlineto{\pgfqpoint{4.064803in}{1.849308in}}%
\pgfpathlineto{\pgfqpoint{4.065648in}{1.850651in}}%
\pgfpathlineto{\pgfqpoint{4.066494in}{2.695504in}}%
\pgfpathlineto{\pgfqpoint{4.068185in}{1.426617in}}%
\pgfpathlineto{\pgfqpoint{4.069031in}{2.906938in}}%
\pgfpathlineto{\pgfqpoint{4.069876in}{2.272527in}}%
\pgfpathlineto{\pgfqpoint{4.070722in}{2.163555in}}%
\pgfpathlineto{\pgfqpoint{4.071567in}{1.215101in}}%
\pgfpathlineto{\pgfqpoint{4.072413in}{1.743834in}}%
\pgfpathlineto{\pgfqpoint{4.073259in}{1.743809in}}%
\pgfpathlineto{\pgfqpoint{4.074104in}{2.906910in}}%
\pgfpathlineto{\pgfqpoint{4.074950in}{2.483976in}}%
\pgfpathlineto{\pgfqpoint{4.077487in}{1.955285in}}%
\pgfpathlineto{\pgfqpoint{4.078332in}{1.955029in}}%
\pgfpathlineto{\pgfqpoint{4.079178in}{1.743494in}}%
\pgfpathlineto{\pgfqpoint{4.080024in}{1.849518in}}%
\pgfpathlineto{\pgfqpoint{4.080869in}{2.061029in}}%
\pgfpathlineto{\pgfqpoint{4.081715in}{1.320873in}}%
\pgfpathlineto{\pgfqpoint{4.082561in}{1.532347in}}%
\pgfpathlineto{\pgfqpoint{4.085097in}{1.955284in}}%
\pgfpathlineto{\pgfqpoint{4.085943in}{1.638096in}}%
\pgfpathlineto{\pgfqpoint{4.086789in}{2.394264in}}%
\pgfpathlineto{\pgfqpoint{4.087634in}{2.272264in}}%
\pgfpathlineto{\pgfqpoint{4.090171in}{1.638077in}}%
\pgfpathlineto{\pgfqpoint{4.091017in}{2.483978in}}%
\pgfpathlineto{\pgfqpoint{4.091862in}{2.483960in}}%
\pgfpathlineto{\pgfqpoint{4.095245in}{1.743809in}}%
\pgfpathlineto{\pgfqpoint{4.096091in}{2.378225in}}%
\pgfpathlineto{\pgfqpoint{4.096936in}{2.272485in}}%
\pgfpathlineto{\pgfqpoint{4.097782in}{1.320870in}}%
\pgfpathlineto{\pgfqpoint{4.100319in}{2.695464in}}%
\pgfpathlineto{\pgfqpoint{4.101164in}{1.636858in}}%
\pgfpathlineto{\pgfqpoint{4.102010in}{2.060955in}}%
\pgfpathlineto{\pgfqpoint{4.102856in}{2.061023in}}%
\pgfpathlineto{\pgfqpoint{4.104547in}{1.426609in}}%
\pgfpathlineto{\pgfqpoint{4.107084in}{2.483970in}}%
\pgfpathlineto{\pgfqpoint{4.109621in}{1.743825in}}%
\pgfpathlineto{\pgfqpoint{4.114694in}{2.378079in}}%
\pgfpathlineto{\pgfqpoint{4.115540in}{1.954831in}}%
\pgfpathlineto{\pgfqpoint{4.116386in}{2.166737in}}%
\pgfpathlineto{\pgfqpoint{4.117231in}{2.695408in}}%
\pgfpathlineto{\pgfqpoint{4.118077in}{2.272047in}}%
\pgfpathlineto{\pgfqpoint{4.118922in}{2.483925in}}%
\pgfpathlineto{\pgfqpoint{4.121459in}{1.092729in}}%
\pgfpathlineto{\pgfqpoint{4.122305in}{2.113888in}}%
\pgfpathlineto{\pgfqpoint{4.123151in}{1.585206in}}%
\pgfpathlineto{\pgfqpoint{4.124842in}{2.113891in}}%
\pgfpathlineto{\pgfqpoint{4.127379in}{1.056524in}}%
\pgfpathlineto{\pgfqpoint{4.129916in}{1.690945in}}%
\pgfpathlineto{\pgfqpoint{4.131607in}{2.558757in}}%
\pgfpathlineto{\pgfqpoint{4.132452in}{2.219627in}}%
\pgfpathlineto{\pgfqpoint{4.133298in}{1.690948in}}%
\pgfpathlineto{\pgfqpoint{4.134989in}{2.536837in}}%
\pgfpathlineto{\pgfqpoint{4.137526in}{2.008153in}}%
\pgfpathlineto{\pgfqpoint{4.139217in}{2.325296in}}%
\pgfpathlineto{\pgfqpoint{4.140063in}{2.536834in}}%
\pgfpathlineto{\pgfqpoint{4.143446in}{1.796660in}}%
\pgfpathlineto{\pgfqpoint{4.144291in}{2.431100in}}%
\pgfpathlineto{\pgfqpoint{4.145137in}{1.268110in}}%
\pgfpathlineto{\pgfqpoint{4.145982in}{2.219627in}}%
\pgfpathlineto{\pgfqpoint{4.147674in}{1.056524in}}%
\pgfpathlineto{\pgfqpoint{4.148519in}{2.431102in}}%
\pgfpathlineto{\pgfqpoint{4.149365in}{2.113926in}}%
\pgfpathlineto{\pgfqpoint{4.150210in}{2.219631in}}%
\pgfpathlineto{\pgfqpoint{4.151056in}{2.113898in}}%
\pgfpathlineto{\pgfqpoint{4.151902in}{1.267999in}}%
\pgfpathlineto{\pgfqpoint{4.153593in}{2.431094in}}%
\pgfpathlineto{\pgfqpoint{4.154439in}{1.479475in}}%
\pgfpathlineto{\pgfqpoint{4.155284in}{1.902448in}}%
\pgfpathlineto{\pgfqpoint{4.156130in}{1.690960in}}%
\pgfpathlineto{\pgfqpoint{4.157821in}{2.536819in}}%
\pgfpathlineto{\pgfqpoint{4.160358in}{1.796679in}}%
\pgfpathlineto{\pgfqpoint{4.162049in}{2.113880in}}%
\pgfpathlineto{\pgfqpoint{4.162895in}{1.056549in}}%
\pgfpathlineto{\pgfqpoint{4.163740in}{1.902417in}}%
\pgfpathlineto{\pgfqpoint{4.164586in}{2.536855in}}%
\pgfpathlineto{\pgfqpoint{4.165432in}{2.113905in}}%
\pgfpathlineto{\pgfqpoint{4.166277in}{1.796684in}}%
\pgfpathlineto{\pgfqpoint{4.167123in}{2.219625in}}%
\pgfpathlineto{\pgfqpoint{4.169660in}{1.162261in}}%
\pgfpathlineto{\pgfqpoint{4.172197in}{2.325356in}}%
\pgfpathlineto{\pgfqpoint{4.173042in}{1.796679in}}%
\pgfpathlineto{\pgfqpoint{4.173888in}{2.219628in}}%
\pgfpathlineto{\pgfqpoint{4.174734in}{2.113888in}}%
\pgfpathlineto{\pgfqpoint{4.177270in}{1.585213in}}%
\pgfpathlineto{\pgfqpoint{4.178962in}{2.219535in}}%
\pgfpathlineto{\pgfqpoint{4.179807in}{1.479488in}}%
\pgfpathlineto{\pgfqpoint{4.180653in}{1.585208in}}%
\pgfpathlineto{\pgfqpoint{4.181499in}{2.536839in}}%
\pgfpathlineto{\pgfqpoint{4.182344in}{1.690887in}}%
\pgfpathlineto{\pgfqpoint{4.183190in}{1.902414in}}%
\pgfpathlineto{\pgfqpoint{4.184035in}{1.585218in}}%
\pgfpathlineto{\pgfqpoint{4.184881in}{1.796681in}}%
\pgfpathlineto{\pgfqpoint{4.185727in}{1.585203in}}%
\pgfpathlineto{\pgfqpoint{4.188264in}{2.325506in}}%
\pgfpathlineto{\pgfqpoint{4.189109in}{1.585226in}}%
\pgfpathlineto{\pgfqpoint{4.189955in}{2.008160in}}%
\pgfpathlineto{\pgfqpoint{4.190800in}{2.642578in}}%
\pgfpathlineto{\pgfqpoint{4.191646in}{1.902384in}}%
\pgfpathlineto{\pgfqpoint{4.192492in}{2.219622in}}%
\pgfpathlineto{\pgfqpoint{4.193337in}{1.902488in}}%
\pgfpathlineto{\pgfqpoint{4.194183in}{2.113890in}}%
\pgfpathlineto{\pgfqpoint{4.195874in}{1.690939in}}%
\pgfpathlineto{\pgfqpoint{4.196720in}{1.796681in}}%
\pgfpathlineto{\pgfqpoint{4.197565in}{2.325366in}}%
\pgfpathlineto{\pgfqpoint{4.198411in}{1.902344in}}%
\pgfpathlineto{\pgfqpoint{4.199257in}{1.267997in}}%
\pgfpathlineto{\pgfqpoint{4.200102in}{2.160802in}}%
\pgfpathlineto{\pgfqpoint{4.200948in}{2.113892in}}%
\pgfpathlineto{\pgfqpoint{4.202639in}{1.796686in}}%
\pgfpathlineto{\pgfqpoint{4.203485in}{2.536769in}}%
\pgfpathlineto{\pgfqpoint{4.205176in}{1.902395in}}%
\pgfpathlineto{\pgfqpoint{4.206022in}{1.902419in}}%
\pgfpathlineto{\pgfqpoint{4.206867in}{2.325362in}}%
\pgfpathlineto{\pgfqpoint{4.207713in}{2.218208in}}%
\pgfpathlineto{\pgfqpoint{4.208559in}{2.220011in}}%
\pgfpathlineto{\pgfqpoint{4.211095in}{1.585207in}}%
\pgfpathlineto{\pgfqpoint{4.212787in}{2.431097in}}%
\pgfpathlineto{\pgfqpoint{4.215324in}{1.585227in}}%
\pgfpathlineto{\pgfqpoint{4.216169in}{2.219630in}}%
\pgfpathlineto{\pgfqpoint{4.217015in}{1.585213in}}%
\pgfpathlineto{\pgfqpoint{4.217860in}{2.536844in}}%
\pgfpathlineto{\pgfqpoint{4.218706in}{2.219614in}}%
\pgfpathlineto{\pgfqpoint{4.220397in}{2.642496in}}%
\pgfpathlineto{\pgfqpoint{4.221243in}{1.585215in}}%
\pgfpathlineto{\pgfqpoint{4.222089in}{2.642548in}}%
\pgfpathlineto{\pgfqpoint{4.222934in}{1.585208in}}%
\pgfpathlineto{\pgfqpoint{4.224625in}{2.536826in}}%
\pgfpathlineto{\pgfqpoint{4.226317in}{1.478903in}}%
\pgfpathlineto{\pgfqpoint{4.227162in}{1.796687in}}%
\pgfpathlineto{\pgfqpoint{4.228853in}{2.431101in}}%
\pgfpathlineto{\pgfqpoint{4.229699in}{2.113889in}}%
\pgfpathlineto{\pgfqpoint{4.230545in}{2.642549in}}%
\pgfpathlineto{\pgfqpoint{4.232236in}{1.796684in}}%
\pgfpathlineto{\pgfqpoint{4.235618in}{2.959787in}}%
\pgfpathlineto{\pgfqpoint{4.238155in}{1.374063in}}%
\pgfpathlineto{\pgfqpoint{4.239001in}{1.796727in}}%
\pgfpathlineto{\pgfqpoint{4.239847in}{2.113770in}}%
\pgfpathlineto{\pgfqpoint{4.241538in}{1.359588in}}%
\pgfpathlineto{\pgfqpoint{4.244075in}{2.420036in}}%
\pgfpathlineto{\pgfqpoint{4.245766in}{1.479427in}}%
\pgfpathlineto{\pgfqpoint{4.248303in}{3.025833in}}%
\pgfpathlineto{\pgfqpoint{4.249994in}{1.638244in}}%
\pgfpathlineto{\pgfqpoint{4.250840in}{2.800986in}}%
\pgfpathlineto{\pgfqpoint{4.251685in}{2.166704in}}%
\pgfpathlineto{\pgfqpoint{4.253377in}{2.589714in}}%
\pgfpathlineto{\pgfqpoint{4.254222in}{1.532339in}}%
\pgfpathlineto{\pgfqpoint{4.255068in}{1.849557in}}%
\pgfpathlineto{\pgfqpoint{4.255913in}{2.166729in}}%
\pgfpathlineto{\pgfqpoint{4.257605in}{1.532942in}}%
\pgfpathlineto{\pgfqpoint{4.260142in}{3.329823in}}%
\pgfpathlineto{\pgfqpoint{4.261833in}{1.426597in}}%
\pgfpathlineto{\pgfqpoint{4.263524in}{1.533931in}}%
\pgfpathlineto{\pgfqpoint{4.264370in}{2.484466in}}%
\pgfpathlineto{\pgfqpoint{4.265215in}{1.638106in}}%
\pgfpathlineto{\pgfqpoint{4.266061in}{2.378232in}}%
\pgfpathlineto{\pgfqpoint{4.266907in}{1.849546in}}%
\pgfpathlineto{\pgfqpoint{4.267752in}{2.166763in}}%
\pgfpathlineto{\pgfqpoint{4.270289in}{1.532336in}}%
\pgfpathlineto{\pgfqpoint{4.273672in}{1.955493in}}%
\pgfpathlineto{\pgfqpoint{4.274517in}{1.743794in}}%
\pgfpathlineto{\pgfqpoint{4.275363in}{2.378259in}}%
\pgfpathlineto{\pgfqpoint{4.276208in}{1.532346in}}%
\pgfpathlineto{\pgfqpoint{4.277054in}{1.849547in}}%
\pgfpathlineto{\pgfqpoint{4.278745in}{2.589701in}}%
\pgfpathlineto{\pgfqpoint{4.280437in}{1.426603in}}%
\pgfpathlineto{\pgfqpoint{4.281282in}{2.906953in}}%
\pgfpathlineto{\pgfqpoint{4.282128in}{1.109569in}}%
\pgfpathlineto{\pgfqpoint{4.282973in}{2.483986in}}%
\pgfpathlineto{\pgfqpoint{4.283819in}{1.955288in}}%
\pgfpathlineto{\pgfqpoint{4.284665in}{2.272434in}}%
\pgfpathlineto{\pgfqpoint{4.285510in}{2.695397in}}%
\pgfpathlineto{\pgfqpoint{4.287202in}{1.638082in}}%
\pgfpathlineto{\pgfqpoint{4.288893in}{2.483769in}}%
\pgfpathlineto{\pgfqpoint{4.290584in}{1.532362in}}%
\pgfpathlineto{\pgfqpoint{4.291430in}{2.483972in}}%
\pgfpathlineto{\pgfqpoint{4.292275in}{2.272467in}}%
\pgfpathlineto{\pgfqpoint{4.293121in}{2.378249in}}%
\pgfpathlineto{\pgfqpoint{4.294812in}{2.061113in}}%
\pgfpathlineto{\pgfqpoint{4.295658in}{2.061264in}}%
\pgfpathlineto{\pgfqpoint{4.296503in}{2.484004in}}%
\pgfpathlineto{\pgfqpoint{4.298195in}{1.532370in}}%
\pgfpathlineto{\pgfqpoint{4.299886in}{1.849690in}}%
\pgfpathlineto{\pgfqpoint{4.300732in}{1.638079in}}%
\pgfpathlineto{\pgfqpoint{4.301577in}{2.061021in}}%
\pgfpathlineto{\pgfqpoint{4.304114in}{1.320945in}}%
\pgfpathlineto{\pgfqpoint{4.305805in}{2.272497in}}%
\pgfpathlineto{\pgfqpoint{4.306651in}{1.638252in}}%
\pgfpathlineto{\pgfqpoint{4.308342in}{2.272497in}}%
\pgfpathlineto{\pgfqpoint{4.309188in}{2.166754in}}%
\pgfpathlineto{\pgfqpoint{4.310033in}{1.109391in}}%
\pgfpathlineto{\pgfqpoint{4.310879in}{2.397937in}}%
\pgfpathlineto{\pgfqpoint{4.311725in}{2.166766in}}%
\pgfpathlineto{\pgfqpoint{4.312570in}{2.166753in}}%
\pgfpathlineto{\pgfqpoint{4.313416in}{1.849531in}}%
\pgfpathlineto{\pgfqpoint{4.314261in}{1.955314in}}%
\pgfpathlineto{\pgfqpoint{4.315107in}{2.906904in}}%
\pgfpathlineto{\pgfqpoint{4.317644in}{1.638075in}}%
\pgfpathlineto{\pgfqpoint{4.319335in}{2.273191in}}%
\pgfpathlineto{\pgfqpoint{4.320181in}{2.272498in}}%
\pgfpathlineto{\pgfqpoint{4.321872in}{1.743814in}}%
\pgfpathlineto{\pgfqpoint{4.323563in}{2.378221in}}%
\pgfpathlineto{\pgfqpoint{4.324409in}{1.532342in}}%
\pgfpathlineto{\pgfqpoint{4.325255in}{2.165938in}}%
\pgfpathlineto{\pgfqpoint{4.326100in}{1.532339in}}%
\pgfpathlineto{\pgfqpoint{4.328637in}{2.483969in}}%
\pgfpathlineto{\pgfqpoint{4.329483in}{2.589704in}}%
\pgfpathlineto{\pgfqpoint{4.330328in}{2.061014in}}%
\pgfpathlineto{\pgfqpoint{4.331174in}{3.012651in}}%
\pgfpathlineto{\pgfqpoint{4.332020in}{2.483955in}}%
\pgfpathlineto{\pgfqpoint{4.332865in}{1.426604in}}%
\pgfpathlineto{\pgfqpoint{4.333711in}{2.378232in}}%
\pgfpathlineto{\pgfqpoint{4.334556in}{1.743814in}}%
\pgfpathlineto{\pgfqpoint{4.336248in}{1.955284in}}%
\pgfpathlineto{\pgfqpoint{4.337093in}{1.850485in}}%
\pgfpathlineto{\pgfqpoint{4.337939in}{2.378248in}}%
\pgfpathlineto{\pgfqpoint{4.338785in}{1.955299in}}%
\pgfpathlineto{\pgfqpoint{4.341321in}{2.378234in}}%
\pgfpathlineto{\pgfqpoint{4.342167in}{2.166758in}}%
\pgfpathlineto{\pgfqpoint{4.343013in}{2.483966in}}%
\pgfpathlineto{\pgfqpoint{4.344704in}{1.849533in}}%
\pgfpathlineto{\pgfqpoint{4.345550in}{2.166760in}}%
\pgfpathlineto{\pgfqpoint{4.346395in}{1.743814in}}%
\pgfpathlineto{\pgfqpoint{4.348086in}{2.378225in}}%
\pgfpathlineto{\pgfqpoint{4.348932in}{1.849551in}}%
\pgfpathlineto{\pgfqpoint{4.349778in}{2.166776in}}%
\pgfpathlineto{\pgfqpoint{4.350623in}{2.272497in}}%
\pgfpathlineto{\pgfqpoint{4.351469in}{2.166711in}}%
\pgfpathlineto{\pgfqpoint{4.352315in}{1.743799in}}%
\pgfpathlineto{\pgfqpoint{4.353160in}{1.849547in}}%
\pgfpathlineto{\pgfqpoint{4.354006in}{2.378232in}}%
\pgfpathlineto{\pgfqpoint{4.354851in}{1.426604in}}%
\pgfpathlineto{\pgfqpoint{4.355697in}{1.849543in}}%
\pgfpathlineto{\pgfqpoint{4.356543in}{2.166806in}}%
\pgfpathlineto{\pgfqpoint{4.357388in}{1.955206in}}%
\pgfpathlineto{\pgfqpoint{4.358234in}{1.849540in}}%
\pgfpathlineto{\pgfqpoint{4.359080in}{2.483982in}}%
\pgfpathlineto{\pgfqpoint{4.359925in}{2.483940in}}%
\pgfpathlineto{\pgfqpoint{4.360771in}{1.426622in}}%
\pgfpathlineto{\pgfqpoint{4.361616in}{2.906914in}}%
\pgfpathlineto{\pgfqpoint{4.362462in}{2.484127in}}%
\pgfpathlineto{\pgfqpoint{4.363308in}{1.743818in}}%
\pgfpathlineto{\pgfqpoint{4.364153in}{2.378251in}}%
\pgfpathlineto{\pgfqpoint{4.364999in}{2.272476in}}%
\pgfpathlineto{\pgfqpoint{4.365845in}{1.109393in}}%
\pgfpathlineto{\pgfqpoint{4.366690in}{1.320432in}}%
\pgfpathlineto{\pgfqpoint{4.367536in}{1.426598in}}%
\pgfpathlineto{\pgfqpoint{4.368381in}{2.483928in}}%
\pgfpathlineto{\pgfqpoint{4.369227in}{1.426595in}}%
\pgfpathlineto{\pgfqpoint{4.370073in}{1.743817in}}%
\pgfpathlineto{\pgfqpoint{4.372610in}{2.906900in}}%
\pgfpathlineto{\pgfqpoint{4.375146in}{1.109394in}}%
\pgfpathlineto{\pgfqpoint{4.375992in}{2.061026in}}%
\pgfpathlineto{\pgfqpoint{4.376838in}{2.060901in}}%
\pgfpathlineto{\pgfqpoint{4.377683in}{2.166759in}}%
\pgfpathlineto{\pgfqpoint{4.378529in}{1.637926in}}%
\pgfpathlineto{\pgfqpoint{4.379375in}{1.743874in}}%
\pgfpathlineto{\pgfqpoint{4.381066in}{2.378232in}}%
\pgfpathlineto{\pgfqpoint{4.381911in}{1.638085in}}%
\pgfpathlineto{\pgfqpoint{4.382757in}{2.378220in}}%
\pgfpathlineto{\pgfqpoint{4.384448in}{1.532348in}}%
\pgfpathlineto{\pgfqpoint{4.385294in}{2.589708in}}%
\pgfpathlineto{\pgfqpoint{4.386140in}{1.955283in}}%
\pgfpathlineto{\pgfqpoint{4.386985in}{1.320851in}}%
\pgfpathlineto{\pgfqpoint{4.387831in}{2.589707in}}%
\pgfpathlineto{\pgfqpoint{4.388676in}{1.321577in}}%
\pgfpathlineto{\pgfqpoint{4.389522in}{2.272489in}}%
\pgfpathlineto{\pgfqpoint{4.390368in}{2.061010in}}%
\pgfpathlineto{\pgfqpoint{4.391213in}{2.272472in}}%
\pgfpathlineto{\pgfqpoint{4.392059in}{1.638077in}}%
\pgfpathlineto{\pgfqpoint{4.392904in}{1.638080in}}%
\pgfpathlineto{\pgfqpoint{4.394596in}{2.378233in}}%
\pgfpathlineto{\pgfqpoint{4.396287in}{1.320856in}}%
\pgfpathlineto{\pgfqpoint{4.398824in}{2.378238in}}%
\pgfpathlineto{\pgfqpoint{4.399669in}{1.533049in}}%
\pgfpathlineto{\pgfqpoint{4.400515in}{2.378243in}}%
\pgfpathlineto{\pgfqpoint{4.401361in}{1.955286in}}%
\pgfpathlineto{\pgfqpoint{4.402206in}{2.378227in}}%
\pgfpathlineto{\pgfqpoint{4.403052in}{1.955286in}}%
\pgfpathlineto{\pgfqpoint{4.403898in}{2.272495in}}%
\pgfpathlineto{\pgfqpoint{4.404743in}{1.955277in}}%
\pgfpathlineto{\pgfqpoint{4.405589in}{1.955291in}}%
\pgfpathlineto{\pgfqpoint{4.407280in}{2.272491in}}%
\pgfpathlineto{\pgfqpoint{4.408126in}{1.638083in}}%
\pgfpathlineto{\pgfqpoint{4.408971in}{2.589669in}}%
\pgfpathlineto{\pgfqpoint{4.409817in}{2.272490in}}%
\pgfpathlineto{\pgfqpoint{4.411508in}{1.215129in}}%
\pgfpathlineto{\pgfqpoint{4.413199in}{2.378233in}}%
\pgfpathlineto{\pgfqpoint{4.414045in}{2.376671in}}%
\pgfpathlineto{\pgfqpoint{4.414891in}{1.849551in}}%
\pgfpathlineto{\pgfqpoint{4.415736in}{2.378234in}}%
\pgfpathlineto{\pgfqpoint{4.417428in}{1.638078in}}%
\pgfpathlineto{\pgfqpoint{4.418273in}{1.955287in}}%
\pgfpathlineto{\pgfqpoint{4.419119in}{1.109389in}}%
\pgfpathlineto{\pgfqpoint{4.420810in}{2.378243in}}%
\pgfpathlineto{\pgfqpoint{4.421656in}{2.061055in}}%
\pgfpathlineto{\pgfqpoint{4.422501in}{2.378159in}}%
\pgfpathlineto{\pgfqpoint{4.423347in}{1.215118in}}%
\pgfpathlineto{\pgfqpoint{4.424193in}{1.955258in}}%
\pgfpathlineto{\pgfqpoint{4.425884in}{2.166563in}}%
\pgfpathlineto{\pgfqpoint{4.426729in}{1.743831in}}%
\pgfpathlineto{\pgfqpoint{4.427575in}{2.589714in}}%
\pgfpathlineto{\pgfqpoint{4.428421in}{1.638081in}}%
\pgfpathlineto{\pgfqpoint{4.429266in}{1.955282in}}%
\pgfpathlineto{\pgfqpoint{4.430958in}{1.743823in}}%
\pgfpathlineto{\pgfqpoint{4.434340in}{2.164311in}}%
\pgfpathlineto{\pgfqpoint{4.435186in}{1.532354in}}%
\pgfpathlineto{\pgfqpoint{4.436031in}{1.849577in}}%
\pgfpathlineto{\pgfqpoint{4.436877in}{1.743816in}}%
\pgfpathlineto{\pgfqpoint{4.437723in}{0.897941in}}%
\pgfpathlineto{\pgfqpoint{4.438568in}{2.061011in}}%
\pgfpathlineto{\pgfqpoint{4.439414in}{1.955286in}}%
\pgfpathlineto{\pgfqpoint{4.440259in}{1.849555in}}%
\pgfpathlineto{\pgfqpoint{4.441105in}{1.532406in}}%
\pgfpathlineto{\pgfqpoint{4.442796in}{2.378241in}}%
\pgfpathlineto{\pgfqpoint{4.443642in}{2.272515in}}%
\pgfpathlineto{\pgfqpoint{4.444488in}{1.955237in}}%
\pgfpathlineto{\pgfqpoint{4.445333in}{2.589697in}}%
\pgfpathlineto{\pgfqpoint{4.446179in}{1.003675in}}%
\pgfpathlineto{\pgfqpoint{4.447024in}{2.166757in}}%
\pgfpathlineto{\pgfqpoint{4.447870in}{1.955455in}}%
\pgfpathlineto{\pgfqpoint{4.448716in}{2.695435in}}%
\pgfpathlineto{\pgfqpoint{4.450407in}{1.426729in}}%
\pgfpathlineto{\pgfqpoint{4.452944in}{2.272435in}}%
\pgfpathlineto{\pgfqpoint{4.454635in}{1.955181in}}%
\pgfpathlineto{\pgfqpoint{4.455481in}{1.955290in}}%
\pgfpathlineto{\pgfqpoint{4.456326in}{1.638130in}}%
\pgfpathlineto{\pgfqpoint{4.457172in}{1.743859in}}%
\pgfpathlineto{\pgfqpoint{4.458018in}{2.589667in}}%
\pgfpathlineto{\pgfqpoint{4.458863in}{2.269960in}}%
\pgfpathlineto{\pgfqpoint{4.459709in}{1.532404in}}%
\pgfpathlineto{\pgfqpoint{4.460554in}{2.166493in}}%
\pgfpathlineto{\pgfqpoint{4.461400in}{1.638108in}}%
\pgfpathlineto{\pgfqpoint{4.464783in}{3.223119in}}%
\pgfpathlineto{\pgfqpoint{4.465628in}{1.849536in}}%
\pgfpathlineto{\pgfqpoint{4.466474in}{1.849562in}}%
\pgfpathlineto{\pgfqpoint{4.468165in}{2.483979in}}%
\pgfpathlineto{\pgfqpoint{4.469011in}{1.743827in}}%
\pgfpathlineto{\pgfqpoint{4.469856in}{1.955243in}}%
\pgfpathlineto{\pgfqpoint{4.472393in}{2.483964in}}%
\pgfpathlineto{\pgfqpoint{4.473239in}{1.320879in}}%
\pgfpathlineto{\pgfqpoint{4.475776in}{2.801181in}}%
\pgfpathlineto{\pgfqpoint{4.477467in}{1.109432in}}%
\pgfpathlineto{\pgfqpoint{4.478312in}{2.272093in}}%
\pgfpathlineto{\pgfqpoint{4.479158in}{1.956287in}}%
\pgfpathlineto{\pgfqpoint{4.480849in}{1.320848in}}%
\pgfpathlineto{\pgfqpoint{4.482541in}{2.166763in}}%
\pgfpathlineto{\pgfqpoint{4.483386in}{1.532347in}}%
\pgfpathlineto{\pgfqpoint{4.485077in}{3.329813in}}%
\pgfpathlineto{\pgfqpoint{4.488460in}{1.638045in}}%
\pgfpathlineto{\pgfqpoint{4.490151in}{2.272877in}}%
\pgfpathlineto{\pgfqpoint{4.490997in}{1.109333in}}%
\pgfpathlineto{\pgfqpoint{4.491842in}{2.060947in}}%
\pgfpathlineto{\pgfqpoint{4.492688in}{1.849558in}}%
\pgfpathlineto{\pgfqpoint{4.493534in}{2.272500in}}%
\pgfpathlineto{\pgfqpoint{4.494379in}{1.743785in}}%
\pgfpathlineto{\pgfqpoint{4.495225in}{2.378350in}}%
\pgfpathlineto{\pgfqpoint{4.496916in}{1.215140in}}%
\pgfpathlineto{\pgfqpoint{4.498607in}{1.955293in}}%
\pgfpathlineto{\pgfqpoint{4.499453in}{1.743613in}}%
\pgfpathlineto{\pgfqpoint{4.500299in}{1.955287in}}%
\pgfpathlineto{\pgfqpoint{4.501144in}{3.224115in}}%
\pgfpathlineto{\pgfqpoint{4.501990in}{2.272522in}}%
\pgfpathlineto{\pgfqpoint{4.502836in}{2.483953in}}%
\pgfpathlineto{\pgfqpoint{4.503681in}{2.060998in}}%
\pgfpathlineto{\pgfqpoint{4.504527in}{2.272512in}}%
\pgfpathlineto{\pgfqpoint{4.505372in}{2.272498in}}%
\pgfpathlineto{\pgfqpoint{4.506218in}{2.378231in}}%
\pgfpathlineto{\pgfqpoint{4.507064in}{1.849584in}}%
\pgfpathlineto{\pgfqpoint{4.507909in}{2.483970in}}%
\pgfpathlineto{\pgfqpoint{4.510446in}{1.003651in}}%
\pgfpathlineto{\pgfqpoint{4.511292in}{2.061022in}}%
\pgfpathlineto{\pgfqpoint{4.512137in}{1.743919in}}%
\pgfpathlineto{\pgfqpoint{4.514674in}{2.378230in}}%
\pgfpathlineto{\pgfqpoint{4.515520in}{2.166930in}}%
\pgfpathlineto{\pgfqpoint{4.516366in}{1.532333in}}%
\pgfpathlineto{\pgfqpoint{4.517211in}{2.272479in}}%
\pgfpathlineto{\pgfqpoint{4.518057in}{2.272419in}}%
\pgfpathlineto{\pgfqpoint{4.520594in}{2.272497in}}%
\pgfpathlineto{\pgfqpoint{4.521439in}{1.532294in}}%
\pgfpathlineto{\pgfqpoint{4.522285in}{1.532338in}}%
\pgfpathlineto{\pgfqpoint{4.523131in}{2.589829in}}%
\pgfpathlineto{\pgfqpoint{4.523976in}{2.060954in}}%
\pgfpathlineto{\pgfqpoint{4.524822in}{2.378227in}}%
\pgfpathlineto{\pgfqpoint{4.527359in}{1.638086in}}%
\pgfpathlineto{\pgfqpoint{4.528204in}{2.378219in}}%
\pgfpathlineto{\pgfqpoint{4.529050in}{1.320792in}}%
\pgfpathlineto{\pgfqpoint{4.529896in}{2.061005in}}%
\pgfpathlineto{\pgfqpoint{4.530741in}{1.637997in}}%
\pgfpathlineto{\pgfqpoint{4.531587in}{2.272459in}}%
\pgfpathlineto{\pgfqpoint{4.532432in}{1.955296in}}%
\pgfpathlineto{\pgfqpoint{4.534124in}{2.907064in}}%
\pgfpathlineto{\pgfqpoint{4.534969in}{1.955416in}}%
\pgfpathlineto{\pgfqpoint{4.535815in}{2.272542in}}%
\pgfpathlineto{\pgfqpoint{4.536661in}{2.378227in}}%
\pgfpathlineto{\pgfqpoint{4.537506in}{1.532339in}}%
\pgfpathlineto{\pgfqpoint{4.538352in}{1.955327in}}%
\pgfpathlineto{\pgfqpoint{4.539197in}{2.378213in}}%
\pgfpathlineto{\pgfqpoint{4.540889in}{1.426616in}}%
\pgfpathlineto{\pgfqpoint{4.543426in}{2.483890in}}%
\pgfpathlineto{\pgfqpoint{4.544271in}{2.060983in}}%
\pgfpathlineto{\pgfqpoint{4.545117in}{2.483280in}}%
\pgfpathlineto{\pgfqpoint{4.545962in}{1.320852in}}%
\pgfpathlineto{\pgfqpoint{4.546808in}{2.061018in}}%
\pgfpathlineto{\pgfqpoint{4.549345in}{2.483959in}}%
\pgfpathlineto{\pgfqpoint{4.551882in}{1.426633in}}%
\pgfpathlineto{\pgfqpoint{4.553573in}{2.378288in}}%
\pgfpathlineto{\pgfqpoint{4.554419in}{2.378231in}}%
\pgfpathlineto{\pgfqpoint{4.555264in}{2.378752in}}%
\pgfpathlineto{\pgfqpoint{4.556110in}{1.320992in}}%
\pgfpathlineto{\pgfqpoint{4.556955in}{2.589689in}}%
\pgfpathlineto{\pgfqpoint{4.557801in}{1.955280in}}%
\pgfpathlineto{\pgfqpoint{4.558647in}{2.695418in}}%
\pgfpathlineto{\pgfqpoint{4.560338in}{2.060707in}}%
\pgfpathlineto{\pgfqpoint{4.561184in}{2.061084in}}%
\pgfpathlineto{\pgfqpoint{4.563720in}{1.426641in}}%
\pgfpathlineto{\pgfqpoint{4.564566in}{1.532335in}}%
\pgfpathlineto{\pgfqpoint{4.565412in}{1.956357in}}%
\pgfpathlineto{\pgfqpoint{4.566257in}{1.955265in}}%
\pgfpathlineto{\pgfqpoint{4.567949in}{1.532283in}}%
\pgfpathlineto{\pgfqpoint{4.569640in}{2.166745in}}%
\pgfpathlineto{\pgfqpoint{4.570485in}{1.638084in}}%
\pgfpathlineto{\pgfqpoint{4.571331in}{2.060991in}}%
\pgfpathlineto{\pgfqpoint{4.572177in}{1.638078in}}%
\pgfpathlineto{\pgfqpoint{4.573868in}{2.165739in}}%
\pgfpathlineto{\pgfqpoint{4.574714in}{2.061009in}}%
\pgfpathlineto{\pgfqpoint{4.576405in}{2.483873in}}%
\pgfpathlineto{\pgfqpoint{4.578096in}{1.532381in}}%
\pgfpathlineto{\pgfqpoint{4.578942in}{1.846125in}}%
\pgfpathlineto{\pgfqpoint{4.579787in}{1.852800in}}%
\pgfpathlineto{\pgfqpoint{4.580633in}{2.906975in}}%
\pgfpathlineto{\pgfqpoint{4.581479in}{1.745186in}}%
\pgfpathlineto{\pgfqpoint{4.582324in}{2.166830in}}%
\pgfpathlineto{\pgfqpoint{4.583170in}{2.058902in}}%
\pgfpathlineto{\pgfqpoint{4.584015in}{1.638196in}}%
\pgfpathlineto{\pgfqpoint{4.584861in}{2.377052in}}%
\pgfpathlineto{\pgfqpoint{4.585707in}{1.955322in}}%
\pgfpathlineto{\pgfqpoint{4.586552in}{1.743499in}}%
\pgfpathlineto{\pgfqpoint{4.588244in}{2.490183in}}%
\pgfpathlineto{\pgfqpoint{4.589089in}{2.484160in}}%
\pgfpathlineto{\pgfqpoint{4.589935in}{1.426617in}}%
\pgfpathlineto{\pgfqpoint{4.590780in}{1.638094in}}%
\pgfpathlineto{\pgfqpoint{4.592472in}{2.166745in}}%
\pgfpathlineto{\pgfqpoint{4.593317in}{2.062005in}}%
\pgfpathlineto{\pgfqpoint{4.595854in}{1.426147in}}%
\pgfpathlineto{\pgfqpoint{4.597545in}{2.061017in}}%
\pgfpathlineto{\pgfqpoint{4.598391in}{1.743859in}}%
\pgfpathlineto{\pgfqpoint{4.599237in}{2.378112in}}%
\pgfpathlineto{\pgfqpoint{4.600082in}{1.849549in}}%
\pgfpathlineto{\pgfqpoint{4.600928in}{2.911636in}}%
\pgfpathlineto{\pgfqpoint{4.601774in}{1.849632in}}%
\pgfpathlineto{\pgfqpoint{4.602619in}{2.483989in}}%
\pgfpathlineto{\pgfqpoint{4.604310in}{1.743896in}}%
\pgfpathlineto{\pgfqpoint{4.605156in}{1.955317in}}%
\pgfpathlineto{\pgfqpoint{4.606002in}{1.743598in}}%
\pgfpathlineto{\pgfqpoint{4.606847in}{2.271822in}}%
\pgfpathlineto{\pgfqpoint{4.607693in}{2.166076in}}%
\pgfpathlineto{\pgfqpoint{4.609384in}{1.849517in}}%
\pgfpathlineto{\pgfqpoint{4.610230in}{2.166575in}}%
\pgfpathlineto{\pgfqpoint{4.611075in}{2.159649in}}%
\pgfpathlineto{\pgfqpoint{4.613612in}{1.531012in}}%
\pgfpathlineto{\pgfqpoint{4.614458in}{2.431023in}}%
\pgfpathlineto{\pgfqpoint{4.615304in}{1.713734in}}%
\pgfpathlineto{\pgfqpoint{4.616149in}{1.902449in}}%
\pgfpathlineto{\pgfqpoint{4.616995in}{1.710065in}}%
\pgfpathlineto{\pgfqpoint{4.617840in}{2.226287in}}%
\pgfpathlineto{\pgfqpoint{4.618686in}{2.219795in}}%
\pgfpathlineto{\pgfqpoint{4.620377in}{1.690905in}}%
\pgfpathlineto{\pgfqpoint{4.621223in}{2.748245in}}%
\pgfpathlineto{\pgfqpoint{4.622069in}{1.056559in}}%
\pgfpathlineto{\pgfqpoint{4.622914in}{2.113878in}}%
\pgfpathlineto{\pgfqpoint{4.623760in}{2.219634in}}%
\pgfpathlineto{\pgfqpoint{4.624605in}{1.690947in}}%
\pgfpathlineto{\pgfqpoint{4.625451in}{2.430897in}}%
\pgfpathlineto{\pgfqpoint{4.626297in}{2.113867in}}%
\pgfpathlineto{\pgfqpoint{4.627142in}{1.268057in}}%
\pgfpathlineto{\pgfqpoint{4.627988in}{2.007916in}}%
\pgfpathlineto{\pgfqpoint{4.628833in}{1.691032in}}%
\pgfpathlineto{\pgfqpoint{4.631370in}{2.325767in}}%
\pgfpathlineto{\pgfqpoint{4.632216in}{1.690955in}}%
\pgfpathlineto{\pgfqpoint{4.633062in}{2.325355in}}%
\pgfpathlineto{\pgfqpoint{4.633907in}{1.479491in}}%
\pgfpathlineto{\pgfqpoint{4.634753in}{2.325203in}}%
\pgfpathlineto{\pgfqpoint{4.635598in}{1.796625in}}%
\pgfpathlineto{\pgfqpoint{4.636444in}{1.585211in}}%
\pgfpathlineto{\pgfqpoint{4.638135in}{2.536494in}}%
\pgfpathlineto{\pgfqpoint{4.638981in}{2.431165in}}%
\pgfpathlineto{\pgfqpoint{4.641518in}{1.585227in}}%
\pgfpathlineto{\pgfqpoint{4.642363in}{1.902330in}}%
\pgfpathlineto{\pgfqpoint{4.643209in}{2.427097in}}%
\pgfpathlineto{\pgfqpoint{4.644055in}{2.008316in}}%
\pgfpathlineto{\pgfqpoint{4.644900in}{1.690865in}}%
\pgfpathlineto{\pgfqpoint{4.645746in}{1.691246in}}%
\pgfpathlineto{\pgfqpoint{4.646592in}{2.431051in}}%
\pgfpathlineto{\pgfqpoint{4.647437in}{2.012417in}}%
\pgfpathlineto{\pgfqpoint{4.648283in}{1.479504in}}%
\pgfpathlineto{\pgfqpoint{4.649128in}{2.536664in}}%
\pgfpathlineto{\pgfqpoint{4.649974in}{1.585816in}}%
\pgfpathlineto{\pgfqpoint{4.650820in}{2.527170in}}%
\pgfpathlineto{\pgfqpoint{4.651665in}{1.901681in}}%
\pgfpathlineto{\pgfqpoint{4.652511in}{2.317687in}}%
\pgfpathlineto{\pgfqpoint{4.653357in}{2.222645in}}%
\pgfpathlineto{\pgfqpoint{4.655048in}{1.698557in}}%
\pgfpathlineto{\pgfqpoint{4.657585in}{2.484004in}}%
\pgfpathlineto{\pgfqpoint{4.658430in}{2.272460in}}%
\pgfpathlineto{\pgfqpoint{4.659276in}{1.532253in}}%
\pgfpathlineto{\pgfqpoint{4.660122in}{2.483772in}}%
\pgfpathlineto{\pgfqpoint{4.660967in}{1.743844in}}%
\pgfpathlineto{\pgfqpoint{4.661813in}{2.166541in}}%
\pgfpathlineto{\pgfqpoint{4.662658in}{1.320925in}}%
\pgfpathlineto{\pgfqpoint{4.663504in}{1.955697in}}%
\pgfpathlineto{\pgfqpoint{4.664350in}{1.321825in}}%
\pgfpathlineto{\pgfqpoint{4.666041in}{2.701637in}}%
\pgfpathlineto{\pgfqpoint{4.667732in}{1.955369in}}%
\pgfpathlineto{\pgfqpoint{4.668578in}{2.061166in}}%
\pgfpathlineto{\pgfqpoint{4.669423in}{1.849578in}}%
\pgfpathlineto{\pgfqpoint{4.670269in}{1.955099in}}%
\pgfpathlineto{\pgfqpoint{4.671115in}{1.955296in}}%
\pgfpathlineto{\pgfqpoint{4.671960in}{2.060939in}}%
\pgfpathlineto{\pgfqpoint{4.672806in}{2.589748in}}%
\pgfpathlineto{\pgfqpoint{4.673652in}{1.215165in}}%
\pgfpathlineto{\pgfqpoint{4.674497in}{1.743796in}}%
\pgfpathlineto{\pgfqpoint{4.675343in}{2.166825in}}%
\pgfpathlineto{\pgfqpoint{4.676188in}{1.743822in}}%
\pgfpathlineto{\pgfqpoint{4.677034in}{1.955223in}}%
\pgfpathlineto{\pgfqpoint{4.677880in}{1.849559in}}%
\pgfpathlineto{\pgfqpoint{4.678725in}{2.060913in}}%
\pgfpathlineto{\pgfqpoint{4.679571in}{1.849632in}}%
\pgfpathlineto{\pgfqpoint{4.680417in}{2.061031in}}%
\pgfpathlineto{\pgfqpoint{4.681262in}{1.955424in}}%
\pgfpathlineto{\pgfqpoint{4.682108in}{1.983130in}}%
\pgfpathlineto{\pgfqpoint{4.682953in}{1.425900in}}%
\pgfpathlineto{\pgfqpoint{4.685490in}{2.801138in}}%
\pgfpathlineto{\pgfqpoint{4.686336in}{1.532786in}}%
\pgfpathlineto{\pgfqpoint{4.687182in}{1.955295in}}%
\pgfpathlineto{\pgfqpoint{4.688027in}{2.484043in}}%
\pgfpathlineto{\pgfqpoint{4.688873in}{2.166758in}}%
\pgfpathlineto{\pgfqpoint{4.689718in}{1.320888in}}%
\pgfpathlineto{\pgfqpoint{4.690564in}{1.955305in}}%
\pgfpathlineto{\pgfqpoint{4.691410in}{2.169916in}}%
\pgfpathlineto{\pgfqpoint{4.693101in}{1.320867in}}%
\pgfpathlineto{\pgfqpoint{4.693947in}{1.955703in}}%
\pgfpathlineto{\pgfqpoint{4.694792in}{1.849565in}}%
\pgfpathlineto{\pgfqpoint{4.698175in}{2.378246in}}%
\pgfpathlineto{\pgfqpoint{4.699866in}{1.955008in}}%
\pgfpathlineto{\pgfqpoint{4.700712in}{1.956179in}}%
\pgfpathlineto{\pgfqpoint{4.701557in}{2.801185in}}%
\pgfpathlineto{\pgfqpoint{4.703248in}{1.743841in}}%
\pgfpathlineto{\pgfqpoint{4.704094in}{1.849408in}}%
\pgfpathlineto{\pgfqpoint{4.704940in}{1.636811in}}%
\pgfpathlineto{\pgfqpoint{4.705785in}{2.483667in}}%
\pgfpathlineto{\pgfqpoint{4.706631in}{1.955260in}}%
\pgfpathlineto{\pgfqpoint{4.707476in}{2.378211in}}%
\pgfpathlineto{\pgfqpoint{4.708322in}{2.272451in}}%
\pgfpathlineto{\pgfqpoint{4.710859in}{2.272498in}}%
\pgfpathlineto{\pgfqpoint{4.711705in}{2.589049in}}%
\pgfpathlineto{\pgfqpoint{4.712550in}{2.167155in}}%
\pgfpathlineto{\pgfqpoint{4.713396in}{2.590174in}}%
\pgfpathlineto{\pgfqpoint{4.715933in}{1.322102in}}%
\pgfpathlineto{\pgfqpoint{4.718470in}{2.146644in}}%
\pgfpathlineto{\pgfqpoint{4.720161in}{2.060988in}}%
\pgfpathlineto{\pgfqpoint{4.721006in}{2.589641in}}%
\pgfpathlineto{\pgfqpoint{4.721852in}{2.272514in}}%
\pgfpathlineto{\pgfqpoint{4.722698in}{1.638801in}}%
\pgfpathlineto{\pgfqpoint{4.724389in}{2.483583in}}%
\pgfpathlineto{\pgfqpoint{4.725235in}{1.639655in}}%
\pgfpathlineto{\pgfqpoint{4.726080in}{1.999316in}}%
\pgfpathlineto{\pgfqpoint{4.726926in}{2.397902in}}%
\pgfpathlineto{\pgfqpoint{4.727771in}{1.639041in}}%
\pgfpathlineto{\pgfqpoint{4.728617in}{1.745156in}}%
\pgfpathlineto{\pgfqpoint{4.729463in}{1.531618in}}%
\pgfpathlineto{\pgfqpoint{4.731154in}{2.383848in}}%
\pgfpathlineto{\pgfqpoint{4.732000in}{1.955759in}}%
\pgfpathlineto{\pgfqpoint{4.733691in}{2.458694in}}%
\pgfpathlineto{\pgfqpoint{4.736228in}{1.881720in}}%
\pgfpathlineto{\pgfqpoint{4.737073in}{1.880388in}}%
\pgfpathlineto{\pgfqpoint{4.737919in}{1.542367in}}%
\pgfpathlineto{\pgfqpoint{4.740456in}{2.535421in}}%
\pgfpathlineto{\pgfqpoint{4.742147in}{1.709177in}}%
\pgfpathlineto{\pgfqpoint{4.742993in}{1.944182in}}%
\pgfpathlineto{\pgfqpoint{4.744684in}{2.433809in}}%
\pgfpathlineto{\pgfqpoint{4.745530in}{2.337629in}}%
\pgfpathlineto{\pgfqpoint{4.746375in}{2.219436in}}%
\pgfpathlineto{\pgfqpoint{4.747221in}{1.585309in}}%
\pgfpathlineto{\pgfqpoint{4.748066in}{2.642532in}}%
\pgfpathlineto{\pgfqpoint{4.749758in}{1.056599in}}%
\pgfpathlineto{\pgfqpoint{4.750603in}{2.642499in}}%
\pgfpathlineto{\pgfqpoint{4.751449in}{1.902457in}}%
\pgfpathlineto{\pgfqpoint{4.752295in}{2.325387in}}%
\pgfpathlineto{\pgfqpoint{4.753140in}{1.796693in}}%
\pgfpathlineto{\pgfqpoint{4.753986in}{1.902412in}}%
\pgfpathlineto{\pgfqpoint{4.754831in}{3.065410in}}%
\pgfpathlineto{\pgfqpoint{4.755677in}{1.796647in}}%
\pgfpathlineto{\pgfqpoint{4.756523in}{2.536936in}}%
\pgfpathlineto{\pgfqpoint{4.757368in}{2.431111in}}%
\pgfpathlineto{\pgfqpoint{4.759905in}{1.268391in}}%
\pgfpathlineto{\pgfqpoint{4.760751in}{1.585242in}}%
\pgfpathlineto{\pgfqpoint{4.762442in}{2.398607in}}%
\pgfpathlineto{\pgfqpoint{4.764979in}{1.585083in}}%
\pgfpathlineto{\pgfqpoint{4.765825in}{2.959791in}}%
\pgfpathlineto{\pgfqpoint{4.766670in}{1.902379in}}%
\pgfpathlineto{\pgfqpoint{4.768361in}{2.113736in}}%
\pgfpathlineto{\pgfqpoint{4.770053in}{2.536228in}}%
\pgfpathlineto{\pgfqpoint{4.770898in}{1.266742in}}%
\pgfpathlineto{\pgfqpoint{4.771744in}{2.380223in}}%
\pgfpathlineto{\pgfqpoint{4.772590in}{2.272393in}}%
\pgfpathlineto{\pgfqpoint{4.773435in}{1.590154in}}%
\pgfpathlineto{\pgfqpoint{4.774281in}{2.482054in}}%
\pgfpathlineto{\pgfqpoint{4.775126in}{2.113966in}}%
\pgfpathlineto{\pgfqpoint{4.775972in}{1.717837in}}%
\pgfpathlineto{\pgfqpoint{4.776818in}{2.509738in}}%
\pgfpathlineto{\pgfqpoint{4.777663in}{2.252200in}}%
\pgfpathlineto{\pgfqpoint{4.778509in}{1.215630in}}%
\pgfpathlineto{\pgfqpoint{4.779355in}{1.931573in}}%
\pgfpathlineto{\pgfqpoint{4.780200in}{2.008165in}}%
\pgfpathlineto{\pgfqpoint{4.781046in}{1.373846in}}%
\pgfpathlineto{\pgfqpoint{4.782737in}{2.113892in}}%
\pgfpathlineto{\pgfqpoint{4.783583in}{1.373730in}}%
\pgfpathlineto{\pgfqpoint{4.786119in}{2.642525in}}%
\pgfpathlineto{\pgfqpoint{4.786965in}{3.382727in}}%
\pgfpathlineto{\pgfqpoint{4.789502in}{2.219630in}}%
\pgfpathlineto{\pgfqpoint{4.792039in}{1.796685in}}%
\pgfpathlineto{\pgfqpoint{4.792884in}{2.642538in}}%
\pgfpathlineto{\pgfqpoint{4.793730in}{2.008071in}}%
\pgfpathlineto{\pgfqpoint{4.794576in}{2.113896in}}%
\pgfpathlineto{\pgfqpoint{4.795421in}{1.797077in}}%
\pgfpathlineto{\pgfqpoint{4.797113in}{2.431588in}}%
\pgfpathlineto{\pgfqpoint{4.797958in}{1.796685in}}%
\pgfpathlineto{\pgfqpoint{4.798804in}{2.113883in}}%
\pgfpathlineto{\pgfqpoint{4.800495in}{2.008169in}}%
\pgfpathlineto{\pgfqpoint{4.802186in}{2.113888in}}%
\pgfpathlineto{\pgfqpoint{4.803032in}{2.113875in}}%
\pgfpathlineto{\pgfqpoint{4.803878in}{2.325348in}}%
\pgfpathlineto{\pgfqpoint{4.808106in}{1.268013in}}%
\pgfpathlineto{\pgfqpoint{4.808951in}{2.008079in}}%
\pgfpathlineto{\pgfqpoint{4.809797in}{1.902606in}}%
\pgfpathlineto{\pgfqpoint{4.812334in}{1.585178in}}%
\pgfpathlineto{\pgfqpoint{4.815716in}{2.854051in}}%
\pgfpathlineto{\pgfqpoint{4.819099in}{1.585210in}}%
\pgfpathlineto{\pgfqpoint{4.819944in}{2.536802in}}%
\pgfpathlineto{\pgfqpoint{4.820790in}{2.325359in}}%
\pgfpathlineto{\pgfqpoint{4.821636in}{1.798703in}}%
\pgfpathlineto{\pgfqpoint{4.822481in}{2.219713in}}%
\pgfpathlineto{\pgfqpoint{4.823327in}{1.902412in}}%
\pgfpathlineto{\pgfqpoint{4.824173in}{2.430442in}}%
\pgfpathlineto{\pgfqpoint{4.825018in}{2.008139in}}%
\pgfpathlineto{\pgfqpoint{4.825864in}{2.114374in}}%
\pgfpathlineto{\pgfqpoint{4.826709in}{2.113929in}}%
\pgfpathlineto{\pgfqpoint{4.827555in}{2.113412in}}%
\pgfpathlineto{\pgfqpoint{4.830092in}{1.796834in}}%
\pgfpathlineto{\pgfqpoint{4.831783in}{2.325346in}}%
\pgfpathlineto{\pgfqpoint{4.833474in}{1.585328in}}%
\pgfpathlineto{\pgfqpoint{4.834320in}{2.324965in}}%
\pgfpathlineto{\pgfqpoint{4.835166in}{1.376768in}}%
\pgfpathlineto{\pgfqpoint{4.836011in}{1.585205in}}%
\pgfpathlineto{\pgfqpoint{4.837703in}{2.213618in}}%
\pgfpathlineto{\pgfqpoint{4.838548in}{2.747966in}}%
\pgfpathlineto{\pgfqpoint{4.840239in}{1.690986in}}%
\pgfpathlineto{\pgfqpoint{4.841931in}{2.009024in}}%
\pgfpathlineto{\pgfqpoint{4.842776in}{1.793893in}}%
\pgfpathlineto{\pgfqpoint{4.844468in}{2.431076in}}%
\pgfpathlineto{\pgfqpoint{4.845313in}{2.113488in}}%
\pgfpathlineto{\pgfqpoint{4.846159in}{2.325402in}}%
\pgfpathlineto{\pgfqpoint{4.847004in}{2.325447in}}%
\pgfpathlineto{\pgfqpoint{4.847850in}{2.114543in}}%
\pgfpathlineto{\pgfqpoint{4.848696in}{2.432149in}}%
\pgfpathlineto{\pgfqpoint{4.849541in}{1.372733in}}%
\pgfpathlineto{\pgfqpoint{4.850387in}{2.218632in}}%
\pgfpathlineto{\pgfqpoint{4.851233in}{2.850943in}}%
\pgfpathlineto{\pgfqpoint{4.853769in}{1.912603in}}%
\pgfpathlineto{\pgfqpoint{4.854615in}{2.245132in}}%
\pgfpathlineto{\pgfqpoint{4.855461in}{1.992863in}}%
\pgfpathlineto{\pgfqpoint{4.856306in}{1.902448in}}%
\pgfpathlineto{\pgfqpoint{4.857152in}{2.430427in}}%
\pgfpathlineto{\pgfqpoint{4.857998in}{1.795974in}}%
\pgfpathlineto{\pgfqpoint{4.858843in}{2.329203in}}%
\pgfpathlineto{\pgfqpoint{4.859689in}{2.218659in}}%
\pgfpathlineto{\pgfqpoint{4.861380in}{1.796770in}}%
\pgfpathlineto{\pgfqpoint{4.862226in}{2.219670in}}%
\pgfpathlineto{\pgfqpoint{4.863071in}{1.690901in}}%
\pgfpathlineto{\pgfqpoint{4.863917in}{1.901924in}}%
\pgfpathlineto{\pgfqpoint{4.864762in}{2.641560in}}%
\pgfpathlineto{\pgfqpoint{4.865608in}{1.585232in}}%
\pgfpathlineto{\pgfqpoint{4.866454in}{1.588400in}}%
\pgfpathlineto{\pgfqpoint{4.868145in}{1.993249in}}%
\pgfpathlineto{\pgfqpoint{4.868991in}{1.902450in}}%
\pgfpathlineto{\pgfqpoint{4.869836in}{1.901540in}}%
\pgfpathlineto{\pgfqpoint{4.870682in}{2.002875in}}%
\pgfpathlineto{\pgfqpoint{4.871527in}{1.569381in}}%
\pgfpathlineto{\pgfqpoint{4.872373in}{1.580246in}}%
\pgfpathlineto{\pgfqpoint{4.873219in}{2.239327in}}%
\pgfpathlineto{\pgfqpoint{4.874064in}{2.009031in}}%
\pgfpathlineto{\pgfqpoint{4.874910in}{1.499139in}}%
\pgfpathlineto{\pgfqpoint{4.875756in}{1.637649in}}%
\pgfpathlineto{\pgfqpoint{4.878292in}{2.590067in}}%
\pgfpathlineto{\pgfqpoint{4.879138in}{1.906302in}}%
\pgfpathlineto{\pgfqpoint{4.879984in}{2.749476in}}%
\pgfpathlineto{\pgfqpoint{4.880829in}{2.442660in}}%
\pgfpathlineto{\pgfqpoint{4.883366in}{1.498575in}}%
\pgfpathlineto{\pgfqpoint{4.885903in}{2.642222in}}%
\pgfpathlineto{\pgfqpoint{4.886749in}{2.114330in}}%
\pgfpathlineto{\pgfqpoint{4.887594in}{2.325274in}}%
\pgfpathlineto{\pgfqpoint{4.890131in}{1.902444in}}%
\pgfpathlineto{\pgfqpoint{4.890977in}{2.431097in}}%
\pgfpathlineto{\pgfqpoint{4.891822in}{2.113901in}}%
\pgfpathlineto{\pgfqpoint{4.892668in}{1.267993in}}%
\pgfpathlineto{\pgfqpoint{4.893514in}{2.113875in}}%
\pgfpathlineto{\pgfqpoint{4.894359in}{1.479516in}}%
\pgfpathlineto{\pgfqpoint{4.896896in}{2.325427in}}%
\pgfpathlineto{\pgfqpoint{4.898587in}{1.373723in}}%
\pgfpathlineto{\pgfqpoint{4.899433in}{2.219633in}}%
\pgfpathlineto{\pgfqpoint{4.900279in}{1.902423in}}%
\pgfpathlineto{\pgfqpoint{4.901124in}{1.373735in}}%
\pgfpathlineto{\pgfqpoint{4.902816in}{2.113877in}}%
\pgfpathlineto{\pgfqpoint{4.903661in}{2.007768in}}%
\pgfpathlineto{\pgfqpoint{4.904507in}{2.008150in}}%
\pgfpathlineto{\pgfqpoint{4.905352in}{1.268014in}}%
\pgfpathlineto{\pgfqpoint{4.906198in}{1.796697in}}%
\pgfpathlineto{\pgfqpoint{4.907889in}{2.431099in}}%
\pgfpathlineto{\pgfqpoint{4.909581in}{1.267995in}}%
\pgfpathlineto{\pgfqpoint{4.910426in}{2.431103in}}%
\pgfpathlineto{\pgfqpoint{4.911272in}{2.219440in}}%
\pgfpathlineto{\pgfqpoint{4.912117in}{2.008182in}}%
\pgfpathlineto{\pgfqpoint{4.912963in}{2.642545in}}%
\pgfpathlineto{\pgfqpoint{4.913809in}{2.219686in}}%
\pgfpathlineto{\pgfqpoint{4.914654in}{2.008878in}}%
\pgfpathlineto{\pgfqpoint{4.915500in}{2.536832in}}%
\pgfpathlineto{\pgfqpoint{4.917191in}{1.690956in}}%
\pgfpathlineto{\pgfqpoint{4.918882in}{1.902425in}}%
\pgfpathlineto{\pgfqpoint{4.919728in}{1.690969in}}%
\pgfpathlineto{\pgfqpoint{4.920574in}{2.748315in}}%
\pgfpathlineto{\pgfqpoint{4.921419in}{1.373741in}}%
\pgfpathlineto{\pgfqpoint{4.922265in}{1.902460in}}%
\pgfpathlineto{\pgfqpoint{4.923111in}{2.431081in}}%
\pgfpathlineto{\pgfqpoint{4.923956in}{1.267996in}}%
\pgfpathlineto{\pgfqpoint{4.924802in}{1.797864in}}%
\pgfpathlineto{\pgfqpoint{4.927339in}{2.536882in}}%
\pgfpathlineto{\pgfqpoint{4.929876in}{1.796683in}}%
\pgfpathlineto{\pgfqpoint{4.930721in}{1.690943in}}%
\pgfpathlineto{\pgfqpoint{4.931567in}{2.748212in}}%
\pgfpathlineto{\pgfqpoint{4.932412in}{2.219628in}}%
\pgfpathlineto{\pgfqpoint{4.933258in}{1.373833in}}%
\pgfpathlineto{\pgfqpoint{4.934104in}{1.691023in}}%
\pgfpathlineto{\pgfqpoint{4.935795in}{2.113891in}}%
\pgfpathlineto{\pgfqpoint{4.936641in}{1.690927in}}%
\pgfpathlineto{\pgfqpoint{4.937486in}{1.796645in}}%
\pgfpathlineto{\pgfqpoint{4.938332in}{2.748314in}}%
\pgfpathlineto{\pgfqpoint{4.939177in}{2.113894in}}%
\pgfpathlineto{\pgfqpoint{4.940023in}{1.902391in}}%
\pgfpathlineto{\pgfqpoint{4.940869in}{2.008219in}}%
\pgfpathlineto{\pgfqpoint{4.941714in}{2.431103in}}%
\pgfpathlineto{\pgfqpoint{4.942560in}{1.373742in}}%
\pgfpathlineto{\pgfqpoint{4.943405in}{2.008154in}}%
\pgfpathlineto{\pgfqpoint{4.944251in}{2.325398in}}%
\pgfpathlineto{\pgfqpoint{4.946788in}{1.162260in}}%
\pgfpathlineto{\pgfqpoint{4.948479in}{1.796671in}}%
\pgfpathlineto{\pgfqpoint{4.949325in}{1.373795in}}%
\pgfpathlineto{\pgfqpoint{4.950170in}{2.642596in}}%
\pgfpathlineto{\pgfqpoint{4.951016in}{2.219624in}}%
\pgfpathlineto{\pgfqpoint{4.951862in}{2.008164in}}%
\pgfpathlineto{\pgfqpoint{4.952707in}{2.219653in}}%
\pgfpathlineto{\pgfqpoint{4.953553in}{1.690936in}}%
\pgfpathlineto{\pgfqpoint{4.955244in}{2.642576in}}%
\pgfpathlineto{\pgfqpoint{4.957781in}{1.479738in}}%
\pgfpathlineto{\pgfqpoint{4.959472in}{2.325113in}}%
\pgfpathlineto{\pgfqpoint{4.960318in}{2.008154in}}%
\pgfpathlineto{\pgfqpoint{4.961164in}{2.008190in}}%
\pgfpathlineto{\pgfqpoint{4.962855in}{1.690980in}}%
\pgfpathlineto{\pgfqpoint{4.964546in}{2.642584in}}%
\pgfpathlineto{\pgfqpoint{4.965392in}{1.902266in}}%
\pgfpathlineto{\pgfqpoint{4.966237in}{2.325393in}}%
\pgfpathlineto{\pgfqpoint{4.967083in}{2.325395in}}%
\pgfpathlineto{\pgfqpoint{4.969620in}{1.268001in}}%
\pgfpathlineto{\pgfqpoint{4.971311in}{2.113901in}}%
\pgfpathlineto{\pgfqpoint{4.972157in}{2.113625in}}%
\pgfpathlineto{\pgfqpoint{4.973002in}{1.162265in}}%
\pgfpathlineto{\pgfqpoint{4.973848in}{2.748303in}}%
\pgfpathlineto{\pgfqpoint{4.974694in}{1.268000in}}%
\pgfpathlineto{\pgfqpoint{4.975539in}{2.113921in}}%
\pgfpathlineto{\pgfqpoint{4.976385in}{2.748186in}}%
\pgfpathlineto{\pgfqpoint{4.977230in}{2.325367in}}%
\pgfpathlineto{\pgfqpoint{4.978076in}{2.008073in}}%
\pgfpathlineto{\pgfqpoint{4.978922in}{2.642609in}}%
\pgfpathlineto{\pgfqpoint{4.979767in}{2.325354in}}%
\pgfpathlineto{\pgfqpoint{4.980613in}{1.688848in}}%
\pgfpathlineto{\pgfqpoint{4.981459in}{1.902517in}}%
\pgfpathlineto{\pgfqpoint{4.982304in}{2.748320in}}%
\pgfpathlineto{\pgfqpoint{4.983150in}{1.585196in}}%
\pgfpathlineto{\pgfqpoint{4.983995in}{2.219880in}}%
\pgfpathlineto{\pgfqpoint{4.984841in}{2.008150in}}%
\pgfpathlineto{\pgfqpoint{4.985687in}{2.008180in}}%
\pgfpathlineto{\pgfqpoint{4.986532in}{2.219634in}}%
\pgfpathlineto{\pgfqpoint{4.989069in}{1.373749in}}%
\pgfpathlineto{\pgfqpoint{4.991606in}{2.219690in}}%
\pgfpathlineto{\pgfqpoint{4.993297in}{2.219621in}}%
\pgfpathlineto{\pgfqpoint{4.995834in}{1.479478in}}%
\pgfpathlineto{\pgfqpoint{4.996680in}{1.690943in}}%
\pgfpathlineto{\pgfqpoint{4.997525in}{2.008201in}}%
\pgfpathlineto{\pgfqpoint{4.998371in}{3.171298in}}%
\pgfpathlineto{\pgfqpoint{4.999217in}{2.431099in}}%
\pgfpathlineto{\pgfqpoint{5.000062in}{1.902366in}}%
\pgfpathlineto{\pgfqpoint{5.000908in}{2.325620in}}%
\pgfpathlineto{\pgfqpoint{5.001754in}{2.431726in}}%
\pgfpathlineto{\pgfqpoint{5.002599in}{1.796580in}}%
\pgfpathlineto{\pgfqpoint{5.003445in}{2.325367in}}%
\pgfpathlineto{\pgfqpoint{5.004290in}{2.008150in}}%
\pgfpathlineto{\pgfqpoint{5.005136in}{2.113688in}}%
\pgfpathlineto{\pgfqpoint{5.006827in}{1.690952in}}%
\pgfpathlineto{\pgfqpoint{5.008519in}{2.536839in}}%
\pgfpathlineto{\pgfqpoint{5.009364in}{1.902402in}}%
\pgfpathlineto{\pgfqpoint{5.010210in}{2.113796in}}%
\pgfpathlineto{\pgfqpoint{5.011901in}{1.796678in}}%
\pgfpathlineto{\pgfqpoint{5.012747in}{2.751033in}}%
\pgfpathlineto{\pgfqpoint{5.013592in}{2.325380in}}%
\pgfpathlineto{\pgfqpoint{5.015284in}{1.796681in}}%
\pgfpathlineto{\pgfqpoint{5.016129in}{2.219624in}}%
\pgfpathlineto{\pgfqpoint{5.016975in}{2.113892in}}%
\pgfpathlineto{\pgfqpoint{5.017820in}{2.113895in}}%
\pgfpathlineto{\pgfqpoint{5.018666in}{2.008185in}}%
\pgfpathlineto{\pgfqpoint{5.019512in}{2.536837in}}%
\pgfpathlineto{\pgfqpoint{5.020357in}{1.796676in}}%
\pgfpathlineto{\pgfqpoint{5.021203in}{2.008160in}}%
\pgfpathlineto{\pgfqpoint{5.022049in}{1.796186in}}%
\pgfpathlineto{\pgfqpoint{5.022894in}{2.113896in}}%
\pgfpathlineto{\pgfqpoint{5.025431in}{1.269681in}}%
\pgfpathlineto{\pgfqpoint{5.026277in}{2.113854in}}%
\pgfpathlineto{\pgfqpoint{5.027122in}{2.008094in}}%
\pgfpathlineto{\pgfqpoint{5.027968in}{1.373735in}}%
\pgfpathlineto{\pgfqpoint{5.029659in}{2.536577in}}%
\pgfpathlineto{\pgfqpoint{5.030505in}{1.479471in}}%
\pgfpathlineto{\pgfqpoint{5.031350in}{2.008146in}}%
\pgfpathlineto{\pgfqpoint{5.032196in}{1.479473in}}%
\pgfpathlineto{\pgfqpoint{5.033042in}{2.854051in}}%
\pgfpathlineto{\pgfqpoint{5.033887in}{2.748302in}}%
\pgfpathlineto{\pgfqpoint{5.034733in}{1.902407in}}%
\pgfpathlineto{\pgfqpoint{5.035578in}{2.325364in}}%
\pgfpathlineto{\pgfqpoint{5.036424in}{2.431102in}}%
\pgfpathlineto{\pgfqpoint{5.037270in}{2.325368in}}%
\pgfpathlineto{\pgfqpoint{5.038115in}{2.008159in}}%
\pgfpathlineto{\pgfqpoint{5.038961in}{2.008202in}}%
\pgfpathlineto{\pgfqpoint{5.040652in}{2.748325in}}%
\pgfpathlineto{\pgfqpoint{5.041498in}{2.536737in}}%
\pgfpathlineto{\pgfqpoint{5.042343in}{2.325392in}}%
\pgfpathlineto{\pgfqpoint{5.043189in}{1.585207in}}%
\pgfpathlineto{\pgfqpoint{5.044035in}{1.796682in}}%
\pgfpathlineto{\pgfqpoint{5.044880in}{2.008169in}}%
\pgfpathlineto{\pgfqpoint{5.045726in}{1.585260in}}%
\pgfpathlineto{\pgfqpoint{5.046572in}{1.796683in}}%
\pgfpathlineto{\pgfqpoint{5.047417in}{2.642576in}}%
\pgfpathlineto{\pgfqpoint{5.048263in}{2.113896in}}%
\pgfpathlineto{\pgfqpoint{5.049108in}{2.325368in}}%
\pgfpathlineto{\pgfqpoint{5.049954in}{1.585201in}}%
\pgfpathlineto{\pgfqpoint{5.052491in}{2.746223in}}%
\pgfpathlineto{\pgfqpoint{5.053337in}{1.902396in}}%
\pgfpathlineto{\pgfqpoint{5.054182in}{2.325365in}}%
\pgfpathlineto{\pgfqpoint{5.055028in}{2.219793in}}%
\pgfpathlineto{\pgfqpoint{5.055873in}{1.690936in}}%
\pgfpathlineto{\pgfqpoint{5.056719in}{1.796683in}}%
\pgfpathlineto{\pgfqpoint{5.057565in}{2.431098in}}%
\pgfpathlineto{\pgfqpoint{5.058410in}{1.268009in}}%
\pgfpathlineto{\pgfqpoint{5.059256in}{1.902367in}}%
\pgfpathlineto{\pgfqpoint{5.061793in}{1.268002in}}%
\pgfpathlineto{\pgfqpoint{5.062638in}{2.008156in}}%
\pgfpathlineto{\pgfqpoint{5.063484in}{1.585208in}}%
\pgfpathlineto{\pgfqpoint{5.064330in}{1.902441in}}%
\pgfpathlineto{\pgfqpoint{5.065175in}{1.796683in}}%
\pgfpathlineto{\pgfqpoint{5.066021in}{1.479492in}}%
\pgfpathlineto{\pgfqpoint{5.067712in}{2.642576in}}%
\pgfpathlineto{\pgfqpoint{5.068558in}{1.585217in}}%
\pgfpathlineto{\pgfqpoint{5.069403in}{1.690819in}}%
\pgfpathlineto{\pgfqpoint{5.070249in}{2.959769in}}%
\pgfpathlineto{\pgfqpoint{5.071095in}{1.902423in}}%
\pgfpathlineto{\pgfqpoint{5.071940in}{2.431105in}}%
\pgfpathlineto{\pgfqpoint{5.072786in}{2.325307in}}%
\pgfpathlineto{\pgfqpoint{5.073632in}{2.008154in}}%
\pgfpathlineto{\pgfqpoint{5.074477in}{2.219584in}}%
\pgfpathlineto{\pgfqpoint{5.075323in}{2.113893in}}%
\pgfpathlineto{\pgfqpoint{5.077014in}{1.585069in}}%
\pgfpathlineto{\pgfqpoint{5.080397in}{2.008153in}}%
\pgfpathlineto{\pgfqpoint{5.081242in}{2.008146in}}%
\pgfpathlineto{\pgfqpoint{5.082088in}{1.796682in}}%
\pgfpathlineto{\pgfqpoint{5.082933in}{2.219542in}}%
\pgfpathlineto{\pgfqpoint{5.083779in}{1.585204in}}%
\pgfpathlineto{\pgfqpoint{5.084625in}{2.431102in}}%
\pgfpathlineto{\pgfqpoint{5.085470in}{1.796650in}}%
\pgfpathlineto{\pgfqpoint{5.086316in}{1.902448in}}%
\pgfpathlineto{\pgfqpoint{5.087162in}{1.373735in}}%
\pgfpathlineto{\pgfqpoint{5.088007in}{2.431094in}}%
\pgfpathlineto{\pgfqpoint{5.088853in}{2.113892in}}%
\pgfpathlineto{\pgfqpoint{5.089698in}{2.008127in}}%
\pgfpathlineto{\pgfqpoint{5.090544in}{2.325321in}}%
\pgfpathlineto{\pgfqpoint{5.092235in}{1.163289in}}%
\pgfpathlineto{\pgfqpoint{5.093081in}{1.690880in}}%
\pgfpathlineto{\pgfqpoint{5.093927in}{1.902375in}}%
\pgfpathlineto{\pgfqpoint{5.094772in}{2.959784in}}%
\pgfpathlineto{\pgfqpoint{5.095618in}{2.748341in}}%
\pgfpathlineto{\pgfqpoint{5.097309in}{1.796760in}}%
\pgfpathlineto{\pgfqpoint{5.098155in}{2.748423in}}%
\pgfpathlineto{\pgfqpoint{5.099846in}{1.267999in}}%
\pgfpathlineto{\pgfqpoint{5.100692in}{2.748301in}}%
\pgfpathlineto{\pgfqpoint{5.101537in}{2.325367in}}%
\pgfpathlineto{\pgfqpoint{5.103228in}{1.585214in}}%
\pgfpathlineto{\pgfqpoint{5.105765in}{2.430981in}}%
\pgfpathlineto{\pgfqpoint{5.108302in}{1.796676in}}%
\pgfpathlineto{\pgfqpoint{5.109148in}{2.852845in}}%
\pgfpathlineto{\pgfqpoint{5.109993in}{2.325342in}}%
\pgfpathlineto{\pgfqpoint{5.110839in}{2.008156in}}%
\pgfpathlineto{\pgfqpoint{5.111685in}{2.113893in}}%
\pgfpathlineto{\pgfqpoint{5.112530in}{2.219626in}}%
\pgfpathlineto{\pgfqpoint{5.113376in}{2.008113in}}%
\pgfpathlineto{\pgfqpoint{5.114221in}{1.269196in}}%
\pgfpathlineto{\pgfqpoint{5.115913in}{1.902409in}}%
\pgfpathlineto{\pgfqpoint{5.116758in}{1.479562in}}%
\pgfpathlineto{\pgfqpoint{5.118450in}{1.904969in}}%
\pgfpathlineto{\pgfqpoint{5.119295in}{1.526497in}}%
\pgfpathlineto{\pgfqpoint{5.120141in}{1.796539in}}%
\pgfpathlineto{\pgfqpoint{5.120986in}{2.536994in}}%
\pgfpathlineto{\pgfqpoint{5.121832in}{2.325343in}}%
\pgfpathlineto{\pgfqpoint{5.122678in}{2.536836in}}%
\pgfpathlineto{\pgfqpoint{5.123523in}{2.008220in}}%
\pgfpathlineto{\pgfqpoint{5.125215in}{2.854049in}}%
\pgfpathlineto{\pgfqpoint{5.126906in}{1.585207in}}%
\pgfpathlineto{\pgfqpoint{5.128597in}{2.536805in}}%
\pgfpathlineto{\pgfqpoint{5.129443in}{1.268006in}}%
\pgfpathlineto{\pgfqpoint{5.130288in}{2.113946in}}%
\pgfpathlineto{\pgfqpoint{5.131134in}{1.373737in}}%
\pgfpathlineto{\pgfqpoint{5.131980in}{1.479471in}}%
\pgfpathlineto{\pgfqpoint{5.134516in}{2.854045in}}%
\pgfpathlineto{\pgfqpoint{5.135362in}{1.056523in}}%
\pgfpathlineto{\pgfqpoint{5.136208in}{2.536823in}}%
\pgfpathlineto{\pgfqpoint{5.137053in}{1.162307in}}%
\pgfpathlineto{\pgfqpoint{5.137899in}{2.536731in}}%
\pgfpathlineto{\pgfqpoint{5.138745in}{1.902396in}}%
\pgfpathlineto{\pgfqpoint{5.139590in}{1.796681in}}%
\pgfpathlineto{\pgfqpoint{5.140436in}{1.902902in}}%
\pgfpathlineto{\pgfqpoint{5.141281in}{2.219628in}}%
\pgfpathlineto{\pgfqpoint{5.142127in}{1.796680in}}%
\pgfpathlineto{\pgfqpoint{5.142973in}{2.853988in}}%
\pgfpathlineto{\pgfqpoint{5.143818in}{2.113862in}}%
\pgfpathlineto{\pgfqpoint{5.144664in}{1.479469in}}%
\pgfpathlineto{\pgfqpoint{5.146355in}{2.536863in}}%
\pgfpathlineto{\pgfqpoint{5.148046in}{1.690957in}}%
\pgfpathlineto{\pgfqpoint{5.148892in}{2.219636in}}%
\pgfpathlineto{\pgfqpoint{5.149738in}{1.902418in}}%
\pgfpathlineto{\pgfqpoint{5.151429in}{2.325362in}}%
\pgfpathlineto{\pgfqpoint{5.152275in}{1.690933in}}%
\pgfpathlineto{\pgfqpoint{5.153120in}{2.219628in}}%
\pgfpathlineto{\pgfqpoint{5.153966in}{1.902437in}}%
\pgfpathlineto{\pgfqpoint{5.154811in}{2.008154in}}%
\pgfpathlineto{\pgfqpoint{5.155657in}{1.690946in}}%
\pgfpathlineto{\pgfqpoint{5.158194in}{2.113888in}}%
\pgfpathlineto{\pgfqpoint{5.160731in}{1.479476in}}%
\pgfpathlineto{\pgfqpoint{5.162422in}{2.113886in}}%
\pgfpathlineto{\pgfqpoint{5.163268in}{2.008148in}}%
\pgfpathlineto{\pgfqpoint{5.164113in}{1.585207in}}%
\pgfpathlineto{\pgfqpoint{5.164959in}{1.796773in}}%
\pgfpathlineto{\pgfqpoint{5.165805in}{1.902421in}}%
\pgfpathlineto{\pgfqpoint{5.167496in}{2.642573in}}%
\pgfpathlineto{\pgfqpoint{5.169187in}{1.902418in}}%
\pgfpathlineto{\pgfqpoint{5.170033in}{2.219628in}}%
\pgfpathlineto{\pgfqpoint{5.170878in}{1.268080in}}%
\pgfpathlineto{\pgfqpoint{5.171724in}{2.008686in}}%
\pgfpathlineto{\pgfqpoint{5.172570in}{2.008160in}}%
\pgfpathlineto{\pgfqpoint{5.173415in}{1.902402in}}%
\pgfpathlineto{\pgfqpoint{5.175106in}{2.114113in}}%
\pgfpathlineto{\pgfqpoint{5.175952in}{2.113889in}}%
\pgfpathlineto{\pgfqpoint{5.177643in}{1.690941in}}%
\pgfpathlineto{\pgfqpoint{5.178489in}{2.536787in}}%
\pgfpathlineto{\pgfqpoint{5.179335in}{2.008120in}}%
\pgfpathlineto{\pgfqpoint{5.180180in}{1.585183in}}%
\pgfpathlineto{\pgfqpoint{5.181026in}{1.902418in}}%
\pgfpathlineto{\pgfqpoint{5.181871in}{2.008153in}}%
\pgfpathlineto{\pgfqpoint{5.182717in}{1.690945in}}%
\pgfpathlineto{\pgfqpoint{5.183563in}{2.219627in}}%
\pgfpathlineto{\pgfqpoint{5.184408in}{2.113881in}}%
\pgfpathlineto{\pgfqpoint{5.185254in}{1.902421in}}%
\pgfpathlineto{\pgfqpoint{5.187791in}{2.536839in}}%
\pgfpathlineto{\pgfqpoint{5.188636in}{1.373565in}}%
\pgfpathlineto{\pgfqpoint{5.188636in}{1.373565in}}%
\pgfusepath{stroke}%
\end{pgfscope}%
\begin{pgfscope}%
\pgfsetrectcap%
\pgfsetmiterjoin%
\pgfsetlinewidth{0.803000pt}%
\definecolor{currentstroke}{rgb}{0.000000,0.000000,0.000000}%
\pgfsetstrokecolor{currentstroke}%
\pgfsetdash{}{0pt}%
\pgfpathmoveto{\pgfqpoint{0.750000in}{0.500000in}}%
\pgfpathlineto{\pgfqpoint{0.750000in}{3.520000in}}%
\pgfusepath{stroke}%
\end{pgfscope}%
\begin{pgfscope}%
\pgfsetrectcap%
\pgfsetmiterjoin%
\pgfsetlinewidth{0.803000pt}%
\definecolor{currentstroke}{rgb}{0.000000,0.000000,0.000000}%
\pgfsetstrokecolor{currentstroke}%
\pgfsetdash{}{0pt}%
\pgfpathmoveto{\pgfqpoint{5.400000in}{0.500000in}}%
\pgfpathlineto{\pgfqpoint{5.400000in}{3.520000in}}%
\pgfusepath{stroke}%
\end{pgfscope}%
\begin{pgfscope}%
\pgfsetrectcap%
\pgfsetmiterjoin%
\pgfsetlinewidth{0.803000pt}%
\definecolor{currentstroke}{rgb}{0.000000,0.000000,0.000000}%
\pgfsetstrokecolor{currentstroke}%
\pgfsetdash{}{0pt}%
\pgfpathmoveto{\pgfqpoint{0.750000in}{0.500000in}}%
\pgfpathlineto{\pgfqpoint{5.400000in}{0.500000in}}%
\pgfusepath{stroke}%
\end{pgfscope}%
\begin{pgfscope}%
\pgfsetrectcap%
\pgfsetmiterjoin%
\pgfsetlinewidth{0.803000pt}%
\definecolor{currentstroke}{rgb}{0.000000,0.000000,0.000000}%
\pgfsetstrokecolor{currentstroke}%
\pgfsetdash{}{0pt}%
\pgfpathmoveto{\pgfqpoint{0.750000in}{3.520000in}}%
\pgfpathlineto{\pgfqpoint{5.400000in}{3.520000in}}%
\pgfusepath{stroke}%
\end{pgfscope}%
\begin{pgfscope}%
\pgfsetbuttcap%
\pgfsetmiterjoin%
\definecolor{currentfill}{rgb}{1.000000,1.000000,1.000000}%
\pgfsetfillcolor{currentfill}%
\pgfsetfillopacity{0.800000}%
\pgfsetlinewidth{1.003750pt}%
\definecolor{currentstroke}{rgb}{0.800000,0.800000,0.800000}%
\pgfsetstrokecolor{currentstroke}%
\pgfsetstrokeopacity{0.800000}%
\pgfsetdash{}{0pt}%
\pgfpathmoveto{\pgfqpoint{0.847222in}{3.205032in}}%
\pgfpathlineto{\pgfqpoint{1.587619in}{3.205032in}}%
\pgfpathquadraticcurveto{\pgfqpoint{1.615397in}{3.205032in}}{\pgfqpoint{1.615397in}{3.232809in}}%
\pgfpathlineto{\pgfqpoint{1.615397in}{3.422778in}}%
\pgfpathquadraticcurveto{\pgfqpoint{1.615397in}{3.450556in}}{\pgfqpoint{1.587619in}{3.450556in}}%
\pgfpathlineto{\pgfqpoint{0.847222in}{3.450556in}}%
\pgfpathquadraticcurveto{\pgfqpoint{0.819444in}{3.450556in}}{\pgfqpoint{0.819444in}{3.422778in}}%
\pgfpathlineto{\pgfqpoint{0.819444in}{3.232809in}}%
\pgfpathquadraticcurveto{\pgfqpoint{0.819444in}{3.205032in}}{\pgfqpoint{0.847222in}{3.205032in}}%
\pgfpathlineto{\pgfqpoint{0.847222in}{3.205032in}}%
\pgfpathclose%
\pgfusepath{stroke,fill}%
\end{pgfscope}%
\begin{pgfscope}%
\pgfsetrectcap%
\pgfsetroundjoin%
\pgfsetlinewidth{1.505625pt}%
\definecolor{currentstroke}{rgb}{1.000000,0.000000,0.000000}%
\pgfsetstrokecolor{currentstroke}%
\pgfsetdash{}{0pt}%
\pgfpathmoveto{\pgfqpoint{0.875000in}{3.338088in}}%
\pgfpathlineto{\pgfqpoint{1.013889in}{3.338088in}}%
\pgfpathlineto{\pgfqpoint{1.152778in}{3.338088in}}%
\pgfusepath{stroke}%
\end{pgfscope}%
\begin{pgfscope}%
\definecolor{textcolor}{rgb}{0.000000,0.000000,0.000000}%
\pgfsetstrokecolor{textcolor}%
\pgfsetfillcolor{textcolor}%
\pgftext[x=1.263889in,y=3.289477in,left,base]{\color{textcolor}\sffamily\fontsize{10.000000}{12.000000}\selectfont SNN}%
\end{pgfscope}%
\end{pgfpicture}%
\makeatother%
\endgroup%

    \caption{Caption}
    \label{fig:my_label}
\end{figure}

\begin{figure}
%% Creator: Matplotlib, PGF backend
%%
%% To include the figure in your LaTeX document, write
%%   \input{<filename>.pgf}
%%
%% Make sure the required packages are loaded in your preamble
%%   \usepackage{pgf}
%%
%% Also ensure that all the required font packages are loaded; for instance,
%% the lmodern package is sometimes necessary when using math font.
%%   \usepackage{lmodern}
%%
%% Figures using additional raster images can only be included by \input if
%% they are in the same directory as the main LaTeX file. For loading figures
%% from other directories you can use the `import` package
%%   \usepackage{import}
%%
%% and then include the figures with
%%   \import{<path to file>}{<filename>.pgf}
%%
%% Matplotlib used the following preamble
%%   \usepackage{fontspec}
%%   \setmainfont{DejaVuSerif.ttf}[Path=\detokenize{C:/I/python38/Lib/site-packages/matplotlib/mpl-data/fonts/ttf/}]
%%   \setsansfont{DejaVuSans.ttf}[Path=\detokenize{C:/I/python38/Lib/site-packages/matplotlib/mpl-data/fonts/ttf/}]
%%   \setmonofont{DejaVuSansMono.ttf}[Path=\detokenize{C:/I/python38/Lib/site-packages/matplotlib/mpl-data/fonts/ttf/}]
%%
\begingroup%
\makeatletter%
\begin{pgfpicture}%
\pgfpathrectangle{\pgfpointorigin}{\pgfqpoint{6.000000in}{4.000000in}}%
\pgfusepath{use as bounding box, clip}%
\begin{pgfscope}%
\pgfsetbuttcap%
\pgfsetmiterjoin%
\pgfsetlinewidth{0.000000pt}%
\definecolor{currentstroke}{rgb}{1.000000,1.000000,1.000000}%
\pgfsetstrokecolor{currentstroke}%
\pgfsetstrokeopacity{0.000000}%
\pgfsetdash{}{0pt}%
\pgfpathmoveto{\pgfqpoint{0.000000in}{0.000000in}}%
\pgfpathlineto{\pgfqpoint{6.000000in}{0.000000in}}%
\pgfpathlineto{\pgfqpoint{6.000000in}{4.000000in}}%
\pgfpathlineto{\pgfqpoint{0.000000in}{4.000000in}}%
\pgfpathlineto{\pgfqpoint{0.000000in}{0.000000in}}%
\pgfpathclose%
\pgfusepath{}%
\end{pgfscope}%
\begin{pgfscope}%
\pgfsetbuttcap%
\pgfsetmiterjoin%
\definecolor{currentfill}{rgb}{1.000000,1.000000,1.000000}%
\pgfsetfillcolor{currentfill}%
\pgfsetlinewidth{0.000000pt}%
\definecolor{currentstroke}{rgb}{0.000000,0.000000,0.000000}%
\pgfsetstrokecolor{currentstroke}%
\pgfsetstrokeopacity{0.000000}%
\pgfsetdash{}{0pt}%
\pgfpathmoveto{\pgfqpoint{0.750000in}{0.500000in}}%
\pgfpathlineto{\pgfqpoint{5.400000in}{0.500000in}}%
\pgfpathlineto{\pgfqpoint{5.400000in}{3.520000in}}%
\pgfpathlineto{\pgfqpoint{0.750000in}{3.520000in}}%
\pgfpathlineto{\pgfqpoint{0.750000in}{0.500000in}}%
\pgfpathclose%
\pgfusepath{fill}%
\end{pgfscope}%
\begin{pgfscope}%
\pgfsetbuttcap%
\pgfsetroundjoin%
\definecolor{currentfill}{rgb}{0.000000,0.000000,0.000000}%
\pgfsetfillcolor{currentfill}%
\pgfsetlinewidth{0.803000pt}%
\definecolor{currentstroke}{rgb}{0.000000,0.000000,0.000000}%
\pgfsetstrokecolor{currentstroke}%
\pgfsetdash{}{0pt}%
\pgfsys@defobject{currentmarker}{\pgfqpoint{0.000000in}{-0.048611in}}{\pgfqpoint{0.000000in}{0.000000in}}{%
\pgfpathmoveto{\pgfqpoint{0.000000in}{0.000000in}}%
\pgfpathlineto{\pgfqpoint{0.000000in}{-0.048611in}}%
\pgfusepath{stroke,fill}%
}%
\begin{pgfscope}%
\pgfsys@transformshift{0.961364in}{0.500000in}%
\pgfsys@useobject{currentmarker}{}%
\end{pgfscope}%
\end{pgfscope}%
\begin{pgfscope}%
\definecolor{textcolor}{rgb}{0.000000,0.000000,0.000000}%
\pgfsetstrokecolor{textcolor}%
\pgfsetfillcolor{textcolor}%
\pgftext[x=0.961364in,y=0.402778in,,top]{\color{textcolor}\sffamily\fontsize{10.000000}{12.000000}\selectfont 0}%
\end{pgfscope}%
\begin{pgfscope}%
\pgfsetbuttcap%
\pgfsetroundjoin%
\definecolor{currentfill}{rgb}{0.000000,0.000000,0.000000}%
\pgfsetfillcolor{currentfill}%
\pgfsetlinewidth{0.803000pt}%
\definecolor{currentstroke}{rgb}{0.000000,0.000000,0.000000}%
\pgfsetstrokecolor{currentstroke}%
\pgfsetdash{}{0pt}%
\pgfsys@defobject{currentmarker}{\pgfqpoint{0.000000in}{-0.048611in}}{\pgfqpoint{0.000000in}{0.000000in}}{%
\pgfpathmoveto{\pgfqpoint{0.000000in}{0.000000in}}%
\pgfpathlineto{\pgfqpoint{0.000000in}{-0.048611in}}%
\pgfusepath{stroke,fill}%
}%
\begin{pgfscope}%
\pgfsys@transformshift{1.806987in}{0.500000in}%
\pgfsys@useobject{currentmarker}{}%
\end{pgfscope}%
\end{pgfscope}%
\begin{pgfscope}%
\definecolor{textcolor}{rgb}{0.000000,0.000000,0.000000}%
\pgfsetstrokecolor{textcolor}%
\pgfsetfillcolor{textcolor}%
\pgftext[x=1.806987in,y=0.402778in,,top]{\color{textcolor}\sffamily\fontsize{10.000000}{12.000000}\selectfont 1000}%
\end{pgfscope}%
\begin{pgfscope}%
\pgfsetbuttcap%
\pgfsetroundjoin%
\definecolor{currentfill}{rgb}{0.000000,0.000000,0.000000}%
\pgfsetfillcolor{currentfill}%
\pgfsetlinewidth{0.803000pt}%
\definecolor{currentstroke}{rgb}{0.000000,0.000000,0.000000}%
\pgfsetstrokecolor{currentstroke}%
\pgfsetdash{}{0pt}%
\pgfsys@defobject{currentmarker}{\pgfqpoint{0.000000in}{-0.048611in}}{\pgfqpoint{0.000000in}{0.000000in}}{%
\pgfpathmoveto{\pgfqpoint{0.000000in}{0.000000in}}%
\pgfpathlineto{\pgfqpoint{0.000000in}{-0.048611in}}%
\pgfusepath{stroke,fill}%
}%
\begin{pgfscope}%
\pgfsys@transformshift{2.652611in}{0.500000in}%
\pgfsys@useobject{currentmarker}{}%
\end{pgfscope}%
\end{pgfscope}%
\begin{pgfscope}%
\definecolor{textcolor}{rgb}{0.000000,0.000000,0.000000}%
\pgfsetstrokecolor{textcolor}%
\pgfsetfillcolor{textcolor}%
\pgftext[x=2.652611in,y=0.402778in,,top]{\color{textcolor}\sffamily\fontsize{10.000000}{12.000000}\selectfont 2000}%
\end{pgfscope}%
\begin{pgfscope}%
\pgfsetbuttcap%
\pgfsetroundjoin%
\definecolor{currentfill}{rgb}{0.000000,0.000000,0.000000}%
\pgfsetfillcolor{currentfill}%
\pgfsetlinewidth{0.803000pt}%
\definecolor{currentstroke}{rgb}{0.000000,0.000000,0.000000}%
\pgfsetstrokecolor{currentstroke}%
\pgfsetdash{}{0pt}%
\pgfsys@defobject{currentmarker}{\pgfqpoint{0.000000in}{-0.048611in}}{\pgfqpoint{0.000000in}{0.000000in}}{%
\pgfpathmoveto{\pgfqpoint{0.000000in}{0.000000in}}%
\pgfpathlineto{\pgfqpoint{0.000000in}{-0.048611in}}%
\pgfusepath{stroke,fill}%
}%
\begin{pgfscope}%
\pgfsys@transformshift{3.498235in}{0.500000in}%
\pgfsys@useobject{currentmarker}{}%
\end{pgfscope}%
\end{pgfscope}%
\begin{pgfscope}%
\definecolor{textcolor}{rgb}{0.000000,0.000000,0.000000}%
\pgfsetstrokecolor{textcolor}%
\pgfsetfillcolor{textcolor}%
\pgftext[x=3.498235in,y=0.402778in,,top]{\color{textcolor}\sffamily\fontsize{10.000000}{12.000000}\selectfont 3000}%
\end{pgfscope}%
\begin{pgfscope}%
\pgfsetbuttcap%
\pgfsetroundjoin%
\definecolor{currentfill}{rgb}{0.000000,0.000000,0.000000}%
\pgfsetfillcolor{currentfill}%
\pgfsetlinewidth{0.803000pt}%
\definecolor{currentstroke}{rgb}{0.000000,0.000000,0.000000}%
\pgfsetstrokecolor{currentstroke}%
\pgfsetdash{}{0pt}%
\pgfsys@defobject{currentmarker}{\pgfqpoint{0.000000in}{-0.048611in}}{\pgfqpoint{0.000000in}{0.000000in}}{%
\pgfpathmoveto{\pgfqpoint{0.000000in}{0.000000in}}%
\pgfpathlineto{\pgfqpoint{0.000000in}{-0.048611in}}%
\pgfusepath{stroke,fill}%
}%
\begin{pgfscope}%
\pgfsys@transformshift{4.343858in}{0.500000in}%
\pgfsys@useobject{currentmarker}{}%
\end{pgfscope}%
\end{pgfscope}%
\begin{pgfscope}%
\definecolor{textcolor}{rgb}{0.000000,0.000000,0.000000}%
\pgfsetstrokecolor{textcolor}%
\pgfsetfillcolor{textcolor}%
\pgftext[x=4.343858in,y=0.402778in,,top]{\color{textcolor}\sffamily\fontsize{10.000000}{12.000000}\selectfont 4000}%
\end{pgfscope}%
\begin{pgfscope}%
\pgfsetbuttcap%
\pgfsetroundjoin%
\definecolor{currentfill}{rgb}{0.000000,0.000000,0.000000}%
\pgfsetfillcolor{currentfill}%
\pgfsetlinewidth{0.803000pt}%
\definecolor{currentstroke}{rgb}{0.000000,0.000000,0.000000}%
\pgfsetstrokecolor{currentstroke}%
\pgfsetdash{}{0pt}%
\pgfsys@defobject{currentmarker}{\pgfqpoint{0.000000in}{-0.048611in}}{\pgfqpoint{0.000000in}{0.000000in}}{%
\pgfpathmoveto{\pgfqpoint{0.000000in}{0.000000in}}%
\pgfpathlineto{\pgfqpoint{0.000000in}{-0.048611in}}%
\pgfusepath{stroke,fill}%
}%
\begin{pgfscope}%
\pgfsys@transformshift{5.189482in}{0.500000in}%
\pgfsys@useobject{currentmarker}{}%
\end{pgfscope}%
\end{pgfscope}%
\begin{pgfscope}%
\definecolor{textcolor}{rgb}{0.000000,0.000000,0.000000}%
\pgfsetstrokecolor{textcolor}%
\pgfsetfillcolor{textcolor}%
\pgftext[x=5.189482in,y=0.402778in,,top]{\color{textcolor}\sffamily\fontsize{10.000000}{12.000000}\selectfont 5000}%
\end{pgfscope}%
\begin{pgfscope}%
\definecolor{textcolor}{rgb}{0.000000,0.000000,0.000000}%
\pgfsetstrokecolor{textcolor}%
\pgfsetfillcolor{textcolor}%
\pgftext[x=3.075000in,y=0.212809in,,top]{\color{textcolor}\sffamily\fontsize{10.000000}{12.000000}\selectfont iteration}%
\end{pgfscope}%
\begin{pgfscope}%
\pgfsetbuttcap%
\pgfsetroundjoin%
\definecolor{currentfill}{rgb}{0.000000,0.000000,0.000000}%
\pgfsetfillcolor{currentfill}%
\pgfsetlinewidth{0.803000pt}%
\definecolor{currentstroke}{rgb}{0.000000,0.000000,0.000000}%
\pgfsetstrokecolor{currentstroke}%
\pgfsetdash{}{0pt}%
\pgfsys@defobject{currentmarker}{\pgfqpoint{-0.048611in}{0.000000in}}{\pgfqpoint{-0.000000in}{0.000000in}}{%
\pgfpathmoveto{\pgfqpoint{-0.000000in}{0.000000in}}%
\pgfpathlineto{\pgfqpoint{-0.048611in}{0.000000in}}%
\pgfusepath{stroke,fill}%
}%
\begin{pgfscope}%
\pgfsys@transformshift{0.750000in}{0.521100in}%
\pgfsys@useobject{currentmarker}{}%
\end{pgfscope}%
\end{pgfscope}%
\begin{pgfscope}%
\definecolor{textcolor}{rgb}{0.000000,0.000000,0.000000}%
\pgfsetstrokecolor{textcolor}%
\pgfsetfillcolor{textcolor}%
\pgftext[x=0.343533in, y=0.468338in, left, base]{\color{textcolor}\sffamily\fontsize{10.000000}{12.000000}\selectfont 0.30}%
\end{pgfscope}%
\begin{pgfscope}%
\pgfsetbuttcap%
\pgfsetroundjoin%
\definecolor{currentfill}{rgb}{0.000000,0.000000,0.000000}%
\pgfsetfillcolor{currentfill}%
\pgfsetlinewidth{0.803000pt}%
\definecolor{currentstroke}{rgb}{0.000000,0.000000,0.000000}%
\pgfsetstrokecolor{currentstroke}%
\pgfsetdash{}{0pt}%
\pgfsys@defobject{currentmarker}{\pgfqpoint{-0.048611in}{0.000000in}}{\pgfqpoint{-0.000000in}{0.000000in}}{%
\pgfpathmoveto{\pgfqpoint{-0.000000in}{0.000000in}}%
\pgfpathlineto{\pgfqpoint{-0.048611in}{0.000000in}}%
\pgfusepath{stroke,fill}%
}%
\begin{pgfscope}%
\pgfsys@transformshift{0.750000in}{0.939474in}%
\pgfsys@useobject{currentmarker}{}%
\end{pgfscope}%
\end{pgfscope}%
\begin{pgfscope}%
\definecolor{textcolor}{rgb}{0.000000,0.000000,0.000000}%
\pgfsetstrokecolor{textcolor}%
\pgfsetfillcolor{textcolor}%
\pgftext[x=0.343533in, y=0.886713in, left, base]{\color{textcolor}\sffamily\fontsize{10.000000}{12.000000}\selectfont 0.35}%
\end{pgfscope}%
\begin{pgfscope}%
\pgfsetbuttcap%
\pgfsetroundjoin%
\definecolor{currentfill}{rgb}{0.000000,0.000000,0.000000}%
\pgfsetfillcolor{currentfill}%
\pgfsetlinewidth{0.803000pt}%
\definecolor{currentstroke}{rgb}{0.000000,0.000000,0.000000}%
\pgfsetstrokecolor{currentstroke}%
\pgfsetdash{}{0pt}%
\pgfsys@defobject{currentmarker}{\pgfqpoint{-0.048611in}{0.000000in}}{\pgfqpoint{-0.000000in}{0.000000in}}{%
\pgfpathmoveto{\pgfqpoint{-0.000000in}{0.000000in}}%
\pgfpathlineto{\pgfqpoint{-0.048611in}{0.000000in}}%
\pgfusepath{stroke,fill}%
}%
\begin{pgfscope}%
\pgfsys@transformshift{0.750000in}{1.357848in}%
\pgfsys@useobject{currentmarker}{}%
\end{pgfscope}%
\end{pgfscope}%
\begin{pgfscope}%
\definecolor{textcolor}{rgb}{0.000000,0.000000,0.000000}%
\pgfsetstrokecolor{textcolor}%
\pgfsetfillcolor{textcolor}%
\pgftext[x=0.343533in, y=1.305087in, left, base]{\color{textcolor}\sffamily\fontsize{10.000000}{12.000000}\selectfont 0.40}%
\end{pgfscope}%
\begin{pgfscope}%
\pgfsetbuttcap%
\pgfsetroundjoin%
\definecolor{currentfill}{rgb}{0.000000,0.000000,0.000000}%
\pgfsetfillcolor{currentfill}%
\pgfsetlinewidth{0.803000pt}%
\definecolor{currentstroke}{rgb}{0.000000,0.000000,0.000000}%
\pgfsetstrokecolor{currentstroke}%
\pgfsetdash{}{0pt}%
\pgfsys@defobject{currentmarker}{\pgfqpoint{-0.048611in}{0.000000in}}{\pgfqpoint{-0.000000in}{0.000000in}}{%
\pgfpathmoveto{\pgfqpoint{-0.000000in}{0.000000in}}%
\pgfpathlineto{\pgfqpoint{-0.048611in}{0.000000in}}%
\pgfusepath{stroke,fill}%
}%
\begin{pgfscope}%
\pgfsys@transformshift{0.750000in}{1.776222in}%
\pgfsys@useobject{currentmarker}{}%
\end{pgfscope}%
\end{pgfscope}%
\begin{pgfscope}%
\definecolor{textcolor}{rgb}{0.000000,0.000000,0.000000}%
\pgfsetstrokecolor{textcolor}%
\pgfsetfillcolor{textcolor}%
\pgftext[x=0.343533in, y=1.723461in, left, base]{\color{textcolor}\sffamily\fontsize{10.000000}{12.000000}\selectfont 0.45}%
\end{pgfscope}%
\begin{pgfscope}%
\pgfsetbuttcap%
\pgfsetroundjoin%
\definecolor{currentfill}{rgb}{0.000000,0.000000,0.000000}%
\pgfsetfillcolor{currentfill}%
\pgfsetlinewidth{0.803000pt}%
\definecolor{currentstroke}{rgb}{0.000000,0.000000,0.000000}%
\pgfsetstrokecolor{currentstroke}%
\pgfsetdash{}{0pt}%
\pgfsys@defobject{currentmarker}{\pgfqpoint{-0.048611in}{0.000000in}}{\pgfqpoint{-0.000000in}{0.000000in}}{%
\pgfpathmoveto{\pgfqpoint{-0.000000in}{0.000000in}}%
\pgfpathlineto{\pgfqpoint{-0.048611in}{0.000000in}}%
\pgfusepath{stroke,fill}%
}%
\begin{pgfscope}%
\pgfsys@transformshift{0.750000in}{2.194596in}%
\pgfsys@useobject{currentmarker}{}%
\end{pgfscope}%
\end{pgfscope}%
\begin{pgfscope}%
\definecolor{textcolor}{rgb}{0.000000,0.000000,0.000000}%
\pgfsetstrokecolor{textcolor}%
\pgfsetfillcolor{textcolor}%
\pgftext[x=0.343533in, y=2.141835in, left, base]{\color{textcolor}\sffamily\fontsize{10.000000}{12.000000}\selectfont 0.50}%
\end{pgfscope}%
\begin{pgfscope}%
\pgfsetbuttcap%
\pgfsetroundjoin%
\definecolor{currentfill}{rgb}{0.000000,0.000000,0.000000}%
\pgfsetfillcolor{currentfill}%
\pgfsetlinewidth{0.803000pt}%
\definecolor{currentstroke}{rgb}{0.000000,0.000000,0.000000}%
\pgfsetstrokecolor{currentstroke}%
\pgfsetdash{}{0pt}%
\pgfsys@defobject{currentmarker}{\pgfqpoint{-0.048611in}{0.000000in}}{\pgfqpoint{-0.000000in}{0.000000in}}{%
\pgfpathmoveto{\pgfqpoint{-0.000000in}{0.000000in}}%
\pgfpathlineto{\pgfqpoint{-0.048611in}{0.000000in}}%
\pgfusepath{stroke,fill}%
}%
\begin{pgfscope}%
\pgfsys@transformshift{0.750000in}{2.612971in}%
\pgfsys@useobject{currentmarker}{}%
\end{pgfscope}%
\end{pgfscope}%
\begin{pgfscope}%
\definecolor{textcolor}{rgb}{0.000000,0.000000,0.000000}%
\pgfsetstrokecolor{textcolor}%
\pgfsetfillcolor{textcolor}%
\pgftext[x=0.343533in, y=2.560209in, left, base]{\color{textcolor}\sffamily\fontsize{10.000000}{12.000000}\selectfont 0.55}%
\end{pgfscope}%
\begin{pgfscope}%
\pgfsetbuttcap%
\pgfsetroundjoin%
\definecolor{currentfill}{rgb}{0.000000,0.000000,0.000000}%
\pgfsetfillcolor{currentfill}%
\pgfsetlinewidth{0.803000pt}%
\definecolor{currentstroke}{rgb}{0.000000,0.000000,0.000000}%
\pgfsetstrokecolor{currentstroke}%
\pgfsetdash{}{0pt}%
\pgfsys@defobject{currentmarker}{\pgfqpoint{-0.048611in}{0.000000in}}{\pgfqpoint{-0.000000in}{0.000000in}}{%
\pgfpathmoveto{\pgfqpoint{-0.000000in}{0.000000in}}%
\pgfpathlineto{\pgfqpoint{-0.048611in}{0.000000in}}%
\pgfusepath{stroke,fill}%
}%
\begin{pgfscope}%
\pgfsys@transformshift{0.750000in}{3.031345in}%
\pgfsys@useobject{currentmarker}{}%
\end{pgfscope}%
\end{pgfscope}%
\begin{pgfscope}%
\definecolor{textcolor}{rgb}{0.000000,0.000000,0.000000}%
\pgfsetstrokecolor{textcolor}%
\pgfsetfillcolor{textcolor}%
\pgftext[x=0.343533in, y=2.978583in, left, base]{\color{textcolor}\sffamily\fontsize{10.000000}{12.000000}\selectfont 0.60}%
\end{pgfscope}%
\begin{pgfscope}%
\pgfsetbuttcap%
\pgfsetroundjoin%
\definecolor{currentfill}{rgb}{0.000000,0.000000,0.000000}%
\pgfsetfillcolor{currentfill}%
\pgfsetlinewidth{0.803000pt}%
\definecolor{currentstroke}{rgb}{0.000000,0.000000,0.000000}%
\pgfsetstrokecolor{currentstroke}%
\pgfsetdash{}{0pt}%
\pgfsys@defobject{currentmarker}{\pgfqpoint{-0.048611in}{0.000000in}}{\pgfqpoint{-0.000000in}{0.000000in}}{%
\pgfpathmoveto{\pgfqpoint{-0.000000in}{0.000000in}}%
\pgfpathlineto{\pgfqpoint{-0.048611in}{0.000000in}}%
\pgfusepath{stroke,fill}%
}%
\begin{pgfscope}%
\pgfsys@transformshift{0.750000in}{3.449719in}%
\pgfsys@useobject{currentmarker}{}%
\end{pgfscope}%
\end{pgfscope}%
\begin{pgfscope}%
\definecolor{textcolor}{rgb}{0.000000,0.000000,0.000000}%
\pgfsetstrokecolor{textcolor}%
\pgfsetfillcolor{textcolor}%
\pgftext[x=0.343533in, y=3.396957in, left, base]{\color{textcolor}\sffamily\fontsize{10.000000}{12.000000}\selectfont 0.65}%
\end{pgfscope}%
\begin{pgfscope}%
\definecolor{textcolor}{rgb}{0.000000,0.000000,0.000000}%
\pgfsetstrokecolor{textcolor}%
\pgfsetfillcolor{textcolor}%
\pgftext[x=0.287977in,y=2.010000in,,bottom,rotate=90.000000]{\color{textcolor}\sffamily\fontsize{10.000000}{12.000000}\selectfont loss function value}%
\end{pgfscope}%
\begin{pgfscope}%
\pgfpathrectangle{\pgfqpoint{0.750000in}{0.500000in}}{\pgfqpoint{4.650000in}{3.020000in}}%
\pgfusepath{clip}%
\pgfsetrectcap%
\pgfsetroundjoin%
\pgfsetlinewidth{1.505625pt}%
\definecolor{currentstroke}{rgb}{0.000000,0.500000,0.000000}%
\pgfsetstrokecolor{currentstroke}%
\pgfsetdash{}{0pt}%
\pgfpathmoveto{\pgfqpoint{0.961364in}{2.229295in}}%
\pgfpathlineto{\pgfqpoint{0.963055in}{2.177994in}}%
\pgfpathlineto{\pgfqpoint{0.964746in}{2.214499in}}%
\pgfpathlineto{\pgfqpoint{0.966437in}{2.191649in}}%
\pgfpathlineto{\pgfqpoint{0.967283in}{2.192411in}}%
\pgfpathlineto{\pgfqpoint{0.968129in}{2.198863in}}%
\pgfpathlineto{\pgfqpoint{0.968974in}{2.179408in}}%
\pgfpathlineto{\pgfqpoint{0.969820in}{2.218841in}}%
\pgfpathlineto{\pgfqpoint{0.970665in}{2.202247in}}%
\pgfpathlineto{\pgfqpoint{0.971511in}{2.180159in}}%
\pgfpathlineto{\pgfqpoint{0.972357in}{2.193619in}}%
\pgfpathlineto{\pgfqpoint{0.973202in}{2.243654in}}%
\pgfpathlineto{\pgfqpoint{0.974048in}{2.159676in}}%
\pgfpathlineto{\pgfqpoint{0.974894in}{2.247315in}}%
\pgfpathlineto{\pgfqpoint{0.975739in}{2.170288in}}%
\pgfpathlineto{\pgfqpoint{0.976585in}{2.180551in}}%
\pgfpathlineto{\pgfqpoint{0.977430in}{2.162617in}}%
\pgfpathlineto{\pgfqpoint{0.978276in}{2.202708in}}%
\pgfpathlineto{\pgfqpoint{0.979122in}{2.169539in}}%
\pgfpathlineto{\pgfqpoint{0.979967in}{2.208752in}}%
\pgfpathlineto{\pgfqpoint{0.980813in}{2.171261in}}%
\pgfpathlineto{\pgfqpoint{0.981659in}{2.226270in}}%
\pgfpathlineto{\pgfqpoint{0.982504in}{2.183675in}}%
\pgfpathlineto{\pgfqpoint{0.983350in}{2.204063in}}%
\pgfpathlineto{\pgfqpoint{0.985041in}{2.175568in}}%
\pgfpathlineto{\pgfqpoint{0.986732in}{2.205002in}}%
\pgfpathlineto{\pgfqpoint{0.988424in}{2.174228in}}%
\pgfpathlineto{\pgfqpoint{0.989269in}{2.222580in}}%
\pgfpathlineto{\pgfqpoint{0.990115in}{2.193372in}}%
\pgfpathlineto{\pgfqpoint{0.990960in}{2.191937in}}%
\pgfpathlineto{\pgfqpoint{0.991806in}{2.154671in}}%
\pgfpathlineto{\pgfqpoint{0.992652in}{2.170147in}}%
\pgfpathlineto{\pgfqpoint{0.995189in}{2.224284in}}%
\pgfpathlineto{\pgfqpoint{0.996034in}{2.162017in}}%
\pgfpathlineto{\pgfqpoint{0.996880in}{2.229071in}}%
\pgfpathlineto{\pgfqpoint{0.997725in}{2.196741in}}%
\pgfpathlineto{\pgfqpoint{0.998571in}{2.151416in}}%
\pgfpathlineto{\pgfqpoint{0.999417in}{2.263603in}}%
\pgfpathlineto{\pgfqpoint{1.000262in}{2.221183in}}%
\pgfpathlineto{\pgfqpoint{1.001108in}{2.183593in}}%
\pgfpathlineto{\pgfqpoint{1.001954in}{2.188596in}}%
\pgfpathlineto{\pgfqpoint{1.002799in}{2.178374in}}%
\pgfpathlineto{\pgfqpoint{1.003645in}{2.219545in}}%
\pgfpathlineto{\pgfqpoint{1.004490in}{2.218526in}}%
\pgfpathlineto{\pgfqpoint{1.005336in}{2.214497in}}%
\pgfpathlineto{\pgfqpoint{1.007027in}{2.226864in}}%
\pgfpathlineto{\pgfqpoint{1.008719in}{2.173197in}}%
\pgfpathlineto{\pgfqpoint{1.010410in}{2.230482in}}%
\pgfpathlineto{\pgfqpoint{1.011255in}{2.220058in}}%
\pgfpathlineto{\pgfqpoint{1.012101in}{2.158038in}}%
\pgfpathlineto{\pgfqpoint{1.012947in}{2.225551in}}%
\pgfpathlineto{\pgfqpoint{1.013792in}{2.183187in}}%
\pgfpathlineto{\pgfqpoint{1.014638in}{2.214582in}}%
\pgfpathlineto{\pgfqpoint{1.015484in}{2.183708in}}%
\pgfpathlineto{\pgfqpoint{1.016329in}{2.208290in}}%
\pgfpathlineto{\pgfqpoint{1.017175in}{2.216368in}}%
\pgfpathlineto{\pgfqpoint{1.018866in}{2.152535in}}%
\pgfpathlineto{\pgfqpoint{1.019712in}{2.210269in}}%
\pgfpathlineto{\pgfqpoint{1.020557in}{2.190454in}}%
\pgfpathlineto{\pgfqpoint{1.021403in}{2.185352in}}%
\pgfpathlineto{\pgfqpoint{1.022249in}{2.236024in}}%
\pgfpathlineto{\pgfqpoint{1.023940in}{2.175095in}}%
\pgfpathlineto{\pgfqpoint{1.024785in}{2.231890in}}%
\pgfpathlineto{\pgfqpoint{1.025631in}{2.196024in}}%
\pgfpathlineto{\pgfqpoint{1.026477in}{2.224165in}}%
\pgfpathlineto{\pgfqpoint{1.027322in}{2.222611in}}%
\pgfpathlineto{\pgfqpoint{1.029859in}{2.161374in}}%
\pgfpathlineto{\pgfqpoint{1.030705in}{2.212312in}}%
\pgfpathlineto{\pgfqpoint{1.031550in}{2.156686in}}%
\pgfpathlineto{\pgfqpoint{1.032396in}{2.189168in}}%
\pgfpathlineto{\pgfqpoint{1.034933in}{2.263655in}}%
\pgfpathlineto{\pgfqpoint{1.035779in}{2.181887in}}%
\pgfpathlineto{\pgfqpoint{1.036624in}{2.206351in}}%
\pgfpathlineto{\pgfqpoint{1.037470in}{2.240599in}}%
\pgfpathlineto{\pgfqpoint{1.040007in}{2.181948in}}%
\pgfpathlineto{\pgfqpoint{1.040852in}{2.217437in}}%
\pgfpathlineto{\pgfqpoint{1.041698in}{2.190934in}}%
\pgfpathlineto{\pgfqpoint{1.042544in}{2.220986in}}%
\pgfpathlineto{\pgfqpoint{1.044235in}{2.163451in}}%
\pgfpathlineto{\pgfqpoint{1.045080in}{2.243643in}}%
\pgfpathlineto{\pgfqpoint{1.045926in}{2.231434in}}%
\pgfpathlineto{\pgfqpoint{1.047617in}{2.165303in}}%
\pgfpathlineto{\pgfqpoint{1.048463in}{2.104649in}}%
\pgfpathlineto{\pgfqpoint{1.049308in}{2.275567in}}%
\pgfpathlineto{\pgfqpoint{1.050154in}{2.135624in}}%
\pgfpathlineto{\pgfqpoint{1.051000in}{2.098983in}}%
\pgfpathlineto{\pgfqpoint{1.051845in}{2.189427in}}%
\pgfpathlineto{\pgfqpoint{1.052691in}{2.180801in}}%
\pgfpathlineto{\pgfqpoint{1.054382in}{2.135337in}}%
\pgfpathlineto{\pgfqpoint{1.056073in}{2.257230in}}%
\pgfpathlineto{\pgfqpoint{1.057765in}{2.132229in}}%
\pgfpathlineto{\pgfqpoint{1.060302in}{2.251201in}}%
\pgfpathlineto{\pgfqpoint{1.061993in}{2.147566in}}%
\pgfpathlineto{\pgfqpoint{1.063684in}{2.213481in}}%
\pgfpathlineto{\pgfqpoint{1.064530in}{2.207112in}}%
\pgfpathlineto{\pgfqpoint{1.065375in}{2.102274in}}%
\pgfpathlineto{\pgfqpoint{1.067912in}{2.248237in}}%
\pgfpathlineto{\pgfqpoint{1.068758in}{2.199538in}}%
\pgfpathlineto{\pgfqpoint{1.069603in}{2.218512in}}%
\pgfpathlineto{\pgfqpoint{1.070449in}{2.215391in}}%
\pgfpathlineto{\pgfqpoint{1.071295in}{2.200176in}}%
\pgfpathlineto{\pgfqpoint{1.072140in}{2.201962in}}%
\pgfpathlineto{\pgfqpoint{1.072986in}{2.130584in}}%
\pgfpathlineto{\pgfqpoint{1.073832in}{2.189094in}}%
\pgfpathlineto{\pgfqpoint{1.074677in}{2.164675in}}%
\pgfpathlineto{\pgfqpoint{1.075523in}{2.242089in}}%
\pgfpathlineto{\pgfqpoint{1.076368in}{2.109607in}}%
\pgfpathlineto{\pgfqpoint{1.077214in}{2.173403in}}%
\pgfpathlineto{\pgfqpoint{1.078060in}{2.144237in}}%
\pgfpathlineto{\pgfqpoint{1.079751in}{2.245019in}}%
\pgfpathlineto{\pgfqpoint{1.080597in}{2.209652in}}%
\pgfpathlineto{\pgfqpoint{1.081442in}{2.200447in}}%
\pgfpathlineto{\pgfqpoint{1.083133in}{2.227518in}}%
\pgfpathlineto{\pgfqpoint{1.084825in}{2.141830in}}%
\pgfpathlineto{\pgfqpoint{1.085670in}{2.183615in}}%
\pgfpathlineto{\pgfqpoint{1.086516in}{2.179992in}}%
\pgfpathlineto{\pgfqpoint{1.087362in}{2.169485in}}%
\pgfpathlineto{\pgfqpoint{1.089053in}{2.257337in}}%
\pgfpathlineto{\pgfqpoint{1.089898in}{2.118299in}}%
\pgfpathlineto{\pgfqpoint{1.090744in}{2.256648in}}%
\pgfpathlineto{\pgfqpoint{1.091590in}{2.118648in}}%
\pgfpathlineto{\pgfqpoint{1.092435in}{2.154672in}}%
\pgfpathlineto{\pgfqpoint{1.094972in}{2.220979in}}%
\pgfpathlineto{\pgfqpoint{1.096663in}{2.122583in}}%
\pgfpathlineto{\pgfqpoint{1.097509in}{2.266689in}}%
\pgfpathlineto{\pgfqpoint{1.098355in}{2.243114in}}%
\pgfpathlineto{\pgfqpoint{1.100046in}{2.156288in}}%
\pgfpathlineto{\pgfqpoint{1.100892in}{2.322447in}}%
\pgfpathlineto{\pgfqpoint{1.102583in}{2.094052in}}%
\pgfpathlineto{\pgfqpoint{1.105120in}{2.184156in}}%
\pgfpathlineto{\pgfqpoint{1.105965in}{2.196009in}}%
\pgfpathlineto{\pgfqpoint{1.106811in}{2.034189in}}%
\pgfpathlineto{\pgfqpoint{1.107657in}{2.224500in}}%
\pgfpathlineto{\pgfqpoint{1.108502in}{2.205799in}}%
\pgfpathlineto{\pgfqpoint{1.109348in}{2.111574in}}%
\pgfpathlineto{\pgfqpoint{1.110193in}{2.137293in}}%
\pgfpathlineto{\pgfqpoint{1.111039in}{2.164838in}}%
\pgfpathlineto{\pgfqpoint{1.111885in}{2.260466in}}%
\pgfpathlineto{\pgfqpoint{1.112730in}{2.246285in}}%
\pgfpathlineto{\pgfqpoint{1.113576in}{2.135841in}}%
\pgfpathlineto{\pgfqpoint{1.114422in}{2.169382in}}%
\pgfpathlineto{\pgfqpoint{1.115267in}{2.243569in}}%
\pgfpathlineto{\pgfqpoint{1.116113in}{2.162728in}}%
\pgfpathlineto{\pgfqpoint{1.116958in}{2.184891in}}%
\pgfpathlineto{\pgfqpoint{1.117804in}{2.152415in}}%
\pgfpathlineto{\pgfqpoint{1.119495in}{2.223249in}}%
\pgfpathlineto{\pgfqpoint{1.120341in}{2.169059in}}%
\pgfpathlineto{\pgfqpoint{1.121187in}{2.210116in}}%
\pgfpathlineto{\pgfqpoint{1.122032in}{2.221051in}}%
\pgfpathlineto{\pgfqpoint{1.122878in}{2.173343in}}%
\pgfpathlineto{\pgfqpoint{1.123723in}{2.306207in}}%
\pgfpathlineto{\pgfqpoint{1.124569in}{2.276817in}}%
\pgfpathlineto{\pgfqpoint{1.125415in}{2.269969in}}%
\pgfpathlineto{\pgfqpoint{1.126260in}{2.296280in}}%
\pgfpathlineto{\pgfqpoint{1.127951in}{2.138428in}}%
\pgfpathlineto{\pgfqpoint{1.129643in}{2.243841in}}%
\pgfpathlineto{\pgfqpoint{1.131334in}{2.109617in}}%
\pgfpathlineto{\pgfqpoint{1.132180in}{2.168151in}}%
\pgfpathlineto{\pgfqpoint{1.133871in}{2.339209in}}%
\pgfpathlineto{\pgfqpoint{1.134716in}{2.043730in}}%
\pgfpathlineto{\pgfqpoint{1.135562in}{2.084756in}}%
\pgfpathlineto{\pgfqpoint{1.136408in}{2.258116in}}%
\pgfpathlineto{\pgfqpoint{1.137253in}{2.129966in}}%
\pgfpathlineto{\pgfqpoint{1.139790in}{2.212507in}}%
\pgfpathlineto{\pgfqpoint{1.140636in}{2.203260in}}%
\pgfpathlineto{\pgfqpoint{1.141481in}{2.034063in}}%
\pgfpathlineto{\pgfqpoint{1.142327in}{2.369274in}}%
\pgfpathlineto{\pgfqpoint{1.143173in}{2.196376in}}%
\pgfpathlineto{\pgfqpoint{1.144018in}{2.098414in}}%
\pgfpathlineto{\pgfqpoint{1.144864in}{2.234718in}}%
\pgfpathlineto{\pgfqpoint{1.145710in}{2.185910in}}%
\pgfpathlineto{\pgfqpoint{1.146555in}{2.149588in}}%
\pgfpathlineto{\pgfqpoint{1.147401in}{2.253176in}}%
\pgfpathlineto{\pgfqpoint{1.148246in}{2.248787in}}%
\pgfpathlineto{\pgfqpoint{1.149092in}{2.133326in}}%
\pgfpathlineto{\pgfqpoint{1.149938in}{2.200748in}}%
\pgfpathlineto{\pgfqpoint{1.150783in}{2.189277in}}%
\pgfpathlineto{\pgfqpoint{1.151629in}{2.224249in}}%
\pgfpathlineto{\pgfqpoint{1.152475in}{2.110551in}}%
\pgfpathlineto{\pgfqpoint{1.153320in}{2.170739in}}%
\pgfpathlineto{\pgfqpoint{1.154166in}{2.158195in}}%
\pgfpathlineto{\pgfqpoint{1.155857in}{2.014468in}}%
\pgfpathlineto{\pgfqpoint{1.157548in}{2.374493in}}%
\pgfpathlineto{\pgfqpoint{1.158394in}{2.064658in}}%
\pgfpathlineto{\pgfqpoint{1.159240in}{2.251950in}}%
\pgfpathlineto{\pgfqpoint{1.160931in}{2.123049in}}%
\pgfpathlineto{\pgfqpoint{1.161776in}{2.232534in}}%
\pgfpathlineto{\pgfqpoint{1.162622in}{2.229212in}}%
\pgfpathlineto{\pgfqpoint{1.164313in}{2.210704in}}%
\pgfpathlineto{\pgfqpoint{1.166005in}{2.313183in}}%
\pgfpathlineto{\pgfqpoint{1.166850in}{2.067805in}}%
\pgfpathlineto{\pgfqpoint{1.167696in}{2.339876in}}%
\pgfpathlineto{\pgfqpoint{1.168541in}{2.116556in}}%
\pgfpathlineto{\pgfqpoint{1.169387in}{2.125142in}}%
\pgfpathlineto{\pgfqpoint{1.170233in}{2.190231in}}%
\pgfpathlineto{\pgfqpoint{1.171078in}{2.103561in}}%
\pgfpathlineto{\pgfqpoint{1.171924in}{2.144968in}}%
\pgfpathlineto{\pgfqpoint{1.172770in}{2.213469in}}%
\pgfpathlineto{\pgfqpoint{1.174461in}{2.072062in}}%
\pgfpathlineto{\pgfqpoint{1.175306in}{2.327653in}}%
\pgfpathlineto{\pgfqpoint{1.176152in}{2.173174in}}%
\pgfpathlineto{\pgfqpoint{1.177843in}{2.389816in}}%
\pgfpathlineto{\pgfqpoint{1.178689in}{2.104600in}}%
\pgfpathlineto{\pgfqpoint{1.179535in}{2.218853in}}%
\pgfpathlineto{\pgfqpoint{1.181226in}{2.072093in}}%
\pgfpathlineto{\pgfqpoint{1.182071in}{2.267135in}}%
\pgfpathlineto{\pgfqpoint{1.182917in}{2.061100in}}%
\pgfpathlineto{\pgfqpoint{1.183763in}{2.147154in}}%
\pgfpathlineto{\pgfqpoint{1.185454in}{2.068948in}}%
\pgfpathlineto{\pgfqpoint{1.186300in}{2.355496in}}%
\pgfpathlineto{\pgfqpoint{1.188836in}{2.030056in}}%
\pgfpathlineto{\pgfqpoint{1.190528in}{2.360423in}}%
\pgfpathlineto{\pgfqpoint{1.193065in}{2.086456in}}%
\pgfpathlineto{\pgfqpoint{1.194756in}{2.019554in}}%
\pgfpathlineto{\pgfqpoint{1.195601in}{2.136165in}}%
\pgfpathlineto{\pgfqpoint{1.196447in}{2.012520in}}%
\pgfpathlineto{\pgfqpoint{1.197293in}{2.110337in}}%
\pgfpathlineto{\pgfqpoint{1.198138in}{2.255635in}}%
\pgfpathlineto{\pgfqpoint{1.198984in}{2.148483in}}%
\pgfpathlineto{\pgfqpoint{1.200675in}{2.292338in}}%
\pgfpathlineto{\pgfqpoint{1.202366in}{2.142483in}}%
\pgfpathlineto{\pgfqpoint{1.203212in}{2.458239in}}%
\pgfpathlineto{\pgfqpoint{1.204058in}{2.035355in}}%
\pgfpathlineto{\pgfqpoint{1.204903in}{2.293942in}}%
\pgfpathlineto{\pgfqpoint{1.206595in}{2.001604in}}%
\pgfpathlineto{\pgfqpoint{1.207440in}{2.181769in}}%
\pgfpathlineto{\pgfqpoint{1.208286in}{2.070586in}}%
\pgfpathlineto{\pgfqpoint{1.209131in}{2.005361in}}%
\pgfpathlineto{\pgfqpoint{1.209977in}{2.310669in}}%
\pgfpathlineto{\pgfqpoint{1.210823in}{2.049931in}}%
\pgfpathlineto{\pgfqpoint{1.211668in}{2.272596in}}%
\pgfpathlineto{\pgfqpoint{1.212514in}{2.210834in}}%
\pgfpathlineto{\pgfqpoint{1.213359in}{1.907448in}}%
\pgfpathlineto{\pgfqpoint{1.214205in}{2.084827in}}%
\pgfpathlineto{\pgfqpoint{1.216742in}{2.267168in}}%
\pgfpathlineto{\pgfqpoint{1.217588in}{2.245563in}}%
\pgfpathlineto{\pgfqpoint{1.219279in}{2.027385in}}%
\pgfpathlineto{\pgfqpoint{1.221816in}{2.377561in}}%
\pgfpathlineto{\pgfqpoint{1.222661in}{2.059260in}}%
\pgfpathlineto{\pgfqpoint{1.223507in}{2.746106in}}%
\pgfpathlineto{\pgfqpoint{1.226044in}{1.977559in}}%
\pgfpathlineto{\pgfqpoint{1.228581in}{2.336797in}}%
\pgfpathlineto{\pgfqpoint{1.229426in}{2.014722in}}%
\pgfpathlineto{\pgfqpoint{1.230272in}{2.253323in}}%
\pgfpathlineto{\pgfqpoint{1.231118in}{2.086711in}}%
\pgfpathlineto{\pgfqpoint{1.231963in}{2.217927in}}%
\pgfpathlineto{\pgfqpoint{1.232809in}{2.434597in}}%
\pgfpathlineto{\pgfqpoint{1.233654in}{2.067158in}}%
\pgfpathlineto{\pgfqpoint{1.234500in}{2.240341in}}%
\pgfpathlineto{\pgfqpoint{1.235346in}{2.113457in}}%
\pgfpathlineto{\pgfqpoint{1.236191in}{2.342114in}}%
\pgfpathlineto{\pgfqpoint{1.237037in}{2.294010in}}%
\pgfpathlineto{\pgfqpoint{1.238728in}{1.846388in}}%
\pgfpathlineto{\pgfqpoint{1.239574in}{2.331404in}}%
\pgfpathlineto{\pgfqpoint{1.240419in}{2.021881in}}%
\pgfpathlineto{\pgfqpoint{1.241265in}{1.923751in}}%
\pgfpathlineto{\pgfqpoint{1.242111in}{2.153314in}}%
\pgfpathlineto{\pgfqpoint{1.242956in}{1.753400in}}%
\pgfpathlineto{\pgfqpoint{1.243802in}{2.116800in}}%
\pgfpathlineto{\pgfqpoint{1.244648in}{2.033587in}}%
\pgfpathlineto{\pgfqpoint{1.245493in}{2.004781in}}%
\pgfpathlineto{\pgfqpoint{1.248030in}{2.385606in}}%
\pgfpathlineto{\pgfqpoint{1.249721in}{2.088054in}}%
\pgfpathlineto{\pgfqpoint{1.250567in}{2.342836in}}%
\pgfpathlineto{\pgfqpoint{1.251413in}{2.054455in}}%
\pgfpathlineto{\pgfqpoint{1.253104in}{2.778536in}}%
\pgfpathlineto{\pgfqpoint{1.254795in}{1.937408in}}%
\pgfpathlineto{\pgfqpoint{1.255641in}{1.957463in}}%
\pgfpathlineto{\pgfqpoint{1.256486in}{2.267161in}}%
\pgfpathlineto{\pgfqpoint{1.257332in}{1.856368in}}%
\pgfpathlineto{\pgfqpoint{1.258178in}{2.362221in}}%
\pgfpathlineto{\pgfqpoint{1.259023in}{2.162710in}}%
\pgfpathlineto{\pgfqpoint{1.259869in}{2.271018in}}%
\pgfpathlineto{\pgfqpoint{1.260714in}{2.195231in}}%
\pgfpathlineto{\pgfqpoint{1.261560in}{2.025946in}}%
\pgfpathlineto{\pgfqpoint{1.264097in}{2.382622in}}%
\pgfpathlineto{\pgfqpoint{1.266634in}{1.991777in}}%
\pgfpathlineto{\pgfqpoint{1.267479in}{2.013350in}}%
\pgfpathlineto{\pgfqpoint{1.268325in}{2.254415in}}%
\pgfpathlineto{\pgfqpoint{1.269171in}{2.025119in}}%
\pgfpathlineto{\pgfqpoint{1.270016in}{2.322764in}}%
\pgfpathlineto{\pgfqpoint{1.270862in}{2.035633in}}%
\pgfpathlineto{\pgfqpoint{1.271708in}{2.163021in}}%
\pgfpathlineto{\pgfqpoint{1.272553in}{2.090684in}}%
\pgfpathlineto{\pgfqpoint{1.273399in}{2.098522in}}%
\pgfpathlineto{\pgfqpoint{1.274244in}{2.919513in}}%
\pgfpathlineto{\pgfqpoint{1.275090in}{2.027329in}}%
\pgfpathlineto{\pgfqpoint{1.275936in}{2.608765in}}%
\pgfpathlineto{\pgfqpoint{1.280164in}{2.112633in}}%
\pgfpathlineto{\pgfqpoint{1.282701in}{2.478677in}}%
\pgfpathlineto{\pgfqpoint{1.283546in}{2.056408in}}%
\pgfpathlineto{\pgfqpoint{1.284392in}{2.344917in}}%
\pgfpathlineto{\pgfqpoint{1.285238in}{2.295269in}}%
\pgfpathlineto{\pgfqpoint{1.286083in}{2.469610in}}%
\pgfpathlineto{\pgfqpoint{1.288620in}{2.081690in}}%
\pgfpathlineto{\pgfqpoint{1.289466in}{2.032877in}}%
\pgfpathlineto{\pgfqpoint{1.290311in}{2.494403in}}%
\pgfpathlineto{\pgfqpoint{1.292002in}{2.034620in}}%
\pgfpathlineto{\pgfqpoint{1.292848in}{2.193606in}}%
\pgfpathlineto{\pgfqpoint{1.293694in}{1.987776in}}%
\pgfpathlineto{\pgfqpoint{1.294539in}{2.430387in}}%
\pgfpathlineto{\pgfqpoint{1.295385in}{2.288085in}}%
\pgfpathlineto{\pgfqpoint{1.296231in}{2.243612in}}%
\pgfpathlineto{\pgfqpoint{1.297076in}{1.929637in}}%
\pgfpathlineto{\pgfqpoint{1.297922in}{2.041251in}}%
\pgfpathlineto{\pgfqpoint{1.298767in}{2.362178in}}%
\pgfpathlineto{\pgfqpoint{1.299613in}{2.163382in}}%
\pgfpathlineto{\pgfqpoint{1.302150in}{1.975806in}}%
\pgfpathlineto{\pgfqpoint{1.302996in}{2.303867in}}%
\pgfpathlineto{\pgfqpoint{1.303841in}{2.108422in}}%
\pgfpathlineto{\pgfqpoint{1.305532in}{2.250515in}}%
\pgfpathlineto{\pgfqpoint{1.306378in}{2.247336in}}%
\pgfpathlineto{\pgfqpoint{1.307224in}{2.052704in}}%
\pgfpathlineto{\pgfqpoint{1.308069in}{2.160372in}}%
\pgfpathlineto{\pgfqpoint{1.308915in}{2.196535in}}%
\pgfpathlineto{\pgfqpoint{1.309761in}{2.100957in}}%
\pgfpathlineto{\pgfqpoint{1.311452in}{2.298462in}}%
\pgfpathlineto{\pgfqpoint{1.312297in}{2.140765in}}%
\pgfpathlineto{\pgfqpoint{1.313989in}{2.368749in}}%
\pgfpathlineto{\pgfqpoint{1.315680in}{2.058473in}}%
\pgfpathlineto{\pgfqpoint{1.316526in}{2.081746in}}%
\pgfpathlineto{\pgfqpoint{1.317371in}{2.346910in}}%
\pgfpathlineto{\pgfqpoint{1.318217in}{2.190358in}}%
\pgfpathlineto{\pgfqpoint{1.319908in}{2.384023in}}%
\pgfpathlineto{\pgfqpoint{1.322445in}{1.999766in}}%
\pgfpathlineto{\pgfqpoint{1.323291in}{2.088079in}}%
\pgfpathlineto{\pgfqpoint{1.324136in}{2.339449in}}%
\pgfpathlineto{\pgfqpoint{1.326673in}{1.980928in}}%
\pgfpathlineto{\pgfqpoint{1.327519in}{2.554024in}}%
\pgfpathlineto{\pgfqpoint{1.328364in}{2.046395in}}%
\pgfpathlineto{\pgfqpoint{1.329210in}{2.246120in}}%
\pgfpathlineto{\pgfqpoint{1.330056in}{2.336771in}}%
\pgfpathlineto{\pgfqpoint{1.330901in}{2.073423in}}%
\pgfpathlineto{\pgfqpoint{1.331747in}{2.348612in}}%
\pgfpathlineto{\pgfqpoint{1.332592in}{2.188396in}}%
\pgfpathlineto{\pgfqpoint{1.333438in}{2.003779in}}%
\pgfpathlineto{\pgfqpoint{1.334284in}{2.068286in}}%
\pgfpathlineto{\pgfqpoint{1.335129in}{2.206828in}}%
\pgfpathlineto{\pgfqpoint{1.335975in}{1.924458in}}%
\pgfpathlineto{\pgfqpoint{1.336821in}{2.292586in}}%
\pgfpathlineto{\pgfqpoint{1.337666in}{1.894929in}}%
\pgfpathlineto{\pgfqpoint{1.339357in}{2.403035in}}%
\pgfpathlineto{\pgfqpoint{1.340203in}{2.025014in}}%
\pgfpathlineto{\pgfqpoint{1.341049in}{2.352374in}}%
\pgfpathlineto{\pgfqpoint{1.341894in}{2.146631in}}%
\pgfpathlineto{\pgfqpoint{1.342740in}{2.261051in}}%
\pgfpathlineto{\pgfqpoint{1.343586in}{2.212609in}}%
\pgfpathlineto{\pgfqpoint{1.344431in}{2.227626in}}%
\pgfpathlineto{\pgfqpoint{1.345277in}{1.968282in}}%
\pgfpathlineto{\pgfqpoint{1.346122in}{2.165250in}}%
\pgfpathlineto{\pgfqpoint{1.346968in}{2.009060in}}%
\pgfpathlineto{\pgfqpoint{1.347814in}{2.341259in}}%
\pgfpathlineto{\pgfqpoint{1.348659in}{2.338206in}}%
\pgfpathlineto{\pgfqpoint{1.349505in}{2.277568in}}%
\pgfpathlineto{\pgfqpoint{1.350351in}{2.322086in}}%
\pgfpathlineto{\pgfqpoint{1.351196in}{2.301360in}}%
\pgfpathlineto{\pgfqpoint{1.352042in}{2.227758in}}%
\pgfpathlineto{\pgfqpoint{1.352887in}{2.315106in}}%
\pgfpathlineto{\pgfqpoint{1.353733in}{2.235966in}}%
\pgfpathlineto{\pgfqpoint{1.354579in}{1.863967in}}%
\pgfpathlineto{\pgfqpoint{1.355424in}{2.060454in}}%
\pgfpathlineto{\pgfqpoint{1.356270in}{2.412792in}}%
\pgfpathlineto{\pgfqpoint{1.357116in}{2.054848in}}%
\pgfpathlineto{\pgfqpoint{1.357961in}{2.468015in}}%
\pgfpathlineto{\pgfqpoint{1.358807in}{2.213238in}}%
\pgfpathlineto{\pgfqpoint{1.359652in}{1.825377in}}%
\pgfpathlineto{\pgfqpoint{1.360498in}{1.874981in}}%
\pgfpathlineto{\pgfqpoint{1.361344in}{1.845153in}}%
\pgfpathlineto{\pgfqpoint{1.362189in}{2.217311in}}%
\pgfpathlineto{\pgfqpoint{1.363035in}{2.118573in}}%
\pgfpathlineto{\pgfqpoint{1.363881in}{2.197429in}}%
\pgfpathlineto{\pgfqpoint{1.364726in}{2.143722in}}%
\pgfpathlineto{\pgfqpoint{1.365572in}{2.153007in}}%
\pgfpathlineto{\pgfqpoint{1.366417in}{2.386564in}}%
\pgfpathlineto{\pgfqpoint{1.367263in}{2.278271in}}%
\pgfpathlineto{\pgfqpoint{1.368109in}{2.082935in}}%
\pgfpathlineto{\pgfqpoint{1.370645in}{2.440794in}}%
\pgfpathlineto{\pgfqpoint{1.371491in}{2.109757in}}%
\pgfpathlineto{\pgfqpoint{1.372337in}{2.404764in}}%
\pgfpathlineto{\pgfqpoint{1.373182in}{2.165957in}}%
\pgfpathlineto{\pgfqpoint{1.374028in}{2.185115in}}%
\pgfpathlineto{\pgfqpoint{1.374874in}{2.123625in}}%
\pgfpathlineto{\pgfqpoint{1.375719in}{2.492554in}}%
\pgfpathlineto{\pgfqpoint{1.376565in}{2.304601in}}%
\pgfpathlineto{\pgfqpoint{1.378256in}{2.141103in}}%
\pgfpathlineto{\pgfqpoint{1.379102in}{2.537057in}}%
\pgfpathlineto{\pgfqpoint{1.379947in}{2.022437in}}%
\pgfpathlineto{\pgfqpoint{1.380793in}{2.170033in}}%
\pgfpathlineto{\pgfqpoint{1.381639in}{2.140288in}}%
\pgfpathlineto{\pgfqpoint{1.383330in}{1.533204in}}%
\pgfpathlineto{\pgfqpoint{1.385021in}{2.165790in}}%
\pgfpathlineto{\pgfqpoint{1.385867in}{2.143414in}}%
\pgfpathlineto{\pgfqpoint{1.386712in}{2.165996in}}%
\pgfpathlineto{\pgfqpoint{1.387558in}{1.918128in}}%
\pgfpathlineto{\pgfqpoint{1.388404in}{2.604881in}}%
\pgfpathlineto{\pgfqpoint{1.389249in}{2.000379in}}%
\pgfpathlineto{\pgfqpoint{1.390095in}{2.142567in}}%
\pgfpathlineto{\pgfqpoint{1.391786in}{2.690306in}}%
\pgfpathlineto{\pgfqpoint{1.393477in}{2.194289in}}%
\pgfpathlineto{\pgfqpoint{1.394323in}{2.550678in}}%
\pgfpathlineto{\pgfqpoint{1.395169in}{2.362377in}}%
\pgfpathlineto{\pgfqpoint{1.396860in}{1.993107in}}%
\pgfpathlineto{\pgfqpoint{1.398551in}{2.410216in}}%
\pgfpathlineto{\pgfqpoint{1.399397in}{2.286795in}}%
\pgfpathlineto{\pgfqpoint{1.400242in}{2.551385in}}%
\pgfpathlineto{\pgfqpoint{1.401088in}{1.978757in}}%
\pgfpathlineto{\pgfqpoint{1.401934in}{2.547740in}}%
\pgfpathlineto{\pgfqpoint{1.402779in}{1.826870in}}%
\pgfpathlineto{\pgfqpoint{1.403625in}{2.331527in}}%
\pgfpathlineto{\pgfqpoint{1.404470in}{2.103073in}}%
\pgfpathlineto{\pgfqpoint{1.405316in}{2.553735in}}%
\pgfpathlineto{\pgfqpoint{1.407853in}{1.906615in}}%
\pgfpathlineto{\pgfqpoint{1.408699in}{2.214274in}}%
\pgfpathlineto{\pgfqpoint{1.409544in}{1.997680in}}%
\pgfpathlineto{\pgfqpoint{1.410390in}{2.050912in}}%
\pgfpathlineto{\pgfqpoint{1.412927in}{2.264356in}}%
\pgfpathlineto{\pgfqpoint{1.413772in}{2.257185in}}%
\pgfpathlineto{\pgfqpoint{1.414618in}{2.055887in}}%
\pgfpathlineto{\pgfqpoint{1.415464in}{2.445286in}}%
\pgfpathlineto{\pgfqpoint{1.416309in}{1.976691in}}%
\pgfpathlineto{\pgfqpoint{1.417155in}{2.309981in}}%
\pgfpathlineto{\pgfqpoint{1.418000in}{2.185366in}}%
\pgfpathlineto{\pgfqpoint{1.418846in}{1.717516in}}%
\pgfpathlineto{\pgfqpoint{1.420537in}{2.792106in}}%
\pgfpathlineto{\pgfqpoint{1.421383in}{1.795094in}}%
\pgfpathlineto{\pgfqpoint{1.422229in}{2.201930in}}%
\pgfpathlineto{\pgfqpoint{1.423920in}{1.755889in}}%
\pgfpathlineto{\pgfqpoint{1.424765in}{1.978485in}}%
\pgfpathlineto{\pgfqpoint{1.425611in}{1.793054in}}%
\pgfpathlineto{\pgfqpoint{1.426457in}{2.157403in}}%
\pgfpathlineto{\pgfqpoint{1.427302in}{2.031401in}}%
\pgfpathlineto{\pgfqpoint{1.428148in}{2.120777in}}%
\pgfpathlineto{\pgfqpoint{1.428994in}{2.645494in}}%
\pgfpathlineto{\pgfqpoint{1.429839in}{2.139109in}}%
\pgfpathlineto{\pgfqpoint{1.430685in}{2.445501in}}%
\pgfpathlineto{\pgfqpoint{1.434067in}{1.935084in}}%
\pgfpathlineto{\pgfqpoint{1.434913in}{2.328699in}}%
\pgfpathlineto{\pgfqpoint{1.435759in}{2.286809in}}%
\pgfpathlineto{\pgfqpoint{1.436604in}{2.364369in}}%
\pgfpathlineto{\pgfqpoint{1.437450in}{2.118647in}}%
\pgfpathlineto{\pgfqpoint{1.439987in}{2.540933in}}%
\pgfpathlineto{\pgfqpoint{1.440832in}{2.190494in}}%
\pgfpathlineto{\pgfqpoint{1.441678in}{2.496780in}}%
\pgfpathlineto{\pgfqpoint{1.443369in}{2.148703in}}%
\pgfpathlineto{\pgfqpoint{1.445060in}{2.535962in}}%
\pgfpathlineto{\pgfqpoint{1.445906in}{2.257996in}}%
\pgfpathlineto{\pgfqpoint{1.446752in}{2.347632in}}%
\pgfpathlineto{\pgfqpoint{1.448443in}{2.197352in}}%
\pgfpathlineto{\pgfqpoint{1.449288in}{1.689206in}}%
\pgfpathlineto{\pgfqpoint{1.450134in}{2.246195in}}%
\pgfpathlineto{\pgfqpoint{1.450980in}{2.236655in}}%
\pgfpathlineto{\pgfqpoint{1.453517in}{1.559859in}}%
\pgfpathlineto{\pgfqpoint{1.455208in}{2.138567in}}%
\pgfpathlineto{\pgfqpoint{1.456053in}{2.031585in}}%
\pgfpathlineto{\pgfqpoint{1.456899in}{2.495257in}}%
\pgfpathlineto{\pgfqpoint{1.457745in}{2.296887in}}%
\pgfpathlineto{\pgfqpoint{1.458590in}{2.244175in}}%
\pgfpathlineto{\pgfqpoint{1.459436in}{2.062174in}}%
\pgfpathlineto{\pgfqpoint{1.460282in}{2.706655in}}%
\pgfpathlineto{\pgfqpoint{1.461973in}{1.788797in}}%
\pgfpathlineto{\pgfqpoint{1.462818in}{1.986903in}}%
\pgfpathlineto{\pgfqpoint{1.463664in}{2.483815in}}%
\pgfpathlineto{\pgfqpoint{1.464510in}{1.793582in}}%
\pgfpathlineto{\pgfqpoint{1.465355in}{1.978347in}}%
\pgfpathlineto{\pgfqpoint{1.466201in}{1.711745in}}%
\pgfpathlineto{\pgfqpoint{1.467047in}{2.394738in}}%
\pgfpathlineto{\pgfqpoint{1.467892in}{2.075087in}}%
\pgfpathlineto{\pgfqpoint{1.469583in}{2.347970in}}%
\pgfpathlineto{\pgfqpoint{1.472120in}{1.953949in}}%
\pgfpathlineto{\pgfqpoint{1.472966in}{2.423589in}}%
\pgfpathlineto{\pgfqpoint{1.473812in}{1.894044in}}%
\pgfpathlineto{\pgfqpoint{1.474657in}{1.931691in}}%
\pgfpathlineto{\pgfqpoint{1.475503in}{2.392755in}}%
\pgfpathlineto{\pgfqpoint{1.476348in}{2.036400in}}%
\pgfpathlineto{\pgfqpoint{1.479731in}{2.576862in}}%
\pgfpathlineto{\pgfqpoint{1.482268in}{1.921481in}}%
\pgfpathlineto{\pgfqpoint{1.483113in}{2.467453in}}%
\pgfpathlineto{\pgfqpoint{1.483959in}{2.209423in}}%
\pgfpathlineto{\pgfqpoint{1.484805in}{2.385709in}}%
\pgfpathlineto{\pgfqpoint{1.485650in}{2.040871in}}%
\pgfpathlineto{\pgfqpoint{1.486496in}{2.534730in}}%
\pgfpathlineto{\pgfqpoint{1.487342in}{2.306647in}}%
\pgfpathlineto{\pgfqpoint{1.488187in}{2.221744in}}%
\pgfpathlineto{\pgfqpoint{1.489033in}{2.460847in}}%
\pgfpathlineto{\pgfqpoint{1.491570in}{1.763814in}}%
\pgfpathlineto{\pgfqpoint{1.493261in}{2.721592in}}%
\pgfpathlineto{\pgfqpoint{1.494107in}{2.202902in}}%
\pgfpathlineto{\pgfqpoint{1.494952in}{2.431395in}}%
\pgfpathlineto{\pgfqpoint{1.497489in}{2.105299in}}%
\pgfpathlineto{\pgfqpoint{1.498335in}{2.237504in}}%
\pgfpathlineto{\pgfqpoint{1.499180in}{2.060009in}}%
\pgfpathlineto{\pgfqpoint{1.500026in}{2.303541in}}%
\pgfpathlineto{\pgfqpoint{1.500872in}{2.203591in}}%
\pgfpathlineto{\pgfqpoint{1.502563in}{1.867817in}}%
\pgfpathlineto{\pgfqpoint{1.504254in}{2.261595in}}%
\pgfpathlineto{\pgfqpoint{1.505100in}{2.653321in}}%
\pgfpathlineto{\pgfqpoint{1.505945in}{2.030811in}}%
\pgfpathlineto{\pgfqpoint{1.506791in}{2.252664in}}%
\pgfpathlineto{\pgfqpoint{1.507637in}{2.614927in}}%
\pgfpathlineto{\pgfqpoint{1.508482in}{2.080535in}}%
\pgfpathlineto{\pgfqpoint{1.509328in}{2.089914in}}%
\pgfpathlineto{\pgfqpoint{1.510173in}{2.441127in}}%
\pgfpathlineto{\pgfqpoint{1.511019in}{2.049712in}}%
\pgfpathlineto{\pgfqpoint{1.511865in}{2.080944in}}%
\pgfpathlineto{\pgfqpoint{1.513556in}{2.397930in}}%
\pgfpathlineto{\pgfqpoint{1.514402in}{2.099175in}}%
\pgfpathlineto{\pgfqpoint{1.515247in}{2.368116in}}%
\pgfpathlineto{\pgfqpoint{1.516093in}{2.194170in}}%
\pgfpathlineto{\pgfqpoint{1.516938in}{1.883754in}}%
\pgfpathlineto{\pgfqpoint{1.517784in}{2.015609in}}%
\pgfpathlineto{\pgfqpoint{1.518630in}{2.051268in}}%
\pgfpathlineto{\pgfqpoint{1.519475in}{1.971488in}}%
\pgfpathlineto{\pgfqpoint{1.520321in}{2.645935in}}%
\pgfpathlineto{\pgfqpoint{1.522858in}{1.874988in}}%
\pgfpathlineto{\pgfqpoint{1.524549in}{2.486364in}}%
\pgfpathlineto{\pgfqpoint{1.525395in}{2.074272in}}%
\pgfpathlineto{\pgfqpoint{1.526240in}{2.378478in}}%
\pgfpathlineto{\pgfqpoint{1.528777in}{2.076538in}}%
\pgfpathlineto{\pgfqpoint{1.529623in}{2.953308in}}%
\pgfpathlineto{\pgfqpoint{1.530468in}{1.501166in}}%
\pgfpathlineto{\pgfqpoint{1.531314in}{1.858020in}}%
\pgfpathlineto{\pgfqpoint{1.532160in}{2.719619in}}%
\pgfpathlineto{\pgfqpoint{1.533005in}{2.289381in}}%
\pgfpathlineto{\pgfqpoint{1.533851in}{1.909703in}}%
\pgfpathlineto{\pgfqpoint{1.534696in}{1.910764in}}%
\pgfpathlineto{\pgfqpoint{1.536388in}{2.535077in}}%
\pgfpathlineto{\pgfqpoint{1.537233in}{2.095326in}}%
\pgfpathlineto{\pgfqpoint{1.538079in}{2.252801in}}%
\pgfpathlineto{\pgfqpoint{1.539770in}{1.993691in}}%
\pgfpathlineto{\pgfqpoint{1.540616in}{2.438665in}}%
\pgfpathlineto{\pgfqpoint{1.541461in}{2.070875in}}%
\pgfpathlineto{\pgfqpoint{1.542307in}{2.363734in}}%
\pgfpathlineto{\pgfqpoint{1.543153in}{2.326379in}}%
\pgfpathlineto{\pgfqpoint{1.543998in}{2.065590in}}%
\pgfpathlineto{\pgfqpoint{1.544844in}{2.189036in}}%
\pgfpathlineto{\pgfqpoint{1.545690in}{2.325659in}}%
\pgfpathlineto{\pgfqpoint{1.546535in}{2.068652in}}%
\pgfpathlineto{\pgfqpoint{1.547381in}{2.934152in}}%
\pgfpathlineto{\pgfqpoint{1.548226in}{2.101799in}}%
\pgfpathlineto{\pgfqpoint{1.549072in}{2.188272in}}%
\pgfpathlineto{\pgfqpoint{1.549918in}{2.650355in}}%
\pgfpathlineto{\pgfqpoint{1.550763in}{1.939990in}}%
\pgfpathlineto{\pgfqpoint{1.551609in}{2.189012in}}%
\pgfpathlineto{\pgfqpoint{1.552455in}{1.899245in}}%
\pgfpathlineto{\pgfqpoint{1.553300in}{2.543771in}}%
\pgfpathlineto{\pgfqpoint{1.554146in}{2.336928in}}%
\pgfpathlineto{\pgfqpoint{1.554991in}{2.392179in}}%
\pgfpathlineto{\pgfqpoint{1.555837in}{1.808632in}}%
\pgfpathlineto{\pgfqpoint{1.556683in}{2.212594in}}%
\pgfpathlineto{\pgfqpoint{1.557528in}{1.912555in}}%
\pgfpathlineto{\pgfqpoint{1.558374in}{2.278200in}}%
\pgfpathlineto{\pgfqpoint{1.559220in}{2.046324in}}%
\pgfpathlineto{\pgfqpoint{1.560065in}{2.054320in}}%
\pgfpathlineto{\pgfqpoint{1.560911in}{2.338794in}}%
\pgfpathlineto{\pgfqpoint{1.561756in}{1.698553in}}%
\pgfpathlineto{\pgfqpoint{1.562602in}{2.027527in}}%
\pgfpathlineto{\pgfqpoint{1.563448in}{1.985926in}}%
\pgfpathlineto{\pgfqpoint{1.564293in}{2.351152in}}%
\pgfpathlineto{\pgfqpoint{1.565139in}{2.062952in}}%
\pgfpathlineto{\pgfqpoint{1.566830in}{2.139323in}}%
\pgfpathlineto{\pgfqpoint{1.568521in}{2.405837in}}%
\pgfpathlineto{\pgfqpoint{1.569367in}{1.834237in}}%
\pgfpathlineto{\pgfqpoint{1.570213in}{2.128575in}}%
\pgfpathlineto{\pgfqpoint{1.571058in}{1.770325in}}%
\pgfpathlineto{\pgfqpoint{1.572750in}{2.381751in}}%
\pgfpathlineto{\pgfqpoint{1.574441in}{2.452668in}}%
\pgfpathlineto{\pgfqpoint{1.575286in}{1.722862in}}%
\pgfpathlineto{\pgfqpoint{1.576132in}{2.093206in}}%
\pgfpathlineto{\pgfqpoint{1.577823in}{1.969668in}}%
\pgfpathlineto{\pgfqpoint{1.578669in}{2.564007in}}%
\pgfpathlineto{\pgfqpoint{1.579515in}{2.330727in}}%
\pgfpathlineto{\pgfqpoint{1.580360in}{1.966921in}}%
\pgfpathlineto{\pgfqpoint{1.581206in}{1.978400in}}%
\pgfpathlineto{\pgfqpoint{1.583743in}{2.182997in}}%
\pgfpathlineto{\pgfqpoint{1.584588in}{1.821431in}}%
\pgfpathlineto{\pgfqpoint{1.585434in}{2.190783in}}%
\pgfpathlineto{\pgfqpoint{1.586280in}{1.678895in}}%
\pgfpathlineto{\pgfqpoint{1.587971in}{2.640737in}}%
\pgfpathlineto{\pgfqpoint{1.590508in}{1.813206in}}%
\pgfpathlineto{\pgfqpoint{1.593045in}{2.232215in}}%
\pgfpathlineto{\pgfqpoint{1.593890in}{2.215214in}}%
\pgfpathlineto{\pgfqpoint{1.594736in}{2.580483in}}%
\pgfpathlineto{\pgfqpoint{1.595581in}{2.119707in}}%
\pgfpathlineto{\pgfqpoint{1.596427in}{2.493975in}}%
\pgfpathlineto{\pgfqpoint{1.597273in}{2.930045in}}%
\pgfpathlineto{\pgfqpoint{1.598964in}{1.948731in}}%
\pgfpathlineto{\pgfqpoint{1.599810in}{2.161367in}}%
\pgfpathlineto{\pgfqpoint{1.600655in}{1.849383in}}%
\pgfpathlineto{\pgfqpoint{1.601501in}{2.297569in}}%
\pgfpathlineto{\pgfqpoint{1.602346in}{2.107523in}}%
\pgfpathlineto{\pgfqpoint{1.603192in}{1.712771in}}%
\pgfpathlineto{\pgfqpoint{1.604038in}{2.364760in}}%
\pgfpathlineto{\pgfqpoint{1.604883in}{2.115043in}}%
\pgfpathlineto{\pgfqpoint{1.605729in}{1.960148in}}%
\pgfpathlineto{\pgfqpoint{1.606574in}{2.157049in}}%
\pgfpathlineto{\pgfqpoint{1.607420in}{2.011060in}}%
\pgfpathlineto{\pgfqpoint{1.608266in}{2.084471in}}%
\pgfpathlineto{\pgfqpoint{1.609957in}{1.631789in}}%
\pgfpathlineto{\pgfqpoint{1.611648in}{2.805218in}}%
\pgfpathlineto{\pgfqpoint{1.613339in}{1.974318in}}%
\pgfpathlineto{\pgfqpoint{1.614185in}{2.142477in}}%
\pgfpathlineto{\pgfqpoint{1.615031in}{2.258322in}}%
\pgfpathlineto{\pgfqpoint{1.615876in}{2.210387in}}%
\pgfpathlineto{\pgfqpoint{1.616722in}{2.095607in}}%
\pgfpathlineto{\pgfqpoint{1.617568in}{2.373319in}}%
\pgfpathlineto{\pgfqpoint{1.619259in}{1.741593in}}%
\pgfpathlineto{\pgfqpoint{1.620950in}{2.382813in}}%
\pgfpathlineto{\pgfqpoint{1.621796in}{2.370514in}}%
\pgfpathlineto{\pgfqpoint{1.622641in}{2.073553in}}%
\pgfpathlineto{\pgfqpoint{1.623487in}{2.321410in}}%
\pgfpathlineto{\pgfqpoint{1.624333in}{1.965906in}}%
\pgfpathlineto{\pgfqpoint{1.625178in}{2.239518in}}%
\pgfpathlineto{\pgfqpoint{1.626024in}{2.570598in}}%
\pgfpathlineto{\pgfqpoint{1.626869in}{1.913060in}}%
\pgfpathlineto{\pgfqpoint{1.627715in}{2.126294in}}%
\pgfpathlineto{\pgfqpoint{1.628561in}{1.889096in}}%
\pgfpathlineto{\pgfqpoint{1.629406in}{1.959030in}}%
\pgfpathlineto{\pgfqpoint{1.630252in}{2.698697in}}%
\pgfpathlineto{\pgfqpoint{1.631098in}{2.281581in}}%
\pgfpathlineto{\pgfqpoint{1.631943in}{2.588123in}}%
\pgfpathlineto{\pgfqpoint{1.632789in}{2.335937in}}%
\pgfpathlineto{\pgfqpoint{1.633634in}{2.185138in}}%
\pgfpathlineto{\pgfqpoint{1.634480in}{2.509917in}}%
\pgfpathlineto{\pgfqpoint{1.635326in}{2.009140in}}%
\pgfpathlineto{\pgfqpoint{1.636171in}{2.275182in}}%
\pgfpathlineto{\pgfqpoint{1.637017in}{2.053731in}}%
\pgfpathlineto{\pgfqpoint{1.637863in}{2.406771in}}%
\pgfpathlineto{\pgfqpoint{1.640399in}{1.480160in}}%
\pgfpathlineto{\pgfqpoint{1.643782in}{2.860447in}}%
\pgfpathlineto{\pgfqpoint{1.646319in}{2.092055in}}%
\pgfpathlineto{\pgfqpoint{1.648856in}{1.764919in}}%
\pgfpathlineto{\pgfqpoint{1.649701in}{2.287020in}}%
\pgfpathlineto{\pgfqpoint{1.650547in}{1.908336in}}%
\pgfpathlineto{\pgfqpoint{1.651393in}{1.699386in}}%
\pgfpathlineto{\pgfqpoint{1.652238in}{2.563204in}}%
\pgfpathlineto{\pgfqpoint{1.653084in}{2.180778in}}%
\pgfpathlineto{\pgfqpoint{1.653929in}{1.688973in}}%
\pgfpathlineto{\pgfqpoint{1.654775in}{2.973125in}}%
\pgfpathlineto{\pgfqpoint{1.655621in}{2.536134in}}%
\pgfpathlineto{\pgfqpoint{1.656466in}{2.804581in}}%
\pgfpathlineto{\pgfqpoint{1.659003in}{2.088818in}}%
\pgfpathlineto{\pgfqpoint{1.659849in}{2.475760in}}%
\pgfpathlineto{\pgfqpoint{1.660694in}{1.842883in}}%
\pgfpathlineto{\pgfqpoint{1.661540in}{2.347284in}}%
\pgfpathlineto{\pgfqpoint{1.662386in}{2.652307in}}%
\pgfpathlineto{\pgfqpoint{1.663231in}{2.506110in}}%
\pgfpathlineto{\pgfqpoint{1.664077in}{2.536867in}}%
\pgfpathlineto{\pgfqpoint{1.664923in}{2.190579in}}%
\pgfpathlineto{\pgfqpoint{1.665768in}{2.891662in}}%
\pgfpathlineto{\pgfqpoint{1.666614in}{2.437342in}}%
\pgfpathlineto{\pgfqpoint{1.667459in}{2.507069in}}%
\pgfpathlineto{\pgfqpoint{1.669151in}{1.779787in}}%
\pgfpathlineto{\pgfqpoint{1.669996in}{2.363120in}}%
\pgfpathlineto{\pgfqpoint{1.670842in}{1.729478in}}%
\pgfpathlineto{\pgfqpoint{1.673379in}{2.682132in}}%
\pgfpathlineto{\pgfqpoint{1.675916in}{1.768742in}}%
\pgfpathlineto{\pgfqpoint{1.676761in}{2.497584in}}%
\pgfpathlineto{\pgfqpoint{1.677607in}{1.636977in}}%
\pgfpathlineto{\pgfqpoint{1.678453in}{2.276828in}}%
\pgfpathlineto{\pgfqpoint{1.679298in}{1.790051in}}%
\pgfpathlineto{\pgfqpoint{1.680144in}{1.940647in}}%
\pgfpathlineto{\pgfqpoint{1.680989in}{1.891426in}}%
\pgfpathlineto{\pgfqpoint{1.682681in}{2.592772in}}%
\pgfpathlineto{\pgfqpoint{1.683526in}{2.526257in}}%
\pgfpathlineto{\pgfqpoint{1.684372in}{1.744340in}}%
\pgfpathlineto{\pgfqpoint{1.685217in}{1.834974in}}%
\pgfpathlineto{\pgfqpoint{1.686063in}{2.482349in}}%
\pgfpathlineto{\pgfqpoint{1.686909in}{1.464770in}}%
\pgfpathlineto{\pgfqpoint{1.687754in}{2.331371in}}%
\pgfpathlineto{\pgfqpoint{1.688600in}{1.980461in}}%
\pgfpathlineto{\pgfqpoint{1.689446in}{2.278246in}}%
\pgfpathlineto{\pgfqpoint{1.690291in}{1.959584in}}%
\pgfpathlineto{\pgfqpoint{1.691137in}{2.321537in}}%
\pgfpathlineto{\pgfqpoint{1.691982in}{2.297496in}}%
\pgfpathlineto{\pgfqpoint{1.692828in}{1.916684in}}%
\pgfpathlineto{\pgfqpoint{1.694519in}{2.473471in}}%
\pgfpathlineto{\pgfqpoint{1.697056in}{1.715569in}}%
\pgfpathlineto{\pgfqpoint{1.697902in}{2.384922in}}%
\pgfpathlineto{\pgfqpoint{1.698747in}{1.995904in}}%
\pgfpathlineto{\pgfqpoint{1.699593in}{2.422140in}}%
\pgfpathlineto{\pgfqpoint{1.700439in}{1.625955in}}%
\pgfpathlineto{\pgfqpoint{1.701284in}{2.255110in}}%
\pgfpathlineto{\pgfqpoint{1.702130in}{2.638516in}}%
\pgfpathlineto{\pgfqpoint{1.703821in}{1.905099in}}%
\pgfpathlineto{\pgfqpoint{1.705512in}{2.470008in}}%
\pgfpathlineto{\pgfqpoint{1.706358in}{1.287594in}}%
\pgfpathlineto{\pgfqpoint{1.707204in}{2.740066in}}%
\pgfpathlineto{\pgfqpoint{1.708049in}{2.163457in}}%
\pgfpathlineto{\pgfqpoint{1.709741in}{1.919582in}}%
\pgfpathlineto{\pgfqpoint{1.712277in}{2.671757in}}%
\pgfpathlineto{\pgfqpoint{1.714814in}{1.880185in}}%
\pgfpathlineto{\pgfqpoint{1.716506in}{2.150187in}}%
\pgfpathlineto{\pgfqpoint{1.717351in}{1.949320in}}%
\pgfpathlineto{\pgfqpoint{1.718197in}{2.108042in}}%
\pgfpathlineto{\pgfqpoint{1.719042in}{2.506615in}}%
\pgfpathlineto{\pgfqpoint{1.721579in}{1.850277in}}%
\pgfpathlineto{\pgfqpoint{1.723271in}{2.400271in}}%
\pgfpathlineto{\pgfqpoint{1.724116in}{2.302659in}}%
\pgfpathlineto{\pgfqpoint{1.724962in}{2.102128in}}%
\pgfpathlineto{\pgfqpoint{1.725807in}{2.602305in}}%
\pgfpathlineto{\pgfqpoint{1.726653in}{2.142636in}}%
\pgfpathlineto{\pgfqpoint{1.727499in}{2.767175in}}%
\pgfpathlineto{\pgfqpoint{1.728344in}{1.901297in}}%
\pgfpathlineto{\pgfqpoint{1.729190in}{2.356779in}}%
\pgfpathlineto{\pgfqpoint{1.730036in}{2.510481in}}%
\pgfpathlineto{\pgfqpoint{1.730881in}{1.691057in}}%
\pgfpathlineto{\pgfqpoint{1.731727in}{2.037218in}}%
\pgfpathlineto{\pgfqpoint{1.732572in}{2.749700in}}%
\pgfpathlineto{\pgfqpoint{1.733418in}{2.267518in}}%
\pgfpathlineto{\pgfqpoint{1.734264in}{2.741993in}}%
\pgfpathlineto{\pgfqpoint{1.735955in}{1.658523in}}%
\pgfpathlineto{\pgfqpoint{1.736801in}{1.856299in}}%
\pgfpathlineto{\pgfqpoint{1.739337in}{2.414914in}}%
\pgfpathlineto{\pgfqpoint{1.741029in}{1.793214in}}%
\pgfpathlineto{\pgfqpoint{1.741874in}{2.166687in}}%
\pgfpathlineto{\pgfqpoint{1.742720in}{1.787351in}}%
\pgfpathlineto{\pgfqpoint{1.743566in}{2.285004in}}%
\pgfpathlineto{\pgfqpoint{1.744411in}{1.759938in}}%
\pgfpathlineto{\pgfqpoint{1.745257in}{2.205134in}}%
\pgfpathlineto{\pgfqpoint{1.746102in}{1.827552in}}%
\pgfpathlineto{\pgfqpoint{1.747794in}{2.451843in}}%
\pgfpathlineto{\pgfqpoint{1.748639in}{1.648544in}}%
\pgfpathlineto{\pgfqpoint{1.749485in}{2.652587in}}%
\pgfpathlineto{\pgfqpoint{1.750331in}{2.341939in}}%
\pgfpathlineto{\pgfqpoint{1.752867in}{1.823623in}}%
\pgfpathlineto{\pgfqpoint{1.753713in}{2.512137in}}%
\pgfpathlineto{\pgfqpoint{1.756250in}{1.740118in}}%
\pgfpathlineto{\pgfqpoint{1.757941in}{2.361225in}}%
\pgfpathlineto{\pgfqpoint{1.758787in}{2.173353in}}%
\pgfpathlineto{\pgfqpoint{1.759632in}{2.476012in}}%
\pgfpathlineto{\pgfqpoint{1.760478in}{1.604592in}}%
\pgfpathlineto{\pgfqpoint{1.761324in}{2.507174in}}%
\pgfpathlineto{\pgfqpoint{1.762169in}{2.124841in}}%
\pgfpathlineto{\pgfqpoint{1.763015in}{2.086396in}}%
\pgfpathlineto{\pgfqpoint{1.763860in}{1.742562in}}%
\pgfpathlineto{\pgfqpoint{1.764706in}{2.312160in}}%
\pgfpathlineto{\pgfqpoint{1.765552in}{2.184063in}}%
\pgfpathlineto{\pgfqpoint{1.766397in}{2.187141in}}%
\pgfpathlineto{\pgfqpoint{1.767243in}{2.305612in}}%
\pgfpathlineto{\pgfqpoint{1.768089in}{1.825980in}}%
\pgfpathlineto{\pgfqpoint{1.768934in}{2.189715in}}%
\pgfpathlineto{\pgfqpoint{1.770625in}{1.933839in}}%
\pgfpathlineto{\pgfqpoint{1.772317in}{2.724044in}}%
\pgfpathlineto{\pgfqpoint{1.773162in}{1.720532in}}%
\pgfpathlineto{\pgfqpoint{1.774008in}{2.307315in}}%
\pgfpathlineto{\pgfqpoint{1.774854in}{1.764100in}}%
\pgfpathlineto{\pgfqpoint{1.775699in}{2.680405in}}%
\pgfpathlineto{\pgfqpoint{1.776545in}{2.324821in}}%
\pgfpathlineto{\pgfqpoint{1.777390in}{2.096043in}}%
\pgfpathlineto{\pgfqpoint{1.779082in}{2.571908in}}%
\pgfpathlineto{\pgfqpoint{1.779927in}{1.811399in}}%
\pgfpathlineto{\pgfqpoint{1.780773in}{2.628505in}}%
\pgfpathlineto{\pgfqpoint{1.781619in}{1.587508in}}%
\pgfpathlineto{\pgfqpoint{1.782464in}{2.301096in}}%
\pgfpathlineto{\pgfqpoint{1.783310in}{2.942245in}}%
\pgfpathlineto{\pgfqpoint{1.784155in}{2.221472in}}%
\pgfpathlineto{\pgfqpoint{1.785001in}{2.248143in}}%
\pgfpathlineto{\pgfqpoint{1.785847in}{2.241100in}}%
\pgfpathlineto{\pgfqpoint{1.786692in}{1.609901in}}%
\pgfpathlineto{\pgfqpoint{1.787538in}{2.375273in}}%
\pgfpathlineto{\pgfqpoint{1.788384in}{2.089926in}}%
\pgfpathlineto{\pgfqpoint{1.789229in}{2.081040in}}%
\pgfpathlineto{\pgfqpoint{1.790075in}{1.755421in}}%
\pgfpathlineto{\pgfqpoint{1.791766in}{2.367198in}}%
\pgfpathlineto{\pgfqpoint{1.792612in}{1.921334in}}%
\pgfpathlineto{\pgfqpoint{1.793457in}{2.125183in}}%
\pgfpathlineto{\pgfqpoint{1.794303in}{3.046485in}}%
\pgfpathlineto{\pgfqpoint{1.795149in}{1.841909in}}%
\pgfpathlineto{\pgfqpoint{1.795994in}{2.444046in}}%
\pgfpathlineto{\pgfqpoint{1.797685in}{1.763186in}}%
\pgfpathlineto{\pgfqpoint{1.798531in}{1.868191in}}%
\pgfpathlineto{\pgfqpoint{1.800222in}{2.558068in}}%
\pgfpathlineto{\pgfqpoint{1.801068in}{2.099467in}}%
\pgfpathlineto{\pgfqpoint{1.801914in}{2.421920in}}%
\pgfpathlineto{\pgfqpoint{1.802759in}{2.641156in}}%
\pgfpathlineto{\pgfqpoint{1.805296in}{1.658834in}}%
\pgfpathlineto{\pgfqpoint{1.806142in}{2.627987in}}%
\pgfpathlineto{\pgfqpoint{1.806987in}{1.781857in}}%
\pgfpathlineto{\pgfqpoint{1.807833in}{1.815617in}}%
\pgfpathlineto{\pgfqpoint{1.809524in}{2.591213in}}%
\pgfpathlineto{\pgfqpoint{1.810370in}{1.991735in}}%
\pgfpathlineto{\pgfqpoint{1.811215in}{2.245494in}}%
\pgfpathlineto{\pgfqpoint{1.812061in}{2.022923in}}%
\pgfpathlineto{\pgfqpoint{1.815444in}{2.941016in}}%
\pgfpathlineto{\pgfqpoint{1.816289in}{2.088467in}}%
\pgfpathlineto{\pgfqpoint{1.817135in}{2.354279in}}%
\pgfpathlineto{\pgfqpoint{1.817980in}{2.581434in}}%
\pgfpathlineto{\pgfqpoint{1.818826in}{2.358310in}}%
\pgfpathlineto{\pgfqpoint{1.820517in}{2.637399in}}%
\pgfpathlineto{\pgfqpoint{1.823900in}{1.861248in}}%
\pgfpathlineto{\pgfqpoint{1.825591in}{2.501375in}}%
\pgfpathlineto{\pgfqpoint{1.826437in}{2.331195in}}%
\pgfpathlineto{\pgfqpoint{1.827282in}{2.584246in}}%
\pgfpathlineto{\pgfqpoint{1.828974in}{1.742840in}}%
\pgfpathlineto{\pgfqpoint{1.830665in}{2.423604in}}%
\pgfpathlineto{\pgfqpoint{1.831510in}{1.673324in}}%
\pgfpathlineto{\pgfqpoint{1.832356in}{1.856272in}}%
\pgfpathlineto{\pgfqpoint{1.834047in}{2.546725in}}%
\pgfpathlineto{\pgfqpoint{1.836584in}{2.056456in}}%
\pgfpathlineto{\pgfqpoint{1.837430in}{2.720001in}}%
\pgfpathlineto{\pgfqpoint{1.838275in}{2.096074in}}%
\pgfpathlineto{\pgfqpoint{1.839121in}{2.205169in}}%
\pgfpathlineto{\pgfqpoint{1.839967in}{2.351308in}}%
\pgfpathlineto{\pgfqpoint{1.840812in}{1.839505in}}%
\pgfpathlineto{\pgfqpoint{1.841658in}{1.942843in}}%
\pgfpathlineto{\pgfqpoint{1.844195in}{2.151715in}}%
\pgfpathlineto{\pgfqpoint{1.845040in}{2.016926in}}%
\pgfpathlineto{\pgfqpoint{1.848423in}{2.820537in}}%
\pgfpathlineto{\pgfqpoint{1.849268in}{3.020411in}}%
\pgfpathlineto{\pgfqpoint{1.850960in}{1.909443in}}%
\pgfpathlineto{\pgfqpoint{1.851805in}{2.748866in}}%
\pgfpathlineto{\pgfqpoint{1.852651in}{1.842477in}}%
\pgfpathlineto{\pgfqpoint{1.853497in}{2.436140in}}%
\pgfpathlineto{\pgfqpoint{1.855188in}{1.792676in}}%
\pgfpathlineto{\pgfqpoint{1.856033in}{2.073426in}}%
\pgfpathlineto{\pgfqpoint{1.857725in}{2.674146in}}%
\pgfpathlineto{\pgfqpoint{1.859416in}{1.692989in}}%
\pgfpathlineto{\pgfqpoint{1.862798in}{2.564430in}}%
\pgfpathlineto{\pgfqpoint{1.863644in}{1.305662in}}%
\pgfpathlineto{\pgfqpoint{1.864490in}{2.752934in}}%
\pgfpathlineto{\pgfqpoint{1.865335in}{1.997222in}}%
\pgfpathlineto{\pgfqpoint{1.867872in}{2.722618in}}%
\pgfpathlineto{\pgfqpoint{1.868718in}{2.131519in}}%
\pgfpathlineto{\pgfqpoint{1.869563in}{2.420107in}}%
\pgfpathlineto{\pgfqpoint{1.871255in}{1.907153in}}%
\pgfpathlineto{\pgfqpoint{1.872100in}{2.411452in}}%
\pgfpathlineto{\pgfqpoint{1.872946in}{1.657549in}}%
\pgfpathlineto{\pgfqpoint{1.874637in}{2.559007in}}%
\pgfpathlineto{\pgfqpoint{1.875483in}{2.025545in}}%
\pgfpathlineto{\pgfqpoint{1.876328in}{2.091082in}}%
\pgfpathlineto{\pgfqpoint{1.877174in}{2.125328in}}%
\pgfpathlineto{\pgfqpoint{1.878020in}{1.983157in}}%
\pgfpathlineto{\pgfqpoint{1.880557in}{2.729531in}}%
\pgfpathlineto{\pgfqpoint{1.883093in}{1.754903in}}%
\pgfpathlineto{\pgfqpoint{1.883939in}{1.838019in}}%
\pgfpathlineto{\pgfqpoint{1.884785in}{2.268775in}}%
\pgfpathlineto{\pgfqpoint{1.885630in}{1.968286in}}%
\pgfpathlineto{\pgfqpoint{1.888167in}{1.599290in}}%
\pgfpathlineto{\pgfqpoint{1.889858in}{2.252549in}}%
\pgfpathlineto{\pgfqpoint{1.890704in}{2.053678in}}%
\pgfpathlineto{\pgfqpoint{1.892395in}{2.672139in}}%
\pgfpathlineto{\pgfqpoint{1.895778in}{2.029038in}}%
\pgfpathlineto{\pgfqpoint{1.896623in}{2.396540in}}%
\pgfpathlineto{\pgfqpoint{1.897469in}{2.112086in}}%
\pgfpathlineto{\pgfqpoint{1.899160in}{2.388674in}}%
\pgfpathlineto{\pgfqpoint{1.900006in}{2.293652in}}%
\pgfpathlineto{\pgfqpoint{1.900852in}{2.543780in}}%
\pgfpathlineto{\pgfqpoint{1.903388in}{2.018299in}}%
\pgfpathlineto{\pgfqpoint{1.904234in}{1.927516in}}%
\pgfpathlineto{\pgfqpoint{1.905925in}{2.568224in}}%
\pgfpathlineto{\pgfqpoint{1.906771in}{2.538064in}}%
\pgfpathlineto{\pgfqpoint{1.907617in}{1.551578in}}%
\pgfpathlineto{\pgfqpoint{1.908462in}{2.092944in}}%
\pgfpathlineto{\pgfqpoint{1.910153in}{2.832312in}}%
\pgfpathlineto{\pgfqpoint{1.911845in}{1.883459in}}%
\pgfpathlineto{\pgfqpoint{1.912690in}{2.129102in}}%
\pgfpathlineto{\pgfqpoint{1.913536in}{2.018115in}}%
\pgfpathlineto{\pgfqpoint{1.914382in}{2.729621in}}%
\pgfpathlineto{\pgfqpoint{1.916073in}{1.765003in}}%
\pgfpathlineto{\pgfqpoint{1.917764in}{2.391329in}}%
\pgfpathlineto{\pgfqpoint{1.918610in}{2.388962in}}%
\pgfpathlineto{\pgfqpoint{1.920301in}{1.830339in}}%
\pgfpathlineto{\pgfqpoint{1.921147in}{2.424387in}}%
\pgfpathlineto{\pgfqpoint{1.922838in}{1.603123in}}%
\pgfpathlineto{\pgfqpoint{1.923683in}{1.606346in}}%
\pgfpathlineto{\pgfqpoint{1.924529in}{2.571167in}}%
\pgfpathlineto{\pgfqpoint{1.925375in}{2.268010in}}%
\pgfpathlineto{\pgfqpoint{1.926220in}{1.189513in}}%
\pgfpathlineto{\pgfqpoint{1.928757in}{2.636162in}}%
\pgfpathlineto{\pgfqpoint{1.929603in}{1.885282in}}%
\pgfpathlineto{\pgfqpoint{1.930448in}{1.972045in}}%
\pgfpathlineto{\pgfqpoint{1.931294in}{2.193395in}}%
\pgfpathlineto{\pgfqpoint{1.932140in}{1.836863in}}%
\pgfpathlineto{\pgfqpoint{1.932985in}{1.967310in}}%
\pgfpathlineto{\pgfqpoint{1.933831in}{2.115155in}}%
\pgfpathlineto{\pgfqpoint{1.934676in}{1.941734in}}%
\pgfpathlineto{\pgfqpoint{1.935522in}{2.322671in}}%
\pgfpathlineto{\pgfqpoint{1.936368in}{2.089087in}}%
\pgfpathlineto{\pgfqpoint{1.937213in}{2.365284in}}%
\pgfpathlineto{\pgfqpoint{1.939750in}{1.872946in}}%
\pgfpathlineto{\pgfqpoint{1.942287in}{2.508205in}}%
\pgfpathlineto{\pgfqpoint{1.943133in}{2.298612in}}%
\pgfpathlineto{\pgfqpoint{1.943978in}{2.150701in}}%
\pgfpathlineto{\pgfqpoint{1.944824in}{1.448096in}}%
\pgfpathlineto{\pgfqpoint{1.945670in}{2.005419in}}%
\pgfpathlineto{\pgfqpoint{1.946515in}{3.123674in}}%
\pgfpathlineto{\pgfqpoint{1.947361in}{2.182750in}}%
\pgfpathlineto{\pgfqpoint{1.948206in}{2.549152in}}%
\pgfpathlineto{\pgfqpoint{1.949052in}{3.133330in}}%
\pgfpathlineto{\pgfqpoint{1.949898in}{2.713229in}}%
\pgfpathlineto{\pgfqpoint{1.950743in}{1.648745in}}%
\pgfpathlineto{\pgfqpoint{1.951589in}{2.347526in}}%
\pgfpathlineto{\pgfqpoint{1.952435in}{2.387789in}}%
\pgfpathlineto{\pgfqpoint{1.953280in}{2.279525in}}%
\pgfpathlineto{\pgfqpoint{1.954971in}{2.762912in}}%
\pgfpathlineto{\pgfqpoint{1.956663in}{1.747375in}}%
\pgfpathlineto{\pgfqpoint{1.958354in}{2.499254in}}%
\pgfpathlineto{\pgfqpoint{1.959200in}{1.465103in}}%
\pgfpathlineto{\pgfqpoint{1.960045in}{2.150833in}}%
\pgfpathlineto{\pgfqpoint{1.960891in}{1.753991in}}%
\pgfpathlineto{\pgfqpoint{1.962582in}{2.531575in}}%
\pgfpathlineto{\pgfqpoint{1.964273in}{1.887974in}}%
\pgfpathlineto{\pgfqpoint{1.965119in}{2.450205in}}%
\pgfpathlineto{\pgfqpoint{1.965965in}{2.133580in}}%
\pgfpathlineto{\pgfqpoint{1.966810in}{1.828980in}}%
\pgfpathlineto{\pgfqpoint{1.967656in}{2.371047in}}%
\pgfpathlineto{\pgfqpoint{1.968501in}{2.137890in}}%
\pgfpathlineto{\pgfqpoint{1.969347in}{1.777585in}}%
\pgfpathlineto{\pgfqpoint{1.970193in}{2.510745in}}%
\pgfpathlineto{\pgfqpoint{1.971038in}{1.601730in}}%
\pgfpathlineto{\pgfqpoint{1.971884in}{1.840629in}}%
\pgfpathlineto{\pgfqpoint{1.973575in}{2.596814in}}%
\pgfpathlineto{\pgfqpoint{1.975266in}{1.734747in}}%
\pgfpathlineto{\pgfqpoint{1.976112in}{2.056315in}}%
\pgfpathlineto{\pgfqpoint{1.976958in}{1.784797in}}%
\pgfpathlineto{\pgfqpoint{1.978649in}{2.597147in}}%
\pgfpathlineto{\pgfqpoint{1.979495in}{2.472608in}}%
\pgfpathlineto{\pgfqpoint{1.980340in}{1.913631in}}%
\pgfpathlineto{\pgfqpoint{1.981186in}{2.239297in}}%
\pgfpathlineto{\pgfqpoint{1.982031in}{1.867470in}}%
\pgfpathlineto{\pgfqpoint{1.982877in}{1.904102in}}%
\pgfpathlineto{\pgfqpoint{1.983723in}{2.542215in}}%
\pgfpathlineto{\pgfqpoint{1.984568in}{2.121181in}}%
\pgfpathlineto{\pgfqpoint{1.986260in}{2.845141in}}%
\pgfpathlineto{\pgfqpoint{1.988796in}{1.692232in}}%
\pgfpathlineto{\pgfqpoint{1.989642in}{2.043471in}}%
\pgfpathlineto{\pgfqpoint{1.990488in}{1.494207in}}%
\pgfpathlineto{\pgfqpoint{1.991333in}{1.883704in}}%
\pgfpathlineto{\pgfqpoint{1.992179in}{1.922295in}}%
\pgfpathlineto{\pgfqpoint{1.993025in}{2.820278in}}%
\pgfpathlineto{\pgfqpoint{1.995561in}{1.737018in}}%
\pgfpathlineto{\pgfqpoint{1.998098in}{1.357711in}}%
\pgfpathlineto{\pgfqpoint{1.998944in}{2.368662in}}%
\pgfpathlineto{\pgfqpoint{1.999790in}{2.304991in}}%
\pgfpathlineto{\pgfqpoint{2.000635in}{1.460494in}}%
\pgfpathlineto{\pgfqpoint{2.001481in}{2.007931in}}%
\pgfpathlineto{\pgfqpoint{2.004018in}{2.496862in}}%
\pgfpathlineto{\pgfqpoint{2.004863in}{2.323561in}}%
\pgfpathlineto{\pgfqpoint{2.005709in}{2.752890in}}%
\pgfpathlineto{\pgfqpoint{2.008246in}{1.674816in}}%
\pgfpathlineto{\pgfqpoint{2.009091in}{1.742847in}}%
\pgfpathlineto{\pgfqpoint{2.009937in}{2.553846in}}%
\pgfpathlineto{\pgfqpoint{2.010783in}{2.269036in}}%
\pgfpathlineto{\pgfqpoint{2.011628in}{2.253614in}}%
\pgfpathlineto{\pgfqpoint{2.012474in}{1.416425in}}%
\pgfpathlineto{\pgfqpoint{2.013319in}{1.954455in}}%
\pgfpathlineto{\pgfqpoint{2.014165in}{2.006089in}}%
\pgfpathlineto{\pgfqpoint{2.015011in}{1.463570in}}%
\pgfpathlineto{\pgfqpoint{2.016702in}{2.858612in}}%
\pgfpathlineto{\pgfqpoint{2.019239in}{1.155103in}}%
\pgfpathlineto{\pgfqpoint{2.020084in}{2.185292in}}%
\pgfpathlineto{\pgfqpoint{2.020930in}{2.019642in}}%
\pgfpathlineto{\pgfqpoint{2.021776in}{1.880201in}}%
\pgfpathlineto{\pgfqpoint{2.022621in}{2.777440in}}%
\pgfpathlineto{\pgfqpoint{2.023467in}{1.713116in}}%
\pgfpathlineto{\pgfqpoint{2.024313in}{2.027328in}}%
\pgfpathlineto{\pgfqpoint{2.026849in}{2.584521in}}%
\pgfpathlineto{\pgfqpoint{2.030232in}{1.210933in}}%
\pgfpathlineto{\pgfqpoint{2.031078in}{2.436713in}}%
\pgfpathlineto{\pgfqpoint{2.031923in}{2.137913in}}%
\pgfpathlineto{\pgfqpoint{2.032769in}{2.515005in}}%
\pgfpathlineto{\pgfqpoint{2.033614in}{2.028470in}}%
\pgfpathlineto{\pgfqpoint{2.034460in}{2.879962in}}%
\pgfpathlineto{\pgfqpoint{2.035306in}{1.642860in}}%
\pgfpathlineto{\pgfqpoint{2.036151in}{1.776572in}}%
\pgfpathlineto{\pgfqpoint{2.036997in}{1.597392in}}%
\pgfpathlineto{\pgfqpoint{2.038688in}{2.672068in}}%
\pgfpathlineto{\pgfqpoint{2.039534in}{1.644394in}}%
\pgfpathlineto{\pgfqpoint{2.040379in}{2.822333in}}%
\pgfpathlineto{\pgfqpoint{2.041225in}{2.092905in}}%
\pgfpathlineto{\pgfqpoint{2.042071in}{2.408408in}}%
\pgfpathlineto{\pgfqpoint{2.043762in}{1.685426in}}%
\pgfpathlineto{\pgfqpoint{2.045453in}{2.545867in}}%
\pgfpathlineto{\pgfqpoint{2.047990in}{1.906517in}}%
\pgfpathlineto{\pgfqpoint{2.048836in}{2.058707in}}%
\pgfpathlineto{\pgfqpoint{2.049681in}{2.839563in}}%
\pgfpathlineto{\pgfqpoint{2.050527in}{2.116429in}}%
\pgfpathlineto{\pgfqpoint{2.051373in}{2.244816in}}%
\pgfpathlineto{\pgfqpoint{2.052218in}{2.125857in}}%
\pgfpathlineto{\pgfqpoint{2.053064in}{2.299627in}}%
\pgfpathlineto{\pgfqpoint{2.054755in}{1.823981in}}%
\pgfpathlineto{\pgfqpoint{2.055601in}{2.417210in}}%
\pgfpathlineto{\pgfqpoint{2.056446in}{1.650274in}}%
\pgfpathlineto{\pgfqpoint{2.057292in}{2.128570in}}%
\pgfpathlineto{\pgfqpoint{2.058138in}{1.804632in}}%
\pgfpathlineto{\pgfqpoint{2.058983in}{2.005126in}}%
\pgfpathlineto{\pgfqpoint{2.059829in}{1.941749in}}%
\pgfpathlineto{\pgfqpoint{2.060674in}{2.153011in}}%
\pgfpathlineto{\pgfqpoint{2.061520in}{1.906242in}}%
\pgfpathlineto{\pgfqpoint{2.062366in}{2.069293in}}%
\pgfpathlineto{\pgfqpoint{2.063211in}{2.042437in}}%
\pgfpathlineto{\pgfqpoint{2.064057in}{1.936708in}}%
\pgfpathlineto{\pgfqpoint{2.064903in}{2.160097in}}%
\pgfpathlineto{\pgfqpoint{2.066594in}{1.544321in}}%
\pgfpathlineto{\pgfqpoint{2.067439in}{2.553251in}}%
\pgfpathlineto{\pgfqpoint{2.068285in}{2.264595in}}%
\pgfpathlineto{\pgfqpoint{2.069131in}{2.115628in}}%
\pgfpathlineto{\pgfqpoint{2.069976in}{1.328891in}}%
\pgfpathlineto{\pgfqpoint{2.070822in}{2.737243in}}%
\pgfpathlineto{\pgfqpoint{2.071668in}{1.940999in}}%
\pgfpathlineto{\pgfqpoint{2.074204in}{2.658348in}}%
\pgfpathlineto{\pgfqpoint{2.075050in}{2.226188in}}%
\pgfpathlineto{\pgfqpoint{2.075896in}{2.852315in}}%
\pgfpathlineto{\pgfqpoint{2.076741in}{1.621961in}}%
\pgfpathlineto{\pgfqpoint{2.077587in}{1.939543in}}%
\pgfpathlineto{\pgfqpoint{2.079278in}{2.090209in}}%
\pgfpathlineto{\pgfqpoint{2.080124in}{2.050697in}}%
\pgfpathlineto{\pgfqpoint{2.080969in}{1.559297in}}%
\pgfpathlineto{\pgfqpoint{2.081815in}{2.441506in}}%
\pgfpathlineto{\pgfqpoint{2.082661in}{2.426255in}}%
\pgfpathlineto{\pgfqpoint{2.083506in}{1.841694in}}%
\pgfpathlineto{\pgfqpoint{2.084352in}{2.556613in}}%
\pgfpathlineto{\pgfqpoint{2.085197in}{2.161538in}}%
\pgfpathlineto{\pgfqpoint{2.086889in}{1.827706in}}%
\pgfpathlineto{\pgfqpoint{2.088580in}{2.719470in}}%
\pgfpathlineto{\pgfqpoint{2.089426in}{1.998193in}}%
\pgfpathlineto{\pgfqpoint{2.090271in}{2.262505in}}%
\pgfpathlineto{\pgfqpoint{2.091117in}{2.324450in}}%
\pgfpathlineto{\pgfqpoint{2.091962in}{1.821879in}}%
\pgfpathlineto{\pgfqpoint{2.092808in}{2.091472in}}%
\pgfpathlineto{\pgfqpoint{2.093654in}{2.600181in}}%
\pgfpathlineto{\pgfqpoint{2.096191in}{2.013557in}}%
\pgfpathlineto{\pgfqpoint{2.097882in}{1.684349in}}%
\pgfpathlineto{\pgfqpoint{2.098727in}{2.576814in}}%
\pgfpathlineto{\pgfqpoint{2.099573in}{1.708891in}}%
\pgfpathlineto{\pgfqpoint{2.100419in}{2.198136in}}%
\pgfpathlineto{\pgfqpoint{2.101264in}{2.483745in}}%
\pgfpathlineto{\pgfqpoint{2.102110in}{1.788491in}}%
\pgfpathlineto{\pgfqpoint{2.103801in}{2.621736in}}%
\pgfpathlineto{\pgfqpoint{2.104647in}{2.083077in}}%
\pgfpathlineto{\pgfqpoint{2.105492in}{2.303209in}}%
\pgfpathlineto{\pgfqpoint{2.106338in}{2.519284in}}%
\pgfpathlineto{\pgfqpoint{2.108029in}{2.082216in}}%
\pgfpathlineto{\pgfqpoint{2.108875in}{2.122989in}}%
\pgfpathlineto{\pgfqpoint{2.109721in}{2.127966in}}%
\pgfpathlineto{\pgfqpoint{2.111412in}{2.503700in}}%
\pgfpathlineto{\pgfqpoint{2.112257in}{1.466873in}}%
\pgfpathlineto{\pgfqpoint{2.113103in}{2.686213in}}%
\pgfpathlineto{\pgfqpoint{2.113949in}{2.177329in}}%
\pgfpathlineto{\pgfqpoint{2.115640in}{1.886779in}}%
\pgfpathlineto{\pgfqpoint{2.116486in}{2.076051in}}%
\pgfpathlineto{\pgfqpoint{2.117331in}{1.415462in}}%
\pgfpathlineto{\pgfqpoint{2.118177in}{1.616003in}}%
\pgfpathlineto{\pgfqpoint{2.119022in}{1.776349in}}%
\pgfpathlineto{\pgfqpoint{2.119868in}{2.452730in}}%
\pgfpathlineto{\pgfqpoint{2.120714in}{2.110331in}}%
\pgfpathlineto{\pgfqpoint{2.121559in}{2.530022in}}%
\pgfpathlineto{\pgfqpoint{2.122405in}{1.446217in}}%
\pgfpathlineto{\pgfqpoint{2.123251in}{1.891465in}}%
\pgfpathlineto{\pgfqpoint{2.125787in}{2.506584in}}%
\pgfpathlineto{\pgfqpoint{2.126633in}{1.786236in}}%
\pgfpathlineto{\pgfqpoint{2.127479in}{2.360352in}}%
\pgfpathlineto{\pgfqpoint{2.128324in}{2.578897in}}%
\pgfpathlineto{\pgfqpoint{2.130016in}{1.739890in}}%
\pgfpathlineto{\pgfqpoint{2.130861in}{2.356058in}}%
\pgfpathlineto{\pgfqpoint{2.131707in}{2.071277in}}%
\pgfpathlineto{\pgfqpoint{2.133398in}{2.226175in}}%
\pgfpathlineto{\pgfqpoint{2.134244in}{1.859939in}}%
\pgfpathlineto{\pgfqpoint{2.137626in}{3.270512in}}%
\pgfpathlineto{\pgfqpoint{2.139317in}{1.787329in}}%
\pgfpathlineto{\pgfqpoint{2.140163in}{1.851135in}}%
\pgfpathlineto{\pgfqpoint{2.141009in}{2.541953in}}%
\pgfpathlineto{\pgfqpoint{2.141854in}{2.280731in}}%
\pgfpathlineto{\pgfqpoint{2.142700in}{2.352966in}}%
\pgfpathlineto{\pgfqpoint{2.143546in}{2.043823in}}%
\pgfpathlineto{\pgfqpoint{2.144391in}{2.263743in}}%
\pgfpathlineto{\pgfqpoint{2.145237in}{2.307823in}}%
\pgfpathlineto{\pgfqpoint{2.146082in}{2.262209in}}%
\pgfpathlineto{\pgfqpoint{2.146928in}{1.946138in}}%
\pgfpathlineto{\pgfqpoint{2.147774in}{2.627599in}}%
\pgfpathlineto{\pgfqpoint{2.148619in}{1.960559in}}%
\pgfpathlineto{\pgfqpoint{2.149465in}{2.445176in}}%
\pgfpathlineto{\pgfqpoint{2.150311in}{1.914487in}}%
\pgfpathlineto{\pgfqpoint{2.151156in}{2.081068in}}%
\pgfpathlineto{\pgfqpoint{2.152002in}{2.370615in}}%
\pgfpathlineto{\pgfqpoint{2.152847in}{1.824191in}}%
\pgfpathlineto{\pgfqpoint{2.153693in}{2.049599in}}%
\pgfpathlineto{\pgfqpoint{2.154539in}{2.028838in}}%
\pgfpathlineto{\pgfqpoint{2.157076in}{2.871582in}}%
\pgfpathlineto{\pgfqpoint{2.159612in}{1.733549in}}%
\pgfpathlineto{\pgfqpoint{2.162149in}{2.534968in}}%
\pgfpathlineto{\pgfqpoint{2.162995in}{2.166031in}}%
\pgfpathlineto{\pgfqpoint{2.163840in}{2.216870in}}%
\pgfpathlineto{\pgfqpoint{2.165532in}{2.622330in}}%
\pgfpathlineto{\pgfqpoint{2.167223in}{1.989746in}}%
\pgfpathlineto{\pgfqpoint{2.168069in}{2.436091in}}%
\pgfpathlineto{\pgfqpoint{2.168914in}{1.329128in}}%
\pgfpathlineto{\pgfqpoint{2.169760in}{2.439442in}}%
\pgfpathlineto{\pgfqpoint{2.170605in}{2.152273in}}%
\pgfpathlineto{\pgfqpoint{2.172297in}{2.505102in}}%
\pgfpathlineto{\pgfqpoint{2.173142in}{2.066108in}}%
\pgfpathlineto{\pgfqpoint{2.173988in}{2.523841in}}%
\pgfpathlineto{\pgfqpoint{2.174834in}{1.690567in}}%
\pgfpathlineto{\pgfqpoint{2.175679in}{2.239359in}}%
\pgfpathlineto{\pgfqpoint{2.176525in}{2.679762in}}%
\pgfpathlineto{\pgfqpoint{2.177370in}{2.058442in}}%
\pgfpathlineto{\pgfqpoint{2.178216in}{2.273994in}}%
\pgfpathlineto{\pgfqpoint{2.179062in}{1.966385in}}%
\pgfpathlineto{\pgfqpoint{2.180753in}{2.509057in}}%
\pgfpathlineto{\pgfqpoint{2.181599in}{2.440484in}}%
\pgfpathlineto{\pgfqpoint{2.183290in}{1.998205in}}%
\pgfpathlineto{\pgfqpoint{2.184135in}{2.379133in}}%
\pgfpathlineto{\pgfqpoint{2.184981in}{2.048727in}}%
\pgfpathlineto{\pgfqpoint{2.185827in}{2.280236in}}%
\pgfpathlineto{\pgfqpoint{2.187518in}{2.774620in}}%
\pgfpathlineto{\pgfqpoint{2.188364in}{2.263364in}}%
\pgfpathlineto{\pgfqpoint{2.189209in}{2.445578in}}%
\pgfpathlineto{\pgfqpoint{2.190055in}{2.341293in}}%
\pgfpathlineto{\pgfqpoint{2.190900in}{2.661883in}}%
\pgfpathlineto{\pgfqpoint{2.191746in}{2.362865in}}%
\pgfpathlineto{\pgfqpoint{2.192592in}{2.390724in}}%
\pgfpathlineto{\pgfqpoint{2.193437in}{2.393450in}}%
\pgfpathlineto{\pgfqpoint{2.194283in}{2.205256in}}%
\pgfpathlineto{\pgfqpoint{2.195129in}{2.616372in}}%
\pgfpathlineto{\pgfqpoint{2.195974in}{2.150186in}}%
\pgfpathlineto{\pgfqpoint{2.196820in}{2.605240in}}%
\pgfpathlineto{\pgfqpoint{2.197665in}{2.584959in}}%
\pgfpathlineto{\pgfqpoint{2.198511in}{1.899521in}}%
\pgfpathlineto{\pgfqpoint{2.199357in}{2.004169in}}%
\pgfpathlineto{\pgfqpoint{2.200202in}{2.130083in}}%
\pgfpathlineto{\pgfqpoint{2.201048in}{1.581321in}}%
\pgfpathlineto{\pgfqpoint{2.201894in}{1.881070in}}%
\pgfpathlineto{\pgfqpoint{2.202739in}{1.896205in}}%
\pgfpathlineto{\pgfqpoint{2.203585in}{2.410933in}}%
\pgfpathlineto{\pgfqpoint{2.204430in}{1.539249in}}%
\pgfpathlineto{\pgfqpoint{2.205276in}{2.587751in}}%
\pgfpathlineto{\pgfqpoint{2.206122in}{2.180327in}}%
\pgfpathlineto{\pgfqpoint{2.206967in}{2.367205in}}%
\pgfpathlineto{\pgfqpoint{2.210350in}{1.360722in}}%
\pgfpathlineto{\pgfqpoint{2.212041in}{1.985938in}}%
\pgfpathlineto{\pgfqpoint{2.212887in}{2.606589in}}%
\pgfpathlineto{\pgfqpoint{2.213732in}{1.856233in}}%
\pgfpathlineto{\pgfqpoint{2.215424in}{2.793250in}}%
\pgfpathlineto{\pgfqpoint{2.217115in}{1.806533in}}%
\pgfpathlineto{\pgfqpoint{2.217960in}{2.061832in}}%
\pgfpathlineto{\pgfqpoint{2.218806in}{1.759872in}}%
\pgfpathlineto{\pgfqpoint{2.219652in}{1.888377in}}%
\pgfpathlineto{\pgfqpoint{2.220497in}{2.374766in}}%
\pgfpathlineto{\pgfqpoint{2.221343in}{1.795763in}}%
\pgfpathlineto{\pgfqpoint{2.222189in}{2.699923in}}%
\pgfpathlineto{\pgfqpoint{2.223034in}{2.195716in}}%
\pgfpathlineto{\pgfqpoint{2.223880in}{2.264369in}}%
\pgfpathlineto{\pgfqpoint{2.224725in}{2.122239in}}%
\pgfpathlineto{\pgfqpoint{2.225571in}{2.436589in}}%
\pgfpathlineto{\pgfqpoint{2.226417in}{1.857050in}}%
\pgfpathlineto{\pgfqpoint{2.228954in}{2.783393in}}%
\pgfpathlineto{\pgfqpoint{2.230645in}{2.049291in}}%
\pgfpathlineto{\pgfqpoint{2.231490in}{2.610181in}}%
\pgfpathlineto{\pgfqpoint{2.232336in}{2.582349in}}%
\pgfpathlineto{\pgfqpoint{2.233182in}{1.746658in}}%
\pgfpathlineto{\pgfqpoint{2.234027in}{2.395671in}}%
\pgfpathlineto{\pgfqpoint{2.235719in}{1.918890in}}%
\pgfpathlineto{\pgfqpoint{2.237410in}{2.049331in}}%
\pgfpathlineto{\pgfqpoint{2.238255in}{0.906709in}}%
\pgfpathlineto{\pgfqpoint{2.239101in}{2.573283in}}%
\pgfpathlineto{\pgfqpoint{2.239947in}{2.219401in}}%
\pgfpathlineto{\pgfqpoint{2.240792in}{1.420643in}}%
\pgfpathlineto{\pgfqpoint{2.241638in}{2.386716in}}%
\pgfpathlineto{\pgfqpoint{2.242483in}{1.997911in}}%
\pgfpathlineto{\pgfqpoint{2.243329in}{2.158610in}}%
\pgfpathlineto{\pgfqpoint{2.244175in}{1.783848in}}%
\pgfpathlineto{\pgfqpoint{2.245020in}{2.427718in}}%
\pgfpathlineto{\pgfqpoint{2.245866in}{2.171970in}}%
\pgfpathlineto{\pgfqpoint{2.246712in}{1.743315in}}%
\pgfpathlineto{\pgfqpoint{2.247557in}{1.929643in}}%
\pgfpathlineto{\pgfqpoint{2.248403in}{2.406211in}}%
\pgfpathlineto{\pgfqpoint{2.249248in}{1.372703in}}%
\pgfpathlineto{\pgfqpoint{2.250094in}{2.736638in}}%
\pgfpathlineto{\pgfqpoint{2.250940in}{2.288045in}}%
\pgfpathlineto{\pgfqpoint{2.251785in}{2.876005in}}%
\pgfpathlineto{\pgfqpoint{2.252631in}{2.241146in}}%
\pgfpathlineto{\pgfqpoint{2.253477in}{2.403363in}}%
\pgfpathlineto{\pgfqpoint{2.254322in}{2.959918in}}%
\pgfpathlineto{\pgfqpoint{2.255168in}{1.861547in}}%
\pgfpathlineto{\pgfqpoint{2.256013in}{2.560940in}}%
\pgfpathlineto{\pgfqpoint{2.256859in}{2.158282in}}%
\pgfpathlineto{\pgfqpoint{2.257705in}{2.217645in}}%
\pgfpathlineto{\pgfqpoint{2.258550in}{2.268305in}}%
\pgfpathlineto{\pgfqpoint{2.259396in}{2.913504in}}%
\pgfpathlineto{\pgfqpoint{2.260242in}{2.389433in}}%
\pgfpathlineto{\pgfqpoint{2.261087in}{2.243678in}}%
\pgfpathlineto{\pgfqpoint{2.261933in}{2.258312in}}%
\pgfpathlineto{\pgfqpoint{2.262778in}{2.557171in}}%
\pgfpathlineto{\pgfqpoint{2.264470in}{1.663017in}}%
\pgfpathlineto{\pgfqpoint{2.267007in}{2.753077in}}%
\pgfpathlineto{\pgfqpoint{2.269543in}{2.000521in}}%
\pgfpathlineto{\pgfqpoint{2.270389in}{1.832367in}}%
\pgfpathlineto{\pgfqpoint{2.271235in}{2.347774in}}%
\pgfpathlineto{\pgfqpoint{2.272080in}{2.077166in}}%
\pgfpathlineto{\pgfqpoint{2.272926in}{2.430371in}}%
\pgfpathlineto{\pgfqpoint{2.273772in}{1.899127in}}%
\pgfpathlineto{\pgfqpoint{2.275463in}{2.630974in}}%
\pgfpathlineto{\pgfqpoint{2.278845in}{1.425766in}}%
\pgfpathlineto{\pgfqpoint{2.281382in}{2.634545in}}%
\pgfpathlineto{\pgfqpoint{2.283073in}{2.666860in}}%
\pgfpathlineto{\pgfqpoint{2.283919in}{1.813240in}}%
\pgfpathlineto{\pgfqpoint{2.284765in}{2.144618in}}%
\pgfpathlineto{\pgfqpoint{2.285610in}{2.660225in}}%
\pgfpathlineto{\pgfqpoint{2.286456in}{2.197947in}}%
\pgfpathlineto{\pgfqpoint{2.287302in}{2.575596in}}%
\pgfpathlineto{\pgfqpoint{2.288147in}{2.281591in}}%
\pgfpathlineto{\pgfqpoint{2.288993in}{1.307474in}}%
\pgfpathlineto{\pgfqpoint{2.289838in}{2.170715in}}%
\pgfpathlineto{\pgfqpoint{2.290684in}{1.892406in}}%
\pgfpathlineto{\pgfqpoint{2.292375in}{2.768143in}}%
\pgfpathlineto{\pgfqpoint{2.294067in}{1.926230in}}%
\pgfpathlineto{\pgfqpoint{2.294912in}{2.304318in}}%
\pgfpathlineto{\pgfqpoint{2.295758in}{2.270980in}}%
\pgfpathlineto{\pgfqpoint{2.296603in}{1.571316in}}%
\pgfpathlineto{\pgfqpoint{2.297449in}{2.521852in}}%
\pgfpathlineto{\pgfqpoint{2.298295in}{2.041150in}}%
\pgfpathlineto{\pgfqpoint{2.299140in}{1.844390in}}%
\pgfpathlineto{\pgfqpoint{2.299986in}{1.914898in}}%
\pgfpathlineto{\pgfqpoint{2.300832in}{2.050902in}}%
\pgfpathlineto{\pgfqpoint{2.301677in}{2.800453in}}%
\pgfpathlineto{\pgfqpoint{2.302523in}{2.560445in}}%
\pgfpathlineto{\pgfqpoint{2.303368in}{1.789649in}}%
\pgfpathlineto{\pgfqpoint{2.304214in}{1.959590in}}%
\pgfpathlineto{\pgfqpoint{2.305060in}{2.330307in}}%
\pgfpathlineto{\pgfqpoint{2.306751in}{1.617465in}}%
\pgfpathlineto{\pgfqpoint{2.307597in}{2.829706in}}%
\pgfpathlineto{\pgfqpoint{2.308442in}{2.621261in}}%
\pgfpathlineto{\pgfqpoint{2.309288in}{2.700126in}}%
\pgfpathlineto{\pgfqpoint{2.310979in}{2.115546in}}%
\pgfpathlineto{\pgfqpoint{2.312670in}{2.426934in}}%
\pgfpathlineto{\pgfqpoint{2.313516in}{1.893645in}}%
\pgfpathlineto{\pgfqpoint{2.314362in}{2.089081in}}%
\pgfpathlineto{\pgfqpoint{2.315207in}{2.706435in}}%
\pgfpathlineto{\pgfqpoint{2.316053in}{2.100579in}}%
\pgfpathlineto{\pgfqpoint{2.316898in}{2.291910in}}%
\pgfpathlineto{\pgfqpoint{2.317744in}{2.445751in}}%
\pgfpathlineto{\pgfqpoint{2.319435in}{1.858084in}}%
\pgfpathlineto{\pgfqpoint{2.320281in}{2.083245in}}%
\pgfpathlineto{\pgfqpoint{2.321972in}{1.699696in}}%
\pgfpathlineto{\pgfqpoint{2.322818in}{2.061689in}}%
\pgfpathlineto{\pgfqpoint{2.323663in}{1.801331in}}%
\pgfpathlineto{\pgfqpoint{2.325355in}{2.359103in}}%
\pgfpathlineto{\pgfqpoint{2.326200in}{2.282491in}}%
\pgfpathlineto{\pgfqpoint{2.327046in}{2.138383in}}%
\pgfpathlineto{\pgfqpoint{2.327891in}{1.382586in}}%
\pgfpathlineto{\pgfqpoint{2.330428in}{2.458357in}}%
\pgfpathlineto{\pgfqpoint{2.331274in}{2.372059in}}%
\pgfpathlineto{\pgfqpoint{2.332120in}{2.400975in}}%
\pgfpathlineto{\pgfqpoint{2.332965in}{1.756303in}}%
\pgfpathlineto{\pgfqpoint{2.333811in}{2.803357in}}%
\pgfpathlineto{\pgfqpoint{2.334656in}{1.460746in}}%
\pgfpathlineto{\pgfqpoint{2.335502in}{2.684468in}}%
\pgfpathlineto{\pgfqpoint{2.336348in}{2.298607in}}%
\pgfpathlineto{\pgfqpoint{2.337193in}{2.779073in}}%
\pgfpathlineto{\pgfqpoint{2.338885in}{1.589122in}}%
\pgfpathlineto{\pgfqpoint{2.340576in}{2.239504in}}%
\pgfpathlineto{\pgfqpoint{2.341421in}{2.925242in}}%
\pgfpathlineto{\pgfqpoint{2.342267in}{2.481695in}}%
\pgfpathlineto{\pgfqpoint{2.343113in}{1.985809in}}%
\pgfpathlineto{\pgfqpoint{2.343958in}{2.175751in}}%
\pgfpathlineto{\pgfqpoint{2.344804in}{2.742804in}}%
\pgfpathlineto{\pgfqpoint{2.345650in}{2.413326in}}%
\pgfpathlineto{\pgfqpoint{2.346495in}{2.382876in}}%
\pgfpathlineto{\pgfqpoint{2.347341in}{2.404476in}}%
\pgfpathlineto{\pgfqpoint{2.348186in}{2.802724in}}%
\pgfpathlineto{\pgfqpoint{2.349032in}{1.868828in}}%
\pgfpathlineto{\pgfqpoint{2.349878in}{2.341311in}}%
\pgfpathlineto{\pgfqpoint{2.350723in}{2.352088in}}%
\pgfpathlineto{\pgfqpoint{2.351569in}{2.291668in}}%
\pgfpathlineto{\pgfqpoint{2.352415in}{2.363885in}}%
\pgfpathlineto{\pgfqpoint{2.353260in}{2.186297in}}%
\pgfpathlineto{\pgfqpoint{2.354106in}{2.488965in}}%
\pgfpathlineto{\pgfqpoint{2.356643in}{1.377389in}}%
\pgfpathlineto{\pgfqpoint{2.357488in}{2.438464in}}%
\pgfpathlineto{\pgfqpoint{2.358334in}{2.386908in}}%
\pgfpathlineto{\pgfqpoint{2.359180in}{2.215603in}}%
\pgfpathlineto{\pgfqpoint{2.360025in}{2.635500in}}%
\pgfpathlineto{\pgfqpoint{2.362562in}{1.521895in}}%
\pgfpathlineto{\pgfqpoint{2.365099in}{2.520590in}}%
\pgfpathlineto{\pgfqpoint{2.365945in}{2.914759in}}%
\pgfpathlineto{\pgfqpoint{2.366790in}{1.711916in}}%
\pgfpathlineto{\pgfqpoint{2.367636in}{1.865562in}}%
\pgfpathlineto{\pgfqpoint{2.368481in}{1.995441in}}%
\pgfpathlineto{\pgfqpoint{2.370173in}{2.843875in}}%
\pgfpathlineto{\pgfqpoint{2.371864in}{1.904903in}}%
\pgfpathlineto{\pgfqpoint{2.372710in}{2.339339in}}%
\pgfpathlineto{\pgfqpoint{2.373555in}{1.840868in}}%
\pgfpathlineto{\pgfqpoint{2.374401in}{2.342208in}}%
\pgfpathlineto{\pgfqpoint{2.375246in}{2.030569in}}%
\pgfpathlineto{\pgfqpoint{2.376092in}{2.086392in}}%
\pgfpathlineto{\pgfqpoint{2.376938in}{2.481090in}}%
\pgfpathlineto{\pgfqpoint{2.377783in}{2.318500in}}%
\pgfpathlineto{\pgfqpoint{2.378629in}{2.472960in}}%
\pgfpathlineto{\pgfqpoint{2.380320in}{1.715129in}}%
\pgfpathlineto{\pgfqpoint{2.382011in}{2.577304in}}%
\pgfpathlineto{\pgfqpoint{2.384548in}{1.790858in}}%
\pgfpathlineto{\pgfqpoint{2.385394in}{2.756937in}}%
\pgfpathlineto{\pgfqpoint{2.386240in}{2.368683in}}%
\pgfpathlineto{\pgfqpoint{2.387931in}{2.428960in}}%
\pgfpathlineto{\pgfqpoint{2.388776in}{2.344844in}}%
\pgfpathlineto{\pgfqpoint{2.389622in}{2.542006in}}%
\pgfpathlineto{\pgfqpoint{2.391313in}{2.027861in}}%
\pgfpathlineto{\pgfqpoint{2.393850in}{2.466862in}}%
\pgfpathlineto{\pgfqpoint{2.394696in}{1.634216in}}%
\pgfpathlineto{\pgfqpoint{2.395541in}{2.308714in}}%
\pgfpathlineto{\pgfqpoint{2.398078in}{1.740156in}}%
\pgfpathlineto{\pgfqpoint{2.398924in}{1.812067in}}%
\pgfpathlineto{\pgfqpoint{2.400615in}{2.738317in}}%
\pgfpathlineto{\pgfqpoint{2.401461in}{2.543075in}}%
\pgfpathlineto{\pgfqpoint{2.403152in}{2.384986in}}%
\pgfpathlineto{\pgfqpoint{2.403998in}{2.650832in}}%
\pgfpathlineto{\pgfqpoint{2.404843in}{2.080312in}}%
\pgfpathlineto{\pgfqpoint{2.405689in}{2.420767in}}%
\pgfpathlineto{\pgfqpoint{2.408226in}{1.675551in}}%
\pgfpathlineto{\pgfqpoint{2.410763in}{2.473276in}}%
\pgfpathlineto{\pgfqpoint{2.411608in}{1.581794in}}%
\pgfpathlineto{\pgfqpoint{2.412454in}{1.889495in}}%
\pgfpathlineto{\pgfqpoint{2.414991in}{2.508546in}}%
\pgfpathlineto{\pgfqpoint{2.415836in}{2.215343in}}%
\pgfpathlineto{\pgfqpoint{2.416682in}{2.937944in}}%
\pgfpathlineto{\pgfqpoint{2.417528in}{2.599423in}}%
\pgfpathlineto{\pgfqpoint{2.418373in}{2.549472in}}%
\pgfpathlineto{\pgfqpoint{2.420064in}{2.751059in}}%
\pgfpathlineto{\pgfqpoint{2.421756in}{1.740521in}}%
\pgfpathlineto{\pgfqpoint{2.423447in}{1.941362in}}%
\pgfpathlineto{\pgfqpoint{2.424293in}{2.492023in}}%
\pgfpathlineto{\pgfqpoint{2.425138in}{2.426864in}}%
\pgfpathlineto{\pgfqpoint{2.425984in}{2.190349in}}%
\pgfpathlineto{\pgfqpoint{2.426829in}{2.454729in}}%
\pgfpathlineto{\pgfqpoint{2.428521in}{1.880397in}}%
\pgfpathlineto{\pgfqpoint{2.430212in}{2.344577in}}%
\pgfpathlineto{\pgfqpoint{2.431058in}{2.015766in}}%
\pgfpathlineto{\pgfqpoint{2.431903in}{2.236077in}}%
\pgfpathlineto{\pgfqpoint{2.433594in}{3.140746in}}%
\pgfpathlineto{\pgfqpoint{2.434440in}{1.642218in}}%
\pgfpathlineto{\pgfqpoint{2.435286in}{2.031372in}}%
\pgfpathlineto{\pgfqpoint{2.436131in}{2.304814in}}%
\pgfpathlineto{\pgfqpoint{2.436977in}{1.760731in}}%
\pgfpathlineto{\pgfqpoint{2.437823in}{2.069728in}}%
\pgfpathlineto{\pgfqpoint{2.440359in}{2.807978in}}%
\pgfpathlineto{\pgfqpoint{2.441205in}{2.108263in}}%
\pgfpathlineto{\pgfqpoint{2.442051in}{2.472395in}}%
\pgfpathlineto{\pgfqpoint{2.442896in}{3.065478in}}%
\pgfpathlineto{\pgfqpoint{2.445433in}{1.950374in}}%
\pgfpathlineto{\pgfqpoint{2.446279in}{1.841490in}}%
\pgfpathlineto{\pgfqpoint{2.447970in}{2.377765in}}%
\pgfpathlineto{\pgfqpoint{2.449661in}{3.007574in}}%
\pgfpathlineto{\pgfqpoint{2.452198in}{1.792394in}}%
\pgfpathlineto{\pgfqpoint{2.453044in}{2.240949in}}%
\pgfpathlineto{\pgfqpoint{2.453889in}{1.770479in}}%
\pgfpathlineto{\pgfqpoint{2.454735in}{2.470308in}}%
\pgfpathlineto{\pgfqpoint{2.455581in}{2.082873in}}%
\pgfpathlineto{\pgfqpoint{2.456426in}{1.928778in}}%
\pgfpathlineto{\pgfqpoint{2.458118in}{2.884204in}}%
\pgfpathlineto{\pgfqpoint{2.461500in}{1.817805in}}%
\pgfpathlineto{\pgfqpoint{2.463191in}{2.726545in}}%
\pgfpathlineto{\pgfqpoint{2.464037in}{1.443020in}}%
\pgfpathlineto{\pgfqpoint{2.464883in}{1.900383in}}%
\pgfpathlineto{\pgfqpoint{2.466574in}{2.427146in}}%
\pgfpathlineto{\pgfqpoint{2.467419in}{1.945102in}}%
\pgfpathlineto{\pgfqpoint{2.468265in}{2.008893in}}%
\pgfpathlineto{\pgfqpoint{2.469111in}{2.316180in}}%
\pgfpathlineto{\pgfqpoint{2.469956in}{1.977983in}}%
\pgfpathlineto{\pgfqpoint{2.470802in}{2.578745in}}%
\pgfpathlineto{\pgfqpoint{2.471648in}{2.510011in}}%
\pgfpathlineto{\pgfqpoint{2.473339in}{1.814158in}}%
\pgfpathlineto{\pgfqpoint{2.474184in}{2.211698in}}%
\pgfpathlineto{\pgfqpoint{2.476721in}{2.594762in}}%
\pgfpathlineto{\pgfqpoint{2.477567in}{2.345928in}}%
\pgfpathlineto{\pgfqpoint{2.478413in}{2.662942in}}%
\pgfpathlineto{\pgfqpoint{2.479258in}{2.516826in}}%
\pgfpathlineto{\pgfqpoint{2.480104in}{2.094375in}}%
\pgfpathlineto{\pgfqpoint{2.480949in}{2.574753in}}%
\pgfpathlineto{\pgfqpoint{2.482641in}{1.888420in}}%
\pgfpathlineto{\pgfqpoint{2.484332in}{2.339079in}}%
\pgfpathlineto{\pgfqpoint{2.485177in}{1.654377in}}%
\pgfpathlineto{\pgfqpoint{2.486023in}{1.856253in}}%
\pgfpathlineto{\pgfqpoint{2.486869in}{2.613166in}}%
\pgfpathlineto{\pgfqpoint{2.487714in}{2.406914in}}%
\pgfpathlineto{\pgfqpoint{2.488560in}{1.870951in}}%
\pgfpathlineto{\pgfqpoint{2.489406in}{2.538116in}}%
\pgfpathlineto{\pgfqpoint{2.490251in}{1.956831in}}%
\pgfpathlineto{\pgfqpoint{2.491097in}{2.526397in}}%
\pgfpathlineto{\pgfqpoint{2.491942in}{2.163552in}}%
\pgfpathlineto{\pgfqpoint{2.492788in}{2.440312in}}%
\pgfpathlineto{\pgfqpoint{2.493634in}{2.137830in}}%
\pgfpathlineto{\pgfqpoint{2.494479in}{2.555063in}}%
\pgfpathlineto{\pgfqpoint{2.497016in}{1.812594in}}%
\pgfpathlineto{\pgfqpoint{2.498707in}{2.264774in}}%
\pgfpathlineto{\pgfqpoint{2.499553in}{1.868131in}}%
\pgfpathlineto{\pgfqpoint{2.500399in}{2.666825in}}%
\pgfpathlineto{\pgfqpoint{2.501244in}{2.142236in}}%
\pgfpathlineto{\pgfqpoint{2.502936in}{2.718072in}}%
\pgfpathlineto{\pgfqpoint{2.506318in}{2.276091in}}%
\pgfpathlineto{\pgfqpoint{2.508009in}{2.382891in}}%
\pgfpathlineto{\pgfqpoint{2.508855in}{2.229862in}}%
\pgfpathlineto{\pgfqpoint{2.509701in}{2.351266in}}%
\pgfpathlineto{\pgfqpoint{2.510546in}{1.623795in}}%
\pgfpathlineto{\pgfqpoint{2.511392in}{2.251233in}}%
\pgfpathlineto{\pgfqpoint{2.512237in}{2.202348in}}%
\pgfpathlineto{\pgfqpoint{2.513929in}{2.052579in}}%
\pgfpathlineto{\pgfqpoint{2.515620in}{2.558803in}}%
\pgfpathlineto{\pgfqpoint{2.516466in}{2.542337in}}%
\pgfpathlineto{\pgfqpoint{2.518157in}{1.886787in}}%
\pgfpathlineto{\pgfqpoint{2.519848in}{1.483163in}}%
\pgfpathlineto{\pgfqpoint{2.521539in}{2.783881in}}%
\pgfpathlineto{\pgfqpoint{2.522385in}{1.664547in}}%
\pgfpathlineto{\pgfqpoint{2.523231in}{2.042510in}}%
\pgfpathlineto{\pgfqpoint{2.524076in}{2.397580in}}%
\pgfpathlineto{\pgfqpoint{2.524922in}{2.116675in}}%
\pgfpathlineto{\pgfqpoint{2.525767in}{2.090906in}}%
\pgfpathlineto{\pgfqpoint{2.526613in}{2.936408in}}%
\pgfpathlineto{\pgfqpoint{2.527459in}{2.738294in}}%
\pgfpathlineto{\pgfqpoint{2.528304in}{1.855466in}}%
\pgfpathlineto{\pgfqpoint{2.529150in}{2.678766in}}%
\pgfpathlineto{\pgfqpoint{2.529996in}{2.337199in}}%
\pgfpathlineto{\pgfqpoint{2.530841in}{2.336379in}}%
\pgfpathlineto{\pgfqpoint{2.531687in}{2.158646in}}%
\pgfpathlineto{\pgfqpoint{2.532532in}{2.554352in}}%
\pgfpathlineto{\pgfqpoint{2.533378in}{2.014088in}}%
\pgfpathlineto{\pgfqpoint{2.534224in}{2.632123in}}%
\pgfpathlineto{\pgfqpoint{2.535069in}{1.537980in}}%
\pgfpathlineto{\pgfqpoint{2.535915in}{1.732566in}}%
\pgfpathlineto{\pgfqpoint{2.536761in}{1.901238in}}%
\pgfpathlineto{\pgfqpoint{2.538452in}{2.993031in}}%
\pgfpathlineto{\pgfqpoint{2.539297in}{2.794392in}}%
\pgfpathlineto{\pgfqpoint{2.540143in}{1.594021in}}%
\pgfpathlineto{\pgfqpoint{2.540989in}{2.209525in}}%
\pgfpathlineto{\pgfqpoint{2.541834in}{1.545933in}}%
\pgfpathlineto{\pgfqpoint{2.542680in}{2.270394in}}%
\pgfpathlineto{\pgfqpoint{2.543526in}{2.203322in}}%
\pgfpathlineto{\pgfqpoint{2.544371in}{1.691469in}}%
\pgfpathlineto{\pgfqpoint{2.545217in}{2.066524in}}%
\pgfpathlineto{\pgfqpoint{2.547754in}{2.461613in}}%
\pgfpathlineto{\pgfqpoint{2.549445in}{2.264151in}}%
\pgfpathlineto{\pgfqpoint{2.551136in}{1.339497in}}%
\pgfpathlineto{\pgfqpoint{2.551982in}{2.630257in}}%
\pgfpathlineto{\pgfqpoint{2.552827in}{2.415599in}}%
\pgfpathlineto{\pgfqpoint{2.553673in}{1.873790in}}%
\pgfpathlineto{\pgfqpoint{2.554519in}{2.354491in}}%
\pgfpathlineto{\pgfqpoint{2.555364in}{2.276736in}}%
\pgfpathlineto{\pgfqpoint{2.556210in}{1.868058in}}%
\pgfpathlineto{\pgfqpoint{2.557056in}{1.868846in}}%
\pgfpathlineto{\pgfqpoint{2.557901in}{1.917317in}}%
\pgfpathlineto{\pgfqpoint{2.558747in}{2.759955in}}%
\pgfpathlineto{\pgfqpoint{2.559592in}{1.834548in}}%
\pgfpathlineto{\pgfqpoint{2.560438in}{2.149941in}}%
\pgfpathlineto{\pgfqpoint{2.561284in}{2.271122in}}%
\pgfpathlineto{\pgfqpoint{2.562129in}{2.171984in}}%
\pgfpathlineto{\pgfqpoint{2.564666in}{2.393824in}}%
\pgfpathlineto{\pgfqpoint{2.565512in}{2.747551in}}%
\pgfpathlineto{\pgfqpoint{2.566357in}{1.884780in}}%
\pgfpathlineto{\pgfqpoint{2.567203in}{2.346387in}}%
\pgfpathlineto{\pgfqpoint{2.568049in}{2.363098in}}%
\pgfpathlineto{\pgfqpoint{2.568894in}{2.170433in}}%
\pgfpathlineto{\pgfqpoint{2.569740in}{2.346937in}}%
\pgfpathlineto{\pgfqpoint{2.571431in}{1.736395in}}%
\pgfpathlineto{\pgfqpoint{2.572277in}{2.008161in}}%
\pgfpathlineto{\pgfqpoint{2.573122in}{2.789320in}}%
\pgfpathlineto{\pgfqpoint{2.573968in}{1.800291in}}%
\pgfpathlineto{\pgfqpoint{2.574814in}{1.913231in}}%
\pgfpathlineto{\pgfqpoint{2.575659in}{2.621487in}}%
\pgfpathlineto{\pgfqpoint{2.576505in}{2.127010in}}%
\pgfpathlineto{\pgfqpoint{2.578196in}{1.552584in}}%
\pgfpathlineto{\pgfqpoint{2.579887in}{2.873044in}}%
\pgfpathlineto{\pgfqpoint{2.580733in}{2.397505in}}%
\pgfpathlineto{\pgfqpoint{2.583270in}{2.674505in}}%
\pgfpathlineto{\pgfqpoint{2.584961in}{1.912275in}}%
\pgfpathlineto{\pgfqpoint{2.585807in}{2.322350in}}%
\pgfpathlineto{\pgfqpoint{2.586652in}{2.059161in}}%
\pgfpathlineto{\pgfqpoint{2.587498in}{1.903876in}}%
\pgfpathlineto{\pgfqpoint{2.588344in}{2.603103in}}%
\pgfpathlineto{\pgfqpoint{2.589189in}{2.062277in}}%
\pgfpathlineto{\pgfqpoint{2.590035in}{2.510142in}}%
\pgfpathlineto{\pgfqpoint{2.591726in}{1.918623in}}%
\pgfpathlineto{\pgfqpoint{2.592572in}{2.411020in}}%
\pgfpathlineto{\pgfqpoint{2.595109in}{1.793071in}}%
\pgfpathlineto{\pgfqpoint{2.597645in}{2.072258in}}%
\pgfpathlineto{\pgfqpoint{2.598491in}{2.029384in}}%
\pgfpathlineto{\pgfqpoint{2.599337in}{2.980927in}}%
\pgfpathlineto{\pgfqpoint{2.600182in}{2.247960in}}%
\pgfpathlineto{\pgfqpoint{2.602719in}{1.811406in}}%
\pgfpathlineto{\pgfqpoint{2.603565in}{2.559372in}}%
\pgfpathlineto{\pgfqpoint{2.604410in}{2.306489in}}%
\pgfpathlineto{\pgfqpoint{2.605256in}{1.607696in}}%
\pgfpathlineto{\pgfqpoint{2.606947in}{2.318882in}}%
\pgfpathlineto{\pgfqpoint{2.607793in}{1.798130in}}%
\pgfpathlineto{\pgfqpoint{2.610330in}{2.879806in}}%
\pgfpathlineto{\pgfqpoint{2.611175in}{2.035959in}}%
\pgfpathlineto{\pgfqpoint{2.612021in}{2.573857in}}%
\pgfpathlineto{\pgfqpoint{2.613712in}{1.702964in}}%
\pgfpathlineto{\pgfqpoint{2.614558in}{1.942502in}}%
\pgfpathlineto{\pgfqpoint{2.615404in}{2.318697in}}%
\pgfpathlineto{\pgfqpoint{2.617940in}{1.047260in}}%
\pgfpathlineto{\pgfqpoint{2.619632in}{2.781306in}}%
\pgfpathlineto{\pgfqpoint{2.622169in}{1.832842in}}%
\pgfpathlineto{\pgfqpoint{2.623014in}{1.916448in}}%
\pgfpathlineto{\pgfqpoint{2.623860in}{1.804775in}}%
\pgfpathlineto{\pgfqpoint{2.627242in}{2.929443in}}%
\pgfpathlineto{\pgfqpoint{2.628088in}{1.930322in}}%
\pgfpathlineto{\pgfqpoint{2.628934in}{2.753862in}}%
\pgfpathlineto{\pgfqpoint{2.630625in}{1.767298in}}%
\pgfpathlineto{\pgfqpoint{2.631470in}{2.748083in}}%
\pgfpathlineto{\pgfqpoint{2.632316in}{2.469408in}}%
\pgfpathlineto{\pgfqpoint{2.633162in}{2.555094in}}%
\pgfpathlineto{\pgfqpoint{2.634853in}{1.942729in}}%
\pgfpathlineto{\pgfqpoint{2.635699in}{2.074802in}}%
\pgfpathlineto{\pgfqpoint{2.636544in}{1.957482in}}%
\pgfpathlineto{\pgfqpoint{2.639081in}{2.922969in}}%
\pgfpathlineto{\pgfqpoint{2.639927in}{1.815952in}}%
\pgfpathlineto{\pgfqpoint{2.640772in}{2.628823in}}%
\pgfpathlineto{\pgfqpoint{2.641618in}{3.017236in}}%
\pgfpathlineto{\pgfqpoint{2.642463in}{1.749640in}}%
\pgfpathlineto{\pgfqpoint{2.643309in}{2.015928in}}%
\pgfpathlineto{\pgfqpoint{2.645000in}{2.042775in}}%
\pgfpathlineto{\pgfqpoint{2.645846in}{2.685992in}}%
\pgfpathlineto{\pgfqpoint{2.646692in}{1.749547in}}%
\pgfpathlineto{\pgfqpoint{2.647537in}{1.939582in}}%
\pgfpathlineto{\pgfqpoint{2.649228in}{2.215757in}}%
\pgfpathlineto{\pgfqpoint{2.650074in}{1.665621in}}%
\pgfpathlineto{\pgfqpoint{2.651765in}{2.388006in}}%
\pgfpathlineto{\pgfqpoint{2.654302in}{2.200469in}}%
\pgfpathlineto{\pgfqpoint{2.655148in}{2.257807in}}%
\pgfpathlineto{\pgfqpoint{2.655993in}{2.483055in}}%
\pgfpathlineto{\pgfqpoint{2.656839in}{2.262402in}}%
\pgfpathlineto{\pgfqpoint{2.657685in}{2.726887in}}%
\pgfpathlineto{\pgfqpoint{2.659376in}{2.008051in}}%
\pgfpathlineto{\pgfqpoint{2.660222in}{2.428397in}}%
\pgfpathlineto{\pgfqpoint{2.661067in}{2.406446in}}%
\pgfpathlineto{\pgfqpoint{2.662758in}{1.699377in}}%
\pgfpathlineto{\pgfqpoint{2.664450in}{1.945068in}}%
\pgfpathlineto{\pgfqpoint{2.666141in}{1.704816in}}%
\pgfpathlineto{\pgfqpoint{2.666987in}{2.216145in}}%
\pgfpathlineto{\pgfqpoint{2.667832in}{1.788619in}}%
\pgfpathlineto{\pgfqpoint{2.668678in}{2.297496in}}%
\pgfpathlineto{\pgfqpoint{2.669523in}{1.981435in}}%
\pgfpathlineto{\pgfqpoint{2.670369in}{2.088132in}}%
\pgfpathlineto{\pgfqpoint{2.671215in}{1.621072in}}%
\pgfpathlineto{\pgfqpoint{2.673752in}{2.513200in}}%
\pgfpathlineto{\pgfqpoint{2.675443in}{2.336886in}}%
\pgfpathlineto{\pgfqpoint{2.677134in}{2.535664in}}%
\pgfpathlineto{\pgfqpoint{2.678825in}{1.731280in}}%
\pgfpathlineto{\pgfqpoint{2.679671in}{2.521706in}}%
\pgfpathlineto{\pgfqpoint{2.680517in}{2.216717in}}%
\pgfpathlineto{\pgfqpoint{2.682208in}{2.090946in}}%
\pgfpathlineto{\pgfqpoint{2.683053in}{2.180073in}}%
\pgfpathlineto{\pgfqpoint{2.683899in}{1.945414in}}%
\pgfpathlineto{\pgfqpoint{2.684745in}{1.952993in}}%
\pgfpathlineto{\pgfqpoint{2.686436in}{2.712359in}}%
\pgfpathlineto{\pgfqpoint{2.688127in}{1.950079in}}%
\pgfpathlineto{\pgfqpoint{2.688973in}{2.374421in}}%
\pgfpathlineto{\pgfqpoint{2.689818in}{2.169516in}}%
\pgfpathlineto{\pgfqpoint{2.690664in}{1.788520in}}%
\pgfpathlineto{\pgfqpoint{2.691510in}{2.449363in}}%
\pgfpathlineto{\pgfqpoint{2.692355in}{2.256019in}}%
\pgfpathlineto{\pgfqpoint{2.693201in}{2.532466in}}%
\pgfpathlineto{\pgfqpoint{2.694047in}{1.995378in}}%
\pgfpathlineto{\pgfqpoint{2.694892in}{3.024951in}}%
\pgfpathlineto{\pgfqpoint{2.695738in}{2.342122in}}%
\pgfpathlineto{\pgfqpoint{2.696583in}{1.640754in}}%
\pgfpathlineto{\pgfqpoint{2.697429in}{3.229707in}}%
\pgfpathlineto{\pgfqpoint{2.698275in}{1.401451in}}%
\pgfpathlineto{\pgfqpoint{2.699120in}{2.631596in}}%
\pgfpathlineto{\pgfqpoint{2.699966in}{2.506736in}}%
\pgfpathlineto{\pgfqpoint{2.700812in}{2.615909in}}%
\pgfpathlineto{\pgfqpoint{2.703348in}{2.144857in}}%
\pgfpathlineto{\pgfqpoint{2.704194in}{2.577492in}}%
\pgfpathlineto{\pgfqpoint{2.705040in}{2.331631in}}%
\pgfpathlineto{\pgfqpoint{2.705885in}{2.640412in}}%
\pgfpathlineto{\pgfqpoint{2.707577in}{1.900625in}}%
\pgfpathlineto{\pgfqpoint{2.708422in}{2.680478in}}%
\pgfpathlineto{\pgfqpoint{2.709268in}{1.588966in}}%
\pgfpathlineto{\pgfqpoint{2.710113in}{2.252853in}}%
\pgfpathlineto{\pgfqpoint{2.710959in}{1.953817in}}%
\pgfpathlineto{\pgfqpoint{2.714342in}{2.903250in}}%
\pgfpathlineto{\pgfqpoint{2.715187in}{1.981987in}}%
\pgfpathlineto{\pgfqpoint{2.716033in}{2.291277in}}%
\pgfpathlineto{\pgfqpoint{2.716878in}{2.251696in}}%
\pgfpathlineto{\pgfqpoint{2.717724in}{1.665178in}}%
\pgfpathlineto{\pgfqpoint{2.718570in}{2.865184in}}%
\pgfpathlineto{\pgfqpoint{2.719415in}{2.175228in}}%
\pgfpathlineto{\pgfqpoint{2.721106in}{1.537804in}}%
\pgfpathlineto{\pgfqpoint{2.721952in}{2.501034in}}%
\pgfpathlineto{\pgfqpoint{2.722798in}{2.306744in}}%
\pgfpathlineto{\pgfqpoint{2.724489in}{2.043458in}}%
\pgfpathlineto{\pgfqpoint{2.727871in}{2.601587in}}%
\pgfpathlineto{\pgfqpoint{2.728717in}{2.710607in}}%
\pgfpathlineto{\pgfqpoint{2.730408in}{1.779607in}}%
\pgfpathlineto{\pgfqpoint{2.731254in}{2.163718in}}%
\pgfpathlineto{\pgfqpoint{2.732100in}{2.190389in}}%
\pgfpathlineto{\pgfqpoint{2.733791in}{1.990772in}}%
\pgfpathlineto{\pgfqpoint{2.734636in}{2.236494in}}%
\pgfpathlineto{\pgfqpoint{2.735482in}{2.176324in}}%
\pgfpathlineto{\pgfqpoint{2.736328in}{2.211016in}}%
\pgfpathlineto{\pgfqpoint{2.737173in}{2.896105in}}%
\pgfpathlineto{\pgfqpoint{2.738865in}{2.131817in}}%
\pgfpathlineto{\pgfqpoint{2.739710in}{2.352204in}}%
\pgfpathlineto{\pgfqpoint{2.741401in}{2.133364in}}%
\pgfpathlineto{\pgfqpoint{2.742247in}{2.132299in}}%
\pgfpathlineto{\pgfqpoint{2.743093in}{2.428525in}}%
\pgfpathlineto{\pgfqpoint{2.745630in}{1.725853in}}%
\pgfpathlineto{\pgfqpoint{2.748166in}{2.371247in}}%
\pgfpathlineto{\pgfqpoint{2.749012in}{2.657433in}}%
\pgfpathlineto{\pgfqpoint{2.750703in}{2.012341in}}%
\pgfpathlineto{\pgfqpoint{2.751549in}{2.403681in}}%
\pgfpathlineto{\pgfqpoint{2.752395in}{2.214140in}}%
\pgfpathlineto{\pgfqpoint{2.753240in}{2.181190in}}%
\pgfpathlineto{\pgfqpoint{2.754086in}{1.930454in}}%
\pgfpathlineto{\pgfqpoint{2.754931in}{2.513003in}}%
\pgfpathlineto{\pgfqpoint{2.755777in}{2.367480in}}%
\pgfpathlineto{\pgfqpoint{2.758314in}{2.237067in}}%
\pgfpathlineto{\pgfqpoint{2.759160in}{2.359065in}}%
\pgfpathlineto{\pgfqpoint{2.760005in}{2.058373in}}%
\pgfpathlineto{\pgfqpoint{2.760851in}{2.298977in}}%
\pgfpathlineto{\pgfqpoint{2.761696in}{2.252246in}}%
\pgfpathlineto{\pgfqpoint{2.762542in}{1.932776in}}%
\pgfpathlineto{\pgfqpoint{2.763388in}{1.999866in}}%
\pgfpathlineto{\pgfqpoint{2.764233in}{2.561358in}}%
\pgfpathlineto{\pgfqpoint{2.765079in}{1.532835in}}%
\pgfpathlineto{\pgfqpoint{2.765925in}{2.290314in}}%
\pgfpathlineto{\pgfqpoint{2.766770in}{2.430844in}}%
\pgfpathlineto{\pgfqpoint{2.768461in}{2.168813in}}%
\pgfpathlineto{\pgfqpoint{2.769307in}{2.429406in}}%
\pgfpathlineto{\pgfqpoint{2.770153in}{2.240176in}}%
\pgfpathlineto{\pgfqpoint{2.770998in}{2.186731in}}%
\pgfpathlineto{\pgfqpoint{2.771844in}{1.889731in}}%
\pgfpathlineto{\pgfqpoint{2.774381in}{2.651099in}}%
\pgfpathlineto{\pgfqpoint{2.776072in}{1.887942in}}%
\pgfpathlineto{\pgfqpoint{2.776918in}{1.887870in}}%
\pgfpathlineto{\pgfqpoint{2.778609in}{2.391507in}}%
\pgfpathlineto{\pgfqpoint{2.780300in}{1.828790in}}%
\pgfpathlineto{\pgfqpoint{2.781146in}{1.924099in}}%
\pgfpathlineto{\pgfqpoint{2.782837in}{2.203466in}}%
\pgfpathlineto{\pgfqpoint{2.783683in}{1.832646in}}%
\pgfpathlineto{\pgfqpoint{2.786220in}{2.395443in}}%
\pgfpathlineto{\pgfqpoint{2.787065in}{2.442320in}}%
\pgfpathlineto{\pgfqpoint{2.789602in}{1.908180in}}%
\pgfpathlineto{\pgfqpoint{2.790448in}{2.434293in}}%
\pgfpathlineto{\pgfqpoint{2.791293in}{2.251796in}}%
\pgfpathlineto{\pgfqpoint{2.792139in}{2.248085in}}%
\pgfpathlineto{\pgfqpoint{2.792985in}{2.030530in}}%
\pgfpathlineto{\pgfqpoint{2.794676in}{2.729123in}}%
\pgfpathlineto{\pgfqpoint{2.795521in}{1.879614in}}%
\pgfpathlineto{\pgfqpoint{2.796367in}{1.940370in}}%
\pgfpathlineto{\pgfqpoint{2.797213in}{2.001438in}}%
\pgfpathlineto{\pgfqpoint{2.798058in}{2.183526in}}%
\pgfpathlineto{\pgfqpoint{2.798904in}{1.725218in}}%
\pgfpathlineto{\pgfqpoint{2.799749in}{2.615306in}}%
\pgfpathlineto{\pgfqpoint{2.800595in}{2.437204in}}%
\pgfpathlineto{\pgfqpoint{2.801441in}{2.575156in}}%
\pgfpathlineto{\pgfqpoint{2.803132in}{1.553925in}}%
\pgfpathlineto{\pgfqpoint{2.804823in}{2.203728in}}%
\pgfpathlineto{\pgfqpoint{2.805669in}{1.767852in}}%
\pgfpathlineto{\pgfqpoint{2.806514in}{2.046380in}}%
\pgfpathlineto{\pgfqpoint{2.807360in}{1.983001in}}%
\pgfpathlineto{\pgfqpoint{2.808206in}{2.484206in}}%
\pgfpathlineto{\pgfqpoint{2.809051in}{2.222426in}}%
\pgfpathlineto{\pgfqpoint{2.809897in}{1.641846in}}%
\pgfpathlineto{\pgfqpoint{2.812434in}{2.436378in}}%
\pgfpathlineto{\pgfqpoint{2.813279in}{2.874935in}}%
\pgfpathlineto{\pgfqpoint{2.814125in}{2.088144in}}%
\pgfpathlineto{\pgfqpoint{2.814971in}{2.205121in}}%
\pgfpathlineto{\pgfqpoint{2.815816in}{2.366400in}}%
\pgfpathlineto{\pgfqpoint{2.817508in}{1.624136in}}%
\pgfpathlineto{\pgfqpoint{2.818353in}{2.742605in}}%
\pgfpathlineto{\pgfqpoint{2.819199in}{2.346962in}}%
\pgfpathlineto{\pgfqpoint{2.820044in}{2.095630in}}%
\pgfpathlineto{\pgfqpoint{2.820890in}{2.426573in}}%
\pgfpathlineto{\pgfqpoint{2.821736in}{2.416537in}}%
\pgfpathlineto{\pgfqpoint{2.822581in}{1.742897in}}%
\pgfpathlineto{\pgfqpoint{2.823427in}{1.917971in}}%
\pgfpathlineto{\pgfqpoint{2.825964in}{2.266306in}}%
\pgfpathlineto{\pgfqpoint{2.826809in}{2.217124in}}%
\pgfpathlineto{\pgfqpoint{2.827655in}{2.857759in}}%
\pgfpathlineto{\pgfqpoint{2.828501in}{1.919750in}}%
\pgfpathlineto{\pgfqpoint{2.829346in}{2.357215in}}%
\pgfpathlineto{\pgfqpoint{2.830192in}{2.400937in}}%
\pgfpathlineto{\pgfqpoint{2.831038in}{2.397536in}}%
\pgfpathlineto{\pgfqpoint{2.831883in}{1.280001in}}%
\pgfpathlineto{\pgfqpoint{2.832729in}{2.116656in}}%
\pgfpathlineto{\pgfqpoint{2.833574in}{1.880778in}}%
\pgfpathlineto{\pgfqpoint{2.834420in}{2.563770in}}%
\pgfpathlineto{\pgfqpoint{2.835266in}{2.084913in}}%
\pgfpathlineto{\pgfqpoint{2.836111in}{2.532579in}}%
\pgfpathlineto{\pgfqpoint{2.836957in}{2.203532in}}%
\pgfpathlineto{\pgfqpoint{2.837803in}{2.296726in}}%
\pgfpathlineto{\pgfqpoint{2.839494in}{2.798827in}}%
\pgfpathlineto{\pgfqpoint{2.840339in}{1.948909in}}%
\pgfpathlineto{\pgfqpoint{2.841185in}{2.490449in}}%
\pgfpathlineto{\pgfqpoint{2.842031in}{2.468061in}}%
\pgfpathlineto{\pgfqpoint{2.842876in}{1.940505in}}%
\pgfpathlineto{\pgfqpoint{2.843722in}{2.207678in}}%
\pgfpathlineto{\pgfqpoint{2.845413in}{2.372112in}}%
\pgfpathlineto{\pgfqpoint{2.847104in}{1.744060in}}%
\pgfpathlineto{\pgfqpoint{2.847950in}{2.205494in}}%
\pgfpathlineto{\pgfqpoint{2.848796in}{1.855879in}}%
\pgfpathlineto{\pgfqpoint{2.849641in}{1.913998in}}%
\pgfpathlineto{\pgfqpoint{2.850487in}{2.389142in}}%
\pgfpathlineto{\pgfqpoint{2.851333in}{1.994944in}}%
\pgfpathlineto{\pgfqpoint{2.852178in}{2.366043in}}%
\pgfpathlineto{\pgfqpoint{2.853024in}{2.217429in}}%
\pgfpathlineto{\pgfqpoint{2.853869in}{1.858358in}}%
\pgfpathlineto{\pgfqpoint{2.857252in}{2.981142in}}%
\pgfpathlineto{\pgfqpoint{2.858098in}{1.888295in}}%
\pgfpathlineto{\pgfqpoint{2.858943in}{2.524837in}}%
\pgfpathlineto{\pgfqpoint{2.859789in}{1.712219in}}%
\pgfpathlineto{\pgfqpoint{2.860634in}{1.891669in}}%
\pgfpathlineto{\pgfqpoint{2.861480in}{1.823936in}}%
\pgfpathlineto{\pgfqpoint{2.862326in}{2.248465in}}%
\pgfpathlineto{\pgfqpoint{2.863171in}{2.169740in}}%
\pgfpathlineto{\pgfqpoint{2.864017in}{2.022802in}}%
\pgfpathlineto{\pgfqpoint{2.865708in}{2.930743in}}%
\pgfpathlineto{\pgfqpoint{2.867399in}{1.736733in}}%
\pgfpathlineto{\pgfqpoint{2.868245in}{1.816733in}}%
\pgfpathlineto{\pgfqpoint{2.869091in}{2.337459in}}%
\pgfpathlineto{\pgfqpoint{2.869936in}{1.834596in}}%
\pgfpathlineto{\pgfqpoint{2.870782in}{2.524245in}}%
\pgfpathlineto{\pgfqpoint{2.871628in}{2.346959in}}%
\pgfpathlineto{\pgfqpoint{2.872473in}{1.959553in}}%
\pgfpathlineto{\pgfqpoint{2.873319in}{2.226033in}}%
\pgfpathlineto{\pgfqpoint{2.874164in}{2.293364in}}%
\pgfpathlineto{\pgfqpoint{2.875010in}{1.769536in}}%
\pgfpathlineto{\pgfqpoint{2.875856in}{2.584184in}}%
\pgfpathlineto{\pgfqpoint{2.876701in}{2.534133in}}%
\pgfpathlineto{\pgfqpoint{2.877547in}{2.276405in}}%
\pgfpathlineto{\pgfqpoint{2.878392in}{2.704496in}}%
\pgfpathlineto{\pgfqpoint{2.880084in}{1.797029in}}%
\pgfpathlineto{\pgfqpoint{2.880929in}{2.617876in}}%
\pgfpathlineto{\pgfqpoint{2.881775in}{2.202321in}}%
\pgfpathlineto{\pgfqpoint{2.882621in}{1.548046in}}%
\pgfpathlineto{\pgfqpoint{2.883466in}{2.321943in}}%
\pgfpathlineto{\pgfqpoint{2.884312in}{1.971588in}}%
\pgfpathlineto{\pgfqpoint{2.886003in}{1.899324in}}%
\pgfpathlineto{\pgfqpoint{2.887694in}{2.231541in}}%
\pgfpathlineto{\pgfqpoint{2.889386in}{1.861190in}}%
\pgfpathlineto{\pgfqpoint{2.890231in}{2.215546in}}%
\pgfpathlineto{\pgfqpoint{2.891077in}{1.794650in}}%
\pgfpathlineto{\pgfqpoint{2.891922in}{2.593806in}}%
\pgfpathlineto{\pgfqpoint{2.892768in}{2.337806in}}%
\pgfpathlineto{\pgfqpoint{2.893614in}{2.297210in}}%
\pgfpathlineto{\pgfqpoint{2.895305in}{1.876163in}}%
\pgfpathlineto{\pgfqpoint{2.896151in}{2.067389in}}%
\pgfpathlineto{\pgfqpoint{2.896996in}{2.042922in}}%
\pgfpathlineto{\pgfqpoint{2.897842in}{2.024129in}}%
\pgfpathlineto{\pgfqpoint{2.899533in}{1.593790in}}%
\pgfpathlineto{\pgfqpoint{2.900379in}{2.648177in}}%
\pgfpathlineto{\pgfqpoint{2.901224in}{2.139144in}}%
\pgfpathlineto{\pgfqpoint{2.902070in}{2.144519in}}%
\pgfpathlineto{\pgfqpoint{2.902916in}{2.572729in}}%
\pgfpathlineto{\pgfqpoint{2.903761in}{2.248716in}}%
\pgfpathlineto{\pgfqpoint{2.904607in}{2.400233in}}%
\pgfpathlineto{\pgfqpoint{2.905452in}{2.028978in}}%
\pgfpathlineto{\pgfqpoint{2.906298in}{2.098848in}}%
\pgfpathlineto{\pgfqpoint{2.907144in}{2.902821in}}%
\pgfpathlineto{\pgfqpoint{2.907989in}{2.004254in}}%
\pgfpathlineto{\pgfqpoint{2.908835in}{2.134402in}}%
\pgfpathlineto{\pgfqpoint{2.909681in}{3.106334in}}%
\pgfpathlineto{\pgfqpoint{2.911372in}{2.284015in}}%
\pgfpathlineto{\pgfqpoint{2.912217in}{2.654156in}}%
\pgfpathlineto{\pgfqpoint{2.913909in}{1.703156in}}%
\pgfpathlineto{\pgfqpoint{2.914754in}{1.808301in}}%
\pgfpathlineto{\pgfqpoint{2.916446in}{2.520629in}}%
\pgfpathlineto{\pgfqpoint{2.917291in}{2.419751in}}%
\pgfpathlineto{\pgfqpoint{2.918137in}{2.357842in}}%
\pgfpathlineto{\pgfqpoint{2.918982in}{2.024745in}}%
\pgfpathlineto{\pgfqpoint{2.919828in}{2.393821in}}%
\pgfpathlineto{\pgfqpoint{2.921519in}{1.401690in}}%
\pgfpathlineto{\pgfqpoint{2.922365in}{2.263168in}}%
\pgfpathlineto{\pgfqpoint{2.923211in}{2.239340in}}%
\pgfpathlineto{\pgfqpoint{2.924056in}{1.671003in}}%
\pgfpathlineto{\pgfqpoint{2.924902in}{2.135170in}}%
\pgfpathlineto{\pgfqpoint{2.925747in}{2.139635in}}%
\pgfpathlineto{\pgfqpoint{2.926593in}{2.761946in}}%
\pgfpathlineto{\pgfqpoint{2.928284in}{2.014170in}}%
\pgfpathlineto{\pgfqpoint{2.929130in}{2.321731in}}%
\pgfpathlineto{\pgfqpoint{2.930821in}{1.794286in}}%
\pgfpathlineto{\pgfqpoint{2.931667in}{2.211109in}}%
\pgfpathlineto{\pgfqpoint{2.932512in}{2.149705in}}%
\pgfpathlineto{\pgfqpoint{2.933358in}{2.159218in}}%
\pgfpathlineto{\pgfqpoint{2.934204in}{2.730821in}}%
\pgfpathlineto{\pgfqpoint{2.935049in}{2.093188in}}%
\pgfpathlineto{\pgfqpoint{2.935895in}{2.474203in}}%
\pgfpathlineto{\pgfqpoint{2.936741in}{2.258197in}}%
\pgfpathlineto{\pgfqpoint{2.937586in}{2.330427in}}%
\pgfpathlineto{\pgfqpoint{2.938432in}{2.280043in}}%
\pgfpathlineto{\pgfqpoint{2.939277in}{2.940454in}}%
\pgfpathlineto{\pgfqpoint{2.940123in}{2.436699in}}%
\pgfpathlineto{\pgfqpoint{2.940969in}{1.803259in}}%
\pgfpathlineto{\pgfqpoint{2.941814in}{2.557922in}}%
\pgfpathlineto{\pgfqpoint{2.942660in}{2.437081in}}%
\pgfpathlineto{\pgfqpoint{2.943506in}{1.885987in}}%
\pgfpathlineto{\pgfqpoint{2.944351in}{3.032779in}}%
\pgfpathlineto{\pgfqpoint{2.946042in}{1.770938in}}%
\pgfpathlineto{\pgfqpoint{2.946888in}{2.915426in}}%
\pgfpathlineto{\pgfqpoint{2.947734in}{2.491042in}}%
\pgfpathlineto{\pgfqpoint{2.948579in}{1.764109in}}%
\pgfpathlineto{\pgfqpoint{2.949425in}{1.803068in}}%
\pgfpathlineto{\pgfqpoint{2.951116in}{2.746299in}}%
\pgfpathlineto{\pgfqpoint{2.952807in}{1.446885in}}%
\pgfpathlineto{\pgfqpoint{2.955344in}{2.537227in}}%
\pgfpathlineto{\pgfqpoint{2.957881in}{2.089671in}}%
\pgfpathlineto{\pgfqpoint{2.958727in}{2.549903in}}%
\pgfpathlineto{\pgfqpoint{2.961264in}{1.314569in}}%
\pgfpathlineto{\pgfqpoint{2.963800in}{2.187718in}}%
\pgfpathlineto{\pgfqpoint{2.964646in}{2.091659in}}%
\pgfpathlineto{\pgfqpoint{2.965492in}{3.296457in}}%
\pgfpathlineto{\pgfqpoint{2.966337in}{2.158839in}}%
\pgfpathlineto{\pgfqpoint{2.967183in}{2.789820in}}%
\pgfpathlineto{\pgfqpoint{2.968029in}{2.050125in}}%
\pgfpathlineto{\pgfqpoint{2.968874in}{2.107493in}}%
\pgfpathlineto{\pgfqpoint{2.970565in}{2.533406in}}%
\pgfpathlineto{\pgfqpoint{2.971411in}{2.462986in}}%
\pgfpathlineto{\pgfqpoint{2.973102in}{1.855827in}}%
\pgfpathlineto{\pgfqpoint{2.973948in}{2.811345in}}%
\pgfpathlineto{\pgfqpoint{2.974794in}{2.282718in}}%
\pgfpathlineto{\pgfqpoint{2.975639in}{2.352187in}}%
\pgfpathlineto{\pgfqpoint{2.976485in}{1.644677in}}%
\pgfpathlineto{\pgfqpoint{2.977330in}{2.222322in}}%
\pgfpathlineto{\pgfqpoint{2.978176in}{1.723810in}}%
\pgfpathlineto{\pgfqpoint{2.979022in}{2.256103in}}%
\pgfpathlineto{\pgfqpoint{2.979867in}{1.815151in}}%
\pgfpathlineto{\pgfqpoint{2.982404in}{2.804242in}}%
\pgfpathlineto{\pgfqpoint{2.983250in}{1.686455in}}%
\pgfpathlineto{\pgfqpoint{2.984095in}{2.344779in}}%
\pgfpathlineto{\pgfqpoint{2.984941in}{2.289333in}}%
\pgfpathlineto{\pgfqpoint{2.986632in}{2.914496in}}%
\pgfpathlineto{\pgfqpoint{2.987478in}{2.556832in}}%
\pgfpathlineto{\pgfqpoint{2.989169in}{2.400371in}}%
\pgfpathlineto{\pgfqpoint{2.990015in}{2.472943in}}%
\pgfpathlineto{\pgfqpoint{2.990860in}{1.832609in}}%
\pgfpathlineto{\pgfqpoint{2.991706in}{2.225965in}}%
\pgfpathlineto{\pgfqpoint{2.993397in}{2.450775in}}%
\pgfpathlineto{\pgfqpoint{2.994243in}{2.224596in}}%
\pgfpathlineto{\pgfqpoint{2.995089in}{2.398594in}}%
\pgfpathlineto{\pgfqpoint{2.995934in}{2.227750in}}%
\pgfpathlineto{\pgfqpoint{2.996780in}{2.751703in}}%
\pgfpathlineto{\pgfqpoint{2.997625in}{2.042215in}}%
\pgfpathlineto{\pgfqpoint{2.998471in}{2.241967in}}%
\pgfpathlineto{\pgfqpoint{3.000162in}{2.747315in}}%
\pgfpathlineto{\pgfqpoint{3.001854in}{2.002342in}}%
\pgfpathlineto{\pgfqpoint{3.002699in}{2.787959in}}%
\pgfpathlineto{\pgfqpoint{3.003545in}{2.404106in}}%
\pgfpathlineto{\pgfqpoint{3.005236in}{2.116330in}}%
\pgfpathlineto{\pgfqpoint{3.006082in}{2.152992in}}%
\pgfpathlineto{\pgfqpoint{3.006927in}{2.271459in}}%
\pgfpathlineto{\pgfqpoint{3.007773in}{1.730395in}}%
\pgfpathlineto{\pgfqpoint{3.008619in}{2.587893in}}%
\pgfpathlineto{\pgfqpoint{3.009464in}{1.774432in}}%
\pgfpathlineto{\pgfqpoint{3.010310in}{2.003267in}}%
\pgfpathlineto{\pgfqpoint{3.011155in}{2.319648in}}%
\pgfpathlineto{\pgfqpoint{3.012001in}{2.282856in}}%
\pgfpathlineto{\pgfqpoint{3.012847in}{2.338951in}}%
\pgfpathlineto{\pgfqpoint{3.013692in}{2.823811in}}%
\pgfpathlineto{\pgfqpoint{3.015384in}{1.688230in}}%
\pgfpathlineto{\pgfqpoint{3.016229in}{2.484459in}}%
\pgfpathlineto{\pgfqpoint{3.017075in}{2.258844in}}%
\pgfpathlineto{\pgfqpoint{3.018766in}{2.447857in}}%
\pgfpathlineto{\pgfqpoint{3.020457in}{1.876844in}}%
\pgfpathlineto{\pgfqpoint{3.022149in}{2.729344in}}%
\pgfpathlineto{\pgfqpoint{3.023840in}{1.871486in}}%
\pgfpathlineto{\pgfqpoint{3.025531in}{2.648518in}}%
\pgfpathlineto{\pgfqpoint{3.027222in}{2.000245in}}%
\pgfpathlineto{\pgfqpoint{3.028068in}{2.003834in}}%
\pgfpathlineto{\pgfqpoint{3.028914in}{2.495507in}}%
\pgfpathlineto{\pgfqpoint{3.029759in}{1.655963in}}%
\pgfpathlineto{\pgfqpoint{3.030605in}{2.265704in}}%
\pgfpathlineto{\pgfqpoint{3.031450in}{2.629747in}}%
\pgfpathlineto{\pgfqpoint{3.032296in}{1.763442in}}%
\pgfpathlineto{\pgfqpoint{3.033142in}{2.641384in}}%
\pgfpathlineto{\pgfqpoint{3.033987in}{2.612238in}}%
\pgfpathlineto{\pgfqpoint{3.034833in}{2.063874in}}%
\pgfpathlineto{\pgfqpoint{3.037370in}{2.718640in}}%
\pgfpathlineto{\pgfqpoint{3.039061in}{1.741025in}}%
\pgfpathlineto{\pgfqpoint{3.039907in}{2.345841in}}%
\pgfpathlineto{\pgfqpoint{3.040752in}{2.009488in}}%
\pgfpathlineto{\pgfqpoint{3.041598in}{1.832915in}}%
\pgfpathlineto{\pgfqpoint{3.045826in}{2.521469in}}%
\pgfpathlineto{\pgfqpoint{3.046672in}{1.982716in}}%
\pgfpathlineto{\pgfqpoint{3.047517in}{2.083923in}}%
\pgfpathlineto{\pgfqpoint{3.050054in}{2.510829in}}%
\pgfpathlineto{\pgfqpoint{3.050900in}{2.465451in}}%
\pgfpathlineto{\pgfqpoint{3.051745in}{1.934563in}}%
\pgfpathlineto{\pgfqpoint{3.052591in}{2.308974in}}%
\pgfpathlineto{\pgfqpoint{3.054282in}{1.955329in}}%
\pgfpathlineto{\pgfqpoint{3.055973in}{2.239156in}}%
\pgfpathlineto{\pgfqpoint{3.056819in}{2.154789in}}%
\pgfpathlineto{\pgfqpoint{3.057665in}{2.419418in}}%
\pgfpathlineto{\pgfqpoint{3.058510in}{2.227084in}}%
\pgfpathlineto{\pgfqpoint{3.060202in}{1.740149in}}%
\pgfpathlineto{\pgfqpoint{3.062738in}{2.528393in}}%
\pgfpathlineto{\pgfqpoint{3.063584in}{1.767335in}}%
\pgfpathlineto{\pgfqpoint{3.064430in}{2.175025in}}%
\pgfpathlineto{\pgfqpoint{3.065275in}{2.245760in}}%
\pgfpathlineto{\pgfqpoint{3.066121in}{1.726416in}}%
\pgfpathlineto{\pgfqpoint{3.066967in}{2.543395in}}%
\pgfpathlineto{\pgfqpoint{3.067812in}{2.375593in}}%
\pgfpathlineto{\pgfqpoint{3.068658in}{1.827347in}}%
\pgfpathlineto{\pgfqpoint{3.069503in}{2.807639in}}%
\pgfpathlineto{\pgfqpoint{3.070349in}{1.775440in}}%
\pgfpathlineto{\pgfqpoint{3.071195in}{2.001445in}}%
\pgfpathlineto{\pgfqpoint{3.072040in}{2.040084in}}%
\pgfpathlineto{\pgfqpoint{3.072886in}{2.812021in}}%
\pgfpathlineto{\pgfqpoint{3.073732in}{2.432987in}}%
\pgfpathlineto{\pgfqpoint{3.076268in}{1.936832in}}%
\pgfpathlineto{\pgfqpoint{3.077960in}{2.244005in}}%
\pgfpathlineto{\pgfqpoint{3.078805in}{1.820200in}}%
\pgfpathlineto{\pgfqpoint{3.079651in}{2.307276in}}%
\pgfpathlineto{\pgfqpoint{3.080497in}{2.088402in}}%
\pgfpathlineto{\pgfqpoint{3.081342in}{1.785218in}}%
\pgfpathlineto{\pgfqpoint{3.083879in}{2.409324in}}%
\pgfpathlineto{\pgfqpoint{3.084725in}{2.213689in}}%
\pgfpathlineto{\pgfqpoint{3.085570in}{1.563817in}}%
\pgfpathlineto{\pgfqpoint{3.086416in}{1.928540in}}%
\pgfpathlineto{\pgfqpoint{3.088107in}{2.524787in}}%
\pgfpathlineto{\pgfqpoint{3.088953in}{1.675717in}}%
\pgfpathlineto{\pgfqpoint{3.090644in}{2.628222in}}%
\pgfpathlineto{\pgfqpoint{3.091490in}{1.612474in}}%
\pgfpathlineto{\pgfqpoint{3.092335in}{2.335183in}}%
\pgfpathlineto{\pgfqpoint{3.093181in}{1.606592in}}%
\pgfpathlineto{\pgfqpoint{3.094027in}{1.880768in}}%
\pgfpathlineto{\pgfqpoint{3.094872in}{1.965189in}}%
\pgfpathlineto{\pgfqpoint{3.095718in}{2.599418in}}%
\pgfpathlineto{\pgfqpoint{3.096563in}{1.942822in}}%
\pgfpathlineto{\pgfqpoint{3.097409in}{2.803030in}}%
\pgfpathlineto{\pgfqpoint{3.098255in}{2.036259in}}%
\pgfpathlineto{\pgfqpoint{3.099100in}{2.304414in}}%
\pgfpathlineto{\pgfqpoint{3.100792in}{1.895151in}}%
\pgfpathlineto{\pgfqpoint{3.101637in}{2.010996in}}%
\pgfpathlineto{\pgfqpoint{3.102483in}{1.972362in}}%
\pgfpathlineto{\pgfqpoint{3.103328in}{1.703736in}}%
\pgfpathlineto{\pgfqpoint{3.105020in}{2.722431in}}%
\pgfpathlineto{\pgfqpoint{3.105865in}{2.642177in}}%
\pgfpathlineto{\pgfqpoint{3.106711in}{1.996774in}}%
\pgfpathlineto{\pgfqpoint{3.107557in}{2.076198in}}%
\pgfpathlineto{\pgfqpoint{3.108402in}{2.169972in}}%
\pgfpathlineto{\pgfqpoint{3.109248in}{1.818158in}}%
\pgfpathlineto{\pgfqpoint{3.111785in}{2.218920in}}%
\pgfpathlineto{\pgfqpoint{3.113476in}{2.494672in}}%
\pgfpathlineto{\pgfqpoint{3.116013in}{1.913399in}}%
\pgfpathlineto{\pgfqpoint{3.116858in}{2.739688in}}%
\pgfpathlineto{\pgfqpoint{3.118550in}{1.988273in}}%
\pgfpathlineto{\pgfqpoint{3.120241in}{1.930952in}}%
\pgfpathlineto{\pgfqpoint{3.121086in}{1.947908in}}%
\pgfpathlineto{\pgfqpoint{3.121932in}{2.615136in}}%
\pgfpathlineto{\pgfqpoint{3.122778in}{2.154613in}}%
\pgfpathlineto{\pgfqpoint{3.123623in}{2.660431in}}%
\pgfpathlineto{\pgfqpoint{3.124469in}{1.715298in}}%
\pgfpathlineto{\pgfqpoint{3.125315in}{2.856963in}}%
\pgfpathlineto{\pgfqpoint{3.126160in}{2.488316in}}%
\pgfpathlineto{\pgfqpoint{3.127006in}{2.464065in}}%
\pgfpathlineto{\pgfqpoint{3.127851in}{2.118242in}}%
\pgfpathlineto{\pgfqpoint{3.128697in}{2.720659in}}%
\pgfpathlineto{\pgfqpoint{3.129543in}{1.979565in}}%
\pgfpathlineto{\pgfqpoint{3.130388in}{2.229217in}}%
\pgfpathlineto{\pgfqpoint{3.132080in}{1.931763in}}%
\pgfpathlineto{\pgfqpoint{3.132925in}{2.841623in}}%
\pgfpathlineto{\pgfqpoint{3.133771in}{1.886032in}}%
\pgfpathlineto{\pgfqpoint{3.134616in}{1.965741in}}%
\pgfpathlineto{\pgfqpoint{3.135462in}{2.369175in}}%
\pgfpathlineto{\pgfqpoint{3.136308in}{2.135123in}}%
\pgfpathlineto{\pgfqpoint{3.137153in}{2.201258in}}%
\pgfpathlineto{\pgfqpoint{3.137999in}{1.338780in}}%
\pgfpathlineto{\pgfqpoint{3.139690in}{2.757446in}}%
\pgfpathlineto{\pgfqpoint{3.140536in}{2.466732in}}%
\pgfpathlineto{\pgfqpoint{3.141381in}{2.630697in}}%
\pgfpathlineto{\pgfqpoint{3.143918in}{1.944640in}}%
\pgfpathlineto{\pgfqpoint{3.144764in}{2.721397in}}%
\pgfpathlineto{\pgfqpoint{3.145610in}{2.294276in}}%
\pgfpathlineto{\pgfqpoint{3.147301in}{2.671182in}}%
\pgfpathlineto{\pgfqpoint{3.148146in}{1.914920in}}%
\pgfpathlineto{\pgfqpoint{3.148992in}{2.024509in}}%
\pgfpathlineto{\pgfqpoint{3.149838in}{2.540035in}}%
\pgfpathlineto{\pgfqpoint{3.150683in}{1.805983in}}%
\pgfpathlineto{\pgfqpoint{3.151529in}{2.165131in}}%
\pgfpathlineto{\pgfqpoint{3.153220in}{2.601778in}}%
\pgfpathlineto{\pgfqpoint{3.154066in}{1.933004in}}%
\pgfpathlineto{\pgfqpoint{3.154911in}{2.497449in}}%
\pgfpathlineto{\pgfqpoint{3.155757in}{1.859121in}}%
\pgfpathlineto{\pgfqpoint{3.156603in}{2.132820in}}%
\pgfpathlineto{\pgfqpoint{3.158294in}{2.445964in}}%
\pgfpathlineto{\pgfqpoint{3.159140in}{2.421363in}}%
\pgfpathlineto{\pgfqpoint{3.159985in}{1.963993in}}%
\pgfpathlineto{\pgfqpoint{3.160831in}{2.446374in}}%
\pgfpathlineto{\pgfqpoint{3.161676in}{2.147278in}}%
\pgfpathlineto{\pgfqpoint{3.162522in}{1.992486in}}%
\pgfpathlineto{\pgfqpoint{3.163368in}{2.003320in}}%
\pgfpathlineto{\pgfqpoint{3.164213in}{2.514627in}}%
\pgfpathlineto{\pgfqpoint{3.165059in}{1.821276in}}%
\pgfpathlineto{\pgfqpoint{3.165905in}{2.242550in}}%
\pgfpathlineto{\pgfqpoint{3.167596in}{2.299018in}}%
\pgfpathlineto{\pgfqpoint{3.168441in}{2.160222in}}%
\pgfpathlineto{\pgfqpoint{3.170133in}{2.626784in}}%
\pgfpathlineto{\pgfqpoint{3.171824in}{1.919590in}}%
\pgfpathlineto{\pgfqpoint{3.172670in}{2.415644in}}%
\pgfpathlineto{\pgfqpoint{3.173515in}{1.988776in}}%
\pgfpathlineto{\pgfqpoint{3.174361in}{2.157315in}}%
\pgfpathlineto{\pgfqpoint{3.175206in}{2.631431in}}%
\pgfpathlineto{\pgfqpoint{3.176898in}{1.784079in}}%
\pgfpathlineto{\pgfqpoint{3.177743in}{2.513339in}}%
\pgfpathlineto{\pgfqpoint{3.178589in}{2.127550in}}%
\pgfpathlineto{\pgfqpoint{3.179435in}{2.431371in}}%
\pgfpathlineto{\pgfqpoint{3.181126in}{1.904006in}}%
\pgfpathlineto{\pgfqpoint{3.182817in}{2.767919in}}%
\pgfpathlineto{\pgfqpoint{3.184508in}{1.829455in}}%
\pgfpathlineto{\pgfqpoint{3.186200in}{2.517517in}}%
\pgfpathlineto{\pgfqpoint{3.187045in}{1.746526in}}%
\pgfpathlineto{\pgfqpoint{3.187891in}{2.444373in}}%
\pgfpathlineto{\pgfqpoint{3.188736in}{1.940626in}}%
\pgfpathlineto{\pgfqpoint{3.189582in}{1.911165in}}%
\pgfpathlineto{\pgfqpoint{3.192119in}{2.759019in}}%
\pgfpathlineto{\pgfqpoint{3.193810in}{1.863600in}}%
\pgfpathlineto{\pgfqpoint{3.194656in}{1.918562in}}%
\pgfpathlineto{\pgfqpoint{3.195501in}{2.466969in}}%
\pgfpathlineto{\pgfqpoint{3.196347in}{2.191102in}}%
\pgfpathlineto{\pgfqpoint{3.197193in}{2.028577in}}%
\pgfpathlineto{\pgfqpoint{3.198038in}{2.712846in}}%
\pgfpathlineto{\pgfqpoint{3.198884in}{2.603769in}}%
\pgfpathlineto{\pgfqpoint{3.200575in}{2.180010in}}%
\pgfpathlineto{\pgfqpoint{3.201421in}{2.511658in}}%
\pgfpathlineto{\pgfqpoint{3.202266in}{2.358183in}}%
\pgfpathlineto{\pgfqpoint{3.203112in}{2.000517in}}%
\pgfpathlineto{\pgfqpoint{3.205649in}{2.742138in}}%
\pgfpathlineto{\pgfqpoint{3.207340in}{2.216741in}}%
\pgfpathlineto{\pgfqpoint{3.208186in}{2.713907in}}%
\pgfpathlineto{\pgfqpoint{3.209031in}{1.547692in}}%
\pgfpathlineto{\pgfqpoint{3.209877in}{2.138468in}}%
\pgfpathlineto{\pgfqpoint{3.210723in}{2.299297in}}%
\pgfpathlineto{\pgfqpoint{3.211568in}{2.168470in}}%
\pgfpathlineto{\pgfqpoint{3.212414in}{2.205193in}}%
\pgfpathlineto{\pgfqpoint{3.214105in}{1.463376in}}%
\pgfpathlineto{\pgfqpoint{3.215796in}{2.315794in}}%
\pgfpathlineto{\pgfqpoint{3.217488in}{1.870738in}}%
\pgfpathlineto{\pgfqpoint{3.218333in}{1.903752in}}%
\pgfpathlineto{\pgfqpoint{3.219179in}{1.922433in}}%
\pgfpathlineto{\pgfqpoint{3.220870in}{2.363096in}}%
\pgfpathlineto{\pgfqpoint{3.221716in}{1.871439in}}%
\pgfpathlineto{\pgfqpoint{3.222561in}{2.502489in}}%
\pgfpathlineto{\pgfqpoint{3.223407in}{1.955927in}}%
\pgfpathlineto{\pgfqpoint{3.224253in}{2.341255in}}%
\pgfpathlineto{\pgfqpoint{3.225098in}{2.179058in}}%
\pgfpathlineto{\pgfqpoint{3.225944in}{2.216034in}}%
\pgfpathlineto{\pgfqpoint{3.226789in}{2.307336in}}%
\pgfpathlineto{\pgfqpoint{3.227635in}{2.191248in}}%
\pgfpathlineto{\pgfqpoint{3.228481in}{2.221798in}}%
\pgfpathlineto{\pgfqpoint{3.229326in}{2.378120in}}%
\pgfpathlineto{\pgfqpoint{3.230172in}{1.967771in}}%
\pgfpathlineto{\pgfqpoint{3.231018in}{2.003832in}}%
\pgfpathlineto{\pgfqpoint{3.231863in}{2.720940in}}%
\pgfpathlineto{\pgfqpoint{3.232709in}{1.650367in}}%
\pgfpathlineto{\pgfqpoint{3.233554in}{2.068700in}}%
\pgfpathlineto{\pgfqpoint{3.234400in}{2.281221in}}%
\pgfpathlineto{\pgfqpoint{3.235246in}{1.736779in}}%
\pgfpathlineto{\pgfqpoint{3.236091in}{2.035371in}}%
\pgfpathlineto{\pgfqpoint{3.236937in}{2.340762in}}%
\pgfpathlineto{\pgfqpoint{3.237783in}{1.984869in}}%
\pgfpathlineto{\pgfqpoint{3.238628in}{2.036598in}}%
\pgfpathlineto{\pgfqpoint{3.239474in}{1.958878in}}%
\pgfpathlineto{\pgfqpoint{3.240319in}{2.274140in}}%
\pgfpathlineto{\pgfqpoint{3.241165in}{1.766999in}}%
\pgfpathlineto{\pgfqpoint{3.242856in}{2.412471in}}%
\pgfpathlineto{\pgfqpoint{3.243702in}{1.964661in}}%
\pgfpathlineto{\pgfqpoint{3.244548in}{2.042602in}}%
\pgfpathlineto{\pgfqpoint{3.245393in}{2.566749in}}%
\pgfpathlineto{\pgfqpoint{3.246239in}{2.333012in}}%
\pgfpathlineto{\pgfqpoint{3.247930in}{1.999542in}}%
\pgfpathlineto{\pgfqpoint{3.248776in}{2.500615in}}%
\pgfpathlineto{\pgfqpoint{3.249621in}{2.368106in}}%
\pgfpathlineto{\pgfqpoint{3.251313in}{1.944266in}}%
\pgfpathlineto{\pgfqpoint{3.252158in}{2.454201in}}%
\pgfpathlineto{\pgfqpoint{3.253004in}{2.109843in}}%
\pgfpathlineto{\pgfqpoint{3.253849in}{2.140502in}}%
\pgfpathlineto{\pgfqpoint{3.254695in}{2.243113in}}%
\pgfpathlineto{\pgfqpoint{3.255541in}{2.665460in}}%
\pgfpathlineto{\pgfqpoint{3.258078in}{1.897680in}}%
\pgfpathlineto{\pgfqpoint{3.258923in}{2.689659in}}%
\pgfpathlineto{\pgfqpoint{3.259769in}{2.156155in}}%
\pgfpathlineto{\pgfqpoint{3.260614in}{1.924467in}}%
\pgfpathlineto{\pgfqpoint{3.261460in}{2.531299in}}%
\pgfpathlineto{\pgfqpoint{3.262306in}{1.571622in}}%
\pgfpathlineto{\pgfqpoint{3.263997in}{2.515785in}}%
\pgfpathlineto{\pgfqpoint{3.264843in}{2.366874in}}%
\pgfpathlineto{\pgfqpoint{3.266534in}{1.741984in}}%
\pgfpathlineto{\pgfqpoint{3.268225in}{2.234948in}}%
\pgfpathlineto{\pgfqpoint{3.269071in}{1.720683in}}%
\pgfpathlineto{\pgfqpoint{3.271608in}{2.577561in}}%
\pgfpathlineto{\pgfqpoint{3.272453in}{1.959646in}}%
\pgfpathlineto{\pgfqpoint{3.273299in}{2.642990in}}%
\pgfpathlineto{\pgfqpoint{3.274144in}{2.259930in}}%
\pgfpathlineto{\pgfqpoint{3.274990in}{1.615390in}}%
\pgfpathlineto{\pgfqpoint{3.275836in}{2.105408in}}%
\pgfpathlineto{\pgfqpoint{3.276681in}{2.726655in}}%
\pgfpathlineto{\pgfqpoint{3.277527in}{2.012688in}}%
\pgfpathlineto{\pgfqpoint{3.278372in}{2.135713in}}%
\pgfpathlineto{\pgfqpoint{3.279218in}{1.940436in}}%
\pgfpathlineto{\pgfqpoint{3.280064in}{2.355436in}}%
\pgfpathlineto{\pgfqpoint{3.280909in}{2.072559in}}%
\pgfpathlineto{\pgfqpoint{3.281755in}{2.158502in}}%
\pgfpathlineto{\pgfqpoint{3.282601in}{2.628884in}}%
\pgfpathlineto{\pgfqpoint{3.283446in}{2.302211in}}%
\pgfpathlineto{\pgfqpoint{3.284292in}{2.459043in}}%
\pgfpathlineto{\pgfqpoint{3.285137in}{1.440540in}}%
\pgfpathlineto{\pgfqpoint{3.285983in}{2.355465in}}%
\pgfpathlineto{\pgfqpoint{3.286829in}{1.875734in}}%
\pgfpathlineto{\pgfqpoint{3.287674in}{2.466777in}}%
\pgfpathlineto{\pgfqpoint{3.288520in}{2.172665in}}%
\pgfpathlineto{\pgfqpoint{3.289366in}{2.031904in}}%
\pgfpathlineto{\pgfqpoint{3.290211in}{2.807089in}}%
\pgfpathlineto{\pgfqpoint{3.291057in}{2.508192in}}%
\pgfpathlineto{\pgfqpoint{3.291902in}{1.838212in}}%
\pgfpathlineto{\pgfqpoint{3.292748in}{1.840079in}}%
\pgfpathlineto{\pgfqpoint{3.293594in}{2.207211in}}%
\pgfpathlineto{\pgfqpoint{3.294439in}{2.070332in}}%
\pgfpathlineto{\pgfqpoint{3.295285in}{1.758501in}}%
\pgfpathlineto{\pgfqpoint{3.296976in}{2.740800in}}%
\pgfpathlineto{\pgfqpoint{3.297822in}{2.275906in}}%
\pgfpathlineto{\pgfqpoint{3.298667in}{2.318128in}}%
\pgfpathlineto{\pgfqpoint{3.299513in}{1.959140in}}%
\pgfpathlineto{\pgfqpoint{3.302050in}{2.766410in}}%
\pgfpathlineto{\pgfqpoint{3.303741in}{1.753017in}}%
\pgfpathlineto{\pgfqpoint{3.304587in}{2.404287in}}%
\pgfpathlineto{\pgfqpoint{3.305432in}{1.995111in}}%
\pgfpathlineto{\pgfqpoint{3.306278in}{2.295224in}}%
\pgfpathlineto{\pgfqpoint{3.307124in}{1.947016in}}%
\pgfpathlineto{\pgfqpoint{3.307969in}{2.177563in}}%
\pgfpathlineto{\pgfqpoint{3.308815in}{2.060352in}}%
\pgfpathlineto{\pgfqpoint{3.310506in}{3.014861in}}%
\pgfpathlineto{\pgfqpoint{3.312197in}{1.825476in}}%
\pgfpathlineto{\pgfqpoint{3.314734in}{2.326305in}}%
\pgfpathlineto{\pgfqpoint{3.316426in}{2.670991in}}%
\pgfpathlineto{\pgfqpoint{3.317271in}{1.673271in}}%
\pgfpathlineto{\pgfqpoint{3.318117in}{2.451380in}}%
\pgfpathlineto{\pgfqpoint{3.318962in}{1.927617in}}%
\pgfpathlineto{\pgfqpoint{3.319808in}{2.397552in}}%
\pgfpathlineto{\pgfqpoint{3.320654in}{2.332298in}}%
\pgfpathlineto{\pgfqpoint{3.322345in}{2.027135in}}%
\pgfpathlineto{\pgfqpoint{3.323191in}{2.208250in}}%
\pgfpathlineto{\pgfqpoint{3.324036in}{2.803938in}}%
\pgfpathlineto{\pgfqpoint{3.325727in}{1.779349in}}%
\pgfpathlineto{\pgfqpoint{3.326573in}{2.318261in}}%
\pgfpathlineto{\pgfqpoint{3.327419in}{1.855423in}}%
\pgfpathlineto{\pgfqpoint{3.328264in}{2.017380in}}%
\pgfpathlineto{\pgfqpoint{3.329956in}{2.023758in}}%
\pgfpathlineto{\pgfqpoint{3.331647in}{2.228242in}}%
\pgfpathlineto{\pgfqpoint{3.332492in}{1.830438in}}%
\pgfpathlineto{\pgfqpoint{3.334184in}{2.491353in}}%
\pgfpathlineto{\pgfqpoint{3.335875in}{1.932706in}}%
\pgfpathlineto{\pgfqpoint{3.336721in}{2.842938in}}%
\pgfpathlineto{\pgfqpoint{3.337566in}{2.297087in}}%
\pgfpathlineto{\pgfqpoint{3.338412in}{1.923503in}}%
\pgfpathlineto{\pgfqpoint{3.339257in}{2.376982in}}%
\pgfpathlineto{\pgfqpoint{3.340103in}{2.308611in}}%
\pgfpathlineto{\pgfqpoint{3.340949in}{2.162146in}}%
\pgfpathlineto{\pgfqpoint{3.341794in}{1.691924in}}%
\pgfpathlineto{\pgfqpoint{3.344331in}{2.498912in}}%
\pgfpathlineto{\pgfqpoint{3.345177in}{1.936522in}}%
\pgfpathlineto{\pgfqpoint{3.346022in}{2.199330in}}%
\pgfpathlineto{\pgfqpoint{3.346868in}{2.106619in}}%
\pgfpathlineto{\pgfqpoint{3.347714in}{2.540103in}}%
\pgfpathlineto{\pgfqpoint{3.348559in}{2.044270in}}%
\pgfpathlineto{\pgfqpoint{3.349405in}{2.471704in}}%
\pgfpathlineto{\pgfqpoint{3.350251in}{2.249499in}}%
\pgfpathlineto{\pgfqpoint{3.351096in}{2.168125in}}%
\pgfpathlineto{\pgfqpoint{3.351942in}{1.617971in}}%
\pgfpathlineto{\pgfqpoint{3.354479in}{2.711720in}}%
\pgfpathlineto{\pgfqpoint{3.356170in}{1.858866in}}%
\pgfpathlineto{\pgfqpoint{3.357861in}{2.317775in}}%
\pgfpathlineto{\pgfqpoint{3.358707in}{2.203861in}}%
\pgfpathlineto{\pgfqpoint{3.359552in}{2.318176in}}%
\pgfpathlineto{\pgfqpoint{3.360398in}{1.728054in}}%
\pgfpathlineto{\pgfqpoint{3.361244in}{2.121188in}}%
\pgfpathlineto{\pgfqpoint{3.362089in}{2.112129in}}%
\pgfpathlineto{\pgfqpoint{3.362935in}{1.875775in}}%
\pgfpathlineto{\pgfqpoint{3.363780in}{2.070065in}}%
\pgfpathlineto{\pgfqpoint{3.364626in}{1.930314in}}%
\pgfpathlineto{\pgfqpoint{3.365472in}{2.591381in}}%
\pgfpathlineto{\pgfqpoint{3.366317in}{1.940653in}}%
\pgfpathlineto{\pgfqpoint{3.367163in}{2.109546in}}%
\pgfpathlineto{\pgfqpoint{3.368009in}{2.248389in}}%
\pgfpathlineto{\pgfqpoint{3.369700in}{1.883833in}}%
\pgfpathlineto{\pgfqpoint{3.370545in}{2.326153in}}%
\pgfpathlineto{\pgfqpoint{3.371391in}{1.943035in}}%
\pgfpathlineto{\pgfqpoint{3.373082in}{2.417601in}}%
\pgfpathlineto{\pgfqpoint{3.373928in}{2.417422in}}%
\pgfpathlineto{\pgfqpoint{3.374774in}{2.063708in}}%
\pgfpathlineto{\pgfqpoint{3.375619in}{2.331016in}}%
\pgfpathlineto{\pgfqpoint{3.378156in}{1.976456in}}%
\pgfpathlineto{\pgfqpoint{3.379002in}{1.930811in}}%
\pgfpathlineto{\pgfqpoint{3.379847in}{2.270794in}}%
\pgfpathlineto{\pgfqpoint{3.380693in}{1.868234in}}%
\pgfpathlineto{\pgfqpoint{3.381539in}{2.135981in}}%
\pgfpathlineto{\pgfqpoint{3.383230in}{1.864059in}}%
\pgfpathlineto{\pgfqpoint{3.384921in}{2.488579in}}%
\pgfpathlineto{\pgfqpoint{3.388304in}{1.906459in}}%
\pgfpathlineto{\pgfqpoint{3.389149in}{2.252890in}}%
\pgfpathlineto{\pgfqpoint{3.389995in}{2.001364in}}%
\pgfpathlineto{\pgfqpoint{3.390840in}{1.416587in}}%
\pgfpathlineto{\pgfqpoint{3.392532in}{2.232318in}}%
\pgfpathlineto{\pgfqpoint{3.393377in}{2.206279in}}%
\pgfpathlineto{\pgfqpoint{3.394223in}{1.904760in}}%
\pgfpathlineto{\pgfqpoint{3.395069in}{2.181541in}}%
\pgfpathlineto{\pgfqpoint{3.395914in}{1.666895in}}%
\pgfpathlineto{\pgfqpoint{3.397605in}{2.814530in}}%
\pgfpathlineto{\pgfqpoint{3.398451in}{1.502614in}}%
\pgfpathlineto{\pgfqpoint{3.399297in}{2.542456in}}%
\pgfpathlineto{\pgfqpoint{3.400142in}{2.010237in}}%
\pgfpathlineto{\pgfqpoint{3.400988in}{2.297220in}}%
\pgfpathlineto{\pgfqpoint{3.401834in}{1.938385in}}%
\pgfpathlineto{\pgfqpoint{3.402679in}{2.656173in}}%
\pgfpathlineto{\pgfqpoint{3.403525in}{1.843052in}}%
\pgfpathlineto{\pgfqpoint{3.404370in}{2.331745in}}%
\pgfpathlineto{\pgfqpoint{3.405216in}{2.184970in}}%
\pgfpathlineto{\pgfqpoint{3.406062in}{2.696444in}}%
\pgfpathlineto{\pgfqpoint{3.408599in}{1.308917in}}%
\pgfpathlineto{\pgfqpoint{3.409444in}{1.977999in}}%
\pgfpathlineto{\pgfqpoint{3.410290in}{1.964191in}}%
\pgfpathlineto{\pgfqpoint{3.411135in}{2.013687in}}%
\pgfpathlineto{\pgfqpoint{3.412827in}{2.419940in}}%
\pgfpathlineto{\pgfqpoint{3.415364in}{2.187770in}}%
\pgfpathlineto{\pgfqpoint{3.416209in}{2.655138in}}%
\pgfpathlineto{\pgfqpoint{3.417055in}{1.954837in}}%
\pgfpathlineto{\pgfqpoint{3.417900in}{2.775953in}}%
\pgfpathlineto{\pgfqpoint{3.418746in}{1.893026in}}%
\pgfpathlineto{\pgfqpoint{3.419592in}{2.570382in}}%
\pgfpathlineto{\pgfqpoint{3.422129in}{1.852125in}}%
\pgfpathlineto{\pgfqpoint{3.422974in}{2.729705in}}%
\pgfpathlineto{\pgfqpoint{3.423820in}{2.026784in}}%
\pgfpathlineto{\pgfqpoint{3.424665in}{2.504534in}}%
\pgfpathlineto{\pgfqpoint{3.425511in}{2.304161in}}%
\pgfpathlineto{\pgfqpoint{3.427202in}{1.656827in}}%
\pgfpathlineto{\pgfqpoint{3.428048in}{2.410449in}}%
\pgfpathlineto{\pgfqpoint{3.428894in}{2.183948in}}%
\pgfpathlineto{\pgfqpoint{3.429739in}{1.982695in}}%
\pgfpathlineto{\pgfqpoint{3.430585in}{2.369336in}}%
\pgfpathlineto{\pgfqpoint{3.431430in}{2.208640in}}%
\pgfpathlineto{\pgfqpoint{3.432276in}{2.244494in}}%
\pgfpathlineto{\pgfqpoint{3.433122in}{2.128554in}}%
\pgfpathlineto{\pgfqpoint{3.433967in}{1.497152in}}%
\pgfpathlineto{\pgfqpoint{3.434813in}{1.738785in}}%
\pgfpathlineto{\pgfqpoint{3.435658in}{2.537274in}}%
\pgfpathlineto{\pgfqpoint{3.436504in}{2.041255in}}%
\pgfpathlineto{\pgfqpoint{3.439041in}{3.042024in}}%
\pgfpathlineto{\pgfqpoint{3.441578in}{1.869217in}}%
\pgfpathlineto{\pgfqpoint{3.443269in}{2.212591in}}%
\pgfpathlineto{\pgfqpoint{3.444115in}{2.137488in}}%
\pgfpathlineto{\pgfqpoint{3.444960in}{2.203558in}}%
\pgfpathlineto{\pgfqpoint{3.446652in}{1.972992in}}%
\pgfpathlineto{\pgfqpoint{3.447497in}{2.640006in}}%
\pgfpathlineto{\pgfqpoint{3.448343in}{1.913335in}}%
\pgfpathlineto{\pgfqpoint{3.449188in}{2.337763in}}%
\pgfpathlineto{\pgfqpoint{3.450034in}{2.314641in}}%
\pgfpathlineto{\pgfqpoint{3.450880in}{2.516422in}}%
\pgfpathlineto{\pgfqpoint{3.451725in}{2.418181in}}%
\pgfpathlineto{\pgfqpoint{3.452571in}{2.219157in}}%
\pgfpathlineto{\pgfqpoint{3.453417in}{2.617798in}}%
\pgfpathlineto{\pgfqpoint{3.455953in}{1.699471in}}%
\pgfpathlineto{\pgfqpoint{3.456799in}{2.322521in}}%
\pgfpathlineto{\pgfqpoint{3.457645in}{2.191422in}}%
\pgfpathlineto{\pgfqpoint{3.458490in}{2.226835in}}%
\pgfpathlineto{\pgfqpoint{3.459336in}{2.390551in}}%
\pgfpathlineto{\pgfqpoint{3.461027in}{1.846409in}}%
\pgfpathlineto{\pgfqpoint{3.461873in}{2.450342in}}%
\pgfpathlineto{\pgfqpoint{3.462718in}{2.262138in}}%
\pgfpathlineto{\pgfqpoint{3.463564in}{2.461498in}}%
\pgfpathlineto{\pgfqpoint{3.464410in}{1.629290in}}%
\pgfpathlineto{\pgfqpoint{3.466947in}{2.739072in}}%
\pgfpathlineto{\pgfqpoint{3.467792in}{2.085192in}}%
\pgfpathlineto{\pgfqpoint{3.468638in}{2.660351in}}%
\pgfpathlineto{\pgfqpoint{3.470329in}{1.652667in}}%
\pgfpathlineto{\pgfqpoint{3.471175in}{2.625899in}}%
\pgfpathlineto{\pgfqpoint{3.472020in}{2.025103in}}%
\pgfpathlineto{\pgfqpoint{3.473712in}{2.676867in}}%
\pgfpathlineto{\pgfqpoint{3.474557in}{2.629287in}}%
\pgfpathlineto{\pgfqpoint{3.475403in}{2.486666in}}%
\pgfpathlineto{\pgfqpoint{3.476248in}{1.994022in}}%
\pgfpathlineto{\pgfqpoint{3.477094in}{2.929359in}}%
\pgfpathlineto{\pgfqpoint{3.477940in}{2.174301in}}%
\pgfpathlineto{\pgfqpoint{3.478785in}{1.858337in}}%
\pgfpathlineto{\pgfqpoint{3.479631in}{2.105955in}}%
\pgfpathlineto{\pgfqpoint{3.480477in}{2.019418in}}%
\pgfpathlineto{\pgfqpoint{3.481322in}{2.424572in}}%
\pgfpathlineto{\pgfqpoint{3.482168in}{2.212872in}}%
\pgfpathlineto{\pgfqpoint{3.483013in}{2.188749in}}%
\pgfpathlineto{\pgfqpoint{3.483859in}{2.053612in}}%
\pgfpathlineto{\pgfqpoint{3.484705in}{2.238032in}}%
\pgfpathlineto{\pgfqpoint{3.485550in}{1.952281in}}%
\pgfpathlineto{\pgfqpoint{3.486396in}{2.182811in}}%
\pgfpathlineto{\pgfqpoint{3.488087in}{2.536170in}}%
\pgfpathlineto{\pgfqpoint{3.489778in}{2.026894in}}%
\pgfpathlineto{\pgfqpoint{3.490624in}{2.610243in}}%
\pgfpathlineto{\pgfqpoint{3.491470in}{2.195216in}}%
\pgfpathlineto{\pgfqpoint{3.493161in}{2.688743in}}%
\pgfpathlineto{\pgfqpoint{3.495698in}{1.670774in}}%
\pgfpathlineto{\pgfqpoint{3.496543in}{1.521784in}}%
\pgfpathlineto{\pgfqpoint{3.497389in}{2.656722in}}%
\pgfpathlineto{\pgfqpoint{3.498235in}{2.092609in}}%
\pgfpathlineto{\pgfqpoint{3.499080in}{1.824890in}}%
\pgfpathlineto{\pgfqpoint{3.499926in}{2.458716in}}%
\pgfpathlineto{\pgfqpoint{3.500772in}{2.062370in}}%
\pgfpathlineto{\pgfqpoint{3.501617in}{1.925626in}}%
\pgfpathlineto{\pgfqpoint{3.502463in}{1.489194in}}%
\pgfpathlineto{\pgfqpoint{3.503308in}{2.304070in}}%
\pgfpathlineto{\pgfqpoint{3.504154in}{2.105115in}}%
\pgfpathlineto{\pgfqpoint{3.505845in}{1.824672in}}%
\pgfpathlineto{\pgfqpoint{3.506691in}{1.827398in}}%
\pgfpathlineto{\pgfqpoint{3.507537in}{1.724941in}}%
\pgfpathlineto{\pgfqpoint{3.510073in}{2.325763in}}%
\pgfpathlineto{\pgfqpoint{3.510919in}{2.537589in}}%
\pgfpathlineto{\pgfqpoint{3.511765in}{1.650097in}}%
\pgfpathlineto{\pgfqpoint{3.514301in}{2.692531in}}%
\pgfpathlineto{\pgfqpoint{3.516838in}{1.854497in}}%
\pgfpathlineto{\pgfqpoint{3.519375in}{2.535690in}}%
\pgfpathlineto{\pgfqpoint{3.520221in}{1.919055in}}%
\pgfpathlineto{\pgfqpoint{3.521066in}{2.122632in}}%
\pgfpathlineto{\pgfqpoint{3.521912in}{2.916851in}}%
\pgfpathlineto{\pgfqpoint{3.522758in}{2.290383in}}%
\pgfpathlineto{\pgfqpoint{3.523603in}{2.452800in}}%
\pgfpathlineto{\pgfqpoint{3.524449in}{1.995305in}}%
\pgfpathlineto{\pgfqpoint{3.525295in}{2.168814in}}%
\pgfpathlineto{\pgfqpoint{3.526986in}{2.067574in}}%
\pgfpathlineto{\pgfqpoint{3.527831in}{2.566558in}}%
\pgfpathlineto{\pgfqpoint{3.528677in}{2.447165in}}%
\pgfpathlineto{\pgfqpoint{3.529523in}{2.356952in}}%
\pgfpathlineto{\pgfqpoint{3.530368in}{1.766219in}}%
\pgfpathlineto{\pgfqpoint{3.531214in}{2.250255in}}%
\pgfpathlineto{\pgfqpoint{3.532060in}{2.416461in}}%
\pgfpathlineto{\pgfqpoint{3.532905in}{1.777671in}}%
\pgfpathlineto{\pgfqpoint{3.533751in}{2.645847in}}%
\pgfpathlineto{\pgfqpoint{3.534596in}{1.914379in}}%
\pgfpathlineto{\pgfqpoint{3.535442in}{2.331722in}}%
\pgfpathlineto{\pgfqpoint{3.536288in}{2.480717in}}%
\pgfpathlineto{\pgfqpoint{3.537133in}{2.059451in}}%
\pgfpathlineto{\pgfqpoint{3.537979in}{2.679597in}}%
\pgfpathlineto{\pgfqpoint{3.538825in}{2.668888in}}%
\pgfpathlineto{\pgfqpoint{3.540516in}{1.755877in}}%
\pgfpathlineto{\pgfqpoint{3.541361in}{1.841397in}}%
\pgfpathlineto{\pgfqpoint{3.542207in}{2.335080in}}%
\pgfpathlineto{\pgfqpoint{3.543053in}{2.248543in}}%
\pgfpathlineto{\pgfqpoint{3.543898in}{2.404470in}}%
\pgfpathlineto{\pgfqpoint{3.544744in}{1.560781in}}%
\pgfpathlineto{\pgfqpoint{3.547281in}{2.548557in}}%
\pgfpathlineto{\pgfqpoint{3.548972in}{1.700090in}}%
\pgfpathlineto{\pgfqpoint{3.550663in}{2.220797in}}%
\pgfpathlineto{\pgfqpoint{3.551509in}{2.076951in}}%
\pgfpathlineto{\pgfqpoint{3.552355in}{2.379196in}}%
\pgfpathlineto{\pgfqpoint{3.553200in}{1.890885in}}%
\pgfpathlineto{\pgfqpoint{3.554046in}{2.028365in}}%
\pgfpathlineto{\pgfqpoint{3.554891in}{2.528329in}}%
\pgfpathlineto{\pgfqpoint{3.555737in}{2.450948in}}%
\pgfpathlineto{\pgfqpoint{3.556583in}{2.514838in}}%
\pgfpathlineto{\pgfqpoint{3.557428in}{2.195791in}}%
\pgfpathlineto{\pgfqpoint{3.558274in}{2.666599in}}%
\pgfpathlineto{\pgfqpoint{3.559965in}{1.707160in}}%
\pgfpathlineto{\pgfqpoint{3.561656in}{2.139296in}}%
\pgfpathlineto{\pgfqpoint{3.562502in}{1.705401in}}%
\pgfpathlineto{\pgfqpoint{3.564193in}{2.249305in}}%
\pgfpathlineto{\pgfqpoint{3.565885in}{1.793325in}}%
\pgfpathlineto{\pgfqpoint{3.568421in}{2.679005in}}%
\pgfpathlineto{\pgfqpoint{3.569267in}{2.094048in}}%
\pgfpathlineto{\pgfqpoint{3.570113in}{2.690556in}}%
\pgfpathlineto{\pgfqpoint{3.570958in}{2.293485in}}%
\pgfpathlineto{\pgfqpoint{3.571804in}{1.597413in}}%
\pgfpathlineto{\pgfqpoint{3.572650in}{2.625850in}}%
\pgfpathlineto{\pgfqpoint{3.573495in}{2.234409in}}%
\pgfpathlineto{\pgfqpoint{3.574341in}{2.304280in}}%
\pgfpathlineto{\pgfqpoint{3.575186in}{2.812412in}}%
\pgfpathlineto{\pgfqpoint{3.576032in}{2.606710in}}%
\pgfpathlineto{\pgfqpoint{3.576878in}{1.629186in}}%
\pgfpathlineto{\pgfqpoint{3.577723in}{2.183667in}}%
\pgfpathlineto{\pgfqpoint{3.578569in}{2.688215in}}%
\pgfpathlineto{\pgfqpoint{3.579415in}{2.298585in}}%
\pgfpathlineto{\pgfqpoint{3.580260in}{1.990007in}}%
\pgfpathlineto{\pgfqpoint{3.581106in}{2.579583in}}%
\pgfpathlineto{\pgfqpoint{3.581951in}{1.626705in}}%
\pgfpathlineto{\pgfqpoint{3.582797in}{1.752654in}}%
\pgfpathlineto{\pgfqpoint{3.584488in}{2.544635in}}%
\pgfpathlineto{\pgfqpoint{3.586180in}{1.775943in}}%
\pgfpathlineto{\pgfqpoint{3.587871in}{2.888663in}}%
\pgfpathlineto{\pgfqpoint{3.589562in}{2.068088in}}%
\pgfpathlineto{\pgfqpoint{3.590408in}{2.137601in}}%
\pgfpathlineto{\pgfqpoint{3.591253in}{1.724682in}}%
\pgfpathlineto{\pgfqpoint{3.592099in}{2.799676in}}%
\pgfpathlineto{\pgfqpoint{3.592944in}{1.699022in}}%
\pgfpathlineto{\pgfqpoint{3.593790in}{2.282465in}}%
\pgfpathlineto{\pgfqpoint{3.594636in}{2.268080in}}%
\pgfpathlineto{\pgfqpoint{3.595481in}{2.202795in}}%
\pgfpathlineto{\pgfqpoint{3.596327in}{2.690748in}}%
\pgfpathlineto{\pgfqpoint{3.597173in}{2.069693in}}%
\pgfpathlineto{\pgfqpoint{3.598018in}{2.259481in}}%
\pgfpathlineto{\pgfqpoint{3.598864in}{1.991869in}}%
\pgfpathlineto{\pgfqpoint{3.599709in}{2.300311in}}%
\pgfpathlineto{\pgfqpoint{3.600555in}{1.990982in}}%
\pgfpathlineto{\pgfqpoint{3.601401in}{2.233224in}}%
\pgfpathlineto{\pgfqpoint{3.602246in}{1.940430in}}%
\pgfpathlineto{\pgfqpoint{3.603092in}{2.385670in}}%
\pgfpathlineto{\pgfqpoint{3.603938in}{1.750185in}}%
\pgfpathlineto{\pgfqpoint{3.604783in}{2.066161in}}%
\pgfpathlineto{\pgfqpoint{3.605629in}{2.379735in}}%
\pgfpathlineto{\pgfqpoint{3.606474in}{2.006438in}}%
\pgfpathlineto{\pgfqpoint{3.608166in}{2.520634in}}%
\pgfpathlineto{\pgfqpoint{3.609857in}{1.850433in}}%
\pgfpathlineto{\pgfqpoint{3.610703in}{2.137065in}}%
\pgfpathlineto{\pgfqpoint{3.611548in}{2.546650in}}%
\pgfpathlineto{\pgfqpoint{3.612394in}{2.497813in}}%
\pgfpathlineto{\pgfqpoint{3.615776in}{1.916960in}}%
\pgfpathlineto{\pgfqpoint{3.616622in}{2.293245in}}%
\pgfpathlineto{\pgfqpoint{3.617468in}{2.250127in}}%
\pgfpathlineto{\pgfqpoint{3.618313in}{1.577672in}}%
\pgfpathlineto{\pgfqpoint{3.619159in}{2.031270in}}%
\pgfpathlineto{\pgfqpoint{3.620004in}{2.151296in}}%
\pgfpathlineto{\pgfqpoint{3.620850in}{2.060004in}}%
\pgfpathlineto{\pgfqpoint{3.621696in}{2.610643in}}%
\pgfpathlineto{\pgfqpoint{3.622541in}{2.110014in}}%
\pgfpathlineto{\pgfqpoint{3.623387in}{2.150525in}}%
\pgfpathlineto{\pgfqpoint{3.626769in}{2.620376in}}%
\pgfpathlineto{\pgfqpoint{3.627615in}{1.778660in}}%
\pgfpathlineto{\pgfqpoint{3.628461in}{2.255879in}}%
\pgfpathlineto{\pgfqpoint{3.629306in}{2.254048in}}%
\pgfpathlineto{\pgfqpoint{3.630152in}{2.602724in}}%
\pgfpathlineto{\pgfqpoint{3.630998in}{2.237685in}}%
\pgfpathlineto{\pgfqpoint{3.631843in}{2.759730in}}%
\pgfpathlineto{\pgfqpoint{3.633534in}{2.191849in}}%
\pgfpathlineto{\pgfqpoint{3.635226in}{2.517362in}}%
\pgfpathlineto{\pgfqpoint{3.637763in}{1.640989in}}%
\pgfpathlineto{\pgfqpoint{3.638608in}{1.755120in}}%
\pgfpathlineto{\pgfqpoint{3.639454in}{1.740886in}}%
\pgfpathlineto{\pgfqpoint{3.640299in}{2.114753in}}%
\pgfpathlineto{\pgfqpoint{3.641145in}{2.033800in}}%
\pgfpathlineto{\pgfqpoint{3.641991in}{2.083027in}}%
\pgfpathlineto{\pgfqpoint{3.642836in}{2.621821in}}%
\pgfpathlineto{\pgfqpoint{3.643682in}{2.333761in}}%
\pgfpathlineto{\pgfqpoint{3.645373in}{2.190194in}}%
\pgfpathlineto{\pgfqpoint{3.646219in}{2.700650in}}%
\pgfpathlineto{\pgfqpoint{3.647064in}{2.678644in}}%
\pgfpathlineto{\pgfqpoint{3.647910in}{2.199780in}}%
\pgfpathlineto{\pgfqpoint{3.648756in}{2.918119in}}%
\pgfpathlineto{\pgfqpoint{3.649601in}{2.645258in}}%
\pgfpathlineto{\pgfqpoint{3.650447in}{2.431947in}}%
\pgfpathlineto{\pgfqpoint{3.651293in}{2.758219in}}%
\pgfpathlineto{\pgfqpoint{3.652984in}{1.563979in}}%
\pgfpathlineto{\pgfqpoint{3.653829in}{2.419477in}}%
\pgfpathlineto{\pgfqpoint{3.654675in}{2.091787in}}%
\pgfpathlineto{\pgfqpoint{3.655521in}{2.376773in}}%
\pgfpathlineto{\pgfqpoint{3.657212in}{1.917730in}}%
\pgfpathlineto{\pgfqpoint{3.659749in}{2.351622in}}%
\pgfpathlineto{\pgfqpoint{3.660594in}{1.845093in}}%
\pgfpathlineto{\pgfqpoint{3.661440in}{2.259988in}}%
\pgfpathlineto{\pgfqpoint{3.662286in}{1.970884in}}%
\pgfpathlineto{\pgfqpoint{3.663131in}{2.177035in}}%
\pgfpathlineto{\pgfqpoint{3.664823in}{2.505232in}}%
\pgfpathlineto{\pgfqpoint{3.666514in}{1.846058in}}%
\pgfpathlineto{\pgfqpoint{3.667359in}{2.814374in}}%
\pgfpathlineto{\pgfqpoint{3.668205in}{2.558055in}}%
\pgfpathlineto{\pgfqpoint{3.669051in}{2.004349in}}%
\pgfpathlineto{\pgfqpoint{3.669896in}{2.607892in}}%
\pgfpathlineto{\pgfqpoint{3.670742in}{2.234044in}}%
\pgfpathlineto{\pgfqpoint{3.671587in}{2.431264in}}%
\pgfpathlineto{\pgfqpoint{3.672433in}{2.370472in}}%
\pgfpathlineto{\pgfqpoint{3.673279in}{2.026545in}}%
\pgfpathlineto{\pgfqpoint{3.674124in}{2.230611in}}%
\pgfpathlineto{\pgfqpoint{3.674970in}{2.620702in}}%
\pgfpathlineto{\pgfqpoint{3.675816in}{1.957670in}}%
\pgfpathlineto{\pgfqpoint{3.676661in}{2.361068in}}%
\pgfpathlineto{\pgfqpoint{3.677507in}{2.209613in}}%
\pgfpathlineto{\pgfqpoint{3.678352in}{2.241322in}}%
\pgfpathlineto{\pgfqpoint{3.679198in}{2.267663in}}%
\pgfpathlineto{\pgfqpoint{3.680044in}{2.499071in}}%
\pgfpathlineto{\pgfqpoint{3.680889in}{1.896559in}}%
\pgfpathlineto{\pgfqpoint{3.681735in}{2.109843in}}%
\pgfpathlineto{\pgfqpoint{3.683426in}{2.276071in}}%
\pgfpathlineto{\pgfqpoint{3.684272in}{2.005245in}}%
\pgfpathlineto{\pgfqpoint{3.685117in}{2.190507in}}%
\pgfpathlineto{\pgfqpoint{3.686809in}{2.108612in}}%
\pgfpathlineto{\pgfqpoint{3.687654in}{1.997409in}}%
\pgfpathlineto{\pgfqpoint{3.688500in}{2.388443in}}%
\pgfpathlineto{\pgfqpoint{3.689346in}{1.741730in}}%
\pgfpathlineto{\pgfqpoint{3.690191in}{2.038517in}}%
\pgfpathlineto{\pgfqpoint{3.691882in}{3.051218in}}%
\pgfpathlineto{\pgfqpoint{3.692728in}{2.773830in}}%
\pgfpathlineto{\pgfqpoint{3.695265in}{1.651545in}}%
\pgfpathlineto{\pgfqpoint{3.696111in}{2.562751in}}%
\pgfpathlineto{\pgfqpoint{3.696956in}{1.839351in}}%
\pgfpathlineto{\pgfqpoint{3.699493in}{2.454511in}}%
\pgfpathlineto{\pgfqpoint{3.700339in}{2.104432in}}%
\pgfpathlineto{\pgfqpoint{3.701184in}{2.193986in}}%
\pgfpathlineto{\pgfqpoint{3.702876in}{1.839334in}}%
\pgfpathlineto{\pgfqpoint{3.703721in}{2.702781in}}%
\pgfpathlineto{\pgfqpoint{3.704567in}{2.049878in}}%
\pgfpathlineto{\pgfqpoint{3.705412in}{2.276851in}}%
\pgfpathlineto{\pgfqpoint{3.706258in}{1.888568in}}%
\pgfpathlineto{\pgfqpoint{3.708795in}{2.741099in}}%
\pgfpathlineto{\pgfqpoint{3.709641in}{2.874921in}}%
\pgfpathlineto{\pgfqpoint{3.711332in}{2.104455in}}%
\pgfpathlineto{\pgfqpoint{3.713023in}{2.509251in}}%
\pgfpathlineto{\pgfqpoint{3.715560in}{2.224387in}}%
\pgfpathlineto{\pgfqpoint{3.716406in}{2.123271in}}%
\pgfpathlineto{\pgfqpoint{3.717251in}{1.692944in}}%
\pgfpathlineto{\pgfqpoint{3.718097in}{2.418583in}}%
\pgfpathlineto{\pgfqpoint{3.718942in}{1.690626in}}%
\pgfpathlineto{\pgfqpoint{3.719788in}{2.239764in}}%
\pgfpathlineto{\pgfqpoint{3.721479in}{1.870460in}}%
\pgfpathlineto{\pgfqpoint{3.722325in}{2.562512in}}%
\pgfpathlineto{\pgfqpoint{3.723171in}{2.412047in}}%
\pgfpathlineto{\pgfqpoint{3.724016in}{2.243739in}}%
\pgfpathlineto{\pgfqpoint{3.725707in}{2.897842in}}%
\pgfpathlineto{\pgfqpoint{3.726553in}{2.862496in}}%
\pgfpathlineto{\pgfqpoint{3.727399in}{1.822257in}}%
\pgfpathlineto{\pgfqpoint{3.728244in}{2.475293in}}%
\pgfpathlineto{\pgfqpoint{3.729090in}{1.729324in}}%
\pgfpathlineto{\pgfqpoint{3.729936in}{2.035056in}}%
\pgfpathlineto{\pgfqpoint{3.730781in}{2.524125in}}%
\pgfpathlineto{\pgfqpoint{3.731627in}{2.343182in}}%
\pgfpathlineto{\pgfqpoint{3.732472in}{2.286948in}}%
\pgfpathlineto{\pgfqpoint{3.734164in}{2.613641in}}%
\pgfpathlineto{\pgfqpoint{3.735855in}{2.082659in}}%
\pgfpathlineto{\pgfqpoint{3.736701in}{2.154225in}}%
\pgfpathlineto{\pgfqpoint{3.737546in}{2.013124in}}%
\pgfpathlineto{\pgfqpoint{3.738392in}{2.500141in}}%
\pgfpathlineto{\pgfqpoint{3.739237in}{2.447423in}}%
\pgfpathlineto{\pgfqpoint{3.740083in}{2.374511in}}%
\pgfpathlineto{\pgfqpoint{3.741774in}{2.094990in}}%
\pgfpathlineto{\pgfqpoint{3.742620in}{2.166880in}}%
\pgfpathlineto{\pgfqpoint{3.744311in}{1.524632in}}%
\pgfpathlineto{\pgfqpoint{3.745157in}{2.305056in}}%
\pgfpathlineto{\pgfqpoint{3.746002in}{1.886015in}}%
\pgfpathlineto{\pgfqpoint{3.746848in}{1.759003in}}%
\pgfpathlineto{\pgfqpoint{3.749385in}{2.478022in}}%
\pgfpathlineto{\pgfqpoint{3.750231in}{1.779979in}}%
\pgfpathlineto{\pgfqpoint{3.751076in}{2.536471in}}%
\pgfpathlineto{\pgfqpoint{3.751922in}{1.997474in}}%
\pgfpathlineto{\pgfqpoint{3.752767in}{2.229599in}}%
\pgfpathlineto{\pgfqpoint{3.753613in}{2.038263in}}%
\pgfpathlineto{\pgfqpoint{3.754459in}{1.597268in}}%
\pgfpathlineto{\pgfqpoint{3.755304in}{1.915195in}}%
\pgfpathlineto{\pgfqpoint{3.758687in}{2.655830in}}%
\pgfpathlineto{\pgfqpoint{3.760378in}{2.128351in}}%
\pgfpathlineto{\pgfqpoint{3.761224in}{2.470338in}}%
\pgfpathlineto{\pgfqpoint{3.762069in}{1.412479in}}%
\pgfpathlineto{\pgfqpoint{3.764606in}{2.627051in}}%
\pgfpathlineto{\pgfqpoint{3.765452in}{2.845465in}}%
\pgfpathlineto{\pgfqpoint{3.766297in}{1.782081in}}%
\pgfpathlineto{\pgfqpoint{3.767143in}{2.148801in}}%
\pgfpathlineto{\pgfqpoint{3.767989in}{2.177735in}}%
\pgfpathlineto{\pgfqpoint{3.769680in}{2.321326in}}%
\pgfpathlineto{\pgfqpoint{3.770525in}{2.316486in}}%
\pgfpathlineto{\pgfqpoint{3.771371in}{1.712500in}}%
\pgfpathlineto{\pgfqpoint{3.772217in}{2.965492in}}%
\pgfpathlineto{\pgfqpoint{3.773062in}{2.143282in}}%
\pgfpathlineto{\pgfqpoint{3.773908in}{2.115554in}}%
\pgfpathlineto{\pgfqpoint{3.774754in}{2.388453in}}%
\pgfpathlineto{\pgfqpoint{3.775599in}{2.150854in}}%
\pgfpathlineto{\pgfqpoint{3.776445in}{2.275793in}}%
\pgfpathlineto{\pgfqpoint{3.777290in}{2.283997in}}%
\pgfpathlineto{\pgfqpoint{3.778136in}{2.009068in}}%
\pgfpathlineto{\pgfqpoint{3.778982in}{2.647857in}}%
\pgfpathlineto{\pgfqpoint{3.779827in}{2.169516in}}%
\pgfpathlineto{\pgfqpoint{3.781519in}{2.000497in}}%
\pgfpathlineto{\pgfqpoint{3.784055in}{2.500095in}}%
\pgfpathlineto{\pgfqpoint{3.786592in}{1.696794in}}%
\pgfpathlineto{\pgfqpoint{3.788284in}{2.553606in}}%
\pgfpathlineto{\pgfqpoint{3.789129in}{1.801674in}}%
\pgfpathlineto{\pgfqpoint{3.790820in}{2.699932in}}%
\pgfpathlineto{\pgfqpoint{3.791666in}{2.647824in}}%
\pgfpathlineto{\pgfqpoint{3.792512in}{2.496923in}}%
\pgfpathlineto{\pgfqpoint{3.793357in}{1.814443in}}%
\pgfpathlineto{\pgfqpoint{3.794203in}{1.942614in}}%
\pgfpathlineto{\pgfqpoint{3.795894in}{2.657599in}}%
\pgfpathlineto{\pgfqpoint{3.797585in}{1.936136in}}%
\pgfpathlineto{\pgfqpoint{3.799277in}{2.352375in}}%
\pgfpathlineto{\pgfqpoint{3.800122in}{2.321002in}}%
\pgfpathlineto{\pgfqpoint{3.800968in}{1.930158in}}%
\pgfpathlineto{\pgfqpoint{3.801814in}{2.238416in}}%
\pgfpathlineto{\pgfqpoint{3.802659in}{2.011680in}}%
\pgfpathlineto{\pgfqpoint{3.803505in}{2.712702in}}%
\pgfpathlineto{\pgfqpoint{3.804350in}{1.869217in}}%
\pgfpathlineto{\pgfqpoint{3.805196in}{1.904002in}}%
\pgfpathlineto{\pgfqpoint{3.806042in}{2.386803in}}%
\pgfpathlineto{\pgfqpoint{3.806887in}{2.332557in}}%
\pgfpathlineto{\pgfqpoint{3.807733in}{1.879522in}}%
\pgfpathlineto{\pgfqpoint{3.808579in}{2.518049in}}%
\pgfpathlineto{\pgfqpoint{3.809424in}{2.489431in}}%
\pgfpathlineto{\pgfqpoint{3.810270in}{1.683968in}}%
\pgfpathlineto{\pgfqpoint{3.812807in}{2.650212in}}%
\pgfpathlineto{\pgfqpoint{3.813652in}{2.168808in}}%
\pgfpathlineto{\pgfqpoint{3.814498in}{2.364617in}}%
\pgfpathlineto{\pgfqpoint{3.816189in}{2.132942in}}%
\pgfpathlineto{\pgfqpoint{3.817035in}{2.379206in}}%
\pgfpathlineto{\pgfqpoint{3.817880in}{2.184473in}}%
\pgfpathlineto{\pgfqpoint{3.818726in}{2.401717in}}%
\pgfpathlineto{\pgfqpoint{3.819572in}{2.381164in}}%
\pgfpathlineto{\pgfqpoint{3.820417in}{2.191589in}}%
\pgfpathlineto{\pgfqpoint{3.821263in}{2.362012in}}%
\pgfpathlineto{\pgfqpoint{3.822109in}{1.873737in}}%
\pgfpathlineto{\pgfqpoint{3.822954in}{2.092657in}}%
\pgfpathlineto{\pgfqpoint{3.823800in}{2.068894in}}%
\pgfpathlineto{\pgfqpoint{3.824645in}{1.955637in}}%
\pgfpathlineto{\pgfqpoint{3.826337in}{2.459419in}}%
\pgfpathlineto{\pgfqpoint{3.827182in}{2.455574in}}%
\pgfpathlineto{\pgfqpoint{3.828028in}{1.919014in}}%
\pgfpathlineto{\pgfqpoint{3.828874in}{2.235284in}}%
\pgfpathlineto{\pgfqpoint{3.829719in}{2.220299in}}%
\pgfpathlineto{\pgfqpoint{3.831410in}{1.770602in}}%
\pgfpathlineto{\pgfqpoint{3.832256in}{2.108183in}}%
\pgfpathlineto{\pgfqpoint{3.833102in}{1.975139in}}%
\pgfpathlineto{\pgfqpoint{3.833947in}{2.027253in}}%
\pgfpathlineto{\pgfqpoint{3.834793in}{2.942116in}}%
\pgfpathlineto{\pgfqpoint{3.835638in}{1.820960in}}%
\pgfpathlineto{\pgfqpoint{3.836484in}{2.166345in}}%
\pgfpathlineto{\pgfqpoint{3.837330in}{2.204518in}}%
\pgfpathlineto{\pgfqpoint{3.838175in}{1.907955in}}%
\pgfpathlineto{\pgfqpoint{3.839021in}{2.290653in}}%
\pgfpathlineto{\pgfqpoint{3.839867in}{1.722414in}}%
\pgfpathlineto{\pgfqpoint{3.840712in}{2.307167in}}%
\pgfpathlineto{\pgfqpoint{3.841558in}{2.015548in}}%
\pgfpathlineto{\pgfqpoint{3.842403in}{2.366826in}}%
\pgfpathlineto{\pgfqpoint{3.843249in}{2.106408in}}%
\pgfpathlineto{\pgfqpoint{3.844095in}{2.273110in}}%
\pgfpathlineto{\pgfqpoint{3.845786in}{3.095288in}}%
\pgfpathlineto{\pgfqpoint{3.848323in}{2.221630in}}%
\pgfpathlineto{\pgfqpoint{3.850014in}{1.902913in}}%
\pgfpathlineto{\pgfqpoint{3.852551in}{2.864220in}}%
\pgfpathlineto{\pgfqpoint{3.854242in}{2.278598in}}%
\pgfpathlineto{\pgfqpoint{3.855088in}{2.652585in}}%
\pgfpathlineto{\pgfqpoint{3.856779in}{1.854556in}}%
\pgfpathlineto{\pgfqpoint{3.857625in}{1.872832in}}%
\pgfpathlineto{\pgfqpoint{3.858470in}{1.895501in}}%
\pgfpathlineto{\pgfqpoint{3.859316in}{2.428741in}}%
\pgfpathlineto{\pgfqpoint{3.861007in}{1.692878in}}%
\pgfpathlineto{\pgfqpoint{3.861853in}{2.255361in}}%
\pgfpathlineto{\pgfqpoint{3.862698in}{1.576993in}}%
\pgfpathlineto{\pgfqpoint{3.863544in}{2.267638in}}%
\pgfpathlineto{\pgfqpoint{3.864390in}{1.888423in}}%
\pgfpathlineto{\pgfqpoint{3.866081in}{2.611529in}}%
\pgfpathlineto{\pgfqpoint{3.866927in}{1.899838in}}%
\pgfpathlineto{\pgfqpoint{3.867772in}{2.310094in}}%
\pgfpathlineto{\pgfqpoint{3.870309in}{1.789141in}}%
\pgfpathlineto{\pgfqpoint{3.872000in}{2.622429in}}%
\pgfpathlineto{\pgfqpoint{3.872846in}{3.119490in}}%
\pgfpathlineto{\pgfqpoint{3.875383in}{2.289654in}}%
\pgfpathlineto{\pgfqpoint{3.876228in}{2.283611in}}%
\pgfpathlineto{\pgfqpoint{3.877074in}{2.051390in}}%
\pgfpathlineto{\pgfqpoint{3.878765in}{2.840701in}}%
\pgfpathlineto{\pgfqpoint{3.880457in}{2.083091in}}%
\pgfpathlineto{\pgfqpoint{3.881302in}{2.951227in}}%
\pgfpathlineto{\pgfqpoint{3.882148in}{1.749851in}}%
\pgfpathlineto{\pgfqpoint{3.882993in}{2.349609in}}%
\pgfpathlineto{\pgfqpoint{3.883839in}{2.079384in}}%
\pgfpathlineto{\pgfqpoint{3.884685in}{2.440856in}}%
\pgfpathlineto{\pgfqpoint{3.885530in}{2.382317in}}%
\pgfpathlineto{\pgfqpoint{3.886376in}{2.094208in}}%
\pgfpathlineto{\pgfqpoint{3.887222in}{2.208481in}}%
\pgfpathlineto{\pgfqpoint{3.888067in}{2.283734in}}%
\pgfpathlineto{\pgfqpoint{3.888913in}{2.257153in}}%
\pgfpathlineto{\pgfqpoint{3.889758in}{1.766878in}}%
\pgfpathlineto{\pgfqpoint{3.890604in}{2.348106in}}%
\pgfpathlineto{\pgfqpoint{3.891450in}{2.117712in}}%
\pgfpathlineto{\pgfqpoint{3.893141in}{1.820818in}}%
\pgfpathlineto{\pgfqpoint{3.895678in}{2.590579in}}%
\pgfpathlineto{\pgfqpoint{3.896523in}{2.054543in}}%
\pgfpathlineto{\pgfqpoint{3.897369in}{2.430241in}}%
\pgfpathlineto{\pgfqpoint{3.898215in}{2.000291in}}%
\pgfpathlineto{\pgfqpoint{3.899060in}{2.467967in}}%
\pgfpathlineto{\pgfqpoint{3.899906in}{1.768085in}}%
\pgfpathlineto{\pgfqpoint{3.900752in}{2.108989in}}%
\pgfpathlineto{\pgfqpoint{3.901597in}{2.007923in}}%
\pgfpathlineto{\pgfqpoint{3.902443in}{2.420965in}}%
\pgfpathlineto{\pgfqpoint{3.904134in}{1.742532in}}%
\pgfpathlineto{\pgfqpoint{3.906671in}{2.289090in}}%
\pgfpathlineto{\pgfqpoint{3.907517in}{2.333109in}}%
\pgfpathlineto{\pgfqpoint{3.908362in}{2.640044in}}%
\pgfpathlineto{\pgfqpoint{3.909208in}{1.801066in}}%
\pgfpathlineto{\pgfqpoint{3.910053in}{2.371735in}}%
\pgfpathlineto{\pgfqpoint{3.910899in}{2.662927in}}%
\pgfpathlineto{\pgfqpoint{3.911745in}{2.157839in}}%
\pgfpathlineto{\pgfqpoint{3.912590in}{2.488590in}}%
\pgfpathlineto{\pgfqpoint{3.913436in}{2.351528in}}%
\pgfpathlineto{\pgfqpoint{3.914281in}{1.824889in}}%
\pgfpathlineto{\pgfqpoint{3.915973in}{2.654805in}}%
\pgfpathlineto{\pgfqpoint{3.918510in}{1.820830in}}%
\pgfpathlineto{\pgfqpoint{3.919355in}{2.355308in}}%
\pgfpathlineto{\pgfqpoint{3.920201in}{2.256272in}}%
\pgfpathlineto{\pgfqpoint{3.921046in}{2.291598in}}%
\pgfpathlineto{\pgfqpoint{3.921892in}{1.886100in}}%
\pgfpathlineto{\pgfqpoint{3.922738in}{2.001173in}}%
\pgfpathlineto{\pgfqpoint{3.925275in}{2.845744in}}%
\pgfpathlineto{\pgfqpoint{3.926120in}{2.230514in}}%
\pgfpathlineto{\pgfqpoint{3.926966in}{2.411254in}}%
\pgfpathlineto{\pgfqpoint{3.929503in}{1.999346in}}%
\pgfpathlineto{\pgfqpoint{3.931194in}{2.802201in}}%
\pgfpathlineto{\pgfqpoint{3.934576in}{1.845953in}}%
\pgfpathlineto{\pgfqpoint{3.937113in}{2.606633in}}%
\pgfpathlineto{\pgfqpoint{3.939650in}{1.665325in}}%
\pgfpathlineto{\pgfqpoint{3.940496in}{2.596589in}}%
\pgfpathlineto{\pgfqpoint{3.941341in}{2.199786in}}%
\pgfpathlineto{\pgfqpoint{3.942187in}{2.352898in}}%
\pgfpathlineto{\pgfqpoint{3.943033in}{1.553026in}}%
\pgfpathlineto{\pgfqpoint{3.943878in}{1.896457in}}%
\pgfpathlineto{\pgfqpoint{3.944724in}{1.776883in}}%
\pgfpathlineto{\pgfqpoint{3.947261in}{2.460347in}}%
\pgfpathlineto{\pgfqpoint{3.948106in}{1.583222in}}%
\pgfpathlineto{\pgfqpoint{3.948952in}{2.073459in}}%
\pgfpathlineto{\pgfqpoint{3.949798in}{2.733805in}}%
\pgfpathlineto{\pgfqpoint{3.950643in}{2.514236in}}%
\pgfpathlineto{\pgfqpoint{3.953180in}{1.649438in}}%
\pgfpathlineto{\pgfqpoint{3.954871in}{2.587511in}}%
\pgfpathlineto{\pgfqpoint{3.955717in}{1.888763in}}%
\pgfpathlineto{\pgfqpoint{3.956563in}{2.678047in}}%
\pgfpathlineto{\pgfqpoint{3.957408in}{2.124237in}}%
\pgfpathlineto{\pgfqpoint{3.958254in}{2.180358in}}%
\pgfpathlineto{\pgfqpoint{3.959100in}{2.387707in}}%
\pgfpathlineto{\pgfqpoint{3.959945in}{1.957959in}}%
\pgfpathlineto{\pgfqpoint{3.961636in}{2.547796in}}%
\pgfpathlineto{\pgfqpoint{3.964173in}{1.625714in}}%
\pgfpathlineto{\pgfqpoint{3.965865in}{1.830729in}}%
\pgfpathlineto{\pgfqpoint{3.966710in}{1.809181in}}%
\pgfpathlineto{\pgfqpoint{3.969247in}{2.537273in}}%
\pgfpathlineto{\pgfqpoint{3.970938in}{2.019433in}}%
\pgfpathlineto{\pgfqpoint{3.972630in}{2.565911in}}%
\pgfpathlineto{\pgfqpoint{3.973475in}{1.773036in}}%
\pgfpathlineto{\pgfqpoint{3.974321in}{2.208323in}}%
\pgfpathlineto{\pgfqpoint{3.975166in}{2.227206in}}%
\pgfpathlineto{\pgfqpoint{3.976012in}{1.959641in}}%
\pgfpathlineto{\pgfqpoint{3.976858in}{2.697051in}}%
\pgfpathlineto{\pgfqpoint{3.977703in}{2.030839in}}%
\pgfpathlineto{\pgfqpoint{3.978549in}{2.267890in}}%
\pgfpathlineto{\pgfqpoint{3.981086in}{2.640501in}}%
\pgfpathlineto{\pgfqpoint{3.981931in}{2.797649in}}%
\pgfpathlineto{\pgfqpoint{3.982777in}{2.565142in}}%
\pgfpathlineto{\pgfqpoint{3.983623in}{2.628560in}}%
\pgfpathlineto{\pgfqpoint{3.984468in}{1.506154in}}%
\pgfpathlineto{\pgfqpoint{3.985314in}{2.124104in}}%
\pgfpathlineto{\pgfqpoint{3.986160in}{2.160735in}}%
\pgfpathlineto{\pgfqpoint{3.987005in}{1.778925in}}%
\pgfpathlineto{\pgfqpoint{3.987851in}{2.168843in}}%
\pgfpathlineto{\pgfqpoint{3.988696in}{2.167516in}}%
\pgfpathlineto{\pgfqpoint{3.989542in}{2.138213in}}%
\pgfpathlineto{\pgfqpoint{3.992079in}{3.159647in}}%
\pgfpathlineto{\pgfqpoint{3.993770in}{2.134244in}}%
\pgfpathlineto{\pgfqpoint{3.995461in}{2.493847in}}%
\pgfpathlineto{\pgfqpoint{3.996307in}{1.839093in}}%
\pgfpathlineto{\pgfqpoint{3.997998in}{2.466753in}}%
\pgfpathlineto{\pgfqpoint{3.998844in}{2.035715in}}%
\pgfpathlineto{\pgfqpoint{3.999689in}{2.701498in}}%
\pgfpathlineto{\pgfqpoint{4.000535in}{1.965937in}}%
\pgfpathlineto{\pgfqpoint{4.001381in}{2.317407in}}%
\pgfpathlineto{\pgfqpoint{4.002226in}{2.309001in}}%
\pgfpathlineto{\pgfqpoint{4.003918in}{2.571618in}}%
\pgfpathlineto{\pgfqpoint{4.004763in}{2.998300in}}%
\pgfpathlineto{\pgfqpoint{4.005609in}{2.227843in}}%
\pgfpathlineto{\pgfqpoint{4.006454in}{2.615814in}}%
\pgfpathlineto{\pgfqpoint{4.007300in}{2.840868in}}%
\pgfpathlineto{\pgfqpoint{4.008991in}{2.010614in}}%
\pgfpathlineto{\pgfqpoint{4.009837in}{2.614725in}}%
\pgfpathlineto{\pgfqpoint{4.012374in}{1.629080in}}%
\pgfpathlineto{\pgfqpoint{4.014065in}{2.078448in}}%
\pgfpathlineto{\pgfqpoint{4.014911in}{1.774541in}}%
\pgfpathlineto{\pgfqpoint{4.015756in}{1.859312in}}%
\pgfpathlineto{\pgfqpoint{4.016602in}{2.745640in}}%
\pgfpathlineto{\pgfqpoint{4.017448in}{1.767313in}}%
\pgfpathlineto{\pgfqpoint{4.018293in}{2.078912in}}%
\pgfpathlineto{\pgfqpoint{4.019139in}{2.536310in}}%
\pgfpathlineto{\pgfqpoint{4.019984in}{2.470312in}}%
\pgfpathlineto{\pgfqpoint{4.020830in}{2.499266in}}%
\pgfpathlineto{\pgfqpoint{4.021676in}{2.388197in}}%
\pgfpathlineto{\pgfqpoint{4.023367in}{1.434504in}}%
\pgfpathlineto{\pgfqpoint{4.024213in}{2.153567in}}%
\pgfpathlineto{\pgfqpoint{4.025058in}{2.098560in}}%
\pgfpathlineto{\pgfqpoint{4.025904in}{2.219584in}}%
\pgfpathlineto{\pgfqpoint{4.026749in}{2.163592in}}%
\pgfpathlineto{\pgfqpoint{4.027595in}{1.795359in}}%
\pgfpathlineto{\pgfqpoint{4.028441in}{1.809854in}}%
\pgfpathlineto{\pgfqpoint{4.030132in}{2.229228in}}%
\pgfpathlineto{\pgfqpoint{4.030978in}{1.834944in}}%
\pgfpathlineto{\pgfqpoint{4.031823in}{1.864975in}}%
\pgfpathlineto{\pgfqpoint{4.032669in}{2.231733in}}%
\pgfpathlineto{\pgfqpoint{4.033514in}{1.689404in}}%
\pgfpathlineto{\pgfqpoint{4.035206in}{2.313306in}}%
\pgfpathlineto{\pgfqpoint{4.036051in}{2.228972in}}%
\pgfpathlineto{\pgfqpoint{4.036897in}{2.625857in}}%
\pgfpathlineto{\pgfqpoint{4.037743in}{2.448523in}}%
\pgfpathlineto{\pgfqpoint{4.039434in}{2.489119in}}%
\pgfpathlineto{\pgfqpoint{4.040279in}{2.021571in}}%
\pgfpathlineto{\pgfqpoint{4.041125in}{2.517998in}}%
\pgfpathlineto{\pgfqpoint{4.041971in}{1.559117in}}%
\pgfpathlineto{\pgfqpoint{4.042816in}{2.130493in}}%
\pgfpathlineto{\pgfqpoint{4.043662in}{2.557273in}}%
\pgfpathlineto{\pgfqpoint{4.044508in}{2.419877in}}%
\pgfpathlineto{\pgfqpoint{4.045353in}{1.774312in}}%
\pgfpathlineto{\pgfqpoint{4.046199in}{1.775348in}}%
\pgfpathlineto{\pgfqpoint{4.047890in}{2.612194in}}%
\pgfpathlineto{\pgfqpoint{4.048736in}{2.475810in}}%
\pgfpathlineto{\pgfqpoint{4.049581in}{1.923358in}}%
\pgfpathlineto{\pgfqpoint{4.050427in}{2.431407in}}%
\pgfpathlineto{\pgfqpoint{4.051273in}{2.170862in}}%
\pgfpathlineto{\pgfqpoint{4.052118in}{1.900598in}}%
\pgfpathlineto{\pgfqpoint{4.052964in}{1.984261in}}%
\pgfpathlineto{\pgfqpoint{4.055501in}{2.583194in}}%
\pgfpathlineto{\pgfqpoint{4.057192in}{1.876148in}}%
\pgfpathlineto{\pgfqpoint{4.058038in}{2.536085in}}%
\pgfpathlineto{\pgfqpoint{4.058883in}{2.519200in}}%
\pgfpathlineto{\pgfqpoint{4.059729in}{1.852376in}}%
\pgfpathlineto{\pgfqpoint{4.060574in}{2.024224in}}%
\pgfpathlineto{\pgfqpoint{4.063111in}{2.561765in}}%
\pgfpathlineto{\pgfqpoint{4.067339in}{2.035792in}}%
\pgfpathlineto{\pgfqpoint{4.068185in}{2.202302in}}%
\pgfpathlineto{\pgfqpoint{4.069031in}{2.522252in}}%
\pgfpathlineto{\pgfqpoint{4.071567in}{1.912073in}}%
\pgfpathlineto{\pgfqpoint{4.072413in}{2.202707in}}%
\pgfpathlineto{\pgfqpoint{4.073259in}{2.010182in}}%
\pgfpathlineto{\pgfqpoint{4.074104in}{2.023144in}}%
\pgfpathlineto{\pgfqpoint{4.075796in}{2.875346in}}%
\pgfpathlineto{\pgfqpoint{4.078332in}{1.872931in}}%
\pgfpathlineto{\pgfqpoint{4.080024in}{2.167236in}}%
\pgfpathlineto{\pgfqpoint{4.080869in}{2.639928in}}%
\pgfpathlineto{\pgfqpoint{4.081715in}{2.351758in}}%
\pgfpathlineto{\pgfqpoint{4.082561in}{2.099584in}}%
\pgfpathlineto{\pgfqpoint{4.083406in}{2.524152in}}%
\pgfpathlineto{\pgfqpoint{4.084252in}{2.183327in}}%
\pgfpathlineto{\pgfqpoint{4.085097in}{2.411151in}}%
\pgfpathlineto{\pgfqpoint{4.085943in}{2.132762in}}%
\pgfpathlineto{\pgfqpoint{4.086789in}{2.616500in}}%
\pgfpathlineto{\pgfqpoint{4.087634in}{2.232255in}}%
\pgfpathlineto{\pgfqpoint{4.088480in}{1.741343in}}%
\pgfpathlineto{\pgfqpoint{4.089326in}{2.432386in}}%
\pgfpathlineto{\pgfqpoint{4.090171in}{2.303849in}}%
\pgfpathlineto{\pgfqpoint{4.091017in}{2.036336in}}%
\pgfpathlineto{\pgfqpoint{4.092708in}{2.609121in}}%
\pgfpathlineto{\pgfqpoint{4.094399in}{2.391994in}}%
\pgfpathlineto{\pgfqpoint{4.096936in}{1.888027in}}%
\pgfpathlineto{\pgfqpoint{4.099473in}{2.714259in}}%
\pgfpathlineto{\pgfqpoint{4.101164in}{1.771082in}}%
\pgfpathlineto{\pgfqpoint{4.102856in}{2.567291in}}%
\pgfpathlineto{\pgfqpoint{4.103701in}{1.748969in}}%
\pgfpathlineto{\pgfqpoint{4.104547in}{2.404513in}}%
\pgfpathlineto{\pgfqpoint{4.105392in}{2.646673in}}%
\pgfpathlineto{\pgfqpoint{4.106238in}{2.091053in}}%
\pgfpathlineto{\pgfqpoint{4.107084in}{2.973090in}}%
\pgfpathlineto{\pgfqpoint{4.107929in}{2.428150in}}%
\pgfpathlineto{\pgfqpoint{4.108775in}{2.327009in}}%
\pgfpathlineto{\pgfqpoint{4.109621in}{1.671524in}}%
\pgfpathlineto{\pgfqpoint{4.110466in}{2.245886in}}%
\pgfpathlineto{\pgfqpoint{4.111312in}{1.797579in}}%
\pgfpathlineto{\pgfqpoint{4.113849in}{2.321463in}}%
\pgfpathlineto{\pgfqpoint{4.114694in}{2.389631in}}%
\pgfpathlineto{\pgfqpoint{4.115540in}{1.372145in}}%
\pgfpathlineto{\pgfqpoint{4.116386in}{2.477412in}}%
\pgfpathlineto{\pgfqpoint{4.117231in}{2.171942in}}%
\pgfpathlineto{\pgfqpoint{4.118077in}{2.051979in}}%
\pgfpathlineto{\pgfqpoint{4.118922in}{2.496710in}}%
\pgfpathlineto{\pgfqpoint{4.119768in}{2.383531in}}%
\pgfpathlineto{\pgfqpoint{4.120614in}{1.621815in}}%
\pgfpathlineto{\pgfqpoint{4.121459in}{1.954180in}}%
\pgfpathlineto{\pgfqpoint{4.122305in}{2.593018in}}%
\pgfpathlineto{\pgfqpoint{4.124842in}{1.451510in}}%
\pgfpathlineto{\pgfqpoint{4.125687in}{2.852450in}}%
\pgfpathlineto{\pgfqpoint{4.126533in}{1.921385in}}%
\pgfpathlineto{\pgfqpoint{4.127379in}{2.305269in}}%
\pgfpathlineto{\pgfqpoint{4.128224in}{1.587381in}}%
\pgfpathlineto{\pgfqpoint{4.129916in}{2.759917in}}%
\pgfpathlineto{\pgfqpoint{4.130761in}{2.709142in}}%
\pgfpathlineto{\pgfqpoint{4.131607in}{1.894392in}}%
\pgfpathlineto{\pgfqpoint{4.132452in}{2.278777in}}%
\pgfpathlineto{\pgfqpoint{4.133298in}{2.639536in}}%
\pgfpathlineto{\pgfqpoint{4.134144in}{2.453134in}}%
\pgfpathlineto{\pgfqpoint{4.134989in}{1.968216in}}%
\pgfpathlineto{\pgfqpoint{4.135835in}{2.166694in}}%
\pgfpathlineto{\pgfqpoint{4.137526in}{2.607881in}}%
\pgfpathlineto{\pgfqpoint{4.138372in}{2.296694in}}%
\pgfpathlineto{\pgfqpoint{4.139217in}{2.422053in}}%
\pgfpathlineto{\pgfqpoint{4.140063in}{2.859167in}}%
\pgfpathlineto{\pgfqpoint{4.140909in}{2.025426in}}%
\pgfpathlineto{\pgfqpoint{4.141754in}{2.081331in}}%
\pgfpathlineto{\pgfqpoint{4.142600in}{2.152487in}}%
\pgfpathlineto{\pgfqpoint{4.143446in}{1.954268in}}%
\pgfpathlineto{\pgfqpoint{4.144291in}{2.513727in}}%
\pgfpathlineto{\pgfqpoint{4.145137in}{2.336150in}}%
\pgfpathlineto{\pgfqpoint{4.145982in}{2.457691in}}%
\pgfpathlineto{\pgfqpoint{4.147674in}{1.975264in}}%
\pgfpathlineto{\pgfqpoint{4.148519in}{2.110834in}}%
\pgfpathlineto{\pgfqpoint{4.149365in}{1.922407in}}%
\pgfpathlineto{\pgfqpoint{4.150210in}{2.421330in}}%
\pgfpathlineto{\pgfqpoint{4.151056in}{1.764158in}}%
\pgfpathlineto{\pgfqpoint{4.151902in}{2.157224in}}%
\pgfpathlineto{\pgfqpoint{4.152747in}{1.954882in}}%
\pgfpathlineto{\pgfqpoint{4.153593in}{2.732998in}}%
\pgfpathlineto{\pgfqpoint{4.154439in}{2.697382in}}%
\pgfpathlineto{\pgfqpoint{4.156130in}{1.624772in}}%
\pgfpathlineto{\pgfqpoint{4.157821in}{2.614766in}}%
\pgfpathlineto{\pgfqpoint{4.158667in}{2.572820in}}%
\pgfpathlineto{\pgfqpoint{4.159512in}{2.345016in}}%
\pgfpathlineto{\pgfqpoint{4.160358in}{2.503110in}}%
\pgfpathlineto{\pgfqpoint{4.162049in}{2.351536in}}%
\pgfpathlineto{\pgfqpoint{4.162895in}{2.468980in}}%
\pgfpathlineto{\pgfqpoint{4.163740in}{1.966840in}}%
\pgfpathlineto{\pgfqpoint{4.164586in}{2.151355in}}%
\pgfpathlineto{\pgfqpoint{4.165432in}{2.634567in}}%
\pgfpathlineto{\pgfqpoint{4.167969in}{1.919892in}}%
\pgfpathlineto{\pgfqpoint{4.168814in}{2.086425in}}%
\pgfpathlineto{\pgfqpoint{4.169660in}{1.676454in}}%
\pgfpathlineto{\pgfqpoint{4.170505in}{1.698881in}}%
\pgfpathlineto{\pgfqpoint{4.173042in}{2.622063in}}%
\pgfpathlineto{\pgfqpoint{4.174734in}{1.927492in}}%
\pgfpathlineto{\pgfqpoint{4.175579in}{1.968493in}}%
\pgfpathlineto{\pgfqpoint{4.178116in}{2.352119in}}%
\pgfpathlineto{\pgfqpoint{4.178962in}{1.968238in}}%
\pgfpathlineto{\pgfqpoint{4.179807in}{2.320417in}}%
\pgfpathlineto{\pgfqpoint{4.180653in}{2.305120in}}%
\pgfpathlineto{\pgfqpoint{4.181499in}{1.813127in}}%
\pgfpathlineto{\pgfqpoint{4.182344in}{2.283174in}}%
\pgfpathlineto{\pgfqpoint{4.183190in}{1.481806in}}%
\pgfpathlineto{\pgfqpoint{4.184035in}{2.404353in}}%
\pgfpathlineto{\pgfqpoint{4.184881in}{2.363191in}}%
\pgfpathlineto{\pgfqpoint{4.185727in}{1.002928in}}%
\pgfpathlineto{\pgfqpoint{4.188264in}{2.858199in}}%
\pgfpathlineto{\pgfqpoint{4.189955in}{3.187986in}}%
\pgfpathlineto{\pgfqpoint{4.191646in}{1.982826in}}%
\pgfpathlineto{\pgfqpoint{4.192492in}{2.820058in}}%
\pgfpathlineto{\pgfqpoint{4.193337in}{2.236852in}}%
\pgfpathlineto{\pgfqpoint{4.195029in}{1.946100in}}%
\pgfpathlineto{\pgfqpoint{4.195874in}{2.107380in}}%
\pgfpathlineto{\pgfqpoint{4.196720in}{2.670211in}}%
\pgfpathlineto{\pgfqpoint{4.197565in}{1.593532in}}%
\pgfpathlineto{\pgfqpoint{4.198411in}{1.969743in}}%
\pgfpathlineto{\pgfqpoint{4.199257in}{2.231882in}}%
\pgfpathlineto{\pgfqpoint{4.200102in}{2.199736in}}%
\pgfpathlineto{\pgfqpoint{4.201794in}{1.933251in}}%
\pgfpathlineto{\pgfqpoint{4.202639in}{2.054278in}}%
\pgfpathlineto{\pgfqpoint{4.203485in}{2.594142in}}%
\pgfpathlineto{\pgfqpoint{4.204330in}{1.968873in}}%
\pgfpathlineto{\pgfqpoint{4.205176in}{2.191073in}}%
\pgfpathlineto{\pgfqpoint{4.206022in}{1.869532in}}%
\pgfpathlineto{\pgfqpoint{4.206867in}{2.593596in}}%
\pgfpathlineto{\pgfqpoint{4.207713in}{1.838686in}}%
\pgfpathlineto{\pgfqpoint{4.208559in}{2.285097in}}%
\pgfpathlineto{\pgfqpoint{4.209404in}{2.322009in}}%
\pgfpathlineto{\pgfqpoint{4.211095in}{1.821761in}}%
\pgfpathlineto{\pgfqpoint{4.211941in}{1.887277in}}%
\pgfpathlineto{\pgfqpoint{4.213632in}{2.620992in}}%
\pgfpathlineto{\pgfqpoint{4.214478in}{1.992449in}}%
\pgfpathlineto{\pgfqpoint{4.215324in}{2.333822in}}%
\pgfpathlineto{\pgfqpoint{4.216169in}{2.305833in}}%
\pgfpathlineto{\pgfqpoint{4.217860in}{1.871365in}}%
\pgfpathlineto{\pgfqpoint{4.218706in}{2.591586in}}%
\pgfpathlineto{\pgfqpoint{4.219552in}{2.581952in}}%
\pgfpathlineto{\pgfqpoint{4.220397in}{1.737206in}}%
\pgfpathlineto{\pgfqpoint{4.221243in}{2.146856in}}%
\pgfpathlineto{\pgfqpoint{4.222089in}{1.953418in}}%
\pgfpathlineto{\pgfqpoint{4.223780in}{2.554744in}}%
\pgfpathlineto{\pgfqpoint{4.225471in}{2.202623in}}%
\pgfpathlineto{\pgfqpoint{4.226317in}{2.876359in}}%
\pgfpathlineto{\pgfqpoint{4.227162in}{1.709184in}}%
\pgfpathlineto{\pgfqpoint{4.228008in}{3.065903in}}%
\pgfpathlineto{\pgfqpoint{4.228853in}{1.856972in}}%
\pgfpathlineto{\pgfqpoint{4.229699in}{2.160737in}}%
\pgfpathlineto{\pgfqpoint{4.232236in}{2.565518in}}%
\pgfpathlineto{\pgfqpoint{4.233082in}{2.850329in}}%
\pgfpathlineto{\pgfqpoint{4.234773in}{1.900992in}}%
\pgfpathlineto{\pgfqpoint{4.236464in}{2.775267in}}%
\pgfpathlineto{\pgfqpoint{4.238155in}{1.828858in}}%
\pgfpathlineto{\pgfqpoint{4.239847in}{2.113588in}}%
\pgfpathlineto{\pgfqpoint{4.240692in}{1.533322in}}%
\pgfpathlineto{\pgfqpoint{4.241538in}{1.750492in}}%
\pgfpathlineto{\pgfqpoint{4.242383in}{2.393811in}}%
\pgfpathlineto{\pgfqpoint{4.243229in}{2.257702in}}%
\pgfpathlineto{\pgfqpoint{4.244075in}{2.118220in}}%
\pgfpathlineto{\pgfqpoint{4.244920in}{2.685965in}}%
\pgfpathlineto{\pgfqpoint{4.245766in}{2.346512in}}%
\pgfpathlineto{\pgfqpoint{4.246612in}{2.195317in}}%
\pgfpathlineto{\pgfqpoint{4.247457in}{2.642609in}}%
\pgfpathlineto{\pgfqpoint{4.248303in}{2.183508in}}%
\pgfpathlineto{\pgfqpoint{4.249148in}{2.516908in}}%
\pgfpathlineto{\pgfqpoint{4.251685in}{1.885628in}}%
\pgfpathlineto{\pgfqpoint{4.254222in}{2.705891in}}%
\pgfpathlineto{\pgfqpoint{4.255913in}{1.776794in}}%
\pgfpathlineto{\pgfqpoint{4.256759in}{3.102785in}}%
\pgfpathlineto{\pgfqpoint{4.257605in}{2.375954in}}%
\pgfpathlineto{\pgfqpoint{4.259296in}{2.533941in}}%
\pgfpathlineto{\pgfqpoint{4.260142in}{2.389510in}}%
\pgfpathlineto{\pgfqpoint{4.260987in}{1.809016in}}%
\pgfpathlineto{\pgfqpoint{4.261833in}{1.957636in}}%
\pgfpathlineto{\pgfqpoint{4.264370in}{2.647679in}}%
\pgfpathlineto{\pgfqpoint{4.265215in}{2.646331in}}%
\pgfpathlineto{\pgfqpoint{4.266061in}{2.074594in}}%
\pgfpathlineto{\pgfqpoint{4.266907in}{2.367996in}}%
\pgfpathlineto{\pgfqpoint{4.267752in}{2.288137in}}%
\pgfpathlineto{\pgfqpoint{4.269443in}{1.420831in}}%
\pgfpathlineto{\pgfqpoint{4.270289in}{2.445026in}}%
\pgfpathlineto{\pgfqpoint{4.271135in}{1.682069in}}%
\pgfpathlineto{\pgfqpoint{4.272826in}{2.215482in}}%
\pgfpathlineto{\pgfqpoint{4.273672in}{1.979006in}}%
\pgfpathlineto{\pgfqpoint{4.274517in}{2.055140in}}%
\pgfpathlineto{\pgfqpoint{4.276208in}{1.917361in}}%
\pgfpathlineto{\pgfqpoint{4.277054in}{2.085000in}}%
\pgfpathlineto{\pgfqpoint{4.277900in}{1.868406in}}%
\pgfpathlineto{\pgfqpoint{4.280437in}{2.684702in}}%
\pgfpathlineto{\pgfqpoint{4.282128in}{1.702575in}}%
\pgfpathlineto{\pgfqpoint{4.283819in}{1.941728in}}%
\pgfpathlineto{\pgfqpoint{4.284665in}{2.496009in}}%
\pgfpathlineto{\pgfqpoint{4.285510in}{2.271181in}}%
\pgfpathlineto{\pgfqpoint{4.286356in}{2.444714in}}%
\pgfpathlineto{\pgfqpoint{4.287202in}{2.068612in}}%
\pgfpathlineto{\pgfqpoint{4.288047in}{2.129409in}}%
\pgfpathlineto{\pgfqpoint{4.289738in}{2.608454in}}%
\pgfpathlineto{\pgfqpoint{4.290584in}{1.526439in}}%
\pgfpathlineto{\pgfqpoint{4.291430in}{2.414001in}}%
\pgfpathlineto{\pgfqpoint{4.293967in}{1.691412in}}%
\pgfpathlineto{\pgfqpoint{4.294812in}{1.630007in}}%
\pgfpathlineto{\pgfqpoint{4.295658in}{2.609353in}}%
\pgfpathlineto{\pgfqpoint{4.296503in}{1.856646in}}%
\pgfpathlineto{\pgfqpoint{4.297349in}{2.232124in}}%
\pgfpathlineto{\pgfqpoint{4.299040in}{1.667991in}}%
\pgfpathlineto{\pgfqpoint{4.300732in}{2.874305in}}%
\pgfpathlineto{\pgfqpoint{4.302423in}{2.311719in}}%
\pgfpathlineto{\pgfqpoint{4.303268in}{2.245422in}}%
\pgfpathlineto{\pgfqpoint{4.304114in}{1.855150in}}%
\pgfpathlineto{\pgfqpoint{4.304960in}{2.452447in}}%
\pgfpathlineto{\pgfqpoint{4.305805in}{2.114365in}}%
\pgfpathlineto{\pgfqpoint{4.306651in}{1.893546in}}%
\pgfpathlineto{\pgfqpoint{4.307496in}{2.019957in}}%
\pgfpathlineto{\pgfqpoint{4.310033in}{2.293668in}}%
\pgfpathlineto{\pgfqpoint{4.310879in}{2.164113in}}%
\pgfpathlineto{\pgfqpoint{4.311725in}{2.198905in}}%
\pgfpathlineto{\pgfqpoint{4.312570in}{1.951722in}}%
\pgfpathlineto{\pgfqpoint{4.313416in}{2.688842in}}%
\pgfpathlineto{\pgfqpoint{4.314261in}{2.007783in}}%
\pgfpathlineto{\pgfqpoint{4.315107in}{2.068515in}}%
\pgfpathlineto{\pgfqpoint{4.315953in}{2.347185in}}%
\pgfpathlineto{\pgfqpoint{4.316798in}{2.275978in}}%
\pgfpathlineto{\pgfqpoint{4.317644in}{1.840168in}}%
\pgfpathlineto{\pgfqpoint{4.318490in}{2.430661in}}%
\pgfpathlineto{\pgfqpoint{4.319335in}{2.044075in}}%
\pgfpathlineto{\pgfqpoint{4.320181in}{2.080705in}}%
\pgfpathlineto{\pgfqpoint{4.321026in}{2.342015in}}%
\pgfpathlineto{\pgfqpoint{4.321872in}{1.730462in}}%
\pgfpathlineto{\pgfqpoint{4.322718in}{2.111095in}}%
\pgfpathlineto{\pgfqpoint{4.324409in}{2.432247in}}%
\pgfpathlineto{\pgfqpoint{4.326100in}{1.571166in}}%
\pgfpathlineto{\pgfqpoint{4.328637in}{2.376961in}}%
\pgfpathlineto{\pgfqpoint{4.329483in}{2.392412in}}%
\pgfpathlineto{\pgfqpoint{4.330328in}{2.495536in}}%
\pgfpathlineto{\pgfqpoint{4.331174in}{2.075262in}}%
\pgfpathlineto{\pgfqpoint{4.332020in}{2.514009in}}%
\pgfpathlineto{\pgfqpoint{4.332865in}{2.304613in}}%
\pgfpathlineto{\pgfqpoint{4.333711in}{2.401599in}}%
\pgfpathlineto{\pgfqpoint{4.337093in}{1.866094in}}%
\pgfpathlineto{\pgfqpoint{4.337939in}{2.409414in}}%
\pgfpathlineto{\pgfqpoint{4.338785in}{2.179190in}}%
\pgfpathlineto{\pgfqpoint{4.339630in}{2.204052in}}%
\pgfpathlineto{\pgfqpoint{4.340476in}{2.460128in}}%
\pgfpathlineto{\pgfqpoint{4.341321in}{1.923841in}}%
\pgfpathlineto{\pgfqpoint{4.342167in}{2.813866in}}%
\pgfpathlineto{\pgfqpoint{4.343013in}{2.451935in}}%
\pgfpathlineto{\pgfqpoint{4.343858in}{1.836997in}}%
\pgfpathlineto{\pgfqpoint{4.345550in}{2.627579in}}%
\pgfpathlineto{\pgfqpoint{4.347241in}{2.112839in}}%
\pgfpathlineto{\pgfqpoint{4.349778in}{2.958274in}}%
\pgfpathlineto{\pgfqpoint{4.352315in}{1.929628in}}%
\pgfpathlineto{\pgfqpoint{4.353160in}{1.782580in}}%
\pgfpathlineto{\pgfqpoint{4.354006in}{2.394842in}}%
\pgfpathlineto{\pgfqpoint{4.354851in}{2.374643in}}%
\pgfpathlineto{\pgfqpoint{4.355697in}{1.836319in}}%
\pgfpathlineto{\pgfqpoint{4.356543in}{2.444245in}}%
\pgfpathlineto{\pgfqpoint{4.357388in}{2.097643in}}%
\pgfpathlineto{\pgfqpoint{4.358234in}{2.307255in}}%
\pgfpathlineto{\pgfqpoint{4.359080in}{2.071574in}}%
\pgfpathlineto{\pgfqpoint{4.359925in}{2.203803in}}%
\pgfpathlineto{\pgfqpoint{4.360771in}{2.201246in}}%
\pgfpathlineto{\pgfqpoint{4.361616in}{1.877640in}}%
\pgfpathlineto{\pgfqpoint{4.362462in}{2.741471in}}%
\pgfpathlineto{\pgfqpoint{4.363308in}{2.330374in}}%
\pgfpathlineto{\pgfqpoint{4.364153in}{1.764168in}}%
\pgfpathlineto{\pgfqpoint{4.364999in}{2.114465in}}%
\pgfpathlineto{\pgfqpoint{4.365845in}{2.394812in}}%
\pgfpathlineto{\pgfqpoint{4.366690in}{1.823568in}}%
\pgfpathlineto{\pgfqpoint{4.367536in}{2.621831in}}%
\pgfpathlineto{\pgfqpoint{4.368381in}{2.520589in}}%
\pgfpathlineto{\pgfqpoint{4.369227in}{1.542866in}}%
\pgfpathlineto{\pgfqpoint{4.370073in}{2.060468in}}%
\pgfpathlineto{\pgfqpoint{4.370918in}{1.772794in}}%
\pgfpathlineto{\pgfqpoint{4.371764in}{2.666544in}}%
\pgfpathlineto{\pgfqpoint{4.372610in}{2.470084in}}%
\pgfpathlineto{\pgfqpoint{4.373455in}{1.710907in}}%
\pgfpathlineto{\pgfqpoint{4.374301in}{1.950526in}}%
\pgfpathlineto{\pgfqpoint{4.375146in}{2.905628in}}%
\pgfpathlineto{\pgfqpoint{4.375992in}{2.341146in}}%
\pgfpathlineto{\pgfqpoint{4.376838in}{2.069459in}}%
\pgfpathlineto{\pgfqpoint{4.377683in}{2.403213in}}%
\pgfpathlineto{\pgfqpoint{4.378529in}{1.728709in}}%
\pgfpathlineto{\pgfqpoint{4.381066in}{3.066008in}}%
\pgfpathlineto{\pgfqpoint{4.381911in}{1.871790in}}%
\pgfpathlineto{\pgfqpoint{4.382757in}{2.207883in}}%
\pgfpathlineto{\pgfqpoint{4.383603in}{2.553538in}}%
\pgfpathlineto{\pgfqpoint{4.384448in}{2.537992in}}%
\pgfpathlineto{\pgfqpoint{4.385294in}{1.806794in}}%
\pgfpathlineto{\pgfqpoint{4.386140in}{2.739149in}}%
\pgfpathlineto{\pgfqpoint{4.386985in}{2.062102in}}%
\pgfpathlineto{\pgfqpoint{4.387831in}{2.539158in}}%
\pgfpathlineto{\pgfqpoint{4.388676in}{1.920061in}}%
\pgfpathlineto{\pgfqpoint{4.389522in}{2.053869in}}%
\pgfpathlineto{\pgfqpoint{4.390368in}{2.080744in}}%
\pgfpathlineto{\pgfqpoint{4.391213in}{2.692021in}}%
\pgfpathlineto{\pgfqpoint{4.392059in}{2.497384in}}%
\pgfpathlineto{\pgfqpoint{4.392904in}{1.768603in}}%
\pgfpathlineto{\pgfqpoint{4.393750in}{1.771812in}}%
\pgfpathlineto{\pgfqpoint{4.394596in}{2.490690in}}%
\pgfpathlineto{\pgfqpoint{4.395441in}{2.237964in}}%
\pgfpathlineto{\pgfqpoint{4.396287in}{1.851481in}}%
\pgfpathlineto{\pgfqpoint{4.397133in}{2.645135in}}%
\pgfpathlineto{\pgfqpoint{4.397978in}{2.160260in}}%
\pgfpathlineto{\pgfqpoint{4.399669in}{1.777817in}}%
\pgfpathlineto{\pgfqpoint{4.400515in}{2.189801in}}%
\pgfpathlineto{\pgfqpoint{4.401361in}{1.634609in}}%
\pgfpathlineto{\pgfqpoint{4.403052in}{2.481335in}}%
\pgfpathlineto{\pgfqpoint{4.403898in}{2.010255in}}%
\pgfpathlineto{\pgfqpoint{4.404743in}{2.567293in}}%
\pgfpathlineto{\pgfqpoint{4.405589in}{1.791476in}}%
\pgfpathlineto{\pgfqpoint{4.406434in}{1.823410in}}%
\pgfpathlineto{\pgfqpoint{4.408971in}{2.729280in}}%
\pgfpathlineto{\pgfqpoint{4.409817in}{2.674612in}}%
\pgfpathlineto{\pgfqpoint{4.411508in}{1.490798in}}%
\pgfpathlineto{\pgfqpoint{4.414045in}{2.505054in}}%
\pgfpathlineto{\pgfqpoint{4.414891in}{1.728318in}}%
\pgfpathlineto{\pgfqpoint{4.415736in}{2.142387in}}%
\pgfpathlineto{\pgfqpoint{4.416582in}{1.776105in}}%
\pgfpathlineto{\pgfqpoint{4.417428in}{2.363982in}}%
\pgfpathlineto{\pgfqpoint{4.418273in}{1.943461in}}%
\pgfpathlineto{\pgfqpoint{4.419119in}{2.269541in}}%
\pgfpathlineto{\pgfqpoint{4.419964in}{1.935132in}}%
\pgfpathlineto{\pgfqpoint{4.421656in}{2.503257in}}%
\pgfpathlineto{\pgfqpoint{4.422501in}{1.698272in}}%
\pgfpathlineto{\pgfqpoint{4.423347in}{1.903719in}}%
\pgfpathlineto{\pgfqpoint{4.425038in}{2.426396in}}%
\pgfpathlineto{\pgfqpoint{4.425884in}{2.061303in}}%
\pgfpathlineto{\pgfqpoint{4.426729in}{2.862770in}}%
\pgfpathlineto{\pgfqpoint{4.427575in}{2.010204in}}%
\pgfpathlineto{\pgfqpoint{4.428421in}{2.260644in}}%
\pgfpathlineto{\pgfqpoint{4.429266in}{2.551337in}}%
\pgfpathlineto{\pgfqpoint{4.430112in}{2.314588in}}%
\pgfpathlineto{\pgfqpoint{4.430958in}{1.257285in}}%
\pgfpathlineto{\pgfqpoint{4.431803in}{1.966083in}}%
\pgfpathlineto{\pgfqpoint{4.432649in}{3.157879in}}%
\pgfpathlineto{\pgfqpoint{4.433494in}{2.494952in}}%
\pgfpathlineto{\pgfqpoint{4.434340in}{2.726252in}}%
\pgfpathlineto{\pgfqpoint{4.435186in}{2.126097in}}%
\pgfpathlineto{\pgfqpoint{4.436031in}{2.525050in}}%
\pgfpathlineto{\pgfqpoint{4.437723in}{1.435023in}}%
\pgfpathlineto{\pgfqpoint{4.439414in}{2.352888in}}%
\pgfpathlineto{\pgfqpoint{4.440259in}{1.534030in}}%
\pgfpathlineto{\pgfqpoint{4.441105in}{2.024408in}}%
\pgfpathlineto{\pgfqpoint{4.441951in}{1.806062in}}%
\pgfpathlineto{\pgfqpoint{4.442796in}{1.878909in}}%
\pgfpathlineto{\pgfqpoint{4.443642in}{2.569187in}}%
\pgfpathlineto{\pgfqpoint{4.444488in}{1.801397in}}%
\pgfpathlineto{\pgfqpoint{4.445333in}{2.451549in}}%
\pgfpathlineto{\pgfqpoint{4.446179in}{2.346328in}}%
\pgfpathlineto{\pgfqpoint{4.447870in}{1.833227in}}%
\pgfpathlineto{\pgfqpoint{4.448716in}{2.210628in}}%
\pgfpathlineto{\pgfqpoint{4.449561in}{1.802850in}}%
\pgfpathlineto{\pgfqpoint{4.450407in}{1.832304in}}%
\pgfpathlineto{\pgfqpoint{4.452098in}{2.708215in}}%
\pgfpathlineto{\pgfqpoint{4.453789in}{2.138058in}}%
\pgfpathlineto{\pgfqpoint{4.454635in}{3.011177in}}%
\pgfpathlineto{\pgfqpoint{4.455481in}{2.524754in}}%
\pgfpathlineto{\pgfqpoint{4.456326in}{2.049497in}}%
\pgfpathlineto{\pgfqpoint{4.458018in}{2.472850in}}%
\pgfpathlineto{\pgfqpoint{4.458863in}{2.010409in}}%
\pgfpathlineto{\pgfqpoint{4.459709in}{2.465646in}}%
\pgfpathlineto{\pgfqpoint{4.460554in}{2.116457in}}%
\pgfpathlineto{\pgfqpoint{4.461400in}{2.191070in}}%
\pgfpathlineto{\pgfqpoint{4.463091in}{2.594315in}}%
\pgfpathlineto{\pgfqpoint{4.463937in}{2.133110in}}%
\pgfpathlineto{\pgfqpoint{4.464783in}{2.282375in}}%
\pgfpathlineto{\pgfqpoint{4.465628in}{2.517736in}}%
\pgfpathlineto{\pgfqpoint{4.467319in}{1.932140in}}%
\pgfpathlineto{\pgfqpoint{4.468165in}{2.681768in}}%
\pgfpathlineto{\pgfqpoint{4.469856in}{1.696801in}}%
\pgfpathlineto{\pgfqpoint{4.471547in}{2.447049in}}%
\pgfpathlineto{\pgfqpoint{4.472393in}{1.998137in}}%
\pgfpathlineto{\pgfqpoint{4.473239in}{2.210126in}}%
\pgfpathlineto{\pgfqpoint{4.474084in}{1.956506in}}%
\pgfpathlineto{\pgfqpoint{4.476621in}{2.310811in}}%
\pgfpathlineto{\pgfqpoint{4.477467in}{2.240846in}}%
\pgfpathlineto{\pgfqpoint{4.478312in}{1.799634in}}%
\pgfpathlineto{\pgfqpoint{4.479158in}{1.921917in}}%
\pgfpathlineto{\pgfqpoint{4.480004in}{1.907517in}}%
\pgfpathlineto{\pgfqpoint{4.480849in}{1.846079in}}%
\pgfpathlineto{\pgfqpoint{4.483386in}{2.164333in}}%
\pgfpathlineto{\pgfqpoint{4.484232in}{3.125315in}}%
\pgfpathlineto{\pgfqpoint{4.485923in}{1.936013in}}%
\pgfpathlineto{\pgfqpoint{4.486769in}{2.002208in}}%
\pgfpathlineto{\pgfqpoint{4.487614in}{2.358371in}}%
\pgfpathlineto{\pgfqpoint{4.488460in}{1.039808in}}%
\pgfpathlineto{\pgfqpoint{4.490151in}{2.719165in}}%
\pgfpathlineto{\pgfqpoint{4.491842in}{2.052262in}}%
\pgfpathlineto{\pgfqpoint{4.492688in}{2.800797in}}%
\pgfpathlineto{\pgfqpoint{4.493534in}{2.057993in}}%
\pgfpathlineto{\pgfqpoint{4.494379in}{2.969110in}}%
\pgfpathlineto{\pgfqpoint{4.496071in}{1.877265in}}%
\pgfpathlineto{\pgfqpoint{4.497762in}{2.632496in}}%
\pgfpathlineto{\pgfqpoint{4.499453in}{1.805534in}}%
\pgfpathlineto{\pgfqpoint{4.500299in}{1.983600in}}%
\pgfpathlineto{\pgfqpoint{4.501144in}{2.816842in}}%
\pgfpathlineto{\pgfqpoint{4.501990in}{2.422633in}}%
\pgfpathlineto{\pgfqpoint{4.502836in}{2.300765in}}%
\pgfpathlineto{\pgfqpoint{4.503681in}{1.485691in}}%
\pgfpathlineto{\pgfqpoint{4.505372in}{3.262657in}}%
\pgfpathlineto{\pgfqpoint{4.507909in}{2.163236in}}%
\pgfpathlineto{\pgfqpoint{4.510446in}{1.738391in}}%
\pgfpathlineto{\pgfqpoint{4.511292in}{2.207288in}}%
\pgfpathlineto{\pgfqpoint{4.512137in}{1.652794in}}%
\pgfpathlineto{\pgfqpoint{4.512983in}{2.809179in}}%
\pgfpathlineto{\pgfqpoint{4.513829in}{2.074162in}}%
\pgfpathlineto{\pgfqpoint{4.514674in}{2.011369in}}%
\pgfpathlineto{\pgfqpoint{4.515520in}{2.530310in}}%
\pgfpathlineto{\pgfqpoint{4.516366in}{2.301389in}}%
\pgfpathlineto{\pgfqpoint{4.517211in}{2.156182in}}%
\pgfpathlineto{\pgfqpoint{4.518057in}{2.167292in}}%
\pgfpathlineto{\pgfqpoint{4.518902in}{1.476202in}}%
\pgfpathlineto{\pgfqpoint{4.519748in}{2.482587in}}%
\pgfpathlineto{\pgfqpoint{4.520594in}{2.481787in}}%
\pgfpathlineto{\pgfqpoint{4.523131in}{1.878461in}}%
\pgfpathlineto{\pgfqpoint{4.523976in}{2.881171in}}%
\pgfpathlineto{\pgfqpoint{4.524822in}{2.238314in}}%
\pgfpathlineto{\pgfqpoint{4.525667in}{1.844067in}}%
\pgfpathlineto{\pgfqpoint{4.526513in}{2.363626in}}%
\pgfpathlineto{\pgfqpoint{4.527359in}{1.775231in}}%
\pgfpathlineto{\pgfqpoint{4.528204in}{2.590528in}}%
\pgfpathlineto{\pgfqpoint{4.529050in}{1.880551in}}%
\pgfpathlineto{\pgfqpoint{4.529896in}{2.075827in}}%
\pgfpathlineto{\pgfqpoint{4.530741in}{2.066225in}}%
\pgfpathlineto{\pgfqpoint{4.531587in}{2.778822in}}%
\pgfpathlineto{\pgfqpoint{4.532432in}{2.279496in}}%
\pgfpathlineto{\pgfqpoint{4.533278in}{2.383785in}}%
\pgfpathlineto{\pgfqpoint{4.534124in}{2.812636in}}%
\pgfpathlineto{\pgfqpoint{4.534969in}{1.904932in}}%
\pgfpathlineto{\pgfqpoint{4.535815in}{2.171051in}}%
\pgfpathlineto{\pgfqpoint{4.537506in}{1.864261in}}%
\pgfpathlineto{\pgfqpoint{4.540043in}{2.878401in}}%
\pgfpathlineto{\pgfqpoint{4.541734in}{2.164106in}}%
\pgfpathlineto{\pgfqpoint{4.542580in}{2.310491in}}%
\pgfpathlineto{\pgfqpoint{4.543426in}{2.414295in}}%
\pgfpathlineto{\pgfqpoint{4.544271in}{2.333342in}}%
\pgfpathlineto{\pgfqpoint{4.546808in}{1.916239in}}%
\pgfpathlineto{\pgfqpoint{4.548499in}{2.481462in}}%
\pgfpathlineto{\pgfqpoint{4.550190in}{1.914118in}}%
\pgfpathlineto{\pgfqpoint{4.551036in}{2.706289in}}%
\pgfpathlineto{\pgfqpoint{4.551882in}{1.881600in}}%
\pgfpathlineto{\pgfqpoint{4.552727in}{2.309392in}}%
\pgfpathlineto{\pgfqpoint{4.553573in}{2.455319in}}%
\pgfpathlineto{\pgfqpoint{4.556110in}{1.827314in}}%
\pgfpathlineto{\pgfqpoint{4.558647in}{2.252292in}}%
\pgfpathlineto{\pgfqpoint{4.559492in}{2.478796in}}%
\pgfpathlineto{\pgfqpoint{4.560338in}{1.976847in}}%
\pgfpathlineto{\pgfqpoint{4.561184in}{2.370407in}}%
\pgfpathlineto{\pgfqpoint{4.562029in}{2.147811in}}%
\pgfpathlineto{\pgfqpoint{4.562875in}{2.222129in}}%
\pgfpathlineto{\pgfqpoint{4.563720in}{0.637273in}}%
\pgfpathlineto{\pgfqpoint{4.566257in}{2.517443in}}%
\pgfpathlineto{\pgfqpoint{4.567949in}{1.938362in}}%
\pgfpathlineto{\pgfqpoint{4.568794in}{2.249695in}}%
\pgfpathlineto{\pgfqpoint{4.571331in}{1.733944in}}%
\pgfpathlineto{\pgfqpoint{4.574714in}{2.904618in}}%
\pgfpathlineto{\pgfqpoint{4.578096in}{1.123854in}}%
\pgfpathlineto{\pgfqpoint{4.578942in}{1.807698in}}%
\pgfpathlineto{\pgfqpoint{4.579787in}{2.000096in}}%
\pgfpathlineto{\pgfqpoint{4.580633in}{1.744787in}}%
\pgfpathlineto{\pgfqpoint{4.581479in}{2.508014in}}%
\pgfpathlineto{\pgfqpoint{4.582324in}{2.257319in}}%
\pgfpathlineto{\pgfqpoint{4.583170in}{1.580873in}}%
\pgfpathlineto{\pgfqpoint{4.584015in}{2.499577in}}%
\pgfpathlineto{\pgfqpoint{4.584861in}{2.104691in}}%
\pgfpathlineto{\pgfqpoint{4.585707in}{2.419069in}}%
\pgfpathlineto{\pgfqpoint{4.586552in}{2.205234in}}%
\pgfpathlineto{\pgfqpoint{4.587398in}{2.418827in}}%
\pgfpathlineto{\pgfqpoint{4.588244in}{1.913400in}}%
\pgfpathlineto{\pgfqpoint{4.589089in}{2.275216in}}%
\pgfpathlineto{\pgfqpoint{4.589935in}{2.545624in}}%
\pgfpathlineto{\pgfqpoint{4.590780in}{2.390173in}}%
\pgfpathlineto{\pgfqpoint{4.591626in}{2.510642in}}%
\pgfpathlineto{\pgfqpoint{4.592472in}{2.001438in}}%
\pgfpathlineto{\pgfqpoint{4.593317in}{2.078713in}}%
\pgfpathlineto{\pgfqpoint{4.595009in}{2.683575in}}%
\pgfpathlineto{\pgfqpoint{4.596700in}{1.802882in}}%
\pgfpathlineto{\pgfqpoint{4.597545in}{2.469206in}}%
\pgfpathlineto{\pgfqpoint{4.598391in}{2.317843in}}%
\pgfpathlineto{\pgfqpoint{4.600082in}{1.379120in}}%
\pgfpathlineto{\pgfqpoint{4.600928in}{2.220680in}}%
\pgfpathlineto{\pgfqpoint{4.601774in}{1.476398in}}%
\pgfpathlineto{\pgfqpoint{4.602619in}{2.252984in}}%
\pgfpathlineto{\pgfqpoint{4.603465in}{1.709060in}}%
\pgfpathlineto{\pgfqpoint{4.605156in}{2.447483in}}%
\pgfpathlineto{\pgfqpoint{4.606002in}{2.021001in}}%
\pgfpathlineto{\pgfqpoint{4.606847in}{2.384278in}}%
\pgfpathlineto{\pgfqpoint{4.607693in}{2.366759in}}%
\pgfpathlineto{\pgfqpoint{4.611075in}{1.809665in}}%
\pgfpathlineto{\pgfqpoint{4.613612in}{2.869422in}}%
\pgfpathlineto{\pgfqpoint{4.614458in}{1.892157in}}%
\pgfpathlineto{\pgfqpoint{4.615304in}{2.028474in}}%
\pgfpathlineto{\pgfqpoint{4.616149in}{1.795504in}}%
\pgfpathlineto{\pgfqpoint{4.616995in}{2.338498in}}%
\pgfpathlineto{\pgfqpoint{4.617840in}{2.223688in}}%
\pgfpathlineto{\pgfqpoint{4.618686in}{1.627192in}}%
\pgfpathlineto{\pgfqpoint{4.619532in}{2.707790in}}%
\pgfpathlineto{\pgfqpoint{4.620377in}{2.084695in}}%
\pgfpathlineto{\pgfqpoint{4.621223in}{2.098296in}}%
\pgfpathlineto{\pgfqpoint{4.622069in}{1.163685in}}%
\pgfpathlineto{\pgfqpoint{4.622914in}{2.782475in}}%
\pgfpathlineto{\pgfqpoint{4.623760in}{2.291520in}}%
\pgfpathlineto{\pgfqpoint{4.624605in}{2.155593in}}%
\pgfpathlineto{\pgfqpoint{4.625451in}{1.751550in}}%
\pgfpathlineto{\pgfqpoint{4.627988in}{2.833651in}}%
\pgfpathlineto{\pgfqpoint{4.628833in}{1.607408in}}%
\pgfpathlineto{\pgfqpoint{4.629679in}{2.138643in}}%
\pgfpathlineto{\pgfqpoint{4.631370in}{2.244799in}}%
\pgfpathlineto{\pgfqpoint{4.632216in}{1.407261in}}%
\pgfpathlineto{\pgfqpoint{4.633062in}{1.768890in}}%
\pgfpathlineto{\pgfqpoint{4.634753in}{2.494570in}}%
\pgfpathlineto{\pgfqpoint{4.635598in}{2.158195in}}%
\pgfpathlineto{\pgfqpoint{4.637290in}{1.544267in}}%
\pgfpathlineto{\pgfqpoint{4.638135in}{2.479849in}}%
\pgfpathlineto{\pgfqpoint{4.638981in}{2.250404in}}%
\pgfpathlineto{\pgfqpoint{4.640672in}{1.786587in}}%
\pgfpathlineto{\pgfqpoint{4.641518in}{1.466707in}}%
\pgfpathlineto{\pgfqpoint{4.643209in}{2.544820in}}%
\pgfpathlineto{\pgfqpoint{4.644900in}{2.023103in}}%
\pgfpathlineto{\pgfqpoint{4.645746in}{2.707280in}}%
\pgfpathlineto{\pgfqpoint{4.647437in}{1.625838in}}%
\pgfpathlineto{\pgfqpoint{4.648283in}{1.733343in}}%
\pgfpathlineto{\pgfqpoint{4.649128in}{2.662263in}}%
\pgfpathlineto{\pgfqpoint{4.649974in}{2.141527in}}%
\pgfpathlineto{\pgfqpoint{4.650820in}{2.432747in}}%
\pgfpathlineto{\pgfqpoint{4.651665in}{2.265956in}}%
\pgfpathlineto{\pgfqpoint{4.652511in}{2.103470in}}%
\pgfpathlineto{\pgfqpoint{4.654202in}{2.585791in}}%
\pgfpathlineto{\pgfqpoint{4.655048in}{1.755032in}}%
\pgfpathlineto{\pgfqpoint{4.655893in}{2.860907in}}%
\pgfpathlineto{\pgfqpoint{4.656739in}{2.123777in}}%
\pgfpathlineto{\pgfqpoint{4.658430in}{2.825785in}}%
\pgfpathlineto{\pgfqpoint{4.659276in}{2.169545in}}%
\pgfpathlineto{\pgfqpoint{4.660122in}{2.584648in}}%
\pgfpathlineto{\pgfqpoint{4.660967in}{2.462682in}}%
\pgfpathlineto{\pgfqpoint{4.661813in}{2.489782in}}%
\pgfpathlineto{\pgfqpoint{4.662658in}{1.752373in}}%
\pgfpathlineto{\pgfqpoint{4.663504in}{2.405030in}}%
\pgfpathlineto{\pgfqpoint{4.664350in}{1.896340in}}%
\pgfpathlineto{\pgfqpoint{4.665195in}{1.659426in}}%
\pgfpathlineto{\pgfqpoint{4.666041in}{2.484756in}}%
\pgfpathlineto{\pgfqpoint{4.666887in}{2.402534in}}%
\pgfpathlineto{\pgfqpoint{4.668578in}{2.228065in}}%
\pgfpathlineto{\pgfqpoint{4.669423in}{2.370718in}}%
\pgfpathlineto{\pgfqpoint{4.670269in}{1.594563in}}%
\pgfpathlineto{\pgfqpoint{4.671115in}{2.113778in}}%
\pgfpathlineto{\pgfqpoint{4.671960in}{1.733574in}}%
\pgfpathlineto{\pgfqpoint{4.673652in}{2.474614in}}%
\pgfpathlineto{\pgfqpoint{4.674497in}{2.261771in}}%
\pgfpathlineto{\pgfqpoint{4.675343in}{2.740134in}}%
\pgfpathlineto{\pgfqpoint{4.676188in}{1.905872in}}%
\pgfpathlineto{\pgfqpoint{4.677034in}{2.194413in}}%
\pgfpathlineto{\pgfqpoint{4.677880in}{2.061870in}}%
\pgfpathlineto{\pgfqpoint{4.678725in}{2.428189in}}%
\pgfpathlineto{\pgfqpoint{4.679571in}{2.050766in}}%
\pgfpathlineto{\pgfqpoint{4.680417in}{2.229134in}}%
\pgfpathlineto{\pgfqpoint{4.681262in}{2.667176in}}%
\pgfpathlineto{\pgfqpoint{4.682953in}{1.714186in}}%
\pgfpathlineto{\pgfqpoint{4.683799in}{2.278360in}}%
\pgfpathlineto{\pgfqpoint{4.684645in}{2.103528in}}%
\pgfpathlineto{\pgfqpoint{4.685490in}{1.701259in}}%
\pgfpathlineto{\pgfqpoint{4.686336in}{2.103123in}}%
\pgfpathlineto{\pgfqpoint{4.687182in}{2.091932in}}%
\pgfpathlineto{\pgfqpoint{4.688027in}{1.807520in}}%
\pgfpathlineto{\pgfqpoint{4.689718in}{2.477422in}}%
\pgfpathlineto{\pgfqpoint{4.691410in}{2.158821in}}%
\pgfpathlineto{\pgfqpoint{4.692255in}{2.347727in}}%
\pgfpathlineto{\pgfqpoint{4.693101in}{1.469743in}}%
\pgfpathlineto{\pgfqpoint{4.693947in}{1.869236in}}%
\pgfpathlineto{\pgfqpoint{4.694792in}{2.455933in}}%
\pgfpathlineto{\pgfqpoint{4.695638in}{1.694527in}}%
\pgfpathlineto{\pgfqpoint{4.697329in}{2.570011in}}%
\pgfpathlineto{\pgfqpoint{4.699020in}{1.817693in}}%
\pgfpathlineto{\pgfqpoint{4.699866in}{2.277388in}}%
\pgfpathlineto{\pgfqpoint{4.700712in}{1.884978in}}%
\pgfpathlineto{\pgfqpoint{4.702403in}{2.426600in}}%
\pgfpathlineto{\pgfqpoint{4.703248in}{1.519549in}}%
\pgfpathlineto{\pgfqpoint{4.704094in}{2.315491in}}%
\pgfpathlineto{\pgfqpoint{4.704940in}{2.138588in}}%
\pgfpathlineto{\pgfqpoint{4.705785in}{2.077553in}}%
\pgfpathlineto{\pgfqpoint{4.709168in}{2.997993in}}%
\pgfpathlineto{\pgfqpoint{4.710013in}{1.445162in}}%
\pgfpathlineto{\pgfqpoint{4.710859in}{2.240685in}}%
\pgfpathlineto{\pgfqpoint{4.711705in}{2.163373in}}%
\pgfpathlineto{\pgfqpoint{4.712550in}{2.560640in}}%
\pgfpathlineto{\pgfqpoint{4.713396in}{2.341675in}}%
\pgfpathlineto{\pgfqpoint{4.715087in}{1.827699in}}%
\pgfpathlineto{\pgfqpoint{4.715933in}{2.031352in}}%
\pgfpathlineto{\pgfqpoint{4.716778in}{2.161360in}}%
\pgfpathlineto{\pgfqpoint{4.717624in}{2.038610in}}%
\pgfpathlineto{\pgfqpoint{4.718470in}{1.442229in}}%
\pgfpathlineto{\pgfqpoint{4.719315in}{1.907975in}}%
\pgfpathlineto{\pgfqpoint{4.721852in}{2.811698in}}%
\pgfpathlineto{\pgfqpoint{4.722698in}{1.623757in}}%
\pgfpathlineto{\pgfqpoint{4.723543in}{1.898274in}}%
\pgfpathlineto{\pgfqpoint{4.724389in}{2.357318in}}%
\pgfpathlineto{\pgfqpoint{4.725235in}{1.637551in}}%
\pgfpathlineto{\pgfqpoint{4.726080in}{2.229846in}}%
\pgfpathlineto{\pgfqpoint{4.726926in}{2.143772in}}%
\pgfpathlineto{\pgfqpoint{4.727771in}{1.527308in}}%
\pgfpathlineto{\pgfqpoint{4.728617in}{2.478406in}}%
\pgfpathlineto{\pgfqpoint{4.729463in}{1.650806in}}%
\pgfpathlineto{\pgfqpoint{4.730308in}{2.916984in}}%
\pgfpathlineto{\pgfqpoint{4.731154in}{2.261700in}}%
\pgfpathlineto{\pgfqpoint{4.732000in}{1.792473in}}%
\pgfpathlineto{\pgfqpoint{4.732845in}{2.670367in}}%
\pgfpathlineto{\pgfqpoint{4.733691in}{1.922955in}}%
\pgfpathlineto{\pgfqpoint{4.734536in}{2.274951in}}%
\pgfpathlineto{\pgfqpoint{4.735382in}{2.072811in}}%
\pgfpathlineto{\pgfqpoint{4.736228in}{2.084478in}}%
\pgfpathlineto{\pgfqpoint{4.737073in}{2.063962in}}%
\pgfpathlineto{\pgfqpoint{4.737919in}{2.606392in}}%
\pgfpathlineto{\pgfqpoint{4.738765in}{2.244244in}}%
\pgfpathlineto{\pgfqpoint{4.739610in}{2.418616in}}%
\pgfpathlineto{\pgfqpoint{4.740456in}{1.981291in}}%
\pgfpathlineto{\pgfqpoint{4.741301in}{2.547668in}}%
\pgfpathlineto{\pgfqpoint{4.742147in}{2.115748in}}%
\pgfpathlineto{\pgfqpoint{4.742993in}{2.562158in}}%
\pgfpathlineto{\pgfqpoint{4.744684in}{1.881601in}}%
\pgfpathlineto{\pgfqpoint{4.745530in}{2.684919in}}%
\pgfpathlineto{\pgfqpoint{4.746375in}{2.113206in}}%
\pgfpathlineto{\pgfqpoint{4.747221in}{2.510363in}}%
\pgfpathlineto{\pgfqpoint{4.748066in}{1.680955in}}%
\pgfpathlineto{\pgfqpoint{4.748912in}{1.808335in}}%
\pgfpathlineto{\pgfqpoint{4.749758in}{1.989910in}}%
\pgfpathlineto{\pgfqpoint{4.750603in}{3.154451in}}%
\pgfpathlineto{\pgfqpoint{4.751449in}{1.864109in}}%
\pgfpathlineto{\pgfqpoint{4.752295in}{2.863564in}}%
\pgfpathlineto{\pgfqpoint{4.753140in}{2.250599in}}%
\pgfpathlineto{\pgfqpoint{4.753986in}{2.710461in}}%
\pgfpathlineto{\pgfqpoint{4.755677in}{2.008590in}}%
\pgfpathlineto{\pgfqpoint{4.756523in}{2.535651in}}%
\pgfpathlineto{\pgfqpoint{4.757368in}{2.209400in}}%
\pgfpathlineto{\pgfqpoint{4.758214in}{2.224960in}}%
\pgfpathlineto{\pgfqpoint{4.759060in}{2.564542in}}%
\pgfpathlineto{\pgfqpoint{4.759905in}{1.891769in}}%
\pgfpathlineto{\pgfqpoint{4.760751in}{1.897125in}}%
\pgfpathlineto{\pgfqpoint{4.761596in}{2.272571in}}%
\pgfpathlineto{\pgfqpoint{4.762442in}{2.218652in}}%
\pgfpathlineto{\pgfqpoint{4.763288in}{1.566921in}}%
\pgfpathlineto{\pgfqpoint{4.764133in}{2.481576in}}%
\pgfpathlineto{\pgfqpoint{4.764979in}{2.381097in}}%
\pgfpathlineto{\pgfqpoint{4.765825in}{2.354250in}}%
\pgfpathlineto{\pgfqpoint{4.766670in}{2.108504in}}%
\pgfpathlineto{\pgfqpoint{4.767516in}{2.399839in}}%
\pgfpathlineto{\pgfqpoint{4.768361in}{2.314977in}}%
\pgfpathlineto{\pgfqpoint{4.769207in}{1.723088in}}%
\pgfpathlineto{\pgfqpoint{4.770053in}{2.092485in}}%
\pgfpathlineto{\pgfqpoint{4.771744in}{1.791137in}}%
\pgfpathlineto{\pgfqpoint{4.772590in}{2.697795in}}%
\pgfpathlineto{\pgfqpoint{4.773435in}{2.435579in}}%
\pgfpathlineto{\pgfqpoint{4.774281in}{2.741097in}}%
\pgfpathlineto{\pgfqpoint{4.775972in}{1.842325in}}%
\pgfpathlineto{\pgfqpoint{4.776818in}{2.038100in}}%
\pgfpathlineto{\pgfqpoint{4.777663in}{1.869937in}}%
\pgfpathlineto{\pgfqpoint{4.778509in}{2.406595in}}%
\pgfpathlineto{\pgfqpoint{4.779355in}{1.799999in}}%
\pgfpathlineto{\pgfqpoint{4.780200in}{2.144308in}}%
\pgfpathlineto{\pgfqpoint{4.781891in}{2.506407in}}%
\pgfpathlineto{\pgfqpoint{4.783583in}{2.025942in}}%
\pgfpathlineto{\pgfqpoint{4.784428in}{2.141988in}}%
\pgfpathlineto{\pgfqpoint{4.786119in}{2.508658in}}%
\pgfpathlineto{\pgfqpoint{4.788656in}{1.931586in}}%
\pgfpathlineto{\pgfqpoint{4.790348in}{2.159890in}}%
\pgfpathlineto{\pgfqpoint{4.791193in}{2.083484in}}%
\pgfpathlineto{\pgfqpoint{4.792039in}{2.406942in}}%
\pgfpathlineto{\pgfqpoint{4.792884in}{1.821408in}}%
\pgfpathlineto{\pgfqpoint{4.793730in}{2.348043in}}%
\pgfpathlineto{\pgfqpoint{4.794576in}{2.135322in}}%
\pgfpathlineto{\pgfqpoint{4.795421in}{1.921059in}}%
\pgfpathlineto{\pgfqpoint{4.797113in}{2.473971in}}%
\pgfpathlineto{\pgfqpoint{4.799649in}{1.886558in}}%
\pgfpathlineto{\pgfqpoint{4.802186in}{2.332132in}}%
\pgfpathlineto{\pgfqpoint{4.803032in}{2.377298in}}%
\pgfpathlineto{\pgfqpoint{4.805569in}{1.743253in}}%
\pgfpathlineto{\pgfqpoint{4.806414in}{2.359369in}}%
\pgfpathlineto{\pgfqpoint{4.807260in}{1.943625in}}%
\pgfpathlineto{\pgfqpoint{4.808106in}{1.481293in}}%
\pgfpathlineto{\pgfqpoint{4.808951in}{2.275939in}}%
\pgfpathlineto{\pgfqpoint{4.809797in}{2.010702in}}%
\pgfpathlineto{\pgfqpoint{4.810643in}{2.001151in}}%
\pgfpathlineto{\pgfqpoint{4.812334in}{2.914720in}}%
\pgfpathlineto{\pgfqpoint{4.813179in}{2.140740in}}%
\pgfpathlineto{\pgfqpoint{4.814025in}{2.392523in}}%
\pgfpathlineto{\pgfqpoint{4.815716in}{2.955412in}}%
\pgfpathlineto{\pgfqpoint{4.817408in}{2.005804in}}%
\pgfpathlineto{\pgfqpoint{4.818253in}{2.664332in}}%
\pgfpathlineto{\pgfqpoint{4.819099in}{1.746030in}}%
\pgfpathlineto{\pgfqpoint{4.820790in}{3.042284in}}%
\pgfpathlineto{\pgfqpoint{4.822481in}{2.172340in}}%
\pgfpathlineto{\pgfqpoint{4.823327in}{2.915689in}}%
\pgfpathlineto{\pgfqpoint{4.824173in}{2.328497in}}%
\pgfpathlineto{\pgfqpoint{4.825018in}{2.639314in}}%
\pgfpathlineto{\pgfqpoint{4.827555in}{1.935171in}}%
\pgfpathlineto{\pgfqpoint{4.828401in}{2.382731in}}%
\pgfpathlineto{\pgfqpoint{4.829246in}{2.375266in}}%
\pgfpathlineto{\pgfqpoint{4.830092in}{2.392493in}}%
\pgfpathlineto{\pgfqpoint{4.832629in}{1.813565in}}%
\pgfpathlineto{\pgfqpoint{4.834320in}{2.266726in}}%
\pgfpathlineto{\pgfqpoint{4.835166in}{1.728146in}}%
\pgfpathlineto{\pgfqpoint{4.837703in}{3.382727in}}%
\pgfpathlineto{\pgfqpoint{4.840239in}{2.014612in}}%
\pgfpathlineto{\pgfqpoint{4.841085in}{1.289024in}}%
\pgfpathlineto{\pgfqpoint{4.843622in}{2.113215in}}%
\pgfpathlineto{\pgfqpoint{4.844468in}{2.420573in}}%
\pgfpathlineto{\pgfqpoint{4.845313in}{1.999955in}}%
\pgfpathlineto{\pgfqpoint{4.847004in}{2.717454in}}%
\pgfpathlineto{\pgfqpoint{4.848696in}{1.995734in}}%
\pgfpathlineto{\pgfqpoint{4.849541in}{2.647637in}}%
\pgfpathlineto{\pgfqpoint{4.850387in}{2.256959in}}%
\pgfpathlineto{\pgfqpoint{4.851233in}{2.650934in}}%
\pgfpathlineto{\pgfqpoint{4.852078in}{2.241089in}}%
\pgfpathlineto{\pgfqpoint{4.852924in}{2.327571in}}%
\pgfpathlineto{\pgfqpoint{4.853769in}{2.651492in}}%
\pgfpathlineto{\pgfqpoint{4.854615in}{2.389024in}}%
\pgfpathlineto{\pgfqpoint{4.856306in}{1.343118in}}%
\pgfpathlineto{\pgfqpoint{4.857998in}{2.392370in}}%
\pgfpathlineto{\pgfqpoint{4.858843in}{2.417115in}}%
\pgfpathlineto{\pgfqpoint{4.859689in}{2.899461in}}%
\pgfpathlineto{\pgfqpoint{4.861380in}{1.742472in}}%
\pgfpathlineto{\pgfqpoint{4.863917in}{2.198385in}}%
\pgfpathlineto{\pgfqpoint{4.864762in}{3.013147in}}%
\pgfpathlineto{\pgfqpoint{4.865608in}{1.927646in}}%
\pgfpathlineto{\pgfqpoint{4.866454in}{2.316958in}}%
\pgfpathlineto{\pgfqpoint{4.867299in}{1.988937in}}%
\pgfpathlineto{\pgfqpoint{4.868145in}{2.207854in}}%
\pgfpathlineto{\pgfqpoint{4.869836in}{1.894613in}}%
\pgfpathlineto{\pgfqpoint{4.870682in}{1.989100in}}%
\pgfpathlineto{\pgfqpoint{4.871527in}{1.771871in}}%
\pgfpathlineto{\pgfqpoint{4.873219in}{2.137024in}}%
\pgfpathlineto{\pgfqpoint{4.874064in}{2.109147in}}%
\pgfpathlineto{\pgfqpoint{4.874910in}{1.738269in}}%
\pgfpathlineto{\pgfqpoint{4.876601in}{2.337563in}}%
\pgfpathlineto{\pgfqpoint{4.877447in}{1.821872in}}%
\pgfpathlineto{\pgfqpoint{4.878292in}{2.959349in}}%
\pgfpathlineto{\pgfqpoint{4.879138in}{2.441358in}}%
\pgfpathlineto{\pgfqpoint{4.879984in}{2.629791in}}%
\pgfpathlineto{\pgfqpoint{4.881675in}{2.007274in}}%
\pgfpathlineto{\pgfqpoint{4.882521in}{2.510473in}}%
\pgfpathlineto{\pgfqpoint{4.884212in}{1.686353in}}%
\pgfpathlineto{\pgfqpoint{4.885057in}{2.521723in}}%
\pgfpathlineto{\pgfqpoint{4.885903in}{2.449857in}}%
\pgfpathlineto{\pgfqpoint{4.886749in}{2.371205in}}%
\pgfpathlineto{\pgfqpoint{4.889286in}{1.987320in}}%
\pgfpathlineto{\pgfqpoint{4.890131in}{2.771314in}}%
\pgfpathlineto{\pgfqpoint{4.892668in}{1.634870in}}%
\pgfpathlineto{\pgfqpoint{4.893514in}{2.402970in}}%
\pgfpathlineto{\pgfqpoint{4.894359in}{2.086178in}}%
\pgfpathlineto{\pgfqpoint{4.895205in}{2.331980in}}%
\pgfpathlineto{\pgfqpoint{4.896051in}{2.162481in}}%
\pgfpathlineto{\pgfqpoint{4.896896in}{2.224068in}}%
\pgfpathlineto{\pgfqpoint{4.897742in}{2.011277in}}%
\pgfpathlineto{\pgfqpoint{4.898587in}{2.459574in}}%
\pgfpathlineto{\pgfqpoint{4.900279in}{1.713936in}}%
\pgfpathlineto{\pgfqpoint{4.901124in}{1.759380in}}%
\pgfpathlineto{\pgfqpoint{4.901970in}{1.911202in}}%
\pgfpathlineto{\pgfqpoint{4.903661in}{1.727307in}}%
\pgfpathlineto{\pgfqpoint{4.907044in}{2.581465in}}%
\pgfpathlineto{\pgfqpoint{4.907889in}{1.923621in}}%
\pgfpathlineto{\pgfqpoint{4.908735in}{2.249155in}}%
\pgfpathlineto{\pgfqpoint{4.909581in}{2.272955in}}%
\pgfpathlineto{\pgfqpoint{4.911272in}{2.572541in}}%
\pgfpathlineto{\pgfqpoint{4.912117in}{2.950945in}}%
\pgfpathlineto{\pgfqpoint{4.912963in}{1.407420in}}%
\pgfpathlineto{\pgfqpoint{4.913809in}{1.744950in}}%
\pgfpathlineto{\pgfqpoint{4.916346in}{2.174118in}}%
\pgfpathlineto{\pgfqpoint{4.917191in}{1.423932in}}%
\pgfpathlineto{\pgfqpoint{4.918037in}{2.084158in}}%
\pgfpathlineto{\pgfqpoint{4.918882in}{1.802891in}}%
\pgfpathlineto{\pgfqpoint{4.921419in}{2.325424in}}%
\pgfpathlineto{\pgfqpoint{4.922265in}{2.442776in}}%
\pgfpathlineto{\pgfqpoint{4.923956in}{1.546997in}}%
\pgfpathlineto{\pgfqpoint{4.924802in}{2.515619in}}%
\pgfpathlineto{\pgfqpoint{4.925647in}{2.059005in}}%
\pgfpathlineto{\pgfqpoint{4.926493in}{1.819222in}}%
\pgfpathlineto{\pgfqpoint{4.927339in}{2.382928in}}%
\pgfpathlineto{\pgfqpoint{4.928184in}{2.053829in}}%
\pgfpathlineto{\pgfqpoint{4.929030in}{1.923624in}}%
\pgfpathlineto{\pgfqpoint{4.929876in}{3.349668in}}%
\pgfpathlineto{\pgfqpoint{4.930721in}{2.249950in}}%
\pgfpathlineto{\pgfqpoint{4.931567in}{2.446262in}}%
\pgfpathlineto{\pgfqpoint{4.932412in}{1.941141in}}%
\pgfpathlineto{\pgfqpoint{4.933258in}{2.545330in}}%
\pgfpathlineto{\pgfqpoint{4.934104in}{2.450511in}}%
\pgfpathlineto{\pgfqpoint{4.937486in}{1.913855in}}%
\pgfpathlineto{\pgfqpoint{4.940023in}{2.369337in}}%
\pgfpathlineto{\pgfqpoint{4.943405in}{1.699430in}}%
\pgfpathlineto{\pgfqpoint{4.944251in}{2.590969in}}%
\pgfpathlineto{\pgfqpoint{4.945097in}{2.014115in}}%
\pgfpathlineto{\pgfqpoint{4.945942in}{1.950048in}}%
\pgfpathlineto{\pgfqpoint{4.946788in}{1.966205in}}%
\pgfpathlineto{\pgfqpoint{4.948479in}{2.250308in}}%
\pgfpathlineto{\pgfqpoint{4.949325in}{1.875047in}}%
\pgfpathlineto{\pgfqpoint{4.951016in}{2.793825in}}%
\pgfpathlineto{\pgfqpoint{4.952707in}{2.136080in}}%
\pgfpathlineto{\pgfqpoint{4.953553in}{2.269163in}}%
\pgfpathlineto{\pgfqpoint{4.954399in}{1.653809in}}%
\pgfpathlineto{\pgfqpoint{4.955244in}{2.708707in}}%
\pgfpathlineto{\pgfqpoint{4.956090in}{1.740341in}}%
\pgfpathlineto{\pgfqpoint{4.956935in}{2.538978in}}%
\pgfpathlineto{\pgfqpoint{4.957781in}{1.791195in}}%
\pgfpathlineto{\pgfqpoint{4.960318in}{3.047166in}}%
\pgfpathlineto{\pgfqpoint{4.961164in}{2.007457in}}%
\pgfpathlineto{\pgfqpoint{4.962009in}{2.265991in}}%
\pgfpathlineto{\pgfqpoint{4.963700in}{2.924542in}}%
\pgfpathlineto{\pgfqpoint{4.965392in}{1.733126in}}%
\pgfpathlineto{\pgfqpoint{4.966237in}{2.014816in}}%
\pgfpathlineto{\pgfqpoint{4.967083in}{2.662402in}}%
\pgfpathlineto{\pgfqpoint{4.967929in}{2.550748in}}%
\pgfpathlineto{\pgfqpoint{4.968774in}{1.729964in}}%
\pgfpathlineto{\pgfqpoint{4.969620in}{2.000246in}}%
\pgfpathlineto{\pgfqpoint{4.970465in}{1.740305in}}%
\pgfpathlineto{\pgfqpoint{4.971311in}{1.851427in}}%
\pgfpathlineto{\pgfqpoint{4.973002in}{1.657070in}}%
\pgfpathlineto{\pgfqpoint{4.973848in}{2.423297in}}%
\pgfpathlineto{\pgfqpoint{4.974694in}{1.763310in}}%
\pgfpathlineto{\pgfqpoint{4.975539in}{2.190379in}}%
\pgfpathlineto{\pgfqpoint{4.976385in}{2.548827in}}%
\pgfpathlineto{\pgfqpoint{4.978076in}{2.136264in}}%
\pgfpathlineto{\pgfqpoint{4.979767in}{2.666956in}}%
\pgfpathlineto{\pgfqpoint{4.980613in}{2.372328in}}%
\pgfpathlineto{\pgfqpoint{4.981459in}{2.387356in}}%
\pgfpathlineto{\pgfqpoint{4.983150in}{1.682510in}}%
\pgfpathlineto{\pgfqpoint{4.984841in}{2.426000in}}%
\pgfpathlineto{\pgfqpoint{4.985687in}{2.248068in}}%
\pgfpathlineto{\pgfqpoint{4.986532in}{2.023499in}}%
\pgfpathlineto{\pgfqpoint{4.988224in}{2.785550in}}%
\pgfpathlineto{\pgfqpoint{4.991606in}{1.541745in}}%
\pgfpathlineto{\pgfqpoint{4.992452in}{3.109854in}}%
\pgfpathlineto{\pgfqpoint{4.993297in}{2.315550in}}%
\pgfpathlineto{\pgfqpoint{4.994989in}{1.677373in}}%
\pgfpathlineto{\pgfqpoint{4.997525in}{2.132278in}}%
\pgfpathlineto{\pgfqpoint{4.998371in}{2.076622in}}%
\pgfpathlineto{\pgfqpoint{5.000062in}{2.965875in}}%
\pgfpathlineto{\pgfqpoint{5.001754in}{1.995420in}}%
\pgfpathlineto{\pgfqpoint{5.002599in}{2.070584in}}%
\pgfpathlineto{\pgfqpoint{5.003445in}{2.563852in}}%
\pgfpathlineto{\pgfqpoint{5.004290in}{2.126508in}}%
\pgfpathlineto{\pgfqpoint{5.005136in}{2.249660in}}%
\pgfpathlineto{\pgfqpoint{5.005982in}{2.841706in}}%
\pgfpathlineto{\pgfqpoint{5.006827in}{2.124108in}}%
\pgfpathlineto{\pgfqpoint{5.007673in}{2.838012in}}%
\pgfpathlineto{\pgfqpoint{5.009364in}{1.915865in}}%
\pgfpathlineto{\pgfqpoint{5.010210in}{2.909932in}}%
\pgfpathlineto{\pgfqpoint{5.011055in}{1.997192in}}%
\pgfpathlineto{\pgfqpoint{5.011901in}{2.059988in}}%
\pgfpathlineto{\pgfqpoint{5.013592in}{2.644968in}}%
\pgfpathlineto{\pgfqpoint{5.014438in}{1.860890in}}%
\pgfpathlineto{\pgfqpoint{5.015284in}{2.050467in}}%
\pgfpathlineto{\pgfqpoint{5.016129in}{1.906097in}}%
\pgfpathlineto{\pgfqpoint{5.018666in}{2.663016in}}%
\pgfpathlineto{\pgfqpoint{5.019512in}{2.460184in}}%
\pgfpathlineto{\pgfqpoint{5.020357in}{2.179288in}}%
\pgfpathlineto{\pgfqpoint{5.021203in}{2.522436in}}%
\pgfpathlineto{\pgfqpoint{5.022049in}{2.255558in}}%
\pgfpathlineto{\pgfqpoint{5.023740in}{2.457606in}}%
\pgfpathlineto{\pgfqpoint{5.024585in}{1.887752in}}%
\pgfpathlineto{\pgfqpoint{5.025431in}{1.933306in}}%
\pgfpathlineto{\pgfqpoint{5.027122in}{2.572331in}}%
\pgfpathlineto{\pgfqpoint{5.028813in}{2.008789in}}%
\pgfpathlineto{\pgfqpoint{5.029659in}{2.724453in}}%
\pgfpathlineto{\pgfqpoint{5.030505in}{2.203824in}}%
\pgfpathlineto{\pgfqpoint{5.031350in}{2.249786in}}%
\pgfpathlineto{\pgfqpoint{5.032196in}{2.228097in}}%
\pgfpathlineto{\pgfqpoint{5.033042in}{2.675638in}}%
\pgfpathlineto{\pgfqpoint{5.033887in}{2.427190in}}%
\pgfpathlineto{\pgfqpoint{5.034733in}{1.907513in}}%
\pgfpathlineto{\pgfqpoint{5.035578in}{1.939135in}}%
\pgfpathlineto{\pgfqpoint{5.037270in}{2.530756in}}%
\pgfpathlineto{\pgfqpoint{5.038115in}{1.796027in}}%
\pgfpathlineto{\pgfqpoint{5.039807in}{3.167098in}}%
\pgfpathlineto{\pgfqpoint{5.040652in}{1.937568in}}%
\pgfpathlineto{\pgfqpoint{5.041498in}{2.392238in}}%
\pgfpathlineto{\pgfqpoint{5.042343in}{2.388153in}}%
\pgfpathlineto{\pgfqpoint{5.043189in}{2.203625in}}%
\pgfpathlineto{\pgfqpoint{5.044035in}{2.253247in}}%
\pgfpathlineto{\pgfqpoint{5.044880in}{2.252820in}}%
\pgfpathlineto{\pgfqpoint{5.045726in}{2.594435in}}%
\pgfpathlineto{\pgfqpoint{5.046572in}{2.371999in}}%
\pgfpathlineto{\pgfqpoint{5.048263in}{1.722963in}}%
\pgfpathlineto{\pgfqpoint{5.049954in}{2.581747in}}%
\pgfpathlineto{\pgfqpoint{5.050800in}{2.000780in}}%
\pgfpathlineto{\pgfqpoint{5.051645in}{2.385589in}}%
\pgfpathlineto{\pgfqpoint{5.052491in}{2.583058in}}%
\pgfpathlineto{\pgfqpoint{5.053337in}{1.780333in}}%
\pgfpathlineto{\pgfqpoint{5.054182in}{1.936099in}}%
\pgfpathlineto{\pgfqpoint{5.056719in}{2.194045in}}%
\pgfpathlineto{\pgfqpoint{5.057565in}{1.945733in}}%
\pgfpathlineto{\pgfqpoint{5.059256in}{2.916420in}}%
\pgfpathlineto{\pgfqpoint{5.060102in}{1.728681in}}%
\pgfpathlineto{\pgfqpoint{5.060947in}{2.410642in}}%
\pgfpathlineto{\pgfqpoint{5.061793in}{1.812472in}}%
\pgfpathlineto{\pgfqpoint{5.062638in}{2.845865in}}%
\pgfpathlineto{\pgfqpoint{5.064330in}{1.146101in}}%
\pgfpathlineto{\pgfqpoint{5.065175in}{2.460138in}}%
\pgfpathlineto{\pgfqpoint{5.066021in}{1.726296in}}%
\pgfpathlineto{\pgfqpoint{5.066867in}{2.438466in}}%
\pgfpathlineto{\pgfqpoint{5.067712in}{2.188963in}}%
\pgfpathlineto{\pgfqpoint{5.068558in}{1.859790in}}%
\pgfpathlineto{\pgfqpoint{5.069403in}{2.013680in}}%
\pgfpathlineto{\pgfqpoint{5.070249in}{2.338153in}}%
\pgfpathlineto{\pgfqpoint{5.071095in}{2.261515in}}%
\pgfpathlineto{\pgfqpoint{5.071940in}{2.171341in}}%
\pgfpathlineto{\pgfqpoint{5.073632in}{2.594627in}}%
\pgfpathlineto{\pgfqpoint{5.074477in}{2.529270in}}%
\pgfpathlineto{\pgfqpoint{5.075323in}{2.001497in}}%
\pgfpathlineto{\pgfqpoint{5.076168in}{2.388937in}}%
\pgfpathlineto{\pgfqpoint{5.078705in}{1.810005in}}%
\pgfpathlineto{\pgfqpoint{5.080397in}{2.763530in}}%
\pgfpathlineto{\pgfqpoint{5.081242in}{2.387761in}}%
\pgfpathlineto{\pgfqpoint{5.082088in}{3.174285in}}%
\pgfpathlineto{\pgfqpoint{5.083779in}{2.063099in}}%
\pgfpathlineto{\pgfqpoint{5.084625in}{2.592147in}}%
\pgfpathlineto{\pgfqpoint{5.085470in}{2.005807in}}%
\pgfpathlineto{\pgfqpoint{5.086316in}{2.120876in}}%
\pgfpathlineto{\pgfqpoint{5.087162in}{2.511572in}}%
\pgfpathlineto{\pgfqpoint{5.088007in}{1.938779in}}%
\pgfpathlineto{\pgfqpoint{5.088853in}{2.265840in}}%
\pgfpathlineto{\pgfqpoint{5.089698in}{3.178699in}}%
\pgfpathlineto{\pgfqpoint{5.090544in}{2.459201in}}%
\pgfpathlineto{\pgfqpoint{5.091390in}{2.646001in}}%
\pgfpathlineto{\pgfqpoint{5.093081in}{2.049486in}}%
\pgfpathlineto{\pgfqpoint{5.093927in}{2.779054in}}%
\pgfpathlineto{\pgfqpoint{5.095618in}{1.548424in}}%
\pgfpathlineto{\pgfqpoint{5.096463in}{1.750488in}}%
\pgfpathlineto{\pgfqpoint{5.099000in}{2.644259in}}%
\pgfpathlineto{\pgfqpoint{5.099846in}{2.642835in}}%
\pgfpathlineto{\pgfqpoint{5.101537in}{1.800703in}}%
\pgfpathlineto{\pgfqpoint{5.103228in}{2.840304in}}%
\pgfpathlineto{\pgfqpoint{5.104074in}{2.379280in}}%
\pgfpathlineto{\pgfqpoint{5.106611in}{1.953870in}}%
\pgfpathlineto{\pgfqpoint{5.107456in}{2.077251in}}%
\pgfpathlineto{\pgfqpoint{5.108302in}{2.656302in}}%
\pgfpathlineto{\pgfqpoint{5.109148in}{2.649615in}}%
\pgfpathlineto{\pgfqpoint{5.109993in}{2.583413in}}%
\pgfpathlineto{\pgfqpoint{5.110839in}{3.110465in}}%
\pgfpathlineto{\pgfqpoint{5.112530in}{1.803427in}}%
\pgfpathlineto{\pgfqpoint{5.113376in}{2.593901in}}%
\pgfpathlineto{\pgfqpoint{5.114221in}{1.800303in}}%
\pgfpathlineto{\pgfqpoint{5.115067in}{2.446441in}}%
\pgfpathlineto{\pgfqpoint{5.115913in}{2.523408in}}%
\pgfpathlineto{\pgfqpoint{5.116758in}{1.935602in}}%
\pgfpathlineto{\pgfqpoint{5.117604in}{1.940720in}}%
\pgfpathlineto{\pgfqpoint{5.121832in}{2.585940in}}%
\pgfpathlineto{\pgfqpoint{5.122678in}{2.185549in}}%
\pgfpathlineto{\pgfqpoint{5.123523in}{2.636361in}}%
\pgfpathlineto{\pgfqpoint{5.126060in}{1.747439in}}%
\pgfpathlineto{\pgfqpoint{5.126906in}{1.810081in}}%
\pgfpathlineto{\pgfqpoint{5.127751in}{1.730820in}}%
\pgfpathlineto{\pgfqpoint{5.128597in}{2.259784in}}%
\pgfpathlineto{\pgfqpoint{5.129443in}{2.007356in}}%
\pgfpathlineto{\pgfqpoint{5.130288in}{2.317981in}}%
\pgfpathlineto{\pgfqpoint{5.131134in}{1.805837in}}%
\pgfpathlineto{\pgfqpoint{5.131980in}{2.192378in}}%
\pgfpathlineto{\pgfqpoint{5.132825in}{1.987011in}}%
\pgfpathlineto{\pgfqpoint{5.133671in}{2.647569in}}%
\pgfpathlineto{\pgfqpoint{5.134516in}{2.338339in}}%
\pgfpathlineto{\pgfqpoint{5.135362in}{1.958636in}}%
\pgfpathlineto{\pgfqpoint{5.136208in}{2.379400in}}%
\pgfpathlineto{\pgfqpoint{5.137053in}{1.479770in}}%
\pgfpathlineto{\pgfqpoint{5.137899in}{2.136131in}}%
\pgfpathlineto{\pgfqpoint{5.138745in}{2.255633in}}%
\pgfpathlineto{\pgfqpoint{5.139590in}{2.184910in}}%
\pgfpathlineto{\pgfqpoint{5.142127in}{1.649963in}}%
\pgfpathlineto{\pgfqpoint{5.142973in}{2.520505in}}%
\pgfpathlineto{\pgfqpoint{5.143818in}{2.051537in}}%
\pgfpathlineto{\pgfqpoint{5.144664in}{2.227648in}}%
\pgfpathlineto{\pgfqpoint{5.145510in}{1.933398in}}%
\pgfpathlineto{\pgfqpoint{5.146355in}{2.856199in}}%
\pgfpathlineto{\pgfqpoint{5.147201in}{1.866496in}}%
\pgfpathlineto{\pgfqpoint{5.148046in}{2.364310in}}%
\pgfpathlineto{\pgfqpoint{5.148892in}{1.867099in}}%
\pgfpathlineto{\pgfqpoint{5.149738in}{2.395471in}}%
\pgfpathlineto{\pgfqpoint{5.152275in}{1.491378in}}%
\pgfpathlineto{\pgfqpoint{5.154811in}{2.451478in}}%
\pgfpathlineto{\pgfqpoint{5.156503in}{2.062207in}}%
\pgfpathlineto{\pgfqpoint{5.157348in}{2.694968in}}%
\pgfpathlineto{\pgfqpoint{5.158194in}{2.336939in}}%
\pgfpathlineto{\pgfqpoint{5.159040in}{2.657746in}}%
\pgfpathlineto{\pgfqpoint{5.159885in}{1.690747in}}%
\pgfpathlineto{\pgfqpoint{5.160731in}{2.722170in}}%
\pgfpathlineto{\pgfqpoint{5.161576in}{1.946218in}}%
\pgfpathlineto{\pgfqpoint{5.163268in}{2.266570in}}%
\pgfpathlineto{\pgfqpoint{5.164959in}{2.260185in}}%
\pgfpathlineto{\pgfqpoint{5.165805in}{2.142728in}}%
\pgfpathlineto{\pgfqpoint{5.166650in}{2.200899in}}%
\pgfpathlineto{\pgfqpoint{5.167496in}{2.517628in}}%
\pgfpathlineto{\pgfqpoint{5.168341in}{2.268778in}}%
\pgfpathlineto{\pgfqpoint{5.169187in}{1.997256in}}%
\pgfpathlineto{\pgfqpoint{5.170033in}{2.461007in}}%
\pgfpathlineto{\pgfqpoint{5.170878in}{1.549542in}}%
\pgfpathlineto{\pgfqpoint{5.171724in}{2.249926in}}%
\pgfpathlineto{\pgfqpoint{5.173415in}{1.402794in}}%
\pgfpathlineto{\pgfqpoint{5.175952in}{2.525631in}}%
\pgfpathlineto{\pgfqpoint{5.177643in}{1.804531in}}%
\pgfpathlineto{\pgfqpoint{5.178489in}{2.192240in}}%
\pgfpathlineto{\pgfqpoint{5.179335in}{1.423467in}}%
\pgfpathlineto{\pgfqpoint{5.180180in}{2.008651in}}%
\pgfpathlineto{\pgfqpoint{5.181026in}{2.123785in}}%
\pgfpathlineto{\pgfqpoint{5.181871in}{1.869778in}}%
\pgfpathlineto{\pgfqpoint{5.182717in}{2.003574in}}%
\pgfpathlineto{\pgfqpoint{5.184408in}{2.647892in}}%
\pgfpathlineto{\pgfqpoint{5.185254in}{2.328360in}}%
\pgfpathlineto{\pgfqpoint{5.186099in}{2.031914in}}%
\pgfpathlineto{\pgfqpoint{5.187791in}{2.517076in}}%
\pgfpathlineto{\pgfqpoint{5.188636in}{1.937737in}}%
\pgfpathlineto{\pgfqpoint{5.188636in}{1.937737in}}%
\pgfusepath{stroke}%
\end{pgfscope}%
\begin{pgfscope}%
\pgfsetrectcap%
\pgfsetmiterjoin%
\pgfsetlinewidth{0.803000pt}%
\definecolor{currentstroke}{rgb}{0.000000,0.000000,0.000000}%
\pgfsetstrokecolor{currentstroke}%
\pgfsetdash{}{0pt}%
\pgfpathmoveto{\pgfqpoint{0.750000in}{0.500000in}}%
\pgfpathlineto{\pgfqpoint{0.750000in}{3.520000in}}%
\pgfusepath{stroke}%
\end{pgfscope}%
\begin{pgfscope}%
\pgfsetrectcap%
\pgfsetmiterjoin%
\pgfsetlinewidth{0.803000pt}%
\definecolor{currentstroke}{rgb}{0.000000,0.000000,0.000000}%
\pgfsetstrokecolor{currentstroke}%
\pgfsetdash{}{0pt}%
\pgfpathmoveto{\pgfqpoint{5.400000in}{0.500000in}}%
\pgfpathlineto{\pgfqpoint{5.400000in}{3.520000in}}%
\pgfusepath{stroke}%
\end{pgfscope}%
\begin{pgfscope}%
\pgfsetrectcap%
\pgfsetmiterjoin%
\pgfsetlinewidth{0.803000pt}%
\definecolor{currentstroke}{rgb}{0.000000,0.000000,0.000000}%
\pgfsetstrokecolor{currentstroke}%
\pgfsetdash{}{0pt}%
\pgfpathmoveto{\pgfqpoint{0.750000in}{0.500000in}}%
\pgfpathlineto{\pgfqpoint{5.400000in}{0.500000in}}%
\pgfusepath{stroke}%
\end{pgfscope}%
\begin{pgfscope}%
\pgfsetrectcap%
\pgfsetmiterjoin%
\pgfsetlinewidth{0.803000pt}%
\definecolor{currentstroke}{rgb}{0.000000,0.000000,0.000000}%
\pgfsetstrokecolor{currentstroke}%
\pgfsetdash{}{0pt}%
\pgfpathmoveto{\pgfqpoint{0.750000in}{3.520000in}}%
\pgfpathlineto{\pgfqpoint{5.400000in}{3.520000in}}%
\pgfusepath{stroke}%
\end{pgfscope}%
\begin{pgfscope}%
\pgfsetbuttcap%
\pgfsetmiterjoin%
\definecolor{currentfill}{rgb}{1.000000,1.000000,1.000000}%
\pgfsetfillcolor{currentfill}%
\pgfsetfillopacity{0.800000}%
\pgfsetlinewidth{1.003750pt}%
\definecolor{currentstroke}{rgb}{0.800000,0.800000,0.800000}%
\pgfsetstrokecolor{currentstroke}%
\pgfsetstrokeopacity{0.800000}%
\pgfsetdash{}{0pt}%
\pgfpathmoveto{\pgfqpoint{0.847222in}{3.205032in}}%
\pgfpathlineto{\pgfqpoint{1.499457in}{3.205032in}}%
\pgfpathquadraticcurveto{\pgfqpoint{1.527235in}{3.205032in}}{\pgfqpoint{1.527235in}{3.232809in}}%
\pgfpathlineto{\pgfqpoint{1.527235in}{3.422778in}}%
\pgfpathquadraticcurveto{\pgfqpoint{1.527235in}{3.450556in}}{\pgfqpoint{1.499457in}{3.450556in}}%
\pgfpathlineto{\pgfqpoint{0.847222in}{3.450556in}}%
\pgfpathquadraticcurveto{\pgfqpoint{0.819444in}{3.450556in}}{\pgfqpoint{0.819444in}{3.422778in}}%
\pgfpathlineto{\pgfqpoint{0.819444in}{3.232809in}}%
\pgfpathquadraticcurveto{\pgfqpoint{0.819444in}{3.205032in}}{\pgfqpoint{0.847222in}{3.205032in}}%
\pgfpathlineto{\pgfqpoint{0.847222in}{3.205032in}}%
\pgfpathclose%
\pgfusepath{stroke,fill}%
\end{pgfscope}%
\begin{pgfscope}%
\pgfsetrectcap%
\pgfsetroundjoin%
\pgfsetlinewidth{1.505625pt}%
\definecolor{currentstroke}{rgb}{0.000000,0.500000,0.000000}%
\pgfsetstrokecolor{currentstroke}%
\pgfsetdash{}{0pt}%
\pgfpathmoveto{\pgfqpoint{0.875000in}{3.338088in}}%
\pgfpathlineto{\pgfqpoint{1.013889in}{3.338088in}}%
\pgfpathlineto{\pgfqpoint{1.152778in}{3.338088in}}%
\pgfusepath{stroke}%
\end{pgfscope}%
\begin{pgfscope}%
\definecolor{textcolor}{rgb}{0.000000,0.000000,0.000000}%
\pgfsetstrokecolor{textcolor}%
\pgfsetfillcolor{textcolor}%
\pgftext[x=1.263889in,y=3.289477in,left,base]{\color{textcolor}\sffamily\fontsize{10.000000}{12.000000}\selectfont NN}%
\end{pgfscope}%
\end{pgfpicture}%
\makeatother%
\endgroup%

    \caption{Caption}
    \label{fig:my_label}
\end{figure}

\begin{figure}
%% Creator: Matplotlib, PGF backend
%%
%% To include the figure in your LaTeX document, write
%%   \input{<filename>.pgf}
%%
%% Make sure the required packages are loaded in your preamble
%%   \usepackage{pgf}
%%
%% Also ensure that all the required font packages are loaded; for instance,
%% the lmodern package is sometimes necessary when using math font.
%%   \usepackage{lmodern}
%%
%% Figures using additional raster images can only be included by \input if
%% they are in the same directory as the main LaTeX file. For loading figures
%% from other directories you can use the `import` package
%%   \usepackage{import}
%%
%% and then include the figures with
%%   \import{<path to file>}{<filename>.pgf}
%%
%% Matplotlib used the following preamble
%%   \usepackage{fontspec}
%%   \setmainfont{DejaVuSerif.ttf}[Path=\detokenize{C:/I/python38/Lib/site-packages/matplotlib/mpl-data/fonts/ttf/}]
%%   \setsansfont{DejaVuSans.ttf}[Path=\detokenize{C:/I/python38/Lib/site-packages/matplotlib/mpl-data/fonts/ttf/}]
%%   \setmonofont{DejaVuSansMono.ttf}[Path=\detokenize{C:/I/python38/Lib/site-packages/matplotlib/mpl-data/fonts/ttf/}]
%%
\begingroup%
\makeatletter%
\begin{pgfpicture}%
\pgfpathrectangle{\pgfpointorigin}{\pgfqpoint{6.000000in}{4.000000in}}%
\pgfusepath{use as bounding box, clip}%
\begin{pgfscope}%
\pgfsetbuttcap%
\pgfsetmiterjoin%
\pgfsetlinewidth{0.000000pt}%
\definecolor{currentstroke}{rgb}{1.000000,1.000000,1.000000}%
\pgfsetstrokecolor{currentstroke}%
\pgfsetstrokeopacity{0.000000}%
\pgfsetdash{}{0pt}%
\pgfpathmoveto{\pgfqpoint{0.000000in}{0.000000in}}%
\pgfpathlineto{\pgfqpoint{6.000000in}{0.000000in}}%
\pgfpathlineto{\pgfqpoint{6.000000in}{4.000000in}}%
\pgfpathlineto{\pgfqpoint{0.000000in}{4.000000in}}%
\pgfpathlineto{\pgfqpoint{0.000000in}{0.000000in}}%
\pgfpathclose%
\pgfusepath{}%
\end{pgfscope}%
\begin{pgfscope}%
\pgfsetbuttcap%
\pgfsetmiterjoin%
\definecolor{currentfill}{rgb}{1.000000,1.000000,1.000000}%
\pgfsetfillcolor{currentfill}%
\pgfsetlinewidth{0.000000pt}%
\definecolor{currentstroke}{rgb}{0.000000,0.000000,0.000000}%
\pgfsetstrokecolor{currentstroke}%
\pgfsetstrokeopacity{0.000000}%
\pgfsetdash{}{0pt}%
\pgfpathmoveto{\pgfqpoint{0.750000in}{0.500000in}}%
\pgfpathlineto{\pgfqpoint{5.400000in}{0.500000in}}%
\pgfpathlineto{\pgfqpoint{5.400000in}{3.520000in}}%
\pgfpathlineto{\pgfqpoint{0.750000in}{3.520000in}}%
\pgfpathlineto{\pgfqpoint{0.750000in}{0.500000in}}%
\pgfpathclose%
\pgfusepath{fill}%
\end{pgfscope}%
\begin{pgfscope}%
\pgfsetbuttcap%
\pgfsetroundjoin%
\definecolor{currentfill}{rgb}{0.000000,0.000000,0.000000}%
\pgfsetfillcolor{currentfill}%
\pgfsetlinewidth{0.803000pt}%
\definecolor{currentstroke}{rgb}{0.000000,0.000000,0.000000}%
\pgfsetstrokecolor{currentstroke}%
\pgfsetdash{}{0pt}%
\pgfsys@defobject{currentmarker}{\pgfqpoint{0.000000in}{-0.048611in}}{\pgfqpoint{0.000000in}{0.000000in}}{%
\pgfpathmoveto{\pgfqpoint{0.000000in}{0.000000in}}%
\pgfpathlineto{\pgfqpoint{0.000000in}{-0.048611in}}%
\pgfusepath{stroke,fill}%
}%
\begin{pgfscope}%
\pgfsys@transformshift{0.961364in}{0.500000in}%
\pgfsys@useobject{currentmarker}{}%
\end{pgfscope}%
\end{pgfscope}%
\begin{pgfscope}%
\definecolor{textcolor}{rgb}{0.000000,0.000000,0.000000}%
\pgfsetstrokecolor{textcolor}%
\pgfsetfillcolor{textcolor}%
\pgftext[x=0.961364in,y=0.402778in,,top]{\color{textcolor}\sffamily\fontsize{10.000000}{12.000000}\selectfont 0}%
\end{pgfscope}%
\begin{pgfscope}%
\pgfsetbuttcap%
\pgfsetroundjoin%
\definecolor{currentfill}{rgb}{0.000000,0.000000,0.000000}%
\pgfsetfillcolor{currentfill}%
\pgfsetlinewidth{0.803000pt}%
\definecolor{currentstroke}{rgb}{0.000000,0.000000,0.000000}%
\pgfsetstrokecolor{currentstroke}%
\pgfsetdash{}{0pt}%
\pgfsys@defobject{currentmarker}{\pgfqpoint{0.000000in}{-0.048611in}}{\pgfqpoint{0.000000in}{0.000000in}}{%
\pgfpathmoveto{\pgfqpoint{0.000000in}{0.000000in}}%
\pgfpathlineto{\pgfqpoint{0.000000in}{-0.048611in}}%
\pgfusepath{stroke,fill}%
}%
\begin{pgfscope}%
\pgfsys@transformshift{1.806987in}{0.500000in}%
\pgfsys@useobject{currentmarker}{}%
\end{pgfscope}%
\end{pgfscope}%
\begin{pgfscope}%
\definecolor{textcolor}{rgb}{0.000000,0.000000,0.000000}%
\pgfsetstrokecolor{textcolor}%
\pgfsetfillcolor{textcolor}%
\pgftext[x=1.806987in,y=0.402778in,,top]{\color{textcolor}\sffamily\fontsize{10.000000}{12.000000}\selectfont 1000}%
\end{pgfscope}%
\begin{pgfscope}%
\pgfsetbuttcap%
\pgfsetroundjoin%
\definecolor{currentfill}{rgb}{0.000000,0.000000,0.000000}%
\pgfsetfillcolor{currentfill}%
\pgfsetlinewidth{0.803000pt}%
\definecolor{currentstroke}{rgb}{0.000000,0.000000,0.000000}%
\pgfsetstrokecolor{currentstroke}%
\pgfsetdash{}{0pt}%
\pgfsys@defobject{currentmarker}{\pgfqpoint{0.000000in}{-0.048611in}}{\pgfqpoint{0.000000in}{0.000000in}}{%
\pgfpathmoveto{\pgfqpoint{0.000000in}{0.000000in}}%
\pgfpathlineto{\pgfqpoint{0.000000in}{-0.048611in}}%
\pgfusepath{stroke,fill}%
}%
\begin{pgfscope}%
\pgfsys@transformshift{2.652611in}{0.500000in}%
\pgfsys@useobject{currentmarker}{}%
\end{pgfscope}%
\end{pgfscope}%
\begin{pgfscope}%
\definecolor{textcolor}{rgb}{0.000000,0.000000,0.000000}%
\pgfsetstrokecolor{textcolor}%
\pgfsetfillcolor{textcolor}%
\pgftext[x=2.652611in,y=0.402778in,,top]{\color{textcolor}\sffamily\fontsize{10.000000}{12.000000}\selectfont 2000}%
\end{pgfscope}%
\begin{pgfscope}%
\pgfsetbuttcap%
\pgfsetroundjoin%
\definecolor{currentfill}{rgb}{0.000000,0.000000,0.000000}%
\pgfsetfillcolor{currentfill}%
\pgfsetlinewidth{0.803000pt}%
\definecolor{currentstroke}{rgb}{0.000000,0.000000,0.000000}%
\pgfsetstrokecolor{currentstroke}%
\pgfsetdash{}{0pt}%
\pgfsys@defobject{currentmarker}{\pgfqpoint{0.000000in}{-0.048611in}}{\pgfqpoint{0.000000in}{0.000000in}}{%
\pgfpathmoveto{\pgfqpoint{0.000000in}{0.000000in}}%
\pgfpathlineto{\pgfqpoint{0.000000in}{-0.048611in}}%
\pgfusepath{stroke,fill}%
}%
\begin{pgfscope}%
\pgfsys@transformshift{3.498235in}{0.500000in}%
\pgfsys@useobject{currentmarker}{}%
\end{pgfscope}%
\end{pgfscope}%
\begin{pgfscope}%
\definecolor{textcolor}{rgb}{0.000000,0.000000,0.000000}%
\pgfsetstrokecolor{textcolor}%
\pgfsetfillcolor{textcolor}%
\pgftext[x=3.498235in,y=0.402778in,,top]{\color{textcolor}\sffamily\fontsize{10.000000}{12.000000}\selectfont 3000}%
\end{pgfscope}%
\begin{pgfscope}%
\pgfsetbuttcap%
\pgfsetroundjoin%
\definecolor{currentfill}{rgb}{0.000000,0.000000,0.000000}%
\pgfsetfillcolor{currentfill}%
\pgfsetlinewidth{0.803000pt}%
\definecolor{currentstroke}{rgb}{0.000000,0.000000,0.000000}%
\pgfsetstrokecolor{currentstroke}%
\pgfsetdash{}{0pt}%
\pgfsys@defobject{currentmarker}{\pgfqpoint{0.000000in}{-0.048611in}}{\pgfqpoint{0.000000in}{0.000000in}}{%
\pgfpathmoveto{\pgfqpoint{0.000000in}{0.000000in}}%
\pgfpathlineto{\pgfqpoint{0.000000in}{-0.048611in}}%
\pgfusepath{stroke,fill}%
}%
\begin{pgfscope}%
\pgfsys@transformshift{4.343858in}{0.500000in}%
\pgfsys@useobject{currentmarker}{}%
\end{pgfscope}%
\end{pgfscope}%
\begin{pgfscope}%
\definecolor{textcolor}{rgb}{0.000000,0.000000,0.000000}%
\pgfsetstrokecolor{textcolor}%
\pgfsetfillcolor{textcolor}%
\pgftext[x=4.343858in,y=0.402778in,,top]{\color{textcolor}\sffamily\fontsize{10.000000}{12.000000}\selectfont 4000}%
\end{pgfscope}%
\begin{pgfscope}%
\pgfsetbuttcap%
\pgfsetroundjoin%
\definecolor{currentfill}{rgb}{0.000000,0.000000,0.000000}%
\pgfsetfillcolor{currentfill}%
\pgfsetlinewidth{0.803000pt}%
\definecolor{currentstroke}{rgb}{0.000000,0.000000,0.000000}%
\pgfsetstrokecolor{currentstroke}%
\pgfsetdash{}{0pt}%
\pgfsys@defobject{currentmarker}{\pgfqpoint{0.000000in}{-0.048611in}}{\pgfqpoint{0.000000in}{0.000000in}}{%
\pgfpathmoveto{\pgfqpoint{0.000000in}{0.000000in}}%
\pgfpathlineto{\pgfqpoint{0.000000in}{-0.048611in}}%
\pgfusepath{stroke,fill}%
}%
\begin{pgfscope}%
\pgfsys@transformshift{5.189482in}{0.500000in}%
\pgfsys@useobject{currentmarker}{}%
\end{pgfscope}%
\end{pgfscope}%
\begin{pgfscope}%
\definecolor{textcolor}{rgb}{0.000000,0.000000,0.000000}%
\pgfsetstrokecolor{textcolor}%
\pgfsetfillcolor{textcolor}%
\pgftext[x=5.189482in,y=0.402778in,,top]{\color{textcolor}\sffamily\fontsize{10.000000}{12.000000}\selectfont 5000}%
\end{pgfscope}%
\begin{pgfscope}%
\definecolor{textcolor}{rgb}{0.000000,0.000000,0.000000}%
\pgfsetstrokecolor{textcolor}%
\pgfsetfillcolor{textcolor}%
\pgftext[x=3.075000in,y=0.212809in,,top]{\color{textcolor}\sffamily\fontsize{10.000000}{12.000000}\selectfont iteration}%
\end{pgfscope}%
\begin{pgfscope}%
\pgfsetbuttcap%
\pgfsetroundjoin%
\definecolor{currentfill}{rgb}{0.000000,0.000000,0.000000}%
\pgfsetfillcolor{currentfill}%
\pgfsetlinewidth{0.803000pt}%
\definecolor{currentstroke}{rgb}{0.000000,0.000000,0.000000}%
\pgfsetstrokecolor{currentstroke}%
\pgfsetdash{}{0pt}%
\pgfsys@defobject{currentmarker}{\pgfqpoint{-0.048611in}{0.000000in}}{\pgfqpoint{-0.000000in}{0.000000in}}{%
\pgfpathmoveto{\pgfqpoint{-0.000000in}{0.000000in}}%
\pgfpathlineto{\pgfqpoint{-0.048611in}{0.000000in}}%
\pgfusepath{stroke,fill}%
}%
\begin{pgfscope}%
\pgfsys@transformshift{0.750000in}{0.521100in}%
\pgfsys@useobject{currentmarker}{}%
\end{pgfscope}%
\end{pgfscope}%
\begin{pgfscope}%
\definecolor{textcolor}{rgb}{0.000000,0.000000,0.000000}%
\pgfsetstrokecolor{textcolor}%
\pgfsetfillcolor{textcolor}%
\pgftext[x=0.343533in, y=0.468338in, left, base]{\color{textcolor}\sffamily\fontsize{10.000000}{12.000000}\selectfont 0.30}%
\end{pgfscope}%
\begin{pgfscope}%
\pgfsetbuttcap%
\pgfsetroundjoin%
\definecolor{currentfill}{rgb}{0.000000,0.000000,0.000000}%
\pgfsetfillcolor{currentfill}%
\pgfsetlinewidth{0.803000pt}%
\definecolor{currentstroke}{rgb}{0.000000,0.000000,0.000000}%
\pgfsetstrokecolor{currentstroke}%
\pgfsetdash{}{0pt}%
\pgfsys@defobject{currentmarker}{\pgfqpoint{-0.048611in}{0.000000in}}{\pgfqpoint{-0.000000in}{0.000000in}}{%
\pgfpathmoveto{\pgfqpoint{-0.000000in}{0.000000in}}%
\pgfpathlineto{\pgfqpoint{-0.048611in}{0.000000in}}%
\pgfusepath{stroke,fill}%
}%
\begin{pgfscope}%
\pgfsys@transformshift{0.750000in}{0.939474in}%
\pgfsys@useobject{currentmarker}{}%
\end{pgfscope}%
\end{pgfscope}%
\begin{pgfscope}%
\definecolor{textcolor}{rgb}{0.000000,0.000000,0.000000}%
\pgfsetstrokecolor{textcolor}%
\pgfsetfillcolor{textcolor}%
\pgftext[x=0.343533in, y=0.886713in, left, base]{\color{textcolor}\sffamily\fontsize{10.000000}{12.000000}\selectfont 0.35}%
\end{pgfscope}%
\begin{pgfscope}%
\pgfsetbuttcap%
\pgfsetroundjoin%
\definecolor{currentfill}{rgb}{0.000000,0.000000,0.000000}%
\pgfsetfillcolor{currentfill}%
\pgfsetlinewidth{0.803000pt}%
\definecolor{currentstroke}{rgb}{0.000000,0.000000,0.000000}%
\pgfsetstrokecolor{currentstroke}%
\pgfsetdash{}{0pt}%
\pgfsys@defobject{currentmarker}{\pgfqpoint{-0.048611in}{0.000000in}}{\pgfqpoint{-0.000000in}{0.000000in}}{%
\pgfpathmoveto{\pgfqpoint{-0.000000in}{0.000000in}}%
\pgfpathlineto{\pgfqpoint{-0.048611in}{0.000000in}}%
\pgfusepath{stroke,fill}%
}%
\begin{pgfscope}%
\pgfsys@transformshift{0.750000in}{1.357848in}%
\pgfsys@useobject{currentmarker}{}%
\end{pgfscope}%
\end{pgfscope}%
\begin{pgfscope}%
\definecolor{textcolor}{rgb}{0.000000,0.000000,0.000000}%
\pgfsetstrokecolor{textcolor}%
\pgfsetfillcolor{textcolor}%
\pgftext[x=0.343533in, y=1.305087in, left, base]{\color{textcolor}\sffamily\fontsize{10.000000}{12.000000}\selectfont 0.40}%
\end{pgfscope}%
\begin{pgfscope}%
\pgfsetbuttcap%
\pgfsetroundjoin%
\definecolor{currentfill}{rgb}{0.000000,0.000000,0.000000}%
\pgfsetfillcolor{currentfill}%
\pgfsetlinewidth{0.803000pt}%
\definecolor{currentstroke}{rgb}{0.000000,0.000000,0.000000}%
\pgfsetstrokecolor{currentstroke}%
\pgfsetdash{}{0pt}%
\pgfsys@defobject{currentmarker}{\pgfqpoint{-0.048611in}{0.000000in}}{\pgfqpoint{-0.000000in}{0.000000in}}{%
\pgfpathmoveto{\pgfqpoint{-0.000000in}{0.000000in}}%
\pgfpathlineto{\pgfqpoint{-0.048611in}{0.000000in}}%
\pgfusepath{stroke,fill}%
}%
\begin{pgfscope}%
\pgfsys@transformshift{0.750000in}{1.776222in}%
\pgfsys@useobject{currentmarker}{}%
\end{pgfscope}%
\end{pgfscope}%
\begin{pgfscope}%
\definecolor{textcolor}{rgb}{0.000000,0.000000,0.000000}%
\pgfsetstrokecolor{textcolor}%
\pgfsetfillcolor{textcolor}%
\pgftext[x=0.343533in, y=1.723461in, left, base]{\color{textcolor}\sffamily\fontsize{10.000000}{12.000000}\selectfont 0.45}%
\end{pgfscope}%
\begin{pgfscope}%
\pgfsetbuttcap%
\pgfsetroundjoin%
\definecolor{currentfill}{rgb}{0.000000,0.000000,0.000000}%
\pgfsetfillcolor{currentfill}%
\pgfsetlinewidth{0.803000pt}%
\definecolor{currentstroke}{rgb}{0.000000,0.000000,0.000000}%
\pgfsetstrokecolor{currentstroke}%
\pgfsetdash{}{0pt}%
\pgfsys@defobject{currentmarker}{\pgfqpoint{-0.048611in}{0.000000in}}{\pgfqpoint{-0.000000in}{0.000000in}}{%
\pgfpathmoveto{\pgfqpoint{-0.000000in}{0.000000in}}%
\pgfpathlineto{\pgfqpoint{-0.048611in}{0.000000in}}%
\pgfusepath{stroke,fill}%
}%
\begin{pgfscope}%
\pgfsys@transformshift{0.750000in}{2.194596in}%
\pgfsys@useobject{currentmarker}{}%
\end{pgfscope}%
\end{pgfscope}%
\begin{pgfscope}%
\definecolor{textcolor}{rgb}{0.000000,0.000000,0.000000}%
\pgfsetstrokecolor{textcolor}%
\pgfsetfillcolor{textcolor}%
\pgftext[x=0.343533in, y=2.141835in, left, base]{\color{textcolor}\sffamily\fontsize{10.000000}{12.000000}\selectfont 0.50}%
\end{pgfscope}%
\begin{pgfscope}%
\pgfsetbuttcap%
\pgfsetroundjoin%
\definecolor{currentfill}{rgb}{0.000000,0.000000,0.000000}%
\pgfsetfillcolor{currentfill}%
\pgfsetlinewidth{0.803000pt}%
\definecolor{currentstroke}{rgb}{0.000000,0.000000,0.000000}%
\pgfsetstrokecolor{currentstroke}%
\pgfsetdash{}{0pt}%
\pgfsys@defobject{currentmarker}{\pgfqpoint{-0.048611in}{0.000000in}}{\pgfqpoint{-0.000000in}{0.000000in}}{%
\pgfpathmoveto{\pgfqpoint{-0.000000in}{0.000000in}}%
\pgfpathlineto{\pgfqpoint{-0.048611in}{0.000000in}}%
\pgfusepath{stroke,fill}%
}%
\begin{pgfscope}%
\pgfsys@transformshift{0.750000in}{2.612971in}%
\pgfsys@useobject{currentmarker}{}%
\end{pgfscope}%
\end{pgfscope}%
\begin{pgfscope}%
\definecolor{textcolor}{rgb}{0.000000,0.000000,0.000000}%
\pgfsetstrokecolor{textcolor}%
\pgfsetfillcolor{textcolor}%
\pgftext[x=0.343533in, y=2.560209in, left, base]{\color{textcolor}\sffamily\fontsize{10.000000}{12.000000}\selectfont 0.55}%
\end{pgfscope}%
\begin{pgfscope}%
\pgfsetbuttcap%
\pgfsetroundjoin%
\definecolor{currentfill}{rgb}{0.000000,0.000000,0.000000}%
\pgfsetfillcolor{currentfill}%
\pgfsetlinewidth{0.803000pt}%
\definecolor{currentstroke}{rgb}{0.000000,0.000000,0.000000}%
\pgfsetstrokecolor{currentstroke}%
\pgfsetdash{}{0pt}%
\pgfsys@defobject{currentmarker}{\pgfqpoint{-0.048611in}{0.000000in}}{\pgfqpoint{-0.000000in}{0.000000in}}{%
\pgfpathmoveto{\pgfqpoint{-0.000000in}{0.000000in}}%
\pgfpathlineto{\pgfqpoint{-0.048611in}{0.000000in}}%
\pgfusepath{stroke,fill}%
}%
\begin{pgfscope}%
\pgfsys@transformshift{0.750000in}{3.031345in}%
\pgfsys@useobject{currentmarker}{}%
\end{pgfscope}%
\end{pgfscope}%
\begin{pgfscope}%
\definecolor{textcolor}{rgb}{0.000000,0.000000,0.000000}%
\pgfsetstrokecolor{textcolor}%
\pgfsetfillcolor{textcolor}%
\pgftext[x=0.343533in, y=2.978583in, left, base]{\color{textcolor}\sffamily\fontsize{10.000000}{12.000000}\selectfont 0.60}%
\end{pgfscope}%
\begin{pgfscope}%
\pgfsetbuttcap%
\pgfsetroundjoin%
\definecolor{currentfill}{rgb}{0.000000,0.000000,0.000000}%
\pgfsetfillcolor{currentfill}%
\pgfsetlinewidth{0.803000pt}%
\definecolor{currentstroke}{rgb}{0.000000,0.000000,0.000000}%
\pgfsetstrokecolor{currentstroke}%
\pgfsetdash{}{0pt}%
\pgfsys@defobject{currentmarker}{\pgfqpoint{-0.048611in}{0.000000in}}{\pgfqpoint{-0.000000in}{0.000000in}}{%
\pgfpathmoveto{\pgfqpoint{-0.000000in}{0.000000in}}%
\pgfpathlineto{\pgfqpoint{-0.048611in}{0.000000in}}%
\pgfusepath{stroke,fill}%
}%
\begin{pgfscope}%
\pgfsys@transformshift{0.750000in}{3.449719in}%
\pgfsys@useobject{currentmarker}{}%
\end{pgfscope}%
\end{pgfscope}%
\begin{pgfscope}%
\definecolor{textcolor}{rgb}{0.000000,0.000000,0.000000}%
\pgfsetstrokecolor{textcolor}%
\pgfsetfillcolor{textcolor}%
\pgftext[x=0.343533in, y=3.396957in, left, base]{\color{textcolor}\sffamily\fontsize{10.000000}{12.000000}\selectfont 0.65}%
\end{pgfscope}%
\begin{pgfscope}%
\definecolor{textcolor}{rgb}{0.000000,0.000000,0.000000}%
\pgfsetstrokecolor{textcolor}%
\pgfsetfillcolor{textcolor}%
\pgftext[x=0.287977in,y=2.010000in,,bottom,rotate=90.000000]{\color{textcolor}\sffamily\fontsize{10.000000}{12.000000}\selectfont loss function value}%
\end{pgfscope}%
\begin{pgfscope}%
\pgfpathrectangle{\pgfqpoint{0.750000in}{0.500000in}}{\pgfqpoint{4.650000in}{3.020000in}}%
\pgfusepath{clip}%
\pgfsetrectcap%
\pgfsetroundjoin%
\pgfsetlinewidth{1.505625pt}%
\definecolor{currentstroke}{rgb}{1.000000,0.000000,0.000000}%
\pgfsetstrokecolor{currentstroke}%
\pgfsetdash{}{0pt}%
\pgfpathmoveto{\pgfqpoint{0.961364in}{2.188774in}}%
\pgfpathlineto{\pgfqpoint{0.962209in}{2.125263in}}%
\pgfpathlineto{\pgfqpoint{0.964746in}{2.273416in}}%
\pgfpathlineto{\pgfqpoint{0.966437in}{2.141432in}}%
\pgfpathlineto{\pgfqpoint{0.967283in}{2.184815in}}%
\pgfpathlineto{\pgfqpoint{0.968129in}{2.232472in}}%
\pgfpathlineto{\pgfqpoint{0.968974in}{2.147937in}}%
\pgfpathlineto{\pgfqpoint{0.969820in}{2.229888in}}%
\pgfpathlineto{\pgfqpoint{0.972357in}{2.074721in}}%
\pgfpathlineto{\pgfqpoint{0.973202in}{2.204054in}}%
\pgfpathlineto{\pgfqpoint{0.974048in}{2.153532in}}%
\pgfpathlineto{\pgfqpoint{0.974894in}{2.176106in}}%
\pgfpathlineto{\pgfqpoint{0.975739in}{2.246386in}}%
\pgfpathlineto{\pgfqpoint{0.976585in}{2.175945in}}%
\pgfpathlineto{\pgfqpoint{0.977430in}{2.261285in}}%
\pgfpathlineto{\pgfqpoint{0.979122in}{2.098111in}}%
\pgfpathlineto{\pgfqpoint{0.981659in}{2.263336in}}%
\pgfpathlineto{\pgfqpoint{0.982504in}{2.261752in}}%
\pgfpathlineto{\pgfqpoint{0.986732in}{2.005974in}}%
\pgfpathlineto{\pgfqpoint{0.987578in}{2.257275in}}%
\pgfpathlineto{\pgfqpoint{0.988424in}{2.151312in}}%
\pgfpathlineto{\pgfqpoint{0.989269in}{2.266544in}}%
\pgfpathlineto{\pgfqpoint{0.990115in}{2.179609in}}%
\pgfpathlineto{\pgfqpoint{0.991806in}{2.036214in}}%
\pgfpathlineto{\pgfqpoint{0.992652in}{2.068493in}}%
\pgfpathlineto{\pgfqpoint{0.993497in}{2.254335in}}%
\pgfpathlineto{\pgfqpoint{0.994343in}{2.158658in}}%
\pgfpathlineto{\pgfqpoint{0.995189in}{2.086126in}}%
\pgfpathlineto{\pgfqpoint{0.997725in}{2.251090in}}%
\pgfpathlineto{\pgfqpoint{0.999417in}{2.161613in}}%
\pgfpathlineto{\pgfqpoint{1.000262in}{2.246873in}}%
\pgfpathlineto{\pgfqpoint{1.001108in}{2.007148in}}%
\pgfpathlineto{\pgfqpoint{1.001954in}{2.162496in}}%
\pgfpathlineto{\pgfqpoint{1.002799in}{2.171662in}}%
\pgfpathlineto{\pgfqpoint{1.003645in}{2.388793in}}%
\pgfpathlineto{\pgfqpoint{1.004490in}{2.292141in}}%
\pgfpathlineto{\pgfqpoint{1.005336in}{2.256588in}}%
\pgfpathlineto{\pgfqpoint{1.007873in}{2.325882in}}%
\pgfpathlineto{\pgfqpoint{1.009564in}{1.963008in}}%
\pgfpathlineto{\pgfqpoint{1.010410in}{2.248538in}}%
\pgfpathlineto{\pgfqpoint{1.011255in}{2.143825in}}%
\pgfpathlineto{\pgfqpoint{1.012101in}{2.130119in}}%
\pgfpathlineto{\pgfqpoint{1.012947in}{2.152158in}}%
\pgfpathlineto{\pgfqpoint{1.013792in}{2.263814in}}%
\pgfpathlineto{\pgfqpoint{1.014638in}{2.087594in}}%
\pgfpathlineto{\pgfqpoint{1.015484in}{2.320088in}}%
\pgfpathlineto{\pgfqpoint{1.016329in}{2.185450in}}%
\pgfpathlineto{\pgfqpoint{1.017175in}{2.251646in}}%
\pgfpathlineto{\pgfqpoint{1.018020in}{2.218590in}}%
\pgfpathlineto{\pgfqpoint{1.018866in}{2.212454in}}%
\pgfpathlineto{\pgfqpoint{1.019712in}{2.389151in}}%
\pgfpathlineto{\pgfqpoint{1.020557in}{2.092262in}}%
\pgfpathlineto{\pgfqpoint{1.021403in}{2.137901in}}%
\pgfpathlineto{\pgfqpoint{1.022249in}{2.165688in}}%
\pgfpathlineto{\pgfqpoint{1.023094in}{2.071836in}}%
\pgfpathlineto{\pgfqpoint{1.023940in}{1.760351in}}%
\pgfpathlineto{\pgfqpoint{1.024785in}{2.195311in}}%
\pgfpathlineto{\pgfqpoint{1.025631in}{2.189629in}}%
\pgfpathlineto{\pgfqpoint{1.026477in}{1.965652in}}%
\pgfpathlineto{\pgfqpoint{1.027322in}{2.235140in}}%
\pgfpathlineto{\pgfqpoint{1.028168in}{2.102035in}}%
\pgfpathlineto{\pgfqpoint{1.029859in}{2.246303in}}%
\pgfpathlineto{\pgfqpoint{1.030705in}{2.139313in}}%
\pgfpathlineto{\pgfqpoint{1.031550in}{1.798132in}}%
\pgfpathlineto{\pgfqpoint{1.032396in}{1.980006in}}%
\pgfpathlineto{\pgfqpoint{1.034087in}{2.356699in}}%
\pgfpathlineto{\pgfqpoint{1.034933in}{2.093596in}}%
\pgfpathlineto{\pgfqpoint{1.035779in}{2.258535in}}%
\pgfpathlineto{\pgfqpoint{1.036624in}{2.190356in}}%
\pgfpathlineto{\pgfqpoint{1.037470in}{2.362202in}}%
\pgfpathlineto{\pgfqpoint{1.040007in}{2.108379in}}%
\pgfpathlineto{\pgfqpoint{1.040852in}{2.484262in}}%
\pgfpathlineto{\pgfqpoint{1.041698in}{2.065351in}}%
\pgfpathlineto{\pgfqpoint{1.042544in}{2.182895in}}%
\pgfpathlineto{\pgfqpoint{1.043389in}{2.169854in}}%
\pgfpathlineto{\pgfqpoint{1.044235in}{1.993457in}}%
\pgfpathlineto{\pgfqpoint{1.045080in}{2.365806in}}%
\pgfpathlineto{\pgfqpoint{1.045926in}{2.166135in}}%
\pgfpathlineto{\pgfqpoint{1.046772in}{2.242933in}}%
\pgfpathlineto{\pgfqpoint{1.047617in}{1.968795in}}%
\pgfpathlineto{\pgfqpoint{1.048463in}{2.144551in}}%
\pgfpathlineto{\pgfqpoint{1.049308in}{2.178417in}}%
\pgfpathlineto{\pgfqpoint{1.050154in}{2.103031in}}%
\pgfpathlineto{\pgfqpoint{1.051845in}{2.314378in}}%
\pgfpathlineto{\pgfqpoint{1.052691in}{2.004285in}}%
\pgfpathlineto{\pgfqpoint{1.053537in}{2.137647in}}%
\pgfpathlineto{\pgfqpoint{1.054382in}{2.198730in}}%
\pgfpathlineto{\pgfqpoint{1.055228in}{2.169050in}}%
\pgfpathlineto{\pgfqpoint{1.056073in}{2.206376in}}%
\pgfpathlineto{\pgfqpoint{1.056919in}{1.972381in}}%
\pgfpathlineto{\pgfqpoint{1.057765in}{2.273894in}}%
\pgfpathlineto{\pgfqpoint{1.058610in}{1.984678in}}%
\pgfpathlineto{\pgfqpoint{1.059456in}{2.158630in}}%
\pgfpathlineto{\pgfqpoint{1.060302in}{2.330890in}}%
\pgfpathlineto{\pgfqpoint{1.061147in}{2.257825in}}%
\pgfpathlineto{\pgfqpoint{1.061993in}{2.256247in}}%
\pgfpathlineto{\pgfqpoint{1.064530in}{2.017854in}}%
\pgfpathlineto{\pgfqpoint{1.065375in}{1.816354in}}%
\pgfpathlineto{\pgfqpoint{1.067067in}{2.523695in}}%
\pgfpathlineto{\pgfqpoint{1.067912in}{2.056229in}}%
\pgfpathlineto{\pgfqpoint{1.068758in}{2.341585in}}%
\pgfpathlineto{\pgfqpoint{1.069603in}{2.438703in}}%
\pgfpathlineto{\pgfqpoint{1.070449in}{2.159808in}}%
\pgfpathlineto{\pgfqpoint{1.071295in}{2.213004in}}%
\pgfpathlineto{\pgfqpoint{1.072140in}{2.350274in}}%
\pgfpathlineto{\pgfqpoint{1.074677in}{2.062971in}}%
\pgfpathlineto{\pgfqpoint{1.075523in}{2.306022in}}%
\pgfpathlineto{\pgfqpoint{1.076368in}{1.838216in}}%
\pgfpathlineto{\pgfqpoint{1.077214in}{2.139073in}}%
\pgfpathlineto{\pgfqpoint{1.078060in}{2.017767in}}%
\pgfpathlineto{\pgfqpoint{1.078905in}{2.426586in}}%
\pgfpathlineto{\pgfqpoint{1.079751in}{2.285148in}}%
\pgfpathlineto{\pgfqpoint{1.080597in}{2.090373in}}%
\pgfpathlineto{\pgfqpoint{1.081442in}{2.455552in}}%
\pgfpathlineto{\pgfqpoint{1.082288in}{2.362475in}}%
\pgfpathlineto{\pgfqpoint{1.083979in}{1.941390in}}%
\pgfpathlineto{\pgfqpoint{1.084825in}{2.402159in}}%
\pgfpathlineto{\pgfqpoint{1.085670in}{2.177887in}}%
\pgfpathlineto{\pgfqpoint{1.086516in}{2.333645in}}%
\pgfpathlineto{\pgfqpoint{1.087362in}{2.252134in}}%
\pgfpathlineto{\pgfqpoint{1.089053in}{2.542147in}}%
\pgfpathlineto{\pgfqpoint{1.089898in}{2.074417in}}%
\pgfpathlineto{\pgfqpoint{1.090744in}{2.423206in}}%
\pgfpathlineto{\pgfqpoint{1.093281in}{1.967164in}}%
\pgfpathlineto{\pgfqpoint{1.094972in}{2.292582in}}%
\pgfpathlineto{\pgfqpoint{1.095818in}{1.951146in}}%
\pgfpathlineto{\pgfqpoint{1.096663in}{1.953640in}}%
\pgfpathlineto{\pgfqpoint{1.097509in}{2.426126in}}%
\pgfpathlineto{\pgfqpoint{1.098355in}{2.300340in}}%
\pgfpathlineto{\pgfqpoint{1.099200in}{2.188552in}}%
\pgfpathlineto{\pgfqpoint{1.100046in}{2.202604in}}%
\pgfpathlineto{\pgfqpoint{1.100892in}{2.237576in}}%
\pgfpathlineto{\pgfqpoint{1.101737in}{1.947687in}}%
\pgfpathlineto{\pgfqpoint{1.102583in}{2.216356in}}%
\pgfpathlineto{\pgfqpoint{1.103428in}{2.204145in}}%
\pgfpathlineto{\pgfqpoint{1.104274in}{2.105147in}}%
\pgfpathlineto{\pgfqpoint{1.105120in}{2.419605in}}%
\pgfpathlineto{\pgfqpoint{1.106811in}{2.051234in}}%
\pgfpathlineto{\pgfqpoint{1.107657in}{2.229046in}}%
\pgfpathlineto{\pgfqpoint{1.109348in}{1.905116in}}%
\pgfpathlineto{\pgfqpoint{1.111885in}{2.396343in}}%
\pgfpathlineto{\pgfqpoint{1.114422in}{1.960712in}}%
\pgfpathlineto{\pgfqpoint{1.115267in}{2.071434in}}%
\pgfpathlineto{\pgfqpoint{1.116958in}{2.167795in}}%
\pgfpathlineto{\pgfqpoint{1.117804in}{2.068775in}}%
\pgfpathlineto{\pgfqpoint{1.118650in}{2.329980in}}%
\pgfpathlineto{\pgfqpoint{1.119495in}{2.169131in}}%
\pgfpathlineto{\pgfqpoint{1.120341in}{2.177422in}}%
\pgfpathlineto{\pgfqpoint{1.121187in}{2.283678in}}%
\pgfpathlineto{\pgfqpoint{1.122032in}{2.184785in}}%
\pgfpathlineto{\pgfqpoint{1.123723in}{2.343641in}}%
\pgfpathlineto{\pgfqpoint{1.124569in}{1.899649in}}%
\pgfpathlineto{\pgfqpoint{1.125415in}{2.293285in}}%
\pgfpathlineto{\pgfqpoint{1.126260in}{2.052083in}}%
\pgfpathlineto{\pgfqpoint{1.127106in}{2.319722in}}%
\pgfpathlineto{\pgfqpoint{1.127951in}{2.102856in}}%
\pgfpathlineto{\pgfqpoint{1.128797in}{2.083686in}}%
\pgfpathlineto{\pgfqpoint{1.129643in}{2.408349in}}%
\pgfpathlineto{\pgfqpoint{1.130488in}{2.339223in}}%
\pgfpathlineto{\pgfqpoint{1.133025in}{2.137883in}}%
\pgfpathlineto{\pgfqpoint{1.133871in}{2.374236in}}%
\pgfpathlineto{\pgfqpoint{1.134716in}{1.835321in}}%
\pgfpathlineto{\pgfqpoint{1.135562in}{2.019344in}}%
\pgfpathlineto{\pgfqpoint{1.137253in}{2.360539in}}%
\pgfpathlineto{\pgfqpoint{1.138945in}{2.225635in}}%
\pgfpathlineto{\pgfqpoint{1.139790in}{2.205183in}}%
\pgfpathlineto{\pgfqpoint{1.140636in}{2.368497in}}%
\pgfpathlineto{\pgfqpoint{1.141481in}{2.034619in}}%
\pgfpathlineto{\pgfqpoint{1.142327in}{2.278479in}}%
\pgfpathlineto{\pgfqpoint{1.144018in}{1.925953in}}%
\pgfpathlineto{\pgfqpoint{1.144864in}{2.354941in}}%
\pgfpathlineto{\pgfqpoint{1.145710in}{2.034055in}}%
\pgfpathlineto{\pgfqpoint{1.148246in}{2.453775in}}%
\pgfpathlineto{\pgfqpoint{1.149938in}{2.042677in}}%
\pgfpathlineto{\pgfqpoint{1.150783in}{2.188753in}}%
\pgfpathlineto{\pgfqpoint{1.151629in}{2.117199in}}%
\pgfpathlineto{\pgfqpoint{1.152475in}{2.048887in}}%
\pgfpathlineto{\pgfqpoint{1.153320in}{2.507548in}}%
\pgfpathlineto{\pgfqpoint{1.155011in}{2.033339in}}%
\pgfpathlineto{\pgfqpoint{1.155857in}{2.103405in}}%
\pgfpathlineto{\pgfqpoint{1.156703in}{1.891946in}}%
\pgfpathlineto{\pgfqpoint{1.157548in}{2.105950in}}%
\pgfpathlineto{\pgfqpoint{1.158394in}{1.789313in}}%
\pgfpathlineto{\pgfqpoint{1.159240in}{2.616262in}}%
\pgfpathlineto{\pgfqpoint{1.160085in}{2.198656in}}%
\pgfpathlineto{\pgfqpoint{1.160931in}{2.244287in}}%
\pgfpathlineto{\pgfqpoint{1.162622in}{1.985564in}}%
\pgfpathlineto{\pgfqpoint{1.165159in}{2.278643in}}%
\pgfpathlineto{\pgfqpoint{1.166005in}{2.346688in}}%
\pgfpathlineto{\pgfqpoint{1.166850in}{2.114229in}}%
\pgfpathlineto{\pgfqpoint{1.167696in}{2.263503in}}%
\pgfpathlineto{\pgfqpoint{1.168541in}{2.411747in}}%
\pgfpathlineto{\pgfqpoint{1.170233in}{2.024760in}}%
\pgfpathlineto{\pgfqpoint{1.172770in}{2.157271in}}%
\pgfpathlineto{\pgfqpoint{1.173615in}{2.161351in}}%
\pgfpathlineto{\pgfqpoint{1.174461in}{1.912995in}}%
\pgfpathlineto{\pgfqpoint{1.175306in}{2.524731in}}%
\pgfpathlineto{\pgfqpoint{1.176152in}{2.285923in}}%
\pgfpathlineto{\pgfqpoint{1.176998in}{2.043466in}}%
\pgfpathlineto{\pgfqpoint{1.177843in}{2.470376in}}%
\pgfpathlineto{\pgfqpoint{1.178689in}{2.159820in}}%
\pgfpathlineto{\pgfqpoint{1.179535in}{2.083191in}}%
\pgfpathlineto{\pgfqpoint{1.180380in}{2.484856in}}%
\pgfpathlineto{\pgfqpoint{1.181226in}{2.248754in}}%
\pgfpathlineto{\pgfqpoint{1.182071in}{2.314463in}}%
\pgfpathlineto{\pgfqpoint{1.182917in}{2.255777in}}%
\pgfpathlineto{\pgfqpoint{1.183763in}{1.807649in}}%
\pgfpathlineto{\pgfqpoint{1.186300in}{2.419865in}}%
\pgfpathlineto{\pgfqpoint{1.187145in}{2.166712in}}%
\pgfpathlineto{\pgfqpoint{1.187991in}{2.266217in}}%
\pgfpathlineto{\pgfqpoint{1.189682in}{1.927752in}}%
\pgfpathlineto{\pgfqpoint{1.191373in}{2.413541in}}%
\pgfpathlineto{\pgfqpoint{1.193910in}{2.024418in}}%
\pgfpathlineto{\pgfqpoint{1.194756in}{2.332866in}}%
\pgfpathlineto{\pgfqpoint{1.195601in}{2.144461in}}%
\pgfpathlineto{\pgfqpoint{1.196447in}{2.296156in}}%
\pgfpathlineto{\pgfqpoint{1.198138in}{1.883120in}}%
\pgfpathlineto{\pgfqpoint{1.199830in}{2.304248in}}%
\pgfpathlineto{\pgfqpoint{1.200675in}{2.298314in}}%
\pgfpathlineto{\pgfqpoint{1.201521in}{2.270086in}}%
\pgfpathlineto{\pgfqpoint{1.202366in}{2.456696in}}%
\pgfpathlineto{\pgfqpoint{1.204903in}{2.058543in}}%
\pgfpathlineto{\pgfqpoint{1.207440in}{2.397497in}}%
\pgfpathlineto{\pgfqpoint{1.209131in}{1.970448in}}%
\pgfpathlineto{\pgfqpoint{1.209977in}{2.323463in}}%
\pgfpathlineto{\pgfqpoint{1.210823in}{2.043621in}}%
\pgfpathlineto{\pgfqpoint{1.211668in}{2.375318in}}%
\pgfpathlineto{\pgfqpoint{1.212514in}{2.333061in}}%
\pgfpathlineto{\pgfqpoint{1.213359in}{1.791109in}}%
\pgfpathlineto{\pgfqpoint{1.214205in}{2.110814in}}%
\pgfpathlineto{\pgfqpoint{1.215051in}{2.067363in}}%
\pgfpathlineto{\pgfqpoint{1.217588in}{2.352764in}}%
\pgfpathlineto{\pgfqpoint{1.219279in}{2.226880in}}%
\pgfpathlineto{\pgfqpoint{1.220124in}{2.276574in}}%
\pgfpathlineto{\pgfqpoint{1.220970in}{2.205326in}}%
\pgfpathlineto{\pgfqpoint{1.221816in}{1.935570in}}%
\pgfpathlineto{\pgfqpoint{1.222661in}{2.112444in}}%
\pgfpathlineto{\pgfqpoint{1.223507in}{2.351143in}}%
\pgfpathlineto{\pgfqpoint{1.226044in}{1.936467in}}%
\pgfpathlineto{\pgfqpoint{1.226889in}{2.237416in}}%
\pgfpathlineto{\pgfqpoint{1.227735in}{1.985847in}}%
\pgfpathlineto{\pgfqpoint{1.228581in}{2.321043in}}%
\pgfpathlineto{\pgfqpoint{1.229426in}{1.949181in}}%
\pgfpathlineto{\pgfqpoint{1.230272in}{2.195977in}}%
\pgfpathlineto{\pgfqpoint{1.231118in}{2.294846in}}%
\pgfpathlineto{\pgfqpoint{1.231963in}{2.148529in}}%
\pgfpathlineto{\pgfqpoint{1.232809in}{2.206845in}}%
\pgfpathlineto{\pgfqpoint{1.233654in}{2.211238in}}%
\pgfpathlineto{\pgfqpoint{1.234500in}{2.369104in}}%
\pgfpathlineto{\pgfqpoint{1.237037in}{1.932201in}}%
\pgfpathlineto{\pgfqpoint{1.237883in}{2.383304in}}%
\pgfpathlineto{\pgfqpoint{1.238728in}{2.021974in}}%
\pgfpathlineto{\pgfqpoint{1.239574in}{2.156706in}}%
\pgfpathlineto{\pgfqpoint{1.240419in}{1.862558in}}%
\pgfpathlineto{\pgfqpoint{1.241265in}{2.283905in}}%
\pgfpathlineto{\pgfqpoint{1.242111in}{1.999659in}}%
\pgfpathlineto{\pgfqpoint{1.242956in}{1.867180in}}%
\pgfpathlineto{\pgfqpoint{1.243802in}{2.164570in}}%
\pgfpathlineto{\pgfqpoint{1.244648in}{1.844663in}}%
\pgfpathlineto{\pgfqpoint{1.245493in}{1.911168in}}%
\pgfpathlineto{\pgfqpoint{1.247184in}{2.235056in}}%
\pgfpathlineto{\pgfqpoint{1.248030in}{2.137812in}}%
\pgfpathlineto{\pgfqpoint{1.248876in}{2.301652in}}%
\pgfpathlineto{\pgfqpoint{1.251413in}{1.886844in}}%
\pgfpathlineto{\pgfqpoint{1.252258in}{2.591167in}}%
\pgfpathlineto{\pgfqpoint{1.253104in}{2.518716in}}%
\pgfpathlineto{\pgfqpoint{1.255641in}{1.938822in}}%
\pgfpathlineto{\pgfqpoint{1.256486in}{2.491012in}}%
\pgfpathlineto{\pgfqpoint{1.257332in}{2.052684in}}%
\pgfpathlineto{\pgfqpoint{1.258178in}{2.495852in}}%
\pgfpathlineto{\pgfqpoint{1.259023in}{2.406467in}}%
\pgfpathlineto{\pgfqpoint{1.259869in}{2.305554in}}%
\pgfpathlineto{\pgfqpoint{1.260714in}{1.968113in}}%
\pgfpathlineto{\pgfqpoint{1.261560in}{2.421718in}}%
\pgfpathlineto{\pgfqpoint{1.262406in}{2.170724in}}%
\pgfpathlineto{\pgfqpoint{1.263251in}{2.102816in}}%
\pgfpathlineto{\pgfqpoint{1.264097in}{2.108407in}}%
\pgfpathlineto{\pgfqpoint{1.265788in}{2.289737in}}%
\pgfpathlineto{\pgfqpoint{1.268325in}{2.082733in}}%
\pgfpathlineto{\pgfqpoint{1.270862in}{2.543166in}}%
\pgfpathlineto{\pgfqpoint{1.271708in}{2.108262in}}%
\pgfpathlineto{\pgfqpoint{1.272553in}{2.131446in}}%
\pgfpathlineto{\pgfqpoint{1.274244in}{2.252655in}}%
\pgfpathlineto{\pgfqpoint{1.275936in}{1.995416in}}%
\pgfpathlineto{\pgfqpoint{1.277627in}{2.424320in}}%
\pgfpathlineto{\pgfqpoint{1.280164in}{2.037766in}}%
\pgfpathlineto{\pgfqpoint{1.281009in}{2.194949in}}%
\pgfpathlineto{\pgfqpoint{1.281855in}{2.029053in}}%
\pgfpathlineto{\pgfqpoint{1.282701in}{2.224259in}}%
\pgfpathlineto{\pgfqpoint{1.283546in}{2.221166in}}%
\pgfpathlineto{\pgfqpoint{1.284392in}{2.237594in}}%
\pgfpathlineto{\pgfqpoint{1.285238in}{2.132766in}}%
\pgfpathlineto{\pgfqpoint{1.286083in}{2.320632in}}%
\pgfpathlineto{\pgfqpoint{1.287774in}{1.877369in}}%
\pgfpathlineto{\pgfqpoint{1.288620in}{2.467889in}}%
\pgfpathlineto{\pgfqpoint{1.289466in}{2.242785in}}%
\pgfpathlineto{\pgfqpoint{1.290311in}{2.259488in}}%
\pgfpathlineto{\pgfqpoint{1.291157in}{2.592468in}}%
\pgfpathlineto{\pgfqpoint{1.292848in}{2.017382in}}%
\pgfpathlineto{\pgfqpoint{1.294539in}{2.396060in}}%
\pgfpathlineto{\pgfqpoint{1.295385in}{2.052408in}}%
\pgfpathlineto{\pgfqpoint{1.296231in}{2.242655in}}%
\pgfpathlineto{\pgfqpoint{1.297076in}{2.179494in}}%
\pgfpathlineto{\pgfqpoint{1.297922in}{2.393952in}}%
\pgfpathlineto{\pgfqpoint{1.298767in}{2.366254in}}%
\pgfpathlineto{\pgfqpoint{1.299613in}{2.389180in}}%
\pgfpathlineto{\pgfqpoint{1.300459in}{2.683862in}}%
\pgfpathlineto{\pgfqpoint{1.302996in}{2.157777in}}%
\pgfpathlineto{\pgfqpoint{1.304687in}{2.234074in}}%
\pgfpathlineto{\pgfqpoint{1.305532in}{2.408534in}}%
\pgfpathlineto{\pgfqpoint{1.306378in}{2.272678in}}%
\pgfpathlineto{\pgfqpoint{1.307224in}{2.275925in}}%
\pgfpathlineto{\pgfqpoint{1.308069in}{2.031374in}}%
\pgfpathlineto{\pgfqpoint{1.308915in}{2.096288in}}%
\pgfpathlineto{\pgfqpoint{1.310606in}{2.436208in}}%
\pgfpathlineto{\pgfqpoint{1.312297in}{1.982563in}}%
\pgfpathlineto{\pgfqpoint{1.313989in}{2.331541in}}%
\pgfpathlineto{\pgfqpoint{1.314834in}{2.030060in}}%
\pgfpathlineto{\pgfqpoint{1.315680in}{2.195705in}}%
\pgfpathlineto{\pgfqpoint{1.316526in}{2.175391in}}%
\pgfpathlineto{\pgfqpoint{1.317371in}{1.935711in}}%
\pgfpathlineto{\pgfqpoint{1.319908in}{2.398577in}}%
\pgfpathlineto{\pgfqpoint{1.321599in}{2.019861in}}%
\pgfpathlineto{\pgfqpoint{1.322445in}{2.335719in}}%
\pgfpathlineto{\pgfqpoint{1.323291in}{2.127252in}}%
\pgfpathlineto{\pgfqpoint{1.324136in}{2.342418in}}%
\pgfpathlineto{\pgfqpoint{1.326673in}{1.869731in}}%
\pgfpathlineto{\pgfqpoint{1.327519in}{2.387310in}}%
\pgfpathlineto{\pgfqpoint{1.328364in}{2.141473in}}%
\pgfpathlineto{\pgfqpoint{1.329210in}{2.073174in}}%
\pgfpathlineto{\pgfqpoint{1.330056in}{2.322059in}}%
\pgfpathlineto{\pgfqpoint{1.330901in}{2.124196in}}%
\pgfpathlineto{\pgfqpoint{1.331747in}{2.140158in}}%
\pgfpathlineto{\pgfqpoint{1.333438in}{1.809072in}}%
\pgfpathlineto{\pgfqpoint{1.334284in}{1.875310in}}%
\pgfpathlineto{\pgfqpoint{1.336821in}{2.320191in}}%
\pgfpathlineto{\pgfqpoint{1.337666in}{1.930444in}}%
\pgfpathlineto{\pgfqpoint{1.339357in}{2.327428in}}%
\pgfpathlineto{\pgfqpoint{1.340203in}{2.319375in}}%
\pgfpathlineto{\pgfqpoint{1.341894in}{2.581683in}}%
\pgfpathlineto{\pgfqpoint{1.342740in}{1.938596in}}%
\pgfpathlineto{\pgfqpoint{1.343586in}{2.265189in}}%
\pgfpathlineto{\pgfqpoint{1.346122in}{2.072932in}}%
\pgfpathlineto{\pgfqpoint{1.346968in}{2.266623in}}%
\pgfpathlineto{\pgfqpoint{1.347814in}{2.131810in}}%
\pgfpathlineto{\pgfqpoint{1.349505in}{2.382889in}}%
\pgfpathlineto{\pgfqpoint{1.350351in}{2.514472in}}%
\pgfpathlineto{\pgfqpoint{1.352887in}{1.994958in}}%
\pgfpathlineto{\pgfqpoint{1.353733in}{2.007141in}}%
\pgfpathlineto{\pgfqpoint{1.354579in}{2.581228in}}%
\pgfpathlineto{\pgfqpoint{1.355424in}{2.148461in}}%
\pgfpathlineto{\pgfqpoint{1.357116in}{2.249987in}}%
\pgfpathlineto{\pgfqpoint{1.357961in}{2.199517in}}%
\pgfpathlineto{\pgfqpoint{1.358807in}{1.963636in}}%
\pgfpathlineto{\pgfqpoint{1.359652in}{2.199535in}}%
\pgfpathlineto{\pgfqpoint{1.360498in}{2.079060in}}%
\pgfpathlineto{\pgfqpoint{1.363035in}{2.389371in}}%
\pgfpathlineto{\pgfqpoint{1.363881in}{2.146837in}}%
\pgfpathlineto{\pgfqpoint{1.364726in}{2.307255in}}%
\pgfpathlineto{\pgfqpoint{1.366417in}{1.885675in}}%
\pgfpathlineto{\pgfqpoint{1.368109in}{2.379069in}}%
\pgfpathlineto{\pgfqpoint{1.368954in}{2.307331in}}%
\pgfpathlineto{\pgfqpoint{1.371491in}{1.947843in}}%
\pgfpathlineto{\pgfqpoint{1.372337in}{2.325057in}}%
\pgfpathlineto{\pgfqpoint{1.373182in}{2.127705in}}%
\pgfpathlineto{\pgfqpoint{1.374874in}{2.239265in}}%
\pgfpathlineto{\pgfqpoint{1.376565in}{2.425240in}}%
\pgfpathlineto{\pgfqpoint{1.377410in}{2.254374in}}%
\pgfpathlineto{\pgfqpoint{1.378256in}{2.270038in}}%
\pgfpathlineto{\pgfqpoint{1.379102in}{2.418370in}}%
\pgfpathlineto{\pgfqpoint{1.379947in}{2.214044in}}%
\pgfpathlineto{\pgfqpoint{1.380793in}{2.540649in}}%
\pgfpathlineto{\pgfqpoint{1.382484in}{1.921882in}}%
\pgfpathlineto{\pgfqpoint{1.383330in}{1.974848in}}%
\pgfpathlineto{\pgfqpoint{1.384175in}{2.265903in}}%
\pgfpathlineto{\pgfqpoint{1.385021in}{2.168079in}}%
\pgfpathlineto{\pgfqpoint{1.385867in}{2.149219in}}%
\pgfpathlineto{\pgfqpoint{1.387558in}{2.315540in}}%
\pgfpathlineto{\pgfqpoint{1.389249in}{1.952764in}}%
\pgfpathlineto{\pgfqpoint{1.390095in}{2.245792in}}%
\pgfpathlineto{\pgfqpoint{1.390940in}{1.924804in}}%
\pgfpathlineto{\pgfqpoint{1.391786in}{2.117996in}}%
\pgfpathlineto{\pgfqpoint{1.392632in}{2.372268in}}%
\pgfpathlineto{\pgfqpoint{1.394323in}{1.850756in}}%
\pgfpathlineto{\pgfqpoint{1.395169in}{2.187908in}}%
\pgfpathlineto{\pgfqpoint{1.396014in}{1.915629in}}%
\pgfpathlineto{\pgfqpoint{1.396860in}{2.187865in}}%
\pgfpathlineto{\pgfqpoint{1.397705in}{2.176869in}}%
\pgfpathlineto{\pgfqpoint{1.399397in}{2.219105in}}%
\pgfpathlineto{\pgfqpoint{1.400242in}{2.166685in}}%
\pgfpathlineto{\pgfqpoint{1.401934in}{1.948303in}}%
\pgfpathlineto{\pgfqpoint{1.403625in}{2.525223in}}%
\pgfpathlineto{\pgfqpoint{1.404470in}{2.209127in}}%
\pgfpathlineto{\pgfqpoint{1.405316in}{2.270362in}}%
\pgfpathlineto{\pgfqpoint{1.407853in}{1.955645in}}%
\pgfpathlineto{\pgfqpoint{1.408699in}{2.147829in}}%
\pgfpathlineto{\pgfqpoint{1.409544in}{1.878726in}}%
\pgfpathlineto{\pgfqpoint{1.411235in}{2.351480in}}%
\pgfpathlineto{\pgfqpoint{1.412081in}{2.261418in}}%
\pgfpathlineto{\pgfqpoint{1.412927in}{2.380818in}}%
\pgfpathlineto{\pgfqpoint{1.413772in}{2.356517in}}%
\pgfpathlineto{\pgfqpoint{1.414618in}{2.369100in}}%
\pgfpathlineto{\pgfqpoint{1.417155in}{2.020847in}}%
\pgfpathlineto{\pgfqpoint{1.418000in}{2.336957in}}%
\pgfpathlineto{\pgfqpoint{1.418846in}{1.944738in}}%
\pgfpathlineto{\pgfqpoint{1.419692in}{2.041980in}}%
\pgfpathlineto{\pgfqpoint{1.420537in}{2.328596in}}%
\pgfpathlineto{\pgfqpoint{1.421383in}{1.975559in}}%
\pgfpathlineto{\pgfqpoint{1.422229in}{2.053338in}}%
\pgfpathlineto{\pgfqpoint{1.423920in}{2.161584in}}%
\pgfpathlineto{\pgfqpoint{1.424765in}{2.119803in}}%
\pgfpathlineto{\pgfqpoint{1.425611in}{1.821762in}}%
\pgfpathlineto{\pgfqpoint{1.426457in}{1.985838in}}%
\pgfpathlineto{\pgfqpoint{1.428148in}{2.517546in}}%
\pgfpathlineto{\pgfqpoint{1.428994in}{1.851596in}}%
\pgfpathlineto{\pgfqpoint{1.429839in}{2.241725in}}%
\pgfpathlineto{\pgfqpoint{1.432376in}{1.910941in}}%
\pgfpathlineto{\pgfqpoint{1.433222in}{2.212182in}}%
\pgfpathlineto{\pgfqpoint{1.434067in}{1.833346in}}%
\pgfpathlineto{\pgfqpoint{1.435759in}{2.429520in}}%
\pgfpathlineto{\pgfqpoint{1.436604in}{2.163094in}}%
\pgfpathlineto{\pgfqpoint{1.437450in}{2.358741in}}%
\pgfpathlineto{\pgfqpoint{1.440832in}{2.007109in}}%
\pgfpathlineto{\pgfqpoint{1.441678in}{2.031564in}}%
\pgfpathlineto{\pgfqpoint{1.445060in}{2.501446in}}%
\pgfpathlineto{\pgfqpoint{1.446752in}{2.180966in}}%
\pgfpathlineto{\pgfqpoint{1.447597in}{2.275986in}}%
\pgfpathlineto{\pgfqpoint{1.449288in}{1.955248in}}%
\pgfpathlineto{\pgfqpoint{1.450134in}{2.002089in}}%
\pgfpathlineto{\pgfqpoint{1.450980in}{2.318846in}}%
\pgfpathlineto{\pgfqpoint{1.451825in}{2.001981in}}%
\pgfpathlineto{\pgfqpoint{1.452671in}{2.067470in}}%
\pgfpathlineto{\pgfqpoint{1.453517in}{2.157264in}}%
\pgfpathlineto{\pgfqpoint{1.454362in}{2.001809in}}%
\pgfpathlineto{\pgfqpoint{1.455208in}{2.322136in}}%
\pgfpathlineto{\pgfqpoint{1.456053in}{2.130046in}}%
\pgfpathlineto{\pgfqpoint{1.458590in}{2.581330in}}%
\pgfpathlineto{\pgfqpoint{1.459436in}{2.114389in}}%
\pgfpathlineto{\pgfqpoint{1.460282in}{2.650308in}}%
\pgfpathlineto{\pgfqpoint{1.461127in}{1.999149in}}%
\pgfpathlineto{\pgfqpoint{1.461973in}{2.196914in}}%
\pgfpathlineto{\pgfqpoint{1.462818in}{2.251130in}}%
\pgfpathlineto{\pgfqpoint{1.463664in}{2.591656in}}%
\pgfpathlineto{\pgfqpoint{1.464510in}{2.138965in}}%
\pgfpathlineto{\pgfqpoint{1.465355in}{2.500882in}}%
\pgfpathlineto{\pgfqpoint{1.466201in}{2.236874in}}%
\pgfpathlineto{\pgfqpoint{1.467047in}{2.501411in}}%
\pgfpathlineto{\pgfqpoint{1.467892in}{2.194449in}}%
\pgfpathlineto{\pgfqpoint{1.468738in}{2.597973in}}%
\pgfpathlineto{\pgfqpoint{1.469583in}{2.174540in}}%
\pgfpathlineto{\pgfqpoint{1.470429in}{2.433851in}}%
\pgfpathlineto{\pgfqpoint{1.472120in}{2.205852in}}%
\pgfpathlineto{\pgfqpoint{1.472966in}{2.269395in}}%
\pgfpathlineto{\pgfqpoint{1.475503in}{1.934133in}}%
\pgfpathlineto{\pgfqpoint{1.478040in}{2.396172in}}%
\pgfpathlineto{\pgfqpoint{1.478885in}{2.258153in}}%
\pgfpathlineto{\pgfqpoint{1.479731in}{2.348262in}}%
\pgfpathlineto{\pgfqpoint{1.482268in}{2.051878in}}%
\pgfpathlineto{\pgfqpoint{1.483113in}{2.305191in}}%
\pgfpathlineto{\pgfqpoint{1.483959in}{2.106823in}}%
\pgfpathlineto{\pgfqpoint{1.484805in}{2.084167in}}%
\pgfpathlineto{\pgfqpoint{1.485650in}{2.434786in}}%
\pgfpathlineto{\pgfqpoint{1.486496in}{2.372023in}}%
\pgfpathlineto{\pgfqpoint{1.487342in}{1.996383in}}%
\pgfpathlineto{\pgfqpoint{1.488187in}{2.251700in}}%
\pgfpathlineto{\pgfqpoint{1.489033in}{2.292722in}}%
\pgfpathlineto{\pgfqpoint{1.489878in}{1.977037in}}%
\pgfpathlineto{\pgfqpoint{1.490724in}{2.425375in}}%
\pgfpathlineto{\pgfqpoint{1.491570in}{2.296238in}}%
\pgfpathlineto{\pgfqpoint{1.492415in}{2.231398in}}%
\pgfpathlineto{\pgfqpoint{1.493261in}{2.612193in}}%
\pgfpathlineto{\pgfqpoint{1.494107in}{2.224833in}}%
\pgfpathlineto{\pgfqpoint{1.494952in}{2.291023in}}%
\pgfpathlineto{\pgfqpoint{1.495798in}{2.487361in}}%
\pgfpathlineto{\pgfqpoint{1.496643in}{2.358624in}}%
\pgfpathlineto{\pgfqpoint{1.499180in}{1.898248in}}%
\pgfpathlineto{\pgfqpoint{1.501717in}{2.747091in}}%
\pgfpathlineto{\pgfqpoint{1.503408in}{1.976086in}}%
\pgfpathlineto{\pgfqpoint{1.505100in}{2.426827in}}%
\pgfpathlineto{\pgfqpoint{1.505945in}{2.226937in}}%
\pgfpathlineto{\pgfqpoint{1.506791in}{2.218610in}}%
\pgfpathlineto{\pgfqpoint{1.509328in}{2.081178in}}%
\pgfpathlineto{\pgfqpoint{1.510173in}{2.250483in}}%
\pgfpathlineto{\pgfqpoint{1.511865in}{1.970408in}}%
\pgfpathlineto{\pgfqpoint{1.512710in}{2.241234in}}%
\pgfpathlineto{\pgfqpoint{1.513556in}{2.212338in}}%
\pgfpathlineto{\pgfqpoint{1.514402in}{2.019423in}}%
\pgfpathlineto{\pgfqpoint{1.515247in}{2.034344in}}%
\pgfpathlineto{\pgfqpoint{1.516093in}{2.597854in}}%
\pgfpathlineto{\pgfqpoint{1.516938in}{2.338071in}}%
\pgfpathlineto{\pgfqpoint{1.517784in}{2.324600in}}%
\pgfpathlineto{\pgfqpoint{1.518630in}{2.351214in}}%
\pgfpathlineto{\pgfqpoint{1.519475in}{2.187312in}}%
\pgfpathlineto{\pgfqpoint{1.520321in}{2.661555in}}%
\pgfpathlineto{\pgfqpoint{1.522012in}{2.009134in}}%
\pgfpathlineto{\pgfqpoint{1.522858in}{2.258279in}}%
\pgfpathlineto{\pgfqpoint{1.523703in}{2.190996in}}%
\pgfpathlineto{\pgfqpoint{1.524549in}{1.737508in}}%
\pgfpathlineto{\pgfqpoint{1.525395in}{2.386213in}}%
\pgfpathlineto{\pgfqpoint{1.526240in}{2.181039in}}%
\pgfpathlineto{\pgfqpoint{1.527086in}{2.059553in}}%
\pgfpathlineto{\pgfqpoint{1.527931in}{2.410554in}}%
\pgfpathlineto{\pgfqpoint{1.528777in}{2.020309in}}%
\pgfpathlineto{\pgfqpoint{1.529623in}{2.374856in}}%
\pgfpathlineto{\pgfqpoint{1.530468in}{1.774757in}}%
\pgfpathlineto{\pgfqpoint{1.531314in}{2.171379in}}%
\pgfpathlineto{\pgfqpoint{1.532160in}{2.432465in}}%
\pgfpathlineto{\pgfqpoint{1.533005in}{2.161389in}}%
\pgfpathlineto{\pgfqpoint{1.533851in}{2.348296in}}%
\pgfpathlineto{\pgfqpoint{1.534696in}{2.285729in}}%
\pgfpathlineto{\pgfqpoint{1.535542in}{2.483563in}}%
\pgfpathlineto{\pgfqpoint{1.537233in}{2.222932in}}%
\pgfpathlineto{\pgfqpoint{1.538079in}{2.275109in}}%
\pgfpathlineto{\pgfqpoint{1.539770in}{1.962502in}}%
\pgfpathlineto{\pgfqpoint{1.540616in}{2.527334in}}%
\pgfpathlineto{\pgfqpoint{1.541461in}{2.041170in}}%
\pgfpathlineto{\pgfqpoint{1.542307in}{2.190181in}}%
\pgfpathlineto{\pgfqpoint{1.543153in}{2.652602in}}%
\pgfpathlineto{\pgfqpoint{1.543998in}{2.613384in}}%
\pgfpathlineto{\pgfqpoint{1.546535in}{2.118845in}}%
\pgfpathlineto{\pgfqpoint{1.547381in}{2.448515in}}%
\pgfpathlineto{\pgfqpoint{1.548226in}{2.336322in}}%
\pgfpathlineto{\pgfqpoint{1.551609in}{2.007699in}}%
\pgfpathlineto{\pgfqpoint{1.552455in}{2.253823in}}%
\pgfpathlineto{\pgfqpoint{1.553300in}{2.138086in}}%
\pgfpathlineto{\pgfqpoint{1.554146in}{2.105311in}}%
\pgfpathlineto{\pgfqpoint{1.554991in}{2.250553in}}%
\pgfpathlineto{\pgfqpoint{1.555837in}{2.212804in}}%
\pgfpathlineto{\pgfqpoint{1.556683in}{1.907840in}}%
\pgfpathlineto{\pgfqpoint{1.557528in}{1.953081in}}%
\pgfpathlineto{\pgfqpoint{1.558374in}{2.362739in}}%
\pgfpathlineto{\pgfqpoint{1.559220in}{2.215244in}}%
\pgfpathlineto{\pgfqpoint{1.560065in}{2.408228in}}%
\pgfpathlineto{\pgfqpoint{1.560911in}{2.315137in}}%
\pgfpathlineto{\pgfqpoint{1.561756in}{2.109015in}}%
\pgfpathlineto{\pgfqpoint{1.562602in}{2.731685in}}%
\pgfpathlineto{\pgfqpoint{1.563448in}{2.305465in}}%
\pgfpathlineto{\pgfqpoint{1.564293in}{2.208625in}}%
\pgfpathlineto{\pgfqpoint{1.565139in}{1.849774in}}%
\pgfpathlineto{\pgfqpoint{1.566830in}{2.616104in}}%
\pgfpathlineto{\pgfqpoint{1.568521in}{2.156262in}}%
\pgfpathlineto{\pgfqpoint{1.569367in}{2.387852in}}%
\pgfpathlineto{\pgfqpoint{1.570213in}{1.973258in}}%
\pgfpathlineto{\pgfqpoint{1.571058in}{2.029454in}}%
\pgfpathlineto{\pgfqpoint{1.572750in}{2.563864in}}%
\pgfpathlineto{\pgfqpoint{1.573595in}{2.263606in}}%
\pgfpathlineto{\pgfqpoint{1.574441in}{2.130304in}}%
\pgfpathlineto{\pgfqpoint{1.576132in}{2.469575in}}%
\pgfpathlineto{\pgfqpoint{1.576978in}{1.976910in}}%
\pgfpathlineto{\pgfqpoint{1.577823in}{2.346783in}}%
\pgfpathlineto{\pgfqpoint{1.579515in}{2.076452in}}%
\pgfpathlineto{\pgfqpoint{1.581206in}{1.690366in}}%
\pgfpathlineto{\pgfqpoint{1.582051in}{2.407494in}}%
\pgfpathlineto{\pgfqpoint{1.582897in}{1.940232in}}%
\pgfpathlineto{\pgfqpoint{1.583743in}{2.313939in}}%
\pgfpathlineto{\pgfqpoint{1.584588in}{1.601355in}}%
\pgfpathlineto{\pgfqpoint{1.585434in}{2.068493in}}%
\pgfpathlineto{\pgfqpoint{1.587125in}{2.412130in}}%
\pgfpathlineto{\pgfqpoint{1.587971in}{2.276440in}}%
\pgfpathlineto{\pgfqpoint{1.590508in}{1.944637in}}%
\pgfpathlineto{\pgfqpoint{1.591353in}{1.954088in}}%
\pgfpathlineto{\pgfqpoint{1.593045in}{2.301131in}}%
\pgfpathlineto{\pgfqpoint{1.593890in}{2.076808in}}%
\pgfpathlineto{\pgfqpoint{1.594736in}{2.376228in}}%
\pgfpathlineto{\pgfqpoint{1.595581in}{2.018758in}}%
\pgfpathlineto{\pgfqpoint{1.596427in}{2.157275in}}%
\pgfpathlineto{\pgfqpoint{1.597273in}{2.192191in}}%
\pgfpathlineto{\pgfqpoint{1.598118in}{2.565489in}}%
\pgfpathlineto{\pgfqpoint{1.598964in}{1.720386in}}%
\pgfpathlineto{\pgfqpoint{1.599810in}{2.231027in}}%
\pgfpathlineto{\pgfqpoint{1.600655in}{2.032977in}}%
\pgfpathlineto{\pgfqpoint{1.601501in}{2.041105in}}%
\pgfpathlineto{\pgfqpoint{1.604038in}{2.662019in}}%
\pgfpathlineto{\pgfqpoint{1.604883in}{1.792354in}}%
\pgfpathlineto{\pgfqpoint{1.605729in}{2.354011in}}%
\pgfpathlineto{\pgfqpoint{1.606574in}{2.347708in}}%
\pgfpathlineto{\pgfqpoint{1.607420in}{2.033145in}}%
\pgfpathlineto{\pgfqpoint{1.608266in}{2.232965in}}%
\pgfpathlineto{\pgfqpoint{1.609111in}{2.075865in}}%
\pgfpathlineto{\pgfqpoint{1.609957in}{2.195992in}}%
\pgfpathlineto{\pgfqpoint{1.610803in}{2.470250in}}%
\pgfpathlineto{\pgfqpoint{1.611648in}{2.270007in}}%
\pgfpathlineto{\pgfqpoint{1.612494in}{1.859978in}}%
\pgfpathlineto{\pgfqpoint{1.613339in}{2.349793in}}%
\pgfpathlineto{\pgfqpoint{1.614185in}{1.874605in}}%
\pgfpathlineto{\pgfqpoint{1.615876in}{2.396170in}}%
\pgfpathlineto{\pgfqpoint{1.616722in}{1.931451in}}%
\pgfpathlineto{\pgfqpoint{1.617568in}{2.194097in}}%
\pgfpathlineto{\pgfqpoint{1.618413in}{2.244870in}}%
\pgfpathlineto{\pgfqpoint{1.619259in}{2.432863in}}%
\pgfpathlineto{\pgfqpoint{1.620104in}{1.886402in}}%
\pgfpathlineto{\pgfqpoint{1.620950in}{2.267972in}}%
\pgfpathlineto{\pgfqpoint{1.622641in}{2.546063in}}%
\pgfpathlineto{\pgfqpoint{1.623487in}{2.171580in}}%
\pgfpathlineto{\pgfqpoint{1.624333in}{2.303967in}}%
\pgfpathlineto{\pgfqpoint{1.625178in}{2.130913in}}%
\pgfpathlineto{\pgfqpoint{1.626024in}{2.156033in}}%
\pgfpathlineto{\pgfqpoint{1.627715in}{2.294474in}}%
\pgfpathlineto{\pgfqpoint{1.628561in}{2.605789in}}%
\pgfpathlineto{\pgfqpoint{1.630252in}{1.976473in}}%
\pgfpathlineto{\pgfqpoint{1.631098in}{2.310504in}}%
\pgfpathlineto{\pgfqpoint{1.631943in}{2.034493in}}%
\pgfpathlineto{\pgfqpoint{1.632789in}{2.151163in}}%
\pgfpathlineto{\pgfqpoint{1.633634in}{2.162844in}}%
\pgfpathlineto{\pgfqpoint{1.634480in}{2.113575in}}%
\pgfpathlineto{\pgfqpoint{1.635326in}{2.545215in}}%
\pgfpathlineto{\pgfqpoint{1.636171in}{2.279537in}}%
\pgfpathlineto{\pgfqpoint{1.637017in}{1.963599in}}%
\pgfpathlineto{\pgfqpoint{1.637863in}{2.103651in}}%
\pgfpathlineto{\pgfqpoint{1.638708in}{2.040176in}}%
\pgfpathlineto{\pgfqpoint{1.639554in}{2.671755in}}%
\pgfpathlineto{\pgfqpoint{1.641245in}{2.086995in}}%
\pgfpathlineto{\pgfqpoint{1.642091in}{2.101442in}}%
\pgfpathlineto{\pgfqpoint{1.642936in}{2.295382in}}%
\pgfpathlineto{\pgfqpoint{1.643782in}{2.294758in}}%
\pgfpathlineto{\pgfqpoint{1.645473in}{1.921681in}}%
\pgfpathlineto{\pgfqpoint{1.646319in}{2.159980in}}%
\pgfpathlineto{\pgfqpoint{1.648010in}{2.060375in}}%
\pgfpathlineto{\pgfqpoint{1.649701in}{2.338013in}}%
\pgfpathlineto{\pgfqpoint{1.651393in}{1.958973in}}%
\pgfpathlineto{\pgfqpoint{1.653084in}{2.234455in}}%
\pgfpathlineto{\pgfqpoint{1.653929in}{2.105977in}}%
\pgfpathlineto{\pgfqpoint{1.654775in}{2.124540in}}%
\pgfpathlineto{\pgfqpoint{1.655621in}{2.297783in}}%
\pgfpathlineto{\pgfqpoint{1.656466in}{2.026952in}}%
\pgfpathlineto{\pgfqpoint{1.657312in}{2.055224in}}%
\pgfpathlineto{\pgfqpoint{1.658158in}{2.198505in}}%
\pgfpathlineto{\pgfqpoint{1.659003in}{2.089597in}}%
\pgfpathlineto{\pgfqpoint{1.661540in}{1.887653in}}%
\pgfpathlineto{\pgfqpoint{1.664077in}{2.647758in}}%
\pgfpathlineto{\pgfqpoint{1.664923in}{2.257941in}}%
\pgfpathlineto{\pgfqpoint{1.665768in}{2.258323in}}%
\pgfpathlineto{\pgfqpoint{1.666614in}{2.196969in}}%
\pgfpathlineto{\pgfqpoint{1.668305in}{1.863844in}}%
\pgfpathlineto{\pgfqpoint{1.669151in}{2.127616in}}%
\pgfpathlineto{\pgfqpoint{1.669996in}{1.870419in}}%
\pgfpathlineto{\pgfqpoint{1.672533in}{2.325038in}}%
\pgfpathlineto{\pgfqpoint{1.675916in}{2.069646in}}%
\pgfpathlineto{\pgfqpoint{1.676761in}{2.379995in}}%
\pgfpathlineto{\pgfqpoint{1.677607in}{2.052491in}}%
\pgfpathlineto{\pgfqpoint{1.678453in}{2.329212in}}%
\pgfpathlineto{\pgfqpoint{1.679298in}{1.995435in}}%
\pgfpathlineto{\pgfqpoint{1.680144in}{2.058933in}}%
\pgfpathlineto{\pgfqpoint{1.680989in}{1.953300in}}%
\pgfpathlineto{\pgfqpoint{1.681835in}{2.200118in}}%
\pgfpathlineto{\pgfqpoint{1.682681in}{2.070429in}}%
\pgfpathlineto{\pgfqpoint{1.683526in}{2.009279in}}%
\pgfpathlineto{\pgfqpoint{1.684372in}{2.065508in}}%
\pgfpathlineto{\pgfqpoint{1.685217in}{2.386176in}}%
\pgfpathlineto{\pgfqpoint{1.686909in}{1.881566in}}%
\pgfpathlineto{\pgfqpoint{1.687754in}{2.306062in}}%
\pgfpathlineto{\pgfqpoint{1.688600in}{2.066871in}}%
\pgfpathlineto{\pgfqpoint{1.689446in}{2.256198in}}%
\pgfpathlineto{\pgfqpoint{1.691137in}{1.938213in}}%
\pgfpathlineto{\pgfqpoint{1.693674in}{2.490773in}}%
\pgfpathlineto{\pgfqpoint{1.696211in}{2.072890in}}%
\pgfpathlineto{\pgfqpoint{1.697056in}{2.168063in}}%
\pgfpathlineto{\pgfqpoint{1.697902in}{2.501219in}}%
\pgfpathlineto{\pgfqpoint{1.700439in}{1.878937in}}%
\pgfpathlineto{\pgfqpoint{1.702130in}{2.132597in}}%
\pgfpathlineto{\pgfqpoint{1.702976in}{2.142876in}}%
\pgfpathlineto{\pgfqpoint{1.704667in}{2.261816in}}%
\pgfpathlineto{\pgfqpoint{1.705512in}{2.197340in}}%
\pgfpathlineto{\pgfqpoint{1.706358in}{1.935422in}}%
\pgfpathlineto{\pgfqpoint{1.707204in}{2.324682in}}%
\pgfpathlineto{\pgfqpoint{1.708049in}{2.257722in}}%
\pgfpathlineto{\pgfqpoint{1.708895in}{1.933611in}}%
\pgfpathlineto{\pgfqpoint{1.709741in}{2.195261in}}%
\pgfpathlineto{\pgfqpoint{1.711432in}{1.996443in}}%
\pgfpathlineto{\pgfqpoint{1.712277in}{2.194415in}}%
\pgfpathlineto{\pgfqpoint{1.713123in}{1.999681in}}%
\pgfpathlineto{\pgfqpoint{1.713969in}{2.264951in}}%
\pgfpathlineto{\pgfqpoint{1.714814in}{2.068420in}}%
\pgfpathlineto{\pgfqpoint{1.718197in}{2.450977in}}%
\pgfpathlineto{\pgfqpoint{1.719042in}{2.064023in}}%
\pgfpathlineto{\pgfqpoint{1.719888in}{2.390171in}}%
\pgfpathlineto{\pgfqpoint{1.720734in}{1.999751in}}%
\pgfpathlineto{\pgfqpoint{1.723271in}{2.520877in}}%
\pgfpathlineto{\pgfqpoint{1.724116in}{2.062748in}}%
\pgfpathlineto{\pgfqpoint{1.724962in}{2.386742in}}%
\pgfpathlineto{\pgfqpoint{1.725807in}{2.456078in}}%
\pgfpathlineto{\pgfqpoint{1.726653in}{1.997989in}}%
\pgfpathlineto{\pgfqpoint{1.727499in}{2.583842in}}%
\pgfpathlineto{\pgfqpoint{1.728344in}{2.387864in}}%
\pgfpathlineto{\pgfqpoint{1.729190in}{2.000708in}}%
\pgfpathlineto{\pgfqpoint{1.730036in}{2.523141in}}%
\pgfpathlineto{\pgfqpoint{1.730881in}{2.059694in}}%
\pgfpathlineto{\pgfqpoint{1.731727in}{2.256653in}}%
\pgfpathlineto{\pgfqpoint{1.732572in}{2.260761in}}%
\pgfpathlineto{\pgfqpoint{1.733418in}{2.326389in}}%
\pgfpathlineto{\pgfqpoint{1.734264in}{2.195655in}}%
\pgfpathlineto{\pgfqpoint{1.735109in}{2.328703in}}%
\pgfpathlineto{\pgfqpoint{1.735955in}{2.126856in}}%
\pgfpathlineto{\pgfqpoint{1.738492in}{2.650720in}}%
\pgfpathlineto{\pgfqpoint{1.739337in}{2.452734in}}%
\pgfpathlineto{\pgfqpoint{1.740183in}{1.870406in}}%
\pgfpathlineto{\pgfqpoint{1.741029in}{1.999514in}}%
\pgfpathlineto{\pgfqpoint{1.741874in}{1.941848in}}%
\pgfpathlineto{\pgfqpoint{1.744411in}{2.256109in}}%
\pgfpathlineto{\pgfqpoint{1.745257in}{2.328818in}}%
\pgfpathlineto{\pgfqpoint{1.746102in}{2.125309in}}%
\pgfpathlineto{\pgfqpoint{1.746948in}{2.311023in}}%
\pgfpathlineto{\pgfqpoint{1.748639in}{1.871401in}}%
\pgfpathlineto{\pgfqpoint{1.749485in}{2.649567in}}%
\pgfpathlineto{\pgfqpoint{1.750331in}{2.059755in}}%
\pgfpathlineto{\pgfqpoint{1.751176in}{1.931029in}}%
\pgfpathlineto{\pgfqpoint{1.753713in}{2.325844in}}%
\pgfpathlineto{\pgfqpoint{1.754559in}{2.067765in}}%
\pgfpathlineto{\pgfqpoint{1.755404in}{2.119492in}}%
\pgfpathlineto{\pgfqpoint{1.757096in}{2.358683in}}%
\pgfpathlineto{\pgfqpoint{1.757941in}{2.247992in}}%
\pgfpathlineto{\pgfqpoint{1.758787in}{2.248288in}}%
\pgfpathlineto{\pgfqpoint{1.759632in}{1.850679in}}%
\pgfpathlineto{\pgfqpoint{1.761324in}{2.538979in}}%
\pgfpathlineto{\pgfqpoint{1.762169in}{2.434243in}}%
\pgfpathlineto{\pgfqpoint{1.763015in}{2.100732in}}%
\pgfpathlineto{\pgfqpoint{1.763860in}{2.355761in}}%
\pgfpathlineto{\pgfqpoint{1.764706in}{2.095845in}}%
\pgfpathlineto{\pgfqpoint{1.765552in}{2.422937in}}%
\pgfpathlineto{\pgfqpoint{1.767243in}{1.924743in}}%
\pgfpathlineto{\pgfqpoint{1.769780in}{2.351204in}}%
\pgfpathlineto{\pgfqpoint{1.770625in}{2.221528in}}%
\pgfpathlineto{\pgfqpoint{1.771471in}{2.739980in}}%
\pgfpathlineto{\pgfqpoint{1.773162in}{1.957652in}}%
\pgfpathlineto{\pgfqpoint{1.774008in}{1.969377in}}%
\pgfpathlineto{\pgfqpoint{1.774854in}{1.844641in}}%
\pgfpathlineto{\pgfqpoint{1.775699in}{2.360092in}}%
\pgfpathlineto{\pgfqpoint{1.776545in}{2.283737in}}%
\pgfpathlineto{\pgfqpoint{1.777390in}{2.210065in}}%
\pgfpathlineto{\pgfqpoint{1.778236in}{1.904986in}}%
\pgfpathlineto{\pgfqpoint{1.779082in}{2.363349in}}%
\pgfpathlineto{\pgfqpoint{1.779927in}{2.232866in}}%
\pgfpathlineto{\pgfqpoint{1.782464in}{2.041890in}}%
\pgfpathlineto{\pgfqpoint{1.784155in}{2.285155in}}%
\pgfpathlineto{\pgfqpoint{1.785001in}{1.976438in}}%
\pgfpathlineto{\pgfqpoint{1.785847in}{2.414801in}}%
\pgfpathlineto{\pgfqpoint{1.786692in}{2.046382in}}%
\pgfpathlineto{\pgfqpoint{1.787538in}{2.488160in}}%
\pgfpathlineto{\pgfqpoint{1.790075in}{1.900296in}}%
\pgfpathlineto{\pgfqpoint{1.790920in}{2.165778in}}%
\pgfpathlineto{\pgfqpoint{1.791766in}{1.971929in}}%
\pgfpathlineto{\pgfqpoint{1.792612in}{1.964230in}}%
\pgfpathlineto{\pgfqpoint{1.795149in}{2.355182in}}%
\pgfpathlineto{\pgfqpoint{1.795994in}{1.822113in}}%
\pgfpathlineto{\pgfqpoint{1.796840in}{2.489154in}}%
\pgfpathlineto{\pgfqpoint{1.797685in}{2.423749in}}%
\pgfpathlineto{\pgfqpoint{1.798531in}{2.031209in}}%
\pgfpathlineto{\pgfqpoint{1.799377in}{2.423241in}}%
\pgfpathlineto{\pgfqpoint{1.800222in}{2.160423in}}%
\pgfpathlineto{\pgfqpoint{1.801068in}{1.915999in}}%
\pgfpathlineto{\pgfqpoint{1.801914in}{2.088373in}}%
\pgfpathlineto{\pgfqpoint{1.802759in}{1.896645in}}%
\pgfpathlineto{\pgfqpoint{1.804450in}{2.242941in}}%
\pgfpathlineto{\pgfqpoint{1.805296in}{2.231196in}}%
\pgfpathlineto{\pgfqpoint{1.806142in}{2.187476in}}%
\pgfpathlineto{\pgfqpoint{1.806987in}{2.223201in}}%
\pgfpathlineto{\pgfqpoint{1.807833in}{2.353372in}}%
\pgfpathlineto{\pgfqpoint{1.809524in}{1.901466in}}%
\pgfpathlineto{\pgfqpoint{1.811215in}{2.226999in}}%
\pgfpathlineto{\pgfqpoint{1.813752in}{1.849693in}}%
\pgfpathlineto{\pgfqpoint{1.814598in}{2.293797in}}%
\pgfpathlineto{\pgfqpoint{1.815444in}{1.970996in}}%
\pgfpathlineto{\pgfqpoint{1.816289in}{2.360267in}}%
\pgfpathlineto{\pgfqpoint{1.817135in}{2.142235in}}%
\pgfpathlineto{\pgfqpoint{1.817980in}{2.031020in}}%
\pgfpathlineto{\pgfqpoint{1.819672in}{2.378839in}}%
\pgfpathlineto{\pgfqpoint{1.822209in}{1.559522in}}%
\pgfpathlineto{\pgfqpoint{1.823054in}{1.946342in}}%
\pgfpathlineto{\pgfqpoint{1.824745in}{2.322933in}}%
\pgfpathlineto{\pgfqpoint{1.825591in}{1.939601in}}%
\pgfpathlineto{\pgfqpoint{1.826437in}{2.139525in}}%
\pgfpathlineto{\pgfqpoint{1.828974in}{2.401161in}}%
\pgfpathlineto{\pgfqpoint{1.829819in}{2.523739in}}%
\pgfpathlineto{\pgfqpoint{1.830665in}{1.919940in}}%
\pgfpathlineto{\pgfqpoint{1.831510in}{2.123019in}}%
\pgfpathlineto{\pgfqpoint{1.832356in}{1.932546in}}%
\pgfpathlineto{\pgfqpoint{1.834893in}{2.327473in}}%
\pgfpathlineto{\pgfqpoint{1.835739in}{1.928678in}}%
\pgfpathlineto{\pgfqpoint{1.836584in}{2.313991in}}%
\pgfpathlineto{\pgfqpoint{1.837430in}{2.236851in}}%
\pgfpathlineto{\pgfqpoint{1.838275in}{2.290764in}}%
\pgfpathlineto{\pgfqpoint{1.839121in}{1.715690in}}%
\pgfpathlineto{\pgfqpoint{1.840812in}{2.371345in}}%
\pgfpathlineto{\pgfqpoint{1.843349in}{1.624083in}}%
\pgfpathlineto{\pgfqpoint{1.844195in}{2.398180in}}%
\pgfpathlineto{\pgfqpoint{1.845040in}{2.148986in}}%
\pgfpathlineto{\pgfqpoint{1.845886in}{1.879003in}}%
\pgfpathlineto{\pgfqpoint{1.846732in}{2.046560in}}%
\pgfpathlineto{\pgfqpoint{1.847577in}{2.458157in}}%
\pgfpathlineto{\pgfqpoint{1.848423in}{2.314011in}}%
\pgfpathlineto{\pgfqpoint{1.849268in}{2.389620in}}%
\pgfpathlineto{\pgfqpoint{1.850960in}{1.996643in}}%
\pgfpathlineto{\pgfqpoint{1.851805in}{2.415154in}}%
\pgfpathlineto{\pgfqpoint{1.852651in}{2.129281in}}%
\pgfpathlineto{\pgfqpoint{1.853497in}{2.196521in}}%
\pgfpathlineto{\pgfqpoint{1.854342in}{2.587193in}}%
\pgfpathlineto{\pgfqpoint{1.855188in}{2.194180in}}%
\pgfpathlineto{\pgfqpoint{1.856033in}{2.388881in}}%
\pgfpathlineto{\pgfqpoint{1.856879in}{2.651668in}}%
\pgfpathlineto{\pgfqpoint{1.858570in}{1.933195in}}%
\pgfpathlineto{\pgfqpoint{1.859416in}{2.193801in}}%
\pgfpathlineto{\pgfqpoint{1.860262in}{2.064246in}}%
\pgfpathlineto{\pgfqpoint{1.862798in}{2.585323in}}%
\pgfpathlineto{\pgfqpoint{1.863644in}{1.812039in}}%
\pgfpathlineto{\pgfqpoint{1.864490in}{2.456748in}}%
\pgfpathlineto{\pgfqpoint{1.865335in}{1.931213in}}%
\pgfpathlineto{\pgfqpoint{1.867027in}{2.259292in}}%
\pgfpathlineto{\pgfqpoint{1.867872in}{2.065144in}}%
\pgfpathlineto{\pgfqpoint{1.868718in}{2.124284in}}%
\pgfpathlineto{\pgfqpoint{1.869563in}{2.127222in}}%
\pgfpathlineto{\pgfqpoint{1.870409in}{2.256925in}}%
\pgfpathlineto{\pgfqpoint{1.871255in}{1.927022in}}%
\pgfpathlineto{\pgfqpoint{1.873792in}{2.483503in}}%
\pgfpathlineto{\pgfqpoint{1.875483in}{2.098514in}}%
\pgfpathlineto{\pgfqpoint{1.876328in}{2.335190in}}%
\pgfpathlineto{\pgfqpoint{1.877174in}{1.964635in}}%
\pgfpathlineto{\pgfqpoint{1.879711in}{2.418816in}}%
\pgfpathlineto{\pgfqpoint{1.881402in}{1.967604in}}%
\pgfpathlineto{\pgfqpoint{1.882248in}{2.174773in}}%
\pgfpathlineto{\pgfqpoint{1.883093in}{2.032321in}}%
\pgfpathlineto{\pgfqpoint{1.883939in}{2.226841in}}%
\pgfpathlineto{\pgfqpoint{1.884785in}{2.098286in}}%
\pgfpathlineto{\pgfqpoint{1.885630in}{1.901101in}}%
\pgfpathlineto{\pgfqpoint{1.886476in}{2.403509in}}%
\pgfpathlineto{\pgfqpoint{1.887322in}{2.374387in}}%
\pgfpathlineto{\pgfqpoint{1.889013in}{1.814045in}}%
\pgfpathlineto{\pgfqpoint{1.889858in}{2.472023in}}%
\pgfpathlineto{\pgfqpoint{1.890704in}{2.098118in}}%
\pgfpathlineto{\pgfqpoint{1.891550in}{2.294229in}}%
\pgfpathlineto{\pgfqpoint{1.892395in}{2.011742in}}%
\pgfpathlineto{\pgfqpoint{1.894087in}{2.303221in}}%
\pgfpathlineto{\pgfqpoint{1.894932in}{2.007348in}}%
\pgfpathlineto{\pgfqpoint{1.895778in}{2.419267in}}%
\pgfpathlineto{\pgfqpoint{1.896623in}{2.081431in}}%
\pgfpathlineto{\pgfqpoint{1.897469in}{2.005799in}}%
\pgfpathlineto{\pgfqpoint{1.899160in}{2.522040in}}%
\pgfpathlineto{\pgfqpoint{1.900006in}{2.129980in}}%
\pgfpathlineto{\pgfqpoint{1.900852in}{2.520977in}}%
\pgfpathlineto{\pgfqpoint{1.901697in}{2.390424in}}%
\pgfpathlineto{\pgfqpoint{1.902543in}{1.802740in}}%
\pgfpathlineto{\pgfqpoint{1.905080in}{2.521361in}}%
\pgfpathlineto{\pgfqpoint{1.905925in}{2.063778in}}%
\pgfpathlineto{\pgfqpoint{1.906771in}{2.063825in}}%
\pgfpathlineto{\pgfqpoint{1.907617in}{2.392188in}}%
\pgfpathlineto{\pgfqpoint{1.908462in}{2.260652in}}%
\pgfpathlineto{\pgfqpoint{1.910153in}{1.867960in}}%
\pgfpathlineto{\pgfqpoint{1.910999in}{2.325596in}}%
\pgfpathlineto{\pgfqpoint{1.911845in}{2.264485in}}%
\pgfpathlineto{\pgfqpoint{1.912690in}{1.933235in}}%
\pgfpathlineto{\pgfqpoint{1.915227in}{2.455335in}}%
\pgfpathlineto{\pgfqpoint{1.916073in}{2.063473in}}%
\pgfpathlineto{\pgfqpoint{1.916918in}{2.521375in}}%
\pgfpathlineto{\pgfqpoint{1.917764in}{2.194093in}}%
\pgfpathlineto{\pgfqpoint{1.918610in}{1.997750in}}%
\pgfpathlineto{\pgfqpoint{1.919455in}{2.325243in}}%
\pgfpathlineto{\pgfqpoint{1.920301in}{1.867713in}}%
\pgfpathlineto{\pgfqpoint{1.921147in}{2.194641in}}%
\pgfpathlineto{\pgfqpoint{1.921992in}{2.259788in}}%
\pgfpathlineto{\pgfqpoint{1.923683in}{2.128378in}}%
\pgfpathlineto{\pgfqpoint{1.924529in}{2.258312in}}%
\pgfpathlineto{\pgfqpoint{1.925375in}{2.063533in}}%
\pgfpathlineto{\pgfqpoint{1.927911in}{2.457067in}}%
\pgfpathlineto{\pgfqpoint{1.930448in}{1.933384in}}%
\pgfpathlineto{\pgfqpoint{1.932140in}{1.998425in}}%
\pgfpathlineto{\pgfqpoint{1.933831in}{2.196557in}}%
\pgfpathlineto{\pgfqpoint{1.934676in}{2.188211in}}%
\pgfpathlineto{\pgfqpoint{1.935522in}{2.129372in}}%
\pgfpathlineto{\pgfqpoint{1.936368in}{1.935264in}}%
\pgfpathlineto{\pgfqpoint{1.938905in}{2.401810in}}%
\pgfpathlineto{\pgfqpoint{1.939750in}{1.916737in}}%
\pgfpathlineto{\pgfqpoint{1.940596in}{2.749197in}}%
\pgfpathlineto{\pgfqpoint{1.941441in}{2.222115in}}%
\pgfpathlineto{\pgfqpoint{1.943133in}{2.033148in}}%
\pgfpathlineto{\pgfqpoint{1.943978in}{2.337717in}}%
\pgfpathlineto{\pgfqpoint{1.944824in}{1.896978in}}%
\pgfpathlineto{\pgfqpoint{1.945670in}{2.160917in}}%
\pgfpathlineto{\pgfqpoint{1.946515in}{2.553256in}}%
\pgfpathlineto{\pgfqpoint{1.947361in}{2.031308in}}%
\pgfpathlineto{\pgfqpoint{1.949052in}{2.677884in}}%
\pgfpathlineto{\pgfqpoint{1.949898in}{1.646276in}}%
\pgfpathlineto{\pgfqpoint{1.950743in}{2.194271in}}%
\pgfpathlineto{\pgfqpoint{1.951589in}{2.440576in}}%
\pgfpathlineto{\pgfqpoint{1.952435in}{2.324603in}}%
\pgfpathlineto{\pgfqpoint{1.953280in}{2.163923in}}%
\pgfpathlineto{\pgfqpoint{1.954126in}{2.419463in}}%
\pgfpathlineto{\pgfqpoint{1.955817in}{2.036323in}}%
\pgfpathlineto{\pgfqpoint{1.956663in}{2.282688in}}%
\pgfpathlineto{\pgfqpoint{1.957508in}{1.968213in}}%
\pgfpathlineto{\pgfqpoint{1.958354in}{2.357881in}}%
\pgfpathlineto{\pgfqpoint{1.959200in}{2.096705in}}%
\pgfpathlineto{\pgfqpoint{1.961736in}{2.313158in}}%
\pgfpathlineto{\pgfqpoint{1.962582in}{1.965605in}}%
\pgfpathlineto{\pgfqpoint{1.963428in}{2.422054in}}%
\pgfpathlineto{\pgfqpoint{1.964273in}{2.291400in}}%
\pgfpathlineto{\pgfqpoint{1.965119in}{2.292922in}}%
\pgfpathlineto{\pgfqpoint{1.965965in}{2.554416in}}%
\pgfpathlineto{\pgfqpoint{1.968501in}{1.967909in}}%
\pgfpathlineto{\pgfqpoint{1.969347in}{2.330655in}}%
\pgfpathlineto{\pgfqpoint{1.970193in}{1.835132in}}%
\pgfpathlineto{\pgfqpoint{1.971038in}{1.966903in}}%
\pgfpathlineto{\pgfqpoint{1.971884in}{2.554279in}}%
\pgfpathlineto{\pgfqpoint{1.972730in}{2.291792in}}%
\pgfpathlineto{\pgfqpoint{1.973575in}{2.289856in}}%
\pgfpathlineto{\pgfqpoint{1.974421in}{2.292750in}}%
\pgfpathlineto{\pgfqpoint{1.975266in}{2.228516in}}%
\pgfpathlineto{\pgfqpoint{1.976112in}{1.967555in}}%
\pgfpathlineto{\pgfqpoint{1.976958in}{2.031844in}}%
\pgfpathlineto{\pgfqpoint{1.978649in}{2.357691in}}%
\pgfpathlineto{\pgfqpoint{1.980340in}{1.836484in}}%
\pgfpathlineto{\pgfqpoint{1.981186in}{2.619269in}}%
\pgfpathlineto{\pgfqpoint{1.982031in}{2.489766in}}%
\pgfpathlineto{\pgfqpoint{1.982877in}{2.161796in}}%
\pgfpathlineto{\pgfqpoint{1.983723in}{2.292965in}}%
\pgfpathlineto{\pgfqpoint{1.984568in}{2.226920in}}%
\pgfpathlineto{\pgfqpoint{1.985414in}{1.833885in}}%
\pgfpathlineto{\pgfqpoint{1.986260in}{2.357017in}}%
\pgfpathlineto{\pgfqpoint{1.987105in}{1.964682in}}%
\pgfpathlineto{\pgfqpoint{1.987951in}{1.704790in}}%
\pgfpathlineto{\pgfqpoint{1.988796in}{2.288110in}}%
\pgfpathlineto{\pgfqpoint{1.989642in}{2.100146in}}%
\pgfpathlineto{\pgfqpoint{1.990488in}{1.968596in}}%
\pgfpathlineto{\pgfqpoint{1.991333in}{2.030499in}}%
\pgfpathlineto{\pgfqpoint{1.992179in}{2.285824in}}%
\pgfpathlineto{\pgfqpoint{1.993025in}{2.164465in}}%
\pgfpathlineto{\pgfqpoint{1.993870in}{1.901575in}}%
\pgfpathlineto{\pgfqpoint{1.994716in}{1.966310in}}%
\pgfpathlineto{\pgfqpoint{1.995561in}{2.048601in}}%
\pgfpathlineto{\pgfqpoint{1.996407in}{1.835947in}}%
\pgfpathlineto{\pgfqpoint{1.997253in}{2.421519in}}%
\pgfpathlineto{\pgfqpoint{1.998098in}{2.162395in}}%
\pgfpathlineto{\pgfqpoint{1.999790in}{2.618655in}}%
\pgfpathlineto{\pgfqpoint{2.001481in}{2.162206in}}%
\pgfpathlineto{\pgfqpoint{2.002326in}{2.226744in}}%
\pgfpathlineto{\pgfqpoint{2.003172in}{1.966636in}}%
\pgfpathlineto{\pgfqpoint{2.004018in}{2.031581in}}%
\pgfpathlineto{\pgfqpoint{2.005709in}{2.292893in}}%
\pgfpathlineto{\pgfqpoint{2.006554in}{2.225014in}}%
\pgfpathlineto{\pgfqpoint{2.007400in}{2.619930in}}%
\pgfpathlineto{\pgfqpoint{2.008246in}{2.030379in}}%
\pgfpathlineto{\pgfqpoint{2.009091in}{2.227752in}}%
\pgfpathlineto{\pgfqpoint{2.009937in}{2.359100in}}%
\pgfpathlineto{\pgfqpoint{2.010783in}{1.968530in}}%
\pgfpathlineto{\pgfqpoint{2.011628in}{2.288333in}}%
\pgfpathlineto{\pgfqpoint{2.014165in}{1.965688in}}%
\pgfpathlineto{\pgfqpoint{2.015856in}{2.292253in}}%
\pgfpathlineto{\pgfqpoint{2.016702in}{1.960808in}}%
\pgfpathlineto{\pgfqpoint{2.017548in}{2.357342in}}%
\pgfpathlineto{\pgfqpoint{2.018393in}{2.289933in}}%
\pgfpathlineto{\pgfqpoint{2.019239in}{2.097355in}}%
\pgfpathlineto{\pgfqpoint{2.020084in}{2.260471in}}%
\pgfpathlineto{\pgfqpoint{2.020930in}{2.000527in}}%
\pgfpathlineto{\pgfqpoint{2.021776in}{2.424046in}}%
\pgfpathlineto{\pgfqpoint{2.022621in}{2.226691in}}%
\pgfpathlineto{\pgfqpoint{2.023467in}{1.771105in}}%
\pgfpathlineto{\pgfqpoint{2.025158in}{2.298524in}}%
\pgfpathlineto{\pgfqpoint{2.026004in}{2.297895in}}%
\pgfpathlineto{\pgfqpoint{2.026849in}{1.965256in}}%
\pgfpathlineto{\pgfqpoint{2.027695in}{2.683529in}}%
\pgfpathlineto{\pgfqpoint{2.028541in}{2.033049in}}%
\pgfpathlineto{\pgfqpoint{2.029386in}{2.293029in}}%
\pgfpathlineto{\pgfqpoint{2.031078in}{2.096702in}}%
\pgfpathlineto{\pgfqpoint{2.031923in}{2.489593in}}%
\pgfpathlineto{\pgfqpoint{2.032769in}{2.031353in}}%
\pgfpathlineto{\pgfqpoint{2.033614in}{2.291915in}}%
\pgfpathlineto{\pgfqpoint{2.034460in}{2.490192in}}%
\pgfpathlineto{\pgfqpoint{2.035306in}{1.833980in}}%
\pgfpathlineto{\pgfqpoint{2.036151in}{1.964030in}}%
\pgfpathlineto{\pgfqpoint{2.036997in}{2.619148in}}%
\pgfpathlineto{\pgfqpoint{2.037843in}{2.423529in}}%
\pgfpathlineto{\pgfqpoint{2.039534in}{2.028111in}}%
\pgfpathlineto{\pgfqpoint{2.040379in}{2.554229in}}%
\pgfpathlineto{\pgfqpoint{2.041225in}{2.161620in}}%
\pgfpathlineto{\pgfqpoint{2.042071in}{2.163324in}}%
\pgfpathlineto{\pgfqpoint{2.042916in}{2.292494in}}%
\pgfpathlineto{\pgfqpoint{2.044608in}{1.770215in}}%
\pgfpathlineto{\pgfqpoint{2.045453in}{2.162003in}}%
\pgfpathlineto{\pgfqpoint{2.046299in}{1.901172in}}%
\pgfpathlineto{\pgfqpoint{2.047144in}{1.965461in}}%
\pgfpathlineto{\pgfqpoint{2.047990in}{2.291684in}}%
\pgfpathlineto{\pgfqpoint{2.048836in}{1.965289in}}%
\pgfpathlineto{\pgfqpoint{2.049681in}{2.616653in}}%
\pgfpathlineto{\pgfqpoint{2.050527in}{2.097073in}}%
\pgfpathlineto{\pgfqpoint{2.051373in}{2.488869in}}%
\pgfpathlineto{\pgfqpoint{2.052218in}{2.292328in}}%
\pgfpathlineto{\pgfqpoint{2.053064in}{2.227788in}}%
\pgfpathlineto{\pgfqpoint{2.053909in}{2.035148in}}%
\pgfpathlineto{\pgfqpoint{2.056446in}{2.553328in}}%
\pgfpathlineto{\pgfqpoint{2.058138in}{2.025445in}}%
\pgfpathlineto{\pgfqpoint{2.058983in}{2.031740in}}%
\pgfpathlineto{\pgfqpoint{2.059829in}{2.356938in}}%
\pgfpathlineto{\pgfqpoint{2.060674in}{2.292358in}}%
\pgfpathlineto{\pgfqpoint{2.063211in}{2.032538in}}%
\pgfpathlineto{\pgfqpoint{2.064057in}{2.099924in}}%
\pgfpathlineto{\pgfqpoint{2.064903in}{2.090521in}}%
\pgfpathlineto{\pgfqpoint{2.065748in}{2.099738in}}%
\pgfpathlineto{\pgfqpoint{2.067439in}{2.294577in}}%
\pgfpathlineto{\pgfqpoint{2.068285in}{2.295993in}}%
\pgfpathlineto{\pgfqpoint{2.069131in}{1.708079in}}%
\pgfpathlineto{\pgfqpoint{2.069976in}{1.835172in}}%
\pgfpathlineto{\pgfqpoint{2.073359in}{2.617506in}}%
\pgfpathlineto{\pgfqpoint{2.075050in}{2.160677in}}%
\pgfpathlineto{\pgfqpoint{2.075896in}{2.456044in}}%
\pgfpathlineto{\pgfqpoint{2.076741in}{1.801987in}}%
\pgfpathlineto{\pgfqpoint{2.077587in}{2.269707in}}%
\pgfpathlineto{\pgfqpoint{2.080124in}{1.933276in}}%
\pgfpathlineto{\pgfqpoint{2.083506in}{2.196128in}}%
\pgfpathlineto{\pgfqpoint{2.084352in}{2.064275in}}%
\pgfpathlineto{\pgfqpoint{2.086889in}{2.456136in}}%
\pgfpathlineto{\pgfqpoint{2.087734in}{2.064303in}}%
\pgfpathlineto{\pgfqpoint{2.088580in}{2.128846in}}%
\pgfpathlineto{\pgfqpoint{2.089426in}{2.260249in}}%
\pgfpathlineto{\pgfqpoint{2.091962in}{2.063836in}}%
\pgfpathlineto{\pgfqpoint{2.092808in}{2.325525in}}%
\pgfpathlineto{\pgfqpoint{2.095345in}{1.996786in}}%
\pgfpathlineto{\pgfqpoint{2.097036in}{2.264298in}}%
\pgfpathlineto{\pgfqpoint{2.097882in}{2.066992in}}%
\pgfpathlineto{\pgfqpoint{2.098727in}{2.390101in}}%
\pgfpathlineto{\pgfqpoint{2.099573in}{2.193199in}}%
\pgfpathlineto{\pgfqpoint{2.101264in}{2.324900in}}%
\pgfpathlineto{\pgfqpoint{2.102956in}{2.065896in}}%
\pgfpathlineto{\pgfqpoint{2.104647in}{2.457181in}}%
\pgfpathlineto{\pgfqpoint{2.105492in}{2.323189in}}%
\pgfpathlineto{\pgfqpoint{2.106338in}{2.585396in}}%
\pgfpathlineto{\pgfqpoint{2.107184in}{1.999075in}}%
\pgfpathlineto{\pgfqpoint{2.108029in}{2.064084in}}%
\pgfpathlineto{\pgfqpoint{2.108875in}{2.197309in}}%
\pgfpathlineto{\pgfqpoint{2.109721in}{1.933990in}}%
\pgfpathlineto{\pgfqpoint{2.110566in}{2.390361in}}%
\pgfpathlineto{\pgfqpoint{2.111412in}{2.260305in}}%
\pgfpathlineto{\pgfqpoint{2.112257in}{2.194982in}}%
\pgfpathlineto{\pgfqpoint{2.113103in}{2.326115in}}%
\pgfpathlineto{\pgfqpoint{2.113949in}{2.325394in}}%
\pgfpathlineto{\pgfqpoint{2.117331in}{1.732892in}}%
\pgfpathlineto{\pgfqpoint{2.119868in}{2.259974in}}%
\pgfpathlineto{\pgfqpoint{2.122405in}{1.801947in}}%
\pgfpathlineto{\pgfqpoint{2.123251in}{2.129572in}}%
\pgfpathlineto{\pgfqpoint{2.124096in}{2.128937in}}%
\pgfpathlineto{\pgfqpoint{2.125787in}{2.063520in}}%
\pgfpathlineto{\pgfqpoint{2.126633in}{2.386619in}}%
\pgfpathlineto{\pgfqpoint{2.127479in}{1.941494in}}%
\pgfpathlineto{\pgfqpoint{2.128324in}{2.262304in}}%
\pgfpathlineto{\pgfqpoint{2.129170in}{2.259183in}}%
\pgfpathlineto{\pgfqpoint{2.130016in}{2.063533in}}%
\pgfpathlineto{\pgfqpoint{2.130861in}{2.520579in}}%
\pgfpathlineto{\pgfqpoint{2.131707in}{2.260987in}}%
\pgfpathlineto{\pgfqpoint{2.132552in}{2.259609in}}%
\pgfpathlineto{\pgfqpoint{2.133398in}{2.324813in}}%
\pgfpathlineto{\pgfqpoint{2.135935in}{1.999106in}}%
\pgfpathlineto{\pgfqpoint{2.136781in}{2.129306in}}%
\pgfpathlineto{\pgfqpoint{2.137626in}{2.001101in}}%
\pgfpathlineto{\pgfqpoint{2.138472in}{2.325981in}}%
\pgfpathlineto{\pgfqpoint{2.139317in}{1.801643in}}%
\pgfpathlineto{\pgfqpoint{2.140163in}{2.196907in}}%
\pgfpathlineto{\pgfqpoint{2.141009in}{2.390575in}}%
\pgfpathlineto{\pgfqpoint{2.141854in}{2.000654in}}%
\pgfpathlineto{\pgfqpoint{2.142700in}{2.391353in}}%
\pgfpathlineto{\pgfqpoint{2.143546in}{2.127756in}}%
\pgfpathlineto{\pgfqpoint{2.145237in}{2.064169in}}%
\pgfpathlineto{\pgfqpoint{2.146082in}{2.129773in}}%
\pgfpathlineto{\pgfqpoint{2.146928in}{2.129421in}}%
\pgfpathlineto{\pgfqpoint{2.148619in}{2.520389in}}%
\pgfpathlineto{\pgfqpoint{2.149465in}{1.998326in}}%
\pgfpathlineto{\pgfqpoint{2.150311in}{2.455957in}}%
\pgfpathlineto{\pgfqpoint{2.151156in}{2.193854in}}%
\pgfpathlineto{\pgfqpoint{2.152002in}{2.456090in}}%
\pgfpathlineto{\pgfqpoint{2.152847in}{1.801918in}}%
\pgfpathlineto{\pgfqpoint{2.153693in}{2.129574in}}%
\pgfpathlineto{\pgfqpoint{2.154539in}{1.868172in}}%
\pgfpathlineto{\pgfqpoint{2.155384in}{2.650852in}}%
\pgfpathlineto{\pgfqpoint{2.156230in}{2.129176in}}%
\pgfpathlineto{\pgfqpoint{2.157076in}{2.064985in}}%
\pgfpathlineto{\pgfqpoint{2.158767in}{2.455802in}}%
\pgfpathlineto{\pgfqpoint{2.160458in}{2.193746in}}%
\pgfpathlineto{\pgfqpoint{2.162149in}{2.586992in}}%
\pgfpathlineto{\pgfqpoint{2.164686in}{1.878541in}}%
\pgfpathlineto{\pgfqpoint{2.166377in}{2.652379in}}%
\pgfpathlineto{\pgfqpoint{2.168914in}{1.606601in}}%
\pgfpathlineto{\pgfqpoint{2.169760in}{2.324760in}}%
\pgfpathlineto{\pgfqpoint{2.170605in}{2.195039in}}%
\pgfpathlineto{\pgfqpoint{2.171451in}{2.325592in}}%
\pgfpathlineto{\pgfqpoint{2.172297in}{1.868785in}}%
\pgfpathlineto{\pgfqpoint{2.173988in}{2.455865in}}%
\pgfpathlineto{\pgfqpoint{2.177370in}{1.737489in}}%
\pgfpathlineto{\pgfqpoint{2.178216in}{2.325637in}}%
\pgfpathlineto{\pgfqpoint{2.179062in}{1.867938in}}%
\pgfpathlineto{\pgfqpoint{2.180753in}{2.129148in}}%
\pgfpathlineto{\pgfqpoint{2.182444in}{1.933152in}}%
\pgfpathlineto{\pgfqpoint{2.184135in}{2.390710in}}%
\pgfpathlineto{\pgfqpoint{2.184981in}{2.129593in}}%
\pgfpathlineto{\pgfqpoint{2.185827in}{2.586942in}}%
\pgfpathlineto{\pgfqpoint{2.186672in}{2.129166in}}%
\pgfpathlineto{\pgfqpoint{2.187518in}{2.195159in}}%
\pgfpathlineto{\pgfqpoint{2.188364in}{2.260080in}}%
\pgfpathlineto{\pgfqpoint{2.189209in}{2.717181in}}%
\pgfpathlineto{\pgfqpoint{2.190055in}{2.390694in}}%
\pgfpathlineto{\pgfqpoint{2.190900in}{2.129132in}}%
\pgfpathlineto{\pgfqpoint{2.191746in}{2.522999in}}%
\pgfpathlineto{\pgfqpoint{2.192592in}{2.260488in}}%
\pgfpathlineto{\pgfqpoint{2.193437in}{2.194661in}}%
\pgfpathlineto{\pgfqpoint{2.194283in}{2.259976in}}%
\pgfpathlineto{\pgfqpoint{2.195129in}{2.259963in}}%
\pgfpathlineto{\pgfqpoint{2.195974in}{2.781653in}}%
\pgfpathlineto{\pgfqpoint{2.196820in}{2.194538in}}%
\pgfpathlineto{\pgfqpoint{2.197665in}{2.456088in}}%
\pgfpathlineto{\pgfqpoint{2.199357in}{2.063923in}}%
\pgfpathlineto{\pgfqpoint{2.200202in}{2.194814in}}%
\pgfpathlineto{\pgfqpoint{2.201894in}{1.737053in}}%
\pgfpathlineto{\pgfqpoint{2.202739in}{1.933039in}}%
\pgfpathlineto{\pgfqpoint{2.203585in}{2.521480in}}%
\pgfpathlineto{\pgfqpoint{2.204430in}{1.933201in}}%
\pgfpathlineto{\pgfqpoint{2.205276in}{2.260433in}}%
\pgfpathlineto{\pgfqpoint{2.207813in}{2.063822in}}%
\pgfpathlineto{\pgfqpoint{2.209504in}{1.998330in}}%
\pgfpathlineto{\pgfqpoint{2.210350in}{1.606184in}}%
\pgfpathlineto{\pgfqpoint{2.212887in}{2.390682in}}%
\pgfpathlineto{\pgfqpoint{2.213732in}{2.391317in}}%
\pgfpathlineto{\pgfqpoint{2.217115in}{1.802481in}}%
\pgfpathlineto{\pgfqpoint{2.219652in}{2.129321in}}%
\pgfpathlineto{\pgfqpoint{2.221343in}{2.129024in}}%
\pgfpathlineto{\pgfqpoint{2.222189in}{2.390593in}}%
\pgfpathlineto{\pgfqpoint{2.223034in}{2.063894in}}%
\pgfpathlineto{\pgfqpoint{2.223880in}{2.325263in}}%
\pgfpathlineto{\pgfqpoint{2.226417in}{2.000960in}}%
\pgfpathlineto{\pgfqpoint{2.227262in}{2.128942in}}%
\pgfpathlineto{\pgfqpoint{2.228108in}{1.998736in}}%
\pgfpathlineto{\pgfqpoint{2.228954in}{2.782945in}}%
\pgfpathlineto{\pgfqpoint{2.229799in}{2.249334in}}%
\pgfpathlineto{\pgfqpoint{2.230645in}{1.671917in}}%
\pgfpathlineto{\pgfqpoint{2.231490in}{2.456024in}}%
\pgfpathlineto{\pgfqpoint{2.232336in}{2.259957in}}%
\pgfpathlineto{\pgfqpoint{2.233182in}{2.129328in}}%
\pgfpathlineto{\pgfqpoint{2.234027in}{2.190492in}}%
\pgfpathlineto{\pgfqpoint{2.234873in}{2.129096in}}%
\pgfpathlineto{\pgfqpoint{2.235719in}{2.298942in}}%
\pgfpathlineto{\pgfqpoint{2.236564in}{2.223408in}}%
\pgfpathlineto{\pgfqpoint{2.237410in}{1.966774in}}%
\pgfpathlineto{\pgfqpoint{2.238255in}{2.085101in}}%
\pgfpathlineto{\pgfqpoint{2.239947in}{2.421332in}}%
\pgfpathlineto{\pgfqpoint{2.240792in}{1.905871in}}%
\pgfpathlineto{\pgfqpoint{2.241638in}{2.228497in}}%
\pgfpathlineto{\pgfqpoint{2.242483in}{2.424165in}}%
\pgfpathlineto{\pgfqpoint{2.243329in}{2.162685in}}%
\pgfpathlineto{\pgfqpoint{2.244175in}{2.491337in}}%
\pgfpathlineto{\pgfqpoint{2.245020in}{2.253697in}}%
\pgfpathlineto{\pgfqpoint{2.245866in}{1.835221in}}%
\pgfpathlineto{\pgfqpoint{2.246712in}{2.488304in}}%
\pgfpathlineto{\pgfqpoint{2.247557in}{2.361608in}}%
\pgfpathlineto{\pgfqpoint{2.248403in}{1.965980in}}%
\pgfpathlineto{\pgfqpoint{2.249248in}{2.353590in}}%
\pgfpathlineto{\pgfqpoint{2.250094in}{2.029370in}}%
\pgfpathlineto{\pgfqpoint{2.250940in}{2.880514in}}%
\pgfpathlineto{\pgfqpoint{2.251785in}{2.486531in}}%
\pgfpathlineto{\pgfqpoint{2.252631in}{2.359486in}}%
\pgfpathlineto{\pgfqpoint{2.253477in}{1.770134in}}%
\pgfpathlineto{\pgfqpoint{2.256013in}{2.423057in}}%
\pgfpathlineto{\pgfqpoint{2.256859in}{1.837399in}}%
\pgfpathlineto{\pgfqpoint{2.257705in}{2.294122in}}%
\pgfpathlineto{\pgfqpoint{2.258550in}{2.230114in}}%
\pgfpathlineto{\pgfqpoint{2.259396in}{2.291865in}}%
\pgfpathlineto{\pgfqpoint{2.261087in}{2.227015in}}%
\pgfpathlineto{\pgfqpoint{2.261933in}{2.228244in}}%
\pgfpathlineto{\pgfqpoint{2.263624in}{2.684525in}}%
\pgfpathlineto{\pgfqpoint{2.264470in}{1.900748in}}%
\pgfpathlineto{\pgfqpoint{2.265315in}{2.292297in}}%
\pgfpathlineto{\pgfqpoint{2.266161in}{2.423373in}}%
\pgfpathlineto{\pgfqpoint{2.267007in}{1.835125in}}%
\pgfpathlineto{\pgfqpoint{2.267852in}{2.458837in}}%
\pgfpathlineto{\pgfqpoint{2.268698in}{2.075775in}}%
\pgfpathlineto{\pgfqpoint{2.271235in}{2.292667in}}%
\pgfpathlineto{\pgfqpoint{2.272080in}{1.893268in}}%
\pgfpathlineto{\pgfqpoint{2.272926in}{2.619225in}}%
\pgfpathlineto{\pgfqpoint{2.273772in}{1.929747in}}%
\pgfpathlineto{\pgfqpoint{2.274617in}{2.184117in}}%
\pgfpathlineto{\pgfqpoint{2.275463in}{2.312574in}}%
\pgfpathlineto{\pgfqpoint{2.278845in}{1.598299in}}%
\pgfpathlineto{\pgfqpoint{2.279691in}{2.448534in}}%
\pgfpathlineto{\pgfqpoint{2.280537in}{2.161419in}}%
\pgfpathlineto{\pgfqpoint{2.281382in}{2.487647in}}%
\pgfpathlineto{\pgfqpoint{2.282228in}{2.447679in}}%
\pgfpathlineto{\pgfqpoint{2.283073in}{1.814250in}}%
\pgfpathlineto{\pgfqpoint{2.283919in}{2.205868in}}%
\pgfpathlineto{\pgfqpoint{2.284765in}{2.395699in}}%
\pgfpathlineto{\pgfqpoint{2.285610in}{2.173202in}}%
\pgfpathlineto{\pgfqpoint{2.286456in}{2.383482in}}%
\pgfpathlineto{\pgfqpoint{2.288993in}{1.883588in}}%
\pgfpathlineto{\pgfqpoint{2.291530in}{2.368380in}}%
\pgfpathlineto{\pgfqpoint{2.292375in}{2.201217in}}%
\pgfpathlineto{\pgfqpoint{2.293221in}{2.644822in}}%
\pgfpathlineto{\pgfqpoint{2.294067in}{1.855469in}}%
\pgfpathlineto{\pgfqpoint{2.294912in}{1.943891in}}%
\pgfpathlineto{\pgfqpoint{2.298295in}{2.746261in}}%
\pgfpathlineto{\pgfqpoint{2.299986in}{2.032380in}}%
\pgfpathlineto{\pgfqpoint{2.300832in}{2.101850in}}%
\pgfpathlineto{\pgfqpoint{2.301677in}{1.918278in}}%
\pgfpathlineto{\pgfqpoint{2.302523in}{2.422271in}}%
\pgfpathlineto{\pgfqpoint{2.303368in}{2.096765in}}%
\pgfpathlineto{\pgfqpoint{2.304214in}{2.487665in}}%
\pgfpathlineto{\pgfqpoint{2.305060in}{2.028002in}}%
\pgfpathlineto{\pgfqpoint{2.305905in}{2.487888in}}%
\pgfpathlineto{\pgfqpoint{2.306751in}{2.350018in}}%
\pgfpathlineto{\pgfqpoint{2.307597in}{2.029952in}}%
\pgfpathlineto{\pgfqpoint{2.308442in}{2.290785in}}%
\pgfpathlineto{\pgfqpoint{2.310133in}{2.025641in}}%
\pgfpathlineto{\pgfqpoint{2.310979in}{2.161194in}}%
\pgfpathlineto{\pgfqpoint{2.311825in}{2.152626in}}%
\pgfpathlineto{\pgfqpoint{2.313516in}{2.009040in}}%
\pgfpathlineto{\pgfqpoint{2.314362in}{2.089291in}}%
\pgfpathlineto{\pgfqpoint{2.315207in}{2.227847in}}%
\pgfpathlineto{\pgfqpoint{2.316898in}{1.731498in}}%
\pgfpathlineto{\pgfqpoint{2.317744in}{2.209193in}}%
\pgfpathlineto{\pgfqpoint{2.318590in}{1.672783in}}%
\pgfpathlineto{\pgfqpoint{2.319435in}{2.035484in}}%
\pgfpathlineto{\pgfqpoint{2.320281in}{2.169543in}}%
\pgfpathlineto{\pgfqpoint{2.321126in}{2.104100in}}%
\pgfpathlineto{\pgfqpoint{2.321972in}{2.163183in}}%
\pgfpathlineto{\pgfqpoint{2.322818in}{2.345078in}}%
\pgfpathlineto{\pgfqpoint{2.323663in}{2.096657in}}%
\pgfpathlineto{\pgfqpoint{2.324509in}{2.553767in}}%
\pgfpathlineto{\pgfqpoint{2.325355in}{2.295554in}}%
\pgfpathlineto{\pgfqpoint{2.326200in}{1.508772in}}%
\pgfpathlineto{\pgfqpoint{2.327046in}{1.963034in}}%
\pgfpathlineto{\pgfqpoint{2.327891in}{2.102669in}}%
\pgfpathlineto{\pgfqpoint{2.328737in}{1.967992in}}%
\pgfpathlineto{\pgfqpoint{2.329583in}{2.034384in}}%
\pgfpathlineto{\pgfqpoint{2.330428in}{2.096314in}}%
\pgfpathlineto{\pgfqpoint{2.331274in}{2.292219in}}%
\pgfpathlineto{\pgfqpoint{2.332120in}{2.030844in}}%
\pgfpathlineto{\pgfqpoint{2.333811in}{2.553344in}}%
\pgfpathlineto{\pgfqpoint{2.334656in}{1.965782in}}%
\pgfpathlineto{\pgfqpoint{2.335502in}{2.096526in}}%
\pgfpathlineto{\pgfqpoint{2.336348in}{2.293653in}}%
\pgfpathlineto{\pgfqpoint{2.337193in}{2.161971in}}%
\pgfpathlineto{\pgfqpoint{2.338039in}{2.031486in}}%
\pgfpathlineto{\pgfqpoint{2.338885in}{2.096905in}}%
\pgfpathlineto{\pgfqpoint{2.339730in}{2.358034in}}%
\pgfpathlineto{\pgfqpoint{2.340576in}{2.227270in}}%
\pgfpathlineto{\pgfqpoint{2.342267in}{1.965814in}}%
\pgfpathlineto{\pgfqpoint{2.344804in}{2.554132in}}%
\pgfpathlineto{\pgfqpoint{2.346495in}{1.966491in}}%
\pgfpathlineto{\pgfqpoint{2.347341in}{2.096665in}}%
\pgfpathlineto{\pgfqpoint{2.348186in}{2.684438in}}%
\pgfpathlineto{\pgfqpoint{2.349032in}{2.292748in}}%
\pgfpathlineto{\pgfqpoint{2.349878in}{2.227287in}}%
\pgfpathlineto{\pgfqpoint{2.350723in}{2.031234in}}%
\pgfpathlineto{\pgfqpoint{2.351569in}{2.358058in}}%
\pgfpathlineto{\pgfqpoint{2.352415in}{2.227271in}}%
\pgfpathlineto{\pgfqpoint{2.353260in}{2.228393in}}%
\pgfpathlineto{\pgfqpoint{2.354106in}{2.488706in}}%
\pgfpathlineto{\pgfqpoint{2.354951in}{1.755958in}}%
\pgfpathlineto{\pgfqpoint{2.355797in}{2.031238in}}%
\pgfpathlineto{\pgfqpoint{2.356643in}{2.099101in}}%
\pgfpathlineto{\pgfqpoint{2.357488in}{1.769735in}}%
\pgfpathlineto{\pgfqpoint{2.359180in}{2.488725in}}%
\pgfpathlineto{\pgfqpoint{2.361716in}{1.900360in}}%
\pgfpathlineto{\pgfqpoint{2.362562in}{2.030881in}}%
\pgfpathlineto{\pgfqpoint{2.364253in}{1.835053in}}%
\pgfpathlineto{\pgfqpoint{2.365099in}{1.704475in}}%
\pgfpathlineto{\pgfqpoint{2.365945in}{2.114221in}}%
\pgfpathlineto{\pgfqpoint{2.366790in}{2.096585in}}%
\pgfpathlineto{\pgfqpoint{2.367636in}{2.096577in}}%
\pgfpathlineto{\pgfqpoint{2.368481in}{2.423411in}}%
\pgfpathlineto{\pgfqpoint{2.369327in}{2.161913in}}%
\pgfpathlineto{\pgfqpoint{2.370173in}{2.358037in}}%
\pgfpathlineto{\pgfqpoint{2.371018in}{1.900529in}}%
\pgfpathlineto{\pgfqpoint{2.372710in}{2.750189in}}%
\pgfpathlineto{\pgfqpoint{2.375246in}{2.031269in}}%
\pgfpathlineto{\pgfqpoint{2.376092in}{1.965725in}}%
\pgfpathlineto{\pgfqpoint{2.377783in}{2.292555in}}%
\pgfpathlineto{\pgfqpoint{2.379475in}{1.965821in}}%
\pgfpathlineto{\pgfqpoint{2.380320in}{2.031561in}}%
\pgfpathlineto{\pgfqpoint{2.381166in}{2.423390in}}%
\pgfpathlineto{\pgfqpoint{2.382011in}{2.423316in}}%
\pgfpathlineto{\pgfqpoint{2.383703in}{2.227166in}}%
\pgfpathlineto{\pgfqpoint{2.384548in}{1.770721in}}%
\pgfpathlineto{\pgfqpoint{2.385394in}{2.488858in}}%
\pgfpathlineto{\pgfqpoint{2.386240in}{2.368310in}}%
\pgfpathlineto{\pgfqpoint{2.387085in}{2.227334in}}%
\pgfpathlineto{\pgfqpoint{2.387931in}{2.423292in}}%
\pgfpathlineto{\pgfqpoint{2.388776in}{2.292583in}}%
\pgfpathlineto{\pgfqpoint{2.389622in}{2.227287in}}%
\pgfpathlineto{\pgfqpoint{2.390468in}{2.684895in}}%
\pgfpathlineto{\pgfqpoint{2.391313in}{1.900669in}}%
\pgfpathlineto{\pgfqpoint{2.392159in}{2.031183in}}%
\pgfpathlineto{\pgfqpoint{2.393850in}{2.554003in}}%
\pgfpathlineto{\pgfqpoint{2.395541in}{1.900494in}}%
\pgfpathlineto{\pgfqpoint{2.397233in}{2.357993in}}%
\pgfpathlineto{\pgfqpoint{2.398078in}{2.161914in}}%
\pgfpathlineto{\pgfqpoint{2.398924in}{2.488700in}}%
\pgfpathlineto{\pgfqpoint{2.399769in}{1.769711in}}%
\pgfpathlineto{\pgfqpoint{2.400615in}{2.358075in}}%
\pgfpathlineto{\pgfqpoint{2.403152in}{2.161824in}}%
\pgfpathlineto{\pgfqpoint{2.403998in}{1.900479in}}%
\pgfpathlineto{\pgfqpoint{2.405689in}{2.357837in}}%
\pgfpathlineto{\pgfqpoint{2.406534in}{2.292556in}}%
\pgfpathlineto{\pgfqpoint{2.407380in}{1.966092in}}%
\pgfpathlineto{\pgfqpoint{2.408226in}{2.488657in}}%
\pgfpathlineto{\pgfqpoint{2.409071in}{2.162001in}}%
\pgfpathlineto{\pgfqpoint{2.409917in}{2.227282in}}%
\pgfpathlineto{\pgfqpoint{2.410763in}{2.553768in}}%
\pgfpathlineto{\pgfqpoint{2.411608in}{1.891172in}}%
\pgfpathlineto{\pgfqpoint{2.412454in}{1.965821in}}%
\pgfpathlineto{\pgfqpoint{2.414991in}{2.222448in}}%
\pgfpathlineto{\pgfqpoint{2.415836in}{1.920981in}}%
\pgfpathlineto{\pgfqpoint{2.416682in}{2.027250in}}%
\pgfpathlineto{\pgfqpoint{2.417528in}{2.209797in}}%
\pgfpathlineto{\pgfqpoint{2.418373in}{1.797806in}}%
\pgfpathlineto{\pgfqpoint{2.419219in}{2.031203in}}%
\pgfpathlineto{\pgfqpoint{2.420910in}{2.587059in}}%
\pgfpathlineto{\pgfqpoint{2.421756in}{2.390658in}}%
\pgfpathlineto{\pgfqpoint{2.423447in}{2.063765in}}%
\pgfpathlineto{\pgfqpoint{2.424293in}{2.521615in}}%
\pgfpathlineto{\pgfqpoint{2.425138in}{1.933067in}}%
\pgfpathlineto{\pgfqpoint{2.425984in}{2.456084in}}%
\pgfpathlineto{\pgfqpoint{2.426829in}{2.325309in}}%
\pgfpathlineto{\pgfqpoint{2.427675in}{2.194630in}}%
\pgfpathlineto{\pgfqpoint{2.428521in}{2.390669in}}%
\pgfpathlineto{\pgfqpoint{2.429366in}{1.671720in}}%
\pgfpathlineto{\pgfqpoint{2.430212in}{1.998503in}}%
\pgfpathlineto{\pgfqpoint{2.432749in}{2.390625in}}%
\pgfpathlineto{\pgfqpoint{2.433594in}{1.998860in}}%
\pgfpathlineto{\pgfqpoint{2.434440in}{2.217836in}}%
\pgfpathlineto{\pgfqpoint{2.435286in}{2.325352in}}%
\pgfpathlineto{\pgfqpoint{2.436977in}{1.865695in}}%
\pgfpathlineto{\pgfqpoint{2.438668in}{2.334116in}}%
\pgfpathlineto{\pgfqpoint{2.439514in}{2.323288in}}%
\pgfpathlineto{\pgfqpoint{2.440359in}{2.324272in}}%
\pgfpathlineto{\pgfqpoint{2.441205in}{1.899072in}}%
\pgfpathlineto{\pgfqpoint{2.442051in}{2.755234in}}%
\pgfpathlineto{\pgfqpoint{2.442896in}{2.105413in}}%
\pgfpathlineto{\pgfqpoint{2.445433in}{2.449992in}}%
\pgfpathlineto{\pgfqpoint{2.446279in}{2.500838in}}%
\pgfpathlineto{\pgfqpoint{2.447124in}{2.008292in}}%
\pgfpathlineto{\pgfqpoint{2.447970in}{2.238017in}}%
\pgfpathlineto{\pgfqpoint{2.448816in}{2.190795in}}%
\pgfpathlineto{\pgfqpoint{2.449661in}{2.194504in}}%
\pgfpathlineto{\pgfqpoint{2.450507in}{2.229109in}}%
\pgfpathlineto{\pgfqpoint{2.451353in}{2.032692in}}%
\pgfpathlineto{\pgfqpoint{2.452198in}{2.357054in}}%
\pgfpathlineto{\pgfqpoint{2.453044in}{1.965254in}}%
\pgfpathlineto{\pgfqpoint{2.453889in}{2.097018in}}%
\pgfpathlineto{\pgfqpoint{2.455581in}{2.311640in}}%
\pgfpathlineto{\pgfqpoint{2.456426in}{2.226713in}}%
\pgfpathlineto{\pgfqpoint{2.457272in}{2.227011in}}%
\pgfpathlineto{\pgfqpoint{2.458118in}{2.494930in}}%
\pgfpathlineto{\pgfqpoint{2.458963in}{1.900203in}}%
\pgfpathlineto{\pgfqpoint{2.459809in}{2.292702in}}%
\pgfpathlineto{\pgfqpoint{2.461500in}{1.769669in}}%
\pgfpathlineto{\pgfqpoint{2.463191in}{2.227317in}}%
\pgfpathlineto{\pgfqpoint{2.464037in}{2.096423in}}%
\pgfpathlineto{\pgfqpoint{2.467419in}{2.423046in}}%
\pgfpathlineto{\pgfqpoint{2.469111in}{1.965195in}}%
\pgfpathlineto{\pgfqpoint{2.469956in}{1.966002in}}%
\pgfpathlineto{\pgfqpoint{2.470802in}{2.358891in}}%
\pgfpathlineto{\pgfqpoint{2.471648in}{1.966653in}}%
\pgfpathlineto{\pgfqpoint{2.472493in}{2.488744in}}%
\pgfpathlineto{\pgfqpoint{2.473339in}{1.966030in}}%
\pgfpathlineto{\pgfqpoint{2.474184in}{2.163970in}}%
\pgfpathlineto{\pgfqpoint{2.475030in}{2.421608in}}%
\pgfpathlineto{\pgfqpoint{2.475876in}{2.096373in}}%
\pgfpathlineto{\pgfqpoint{2.476721in}{2.358406in}}%
\pgfpathlineto{\pgfqpoint{2.477567in}{2.489120in}}%
\pgfpathlineto{\pgfqpoint{2.480104in}{2.031433in}}%
\pgfpathlineto{\pgfqpoint{2.480949in}{2.292454in}}%
\pgfpathlineto{\pgfqpoint{2.481795in}{2.164258in}}%
\pgfpathlineto{\pgfqpoint{2.482641in}{2.032059in}}%
\pgfpathlineto{\pgfqpoint{2.484332in}{2.292695in}}%
\pgfpathlineto{\pgfqpoint{2.485177in}{2.236796in}}%
\pgfpathlineto{\pgfqpoint{2.486023in}{2.357994in}}%
\pgfpathlineto{\pgfqpoint{2.487714in}{2.031503in}}%
\pgfpathlineto{\pgfqpoint{2.488560in}{2.421386in}}%
\pgfpathlineto{\pgfqpoint{2.489406in}{2.292742in}}%
\pgfpathlineto{\pgfqpoint{2.491097in}{2.684721in}}%
\pgfpathlineto{\pgfqpoint{2.491942in}{1.900458in}}%
\pgfpathlineto{\pgfqpoint{2.492788in}{2.162015in}}%
\pgfpathlineto{\pgfqpoint{2.493634in}{2.091824in}}%
\pgfpathlineto{\pgfqpoint{2.494479in}{2.551846in}}%
\pgfpathlineto{\pgfqpoint{2.495325in}{2.292550in}}%
\pgfpathlineto{\pgfqpoint{2.496171in}{2.292749in}}%
\pgfpathlineto{\pgfqpoint{2.497016in}{2.013489in}}%
\pgfpathlineto{\pgfqpoint{2.497862in}{2.450977in}}%
\pgfpathlineto{\pgfqpoint{2.498707in}{2.388659in}}%
\pgfpathlineto{\pgfqpoint{2.499553in}{2.062151in}}%
\pgfpathlineto{\pgfqpoint{2.500399in}{2.651467in}}%
\pgfpathlineto{\pgfqpoint{2.501244in}{2.256123in}}%
\pgfpathlineto{\pgfqpoint{2.502090in}{2.259715in}}%
\pgfpathlineto{\pgfqpoint{2.502936in}{2.586460in}}%
\pgfpathlineto{\pgfqpoint{2.504627in}{1.998099in}}%
\pgfpathlineto{\pgfqpoint{2.505472in}{2.063613in}}%
\pgfpathlineto{\pgfqpoint{2.506318in}{2.389494in}}%
\pgfpathlineto{\pgfqpoint{2.507164in}{1.803213in}}%
\pgfpathlineto{\pgfqpoint{2.508009in}{2.260150in}}%
\pgfpathlineto{\pgfqpoint{2.508855in}{2.194748in}}%
\pgfpathlineto{\pgfqpoint{2.509701in}{2.259930in}}%
\pgfpathlineto{\pgfqpoint{2.510546in}{2.130635in}}%
\pgfpathlineto{\pgfqpoint{2.513083in}{2.326445in}}%
\pgfpathlineto{\pgfqpoint{2.515620in}{1.998444in}}%
\pgfpathlineto{\pgfqpoint{2.516466in}{2.390851in}}%
\pgfpathlineto{\pgfqpoint{2.519002in}{1.868143in}}%
\pgfpathlineto{\pgfqpoint{2.520694in}{2.651908in}}%
\pgfpathlineto{\pgfqpoint{2.521539in}{2.521513in}}%
\pgfpathlineto{\pgfqpoint{2.522385in}{1.932455in}}%
\pgfpathlineto{\pgfqpoint{2.523231in}{2.063632in}}%
\pgfpathlineto{\pgfqpoint{2.524076in}{1.934296in}}%
\pgfpathlineto{\pgfqpoint{2.525767in}{2.392552in}}%
\pgfpathlineto{\pgfqpoint{2.527459in}{2.129163in}}%
\pgfpathlineto{\pgfqpoint{2.528304in}{2.129381in}}%
\pgfpathlineto{\pgfqpoint{2.529150in}{2.521238in}}%
\pgfpathlineto{\pgfqpoint{2.530841in}{2.129351in}}%
\pgfpathlineto{\pgfqpoint{2.531687in}{2.324840in}}%
\pgfpathlineto{\pgfqpoint{2.532532in}{2.064148in}}%
\pgfpathlineto{\pgfqpoint{2.533378in}{2.325679in}}%
\pgfpathlineto{\pgfqpoint{2.534224in}{2.216377in}}%
\pgfpathlineto{\pgfqpoint{2.535069in}{1.998383in}}%
\pgfpathlineto{\pgfqpoint{2.536761in}{2.521172in}}%
\pgfpathlineto{\pgfqpoint{2.539297in}{1.868839in}}%
\pgfpathlineto{\pgfqpoint{2.540143in}{2.326108in}}%
\pgfpathlineto{\pgfqpoint{2.540989in}{1.998837in}}%
\pgfpathlineto{\pgfqpoint{2.542680in}{2.651990in}}%
\pgfpathlineto{\pgfqpoint{2.543526in}{2.259309in}}%
\pgfpathlineto{\pgfqpoint{2.544371in}{2.263647in}}%
\pgfpathlineto{\pgfqpoint{2.545217in}{2.325626in}}%
\pgfpathlineto{\pgfqpoint{2.546908in}{2.194347in}}%
\pgfpathlineto{\pgfqpoint{2.547754in}{2.321021in}}%
\pgfpathlineto{\pgfqpoint{2.548599in}{2.260362in}}%
\pgfpathlineto{\pgfqpoint{2.550291in}{2.129160in}}%
\pgfpathlineto{\pgfqpoint{2.552827in}{2.325066in}}%
\pgfpathlineto{\pgfqpoint{2.555364in}{2.067728in}}%
\pgfpathlineto{\pgfqpoint{2.556210in}{2.124796in}}%
\pgfpathlineto{\pgfqpoint{2.557056in}{2.001879in}}%
\pgfpathlineto{\pgfqpoint{2.557901in}{2.515794in}}%
\pgfpathlineto{\pgfqpoint{2.558747in}{2.254576in}}%
\pgfpathlineto{\pgfqpoint{2.559592in}{1.997020in}}%
\pgfpathlineto{\pgfqpoint{2.560438in}{2.026662in}}%
\pgfpathlineto{\pgfqpoint{2.561284in}{2.285037in}}%
\pgfpathlineto{\pgfqpoint{2.562129in}{2.091477in}}%
\pgfpathlineto{\pgfqpoint{2.562975in}{2.294275in}}%
\pgfpathlineto{\pgfqpoint{2.563820in}{2.228729in}}%
\pgfpathlineto{\pgfqpoint{2.564666in}{2.162501in}}%
\pgfpathlineto{\pgfqpoint{2.565512in}{2.423349in}}%
\pgfpathlineto{\pgfqpoint{2.566357in}{1.835455in}}%
\pgfpathlineto{\pgfqpoint{2.567203in}{2.162832in}}%
\pgfpathlineto{\pgfqpoint{2.568049in}{2.418677in}}%
\pgfpathlineto{\pgfqpoint{2.568894in}{2.236998in}}%
\pgfpathlineto{\pgfqpoint{2.570585in}{1.763815in}}%
\pgfpathlineto{\pgfqpoint{2.572277in}{2.206380in}}%
\pgfpathlineto{\pgfqpoint{2.573122in}{1.900622in}}%
\pgfpathlineto{\pgfqpoint{2.573968in}{2.229108in}}%
\pgfpathlineto{\pgfqpoint{2.574814in}{1.707768in}}%
\pgfpathlineto{\pgfqpoint{2.577350in}{2.357843in}}%
\pgfpathlineto{\pgfqpoint{2.578196in}{2.357730in}}%
\pgfpathlineto{\pgfqpoint{2.579042in}{2.488131in}}%
\pgfpathlineto{\pgfqpoint{2.579887in}{2.096258in}}%
\pgfpathlineto{\pgfqpoint{2.581579in}{2.488718in}}%
\pgfpathlineto{\pgfqpoint{2.583270in}{1.834879in}}%
\pgfpathlineto{\pgfqpoint{2.584961in}{2.292755in}}%
\pgfpathlineto{\pgfqpoint{2.586652in}{2.097231in}}%
\pgfpathlineto{\pgfqpoint{2.587498in}{1.900608in}}%
\pgfpathlineto{\pgfqpoint{2.588344in}{2.161931in}}%
\pgfpathlineto{\pgfqpoint{2.589189in}{1.901707in}}%
\pgfpathlineto{\pgfqpoint{2.590035in}{2.358326in}}%
\pgfpathlineto{\pgfqpoint{2.590880in}{2.292447in}}%
\pgfpathlineto{\pgfqpoint{2.591726in}{2.162554in}}%
\pgfpathlineto{\pgfqpoint{2.592572in}{2.423340in}}%
\pgfpathlineto{\pgfqpoint{2.593417in}{1.965513in}}%
\pgfpathlineto{\pgfqpoint{2.594263in}{2.161532in}}%
\pgfpathlineto{\pgfqpoint{2.595109in}{2.097460in}}%
\pgfpathlineto{\pgfqpoint{2.596800in}{2.228838in}}%
\pgfpathlineto{\pgfqpoint{2.597645in}{2.485353in}}%
\pgfpathlineto{\pgfqpoint{2.598491in}{2.295639in}}%
\pgfpathlineto{\pgfqpoint{2.599337in}{2.292835in}}%
\pgfpathlineto{\pgfqpoint{2.601028in}{2.031912in}}%
\pgfpathlineto{\pgfqpoint{2.601874in}{2.096426in}}%
\pgfpathlineto{\pgfqpoint{2.602719in}{2.034688in}}%
\pgfpathlineto{\pgfqpoint{2.603565in}{2.359099in}}%
\pgfpathlineto{\pgfqpoint{2.605256in}{2.031869in}}%
\pgfpathlineto{\pgfqpoint{2.606102in}{2.162187in}}%
\pgfpathlineto{\pgfqpoint{2.607793in}{1.965034in}}%
\pgfpathlineto{\pgfqpoint{2.608639in}{2.096497in}}%
\pgfpathlineto{\pgfqpoint{2.609484in}{1.966240in}}%
\pgfpathlineto{\pgfqpoint{2.612021in}{2.357931in}}%
\pgfpathlineto{\pgfqpoint{2.612867in}{2.358044in}}%
\pgfpathlineto{\pgfqpoint{2.613712in}{1.965797in}}%
\pgfpathlineto{\pgfqpoint{2.614558in}{2.423297in}}%
\pgfpathlineto{\pgfqpoint{2.615404in}{2.161991in}}%
\pgfpathlineto{\pgfqpoint{2.617095in}{2.227360in}}%
\pgfpathlineto{\pgfqpoint{2.617940in}{2.161929in}}%
\pgfpathlineto{\pgfqpoint{2.618786in}{1.965562in}}%
\pgfpathlineto{\pgfqpoint{2.621323in}{2.554079in}}%
\pgfpathlineto{\pgfqpoint{2.623860in}{1.835265in}}%
\pgfpathlineto{\pgfqpoint{2.626397in}{2.619472in}}%
\pgfpathlineto{\pgfqpoint{2.628088in}{2.096532in}}%
\pgfpathlineto{\pgfqpoint{2.630625in}{2.367196in}}%
\pgfpathlineto{\pgfqpoint{2.631470in}{1.750696in}}%
\pgfpathlineto{\pgfqpoint{2.632316in}{2.043084in}}%
\pgfpathlineto{\pgfqpoint{2.634853in}{2.554418in}}%
\pgfpathlineto{\pgfqpoint{2.636544in}{1.729383in}}%
\pgfpathlineto{\pgfqpoint{2.639081in}{2.781941in}}%
\pgfpathlineto{\pgfqpoint{2.639927in}{1.929368in}}%
\pgfpathlineto{\pgfqpoint{2.640772in}{1.996206in}}%
\pgfpathlineto{\pgfqpoint{2.641618in}{2.587781in}}%
\pgfpathlineto{\pgfqpoint{2.642463in}{2.318284in}}%
\pgfpathlineto{\pgfqpoint{2.644155in}{1.704399in}}%
\pgfpathlineto{\pgfqpoint{2.645000in}{2.245969in}}%
\pgfpathlineto{\pgfqpoint{2.645846in}{2.212478in}}%
\pgfpathlineto{\pgfqpoint{2.646692in}{2.193638in}}%
\pgfpathlineto{\pgfqpoint{2.648383in}{2.476592in}}%
\pgfpathlineto{\pgfqpoint{2.650074in}{1.802145in}}%
\pgfpathlineto{\pgfqpoint{2.650920in}{2.389796in}}%
\pgfpathlineto{\pgfqpoint{2.651765in}{2.195573in}}%
\pgfpathlineto{\pgfqpoint{2.653457in}{1.933175in}}%
\pgfpathlineto{\pgfqpoint{2.654302in}{2.390835in}}%
\pgfpathlineto{\pgfqpoint{2.655148in}{2.260226in}}%
\pgfpathlineto{\pgfqpoint{2.655993in}{2.390582in}}%
\pgfpathlineto{\pgfqpoint{2.656839in}{1.998556in}}%
\pgfpathlineto{\pgfqpoint{2.657685in}{2.260085in}}%
\pgfpathlineto{\pgfqpoint{2.658530in}{2.390760in}}%
\pgfpathlineto{\pgfqpoint{2.661913in}{1.934402in}}%
\pgfpathlineto{\pgfqpoint{2.662758in}{2.193890in}}%
\pgfpathlineto{\pgfqpoint{2.663604in}{1.802147in}}%
\pgfpathlineto{\pgfqpoint{2.665295in}{2.586514in}}%
\pgfpathlineto{\pgfqpoint{2.667832in}{1.867781in}}%
\pgfpathlineto{\pgfqpoint{2.668678in}{1.933025in}}%
\pgfpathlineto{\pgfqpoint{2.671215in}{2.587029in}}%
\pgfpathlineto{\pgfqpoint{2.674597in}{1.671702in}}%
\pgfpathlineto{\pgfqpoint{2.675443in}{2.260020in}}%
\pgfpathlineto{\pgfqpoint{2.676288in}{2.195391in}}%
\pgfpathlineto{\pgfqpoint{2.677134in}{2.259890in}}%
\pgfpathlineto{\pgfqpoint{2.677980in}{1.671679in}}%
\pgfpathlineto{\pgfqpoint{2.679671in}{2.978976in}}%
\pgfpathlineto{\pgfqpoint{2.681362in}{1.933133in}}%
\pgfpathlineto{\pgfqpoint{2.682208in}{2.128662in}}%
\pgfpathlineto{\pgfqpoint{2.683053in}{1.606439in}}%
\pgfpathlineto{\pgfqpoint{2.683899in}{1.802904in}}%
\pgfpathlineto{\pgfqpoint{2.684745in}{2.586720in}}%
\pgfpathlineto{\pgfqpoint{2.685590in}{2.390661in}}%
\pgfpathlineto{\pgfqpoint{2.686436in}{2.325435in}}%
\pgfpathlineto{\pgfqpoint{2.687282in}{1.933251in}}%
\pgfpathlineto{\pgfqpoint{2.688127in}{2.259970in}}%
\pgfpathlineto{\pgfqpoint{2.688973in}{1.998461in}}%
\pgfpathlineto{\pgfqpoint{2.690664in}{2.194641in}}%
\pgfpathlineto{\pgfqpoint{2.691510in}{1.998483in}}%
\pgfpathlineto{\pgfqpoint{2.692355in}{2.456011in}}%
\pgfpathlineto{\pgfqpoint{2.693201in}{2.194560in}}%
\pgfpathlineto{\pgfqpoint{2.694047in}{2.129226in}}%
\pgfpathlineto{\pgfqpoint{2.694892in}{1.933105in}}%
\pgfpathlineto{\pgfqpoint{2.695738in}{2.325366in}}%
\pgfpathlineto{\pgfqpoint{2.696583in}{1.998466in}}%
\pgfpathlineto{\pgfqpoint{2.697429in}{2.782866in}}%
\pgfpathlineto{\pgfqpoint{2.699120in}{1.868325in}}%
\pgfpathlineto{\pgfqpoint{2.700812in}{2.521115in}}%
\pgfpathlineto{\pgfqpoint{2.702503in}{1.998398in}}%
\pgfpathlineto{\pgfqpoint{2.703348in}{2.521492in}}%
\pgfpathlineto{\pgfqpoint{2.704194in}{2.325275in}}%
\pgfpathlineto{\pgfqpoint{2.706731in}{1.933152in}}%
\pgfpathlineto{\pgfqpoint{2.707577in}{2.586800in}}%
\pgfpathlineto{\pgfqpoint{2.708422in}{2.390442in}}%
\pgfpathlineto{\pgfqpoint{2.710113in}{1.802365in}}%
\pgfpathlineto{\pgfqpoint{2.712650in}{2.194492in}}%
\pgfpathlineto{\pgfqpoint{2.713496in}{2.259304in}}%
\pgfpathlineto{\pgfqpoint{2.715187in}{2.586814in}}%
\pgfpathlineto{\pgfqpoint{2.716878in}{2.323816in}}%
\pgfpathlineto{\pgfqpoint{2.717724in}{2.652208in}}%
\pgfpathlineto{\pgfqpoint{2.719415in}{1.867753in}}%
\pgfpathlineto{\pgfqpoint{2.722798in}{2.455016in}}%
\pgfpathlineto{\pgfqpoint{2.723643in}{1.868756in}}%
\pgfpathlineto{\pgfqpoint{2.724489in}{1.998329in}}%
\pgfpathlineto{\pgfqpoint{2.725335in}{1.933326in}}%
\pgfpathlineto{\pgfqpoint{2.726180in}{2.325108in}}%
\pgfpathlineto{\pgfqpoint{2.727026in}{2.194909in}}%
\pgfpathlineto{\pgfqpoint{2.727871in}{2.130020in}}%
\pgfpathlineto{\pgfqpoint{2.729563in}{2.390345in}}%
\pgfpathlineto{\pgfqpoint{2.731254in}{2.079398in}}%
\pgfpathlineto{\pgfqpoint{2.732100in}{2.185858in}}%
\pgfpathlineto{\pgfqpoint{2.732945in}{2.063932in}}%
\pgfpathlineto{\pgfqpoint{2.733791in}{2.586498in}}%
\pgfpathlineto{\pgfqpoint{2.734636in}{2.259676in}}%
\pgfpathlineto{\pgfqpoint{2.735482in}{2.521374in}}%
\pgfpathlineto{\pgfqpoint{2.736328in}{1.933161in}}%
\pgfpathlineto{\pgfqpoint{2.737173in}{2.133526in}}%
\pgfpathlineto{\pgfqpoint{2.738019in}{2.322545in}}%
\pgfpathlineto{\pgfqpoint{2.738865in}{2.034276in}}%
\pgfpathlineto{\pgfqpoint{2.739710in}{2.618483in}}%
\pgfpathlineto{\pgfqpoint{2.741401in}{1.914623in}}%
\pgfpathlineto{\pgfqpoint{2.743093in}{2.227657in}}%
\pgfpathlineto{\pgfqpoint{2.743938in}{1.966090in}}%
\pgfpathlineto{\pgfqpoint{2.744784in}{2.356218in}}%
\pgfpathlineto{\pgfqpoint{2.745630in}{1.965340in}}%
\pgfpathlineto{\pgfqpoint{2.746475in}{2.096872in}}%
\pgfpathlineto{\pgfqpoint{2.747321in}{2.096616in}}%
\pgfpathlineto{\pgfqpoint{2.748166in}{2.552423in}}%
\pgfpathlineto{\pgfqpoint{2.749012in}{1.896004in}}%
\pgfpathlineto{\pgfqpoint{2.749858in}{2.096517in}}%
\pgfpathlineto{\pgfqpoint{2.750703in}{2.292652in}}%
\pgfpathlineto{\pgfqpoint{2.751549in}{1.875838in}}%
\pgfpathlineto{\pgfqpoint{2.752395in}{2.467749in}}%
\pgfpathlineto{\pgfqpoint{2.753240in}{2.365700in}}%
\pgfpathlineto{\pgfqpoint{2.754086in}{1.935716in}}%
\pgfpathlineto{\pgfqpoint{2.754931in}{2.147860in}}%
\pgfpathlineto{\pgfqpoint{2.755777in}{2.259497in}}%
\pgfpathlineto{\pgfqpoint{2.756623in}{2.651623in}}%
\pgfpathlineto{\pgfqpoint{2.757468in}{2.390248in}}%
\pgfpathlineto{\pgfqpoint{2.758314in}{2.521340in}}%
\pgfpathlineto{\pgfqpoint{2.760005in}{2.259918in}}%
\pgfpathlineto{\pgfqpoint{2.760851in}{2.389506in}}%
\pgfpathlineto{\pgfqpoint{2.761696in}{1.671681in}}%
\pgfpathlineto{\pgfqpoint{2.764233in}{2.717443in}}%
\pgfpathlineto{\pgfqpoint{2.765079in}{1.998493in}}%
\pgfpathlineto{\pgfqpoint{2.765925in}{2.129270in}}%
\pgfpathlineto{\pgfqpoint{2.766770in}{2.194710in}}%
\pgfpathlineto{\pgfqpoint{2.767616in}{1.998573in}}%
\pgfpathlineto{\pgfqpoint{2.768461in}{2.063901in}}%
\pgfpathlineto{\pgfqpoint{2.770998in}{2.260003in}}%
\pgfpathlineto{\pgfqpoint{2.771844in}{2.063681in}}%
\pgfpathlineto{\pgfqpoint{2.772690in}{2.258302in}}%
\pgfpathlineto{\pgfqpoint{2.773535in}{1.933106in}}%
\pgfpathlineto{\pgfqpoint{2.776072in}{2.521520in}}%
\pgfpathlineto{\pgfqpoint{2.776918in}{1.998513in}}%
\pgfpathlineto{\pgfqpoint{2.777763in}{2.390683in}}%
\pgfpathlineto{\pgfqpoint{2.778609in}{2.586632in}}%
\pgfpathlineto{\pgfqpoint{2.781146in}{1.802366in}}%
\pgfpathlineto{\pgfqpoint{2.783683in}{2.260067in}}%
\pgfpathlineto{\pgfqpoint{2.784528in}{2.325334in}}%
\pgfpathlineto{\pgfqpoint{2.785374in}{1.933134in}}%
\pgfpathlineto{\pgfqpoint{2.786220in}{1.998519in}}%
\pgfpathlineto{\pgfqpoint{2.787065in}{1.998430in}}%
\pgfpathlineto{\pgfqpoint{2.787911in}{1.933111in}}%
\pgfpathlineto{\pgfqpoint{2.788756in}{2.325038in}}%
\pgfpathlineto{\pgfqpoint{2.789602in}{2.129238in}}%
\pgfpathlineto{\pgfqpoint{2.790448in}{1.933113in}}%
\pgfpathlineto{\pgfqpoint{2.793830in}{2.521226in}}%
\pgfpathlineto{\pgfqpoint{2.794676in}{2.455195in}}%
\pgfpathlineto{\pgfqpoint{2.797213in}{1.999575in}}%
\pgfpathlineto{\pgfqpoint{2.798058in}{2.652221in}}%
\pgfpathlineto{\pgfqpoint{2.798904in}{1.933120in}}%
\pgfpathlineto{\pgfqpoint{2.799749in}{2.390774in}}%
\pgfpathlineto{\pgfqpoint{2.802286in}{2.129358in}}%
\pgfpathlineto{\pgfqpoint{2.804823in}{2.260396in}}%
\pgfpathlineto{\pgfqpoint{2.806514in}{1.606292in}}%
\pgfpathlineto{\pgfqpoint{2.807360in}{1.932937in}}%
\pgfpathlineto{\pgfqpoint{2.809897in}{2.456251in}}%
\pgfpathlineto{\pgfqpoint{2.810743in}{2.259856in}}%
\pgfpathlineto{\pgfqpoint{2.811588in}{2.260070in}}%
\pgfpathlineto{\pgfqpoint{2.812434in}{2.325302in}}%
\pgfpathlineto{\pgfqpoint{2.813279in}{2.717606in}}%
\pgfpathlineto{\pgfqpoint{2.814125in}{2.047259in}}%
\pgfpathlineto{\pgfqpoint{2.814971in}{2.063897in}}%
\pgfpathlineto{\pgfqpoint{2.815816in}{2.129813in}}%
\pgfpathlineto{\pgfqpoint{2.818353in}{2.456556in}}%
\pgfpathlineto{\pgfqpoint{2.819199in}{2.064004in}}%
\pgfpathlineto{\pgfqpoint{2.820044in}{2.194465in}}%
\pgfpathlineto{\pgfqpoint{2.820890in}{2.390820in}}%
\pgfpathlineto{\pgfqpoint{2.822581in}{1.867720in}}%
\pgfpathlineto{\pgfqpoint{2.823427in}{2.063401in}}%
\pgfpathlineto{\pgfqpoint{2.824273in}{1.737360in}}%
\pgfpathlineto{\pgfqpoint{2.825118in}{2.325341in}}%
\pgfpathlineto{\pgfqpoint{2.825964in}{1.998540in}}%
\pgfpathlineto{\pgfqpoint{2.826809in}{2.194828in}}%
\pgfpathlineto{\pgfqpoint{2.827655in}{1.997504in}}%
\pgfpathlineto{\pgfqpoint{2.828501in}{2.078858in}}%
\pgfpathlineto{\pgfqpoint{2.830192in}{2.644638in}}%
\pgfpathlineto{\pgfqpoint{2.831883in}{2.063608in}}%
\pgfpathlineto{\pgfqpoint{2.832729in}{2.308239in}}%
\pgfpathlineto{\pgfqpoint{2.833574in}{2.073313in}}%
\pgfpathlineto{\pgfqpoint{2.835266in}{2.690474in}}%
\pgfpathlineto{\pgfqpoint{2.837803in}{1.965790in}}%
\pgfpathlineto{\pgfqpoint{2.838648in}{1.965964in}}%
\pgfpathlineto{\pgfqpoint{2.841185in}{2.423350in}}%
\pgfpathlineto{\pgfqpoint{2.842031in}{2.096549in}}%
\pgfpathlineto{\pgfqpoint{2.842876in}{2.292624in}}%
\pgfpathlineto{\pgfqpoint{2.844568in}{1.704310in}}%
\pgfpathlineto{\pgfqpoint{2.845413in}{2.488562in}}%
\pgfpathlineto{\pgfqpoint{2.846259in}{2.096780in}}%
\pgfpathlineto{\pgfqpoint{2.847104in}{2.096537in}}%
\pgfpathlineto{\pgfqpoint{2.847950in}{2.031188in}}%
\pgfpathlineto{\pgfqpoint{2.848796in}{2.292687in}}%
\pgfpathlineto{\pgfqpoint{2.849641in}{2.227225in}}%
\pgfpathlineto{\pgfqpoint{2.850487in}{2.031221in}}%
\pgfpathlineto{\pgfqpoint{2.851333in}{2.227773in}}%
\pgfpathlineto{\pgfqpoint{2.852178in}{2.162019in}}%
\pgfpathlineto{\pgfqpoint{2.853024in}{1.703545in}}%
\pgfpathlineto{\pgfqpoint{2.853869in}{1.965843in}}%
\pgfpathlineto{\pgfqpoint{2.854715in}{2.423276in}}%
\pgfpathlineto{\pgfqpoint{2.855561in}{1.900382in}}%
\pgfpathlineto{\pgfqpoint{2.856406in}{2.292572in}}%
\pgfpathlineto{\pgfqpoint{2.857252in}{2.749624in}}%
\pgfpathlineto{\pgfqpoint{2.858098in}{1.835479in}}%
\pgfpathlineto{\pgfqpoint{2.858943in}{2.424301in}}%
\pgfpathlineto{\pgfqpoint{2.859789in}{2.292629in}}%
\pgfpathlineto{\pgfqpoint{2.860634in}{1.573510in}}%
\pgfpathlineto{\pgfqpoint{2.861480in}{2.293107in}}%
\pgfpathlineto{\pgfqpoint{2.862326in}{2.031267in}}%
\pgfpathlineto{\pgfqpoint{2.863171in}{1.702299in}}%
\pgfpathlineto{\pgfqpoint{2.864017in}{2.619349in}}%
\pgfpathlineto{\pgfqpoint{2.864863in}{2.095668in}}%
\pgfpathlineto{\pgfqpoint{2.865708in}{2.227167in}}%
\pgfpathlineto{\pgfqpoint{2.866554in}{1.966826in}}%
\pgfpathlineto{\pgfqpoint{2.867399in}{2.358402in}}%
\pgfpathlineto{\pgfqpoint{2.868245in}{2.031065in}}%
\pgfpathlineto{\pgfqpoint{2.869091in}{2.552984in}}%
\pgfpathlineto{\pgfqpoint{2.869936in}{1.965968in}}%
\pgfpathlineto{\pgfqpoint{2.870782in}{2.358111in}}%
\pgfpathlineto{\pgfqpoint{2.871628in}{2.618603in}}%
\pgfpathlineto{\pgfqpoint{2.874164in}{1.965966in}}%
\pgfpathlineto{\pgfqpoint{2.875010in}{1.897629in}}%
\pgfpathlineto{\pgfqpoint{2.875856in}{2.423524in}}%
\pgfpathlineto{\pgfqpoint{2.876701in}{2.227593in}}%
\pgfpathlineto{\pgfqpoint{2.877547in}{2.293128in}}%
\pgfpathlineto{\pgfqpoint{2.879238in}{2.161851in}}%
\pgfpathlineto{\pgfqpoint{2.880084in}{1.901160in}}%
\pgfpathlineto{\pgfqpoint{2.880929in}{2.296574in}}%
\pgfpathlineto{\pgfqpoint{2.881775in}{2.224132in}}%
\pgfpathlineto{\pgfqpoint{2.882621in}{2.094982in}}%
\pgfpathlineto{\pgfqpoint{2.883466in}{2.551588in}}%
\pgfpathlineto{\pgfqpoint{2.884312in}{2.339623in}}%
\pgfpathlineto{\pgfqpoint{2.886849in}{1.933284in}}%
\pgfpathlineto{\pgfqpoint{2.888540in}{2.456353in}}%
\pgfpathlineto{\pgfqpoint{2.889386in}{1.867319in}}%
\pgfpathlineto{\pgfqpoint{2.890231in}{2.195209in}}%
\pgfpathlineto{\pgfqpoint{2.891077in}{2.063184in}}%
\pgfpathlineto{\pgfqpoint{2.891922in}{2.523873in}}%
\pgfpathlineto{\pgfqpoint{2.892768in}{2.064333in}}%
\pgfpathlineto{\pgfqpoint{2.893614in}{2.064411in}}%
\pgfpathlineto{\pgfqpoint{2.894459in}{2.060539in}}%
\pgfpathlineto{\pgfqpoint{2.895305in}{1.999614in}}%
\pgfpathlineto{\pgfqpoint{2.896151in}{1.792671in}}%
\pgfpathlineto{\pgfqpoint{2.897842in}{2.396194in}}%
\pgfpathlineto{\pgfqpoint{2.899533in}{1.802961in}}%
\pgfpathlineto{\pgfqpoint{2.900379in}{2.193127in}}%
\pgfpathlineto{\pgfqpoint{2.901224in}{2.060985in}}%
\pgfpathlineto{\pgfqpoint{2.902916in}{2.397373in}}%
\pgfpathlineto{\pgfqpoint{2.903761in}{2.381657in}}%
\pgfpathlineto{\pgfqpoint{2.905452in}{1.909886in}}%
\pgfpathlineto{\pgfqpoint{2.907989in}{2.453124in}}%
\pgfpathlineto{\pgfqpoint{2.909681in}{1.926690in}}%
\pgfpathlineto{\pgfqpoint{2.910526in}{2.305866in}}%
\pgfpathlineto{\pgfqpoint{2.911372in}{2.120526in}}%
\pgfpathlineto{\pgfqpoint{2.912217in}{2.206849in}}%
\pgfpathlineto{\pgfqpoint{2.913909in}{1.769720in}}%
\pgfpathlineto{\pgfqpoint{2.916446in}{2.488418in}}%
\pgfpathlineto{\pgfqpoint{2.917291in}{1.900526in}}%
\pgfpathlineto{\pgfqpoint{2.918982in}{2.423357in}}%
\pgfpathlineto{\pgfqpoint{2.921519in}{1.443000in}}%
\pgfpathlineto{\pgfqpoint{2.922365in}{2.227162in}}%
\pgfpathlineto{\pgfqpoint{2.923211in}{2.161786in}}%
\pgfpathlineto{\pgfqpoint{2.924056in}{2.292483in}}%
\pgfpathlineto{\pgfqpoint{2.924902in}{2.031845in}}%
\pgfpathlineto{\pgfqpoint{2.925747in}{2.232262in}}%
\pgfpathlineto{\pgfqpoint{2.926593in}{2.031357in}}%
\pgfpathlineto{\pgfqpoint{2.928284in}{2.423396in}}%
\pgfpathlineto{\pgfqpoint{2.929130in}{1.835614in}}%
\pgfpathlineto{\pgfqpoint{2.929976in}{2.357243in}}%
\pgfpathlineto{\pgfqpoint{2.930821in}{2.031168in}}%
\pgfpathlineto{\pgfqpoint{2.932512in}{2.262355in}}%
\pgfpathlineto{\pgfqpoint{2.933358in}{1.965854in}}%
\pgfpathlineto{\pgfqpoint{2.934204in}{2.359747in}}%
\pgfpathlineto{\pgfqpoint{2.935049in}{2.271823in}}%
\pgfpathlineto{\pgfqpoint{2.935895in}{2.292625in}}%
\pgfpathlineto{\pgfqpoint{2.936741in}{2.000584in}}%
\pgfpathlineto{\pgfqpoint{2.937586in}{2.015605in}}%
\pgfpathlineto{\pgfqpoint{2.939277in}{2.370257in}}%
\pgfpathlineto{\pgfqpoint{2.941814in}{1.896540in}}%
\pgfpathlineto{\pgfqpoint{2.944351in}{2.357976in}}%
\pgfpathlineto{\pgfqpoint{2.946042in}{1.996079in}}%
\pgfpathlineto{\pgfqpoint{2.946888in}{2.613059in}}%
\pgfpathlineto{\pgfqpoint{2.947734in}{2.292602in}}%
\pgfpathlineto{\pgfqpoint{2.948579in}{2.234892in}}%
\pgfpathlineto{\pgfqpoint{2.949425in}{2.291678in}}%
\pgfpathlineto{\pgfqpoint{2.950271in}{2.096562in}}%
\pgfpathlineto{\pgfqpoint{2.951116in}{2.488386in}}%
\pgfpathlineto{\pgfqpoint{2.952807in}{1.704344in}}%
\pgfpathlineto{\pgfqpoint{2.954499in}{2.488444in}}%
\pgfpathlineto{\pgfqpoint{2.956190in}{2.227344in}}%
\pgfpathlineto{\pgfqpoint{2.957035in}{2.226953in}}%
\pgfpathlineto{\pgfqpoint{2.958727in}{2.619541in}}%
\pgfpathlineto{\pgfqpoint{2.959572in}{2.031177in}}%
\pgfpathlineto{\pgfqpoint{2.960418in}{2.619434in}}%
\pgfpathlineto{\pgfqpoint{2.962109in}{2.031369in}}%
\pgfpathlineto{\pgfqpoint{2.962955in}{2.421753in}}%
\pgfpathlineto{\pgfqpoint{2.963800in}{2.031210in}}%
\pgfpathlineto{\pgfqpoint{2.964646in}{2.164696in}}%
\pgfpathlineto{\pgfqpoint{2.965492in}{2.488582in}}%
\pgfpathlineto{\pgfqpoint{2.966337in}{1.769806in}}%
\pgfpathlineto{\pgfqpoint{2.967183in}{2.031064in}}%
\pgfpathlineto{\pgfqpoint{2.968874in}{2.488665in}}%
\pgfpathlineto{\pgfqpoint{2.970565in}{2.096887in}}%
\pgfpathlineto{\pgfqpoint{2.971411in}{2.224488in}}%
\pgfpathlineto{\pgfqpoint{2.972257in}{2.228492in}}%
\pgfpathlineto{\pgfqpoint{2.973102in}{2.174610in}}%
\pgfpathlineto{\pgfqpoint{2.973948in}{2.358159in}}%
\pgfpathlineto{\pgfqpoint{2.974794in}{2.227305in}}%
\pgfpathlineto{\pgfqpoint{2.975639in}{2.180442in}}%
\pgfpathlineto{\pgfqpoint{2.976485in}{2.031144in}}%
\pgfpathlineto{\pgfqpoint{2.978176in}{2.232239in}}%
\pgfpathlineto{\pgfqpoint{2.979022in}{2.031094in}}%
\pgfpathlineto{\pgfqpoint{2.979867in}{2.291596in}}%
\pgfpathlineto{\pgfqpoint{2.980713in}{2.030090in}}%
\pgfpathlineto{\pgfqpoint{2.981559in}{2.423341in}}%
\pgfpathlineto{\pgfqpoint{2.982404in}{2.160727in}}%
\pgfpathlineto{\pgfqpoint{2.984095in}{1.900015in}}%
\pgfpathlineto{\pgfqpoint{2.984941in}{2.378870in}}%
\pgfpathlineto{\pgfqpoint{2.985787in}{2.096778in}}%
\pgfpathlineto{\pgfqpoint{2.987478in}{2.686580in}}%
\pgfpathlineto{\pgfqpoint{2.988324in}{1.733110in}}%
\pgfpathlineto{\pgfqpoint{2.989169in}{2.083080in}}%
\pgfpathlineto{\pgfqpoint{2.990015in}{2.134083in}}%
\pgfpathlineto{\pgfqpoint{2.991706in}{2.649547in}}%
\pgfpathlineto{\pgfqpoint{2.992552in}{2.521466in}}%
\pgfpathlineto{\pgfqpoint{2.994243in}{2.065133in}}%
\pgfpathlineto{\pgfqpoint{2.996780in}{2.326706in}}%
\pgfpathlineto{\pgfqpoint{2.998471in}{1.737943in}}%
\pgfpathlineto{\pgfqpoint{2.999317in}{1.867267in}}%
\pgfpathlineto{\pgfqpoint{3.000162in}{2.258920in}}%
\pgfpathlineto{\pgfqpoint{3.001008in}{1.867912in}}%
\pgfpathlineto{\pgfqpoint{3.001854in}{2.586442in}}%
\pgfpathlineto{\pgfqpoint{3.002699in}{2.390812in}}%
\pgfpathlineto{\pgfqpoint{3.003545in}{2.520997in}}%
\pgfpathlineto{\pgfqpoint{3.006082in}{1.737485in}}%
\pgfpathlineto{\pgfqpoint{3.006927in}{2.521094in}}%
\pgfpathlineto{\pgfqpoint{3.007773in}{2.129663in}}%
\pgfpathlineto{\pgfqpoint{3.008619in}{2.129257in}}%
\pgfpathlineto{\pgfqpoint{3.009464in}{2.465351in}}%
\pgfpathlineto{\pgfqpoint{3.010310in}{2.002770in}}%
\pgfpathlineto{\pgfqpoint{3.012001in}{2.716243in}}%
\pgfpathlineto{\pgfqpoint{3.014538in}{1.994349in}}%
\pgfpathlineto{\pgfqpoint{3.015384in}{2.063994in}}%
\pgfpathlineto{\pgfqpoint{3.016229in}{2.195555in}}%
\pgfpathlineto{\pgfqpoint{3.017075in}{2.130608in}}%
\pgfpathlineto{\pgfqpoint{3.017920in}{2.129846in}}%
\pgfpathlineto{\pgfqpoint{3.020457in}{2.325335in}}%
\pgfpathlineto{\pgfqpoint{3.021303in}{1.802492in}}%
\pgfpathlineto{\pgfqpoint{3.022994in}{2.652449in}}%
\pgfpathlineto{\pgfqpoint{3.023840in}{2.129039in}}%
\pgfpathlineto{\pgfqpoint{3.024685in}{2.194500in}}%
\pgfpathlineto{\pgfqpoint{3.025531in}{2.129260in}}%
\pgfpathlineto{\pgfqpoint{3.026377in}{2.259857in}}%
\pgfpathlineto{\pgfqpoint{3.027222in}{1.802928in}}%
\pgfpathlineto{\pgfqpoint{3.028068in}{1.998673in}}%
\pgfpathlineto{\pgfqpoint{3.028914in}{2.848278in}}%
\pgfpathlineto{\pgfqpoint{3.030605in}{1.932758in}}%
\pgfpathlineto{\pgfqpoint{3.031450in}{2.521568in}}%
\pgfpathlineto{\pgfqpoint{3.032296in}{2.129343in}}%
\pgfpathlineto{\pgfqpoint{3.033142in}{1.933347in}}%
\pgfpathlineto{\pgfqpoint{3.033987in}{2.195594in}}%
\pgfpathlineto{\pgfqpoint{3.034833in}{1.736923in}}%
\pgfpathlineto{\pgfqpoint{3.035678in}{2.455973in}}%
\pgfpathlineto{\pgfqpoint{3.036524in}{2.129125in}}%
\pgfpathlineto{\pgfqpoint{3.038215in}{2.390687in}}%
\pgfpathlineto{\pgfqpoint{3.039061in}{2.062980in}}%
\pgfpathlineto{\pgfqpoint{3.039907in}{2.194547in}}%
\pgfpathlineto{\pgfqpoint{3.040752in}{2.129311in}}%
\pgfpathlineto{\pgfqpoint{3.043289in}{2.521485in}}%
\pgfpathlineto{\pgfqpoint{3.044135in}{2.194943in}}%
\pgfpathlineto{\pgfqpoint{3.044980in}{2.652468in}}%
\pgfpathlineto{\pgfqpoint{3.045826in}{2.129137in}}%
\pgfpathlineto{\pgfqpoint{3.046672in}{2.521312in}}%
\pgfpathlineto{\pgfqpoint{3.047517in}{2.259796in}}%
\pgfpathlineto{\pgfqpoint{3.048363in}{2.260141in}}%
\pgfpathlineto{\pgfqpoint{3.049208in}{2.259198in}}%
\pgfpathlineto{\pgfqpoint{3.050054in}{2.390018in}}%
\pgfpathlineto{\pgfqpoint{3.052591in}{1.867985in}}%
\pgfpathlineto{\pgfqpoint{3.055128in}{2.324617in}}%
\pgfpathlineto{\pgfqpoint{3.055973in}{2.260616in}}%
\pgfpathlineto{\pgfqpoint{3.057665in}{2.258701in}}%
\pgfpathlineto{\pgfqpoint{3.059356in}{1.736783in}}%
\pgfpathlineto{\pgfqpoint{3.061893in}{2.456837in}}%
\pgfpathlineto{\pgfqpoint{3.063584in}{2.258347in}}%
\pgfpathlineto{\pgfqpoint{3.064430in}{2.455090in}}%
\pgfpathlineto{\pgfqpoint{3.065275in}{1.802416in}}%
\pgfpathlineto{\pgfqpoint{3.066121in}{1.998577in}}%
\pgfpathlineto{\pgfqpoint{3.069503in}{2.717550in}}%
\pgfpathlineto{\pgfqpoint{3.070349in}{2.194539in}}%
\pgfpathlineto{\pgfqpoint{3.071195in}{2.390634in}}%
\pgfpathlineto{\pgfqpoint{3.072886in}{2.063948in}}%
\pgfpathlineto{\pgfqpoint{3.073732in}{2.325348in}}%
\pgfpathlineto{\pgfqpoint{3.074577in}{2.194607in}}%
\pgfpathlineto{\pgfqpoint{3.075423in}{2.259959in}}%
\pgfpathlineto{\pgfqpoint{3.076268in}{2.453400in}}%
\pgfpathlineto{\pgfqpoint{3.077114in}{2.325325in}}%
\pgfpathlineto{\pgfqpoint{3.077960in}{1.998483in}}%
\pgfpathlineto{\pgfqpoint{3.078805in}{2.259056in}}%
\pgfpathlineto{\pgfqpoint{3.079651in}{2.259847in}}%
\pgfpathlineto{\pgfqpoint{3.080497in}{1.934156in}}%
\pgfpathlineto{\pgfqpoint{3.081342in}{2.129249in}}%
\pgfpathlineto{\pgfqpoint{3.083033in}{2.259951in}}%
\pgfpathlineto{\pgfqpoint{3.083879in}{2.456033in}}%
\pgfpathlineto{\pgfqpoint{3.084725in}{1.738180in}}%
\pgfpathlineto{\pgfqpoint{3.085570in}{2.131419in}}%
\pgfpathlineto{\pgfqpoint{3.086416in}{2.129197in}}%
\pgfpathlineto{\pgfqpoint{3.087262in}{2.387959in}}%
\pgfpathlineto{\pgfqpoint{3.089798in}{1.934079in}}%
\pgfpathlineto{\pgfqpoint{3.090644in}{2.325145in}}%
\pgfpathlineto{\pgfqpoint{3.091490in}{2.063487in}}%
\pgfpathlineto{\pgfqpoint{3.092335in}{1.944729in}}%
\pgfpathlineto{\pgfqpoint{3.093181in}{2.455811in}}%
\pgfpathlineto{\pgfqpoint{3.094027in}{2.067048in}}%
\pgfpathlineto{\pgfqpoint{3.094872in}{1.998655in}}%
\pgfpathlineto{\pgfqpoint{3.097409in}{2.521376in}}%
\pgfpathlineto{\pgfqpoint{3.098255in}{1.802453in}}%
\pgfpathlineto{\pgfqpoint{3.099100in}{2.063881in}}%
\pgfpathlineto{\pgfqpoint{3.100792in}{2.390847in}}%
\pgfpathlineto{\pgfqpoint{3.101637in}{1.802387in}}%
\pgfpathlineto{\pgfqpoint{3.102483in}{2.259830in}}%
\pgfpathlineto{\pgfqpoint{3.105020in}{2.063999in}}%
\pgfpathlineto{\pgfqpoint{3.106711in}{2.456422in}}%
\pgfpathlineto{\pgfqpoint{3.108402in}{1.998615in}}%
\pgfpathlineto{\pgfqpoint{3.110093in}{2.325453in}}%
\pgfpathlineto{\pgfqpoint{3.111785in}{2.321891in}}%
\pgfpathlineto{\pgfqpoint{3.112630in}{2.390791in}}%
\pgfpathlineto{\pgfqpoint{3.113476in}{2.129308in}}%
\pgfpathlineto{\pgfqpoint{3.114322in}{2.259899in}}%
\pgfpathlineto{\pgfqpoint{3.115167in}{2.193973in}}%
\pgfpathlineto{\pgfqpoint{3.116013in}{2.390631in}}%
\pgfpathlineto{\pgfqpoint{3.119395in}{1.737907in}}%
\pgfpathlineto{\pgfqpoint{3.120241in}{2.453728in}}%
\pgfpathlineto{\pgfqpoint{3.121086in}{2.452311in}}%
\pgfpathlineto{\pgfqpoint{3.121932in}{2.031540in}}%
\pgfpathlineto{\pgfqpoint{3.122778in}{2.431108in}}%
\pgfpathlineto{\pgfqpoint{3.123623in}{1.904083in}}%
\pgfpathlineto{\pgfqpoint{3.124469in}{2.488842in}}%
\pgfpathlineto{\pgfqpoint{3.125315in}{1.900431in}}%
\pgfpathlineto{\pgfqpoint{3.126160in}{2.684804in}}%
\pgfpathlineto{\pgfqpoint{3.127006in}{2.161905in}}%
\pgfpathlineto{\pgfqpoint{3.127851in}{2.031184in}}%
\pgfpathlineto{\pgfqpoint{3.130388in}{2.358060in}}%
\pgfpathlineto{\pgfqpoint{3.131234in}{2.423300in}}%
\pgfpathlineto{\pgfqpoint{3.132080in}{2.031201in}}%
\pgfpathlineto{\pgfqpoint{3.132925in}{2.161939in}}%
\pgfpathlineto{\pgfqpoint{3.133771in}{2.161874in}}%
\pgfpathlineto{\pgfqpoint{3.135462in}{1.845436in}}%
\pgfpathlineto{\pgfqpoint{3.136308in}{2.255363in}}%
\pgfpathlineto{\pgfqpoint{3.137153in}{2.059082in}}%
\pgfpathlineto{\pgfqpoint{3.137999in}{2.302645in}}%
\pgfpathlineto{\pgfqpoint{3.138845in}{2.161902in}}%
\pgfpathlineto{\pgfqpoint{3.140536in}{1.965823in}}%
\pgfpathlineto{\pgfqpoint{3.141381in}{2.423753in}}%
\pgfpathlineto{\pgfqpoint{3.142227in}{2.096210in}}%
\pgfpathlineto{\pgfqpoint{3.144764in}{2.358021in}}%
\pgfpathlineto{\pgfqpoint{3.145610in}{2.161711in}}%
\pgfpathlineto{\pgfqpoint{3.147301in}{2.423395in}}%
\pgfpathlineto{\pgfqpoint{3.148146in}{1.835696in}}%
\pgfpathlineto{\pgfqpoint{3.150683in}{2.554233in}}%
\pgfpathlineto{\pgfqpoint{3.153220in}{1.845656in}}%
\pgfpathlineto{\pgfqpoint{3.154911in}{2.358249in}}%
\pgfpathlineto{\pgfqpoint{3.155757in}{2.423403in}}%
\pgfpathlineto{\pgfqpoint{3.156603in}{1.835114in}}%
\pgfpathlineto{\pgfqpoint{3.157448in}{2.488842in}}%
\pgfpathlineto{\pgfqpoint{3.158294in}{2.161922in}}%
\pgfpathlineto{\pgfqpoint{3.159140in}{1.965781in}}%
\pgfpathlineto{\pgfqpoint{3.160831in}{2.488735in}}%
\pgfpathlineto{\pgfqpoint{3.161676in}{2.422963in}}%
\pgfpathlineto{\pgfqpoint{3.162522in}{2.358074in}}%
\pgfpathlineto{\pgfqpoint{3.163368in}{2.488399in}}%
\pgfpathlineto{\pgfqpoint{3.164213in}{2.161884in}}%
\pgfpathlineto{\pgfqpoint{3.165059in}{2.162168in}}%
\pgfpathlineto{\pgfqpoint{3.165905in}{2.340558in}}%
\pgfpathlineto{\pgfqpoint{3.166750in}{2.263963in}}%
\pgfpathlineto{\pgfqpoint{3.168441in}{1.769685in}}%
\pgfpathlineto{\pgfqpoint{3.170133in}{2.292690in}}%
\pgfpathlineto{\pgfqpoint{3.171824in}{1.900949in}}%
\pgfpathlineto{\pgfqpoint{3.172670in}{2.488715in}}%
\pgfpathlineto{\pgfqpoint{3.173515in}{1.965944in}}%
\pgfpathlineto{\pgfqpoint{3.174361in}{2.162748in}}%
\pgfpathlineto{\pgfqpoint{3.175206in}{1.966579in}}%
\pgfpathlineto{\pgfqpoint{3.176052in}{2.553912in}}%
\pgfpathlineto{\pgfqpoint{3.176898in}{2.357919in}}%
\pgfpathlineto{\pgfqpoint{3.179435in}{1.965763in}}%
\pgfpathlineto{\pgfqpoint{3.180280in}{1.900133in}}%
\pgfpathlineto{\pgfqpoint{3.181971in}{2.162354in}}%
\pgfpathlineto{\pgfqpoint{3.182817in}{2.144215in}}%
\pgfpathlineto{\pgfqpoint{3.186200in}{2.422959in}}%
\pgfpathlineto{\pgfqpoint{3.188736in}{1.901148in}}%
\pgfpathlineto{\pgfqpoint{3.190428in}{2.554955in}}%
\pgfpathlineto{\pgfqpoint{3.191273in}{2.162120in}}%
\pgfpathlineto{\pgfqpoint{3.192119in}{2.292678in}}%
\pgfpathlineto{\pgfqpoint{3.192965in}{2.290245in}}%
\pgfpathlineto{\pgfqpoint{3.193810in}{1.900595in}}%
\pgfpathlineto{\pgfqpoint{3.194656in}{2.096477in}}%
\pgfpathlineto{\pgfqpoint{3.195501in}{2.031236in}}%
\pgfpathlineto{\pgfqpoint{3.199729in}{2.554048in}}%
\pgfpathlineto{\pgfqpoint{3.201421in}{2.031260in}}%
\pgfpathlineto{\pgfqpoint{3.202266in}{2.423235in}}%
\pgfpathlineto{\pgfqpoint{3.203112in}{2.227217in}}%
\pgfpathlineto{\pgfqpoint{3.203958in}{1.965979in}}%
\pgfpathlineto{\pgfqpoint{3.204803in}{2.098222in}}%
\pgfpathlineto{\pgfqpoint{3.205649in}{2.227333in}}%
\pgfpathlineto{\pgfqpoint{3.206494in}{2.619324in}}%
\pgfpathlineto{\pgfqpoint{3.207340in}{2.421353in}}%
\pgfpathlineto{\pgfqpoint{3.209031in}{2.031150in}}%
\pgfpathlineto{\pgfqpoint{3.209877in}{2.488836in}}%
\pgfpathlineto{\pgfqpoint{3.211568in}{1.767961in}}%
\pgfpathlineto{\pgfqpoint{3.213259in}{2.096643in}}%
\pgfpathlineto{\pgfqpoint{3.214105in}{2.031355in}}%
\pgfpathlineto{\pgfqpoint{3.214951in}{2.404820in}}%
\pgfpathlineto{\pgfqpoint{3.215796in}{2.257104in}}%
\pgfpathlineto{\pgfqpoint{3.216642in}{2.054601in}}%
\pgfpathlineto{\pgfqpoint{3.217488in}{2.334795in}}%
\pgfpathlineto{\pgfqpoint{3.218333in}{2.244739in}}%
\pgfpathlineto{\pgfqpoint{3.219179in}{2.077136in}}%
\pgfpathlineto{\pgfqpoint{3.220870in}{2.589211in}}%
\pgfpathlineto{\pgfqpoint{3.223407in}{1.933740in}}%
\pgfpathlineto{\pgfqpoint{3.225098in}{2.325714in}}%
\pgfpathlineto{\pgfqpoint{3.225944in}{1.998546in}}%
\pgfpathlineto{\pgfqpoint{3.226789in}{2.523186in}}%
\pgfpathlineto{\pgfqpoint{3.229326in}{1.737213in}}%
\pgfpathlineto{\pgfqpoint{3.230172in}{2.129249in}}%
\pgfpathlineto{\pgfqpoint{3.231018in}{1.870557in}}%
\pgfpathlineto{\pgfqpoint{3.233554in}{2.465656in}}%
\pgfpathlineto{\pgfqpoint{3.234400in}{2.456198in}}%
\pgfpathlineto{\pgfqpoint{3.235246in}{2.194622in}}%
\pgfpathlineto{\pgfqpoint{3.236091in}{2.586804in}}%
\pgfpathlineto{\pgfqpoint{3.236937in}{1.671871in}}%
\pgfpathlineto{\pgfqpoint{3.237783in}{2.129245in}}%
\pgfpathlineto{\pgfqpoint{3.238628in}{2.586566in}}%
\pgfpathlineto{\pgfqpoint{3.239474in}{2.390634in}}%
\pgfpathlineto{\pgfqpoint{3.242011in}{1.998534in}}%
\pgfpathlineto{\pgfqpoint{3.242856in}{2.325254in}}%
\pgfpathlineto{\pgfqpoint{3.243702in}{2.259949in}}%
\pgfpathlineto{\pgfqpoint{3.244548in}{1.933118in}}%
\pgfpathlineto{\pgfqpoint{3.246239in}{2.393190in}}%
\pgfpathlineto{\pgfqpoint{3.247084in}{2.194397in}}%
\pgfpathlineto{\pgfqpoint{3.247930in}{2.456271in}}%
\pgfpathlineto{\pgfqpoint{3.248776in}{1.933105in}}%
\pgfpathlineto{\pgfqpoint{3.249621in}{2.259972in}}%
\pgfpathlineto{\pgfqpoint{3.251313in}{1.802335in}}%
\pgfpathlineto{\pgfqpoint{3.252158in}{2.456062in}}%
\pgfpathlineto{\pgfqpoint{3.253004in}{2.129277in}}%
\pgfpathlineto{\pgfqpoint{3.253849in}{1.998524in}}%
\pgfpathlineto{\pgfqpoint{3.254695in}{2.129179in}}%
\pgfpathlineto{\pgfqpoint{3.255541in}{1.933146in}}%
\pgfpathlineto{\pgfqpoint{3.257232in}{2.455998in}}%
\pgfpathlineto{\pgfqpoint{3.258923in}{2.325335in}}%
\pgfpathlineto{\pgfqpoint{3.261460in}{1.998538in}}%
\pgfpathlineto{\pgfqpoint{3.262306in}{2.129280in}}%
\pgfpathlineto{\pgfqpoint{3.263151in}{2.521519in}}%
\pgfpathlineto{\pgfqpoint{3.265688in}{1.671702in}}%
\pgfpathlineto{\pgfqpoint{3.266534in}{2.533244in}}%
\pgfpathlineto{\pgfqpoint{3.267379in}{2.063866in}}%
\pgfpathlineto{\pgfqpoint{3.268225in}{1.998538in}}%
\pgfpathlineto{\pgfqpoint{3.269071in}{2.259959in}}%
\pgfpathlineto{\pgfqpoint{3.269916in}{2.194735in}}%
\pgfpathlineto{\pgfqpoint{3.270762in}{2.026923in}}%
\pgfpathlineto{\pgfqpoint{3.271608in}{2.374190in}}%
\pgfpathlineto{\pgfqpoint{3.272453in}{2.292221in}}%
\pgfpathlineto{\pgfqpoint{3.274144in}{2.102102in}}%
\pgfpathlineto{\pgfqpoint{3.276681in}{2.554077in}}%
\pgfpathlineto{\pgfqpoint{3.279218in}{2.031128in}}%
\pgfpathlineto{\pgfqpoint{3.280064in}{2.096479in}}%
\pgfpathlineto{\pgfqpoint{3.280909in}{2.031258in}}%
\pgfpathlineto{\pgfqpoint{3.283446in}{2.815594in}}%
\pgfpathlineto{\pgfqpoint{3.284292in}{1.965815in}}%
\pgfpathlineto{\pgfqpoint{3.285137in}{2.096533in}}%
\pgfpathlineto{\pgfqpoint{3.286829in}{2.423494in}}%
\pgfpathlineto{\pgfqpoint{3.287674in}{2.161891in}}%
\pgfpathlineto{\pgfqpoint{3.288520in}{2.357981in}}%
\pgfpathlineto{\pgfqpoint{3.289366in}{2.227294in}}%
\pgfpathlineto{\pgfqpoint{3.290211in}{2.488763in}}%
\pgfpathlineto{\pgfqpoint{3.293594in}{1.508198in}}%
\pgfpathlineto{\pgfqpoint{3.296131in}{2.358019in}}%
\pgfpathlineto{\pgfqpoint{3.296976in}{2.554252in}}%
\pgfpathlineto{\pgfqpoint{3.299513in}{1.835051in}}%
\pgfpathlineto{\pgfqpoint{3.300359in}{2.554232in}}%
\pgfpathlineto{\pgfqpoint{3.301204in}{2.096518in}}%
\pgfpathlineto{\pgfqpoint{3.302050in}{2.357979in}}%
\pgfpathlineto{\pgfqpoint{3.302896in}{2.292697in}}%
\pgfpathlineto{\pgfqpoint{3.303741in}{2.227244in}}%
\pgfpathlineto{\pgfqpoint{3.305432in}{2.029318in}}%
\pgfpathlineto{\pgfqpoint{3.306278in}{2.031169in}}%
\pgfpathlineto{\pgfqpoint{3.307124in}{2.554084in}}%
\pgfpathlineto{\pgfqpoint{3.309661in}{1.704375in}}%
\pgfpathlineto{\pgfqpoint{3.310506in}{2.684836in}}%
\pgfpathlineto{\pgfqpoint{3.311352in}{1.900434in}}%
\pgfpathlineto{\pgfqpoint{3.312197in}{2.292605in}}%
\pgfpathlineto{\pgfqpoint{3.313043in}{1.835061in}}%
\pgfpathlineto{\pgfqpoint{3.313889in}{1.965924in}}%
\pgfpathlineto{\pgfqpoint{3.314734in}{1.965763in}}%
\pgfpathlineto{\pgfqpoint{3.315580in}{2.031263in}}%
\pgfpathlineto{\pgfqpoint{3.316426in}{2.488678in}}%
\pgfpathlineto{\pgfqpoint{3.317271in}{2.358035in}}%
\pgfpathlineto{\pgfqpoint{3.318962in}{2.227245in}}%
\pgfpathlineto{\pgfqpoint{3.321499in}{2.422992in}}%
\pgfpathlineto{\pgfqpoint{3.322345in}{2.031307in}}%
\pgfpathlineto{\pgfqpoint{3.323191in}{2.096527in}}%
\pgfpathlineto{\pgfqpoint{3.324036in}{2.096448in}}%
\pgfpathlineto{\pgfqpoint{3.324882in}{2.292776in}}%
\pgfpathlineto{\pgfqpoint{3.326573in}{1.965809in}}%
\pgfpathlineto{\pgfqpoint{3.328264in}{2.161983in}}%
\pgfpathlineto{\pgfqpoint{3.329110in}{1.835493in}}%
\pgfpathlineto{\pgfqpoint{3.329956in}{2.161923in}}%
\pgfpathlineto{\pgfqpoint{3.330801in}{1.965731in}}%
\pgfpathlineto{\pgfqpoint{3.331647in}{2.164587in}}%
\pgfpathlineto{\pgfqpoint{3.332492in}{2.160055in}}%
\pgfpathlineto{\pgfqpoint{3.333338in}{2.096615in}}%
\pgfpathlineto{\pgfqpoint{3.334184in}{2.358075in}}%
\pgfpathlineto{\pgfqpoint{3.335029in}{2.030753in}}%
\pgfpathlineto{\pgfqpoint{3.335875in}{2.161658in}}%
\pgfpathlineto{\pgfqpoint{3.336721in}{2.423355in}}%
\pgfpathlineto{\pgfqpoint{3.337566in}{1.704319in}}%
\pgfpathlineto{\pgfqpoint{3.338412in}{1.900358in}}%
\pgfpathlineto{\pgfqpoint{3.339257in}{2.488604in}}%
\pgfpathlineto{\pgfqpoint{3.340103in}{2.292631in}}%
\pgfpathlineto{\pgfqpoint{3.340949in}{1.769730in}}%
\pgfpathlineto{\pgfqpoint{3.341794in}{2.030829in}}%
\pgfpathlineto{\pgfqpoint{3.344331in}{2.554139in}}%
\pgfpathlineto{\pgfqpoint{3.345177in}{2.291741in}}%
\pgfpathlineto{\pgfqpoint{3.346022in}{2.292510in}}%
\pgfpathlineto{\pgfqpoint{3.346868in}{2.488432in}}%
\pgfpathlineto{\pgfqpoint{3.349405in}{1.969554in}}%
\pgfpathlineto{\pgfqpoint{3.350251in}{1.960980in}}%
\pgfpathlineto{\pgfqpoint{3.351096in}{2.231687in}}%
\pgfpathlineto{\pgfqpoint{3.351942in}{2.136449in}}%
\pgfpathlineto{\pgfqpoint{3.352787in}{2.043611in}}%
\pgfpathlineto{\pgfqpoint{3.353633in}{2.400985in}}%
\pgfpathlineto{\pgfqpoint{3.354479in}{2.206434in}}%
\pgfpathlineto{\pgfqpoint{3.355324in}{2.129229in}}%
\pgfpathlineto{\pgfqpoint{3.356170in}{2.194629in}}%
\pgfpathlineto{\pgfqpoint{3.357015in}{1.998426in}}%
\pgfpathlineto{\pgfqpoint{3.358707in}{2.390750in}}%
\pgfpathlineto{\pgfqpoint{3.359552in}{1.998562in}}%
\pgfpathlineto{\pgfqpoint{3.360398in}{2.063753in}}%
\pgfpathlineto{\pgfqpoint{3.361244in}{2.194683in}}%
\pgfpathlineto{\pgfqpoint{3.362089in}{2.129417in}}%
\pgfpathlineto{\pgfqpoint{3.362935in}{1.867642in}}%
\pgfpathlineto{\pgfqpoint{3.363780in}{2.652188in}}%
\pgfpathlineto{\pgfqpoint{3.364626in}{2.324390in}}%
\pgfpathlineto{\pgfqpoint{3.365472in}{2.586839in}}%
\pgfpathlineto{\pgfqpoint{3.366317in}{2.256098in}}%
\pgfpathlineto{\pgfqpoint{3.367163in}{2.455496in}}%
\pgfpathlineto{\pgfqpoint{3.368009in}{2.390311in}}%
\pgfpathlineto{\pgfqpoint{3.368854in}{1.347059in}}%
\pgfpathlineto{\pgfqpoint{3.369700in}{2.130968in}}%
\pgfpathlineto{\pgfqpoint{3.370545in}{1.996774in}}%
\pgfpathlineto{\pgfqpoint{3.371391in}{2.061582in}}%
\pgfpathlineto{\pgfqpoint{3.372237in}{2.068996in}}%
\pgfpathlineto{\pgfqpoint{3.373928in}{2.326301in}}%
\pgfpathlineto{\pgfqpoint{3.376465in}{1.695750in}}%
\pgfpathlineto{\pgfqpoint{3.377310in}{2.317632in}}%
\pgfpathlineto{\pgfqpoint{3.378156in}{2.312114in}}%
\pgfpathlineto{\pgfqpoint{3.379002in}{2.125443in}}%
\pgfpathlineto{\pgfqpoint{3.379847in}{2.324097in}}%
\pgfpathlineto{\pgfqpoint{3.380693in}{1.931695in}}%
\pgfpathlineto{\pgfqpoint{3.381539in}{2.585125in}}%
\pgfpathlineto{\pgfqpoint{3.382384in}{2.456436in}}%
\pgfpathlineto{\pgfqpoint{3.383230in}{2.324477in}}%
\pgfpathlineto{\pgfqpoint{3.384075in}{1.913358in}}%
\pgfpathlineto{\pgfqpoint{3.384921in}{2.061657in}}%
\pgfpathlineto{\pgfqpoint{3.385767in}{1.868696in}}%
\pgfpathlineto{\pgfqpoint{3.387458in}{2.323277in}}%
\pgfpathlineto{\pgfqpoint{3.388304in}{2.315712in}}%
\pgfpathlineto{\pgfqpoint{3.389149in}{2.325191in}}%
\pgfpathlineto{\pgfqpoint{3.391686in}{1.721926in}}%
\pgfpathlineto{\pgfqpoint{3.392532in}{1.914933in}}%
\pgfpathlineto{\pgfqpoint{3.393377in}{1.898520in}}%
\pgfpathlineto{\pgfqpoint{3.396760in}{2.407471in}}%
\pgfpathlineto{\pgfqpoint{3.397605in}{1.932820in}}%
\pgfpathlineto{\pgfqpoint{3.398451in}{2.257873in}}%
\pgfpathlineto{\pgfqpoint{3.399297in}{2.130402in}}%
\pgfpathlineto{\pgfqpoint{3.400142in}{2.182900in}}%
\pgfpathlineto{\pgfqpoint{3.400988in}{2.455639in}}%
\pgfpathlineto{\pgfqpoint{3.403525in}{2.048577in}}%
\pgfpathlineto{\pgfqpoint{3.404370in}{2.122534in}}%
\pgfpathlineto{\pgfqpoint{3.405216in}{2.117981in}}%
\pgfpathlineto{\pgfqpoint{3.406062in}{2.488034in}}%
\pgfpathlineto{\pgfqpoint{3.406907in}{1.714457in}}%
\pgfpathlineto{\pgfqpoint{3.407753in}{2.024230in}}%
\pgfpathlineto{\pgfqpoint{3.408599in}{1.704196in}}%
\pgfpathlineto{\pgfqpoint{3.409444in}{2.225639in}}%
\pgfpathlineto{\pgfqpoint{3.410290in}{1.899407in}}%
\pgfpathlineto{\pgfqpoint{3.411135in}{2.095030in}}%
\pgfpathlineto{\pgfqpoint{3.411981in}{1.897954in}}%
\pgfpathlineto{\pgfqpoint{3.412827in}{2.179534in}}%
\pgfpathlineto{\pgfqpoint{3.413672in}{2.019881in}}%
\pgfpathlineto{\pgfqpoint{3.415364in}{2.307224in}}%
\pgfpathlineto{\pgfqpoint{3.417055in}{1.930339in}}%
\pgfpathlineto{\pgfqpoint{3.417900in}{2.552343in}}%
\pgfpathlineto{\pgfqpoint{3.418746in}{2.176298in}}%
\pgfpathlineto{\pgfqpoint{3.420437in}{2.423148in}}%
\pgfpathlineto{\pgfqpoint{3.422129in}{2.094859in}}%
\pgfpathlineto{\pgfqpoint{3.422974in}{2.293918in}}%
\pgfpathlineto{\pgfqpoint{3.423820in}{2.141675in}}%
\pgfpathlineto{\pgfqpoint{3.424665in}{2.229767in}}%
\pgfpathlineto{\pgfqpoint{3.425511in}{1.857965in}}%
\pgfpathlineto{\pgfqpoint{3.426357in}{1.862389in}}%
\pgfpathlineto{\pgfqpoint{3.428048in}{2.619394in}}%
\pgfpathlineto{\pgfqpoint{3.428894in}{2.279134in}}%
\pgfpathlineto{\pgfqpoint{3.429739in}{1.884311in}}%
\pgfpathlineto{\pgfqpoint{3.430585in}{2.384931in}}%
\pgfpathlineto{\pgfqpoint{3.431430in}{2.104182in}}%
\pgfpathlineto{\pgfqpoint{3.432276in}{2.060858in}}%
\pgfpathlineto{\pgfqpoint{3.433967in}{2.520058in}}%
\pgfpathlineto{\pgfqpoint{3.436504in}{1.796745in}}%
\pgfpathlineto{\pgfqpoint{3.437350in}{2.401202in}}%
\pgfpathlineto{\pgfqpoint{3.438195in}{1.803328in}}%
\pgfpathlineto{\pgfqpoint{3.439041in}{2.375205in}}%
\pgfpathlineto{\pgfqpoint{3.439887in}{2.251164in}}%
\pgfpathlineto{\pgfqpoint{3.440732in}{2.297290in}}%
\pgfpathlineto{\pgfqpoint{3.441578in}{1.642147in}}%
\pgfpathlineto{\pgfqpoint{3.442423in}{1.963057in}}%
\pgfpathlineto{\pgfqpoint{3.444115in}{2.159773in}}%
\pgfpathlineto{\pgfqpoint{3.444960in}{1.961371in}}%
\pgfpathlineto{\pgfqpoint{3.445806in}{1.965074in}}%
\pgfpathlineto{\pgfqpoint{3.446652in}{2.160635in}}%
\pgfpathlineto{\pgfqpoint{3.447497in}{2.946778in}}%
\pgfpathlineto{\pgfqpoint{3.449188in}{1.872412in}}%
\pgfpathlineto{\pgfqpoint{3.451725in}{2.391628in}}%
\pgfpathlineto{\pgfqpoint{3.452571in}{2.176261in}}%
\pgfpathlineto{\pgfqpoint{3.453417in}{2.383793in}}%
\pgfpathlineto{\pgfqpoint{3.455108in}{2.041185in}}%
\pgfpathlineto{\pgfqpoint{3.456799in}{2.408092in}}%
\pgfpathlineto{\pgfqpoint{3.458490in}{2.071846in}}%
\pgfpathlineto{\pgfqpoint{3.459336in}{2.319389in}}%
\pgfpathlineto{\pgfqpoint{3.460182in}{2.062874in}}%
\pgfpathlineto{\pgfqpoint{3.461027in}{2.192027in}}%
\pgfpathlineto{\pgfqpoint{3.461873in}{2.330836in}}%
\pgfpathlineto{\pgfqpoint{3.462718in}{2.321089in}}%
\pgfpathlineto{\pgfqpoint{3.464410in}{2.127854in}}%
\pgfpathlineto{\pgfqpoint{3.465255in}{2.719029in}}%
\pgfpathlineto{\pgfqpoint{3.466101in}{2.137143in}}%
\pgfpathlineto{\pgfqpoint{3.466947in}{2.323223in}}%
\pgfpathlineto{\pgfqpoint{3.467792in}{2.195501in}}%
\pgfpathlineto{\pgfqpoint{3.468638in}{2.388499in}}%
\pgfpathlineto{\pgfqpoint{3.470329in}{2.002323in}}%
\pgfpathlineto{\pgfqpoint{3.471175in}{2.454327in}}%
\pgfpathlineto{\pgfqpoint{3.472020in}{2.390987in}}%
\pgfpathlineto{\pgfqpoint{3.472866in}{2.188129in}}%
\pgfpathlineto{\pgfqpoint{3.473712in}{2.322221in}}%
\pgfpathlineto{\pgfqpoint{3.474557in}{2.519286in}}%
\pgfpathlineto{\pgfqpoint{3.475403in}{2.385762in}}%
\pgfpathlineto{\pgfqpoint{3.476248in}{2.128890in}}%
\pgfpathlineto{\pgfqpoint{3.477940in}{2.457316in}}%
\pgfpathlineto{\pgfqpoint{3.478785in}{1.868219in}}%
\pgfpathlineto{\pgfqpoint{3.479631in}{2.304485in}}%
\pgfpathlineto{\pgfqpoint{3.480477in}{2.586553in}}%
\pgfpathlineto{\pgfqpoint{3.484705in}{1.737435in}}%
\pgfpathlineto{\pgfqpoint{3.488087in}{2.719423in}}%
\pgfpathlineto{\pgfqpoint{3.489778in}{2.125552in}}%
\pgfpathlineto{\pgfqpoint{3.490624in}{2.260031in}}%
\pgfpathlineto{\pgfqpoint{3.491470in}{2.063794in}}%
\pgfpathlineto{\pgfqpoint{3.492315in}{1.475594in}}%
\pgfpathlineto{\pgfqpoint{3.493161in}{2.652248in}}%
\pgfpathlineto{\pgfqpoint{3.494007in}{2.325362in}}%
\pgfpathlineto{\pgfqpoint{3.496543in}{1.867783in}}%
\pgfpathlineto{\pgfqpoint{3.497389in}{2.586296in}}%
\pgfpathlineto{\pgfqpoint{3.498235in}{2.194931in}}%
\pgfpathlineto{\pgfqpoint{3.499926in}{2.586869in}}%
\pgfpathlineto{\pgfqpoint{3.501617in}{1.802395in}}%
\pgfpathlineto{\pgfqpoint{3.502463in}{2.194530in}}%
\pgfpathlineto{\pgfqpoint{3.503308in}{1.932594in}}%
\pgfpathlineto{\pgfqpoint{3.504154in}{1.867805in}}%
\pgfpathlineto{\pgfqpoint{3.505000in}{2.325342in}}%
\pgfpathlineto{\pgfqpoint{3.505845in}{1.736697in}}%
\pgfpathlineto{\pgfqpoint{3.506691in}{2.521046in}}%
\pgfpathlineto{\pgfqpoint{3.507537in}{1.735395in}}%
\pgfpathlineto{\pgfqpoint{3.508382in}{2.000395in}}%
\pgfpathlineto{\pgfqpoint{3.509228in}{2.194719in}}%
\pgfpathlineto{\pgfqpoint{3.510073in}{2.129096in}}%
\pgfpathlineto{\pgfqpoint{3.510919in}{2.193662in}}%
\pgfpathlineto{\pgfqpoint{3.511765in}{1.736998in}}%
\pgfpathlineto{\pgfqpoint{3.512610in}{2.455708in}}%
\pgfpathlineto{\pgfqpoint{3.513456in}{1.998519in}}%
\pgfpathlineto{\pgfqpoint{3.515147in}{2.390684in}}%
\pgfpathlineto{\pgfqpoint{3.515993in}{1.867813in}}%
\pgfpathlineto{\pgfqpoint{3.516838in}{2.005059in}}%
\pgfpathlineto{\pgfqpoint{3.517684in}{2.256939in}}%
\pgfpathlineto{\pgfqpoint{3.518530in}{1.868240in}}%
\pgfpathlineto{\pgfqpoint{3.519375in}{2.063876in}}%
\pgfpathlineto{\pgfqpoint{3.521912in}{2.456278in}}%
\pgfpathlineto{\pgfqpoint{3.522758in}{2.324286in}}%
\pgfpathlineto{\pgfqpoint{3.523603in}{2.391670in}}%
\pgfpathlineto{\pgfqpoint{3.524449in}{2.131884in}}%
\pgfpathlineto{\pgfqpoint{3.525295in}{2.241323in}}%
\pgfpathlineto{\pgfqpoint{3.526140in}{2.385727in}}%
\pgfpathlineto{\pgfqpoint{3.526986in}{1.868286in}}%
\pgfpathlineto{\pgfqpoint{3.527831in}{2.284336in}}%
\pgfpathlineto{\pgfqpoint{3.528677in}{2.399200in}}%
\pgfpathlineto{\pgfqpoint{3.529523in}{2.391188in}}%
\pgfpathlineto{\pgfqpoint{3.530368in}{1.921632in}}%
\pgfpathlineto{\pgfqpoint{3.531214in}{2.324185in}}%
\pgfpathlineto{\pgfqpoint{3.532060in}{2.140134in}}%
\pgfpathlineto{\pgfqpoint{3.532905in}{1.648397in}}%
\pgfpathlineto{\pgfqpoint{3.533751in}{2.332716in}}%
\pgfpathlineto{\pgfqpoint{3.534596in}{2.207541in}}%
\pgfpathlineto{\pgfqpoint{3.535442in}{2.442346in}}%
\pgfpathlineto{\pgfqpoint{3.536288in}{2.237865in}}%
\pgfpathlineto{\pgfqpoint{3.537133in}{2.258638in}}%
\pgfpathlineto{\pgfqpoint{3.537979in}{2.335567in}}%
\pgfpathlineto{\pgfqpoint{3.538825in}{2.076355in}}%
\pgfpathlineto{\pgfqpoint{3.539670in}{2.195523in}}%
\pgfpathlineto{\pgfqpoint{3.540516in}{2.120450in}}%
\pgfpathlineto{\pgfqpoint{3.541361in}{2.195027in}}%
\pgfpathlineto{\pgfqpoint{3.543053in}{2.497020in}}%
\pgfpathlineto{\pgfqpoint{3.543898in}{1.917207in}}%
\pgfpathlineto{\pgfqpoint{3.544744in}{2.086368in}}%
\pgfpathlineto{\pgfqpoint{3.546435in}{1.857282in}}%
\pgfpathlineto{\pgfqpoint{3.548126in}{2.226061in}}%
\pgfpathlineto{\pgfqpoint{3.548972in}{2.213521in}}%
\pgfpathlineto{\pgfqpoint{3.549818in}{1.971070in}}%
\pgfpathlineto{\pgfqpoint{3.550663in}{2.217520in}}%
\pgfpathlineto{\pgfqpoint{3.551509in}{2.043426in}}%
\pgfpathlineto{\pgfqpoint{3.552355in}{2.238187in}}%
\pgfpathlineto{\pgfqpoint{3.553200in}{2.236915in}}%
\pgfpathlineto{\pgfqpoint{3.554046in}{1.731949in}}%
\pgfpathlineto{\pgfqpoint{3.555737in}{2.508476in}}%
\pgfpathlineto{\pgfqpoint{3.556583in}{1.963043in}}%
\pgfpathlineto{\pgfqpoint{3.557428in}{2.303337in}}%
\pgfpathlineto{\pgfqpoint{3.558274in}{2.619411in}}%
\pgfpathlineto{\pgfqpoint{3.559965in}{2.161604in}}%
\pgfpathlineto{\pgfqpoint{3.560811in}{2.165616in}}%
\pgfpathlineto{\pgfqpoint{3.561656in}{2.223297in}}%
\pgfpathlineto{\pgfqpoint{3.562502in}{1.974984in}}%
\pgfpathlineto{\pgfqpoint{3.563348in}{2.217318in}}%
\pgfpathlineto{\pgfqpoint{3.564193in}{2.088973in}}%
\pgfpathlineto{\pgfqpoint{3.565039in}{2.196674in}}%
\pgfpathlineto{\pgfqpoint{3.565885in}{2.001178in}}%
\pgfpathlineto{\pgfqpoint{3.566730in}{2.584114in}}%
\pgfpathlineto{\pgfqpoint{3.567576in}{1.802117in}}%
\pgfpathlineto{\pgfqpoint{3.568421in}{2.169802in}}%
\pgfpathlineto{\pgfqpoint{3.569267in}{2.204534in}}%
\pgfpathlineto{\pgfqpoint{3.570113in}{2.201689in}}%
\pgfpathlineto{\pgfqpoint{3.571804in}{1.865724in}}%
\pgfpathlineto{\pgfqpoint{3.572650in}{2.604364in}}%
\pgfpathlineto{\pgfqpoint{3.573495in}{2.187552in}}%
\pgfpathlineto{\pgfqpoint{3.574341in}{1.609358in}}%
\pgfpathlineto{\pgfqpoint{3.576032in}{2.583281in}}%
\pgfpathlineto{\pgfqpoint{3.576878in}{1.994058in}}%
\pgfpathlineto{\pgfqpoint{3.577723in}{2.100484in}}%
\pgfpathlineto{\pgfqpoint{3.578569in}{2.526096in}}%
\pgfpathlineto{\pgfqpoint{3.579415in}{2.125312in}}%
\pgfpathlineto{\pgfqpoint{3.580260in}{2.197729in}}%
\pgfpathlineto{\pgfqpoint{3.581106in}{2.227826in}}%
\pgfpathlineto{\pgfqpoint{3.581951in}{1.538341in}}%
\pgfpathlineto{\pgfqpoint{3.582797in}{1.973378in}}%
\pgfpathlineto{\pgfqpoint{3.583643in}{1.986776in}}%
\pgfpathlineto{\pgfqpoint{3.584488in}{2.383977in}}%
\pgfpathlineto{\pgfqpoint{3.585334in}{2.068937in}}%
\pgfpathlineto{\pgfqpoint{3.586180in}{2.036774in}}%
\pgfpathlineto{\pgfqpoint{3.588716in}{2.767410in}}%
\pgfpathlineto{\pgfqpoint{3.590408in}{2.066069in}}%
\pgfpathlineto{\pgfqpoint{3.591253in}{2.255365in}}%
\pgfpathlineto{\pgfqpoint{3.592099in}{2.071977in}}%
\pgfpathlineto{\pgfqpoint{3.592944in}{2.521964in}}%
\pgfpathlineto{\pgfqpoint{3.593790in}{2.366511in}}%
\pgfpathlineto{\pgfqpoint{3.595481in}{1.901077in}}%
\pgfpathlineto{\pgfqpoint{3.596327in}{2.615321in}}%
\pgfpathlineto{\pgfqpoint{3.597173in}{1.893628in}}%
\pgfpathlineto{\pgfqpoint{3.598018in}{2.170018in}}%
\pgfpathlineto{\pgfqpoint{3.598864in}{2.495076in}}%
\pgfpathlineto{\pgfqpoint{3.601401in}{1.855192in}}%
\pgfpathlineto{\pgfqpoint{3.602246in}{2.420335in}}%
\pgfpathlineto{\pgfqpoint{3.603092in}{1.972506in}}%
\pgfpathlineto{\pgfqpoint{3.603938in}{2.405936in}}%
\pgfpathlineto{\pgfqpoint{3.604783in}{2.312622in}}%
\pgfpathlineto{\pgfqpoint{3.605629in}{2.256244in}}%
\pgfpathlineto{\pgfqpoint{3.606474in}{2.262916in}}%
\pgfpathlineto{\pgfqpoint{3.608166in}{2.636519in}}%
\pgfpathlineto{\pgfqpoint{3.609011in}{1.806948in}}%
\pgfpathlineto{\pgfqpoint{3.609857in}{1.973930in}}%
\pgfpathlineto{\pgfqpoint{3.611548in}{2.778531in}}%
\pgfpathlineto{\pgfqpoint{3.612394in}{2.235441in}}%
\pgfpathlineto{\pgfqpoint{3.613239in}{2.574419in}}%
\pgfpathlineto{\pgfqpoint{3.614085in}{2.486770in}}%
\pgfpathlineto{\pgfqpoint{3.614931in}{1.953524in}}%
\pgfpathlineto{\pgfqpoint{3.615776in}{2.244249in}}%
\pgfpathlineto{\pgfqpoint{3.616622in}{2.080676in}}%
\pgfpathlineto{\pgfqpoint{3.617468in}{2.466117in}}%
\pgfpathlineto{\pgfqpoint{3.618313in}{2.163294in}}%
\pgfpathlineto{\pgfqpoint{3.619159in}{1.986448in}}%
\pgfpathlineto{\pgfqpoint{3.621696in}{2.258695in}}%
\pgfpathlineto{\pgfqpoint{3.623387in}{1.926458in}}%
\pgfpathlineto{\pgfqpoint{3.624233in}{2.205447in}}%
\pgfpathlineto{\pgfqpoint{3.625078in}{1.867545in}}%
\pgfpathlineto{\pgfqpoint{3.625924in}{2.484281in}}%
\pgfpathlineto{\pgfqpoint{3.626769in}{2.058813in}}%
\pgfpathlineto{\pgfqpoint{3.629306in}{2.572126in}}%
\pgfpathlineto{\pgfqpoint{3.631843in}{2.111570in}}%
\pgfpathlineto{\pgfqpoint{3.632689in}{2.421190in}}%
\pgfpathlineto{\pgfqpoint{3.633534in}{2.206555in}}%
\pgfpathlineto{\pgfqpoint{3.634380in}{1.981959in}}%
\pgfpathlineto{\pgfqpoint{3.635226in}{2.079444in}}%
\pgfpathlineto{\pgfqpoint{3.636071in}{2.410577in}}%
\pgfpathlineto{\pgfqpoint{3.636917in}{1.788407in}}%
\pgfpathlineto{\pgfqpoint{3.637763in}{2.082727in}}%
\pgfpathlineto{\pgfqpoint{3.638608in}{2.101631in}}%
\pgfpathlineto{\pgfqpoint{3.639454in}{2.384146in}}%
\pgfpathlineto{\pgfqpoint{3.640299in}{1.920793in}}%
\pgfpathlineto{\pgfqpoint{3.641145in}{2.101557in}}%
\pgfpathlineto{\pgfqpoint{3.641991in}{2.488062in}}%
\pgfpathlineto{\pgfqpoint{3.642836in}{2.227654in}}%
\pgfpathlineto{\pgfqpoint{3.643682in}{1.835867in}}%
\pgfpathlineto{\pgfqpoint{3.644528in}{1.998919in}}%
\pgfpathlineto{\pgfqpoint{3.647064in}{2.437612in}}%
\pgfpathlineto{\pgfqpoint{3.648756in}{2.175917in}}%
\pgfpathlineto{\pgfqpoint{3.649601in}{2.287074in}}%
\pgfpathlineto{\pgfqpoint{3.650447in}{2.162023in}}%
\pgfpathlineto{\pgfqpoint{3.651293in}{2.549754in}}%
\pgfpathlineto{\pgfqpoint{3.652984in}{2.065692in}}%
\pgfpathlineto{\pgfqpoint{3.653829in}{2.292985in}}%
\pgfpathlineto{\pgfqpoint{3.654675in}{2.161956in}}%
\pgfpathlineto{\pgfqpoint{3.656366in}{2.161308in}}%
\pgfpathlineto{\pgfqpoint{3.657212in}{2.453780in}}%
\pgfpathlineto{\pgfqpoint{3.658058in}{2.346368in}}%
\pgfpathlineto{\pgfqpoint{3.658903in}{2.193742in}}%
\pgfpathlineto{\pgfqpoint{3.659749in}{2.583930in}}%
\pgfpathlineto{\pgfqpoint{3.662286in}{1.785342in}}%
\pgfpathlineto{\pgfqpoint{3.663131in}{2.434712in}}%
\pgfpathlineto{\pgfqpoint{3.663977in}{2.161830in}}%
\pgfpathlineto{\pgfqpoint{3.664823in}{1.996590in}}%
\pgfpathlineto{\pgfqpoint{3.667359in}{2.301404in}}%
\pgfpathlineto{\pgfqpoint{3.668205in}{2.196206in}}%
\pgfpathlineto{\pgfqpoint{3.669051in}{1.810516in}}%
\pgfpathlineto{\pgfqpoint{3.669896in}{2.542565in}}%
\pgfpathlineto{\pgfqpoint{3.670742in}{2.196401in}}%
\pgfpathlineto{\pgfqpoint{3.671587in}{2.238896in}}%
\pgfpathlineto{\pgfqpoint{3.672433in}{2.430704in}}%
\pgfpathlineto{\pgfqpoint{3.673279in}{1.751123in}}%
\pgfpathlineto{\pgfqpoint{3.674124in}{2.330195in}}%
\pgfpathlineto{\pgfqpoint{3.674970in}{2.242631in}}%
\pgfpathlineto{\pgfqpoint{3.675816in}{2.292800in}}%
\pgfpathlineto{\pgfqpoint{3.676661in}{2.158483in}}%
\pgfpathlineto{\pgfqpoint{3.677507in}{2.481755in}}%
\pgfpathlineto{\pgfqpoint{3.678352in}{1.806140in}}%
\pgfpathlineto{\pgfqpoint{3.679198in}{2.161485in}}%
\pgfpathlineto{\pgfqpoint{3.680044in}{2.070330in}}%
\pgfpathlineto{\pgfqpoint{3.680889in}{2.079020in}}%
\pgfpathlineto{\pgfqpoint{3.683426in}{2.498899in}}%
\pgfpathlineto{\pgfqpoint{3.684272in}{1.898746in}}%
\pgfpathlineto{\pgfqpoint{3.685117in}{2.000853in}}%
\pgfpathlineto{\pgfqpoint{3.685963in}{1.991771in}}%
\pgfpathlineto{\pgfqpoint{3.688500in}{2.523312in}}%
\pgfpathlineto{\pgfqpoint{3.690191in}{2.130850in}}%
\pgfpathlineto{\pgfqpoint{3.691037in}{2.194677in}}%
\pgfpathlineto{\pgfqpoint{3.691882in}{2.386637in}}%
\pgfpathlineto{\pgfqpoint{3.694419in}{1.772586in}}%
\pgfpathlineto{\pgfqpoint{3.695265in}{2.010405in}}%
\pgfpathlineto{\pgfqpoint{3.696111in}{2.595256in}}%
\pgfpathlineto{\pgfqpoint{3.696956in}{2.431187in}}%
\pgfpathlineto{\pgfqpoint{3.697802in}{2.209285in}}%
\pgfpathlineto{\pgfqpoint{3.698647in}{2.396404in}}%
\pgfpathlineto{\pgfqpoint{3.700339in}{1.847791in}}%
\pgfpathlineto{\pgfqpoint{3.701184in}{2.522275in}}%
\pgfpathlineto{\pgfqpoint{3.702030in}{2.259745in}}%
\pgfpathlineto{\pgfqpoint{3.702876in}{2.327459in}}%
\pgfpathlineto{\pgfqpoint{3.703721in}{2.324476in}}%
\pgfpathlineto{\pgfqpoint{3.705412in}{1.737075in}}%
\pgfpathlineto{\pgfqpoint{3.706258in}{1.867971in}}%
\pgfpathlineto{\pgfqpoint{3.707104in}{2.324883in}}%
\pgfpathlineto{\pgfqpoint{3.707949in}{2.064130in}}%
\pgfpathlineto{\pgfqpoint{3.708795in}{1.802437in}}%
\pgfpathlineto{\pgfqpoint{3.709641in}{2.463394in}}%
\pgfpathlineto{\pgfqpoint{3.710486in}{2.010894in}}%
\pgfpathlineto{\pgfqpoint{3.711332in}{2.129162in}}%
\pgfpathlineto{\pgfqpoint{3.712177in}{2.063817in}}%
\pgfpathlineto{\pgfqpoint{3.713869in}{2.194589in}}%
\pgfpathlineto{\pgfqpoint{3.714714in}{1.933107in}}%
\pgfpathlineto{\pgfqpoint{3.715560in}{2.063856in}}%
\pgfpathlineto{\pgfqpoint{3.716406in}{2.325350in}}%
\pgfpathlineto{\pgfqpoint{3.717251in}{2.063702in}}%
\pgfpathlineto{\pgfqpoint{3.718097in}{2.644383in}}%
\pgfpathlineto{\pgfqpoint{3.718942in}{1.998356in}}%
\pgfpathlineto{\pgfqpoint{3.719788in}{2.455881in}}%
\pgfpathlineto{\pgfqpoint{3.720634in}{1.933136in}}%
\pgfpathlineto{\pgfqpoint{3.721479in}{2.081041in}}%
\pgfpathlineto{\pgfqpoint{3.723171in}{1.997472in}}%
\pgfpathlineto{\pgfqpoint{3.725707in}{2.510975in}}%
\pgfpathlineto{\pgfqpoint{3.728244in}{1.668486in}}%
\pgfpathlineto{\pgfqpoint{3.729090in}{2.522208in}}%
\pgfpathlineto{\pgfqpoint{3.729936in}{2.198267in}}%
\pgfpathlineto{\pgfqpoint{3.731627in}{2.391885in}}%
\pgfpathlineto{\pgfqpoint{3.732472in}{2.259851in}}%
\pgfpathlineto{\pgfqpoint{3.733318in}{1.867806in}}%
\pgfpathlineto{\pgfqpoint{3.734164in}{2.390603in}}%
\pgfpathlineto{\pgfqpoint{3.735009in}{2.389544in}}%
\pgfpathlineto{\pgfqpoint{3.735855in}{1.997824in}}%
\pgfpathlineto{\pgfqpoint{3.736701in}{2.389665in}}%
\pgfpathlineto{\pgfqpoint{3.737546in}{1.998595in}}%
\pgfpathlineto{\pgfqpoint{3.738392in}{2.913580in}}%
\pgfpathlineto{\pgfqpoint{3.739237in}{2.001329in}}%
\pgfpathlineto{\pgfqpoint{3.740083in}{2.259791in}}%
\pgfpathlineto{\pgfqpoint{3.740929in}{2.063516in}}%
\pgfpathlineto{\pgfqpoint{3.741774in}{2.129613in}}%
\pgfpathlineto{\pgfqpoint{3.742620in}{2.489732in}}%
\pgfpathlineto{\pgfqpoint{3.743466in}{2.390660in}}%
\pgfpathlineto{\pgfqpoint{3.745157in}{1.932204in}}%
\pgfpathlineto{\pgfqpoint{3.746002in}{1.948200in}}%
\pgfpathlineto{\pgfqpoint{3.748539in}{2.384454in}}%
\pgfpathlineto{\pgfqpoint{3.749385in}{1.866448in}}%
\pgfpathlineto{\pgfqpoint{3.750231in}{2.127184in}}%
\pgfpathlineto{\pgfqpoint{3.751076in}{2.128870in}}%
\pgfpathlineto{\pgfqpoint{3.751922in}{2.120487in}}%
\pgfpathlineto{\pgfqpoint{3.753613in}{2.586730in}}%
\pgfpathlineto{\pgfqpoint{3.756150in}{2.161911in}}%
\pgfpathlineto{\pgfqpoint{3.756995in}{2.096645in}}%
\pgfpathlineto{\pgfqpoint{3.757841in}{2.306297in}}%
\pgfpathlineto{\pgfqpoint{3.758687in}{2.162978in}}%
\pgfpathlineto{\pgfqpoint{3.759532in}{1.899244in}}%
\pgfpathlineto{\pgfqpoint{3.760378in}{1.998471in}}%
\pgfpathlineto{\pgfqpoint{3.762915in}{2.716758in}}%
\pgfpathlineto{\pgfqpoint{3.763760in}{2.062663in}}%
\pgfpathlineto{\pgfqpoint{3.764606in}{2.193445in}}%
\pgfpathlineto{\pgfqpoint{3.765452in}{2.194167in}}%
\pgfpathlineto{\pgfqpoint{3.766297in}{2.390626in}}%
\pgfpathlineto{\pgfqpoint{3.767143in}{2.100162in}}%
\pgfpathlineto{\pgfqpoint{3.767989in}{2.652032in}}%
\pgfpathlineto{\pgfqpoint{3.768834in}{2.192910in}}%
\pgfpathlineto{\pgfqpoint{3.769680in}{2.194577in}}%
\pgfpathlineto{\pgfqpoint{3.770525in}{2.259825in}}%
\pgfpathlineto{\pgfqpoint{3.771371in}{2.031515in}}%
\pgfpathlineto{\pgfqpoint{3.772217in}{2.390721in}}%
\pgfpathlineto{\pgfqpoint{3.773908in}{1.998112in}}%
\pgfpathlineto{\pgfqpoint{3.774754in}{2.586891in}}%
\pgfpathlineto{\pgfqpoint{3.775599in}{2.586710in}}%
\pgfpathlineto{\pgfqpoint{3.776445in}{2.519381in}}%
\pgfpathlineto{\pgfqpoint{3.778982in}{1.785566in}}%
\pgfpathlineto{\pgfqpoint{3.779827in}{2.135558in}}%
\pgfpathlineto{\pgfqpoint{3.780673in}{2.129303in}}%
\pgfpathlineto{\pgfqpoint{3.781519in}{2.178617in}}%
\pgfpathlineto{\pgfqpoint{3.782364in}{1.959563in}}%
\pgfpathlineto{\pgfqpoint{3.783210in}{2.435440in}}%
\pgfpathlineto{\pgfqpoint{3.784055in}{2.299625in}}%
\pgfpathlineto{\pgfqpoint{3.784901in}{2.096547in}}%
\pgfpathlineto{\pgfqpoint{3.785747in}{2.226638in}}%
\pgfpathlineto{\pgfqpoint{3.786592in}{2.423363in}}%
\pgfpathlineto{\pgfqpoint{3.788284in}{2.031175in}}%
\pgfpathlineto{\pgfqpoint{3.789129in}{2.096580in}}%
\pgfpathlineto{\pgfqpoint{3.789975in}{2.227247in}}%
\pgfpathlineto{\pgfqpoint{3.790820in}{2.618816in}}%
\pgfpathlineto{\pgfqpoint{3.791666in}{2.096437in}}%
\pgfpathlineto{\pgfqpoint{3.792512in}{2.161914in}}%
\pgfpathlineto{\pgfqpoint{3.793357in}{2.090401in}}%
\pgfpathlineto{\pgfqpoint{3.794203in}{2.750311in}}%
\pgfpathlineto{\pgfqpoint{3.795049in}{2.096744in}}%
\pgfpathlineto{\pgfqpoint{3.795894in}{2.294484in}}%
\pgfpathlineto{\pgfqpoint{3.797585in}{2.188305in}}%
\pgfpathlineto{\pgfqpoint{3.798431in}{2.292047in}}%
\pgfpathlineto{\pgfqpoint{3.800968in}{2.083728in}}%
\pgfpathlineto{\pgfqpoint{3.802659in}{2.260021in}}%
\pgfpathlineto{\pgfqpoint{3.803505in}{1.998403in}}%
\pgfpathlineto{\pgfqpoint{3.804350in}{2.518459in}}%
\pgfpathlineto{\pgfqpoint{3.805196in}{2.194785in}}%
\pgfpathlineto{\pgfqpoint{3.806042in}{2.193976in}}%
\pgfpathlineto{\pgfqpoint{3.806887in}{2.127820in}}%
\pgfpathlineto{\pgfqpoint{3.807733in}{2.132450in}}%
\pgfpathlineto{\pgfqpoint{3.809424in}{2.519849in}}%
\pgfpathlineto{\pgfqpoint{3.810270in}{2.031190in}}%
\pgfpathlineto{\pgfqpoint{3.811961in}{2.647844in}}%
\pgfpathlineto{\pgfqpoint{3.814498in}{1.757369in}}%
\pgfpathlineto{\pgfqpoint{3.817880in}{2.586126in}}%
\pgfpathlineto{\pgfqpoint{3.818726in}{2.082109in}}%
\pgfpathlineto{\pgfqpoint{3.819572in}{2.554255in}}%
\pgfpathlineto{\pgfqpoint{3.820417in}{2.378265in}}%
\pgfpathlineto{\pgfqpoint{3.822109in}{1.668416in}}%
\pgfpathlineto{\pgfqpoint{3.822954in}{2.390780in}}%
\pgfpathlineto{\pgfqpoint{3.823800in}{2.285000in}}%
\pgfpathlineto{\pgfqpoint{3.826337in}{2.128035in}}%
\pgfpathlineto{\pgfqpoint{3.828874in}{2.354629in}}%
\pgfpathlineto{\pgfqpoint{3.831410in}{1.767193in}}%
\pgfpathlineto{\pgfqpoint{3.832256in}{1.767576in}}%
\pgfpathlineto{\pgfqpoint{3.836484in}{2.455807in}}%
\pgfpathlineto{\pgfqpoint{3.838175in}{1.934253in}}%
\pgfpathlineto{\pgfqpoint{3.839021in}{1.998492in}}%
\pgfpathlineto{\pgfqpoint{3.839867in}{1.868053in}}%
\pgfpathlineto{\pgfqpoint{3.843249in}{2.652078in}}%
\pgfpathlineto{\pgfqpoint{3.844940in}{2.194602in}}%
\pgfpathlineto{\pgfqpoint{3.846632in}{2.586163in}}%
\pgfpathlineto{\pgfqpoint{3.849168in}{1.867752in}}%
\pgfpathlineto{\pgfqpoint{3.850014in}{1.998410in}}%
\pgfpathlineto{\pgfqpoint{3.850860in}{2.194659in}}%
\pgfpathlineto{\pgfqpoint{3.851705in}{2.194564in}}%
\pgfpathlineto{\pgfqpoint{3.852551in}{1.998505in}}%
\pgfpathlineto{\pgfqpoint{3.853397in}{2.456016in}}%
\pgfpathlineto{\pgfqpoint{3.854242in}{2.194609in}}%
\pgfpathlineto{\pgfqpoint{3.855088in}{2.325345in}}%
\pgfpathlineto{\pgfqpoint{3.855933in}{2.717331in}}%
\pgfpathlineto{\pgfqpoint{3.856779in}{2.456060in}}%
\pgfpathlineto{\pgfqpoint{3.857625in}{2.390684in}}%
\pgfpathlineto{\pgfqpoint{3.858470in}{2.063883in}}%
\pgfpathlineto{\pgfqpoint{3.859316in}{2.259938in}}%
\pgfpathlineto{\pgfqpoint{3.860162in}{2.390729in}}%
\pgfpathlineto{\pgfqpoint{3.861007in}{1.802489in}}%
\pgfpathlineto{\pgfqpoint{3.862698in}{2.717419in}}%
\pgfpathlineto{\pgfqpoint{3.864390in}{2.063824in}}%
\pgfpathlineto{\pgfqpoint{3.865235in}{2.578622in}}%
\pgfpathlineto{\pgfqpoint{3.866081in}{2.455482in}}%
\pgfpathlineto{\pgfqpoint{3.867772in}{1.994127in}}%
\pgfpathlineto{\pgfqpoint{3.868618in}{2.096541in}}%
\pgfpathlineto{\pgfqpoint{3.870309in}{2.358017in}}%
\pgfpathlineto{\pgfqpoint{3.871155in}{2.096584in}}%
\pgfpathlineto{\pgfqpoint{3.872000in}{2.292053in}}%
\pgfpathlineto{\pgfqpoint{3.873692in}{2.487796in}}%
\pgfpathlineto{\pgfqpoint{3.875383in}{2.031163in}}%
\pgfpathlineto{\pgfqpoint{3.876228in}{2.091065in}}%
\pgfpathlineto{\pgfqpoint{3.878765in}{2.357989in}}%
\pgfpathlineto{\pgfqpoint{3.879611in}{2.162132in}}%
\pgfpathlineto{\pgfqpoint{3.880457in}{2.358010in}}%
\pgfpathlineto{\pgfqpoint{3.881302in}{2.191993in}}%
\pgfpathlineto{\pgfqpoint{3.882148in}{2.302555in}}%
\pgfpathlineto{\pgfqpoint{3.882993in}{2.378821in}}%
\pgfpathlineto{\pgfqpoint{3.883839in}{2.030759in}}%
\pgfpathlineto{\pgfqpoint{3.884685in}{2.423846in}}%
\pgfpathlineto{\pgfqpoint{3.885530in}{1.638736in}}%
\pgfpathlineto{\pgfqpoint{3.886376in}{2.227462in}}%
\pgfpathlineto{\pgfqpoint{3.887222in}{2.292482in}}%
\pgfpathlineto{\pgfqpoint{3.888067in}{2.096638in}}%
\pgfpathlineto{\pgfqpoint{3.888913in}{2.423204in}}%
\pgfpathlineto{\pgfqpoint{3.889758in}{2.358050in}}%
\pgfpathlineto{\pgfqpoint{3.891450in}{1.768067in}}%
\pgfpathlineto{\pgfqpoint{3.893141in}{2.619533in}}%
\pgfpathlineto{\pgfqpoint{3.893987in}{1.841768in}}%
\pgfpathlineto{\pgfqpoint{3.894832in}{2.226944in}}%
\pgfpathlineto{\pgfqpoint{3.895678in}{2.227148in}}%
\pgfpathlineto{\pgfqpoint{3.896523in}{2.077953in}}%
\pgfpathlineto{\pgfqpoint{3.897369in}{2.161917in}}%
\pgfpathlineto{\pgfqpoint{3.899060in}{2.358037in}}%
\pgfpathlineto{\pgfqpoint{3.902443in}{1.900433in}}%
\pgfpathlineto{\pgfqpoint{3.903288in}{1.965774in}}%
\pgfpathlineto{\pgfqpoint{3.904134in}{2.554142in}}%
\pgfpathlineto{\pgfqpoint{3.906671in}{1.573583in}}%
\pgfpathlineto{\pgfqpoint{3.907517in}{1.504570in}}%
\pgfpathlineto{\pgfqpoint{3.908362in}{2.423389in}}%
\pgfpathlineto{\pgfqpoint{3.909208in}{1.965764in}}%
\pgfpathlineto{\pgfqpoint{3.910053in}{2.031173in}}%
\pgfpathlineto{\pgfqpoint{3.910899in}{2.423561in}}%
\pgfpathlineto{\pgfqpoint{3.911745in}{2.292570in}}%
\pgfpathlineto{\pgfqpoint{3.912590in}{2.097754in}}%
\pgfpathlineto{\pgfqpoint{3.913436in}{2.488811in}}%
\pgfpathlineto{\pgfqpoint{3.914281in}{1.835215in}}%
\pgfpathlineto{\pgfqpoint{3.915127in}{2.684877in}}%
\pgfpathlineto{\pgfqpoint{3.915973in}{2.283564in}}%
\pgfpathlineto{\pgfqpoint{3.917664in}{2.423065in}}%
\pgfpathlineto{\pgfqpoint{3.918510in}{2.096520in}}%
\pgfpathlineto{\pgfqpoint{3.919355in}{2.096544in}}%
\pgfpathlineto{\pgfqpoint{3.920201in}{2.488766in}}%
\pgfpathlineto{\pgfqpoint{3.921046in}{2.161925in}}%
\pgfpathlineto{\pgfqpoint{3.921892in}{2.423399in}}%
\pgfpathlineto{\pgfqpoint{3.922738in}{2.358020in}}%
\pgfpathlineto{\pgfqpoint{3.923583in}{1.848226in}}%
\pgfpathlineto{\pgfqpoint{3.924429in}{2.227312in}}%
\pgfpathlineto{\pgfqpoint{3.925275in}{2.227287in}}%
\pgfpathlineto{\pgfqpoint{3.926966in}{2.488760in}}%
\pgfpathlineto{\pgfqpoint{3.929503in}{2.096547in}}%
\pgfpathlineto{\pgfqpoint{3.930348in}{2.551358in}}%
\pgfpathlineto{\pgfqpoint{3.931194in}{2.161928in}}%
\pgfpathlineto{\pgfqpoint{3.932040in}{2.477715in}}%
\pgfpathlineto{\pgfqpoint{3.933731in}{2.161854in}}%
\pgfpathlineto{\pgfqpoint{3.935422in}{2.431185in}}%
\pgfpathlineto{\pgfqpoint{3.937113in}{2.217823in}}%
\pgfpathlineto{\pgfqpoint{3.937959in}{2.292019in}}%
\pgfpathlineto{\pgfqpoint{3.938805in}{2.227280in}}%
\pgfpathlineto{\pgfqpoint{3.939650in}{2.019782in}}%
\pgfpathlineto{\pgfqpoint{3.940496in}{2.558688in}}%
\pgfpathlineto{\pgfqpoint{3.941341in}{1.867709in}}%
\pgfpathlineto{\pgfqpoint{3.942187in}{2.343481in}}%
\pgfpathlineto{\pgfqpoint{3.943878in}{1.826595in}}%
\pgfpathlineto{\pgfqpoint{3.944724in}{2.599030in}}%
\pgfpathlineto{\pgfqpoint{3.945570in}{2.227106in}}%
\pgfpathlineto{\pgfqpoint{3.946415in}{1.964848in}}%
\pgfpathlineto{\pgfqpoint{3.947261in}{2.016970in}}%
\pgfpathlineto{\pgfqpoint{3.948952in}{2.291277in}}%
\pgfpathlineto{\pgfqpoint{3.949798in}{2.063850in}}%
\pgfpathlineto{\pgfqpoint{3.950643in}{2.325358in}}%
\pgfpathlineto{\pgfqpoint{3.951489in}{1.933046in}}%
\pgfpathlineto{\pgfqpoint{3.952335in}{2.521345in}}%
\pgfpathlineto{\pgfqpoint{3.953180in}{2.063998in}}%
\pgfpathlineto{\pgfqpoint{3.954026in}{2.063861in}}%
\pgfpathlineto{\pgfqpoint{3.955717in}{2.586823in}}%
\pgfpathlineto{\pgfqpoint{3.956563in}{1.868044in}}%
\pgfpathlineto{\pgfqpoint{3.957408in}{1.998378in}}%
\pgfpathlineto{\pgfqpoint{3.958254in}{2.456022in}}%
\pgfpathlineto{\pgfqpoint{3.959100in}{2.454900in}}%
\pgfpathlineto{\pgfqpoint{3.959945in}{2.063807in}}%
\pgfpathlineto{\pgfqpoint{3.960791in}{2.782927in}}%
\pgfpathlineto{\pgfqpoint{3.961636in}{2.325691in}}%
\pgfpathlineto{\pgfqpoint{3.962482in}{2.652035in}}%
\pgfpathlineto{\pgfqpoint{3.964173in}{1.867320in}}%
\pgfpathlineto{\pgfqpoint{3.965019in}{2.063891in}}%
\pgfpathlineto{\pgfqpoint{3.965865in}{1.997983in}}%
\pgfpathlineto{\pgfqpoint{3.966710in}{2.064239in}}%
\pgfpathlineto{\pgfqpoint{3.969247in}{2.978983in}}%
\pgfpathlineto{\pgfqpoint{3.970938in}{1.802473in}}%
\pgfpathlineto{\pgfqpoint{3.973475in}{2.325180in}}%
\pgfpathlineto{\pgfqpoint{3.974321in}{2.325352in}}%
\pgfpathlineto{\pgfqpoint{3.975166in}{2.392719in}}%
\pgfpathlineto{\pgfqpoint{3.976012in}{2.782401in}}%
\pgfpathlineto{\pgfqpoint{3.976858in}{2.063856in}}%
\pgfpathlineto{\pgfqpoint{3.977703in}{2.260004in}}%
\pgfpathlineto{\pgfqpoint{3.978549in}{2.194634in}}%
\pgfpathlineto{\pgfqpoint{3.980240in}{1.933012in}}%
\pgfpathlineto{\pgfqpoint{3.982777in}{2.588607in}}%
\pgfpathlineto{\pgfqpoint{3.985314in}{1.933034in}}%
\pgfpathlineto{\pgfqpoint{3.986160in}{2.325340in}}%
\pgfpathlineto{\pgfqpoint{3.987005in}{2.259938in}}%
\pgfpathlineto{\pgfqpoint{3.987851in}{1.802342in}}%
\pgfpathlineto{\pgfqpoint{3.988696in}{1.867892in}}%
\pgfpathlineto{\pgfqpoint{3.991233in}{2.325366in}}%
\pgfpathlineto{\pgfqpoint{3.992079in}{1.607634in}}%
\pgfpathlineto{\pgfqpoint{3.992924in}{2.259953in}}%
\pgfpathlineto{\pgfqpoint{3.993770in}{2.194497in}}%
\pgfpathlineto{\pgfqpoint{3.994616in}{2.260069in}}%
\pgfpathlineto{\pgfqpoint{3.995461in}{2.586832in}}%
\pgfpathlineto{\pgfqpoint{3.997998in}{2.063859in}}%
\pgfpathlineto{\pgfqpoint{3.998844in}{1.737046in}}%
\pgfpathlineto{\pgfqpoint{3.999689in}{1.932707in}}%
\pgfpathlineto{\pgfqpoint{4.000535in}{2.129189in}}%
\pgfpathlineto{\pgfqpoint{4.001381in}{1.802533in}}%
\pgfpathlineto{\pgfqpoint{4.002226in}{2.652191in}}%
\pgfpathlineto{\pgfqpoint{4.003072in}{1.877356in}}%
\pgfpathlineto{\pgfqpoint{4.003918in}{2.325039in}}%
\pgfpathlineto{\pgfqpoint{4.004763in}{2.259992in}}%
\pgfpathlineto{\pgfqpoint{4.005609in}{2.325151in}}%
\pgfpathlineto{\pgfqpoint{4.006454in}{1.999257in}}%
\pgfpathlineto{\pgfqpoint{4.007300in}{2.456094in}}%
\pgfpathlineto{\pgfqpoint{4.008146in}{1.867747in}}%
\pgfpathlineto{\pgfqpoint{4.008991in}{2.194559in}}%
\pgfpathlineto{\pgfqpoint{4.011528in}{2.456075in}}%
\pgfpathlineto{\pgfqpoint{4.012374in}{1.998482in}}%
\pgfpathlineto{\pgfqpoint{4.013219in}{2.194584in}}%
\pgfpathlineto{\pgfqpoint{4.014065in}{2.369999in}}%
\pgfpathlineto{\pgfqpoint{4.014911in}{2.063842in}}%
\pgfpathlineto{\pgfqpoint{4.015756in}{2.292231in}}%
\pgfpathlineto{\pgfqpoint{4.017448in}{2.032316in}}%
\pgfpathlineto{\pgfqpoint{4.018293in}{2.423392in}}%
\pgfpathlineto{\pgfqpoint{4.019139in}{1.965973in}}%
\pgfpathlineto{\pgfqpoint{4.019984in}{2.357882in}}%
\pgfpathlineto{\pgfqpoint{4.020830in}{2.227202in}}%
\pgfpathlineto{\pgfqpoint{4.021676in}{2.031358in}}%
\pgfpathlineto{\pgfqpoint{4.023367in}{2.619526in}}%
\pgfpathlineto{\pgfqpoint{4.025058in}{1.769521in}}%
\pgfpathlineto{\pgfqpoint{4.025904in}{2.227289in}}%
\pgfpathlineto{\pgfqpoint{4.026749in}{1.900834in}}%
\pgfpathlineto{\pgfqpoint{4.030132in}{2.358067in}}%
\pgfpathlineto{\pgfqpoint{4.030978in}{2.292703in}}%
\pgfpathlineto{\pgfqpoint{4.031823in}{2.422254in}}%
\pgfpathlineto{\pgfqpoint{4.034360in}{2.031111in}}%
\pgfpathlineto{\pgfqpoint{4.036897in}{2.358360in}}%
\pgfpathlineto{\pgfqpoint{4.038588in}{1.900371in}}%
\pgfpathlineto{\pgfqpoint{4.041125in}{2.565650in}}%
\pgfpathlineto{\pgfqpoint{4.041971in}{1.769381in}}%
\pgfpathlineto{\pgfqpoint{4.042816in}{2.293736in}}%
\pgfpathlineto{\pgfqpoint{4.043662in}{2.554236in}}%
\pgfpathlineto{\pgfqpoint{4.044508in}{2.358097in}}%
\pgfpathlineto{\pgfqpoint{4.045353in}{2.290788in}}%
\pgfpathlineto{\pgfqpoint{4.046199in}{1.966874in}}%
\pgfpathlineto{\pgfqpoint{4.047044in}{2.292840in}}%
\pgfpathlineto{\pgfqpoint{4.047890in}{1.835386in}}%
\pgfpathlineto{\pgfqpoint{4.048736in}{2.163904in}}%
\pgfpathlineto{\pgfqpoint{4.049581in}{2.032628in}}%
\pgfpathlineto{\pgfqpoint{4.050427in}{2.292409in}}%
\pgfpathlineto{\pgfqpoint{4.051273in}{2.235751in}}%
\pgfpathlineto{\pgfqpoint{4.052118in}{1.965777in}}%
\pgfpathlineto{\pgfqpoint{4.052964in}{2.554112in}}%
\pgfpathlineto{\pgfqpoint{4.053809in}{2.358270in}}%
\pgfpathlineto{\pgfqpoint{4.055501in}{1.966109in}}%
\pgfpathlineto{\pgfqpoint{4.056346in}{2.096600in}}%
\pgfpathlineto{\pgfqpoint{4.058038in}{2.227295in}}%
\pgfpathlineto{\pgfqpoint{4.059729in}{2.619160in}}%
\pgfpathlineto{\pgfqpoint{4.062266in}{1.769694in}}%
\pgfpathlineto{\pgfqpoint{4.063111in}{1.638945in}}%
\pgfpathlineto{\pgfqpoint{4.063957in}{2.357972in}}%
\pgfpathlineto{\pgfqpoint{4.064803in}{2.096391in}}%
\pgfpathlineto{\pgfqpoint{4.065648in}{2.097221in}}%
\pgfpathlineto{\pgfqpoint{4.066494in}{2.619545in}}%
\pgfpathlineto{\pgfqpoint{4.068185in}{1.835065in}}%
\pgfpathlineto{\pgfqpoint{4.069031in}{2.750263in}}%
\pgfpathlineto{\pgfqpoint{4.069876in}{2.358043in}}%
\pgfpathlineto{\pgfqpoint{4.070722in}{2.290671in}}%
\pgfpathlineto{\pgfqpoint{4.071567in}{1.704297in}}%
\pgfpathlineto{\pgfqpoint{4.072413in}{2.031182in}}%
\pgfpathlineto{\pgfqpoint{4.073259in}{2.031167in}}%
\pgfpathlineto{\pgfqpoint{4.074104in}{2.750245in}}%
\pgfpathlineto{\pgfqpoint{4.074950in}{2.488769in}}%
\pgfpathlineto{\pgfqpoint{4.077487in}{2.161910in}}%
\pgfpathlineto{\pgfqpoint{4.078332in}{2.161752in}}%
\pgfpathlineto{\pgfqpoint{4.079178in}{2.030972in}}%
\pgfpathlineto{\pgfqpoint{4.080024in}{2.096520in}}%
\pgfpathlineto{\pgfqpoint{4.080869in}{2.227285in}}%
\pgfpathlineto{\pgfqpoint{4.081715in}{1.769690in}}%
\pgfpathlineto{\pgfqpoint{4.082561in}{1.900432in}}%
\pgfpathlineto{\pgfqpoint{4.085097in}{2.161910in}}%
\pgfpathlineto{\pgfqpoint{4.085943in}{1.965811in}}%
\pgfpathlineto{\pgfqpoint{4.086789in}{2.433305in}}%
\pgfpathlineto{\pgfqpoint{4.087634in}{2.357880in}}%
\pgfpathlineto{\pgfqpoint{4.090171in}{1.965799in}}%
\pgfpathlineto{\pgfqpoint{4.091017in}{2.488771in}}%
\pgfpathlineto{\pgfqpoint{4.091862in}{2.488759in}}%
\pgfpathlineto{\pgfqpoint{4.095245in}{2.031167in}}%
\pgfpathlineto{\pgfqpoint{4.096091in}{2.423390in}}%
\pgfpathlineto{\pgfqpoint{4.096936in}{2.358017in}}%
\pgfpathlineto{\pgfqpoint{4.097782in}{1.769688in}}%
\pgfpathlineto{\pgfqpoint{4.100319in}{2.619520in}}%
\pgfpathlineto{\pgfqpoint{4.101164in}{1.965045in}}%
\pgfpathlineto{\pgfqpoint{4.102010in}{2.227240in}}%
\pgfpathlineto{\pgfqpoint{4.102856in}{2.227282in}}%
\pgfpathlineto{\pgfqpoint{4.104547in}{1.835060in}}%
\pgfpathlineto{\pgfqpoint{4.107084in}{2.488765in}}%
\pgfpathlineto{\pgfqpoint{4.109621in}{2.031177in}}%
\pgfpathlineto{\pgfqpoint{4.114694in}{2.423299in}}%
\pgfpathlineto{\pgfqpoint{4.115540in}{2.161629in}}%
\pgfpathlineto{\pgfqpoint{4.116386in}{2.292638in}}%
\pgfpathlineto{\pgfqpoint{4.117231in}{2.619486in}}%
\pgfpathlineto{\pgfqpoint{4.118077in}{2.357746in}}%
\pgfpathlineto{\pgfqpoint{4.118922in}{2.488738in}}%
\pgfpathlineto{\pgfqpoint{4.121459in}{1.628642in}}%
\pgfpathlineto{\pgfqpoint{4.122305in}{2.259965in}}%
\pgfpathlineto{\pgfqpoint{4.123151in}{1.933111in}}%
\pgfpathlineto{\pgfqpoint{4.124842in}{2.259967in}}%
\pgfpathlineto{\pgfqpoint{4.127379in}{1.606258in}}%
\pgfpathlineto{\pgfqpoint{4.129916in}{1.998484in}}%
\pgfpathlineto{\pgfqpoint{4.131607in}{2.535002in}}%
\pgfpathlineto{\pgfqpoint{4.132452in}{2.325337in}}%
\pgfpathlineto{\pgfqpoint{4.133298in}{1.998486in}}%
\pgfpathlineto{\pgfqpoint{4.134989in}{2.521450in}}%
\pgfpathlineto{\pgfqpoint{4.137526in}{2.194595in}}%
\pgfpathlineto{\pgfqpoint{4.139217in}{2.390666in}}%
\pgfpathlineto{\pgfqpoint{4.140063in}{2.521448in}}%
\pgfpathlineto{\pgfqpoint{4.143446in}{2.063841in}}%
\pgfpathlineto{\pgfqpoint{4.144291in}{2.456079in}}%
\pgfpathlineto{\pgfqpoint{4.145137in}{1.737069in}}%
\pgfpathlineto{\pgfqpoint{4.145982in}{2.325337in}}%
\pgfpathlineto{\pgfqpoint{4.147674in}{1.606258in}}%
\pgfpathlineto{\pgfqpoint{4.148519in}{2.456080in}}%
\pgfpathlineto{\pgfqpoint{4.149365in}{2.259988in}}%
\pgfpathlineto{\pgfqpoint{4.150210in}{2.325340in}}%
\pgfpathlineto{\pgfqpoint{4.151056in}{2.259971in}}%
\pgfpathlineto{\pgfqpoint{4.151902in}{1.737000in}}%
\pgfpathlineto{\pgfqpoint{4.153593in}{2.456075in}}%
\pgfpathlineto{\pgfqpoint{4.154439in}{1.867744in}}%
\pgfpathlineto{\pgfqpoint{4.155284in}{2.129244in}}%
\pgfpathlineto{\pgfqpoint{4.156130in}{1.998493in}}%
\pgfpathlineto{\pgfqpoint{4.157821in}{2.521439in}}%
\pgfpathlineto{\pgfqpoint{4.160358in}{2.063853in}}%
\pgfpathlineto{\pgfqpoint{4.162049in}{2.259960in}}%
\pgfpathlineto{\pgfqpoint{4.162895in}{1.606274in}}%
\pgfpathlineto{\pgfqpoint{4.163740in}{2.129225in}}%
\pgfpathlineto{\pgfqpoint{4.164586in}{2.521461in}}%
\pgfpathlineto{\pgfqpoint{4.165432in}{2.259975in}}%
\pgfpathlineto{\pgfqpoint{4.166277in}{2.063856in}}%
\pgfpathlineto{\pgfqpoint{4.167123in}{2.325336in}}%
\pgfpathlineto{\pgfqpoint{4.169660in}{1.671629in}}%
\pgfpathlineto{\pgfqpoint{4.172197in}{2.390704in}}%
\pgfpathlineto{\pgfqpoint{4.173042in}{2.063853in}}%
\pgfpathlineto{\pgfqpoint{4.173888in}{2.325338in}}%
\pgfpathlineto{\pgfqpoint{4.174734in}{2.259965in}}%
\pgfpathlineto{\pgfqpoint{4.177270in}{1.933116in}}%
\pgfpathlineto{\pgfqpoint{4.178962in}{2.325281in}}%
\pgfpathlineto{\pgfqpoint{4.179807in}{1.867752in}}%
\pgfpathlineto{\pgfqpoint{4.180653in}{1.933113in}}%
\pgfpathlineto{\pgfqpoint{4.181499in}{2.521451in}}%
\pgfpathlineto{\pgfqpoint{4.182344in}{1.998448in}}%
\pgfpathlineto{\pgfqpoint{4.183190in}{2.129223in}}%
\pgfpathlineto{\pgfqpoint{4.184035in}{1.933119in}}%
\pgfpathlineto{\pgfqpoint{4.184881in}{2.063855in}}%
\pgfpathlineto{\pgfqpoint{4.185727in}{1.933110in}}%
\pgfpathlineto{\pgfqpoint{4.188264in}{2.390796in}}%
\pgfpathlineto{\pgfqpoint{4.189109in}{1.933124in}}%
\pgfpathlineto{\pgfqpoint{4.189955in}{2.194599in}}%
\pgfpathlineto{\pgfqpoint{4.190800in}{2.586824in}}%
\pgfpathlineto{\pgfqpoint{4.191646in}{2.129204in}}%
\pgfpathlineto{\pgfqpoint{4.192492in}{2.325334in}}%
\pgfpathlineto{\pgfqpoint{4.193337in}{2.129269in}}%
\pgfpathlineto{\pgfqpoint{4.194183in}{2.259966in}}%
\pgfpathlineto{\pgfqpoint{4.195874in}{1.998480in}}%
\pgfpathlineto{\pgfqpoint{4.196720in}{2.063854in}}%
\pgfpathlineto{\pgfqpoint{4.197565in}{2.390710in}}%
\pgfpathlineto{\pgfqpoint{4.198411in}{2.129180in}}%
\pgfpathlineto{\pgfqpoint{4.199257in}{1.736999in}}%
\pgfpathlineto{\pgfqpoint{4.200102in}{2.288969in}}%
\pgfpathlineto{\pgfqpoint{4.200948in}{2.259968in}}%
\pgfpathlineto{\pgfqpoint{4.202639in}{2.063858in}}%
\pgfpathlineto{\pgfqpoint{4.203485in}{2.521408in}}%
\pgfpathlineto{\pgfqpoint{4.205176in}{2.129212in}}%
\pgfpathlineto{\pgfqpoint{4.206022in}{2.129226in}}%
\pgfpathlineto{\pgfqpoint{4.206867in}{2.390707in}}%
\pgfpathlineto{\pgfqpoint{4.207713in}{2.324460in}}%
\pgfpathlineto{\pgfqpoint{4.208559in}{2.325575in}}%
\pgfpathlineto{\pgfqpoint{4.211095in}{1.933112in}}%
\pgfpathlineto{\pgfqpoint{4.212787in}{2.456077in}}%
\pgfpathlineto{\pgfqpoint{4.215324in}{1.933125in}}%
\pgfpathlineto{\pgfqpoint{4.216169in}{2.325339in}}%
\pgfpathlineto{\pgfqpoint{4.217015in}{1.933116in}}%
\pgfpathlineto{\pgfqpoint{4.217860in}{2.521454in}}%
\pgfpathlineto{\pgfqpoint{4.218706in}{2.325329in}}%
\pgfpathlineto{\pgfqpoint{4.220397in}{2.586773in}}%
\pgfpathlineto{\pgfqpoint{4.221243in}{1.933117in}}%
\pgfpathlineto{\pgfqpoint{4.222089in}{2.586805in}}%
\pgfpathlineto{\pgfqpoint{4.222934in}{1.933113in}}%
\pgfpathlineto{\pgfqpoint{4.224625in}{2.521443in}}%
\pgfpathlineto{\pgfqpoint{4.226317in}{1.867391in}}%
\pgfpathlineto{\pgfqpoint{4.227162in}{2.063858in}}%
\pgfpathlineto{\pgfqpoint{4.228853in}{2.456080in}}%
\pgfpathlineto{\pgfqpoint{4.229699in}{2.259966in}}%
\pgfpathlineto{\pgfqpoint{4.230545in}{2.586806in}}%
\pgfpathlineto{\pgfqpoint{4.232236in}{2.063856in}}%
\pgfpathlineto{\pgfqpoint{4.235618in}{2.782936in}}%
\pgfpathlineto{\pgfqpoint{4.238155in}{1.802574in}}%
\pgfpathlineto{\pgfqpoint{4.239001in}{2.063883in}}%
\pgfpathlineto{\pgfqpoint{4.239847in}{2.259892in}}%
\pgfpathlineto{\pgfqpoint{4.241538in}{1.793625in}}%
\pgfpathlineto{\pgfqpoint{4.244075in}{2.449239in}}%
\pgfpathlineto{\pgfqpoint{4.245766in}{1.867715in}}%
\pgfpathlineto{\pgfqpoint{4.248303in}{2.823768in}}%
\pgfpathlineto{\pgfqpoint{4.249994in}{1.965902in}}%
\pgfpathlineto{\pgfqpoint{4.250840in}{2.684758in}}%
\pgfpathlineto{\pgfqpoint{4.251685in}{2.292618in}}%
\pgfpathlineto{\pgfqpoint{4.253377in}{2.554141in}}%
\pgfpathlineto{\pgfqpoint{4.254222in}{1.900427in}}%
\pgfpathlineto{\pgfqpoint{4.255068in}{2.096544in}}%
\pgfpathlineto{\pgfqpoint{4.255913in}{2.292633in}}%
\pgfpathlineto{\pgfqpoint{4.257605in}{1.900800in}}%
\pgfpathlineto{\pgfqpoint{4.260142in}{3.011707in}}%
\pgfpathlineto{\pgfqpoint{4.261833in}{1.835052in}}%
\pgfpathlineto{\pgfqpoint{4.263524in}{1.901411in}}%
\pgfpathlineto{\pgfqpoint{4.264370in}{2.489072in}}%
\pgfpathlineto{\pgfqpoint{4.265215in}{1.965817in}}%
\pgfpathlineto{\pgfqpoint{4.266061in}{2.423394in}}%
\pgfpathlineto{\pgfqpoint{4.266907in}{2.096538in}}%
\pgfpathlineto{\pgfqpoint{4.267752in}{2.292655in}}%
\pgfpathlineto{\pgfqpoint{4.270289in}{1.900425in}}%
\pgfpathlineto{\pgfqpoint{4.273672in}{2.162039in}}%
\pgfpathlineto{\pgfqpoint{4.274517in}{2.031157in}}%
\pgfpathlineto{\pgfqpoint{4.275363in}{2.423410in}}%
\pgfpathlineto{\pgfqpoint{4.276208in}{1.900431in}}%
\pgfpathlineto{\pgfqpoint{4.277054in}{2.096538in}}%
\pgfpathlineto{\pgfqpoint{4.278745in}{2.554133in}}%
\pgfpathlineto{\pgfqpoint{4.280437in}{1.835057in}}%
\pgfpathlineto{\pgfqpoint{4.281282in}{2.750272in}}%
\pgfpathlineto{\pgfqpoint{4.282128in}{1.639052in}}%
\pgfpathlineto{\pgfqpoint{4.282973in}{2.488776in}}%
\pgfpathlineto{\pgfqpoint{4.283819in}{2.161912in}}%
\pgfpathlineto{\pgfqpoint{4.284665in}{2.357985in}}%
\pgfpathlineto{\pgfqpoint{4.285510in}{2.619479in}}%
\pgfpathlineto{\pgfqpoint{4.287202in}{1.965802in}}%
\pgfpathlineto{\pgfqpoint{4.288893in}{2.488641in}}%
\pgfpathlineto{\pgfqpoint{4.290584in}{1.900441in}}%
\pgfpathlineto{\pgfqpoint{4.291430in}{2.488767in}}%
\pgfpathlineto{\pgfqpoint{4.292275in}{2.358005in}}%
\pgfpathlineto{\pgfqpoint{4.293121in}{2.423404in}}%
\pgfpathlineto{\pgfqpoint{4.294812in}{2.227337in}}%
\pgfpathlineto{\pgfqpoint{4.295658in}{2.227431in}}%
\pgfpathlineto{\pgfqpoint{4.296503in}{2.488787in}}%
\pgfpathlineto{\pgfqpoint{4.298195in}{1.900446in}}%
\pgfpathlineto{\pgfqpoint{4.299886in}{2.096627in}}%
\pgfpathlineto{\pgfqpoint{4.300732in}{1.965800in}}%
\pgfpathlineto{\pgfqpoint{4.301577in}{2.227280in}}%
\pgfpathlineto{\pgfqpoint{4.304114in}{1.769734in}}%
\pgfpathlineto{\pgfqpoint{4.305805in}{2.358024in}}%
\pgfpathlineto{\pgfqpoint{4.306651in}{1.965907in}}%
\pgfpathlineto{\pgfqpoint{4.308342in}{2.358024in}}%
\pgfpathlineto{\pgfqpoint{4.309188in}{2.292649in}}%
\pgfpathlineto{\pgfqpoint{4.310033in}{1.638943in}}%
\pgfpathlineto{\pgfqpoint{4.310879in}{2.435576in}}%
\pgfpathlineto{\pgfqpoint{4.311725in}{2.292657in}}%
\pgfpathlineto{\pgfqpoint{4.312570in}{2.292648in}}%
\pgfpathlineto{\pgfqpoint{4.313416in}{2.096529in}}%
\pgfpathlineto{\pgfqpoint{4.314261in}{2.161928in}}%
\pgfpathlineto{\pgfqpoint{4.315107in}{2.750241in}}%
\pgfpathlineto{\pgfqpoint{4.317644in}{1.965797in}}%
\pgfpathlineto{\pgfqpoint{4.319335in}{2.358453in}}%
\pgfpathlineto{\pgfqpoint{4.320181in}{2.358024in}}%
\pgfpathlineto{\pgfqpoint{4.321872in}{2.031170in}}%
\pgfpathlineto{\pgfqpoint{4.323563in}{2.423387in}}%
\pgfpathlineto{\pgfqpoint{4.324409in}{1.900429in}}%
\pgfpathlineto{\pgfqpoint{4.325255in}{2.292145in}}%
\pgfpathlineto{\pgfqpoint{4.326100in}{1.900427in}}%
\pgfpathlineto{\pgfqpoint{4.328637in}{2.488765in}}%
\pgfpathlineto{\pgfqpoint{4.329483in}{2.554135in}}%
\pgfpathlineto{\pgfqpoint{4.330328in}{2.227276in}}%
\pgfpathlineto{\pgfqpoint{4.331174in}{2.815619in}}%
\pgfpathlineto{\pgfqpoint{4.332020in}{2.488756in}}%
\pgfpathlineto{\pgfqpoint{4.332865in}{1.835057in}}%
\pgfpathlineto{\pgfqpoint{4.333711in}{2.423394in}}%
\pgfpathlineto{\pgfqpoint{4.334556in}{2.031170in}}%
\pgfpathlineto{\pgfqpoint{4.336248in}{2.161910in}}%
\pgfpathlineto{\pgfqpoint{4.337093in}{2.097118in}}%
\pgfpathlineto{\pgfqpoint{4.337939in}{2.423404in}}%
\pgfpathlineto{\pgfqpoint{4.338785in}{2.161919in}}%
\pgfpathlineto{\pgfqpoint{4.341321in}{2.423395in}}%
\pgfpathlineto{\pgfqpoint{4.342167in}{2.292651in}}%
\pgfpathlineto{\pgfqpoint{4.343013in}{2.488763in}}%
\pgfpathlineto{\pgfqpoint{4.344704in}{2.096530in}}%
\pgfpathlineto{\pgfqpoint{4.345550in}{2.292653in}}%
\pgfpathlineto{\pgfqpoint{4.346395in}{2.031170in}}%
\pgfpathlineto{\pgfqpoint{4.348086in}{2.423389in}}%
\pgfpathlineto{\pgfqpoint{4.348932in}{2.096541in}}%
\pgfpathlineto{\pgfqpoint{4.349778in}{2.292663in}}%
\pgfpathlineto{\pgfqpoint{4.350623in}{2.358024in}}%
\pgfpathlineto{\pgfqpoint{4.351469in}{2.292622in}}%
\pgfpathlineto{\pgfqpoint{4.352315in}{2.031161in}}%
\pgfpathlineto{\pgfqpoint{4.353160in}{2.096538in}}%
\pgfpathlineto{\pgfqpoint{4.354006in}{2.423394in}}%
\pgfpathlineto{\pgfqpoint{4.354851in}{1.835057in}}%
\pgfpathlineto{\pgfqpoint{4.355697in}{2.096536in}}%
\pgfpathlineto{\pgfqpoint{4.356543in}{2.292681in}}%
\pgfpathlineto{\pgfqpoint{4.357388in}{2.161861in}}%
\pgfpathlineto{\pgfqpoint{4.358234in}{2.096534in}}%
\pgfpathlineto{\pgfqpoint{4.359080in}{2.488773in}}%
\pgfpathlineto{\pgfqpoint{4.359925in}{2.488747in}}%
\pgfpathlineto{\pgfqpoint{4.360771in}{1.835068in}}%
\pgfpathlineto{\pgfqpoint{4.361616in}{2.750248in}}%
\pgfpathlineto{\pgfqpoint{4.362462in}{2.488863in}}%
\pgfpathlineto{\pgfqpoint{4.363308in}{2.031172in}}%
\pgfpathlineto{\pgfqpoint{4.364153in}{2.423406in}}%
\pgfpathlineto{\pgfqpoint{4.364999in}{2.358011in}}%
\pgfpathlineto{\pgfqpoint{4.365845in}{1.638944in}}%
\pgfpathlineto{\pgfqpoint{4.366690in}{1.769417in}}%
\pgfpathlineto{\pgfqpoint{4.367536in}{1.835053in}}%
\pgfpathlineto{\pgfqpoint{4.368381in}{2.488740in}}%
\pgfpathlineto{\pgfqpoint{4.369227in}{1.835052in}}%
\pgfpathlineto{\pgfqpoint{4.370073in}{2.031172in}}%
\pgfpathlineto{\pgfqpoint{4.372610in}{2.750239in}}%
\pgfpathlineto{\pgfqpoint{4.375146in}{1.638945in}}%
\pgfpathlineto{\pgfqpoint{4.375992in}{2.227284in}}%
\pgfpathlineto{\pgfqpoint{4.376838in}{2.227206in}}%
\pgfpathlineto{\pgfqpoint{4.377683in}{2.292652in}}%
\pgfpathlineto{\pgfqpoint{4.378529in}{1.965705in}}%
\pgfpathlineto{\pgfqpoint{4.379375in}{2.031207in}}%
\pgfpathlineto{\pgfqpoint{4.381066in}{2.423394in}}%
\pgfpathlineto{\pgfqpoint{4.381911in}{1.965803in}}%
\pgfpathlineto{\pgfqpoint{4.382757in}{2.423386in}}%
\pgfpathlineto{\pgfqpoint{4.384448in}{1.900433in}}%
\pgfpathlineto{\pgfqpoint{4.385294in}{2.554137in}}%
\pgfpathlineto{\pgfqpoint{4.386140in}{2.161909in}}%
\pgfpathlineto{\pgfqpoint{4.386985in}{1.769676in}}%
\pgfpathlineto{\pgfqpoint{4.387831in}{2.554137in}}%
\pgfpathlineto{\pgfqpoint{4.388676in}{1.770125in}}%
\pgfpathlineto{\pgfqpoint{4.389522in}{2.358019in}}%
\pgfpathlineto{\pgfqpoint{4.390368in}{2.227274in}}%
\pgfpathlineto{\pgfqpoint{4.391213in}{2.358008in}}%
\pgfpathlineto{\pgfqpoint{4.392904in}{1.965800in}}%
\pgfpathlineto{\pgfqpoint{4.394596in}{2.423394in}}%
\pgfpathlineto{\pgfqpoint{4.396287in}{1.769679in}}%
\pgfpathlineto{\pgfqpoint{4.398824in}{2.423398in}}%
\pgfpathlineto{\pgfqpoint{4.399669in}{1.900866in}}%
\pgfpathlineto{\pgfqpoint{4.400515in}{2.423401in}}%
\pgfpathlineto{\pgfqpoint{4.401361in}{2.161911in}}%
\pgfpathlineto{\pgfqpoint{4.402206in}{2.423391in}}%
\pgfpathlineto{\pgfqpoint{4.403052in}{2.161911in}}%
\pgfpathlineto{\pgfqpoint{4.403898in}{2.358023in}}%
\pgfpathlineto{\pgfqpoint{4.404743in}{2.161905in}}%
\pgfpathlineto{\pgfqpoint{4.405589in}{2.161913in}}%
\pgfpathlineto{\pgfqpoint{4.407280in}{2.358020in}}%
\pgfpathlineto{\pgfqpoint{4.408126in}{1.965803in}}%
\pgfpathlineto{\pgfqpoint{4.408971in}{2.554113in}}%
\pgfpathlineto{\pgfqpoint{4.409817in}{2.358020in}}%
\pgfpathlineto{\pgfqpoint{4.411508in}{1.704315in}}%
\pgfpathlineto{\pgfqpoint{4.413199in}{2.423394in}}%
\pgfpathlineto{\pgfqpoint{4.414045in}{2.422429in}}%
\pgfpathlineto{\pgfqpoint{4.414891in}{2.096541in}}%
\pgfpathlineto{\pgfqpoint{4.415736in}{2.423395in}}%
\pgfpathlineto{\pgfqpoint{4.417428in}{1.965799in}}%
\pgfpathlineto{\pgfqpoint{4.418273in}{2.161911in}}%
\pgfpathlineto{\pgfqpoint{4.419119in}{1.638941in}}%
\pgfpathlineto{\pgfqpoint{4.420810in}{2.423401in}}%
\pgfpathlineto{\pgfqpoint{4.421656in}{2.227301in}}%
\pgfpathlineto{\pgfqpoint{4.422501in}{2.423349in}}%
\pgfpathlineto{\pgfqpoint{4.423347in}{1.704307in}}%
\pgfpathlineto{\pgfqpoint{4.424193in}{2.161894in}}%
\pgfpathlineto{\pgfqpoint{4.425884in}{2.292531in}}%
\pgfpathlineto{\pgfqpoint{4.426729in}{2.031181in}}%
\pgfpathlineto{\pgfqpoint{4.427575in}{2.554141in}}%
\pgfpathlineto{\pgfqpoint{4.428421in}{1.965801in}}%
\pgfpathlineto{\pgfqpoint{4.429266in}{2.161908in}}%
\pgfpathlineto{\pgfqpoint{4.430958in}{2.031175in}}%
\pgfpathlineto{\pgfqpoint{4.434340in}{2.291139in}}%
\pgfpathlineto{\pgfqpoint{4.435186in}{1.900436in}}%
\pgfpathlineto{\pgfqpoint{4.436031in}{2.096557in}}%
\pgfpathlineto{\pgfqpoint{4.436877in}{2.031171in}}%
\pgfpathlineto{\pgfqpoint{4.437723in}{1.508215in}}%
\pgfpathlineto{\pgfqpoint{4.438568in}{2.227274in}}%
\pgfpathlineto{\pgfqpoint{4.439414in}{2.161910in}}%
\pgfpathlineto{\pgfqpoint{4.440259in}{2.096544in}}%
\pgfpathlineto{\pgfqpoint{4.441105in}{1.900469in}}%
\pgfpathlineto{\pgfqpoint{4.442796in}{2.423399in}}%
\pgfpathlineto{\pgfqpoint{4.443642in}{2.358035in}}%
\pgfpathlineto{\pgfqpoint{4.444488in}{2.161880in}}%
\pgfpathlineto{\pgfqpoint{4.445333in}{2.554130in}}%
\pgfpathlineto{\pgfqpoint{4.446179in}{1.573585in}}%
\pgfpathlineto{\pgfqpoint{4.447024in}{2.292651in}}%
\pgfpathlineto{\pgfqpoint{4.447870in}{2.162015in}}%
\pgfpathlineto{\pgfqpoint{4.448716in}{2.619502in}}%
\pgfpathlineto{\pgfqpoint{4.450407in}{1.835134in}}%
\pgfpathlineto{\pgfqpoint{4.452944in}{2.357985in}}%
\pgfpathlineto{\pgfqpoint{4.454635in}{2.161846in}}%
\pgfpathlineto{\pgfqpoint{4.455481in}{2.161913in}}%
\pgfpathlineto{\pgfqpoint{4.456326in}{1.965832in}}%
\pgfpathlineto{\pgfqpoint{4.457172in}{2.031197in}}%
\pgfpathlineto{\pgfqpoint{4.458018in}{2.554112in}}%
\pgfpathlineto{\pgfqpoint{4.458863in}{2.356455in}}%
\pgfpathlineto{\pgfqpoint{4.459709in}{1.900467in}}%
\pgfpathlineto{\pgfqpoint{4.460554in}{2.292488in}}%
\pgfpathlineto{\pgfqpoint{4.461400in}{1.965818in}}%
\pgfpathlineto{\pgfqpoint{4.464783in}{2.945739in}}%
\pgfpathlineto{\pgfqpoint{4.465628in}{2.096531in}}%
\pgfpathlineto{\pgfqpoint{4.466474in}{2.096548in}}%
\pgfpathlineto{\pgfqpoint{4.468165in}{2.488771in}}%
\pgfpathlineto{\pgfqpoint{4.469011in}{2.031178in}}%
\pgfpathlineto{\pgfqpoint{4.469856in}{2.161884in}}%
\pgfpathlineto{\pgfqpoint{4.472393in}{2.488762in}}%
\pgfpathlineto{\pgfqpoint{4.473239in}{1.769693in}}%
\pgfpathlineto{\pgfqpoint{4.475776in}{2.684879in}}%
\pgfpathlineto{\pgfqpoint{4.477467in}{1.638968in}}%
\pgfpathlineto{\pgfqpoint{4.478312in}{2.357774in}}%
\pgfpathlineto{\pgfqpoint{4.479158in}{2.162530in}}%
\pgfpathlineto{\pgfqpoint{4.480849in}{1.769674in}}%
\pgfpathlineto{\pgfqpoint{4.482541in}{2.292655in}}%
\pgfpathlineto{\pgfqpoint{4.483386in}{1.900432in}}%
\pgfpathlineto{\pgfqpoint{4.485077in}{3.011702in}}%
\pgfpathlineto{\pgfqpoint{4.488460in}{1.965779in}}%
\pgfpathlineto{\pgfqpoint{4.490151in}{2.358259in}}%
\pgfpathlineto{\pgfqpoint{4.490997in}{1.638907in}}%
\pgfpathlineto{\pgfqpoint{4.491842in}{2.227235in}}%
\pgfpathlineto{\pgfqpoint{4.492688in}{2.096545in}}%
\pgfpathlineto{\pgfqpoint{4.493534in}{2.358026in}}%
\pgfpathlineto{\pgfqpoint{4.494379in}{2.031152in}}%
\pgfpathlineto{\pgfqpoint{4.495225in}{2.423467in}}%
\pgfpathlineto{\pgfqpoint{4.496916in}{1.704321in}}%
\pgfpathlineto{\pgfqpoint{4.498607in}{2.161915in}}%
\pgfpathlineto{\pgfqpoint{4.499453in}{2.031045in}}%
\pgfpathlineto{\pgfqpoint{4.500299in}{2.161911in}}%
\pgfpathlineto{\pgfqpoint{4.501144in}{2.946354in}}%
\pgfpathlineto{\pgfqpoint{4.501990in}{2.358039in}}%
\pgfpathlineto{\pgfqpoint{4.502836in}{2.488755in}}%
\pgfpathlineto{\pgfqpoint{4.503681in}{2.227266in}}%
\pgfpathlineto{\pgfqpoint{4.504527in}{2.358033in}}%
\pgfpathlineto{\pgfqpoint{4.505372in}{2.358024in}}%
\pgfpathlineto{\pgfqpoint{4.506218in}{2.423393in}}%
\pgfpathlineto{\pgfqpoint{4.507064in}{2.096561in}}%
\pgfpathlineto{\pgfqpoint{4.507909in}{2.488765in}}%
\pgfpathlineto{\pgfqpoint{4.510446in}{1.573569in}}%
\pgfpathlineto{\pgfqpoint{4.511292in}{2.227281in}}%
\pgfpathlineto{\pgfqpoint{4.512137in}{2.031235in}}%
\pgfpathlineto{\pgfqpoint{4.514674in}{2.423392in}}%
\pgfpathlineto{\pgfqpoint{4.515520in}{2.292758in}}%
\pgfpathlineto{\pgfqpoint{4.516366in}{1.900423in}}%
\pgfpathlineto{\pgfqpoint{4.517211in}{2.358013in}}%
\pgfpathlineto{\pgfqpoint{4.518057in}{2.357976in}}%
\pgfpathlineto{\pgfqpoint{4.520594in}{2.358024in}}%
\pgfpathlineto{\pgfqpoint{4.521439in}{1.900399in}}%
\pgfpathlineto{\pgfqpoint{4.522285in}{1.900426in}}%
\pgfpathlineto{\pgfqpoint{4.523131in}{2.554212in}}%
\pgfpathlineto{\pgfqpoint{4.523976in}{2.227239in}}%
\pgfpathlineto{\pgfqpoint{4.524822in}{2.423391in}}%
\pgfpathlineto{\pgfqpoint{4.527359in}{1.965804in}}%
\pgfpathlineto{\pgfqpoint{4.528204in}{2.423386in}}%
\pgfpathlineto{\pgfqpoint{4.529050in}{1.769640in}}%
\pgfpathlineto{\pgfqpoint{4.529896in}{2.227271in}}%
\pgfpathlineto{\pgfqpoint{4.530741in}{1.965749in}}%
\pgfpathlineto{\pgfqpoint{4.531587in}{2.358000in}}%
\pgfpathlineto{\pgfqpoint{4.532432in}{2.161917in}}%
\pgfpathlineto{\pgfqpoint{4.534124in}{2.750340in}}%
\pgfpathlineto{\pgfqpoint{4.534969in}{2.161991in}}%
\pgfpathlineto{\pgfqpoint{4.535815in}{2.358052in}}%
\pgfpathlineto{\pgfqpoint{4.536661in}{2.423391in}}%
\pgfpathlineto{\pgfqpoint{4.537506in}{1.900427in}}%
\pgfpathlineto{\pgfqpoint{4.538352in}{2.161936in}}%
\pgfpathlineto{\pgfqpoint{4.539197in}{2.423382in}}%
\pgfpathlineto{\pgfqpoint{4.540889in}{1.835064in}}%
\pgfpathlineto{\pgfqpoint{4.543426in}{2.488716in}}%
\pgfpathlineto{\pgfqpoint{4.544271in}{2.227257in}}%
\pgfpathlineto{\pgfqpoint{4.545117in}{2.488339in}}%
\pgfpathlineto{\pgfqpoint{4.545962in}{1.769677in}}%
\pgfpathlineto{\pgfqpoint{4.546808in}{2.227278in}}%
\pgfpathlineto{\pgfqpoint{4.549345in}{2.488759in}}%
\pgfpathlineto{\pgfqpoint{4.551882in}{1.835075in}}%
\pgfpathlineto{\pgfqpoint{4.553573in}{2.423428in}}%
\pgfpathlineto{\pgfqpoint{4.554419in}{2.423393in}}%
\pgfpathlineto{\pgfqpoint{4.555264in}{2.423715in}}%
\pgfpathlineto{\pgfqpoint{4.556110in}{1.769764in}}%
\pgfpathlineto{\pgfqpoint{4.556955in}{2.554126in}}%
\pgfpathlineto{\pgfqpoint{4.557801in}{2.161907in}}%
\pgfpathlineto{\pgfqpoint{4.558647in}{2.619492in}}%
\pgfpathlineto{\pgfqpoint{4.560338in}{2.227086in}}%
\pgfpathlineto{\pgfqpoint{4.561184in}{2.227319in}}%
\pgfpathlineto{\pgfqpoint{4.563720in}{1.835080in}}%
\pgfpathlineto{\pgfqpoint{4.564566in}{1.900424in}}%
\pgfpathlineto{\pgfqpoint{4.565412in}{2.162573in}}%
\pgfpathlineto{\pgfqpoint{4.566257in}{2.161898in}}%
\pgfpathlineto{\pgfqpoint{4.567949in}{1.900392in}}%
\pgfpathlineto{\pgfqpoint{4.569640in}{2.292643in}}%
\pgfpathlineto{\pgfqpoint{4.570485in}{1.965803in}}%
\pgfpathlineto{\pgfqpoint{4.571331in}{2.227262in}}%
\pgfpathlineto{\pgfqpoint{4.572177in}{1.965799in}}%
\pgfpathlineto{\pgfqpoint{4.573868in}{2.292021in}}%
\pgfpathlineto{\pgfqpoint{4.574714in}{2.227273in}}%
\pgfpathlineto{\pgfqpoint{4.576405in}{2.488705in}}%
\pgfpathlineto{\pgfqpoint{4.578096in}{1.900453in}}%
\pgfpathlineto{\pgfqpoint{4.578942in}{2.094423in}}%
\pgfpathlineto{\pgfqpoint{4.579787in}{2.098549in}}%
\pgfpathlineto{\pgfqpoint{4.580633in}{2.750285in}}%
\pgfpathlineto{\pgfqpoint{4.581479in}{2.032018in}}%
\pgfpathlineto{\pgfqpoint{4.582324in}{2.292696in}}%
\pgfpathlineto{\pgfqpoint{4.583170in}{2.225971in}}%
\pgfpathlineto{\pgfqpoint{4.584015in}{1.965872in}}%
\pgfpathlineto{\pgfqpoint{4.584861in}{2.422664in}}%
\pgfpathlineto{\pgfqpoint{4.585707in}{2.161933in}}%
\pgfpathlineto{\pgfqpoint{4.586552in}{2.030975in}}%
\pgfpathlineto{\pgfqpoint{4.588244in}{2.492607in}}%
\pgfpathlineto{\pgfqpoint{4.589089in}{2.488883in}}%
\pgfpathlineto{\pgfqpoint{4.589935in}{1.835065in}}%
\pgfpathlineto{\pgfqpoint{4.590780in}{1.965809in}}%
\pgfpathlineto{\pgfqpoint{4.592472in}{2.292644in}}%
\pgfpathlineto{\pgfqpoint{4.593317in}{2.227889in}}%
\pgfpathlineto{\pgfqpoint{4.595854in}{1.834775in}}%
\pgfpathlineto{\pgfqpoint{4.597545in}{2.227278in}}%
\pgfpathlineto{\pgfqpoint{4.598391in}{2.031197in}}%
\pgfpathlineto{\pgfqpoint{4.599237in}{2.423319in}}%
\pgfpathlineto{\pgfqpoint{4.600082in}{2.096540in}}%
\pgfpathlineto{\pgfqpoint{4.600928in}{2.753167in}}%
\pgfpathlineto{\pgfqpoint{4.601774in}{2.096591in}}%
\pgfpathlineto{\pgfqpoint{4.602619in}{2.488777in}}%
\pgfpathlineto{\pgfqpoint{4.604310in}{2.031221in}}%
\pgfpathlineto{\pgfqpoint{4.605156in}{2.161930in}}%
\pgfpathlineto{\pgfqpoint{4.606002in}{2.031036in}}%
\pgfpathlineto{\pgfqpoint{4.606847in}{2.357607in}}%
\pgfpathlineto{\pgfqpoint{4.607693in}{2.292230in}}%
\pgfpathlineto{\pgfqpoint{4.609384in}{2.096520in}}%
\pgfpathlineto{\pgfqpoint{4.610230in}{2.292539in}}%
\pgfpathlineto{\pgfqpoint{4.611075in}{2.288256in}}%
\pgfpathlineto{\pgfqpoint{4.613612in}{1.899606in}}%
\pgfpathlineto{\pgfqpoint{4.614458in}{2.456031in}}%
\pgfpathlineto{\pgfqpoint{4.615304in}{2.012573in}}%
\pgfpathlineto{\pgfqpoint{4.616149in}{2.129245in}}%
\pgfpathlineto{\pgfqpoint{4.616995in}{2.010305in}}%
\pgfpathlineto{\pgfqpoint{4.617840in}{2.329455in}}%
\pgfpathlineto{\pgfqpoint{4.618686in}{2.325441in}}%
\pgfpathlineto{\pgfqpoint{4.620377in}{1.998459in}}%
\pgfpathlineto{\pgfqpoint{4.621223in}{2.652151in}}%
\pgfpathlineto{\pgfqpoint{4.622069in}{1.606280in}}%
\pgfpathlineto{\pgfqpoint{4.622914in}{2.259959in}}%
\pgfpathlineto{\pgfqpoint{4.623760in}{2.325342in}}%
\pgfpathlineto{\pgfqpoint{4.624605in}{1.998485in}}%
\pgfpathlineto{\pgfqpoint{4.625451in}{2.455954in}}%
\pgfpathlineto{\pgfqpoint{4.626297in}{2.259952in}}%
\pgfpathlineto{\pgfqpoint{4.627142in}{1.737037in}}%
\pgfpathlineto{\pgfqpoint{4.627988in}{2.194449in}}%
\pgfpathlineto{\pgfqpoint{4.628833in}{1.998537in}}%
\pgfpathlineto{\pgfqpoint{4.631370in}{2.390958in}}%
\pgfpathlineto{\pgfqpoint{4.632216in}{1.998490in}}%
\pgfpathlineto{\pgfqpoint{4.633062in}{2.390703in}}%
\pgfpathlineto{\pgfqpoint{4.633907in}{1.867754in}}%
\pgfpathlineto{\pgfqpoint{4.634753in}{2.390609in}}%
\pgfpathlineto{\pgfqpoint{4.635598in}{2.063820in}}%
\pgfpathlineto{\pgfqpoint{4.636444in}{1.933114in}}%
\pgfpathlineto{\pgfqpoint{4.638135in}{2.521238in}}%
\pgfpathlineto{\pgfqpoint{4.638981in}{2.456119in}}%
\pgfpathlineto{\pgfqpoint{4.641518in}{1.933124in}}%
\pgfpathlineto{\pgfqpoint{4.642363in}{2.129171in}}%
\pgfpathlineto{\pgfqpoint{4.643209in}{2.453605in}}%
\pgfpathlineto{\pgfqpoint{4.644055in}{2.194696in}}%
\pgfpathlineto{\pgfqpoint{4.644900in}{1.998434in}}%
\pgfpathlineto{\pgfqpoint{4.645746in}{1.998670in}}%
\pgfpathlineto{\pgfqpoint{4.646592in}{2.456049in}}%
\pgfpathlineto{\pgfqpoint{4.647437in}{2.197231in}}%
\pgfpathlineto{\pgfqpoint{4.648283in}{1.867762in}}%
\pgfpathlineto{\pgfqpoint{4.649128in}{2.521343in}}%
\pgfpathlineto{\pgfqpoint{4.649974in}{1.933489in}}%
\pgfpathlineto{\pgfqpoint{4.650820in}{2.515474in}}%
\pgfpathlineto{\pgfqpoint{4.651665in}{2.128770in}}%
\pgfpathlineto{\pgfqpoint{4.652511in}{2.385962in}}%
\pgfpathlineto{\pgfqpoint{4.653357in}{2.327203in}}%
\pgfpathlineto{\pgfqpoint{4.655048in}{2.003190in}}%
\pgfpathlineto{\pgfqpoint{4.657585in}{2.488787in}}%
\pgfpathlineto{\pgfqpoint{4.658430in}{2.358001in}}%
\pgfpathlineto{\pgfqpoint{4.659276in}{1.900374in}}%
\pgfpathlineto{\pgfqpoint{4.660122in}{2.488643in}}%
\pgfpathlineto{\pgfqpoint{4.660967in}{2.031188in}}%
\pgfpathlineto{\pgfqpoint{4.661813in}{2.292518in}}%
\pgfpathlineto{\pgfqpoint{4.662658in}{1.769722in}}%
\pgfpathlineto{\pgfqpoint{4.663504in}{2.162165in}}%
\pgfpathlineto{\pgfqpoint{4.664350in}{1.770278in}}%
\pgfpathlineto{\pgfqpoint{4.666041in}{2.623337in}}%
\pgfpathlineto{\pgfqpoint{4.667732in}{2.161962in}}%
\pgfpathlineto{\pgfqpoint{4.668578in}{2.227370in}}%
\pgfpathlineto{\pgfqpoint{4.669423in}{2.096557in}}%
\pgfpathlineto{\pgfqpoint{4.670269in}{2.161795in}}%
\pgfpathlineto{\pgfqpoint{4.671115in}{2.161917in}}%
\pgfpathlineto{\pgfqpoint{4.671960in}{2.227230in}}%
\pgfpathlineto{\pgfqpoint{4.672806in}{2.554162in}}%
\pgfpathlineto{\pgfqpoint{4.673652in}{1.704336in}}%
\pgfpathlineto{\pgfqpoint{4.674497in}{2.031159in}}%
\pgfpathlineto{\pgfqpoint{4.675343in}{2.292693in}}%
\pgfpathlineto{\pgfqpoint{4.676188in}{2.031175in}}%
\pgfpathlineto{\pgfqpoint{4.677034in}{2.161872in}}%
\pgfpathlineto{\pgfqpoint{4.677880in}{2.096545in}}%
\pgfpathlineto{\pgfqpoint{4.678725in}{2.227214in}}%
\pgfpathlineto{\pgfqpoint{4.679571in}{2.096591in}}%
\pgfpathlineto{\pgfqpoint{4.680417in}{2.227286in}}%
\pgfpathlineto{\pgfqpoint{4.681262in}{2.161996in}}%
\pgfpathlineto{\pgfqpoint{4.682108in}{2.179125in}}%
\pgfpathlineto{\pgfqpoint{4.682953in}{1.834622in}}%
\pgfpathlineto{\pgfqpoint{4.685490in}{2.684852in}}%
\pgfpathlineto{\pgfqpoint{4.686336in}{1.900703in}}%
\pgfpathlineto{\pgfqpoint{4.687182in}{2.161916in}}%
\pgfpathlineto{\pgfqpoint{4.688027in}{2.488811in}}%
\pgfpathlineto{\pgfqpoint{4.688873in}{2.292652in}}%
\pgfpathlineto{\pgfqpoint{4.689718in}{1.769699in}}%
\pgfpathlineto{\pgfqpoint{4.690564in}{2.161923in}}%
\pgfpathlineto{\pgfqpoint{4.691410in}{2.294604in}}%
\pgfpathlineto{\pgfqpoint{4.693101in}{1.769686in}}%
\pgfpathlineto{\pgfqpoint{4.693947in}{2.162169in}}%
\pgfpathlineto{\pgfqpoint{4.694792in}{2.096549in}}%
\pgfpathlineto{\pgfqpoint{4.698175in}{2.423403in}}%
\pgfpathlineto{\pgfqpoint{4.699866in}{2.161739in}}%
\pgfpathlineto{\pgfqpoint{4.700712in}{2.162463in}}%
\pgfpathlineto{\pgfqpoint{4.701557in}{2.684881in}}%
\pgfpathlineto{\pgfqpoint{4.703248in}{2.031187in}}%
\pgfpathlineto{\pgfqpoint{4.704094in}{2.096452in}}%
\pgfpathlineto{\pgfqpoint{4.704940in}{1.965016in}}%
\pgfpathlineto{\pgfqpoint{4.705785in}{2.488578in}}%
\pgfpathlineto{\pgfqpoint{4.706631in}{2.161895in}}%
\pgfpathlineto{\pgfqpoint{4.707476in}{2.423381in}}%
\pgfpathlineto{\pgfqpoint{4.708322in}{2.357995in}}%
\pgfpathlineto{\pgfqpoint{4.710859in}{2.358025in}}%
\pgfpathlineto{\pgfqpoint{4.711705in}{2.553730in}}%
\pgfpathlineto{\pgfqpoint{4.712550in}{2.292897in}}%
\pgfpathlineto{\pgfqpoint{4.713396in}{2.554425in}}%
\pgfpathlineto{\pgfqpoint{4.715933in}{1.770450in}}%
\pgfpathlineto{\pgfqpoint{4.718470in}{2.280216in}}%
\pgfpathlineto{\pgfqpoint{4.720161in}{2.227260in}}%
\pgfpathlineto{\pgfqpoint{4.721006in}{2.554096in}}%
\pgfpathlineto{\pgfqpoint{4.721852in}{2.358034in}}%
\pgfpathlineto{\pgfqpoint{4.722698in}{1.966246in}}%
\pgfpathlineto{\pgfqpoint{4.724389in}{2.488526in}}%
\pgfpathlineto{\pgfqpoint{4.725235in}{1.966774in}}%
\pgfpathlineto{\pgfqpoint{4.726080in}{2.189132in}}%
\pgfpathlineto{\pgfqpoint{4.726926in}{2.435555in}}%
\pgfpathlineto{\pgfqpoint{4.727771in}{1.966395in}}%
\pgfpathlineto{\pgfqpoint{4.728617in}{2.031999in}}%
\pgfpathlineto{\pgfqpoint{4.729463in}{1.899981in}}%
\pgfpathlineto{\pgfqpoint{4.731154in}{2.426866in}}%
\pgfpathlineto{\pgfqpoint{4.732000in}{2.162203in}}%
\pgfpathlineto{\pgfqpoint{4.733691in}{2.473139in}}%
\pgfpathlineto{\pgfqpoint{4.736228in}{2.116429in}}%
\pgfpathlineto{\pgfqpoint{4.737073in}{2.115605in}}%
\pgfpathlineto{\pgfqpoint{4.737919in}{1.906626in}}%
\pgfpathlineto{\pgfqpoint{4.740456in}{2.520574in}}%
\pgfpathlineto{\pgfqpoint{4.742147in}{2.009755in}}%
\pgfpathlineto{\pgfqpoint{4.742993in}{2.155046in}}%
\pgfpathlineto{\pgfqpoint{4.744684in}{2.457754in}}%
\pgfpathlineto{\pgfqpoint{4.745530in}{2.398291in}}%
\pgfpathlineto{\pgfqpoint{4.746375in}{2.325220in}}%
\pgfpathlineto{\pgfqpoint{4.747221in}{1.933175in}}%
\pgfpathlineto{\pgfqpoint{4.748066in}{2.586795in}}%
\pgfpathlineto{\pgfqpoint{4.749758in}{1.606304in}}%
\pgfpathlineto{\pgfqpoint{4.750603in}{2.586775in}}%
\pgfpathlineto{\pgfqpoint{4.751449in}{2.129249in}}%
\pgfpathlineto{\pgfqpoint{4.752295in}{2.390723in}}%
\pgfpathlineto{\pgfqpoint{4.753140in}{2.063862in}}%
\pgfpathlineto{\pgfqpoint{4.753986in}{2.129222in}}%
\pgfpathlineto{\pgfqpoint{4.754831in}{2.848236in}}%
\pgfpathlineto{\pgfqpoint{4.755677in}{2.063834in}}%
\pgfpathlineto{\pgfqpoint{4.756523in}{2.521512in}}%
\pgfpathlineto{\pgfqpoint{4.757368in}{2.456086in}}%
\pgfpathlineto{\pgfqpoint{4.759905in}{1.737243in}}%
\pgfpathlineto{\pgfqpoint{4.760751in}{1.933134in}}%
\pgfpathlineto{\pgfqpoint{4.762442in}{2.435991in}}%
\pgfpathlineto{\pgfqpoint{4.764979in}{1.933035in}}%
\pgfpathlineto{\pgfqpoint{4.765825in}{2.782939in}}%
\pgfpathlineto{\pgfqpoint{4.766670in}{2.129201in}}%
\pgfpathlineto{\pgfqpoint{4.768361in}{2.259871in}}%
\pgfpathlineto{\pgfqpoint{4.770053in}{2.521074in}}%
\pgfpathlineto{\pgfqpoint{4.770898in}{1.736223in}}%
\pgfpathlineto{\pgfqpoint{4.771744in}{2.424625in}}%
\pgfpathlineto{\pgfqpoint{4.772590in}{2.357960in}}%
\pgfpathlineto{\pgfqpoint{4.773435in}{1.936170in}}%
\pgfpathlineto{\pgfqpoint{4.774281in}{2.487581in}}%
\pgfpathlineto{\pgfqpoint{4.775126in}{2.260013in}}%
\pgfpathlineto{\pgfqpoint{4.775972in}{2.015110in}}%
\pgfpathlineto{\pgfqpoint{4.776818in}{2.504697in}}%
\pgfpathlineto{\pgfqpoint{4.777663in}{2.345476in}}%
\pgfpathlineto{\pgfqpoint{4.778509in}{1.704624in}}%
\pgfpathlineto{\pgfqpoint{4.779355in}{2.147250in}}%
\pgfpathlineto{\pgfqpoint{4.780200in}{2.194602in}}%
\pgfpathlineto{\pgfqpoint{4.781046in}{1.802440in}}%
\pgfpathlineto{\pgfqpoint{4.782737in}{2.259968in}}%
\pgfpathlineto{\pgfqpoint{4.783583in}{1.802368in}}%
\pgfpathlineto{\pgfqpoint{4.786119in}{2.586791in}}%
\pgfpathlineto{\pgfqpoint{4.786965in}{3.044415in}}%
\pgfpathlineto{\pgfqpoint{4.789502in}{2.325339in}}%
\pgfpathlineto{\pgfqpoint{4.792039in}{2.063857in}}%
\pgfpathlineto{\pgfqpoint{4.792884in}{2.586799in}}%
\pgfpathlineto{\pgfqpoint{4.793730in}{2.194544in}}%
\pgfpathlineto{\pgfqpoint{4.794576in}{2.259970in}}%
\pgfpathlineto{\pgfqpoint{4.795421in}{2.064099in}}%
\pgfpathlineto{\pgfqpoint{4.797113in}{2.456381in}}%
\pgfpathlineto{\pgfqpoint{4.797958in}{2.063857in}}%
\pgfpathlineto{\pgfqpoint{4.798804in}{2.259962in}}%
\pgfpathlineto{\pgfqpoint{4.800495in}{2.194605in}}%
\pgfpathlineto{\pgfqpoint{4.802186in}{2.259965in}}%
\pgfpathlineto{\pgfqpoint{4.803032in}{2.259957in}}%
\pgfpathlineto{\pgfqpoint{4.803878in}{2.390699in}}%
\pgfpathlineto{\pgfqpoint{4.808106in}{1.737009in}}%
\pgfpathlineto{\pgfqpoint{4.808951in}{2.194550in}}%
\pgfpathlineto{\pgfqpoint{4.809797in}{2.129341in}}%
\pgfpathlineto{\pgfqpoint{4.812334in}{1.933094in}}%
\pgfpathlineto{\pgfqpoint{4.815716in}{2.717566in}}%
\pgfpathlineto{\pgfqpoint{4.819099in}{1.933114in}}%
\pgfpathlineto{\pgfqpoint{4.819944in}{2.521428in}}%
\pgfpathlineto{\pgfqpoint{4.820790in}{2.390705in}}%
\pgfpathlineto{\pgfqpoint{4.821636in}{2.065105in}}%
\pgfpathlineto{\pgfqpoint{4.822481in}{2.325391in}}%
\pgfpathlineto{\pgfqpoint{4.823327in}{2.129222in}}%
\pgfpathlineto{\pgfqpoint{4.824173in}{2.455672in}}%
\pgfpathlineto{\pgfqpoint{4.825018in}{2.194587in}}%
\pgfpathlineto{\pgfqpoint{4.825864in}{2.260266in}}%
\pgfpathlineto{\pgfqpoint{4.826709in}{2.259990in}}%
\pgfpathlineto{\pgfqpoint{4.827555in}{2.259671in}}%
\pgfpathlineto{\pgfqpoint{4.830092in}{2.063949in}}%
\pgfpathlineto{\pgfqpoint{4.831783in}{2.390697in}}%
\pgfpathlineto{\pgfqpoint{4.833474in}{1.933187in}}%
\pgfpathlineto{\pgfqpoint{4.834320in}{2.390462in}}%
\pgfpathlineto{\pgfqpoint{4.835166in}{1.804246in}}%
\pgfpathlineto{\pgfqpoint{4.836011in}{1.933111in}}%
\pgfpathlineto{\pgfqpoint{4.837703in}{2.321623in}}%
\pgfpathlineto{\pgfqpoint{4.838548in}{2.651979in}}%
\pgfpathlineto{\pgfqpoint{4.840239in}{1.998509in}}%
\pgfpathlineto{\pgfqpoint{4.841931in}{2.195134in}}%
\pgfpathlineto{\pgfqpoint{4.842776in}{2.062131in}}%
\pgfpathlineto{\pgfqpoint{4.844468in}{2.456064in}}%
\pgfpathlineto{\pgfqpoint{4.845313in}{2.259718in}}%
\pgfpathlineto{\pgfqpoint{4.846159in}{2.390732in}}%
\pgfpathlineto{\pgfqpoint{4.847004in}{2.390760in}}%
\pgfpathlineto{\pgfqpoint{4.847850in}{2.260370in}}%
\pgfpathlineto{\pgfqpoint{4.848696in}{2.456728in}}%
\pgfpathlineto{\pgfqpoint{4.849541in}{1.801752in}}%
\pgfpathlineto{\pgfqpoint{4.850387in}{2.324722in}}%
\pgfpathlineto{\pgfqpoint{4.851233in}{2.715644in}}%
\pgfpathlineto{\pgfqpoint{4.853769in}{2.135522in}}%
\pgfpathlineto{\pgfqpoint{4.854615in}{2.341106in}}%
\pgfpathlineto{\pgfqpoint{4.855461in}{2.185142in}}%
\pgfpathlineto{\pgfqpoint{4.856306in}{2.129244in}}%
\pgfpathlineto{\pgfqpoint{4.857152in}{2.455663in}}%
\pgfpathlineto{\pgfqpoint{4.857998in}{2.063417in}}%
\pgfpathlineto{\pgfqpoint{4.858843in}{2.393082in}}%
\pgfpathlineto{\pgfqpoint{4.859689in}{2.324739in}}%
\pgfpathlineto{\pgfqpoint{4.861380in}{2.063910in}}%
\pgfpathlineto{\pgfqpoint{4.862226in}{2.325364in}}%
\pgfpathlineto{\pgfqpoint{4.863071in}{1.998457in}}%
\pgfpathlineto{\pgfqpoint{4.863917in}{2.128920in}}%
\pgfpathlineto{\pgfqpoint{4.864762in}{2.586194in}}%
\pgfpathlineto{\pgfqpoint{4.865608in}{1.933128in}}%
\pgfpathlineto{\pgfqpoint{4.866454in}{1.935086in}}%
\pgfpathlineto{\pgfqpoint{4.868145in}{2.185381in}}%
\pgfpathlineto{\pgfqpoint{4.868991in}{2.129245in}}%
\pgfpathlineto{\pgfqpoint{4.869836in}{2.128682in}}%
\pgfpathlineto{\pgfqpoint{4.870682in}{2.191332in}}%
\pgfpathlineto{\pgfqpoint{4.871527in}{1.923328in}}%
\pgfpathlineto{\pgfqpoint{4.872373in}{1.930045in}}%
\pgfpathlineto{\pgfqpoint{4.873219in}{2.337517in}}%
\pgfpathlineto{\pgfqpoint{4.874064in}{2.195138in}}%
\pgfpathlineto{\pgfqpoint{4.874910in}{1.879901in}}%
\pgfpathlineto{\pgfqpoint{4.875756in}{1.965534in}}%
\pgfpathlineto{\pgfqpoint{4.878292in}{2.554359in}}%
\pgfpathlineto{\pgfqpoint{4.879138in}{2.131626in}}%
\pgfpathlineto{\pgfqpoint{4.879984in}{2.652913in}}%
\pgfpathlineto{\pgfqpoint{4.880829in}{2.463226in}}%
\pgfpathlineto{\pgfqpoint{4.883366in}{1.879553in}}%
\pgfpathlineto{\pgfqpoint{4.885903in}{2.586604in}}%
\pgfpathlineto{\pgfqpoint{4.886749in}{2.260239in}}%
\pgfpathlineto{\pgfqpoint{4.887594in}{2.390653in}}%
\pgfpathlineto{\pgfqpoint{4.890131in}{2.129241in}}%
\pgfpathlineto{\pgfqpoint{4.890977in}{2.456077in}}%
\pgfpathlineto{\pgfqpoint{4.891822in}{2.259973in}}%
\pgfpathlineto{\pgfqpoint{4.892668in}{1.736997in}}%
\pgfpathlineto{\pgfqpoint{4.893514in}{2.259957in}}%
\pgfpathlineto{\pgfqpoint{4.894359in}{1.867769in}}%
\pgfpathlineto{\pgfqpoint{4.896896in}{2.390748in}}%
\pgfpathlineto{\pgfqpoint{4.898587in}{1.802363in}}%
\pgfpathlineto{\pgfqpoint{4.899433in}{2.325341in}}%
\pgfpathlineto{\pgfqpoint{4.900279in}{2.129229in}}%
\pgfpathlineto{\pgfqpoint{4.901124in}{1.802371in}}%
\pgfpathlineto{\pgfqpoint{4.902816in}{2.259958in}}%
\pgfpathlineto{\pgfqpoint{4.903661in}{2.194357in}}%
\pgfpathlineto{\pgfqpoint{4.904507in}{2.194593in}}%
\pgfpathlineto{\pgfqpoint{4.905352in}{1.737010in}}%
\pgfpathlineto{\pgfqpoint{4.906198in}{2.063865in}}%
\pgfpathlineto{\pgfqpoint{4.907889in}{2.456078in}}%
\pgfpathlineto{\pgfqpoint{4.909581in}{1.736998in}}%
\pgfpathlineto{\pgfqpoint{4.910426in}{2.456081in}}%
\pgfpathlineto{\pgfqpoint{4.911272in}{2.325222in}}%
\pgfpathlineto{\pgfqpoint{4.912117in}{2.194613in}}%
\pgfpathlineto{\pgfqpoint{4.912963in}{2.586803in}}%
\pgfpathlineto{\pgfqpoint{4.913809in}{2.325374in}}%
\pgfpathlineto{\pgfqpoint{4.914654in}{2.195044in}}%
\pgfpathlineto{\pgfqpoint{4.915500in}{2.521447in}}%
\pgfpathlineto{\pgfqpoint{4.917191in}{1.998491in}}%
\pgfpathlineto{\pgfqpoint{4.918882in}{2.129230in}}%
\pgfpathlineto{\pgfqpoint{4.919728in}{1.998499in}}%
\pgfpathlineto{\pgfqpoint{4.920574in}{2.652195in}}%
\pgfpathlineto{\pgfqpoint{4.921419in}{1.802375in}}%
\pgfpathlineto{\pgfqpoint{4.922265in}{2.129252in}}%
\pgfpathlineto{\pgfqpoint{4.923111in}{2.456067in}}%
\pgfpathlineto{\pgfqpoint{4.923956in}{1.736999in}}%
\pgfpathlineto{\pgfqpoint{4.924802in}{2.064585in}}%
\pgfpathlineto{\pgfqpoint{4.927339in}{2.521478in}}%
\pgfpathlineto{\pgfqpoint{4.929876in}{2.063856in}}%
\pgfpathlineto{\pgfqpoint{4.930721in}{1.998482in}}%
\pgfpathlineto{\pgfqpoint{4.931567in}{2.652131in}}%
\pgfpathlineto{\pgfqpoint{4.932412in}{2.325338in}}%
\pgfpathlineto{\pgfqpoint{4.933258in}{1.802432in}}%
\pgfpathlineto{\pgfqpoint{4.934104in}{1.998532in}}%
\pgfpathlineto{\pgfqpoint{4.935795in}{2.259967in}}%
\pgfpathlineto{\pgfqpoint{4.936641in}{1.998473in}}%
\pgfpathlineto{\pgfqpoint{4.937486in}{2.063832in}}%
\pgfpathlineto{\pgfqpoint{4.938332in}{2.652194in}}%
\pgfpathlineto{\pgfqpoint{4.939177in}{2.259969in}}%
\pgfpathlineto{\pgfqpoint{4.940023in}{2.129209in}}%
\pgfpathlineto{\pgfqpoint{4.940869in}{2.194636in}}%
\pgfpathlineto{\pgfqpoint{4.941714in}{2.456081in}}%
\pgfpathlineto{\pgfqpoint{4.942560in}{1.802375in}}%
\pgfpathlineto{\pgfqpoint{4.943405in}{2.194596in}}%
\pgfpathlineto{\pgfqpoint{4.944251in}{2.390730in}}%
\pgfpathlineto{\pgfqpoint{4.946788in}{1.671628in}}%
\pgfpathlineto{\pgfqpoint{4.948479in}{2.063848in}}%
\pgfpathlineto{\pgfqpoint{4.949325in}{1.802408in}}%
\pgfpathlineto{\pgfqpoint{4.950170in}{2.586835in}}%
\pgfpathlineto{\pgfqpoint{4.951016in}{2.325335in}}%
\pgfpathlineto{\pgfqpoint{4.951862in}{2.194602in}}%
\pgfpathlineto{\pgfqpoint{4.952707in}{2.325353in}}%
\pgfpathlineto{\pgfqpoint{4.953553in}{1.998478in}}%
\pgfpathlineto{\pgfqpoint{4.955244in}{2.586823in}}%
\pgfpathlineto{\pgfqpoint{4.957781in}{1.867907in}}%
\pgfpathlineto{\pgfqpoint{4.959472in}{2.390554in}}%
\pgfpathlineto{\pgfqpoint{4.960318in}{2.194596in}}%
\pgfpathlineto{\pgfqpoint{4.961164in}{2.194618in}}%
\pgfpathlineto{\pgfqpoint{4.962855in}{1.998505in}}%
\pgfpathlineto{\pgfqpoint{4.964546in}{2.586828in}}%
\pgfpathlineto{\pgfqpoint{4.965392in}{2.129132in}}%
\pgfpathlineto{\pgfqpoint{4.966237in}{2.390727in}}%
\pgfpathlineto{\pgfqpoint{4.967083in}{2.390728in}}%
\pgfpathlineto{\pgfqpoint{4.969620in}{1.737002in}}%
\pgfpathlineto{\pgfqpoint{4.971311in}{2.259973in}}%
\pgfpathlineto{\pgfqpoint{4.972157in}{2.259802in}}%
\pgfpathlineto{\pgfqpoint{4.973002in}{1.671632in}}%
\pgfpathlineto{\pgfqpoint{4.973848in}{2.652188in}}%
\pgfpathlineto{\pgfqpoint{4.974694in}{1.737001in}}%
\pgfpathlineto{\pgfqpoint{4.975539in}{2.259985in}}%
\pgfpathlineto{\pgfqpoint{4.976385in}{2.652115in}}%
\pgfpathlineto{\pgfqpoint{4.977230in}{2.390710in}}%
\pgfpathlineto{\pgfqpoint{4.978076in}{2.194546in}}%
\pgfpathlineto{\pgfqpoint{4.978922in}{2.586843in}}%
\pgfpathlineto{\pgfqpoint{4.979767in}{2.390702in}}%
\pgfpathlineto{\pgfqpoint{4.980613in}{1.997188in}}%
\pgfpathlineto{\pgfqpoint{4.981459in}{2.129287in}}%
\pgfpathlineto{\pgfqpoint{4.982304in}{2.652198in}}%
\pgfpathlineto{\pgfqpoint{4.983150in}{1.933106in}}%
\pgfpathlineto{\pgfqpoint{4.983995in}{2.325494in}}%
\pgfpathlineto{\pgfqpoint{4.984841in}{2.194594in}}%
\pgfpathlineto{\pgfqpoint{4.985687in}{2.194612in}}%
\pgfpathlineto{\pgfqpoint{4.986532in}{2.325342in}}%
\pgfpathlineto{\pgfqpoint{4.989069in}{1.802380in}}%
\pgfpathlineto{\pgfqpoint{4.991606in}{2.325376in}}%
\pgfpathlineto{\pgfqpoint{4.993297in}{2.325334in}}%
\pgfpathlineto{\pgfqpoint{4.995834in}{1.867746in}}%
\pgfpathlineto{\pgfqpoint{4.996680in}{1.998483in}}%
\pgfpathlineto{\pgfqpoint{4.997525in}{2.194625in}}%
\pgfpathlineto{\pgfqpoint{4.998371in}{2.913701in}}%
\pgfpathlineto{\pgfqpoint{4.999217in}{2.456078in}}%
\pgfpathlineto{\pgfqpoint{5.000062in}{2.129194in}}%
\pgfpathlineto{\pgfqpoint{5.000908in}{2.390867in}}%
\pgfpathlineto{\pgfqpoint{5.001754in}{2.456466in}}%
\pgfpathlineto{\pgfqpoint{5.002599in}{2.063792in}}%
\pgfpathlineto{\pgfqpoint{5.003445in}{2.390710in}}%
\pgfpathlineto{\pgfqpoint{5.004290in}{2.194593in}}%
\pgfpathlineto{\pgfqpoint{5.005136in}{2.259841in}}%
\pgfpathlineto{\pgfqpoint{5.006827in}{1.998488in}}%
\pgfpathlineto{\pgfqpoint{5.008519in}{2.521451in}}%
\pgfpathlineto{\pgfqpoint{5.009364in}{2.129215in}}%
\pgfpathlineto{\pgfqpoint{5.010210in}{2.259909in}}%
\pgfpathlineto{\pgfqpoint{5.011901in}{2.063853in}}%
\pgfpathlineto{\pgfqpoint{5.012747in}{2.653875in}}%
\pgfpathlineto{\pgfqpoint{5.013592in}{2.390718in}}%
\pgfpathlineto{\pgfqpoint{5.015284in}{2.063854in}}%
\pgfpathlineto{\pgfqpoint{5.016129in}{2.325335in}}%
\pgfpathlineto{\pgfqpoint{5.016975in}{2.259967in}}%
\pgfpathlineto{\pgfqpoint{5.017820in}{2.259969in}}%
\pgfpathlineto{\pgfqpoint{5.018666in}{2.194615in}}%
\pgfpathlineto{\pgfqpoint{5.019512in}{2.521450in}}%
\pgfpathlineto{\pgfqpoint{5.020357in}{2.063851in}}%
\pgfpathlineto{\pgfqpoint{5.021203in}{2.194599in}}%
\pgfpathlineto{\pgfqpoint{5.022049in}{2.063549in}}%
\pgfpathlineto{\pgfqpoint{5.022894in}{2.259970in}}%
\pgfpathlineto{\pgfqpoint{5.025431in}{1.738040in}}%
\pgfpathlineto{\pgfqpoint{5.026277in}{2.259944in}}%
\pgfpathlineto{\pgfqpoint{5.027122in}{2.194559in}}%
\pgfpathlineto{\pgfqpoint{5.027968in}{1.802371in}}%
\pgfpathlineto{\pgfqpoint{5.029659in}{2.521290in}}%
\pgfpathlineto{\pgfqpoint{5.030505in}{1.867742in}}%
\pgfpathlineto{\pgfqpoint{5.031350in}{2.194591in}}%
\pgfpathlineto{\pgfqpoint{5.032196in}{1.867743in}}%
\pgfpathlineto{\pgfqpoint{5.033042in}{2.717565in}}%
\pgfpathlineto{\pgfqpoint{5.033887in}{2.652187in}}%
\pgfpathlineto{\pgfqpoint{5.034733in}{2.129219in}}%
\pgfpathlineto{\pgfqpoint{5.035578in}{2.390708in}}%
\pgfpathlineto{\pgfqpoint{5.036424in}{2.456080in}}%
\pgfpathlineto{\pgfqpoint{5.037270in}{2.390711in}}%
\pgfpathlineto{\pgfqpoint{5.038115in}{2.194599in}}%
\pgfpathlineto{\pgfqpoint{5.038961in}{2.194626in}}%
\pgfpathlineto{\pgfqpoint{5.040652in}{2.652201in}}%
\pgfpathlineto{\pgfqpoint{5.041498in}{2.521388in}}%
\pgfpathlineto{\pgfqpoint{5.042343in}{2.390726in}}%
\pgfpathlineto{\pgfqpoint{5.043189in}{1.933112in}}%
\pgfpathlineto{\pgfqpoint{5.044035in}{2.063855in}}%
\pgfpathlineto{\pgfqpoint{5.044880in}{2.194605in}}%
\pgfpathlineto{\pgfqpoint{5.045726in}{1.933145in}}%
\pgfpathlineto{\pgfqpoint{5.046572in}{2.063856in}}%
\pgfpathlineto{\pgfqpoint{5.047417in}{2.586823in}}%
\pgfpathlineto{\pgfqpoint{5.048263in}{2.259970in}}%
\pgfpathlineto{\pgfqpoint{5.049108in}{2.390711in}}%
\pgfpathlineto{\pgfqpoint{5.049954in}{1.933108in}}%
\pgfpathlineto{\pgfqpoint{5.052491in}{2.650901in}}%
\pgfpathlineto{\pgfqpoint{5.053337in}{2.129212in}}%
\pgfpathlineto{\pgfqpoint{5.054182in}{2.390709in}}%
\pgfpathlineto{\pgfqpoint{5.055028in}{2.325440in}}%
\pgfpathlineto{\pgfqpoint{5.055873in}{1.998479in}}%
\pgfpathlineto{\pgfqpoint{5.056719in}{2.063856in}}%
\pgfpathlineto{\pgfqpoint{5.057565in}{2.456078in}}%
\pgfpathlineto{\pgfqpoint{5.058410in}{1.737007in}}%
\pgfpathlineto{\pgfqpoint{5.059256in}{2.129194in}}%
\pgfpathlineto{\pgfqpoint{5.061793in}{1.737003in}}%
\pgfpathlineto{\pgfqpoint{5.062638in}{2.194597in}}%
\pgfpathlineto{\pgfqpoint{5.063484in}{1.933113in}}%
\pgfpathlineto{\pgfqpoint{5.064330in}{2.129239in}}%
\pgfpathlineto{\pgfqpoint{5.065175in}{2.063855in}}%
\pgfpathlineto{\pgfqpoint{5.066021in}{1.867755in}}%
\pgfpathlineto{\pgfqpoint{5.067712in}{2.586822in}}%
\pgfpathlineto{\pgfqpoint{5.068558in}{1.933118in}}%
\pgfpathlineto{\pgfqpoint{5.069403in}{1.998406in}}%
\pgfpathlineto{\pgfqpoint{5.070249in}{2.782925in}}%
\pgfpathlineto{\pgfqpoint{5.071095in}{2.129228in}}%
\pgfpathlineto{\pgfqpoint{5.071940in}{2.456082in}}%
\pgfpathlineto{\pgfqpoint{5.072786in}{2.390673in}}%
\pgfpathlineto{\pgfqpoint{5.073632in}{2.194596in}}%
\pgfpathlineto{\pgfqpoint{5.074477in}{2.325311in}}%
\pgfpathlineto{\pgfqpoint{5.075323in}{2.259968in}}%
\pgfpathlineto{\pgfqpoint{5.077014in}{1.933027in}}%
\pgfpathlineto{\pgfqpoint{5.081242in}{2.194591in}}%
\pgfpathlineto{\pgfqpoint{5.082088in}{2.063855in}}%
\pgfpathlineto{\pgfqpoint{5.082933in}{2.325285in}}%
\pgfpathlineto{\pgfqpoint{5.083779in}{1.933110in}}%
\pgfpathlineto{\pgfqpoint{5.084625in}{2.456080in}}%
\pgfpathlineto{\pgfqpoint{5.085470in}{2.063835in}}%
\pgfpathlineto{\pgfqpoint{5.086316in}{2.129244in}}%
\pgfpathlineto{\pgfqpoint{5.087162in}{1.802371in}}%
\pgfpathlineto{\pgfqpoint{5.088007in}{2.456075in}}%
\pgfpathlineto{\pgfqpoint{5.088853in}{2.259968in}}%
\pgfpathlineto{\pgfqpoint{5.089698in}{2.194579in}}%
\pgfpathlineto{\pgfqpoint{5.090544in}{2.390682in}}%
\pgfpathlineto{\pgfqpoint{5.092235in}{1.672264in}}%
\pgfpathlineto{\pgfqpoint{5.093081in}{1.998444in}}%
\pgfpathlineto{\pgfqpoint{5.093927in}{2.129199in}}%
\pgfpathlineto{\pgfqpoint{5.094772in}{2.782934in}}%
\pgfpathlineto{\pgfqpoint{5.095618in}{2.652211in}}%
\pgfpathlineto{\pgfqpoint{5.097309in}{2.063903in}}%
\pgfpathlineto{\pgfqpoint{5.098155in}{2.652261in}}%
\pgfpathlineto{\pgfqpoint{5.099846in}{1.737001in}}%
\pgfpathlineto{\pgfqpoint{5.100692in}{2.652186in}}%
\pgfpathlineto{\pgfqpoint{5.101537in}{2.390710in}}%
\pgfpathlineto{\pgfqpoint{5.103228in}{1.933117in}}%
\pgfpathlineto{\pgfqpoint{5.105765in}{2.456005in}}%
\pgfpathlineto{\pgfqpoint{5.108302in}{2.063851in}}%
\pgfpathlineto{\pgfqpoint{5.109148in}{2.716819in}}%
\pgfpathlineto{\pgfqpoint{5.109993in}{2.390695in}}%
\pgfpathlineto{\pgfqpoint{5.110839in}{2.194597in}}%
\pgfpathlineto{\pgfqpoint{5.111685in}{2.259968in}}%
\pgfpathlineto{\pgfqpoint{5.112530in}{2.325337in}}%
\pgfpathlineto{\pgfqpoint{5.113376in}{2.194570in}}%
\pgfpathlineto{\pgfqpoint{5.114221in}{1.737741in}}%
\pgfpathlineto{\pgfqpoint{5.115913in}{2.129220in}}%
\pgfpathlineto{\pgfqpoint{5.116758in}{1.867798in}}%
\pgfpathlineto{\pgfqpoint{5.118450in}{2.130803in}}%
\pgfpathlineto{\pgfqpoint{5.119295in}{1.896815in}}%
\pgfpathlineto{\pgfqpoint{5.120141in}{2.063767in}}%
\pgfpathlineto{\pgfqpoint{5.120986in}{2.521547in}}%
\pgfpathlineto{\pgfqpoint{5.121832in}{2.390695in}}%
\pgfpathlineto{\pgfqpoint{5.122678in}{2.521450in}}%
\pgfpathlineto{\pgfqpoint{5.123523in}{2.194637in}}%
\pgfpathlineto{\pgfqpoint{5.125215in}{2.717564in}}%
\pgfpathlineto{\pgfqpoint{5.126906in}{1.933112in}}%
\pgfpathlineto{\pgfqpoint{5.128597in}{2.521430in}}%
\pgfpathlineto{\pgfqpoint{5.129443in}{1.737005in}}%
\pgfpathlineto{\pgfqpoint{5.130288in}{2.260001in}}%
\pgfpathlineto{\pgfqpoint{5.131134in}{1.802372in}}%
\pgfpathlineto{\pgfqpoint{5.131980in}{1.867742in}}%
\pgfpathlineto{\pgfqpoint{5.134516in}{2.717562in}}%
\pgfpathlineto{\pgfqpoint{5.135362in}{1.606257in}}%
\pgfpathlineto{\pgfqpoint{5.136208in}{2.521441in}}%
\pgfpathlineto{\pgfqpoint{5.137053in}{1.671658in}}%
\pgfpathlineto{\pgfqpoint{5.137899in}{2.521384in}}%
\pgfpathlineto{\pgfqpoint{5.138745in}{2.129212in}}%
\pgfpathlineto{\pgfqpoint{5.139590in}{2.063854in}}%
\pgfpathlineto{\pgfqpoint{5.140436in}{2.129525in}}%
\pgfpathlineto{\pgfqpoint{5.141281in}{2.325338in}}%
\pgfpathlineto{\pgfqpoint{5.142127in}{2.063854in}}%
\pgfpathlineto{\pgfqpoint{5.142973in}{2.717526in}}%
\pgfpathlineto{\pgfqpoint{5.143818in}{2.259949in}}%
\pgfpathlineto{\pgfqpoint{5.144664in}{1.867741in}}%
\pgfpathlineto{\pgfqpoint{5.146355in}{2.521466in}}%
\pgfpathlineto{\pgfqpoint{5.148046in}{1.998491in}}%
\pgfpathlineto{\pgfqpoint{5.148892in}{2.325343in}}%
\pgfpathlineto{\pgfqpoint{5.149738in}{2.129226in}}%
\pgfpathlineto{\pgfqpoint{5.151429in}{2.390707in}}%
\pgfpathlineto{\pgfqpoint{5.152275in}{1.998476in}}%
\pgfpathlineto{\pgfqpoint{5.153120in}{2.325338in}}%
\pgfpathlineto{\pgfqpoint{5.153966in}{2.129237in}}%
\pgfpathlineto{\pgfqpoint{5.154811in}{2.194596in}}%
\pgfpathlineto{\pgfqpoint{5.155657in}{1.998485in}}%
\pgfpathlineto{\pgfqpoint{5.158194in}{2.259965in}}%
\pgfpathlineto{\pgfqpoint{5.160731in}{1.867745in}}%
\pgfpathlineto{\pgfqpoint{5.162422in}{2.259964in}}%
\pgfpathlineto{\pgfqpoint{5.163268in}{2.194592in}}%
\pgfpathlineto{\pgfqpoint{5.164113in}{1.933112in}}%
\pgfpathlineto{\pgfqpoint{5.164959in}{2.063911in}}%
\pgfpathlineto{\pgfqpoint{5.165805in}{2.129227in}}%
\pgfpathlineto{\pgfqpoint{5.167496in}{2.586821in}}%
\pgfpathlineto{\pgfqpoint{5.169187in}{2.129225in}}%
\pgfpathlineto{\pgfqpoint{5.170033in}{2.325338in}}%
\pgfpathlineto{\pgfqpoint{5.170878in}{1.737051in}}%
\pgfpathlineto{\pgfqpoint{5.171724in}{2.194925in}}%
\pgfpathlineto{\pgfqpoint{5.172570in}{2.194599in}}%
\pgfpathlineto{\pgfqpoint{5.173415in}{2.129216in}}%
\pgfpathlineto{\pgfqpoint{5.175106in}{2.260105in}}%
\pgfpathlineto{\pgfqpoint{5.175952in}{2.259966in}}%
\pgfpathlineto{\pgfqpoint{5.177643in}{1.998482in}}%
\pgfpathlineto{\pgfqpoint{5.178489in}{2.521419in}}%
\pgfpathlineto{\pgfqpoint{5.179335in}{2.194575in}}%
\pgfpathlineto{\pgfqpoint{5.180180in}{1.933097in}}%
\pgfpathlineto{\pgfqpoint{5.181026in}{2.129225in}}%
\pgfpathlineto{\pgfqpoint{5.181871in}{2.194595in}}%
\pgfpathlineto{\pgfqpoint{5.182717in}{1.998484in}}%
\pgfpathlineto{\pgfqpoint{5.183563in}{2.325337in}}%
\pgfpathlineto{\pgfqpoint{5.184408in}{2.259961in}}%
\pgfpathlineto{\pgfqpoint{5.185254in}{2.129227in}}%
\pgfpathlineto{\pgfqpoint{5.187791in}{2.521451in}}%
\pgfpathlineto{\pgfqpoint{5.188636in}{1.802266in}}%
\pgfpathlineto{\pgfqpoint{5.188636in}{1.802266in}}%
\pgfusepath{stroke}%
\end{pgfscope}%
\begin{pgfscope}%
\pgfpathrectangle{\pgfqpoint{0.750000in}{0.500000in}}{\pgfqpoint{4.650000in}{3.020000in}}%
\pgfusepath{clip}%
\pgfsetrectcap%
\pgfsetroundjoin%
\pgfsetlinewidth{1.505625pt}%
\definecolor{currentstroke}{rgb}{0.000000,0.500000,0.000000}%
\pgfsetstrokecolor{currentstroke}%
\pgfsetdash{}{0pt}%
\pgfpathmoveto{\pgfqpoint{0.961364in}{2.229295in}}%
\pgfpathlineto{\pgfqpoint{0.963055in}{2.177994in}}%
\pgfpathlineto{\pgfqpoint{0.964746in}{2.214499in}}%
\pgfpathlineto{\pgfqpoint{0.966437in}{2.191649in}}%
\pgfpathlineto{\pgfqpoint{0.967283in}{2.192411in}}%
\pgfpathlineto{\pgfqpoint{0.968129in}{2.198863in}}%
\pgfpathlineto{\pgfqpoint{0.968974in}{2.179408in}}%
\pgfpathlineto{\pgfqpoint{0.969820in}{2.218841in}}%
\pgfpathlineto{\pgfqpoint{0.970665in}{2.202247in}}%
\pgfpathlineto{\pgfqpoint{0.971511in}{2.180159in}}%
\pgfpathlineto{\pgfqpoint{0.972357in}{2.193619in}}%
\pgfpathlineto{\pgfqpoint{0.973202in}{2.243654in}}%
\pgfpathlineto{\pgfqpoint{0.974048in}{2.159676in}}%
\pgfpathlineto{\pgfqpoint{0.974894in}{2.247315in}}%
\pgfpathlineto{\pgfqpoint{0.975739in}{2.170288in}}%
\pgfpathlineto{\pgfqpoint{0.976585in}{2.180551in}}%
\pgfpathlineto{\pgfqpoint{0.977430in}{2.162617in}}%
\pgfpathlineto{\pgfqpoint{0.978276in}{2.202708in}}%
\pgfpathlineto{\pgfqpoint{0.979122in}{2.169539in}}%
\pgfpathlineto{\pgfqpoint{0.979967in}{2.208752in}}%
\pgfpathlineto{\pgfqpoint{0.980813in}{2.171261in}}%
\pgfpathlineto{\pgfqpoint{0.981659in}{2.226270in}}%
\pgfpathlineto{\pgfqpoint{0.982504in}{2.183675in}}%
\pgfpathlineto{\pgfqpoint{0.983350in}{2.204063in}}%
\pgfpathlineto{\pgfqpoint{0.985041in}{2.175568in}}%
\pgfpathlineto{\pgfqpoint{0.986732in}{2.205002in}}%
\pgfpathlineto{\pgfqpoint{0.988424in}{2.174228in}}%
\pgfpathlineto{\pgfqpoint{0.989269in}{2.222580in}}%
\pgfpathlineto{\pgfqpoint{0.990115in}{2.193372in}}%
\pgfpathlineto{\pgfqpoint{0.990960in}{2.191937in}}%
\pgfpathlineto{\pgfqpoint{0.991806in}{2.154671in}}%
\pgfpathlineto{\pgfqpoint{0.992652in}{2.170147in}}%
\pgfpathlineto{\pgfqpoint{0.995189in}{2.224284in}}%
\pgfpathlineto{\pgfqpoint{0.996034in}{2.162017in}}%
\pgfpathlineto{\pgfqpoint{0.996880in}{2.229071in}}%
\pgfpathlineto{\pgfqpoint{0.997725in}{2.196741in}}%
\pgfpathlineto{\pgfqpoint{0.998571in}{2.151416in}}%
\pgfpathlineto{\pgfqpoint{0.999417in}{2.263603in}}%
\pgfpathlineto{\pgfqpoint{1.000262in}{2.221183in}}%
\pgfpathlineto{\pgfqpoint{1.001108in}{2.183593in}}%
\pgfpathlineto{\pgfqpoint{1.001954in}{2.188596in}}%
\pgfpathlineto{\pgfqpoint{1.002799in}{2.178374in}}%
\pgfpathlineto{\pgfqpoint{1.003645in}{2.219545in}}%
\pgfpathlineto{\pgfqpoint{1.004490in}{2.218526in}}%
\pgfpathlineto{\pgfqpoint{1.005336in}{2.214497in}}%
\pgfpathlineto{\pgfqpoint{1.007027in}{2.226864in}}%
\pgfpathlineto{\pgfqpoint{1.008719in}{2.173197in}}%
\pgfpathlineto{\pgfqpoint{1.010410in}{2.230482in}}%
\pgfpathlineto{\pgfqpoint{1.011255in}{2.220058in}}%
\pgfpathlineto{\pgfqpoint{1.012101in}{2.158038in}}%
\pgfpathlineto{\pgfqpoint{1.012947in}{2.225551in}}%
\pgfpathlineto{\pgfqpoint{1.013792in}{2.183187in}}%
\pgfpathlineto{\pgfqpoint{1.014638in}{2.214582in}}%
\pgfpathlineto{\pgfqpoint{1.015484in}{2.183708in}}%
\pgfpathlineto{\pgfqpoint{1.016329in}{2.208290in}}%
\pgfpathlineto{\pgfqpoint{1.017175in}{2.216368in}}%
\pgfpathlineto{\pgfqpoint{1.018866in}{2.152535in}}%
\pgfpathlineto{\pgfqpoint{1.019712in}{2.210269in}}%
\pgfpathlineto{\pgfqpoint{1.020557in}{2.190454in}}%
\pgfpathlineto{\pgfqpoint{1.021403in}{2.185352in}}%
\pgfpathlineto{\pgfqpoint{1.022249in}{2.236024in}}%
\pgfpathlineto{\pgfqpoint{1.023940in}{2.175095in}}%
\pgfpathlineto{\pgfqpoint{1.024785in}{2.231890in}}%
\pgfpathlineto{\pgfqpoint{1.025631in}{2.196024in}}%
\pgfpathlineto{\pgfqpoint{1.026477in}{2.224165in}}%
\pgfpathlineto{\pgfqpoint{1.027322in}{2.222611in}}%
\pgfpathlineto{\pgfqpoint{1.029859in}{2.161374in}}%
\pgfpathlineto{\pgfqpoint{1.030705in}{2.212312in}}%
\pgfpathlineto{\pgfqpoint{1.031550in}{2.156686in}}%
\pgfpathlineto{\pgfqpoint{1.032396in}{2.189168in}}%
\pgfpathlineto{\pgfqpoint{1.034933in}{2.263655in}}%
\pgfpathlineto{\pgfqpoint{1.035779in}{2.181887in}}%
\pgfpathlineto{\pgfqpoint{1.036624in}{2.206351in}}%
\pgfpathlineto{\pgfqpoint{1.037470in}{2.240599in}}%
\pgfpathlineto{\pgfqpoint{1.040007in}{2.181948in}}%
\pgfpathlineto{\pgfqpoint{1.040852in}{2.217437in}}%
\pgfpathlineto{\pgfqpoint{1.041698in}{2.190934in}}%
\pgfpathlineto{\pgfqpoint{1.042544in}{2.220986in}}%
\pgfpathlineto{\pgfqpoint{1.044235in}{2.163451in}}%
\pgfpathlineto{\pgfqpoint{1.045080in}{2.243643in}}%
\pgfpathlineto{\pgfqpoint{1.045926in}{2.231434in}}%
\pgfpathlineto{\pgfqpoint{1.047617in}{2.165303in}}%
\pgfpathlineto{\pgfqpoint{1.048463in}{2.104649in}}%
\pgfpathlineto{\pgfqpoint{1.049308in}{2.275567in}}%
\pgfpathlineto{\pgfqpoint{1.050154in}{2.135624in}}%
\pgfpathlineto{\pgfqpoint{1.051000in}{2.098983in}}%
\pgfpathlineto{\pgfqpoint{1.051845in}{2.189427in}}%
\pgfpathlineto{\pgfqpoint{1.052691in}{2.180801in}}%
\pgfpathlineto{\pgfqpoint{1.054382in}{2.135337in}}%
\pgfpathlineto{\pgfqpoint{1.056073in}{2.257230in}}%
\pgfpathlineto{\pgfqpoint{1.057765in}{2.132229in}}%
\pgfpathlineto{\pgfqpoint{1.060302in}{2.251201in}}%
\pgfpathlineto{\pgfqpoint{1.061993in}{2.147566in}}%
\pgfpathlineto{\pgfqpoint{1.063684in}{2.213481in}}%
\pgfpathlineto{\pgfqpoint{1.064530in}{2.207112in}}%
\pgfpathlineto{\pgfqpoint{1.065375in}{2.102274in}}%
\pgfpathlineto{\pgfqpoint{1.067912in}{2.248237in}}%
\pgfpathlineto{\pgfqpoint{1.068758in}{2.199538in}}%
\pgfpathlineto{\pgfqpoint{1.069603in}{2.218512in}}%
\pgfpathlineto{\pgfqpoint{1.070449in}{2.215391in}}%
\pgfpathlineto{\pgfqpoint{1.071295in}{2.200176in}}%
\pgfpathlineto{\pgfqpoint{1.072140in}{2.201962in}}%
\pgfpathlineto{\pgfqpoint{1.072986in}{2.130584in}}%
\pgfpathlineto{\pgfqpoint{1.073832in}{2.189094in}}%
\pgfpathlineto{\pgfqpoint{1.074677in}{2.164675in}}%
\pgfpathlineto{\pgfqpoint{1.075523in}{2.242089in}}%
\pgfpathlineto{\pgfqpoint{1.076368in}{2.109607in}}%
\pgfpathlineto{\pgfqpoint{1.077214in}{2.173403in}}%
\pgfpathlineto{\pgfqpoint{1.078060in}{2.144237in}}%
\pgfpathlineto{\pgfqpoint{1.079751in}{2.245019in}}%
\pgfpathlineto{\pgfqpoint{1.080597in}{2.209652in}}%
\pgfpathlineto{\pgfqpoint{1.081442in}{2.200447in}}%
\pgfpathlineto{\pgfqpoint{1.083133in}{2.227518in}}%
\pgfpathlineto{\pgfqpoint{1.084825in}{2.141830in}}%
\pgfpathlineto{\pgfqpoint{1.085670in}{2.183615in}}%
\pgfpathlineto{\pgfqpoint{1.086516in}{2.179992in}}%
\pgfpathlineto{\pgfqpoint{1.087362in}{2.169485in}}%
\pgfpathlineto{\pgfqpoint{1.089053in}{2.257337in}}%
\pgfpathlineto{\pgfqpoint{1.089898in}{2.118299in}}%
\pgfpathlineto{\pgfqpoint{1.090744in}{2.256648in}}%
\pgfpathlineto{\pgfqpoint{1.091590in}{2.118648in}}%
\pgfpathlineto{\pgfqpoint{1.092435in}{2.154672in}}%
\pgfpathlineto{\pgfqpoint{1.094972in}{2.220979in}}%
\pgfpathlineto{\pgfqpoint{1.096663in}{2.122583in}}%
\pgfpathlineto{\pgfqpoint{1.097509in}{2.266689in}}%
\pgfpathlineto{\pgfqpoint{1.098355in}{2.243114in}}%
\pgfpathlineto{\pgfqpoint{1.100046in}{2.156288in}}%
\pgfpathlineto{\pgfqpoint{1.100892in}{2.322447in}}%
\pgfpathlineto{\pgfqpoint{1.102583in}{2.094052in}}%
\pgfpathlineto{\pgfqpoint{1.105120in}{2.184156in}}%
\pgfpathlineto{\pgfqpoint{1.105965in}{2.196009in}}%
\pgfpathlineto{\pgfqpoint{1.106811in}{2.034189in}}%
\pgfpathlineto{\pgfqpoint{1.107657in}{2.224500in}}%
\pgfpathlineto{\pgfqpoint{1.108502in}{2.205799in}}%
\pgfpathlineto{\pgfqpoint{1.109348in}{2.111574in}}%
\pgfpathlineto{\pgfqpoint{1.110193in}{2.137293in}}%
\pgfpathlineto{\pgfqpoint{1.111039in}{2.164838in}}%
\pgfpathlineto{\pgfqpoint{1.111885in}{2.260466in}}%
\pgfpathlineto{\pgfqpoint{1.112730in}{2.246285in}}%
\pgfpathlineto{\pgfqpoint{1.113576in}{2.135841in}}%
\pgfpathlineto{\pgfqpoint{1.114422in}{2.169382in}}%
\pgfpathlineto{\pgfqpoint{1.115267in}{2.243569in}}%
\pgfpathlineto{\pgfqpoint{1.116113in}{2.162728in}}%
\pgfpathlineto{\pgfqpoint{1.116958in}{2.184891in}}%
\pgfpathlineto{\pgfqpoint{1.117804in}{2.152415in}}%
\pgfpathlineto{\pgfqpoint{1.119495in}{2.223249in}}%
\pgfpathlineto{\pgfqpoint{1.120341in}{2.169059in}}%
\pgfpathlineto{\pgfqpoint{1.121187in}{2.210116in}}%
\pgfpathlineto{\pgfqpoint{1.122032in}{2.221051in}}%
\pgfpathlineto{\pgfqpoint{1.122878in}{2.173343in}}%
\pgfpathlineto{\pgfqpoint{1.123723in}{2.306207in}}%
\pgfpathlineto{\pgfqpoint{1.124569in}{2.276817in}}%
\pgfpathlineto{\pgfqpoint{1.125415in}{2.269969in}}%
\pgfpathlineto{\pgfqpoint{1.126260in}{2.296280in}}%
\pgfpathlineto{\pgfqpoint{1.127951in}{2.138428in}}%
\pgfpathlineto{\pgfqpoint{1.129643in}{2.243841in}}%
\pgfpathlineto{\pgfqpoint{1.131334in}{2.109617in}}%
\pgfpathlineto{\pgfqpoint{1.132180in}{2.168151in}}%
\pgfpathlineto{\pgfqpoint{1.133871in}{2.339209in}}%
\pgfpathlineto{\pgfqpoint{1.134716in}{2.043730in}}%
\pgfpathlineto{\pgfqpoint{1.135562in}{2.084756in}}%
\pgfpathlineto{\pgfqpoint{1.136408in}{2.258116in}}%
\pgfpathlineto{\pgfqpoint{1.137253in}{2.129966in}}%
\pgfpathlineto{\pgfqpoint{1.139790in}{2.212507in}}%
\pgfpathlineto{\pgfqpoint{1.140636in}{2.203260in}}%
\pgfpathlineto{\pgfqpoint{1.141481in}{2.034063in}}%
\pgfpathlineto{\pgfqpoint{1.142327in}{2.369274in}}%
\pgfpathlineto{\pgfqpoint{1.143173in}{2.196376in}}%
\pgfpathlineto{\pgfqpoint{1.144018in}{2.098414in}}%
\pgfpathlineto{\pgfqpoint{1.144864in}{2.234718in}}%
\pgfpathlineto{\pgfqpoint{1.145710in}{2.185910in}}%
\pgfpathlineto{\pgfqpoint{1.146555in}{2.149588in}}%
\pgfpathlineto{\pgfqpoint{1.147401in}{2.253176in}}%
\pgfpathlineto{\pgfqpoint{1.148246in}{2.248787in}}%
\pgfpathlineto{\pgfqpoint{1.149092in}{2.133326in}}%
\pgfpathlineto{\pgfqpoint{1.149938in}{2.200748in}}%
\pgfpathlineto{\pgfqpoint{1.150783in}{2.189277in}}%
\pgfpathlineto{\pgfqpoint{1.151629in}{2.224249in}}%
\pgfpathlineto{\pgfqpoint{1.152475in}{2.110551in}}%
\pgfpathlineto{\pgfqpoint{1.153320in}{2.170739in}}%
\pgfpathlineto{\pgfqpoint{1.154166in}{2.158195in}}%
\pgfpathlineto{\pgfqpoint{1.155857in}{2.014468in}}%
\pgfpathlineto{\pgfqpoint{1.157548in}{2.374493in}}%
\pgfpathlineto{\pgfqpoint{1.158394in}{2.064658in}}%
\pgfpathlineto{\pgfqpoint{1.159240in}{2.251950in}}%
\pgfpathlineto{\pgfqpoint{1.160931in}{2.123049in}}%
\pgfpathlineto{\pgfqpoint{1.161776in}{2.232534in}}%
\pgfpathlineto{\pgfqpoint{1.162622in}{2.229212in}}%
\pgfpathlineto{\pgfqpoint{1.164313in}{2.210704in}}%
\pgfpathlineto{\pgfqpoint{1.166005in}{2.313183in}}%
\pgfpathlineto{\pgfqpoint{1.166850in}{2.067805in}}%
\pgfpathlineto{\pgfqpoint{1.167696in}{2.339876in}}%
\pgfpathlineto{\pgfqpoint{1.168541in}{2.116556in}}%
\pgfpathlineto{\pgfqpoint{1.169387in}{2.125142in}}%
\pgfpathlineto{\pgfqpoint{1.170233in}{2.190231in}}%
\pgfpathlineto{\pgfqpoint{1.171078in}{2.103561in}}%
\pgfpathlineto{\pgfqpoint{1.171924in}{2.144968in}}%
\pgfpathlineto{\pgfqpoint{1.172770in}{2.213469in}}%
\pgfpathlineto{\pgfqpoint{1.174461in}{2.072062in}}%
\pgfpathlineto{\pgfqpoint{1.175306in}{2.327653in}}%
\pgfpathlineto{\pgfqpoint{1.176152in}{2.173174in}}%
\pgfpathlineto{\pgfqpoint{1.177843in}{2.389816in}}%
\pgfpathlineto{\pgfqpoint{1.178689in}{2.104600in}}%
\pgfpathlineto{\pgfqpoint{1.179535in}{2.218853in}}%
\pgfpathlineto{\pgfqpoint{1.181226in}{2.072093in}}%
\pgfpathlineto{\pgfqpoint{1.182071in}{2.267135in}}%
\pgfpathlineto{\pgfqpoint{1.182917in}{2.061100in}}%
\pgfpathlineto{\pgfqpoint{1.183763in}{2.147154in}}%
\pgfpathlineto{\pgfqpoint{1.185454in}{2.068948in}}%
\pgfpathlineto{\pgfqpoint{1.186300in}{2.355496in}}%
\pgfpathlineto{\pgfqpoint{1.188836in}{2.030056in}}%
\pgfpathlineto{\pgfqpoint{1.190528in}{2.360423in}}%
\pgfpathlineto{\pgfqpoint{1.193065in}{2.086456in}}%
\pgfpathlineto{\pgfqpoint{1.194756in}{2.019554in}}%
\pgfpathlineto{\pgfqpoint{1.195601in}{2.136165in}}%
\pgfpathlineto{\pgfqpoint{1.196447in}{2.012520in}}%
\pgfpathlineto{\pgfqpoint{1.197293in}{2.110337in}}%
\pgfpathlineto{\pgfqpoint{1.198138in}{2.255635in}}%
\pgfpathlineto{\pgfqpoint{1.198984in}{2.148483in}}%
\pgfpathlineto{\pgfqpoint{1.200675in}{2.292338in}}%
\pgfpathlineto{\pgfqpoint{1.202366in}{2.142483in}}%
\pgfpathlineto{\pgfqpoint{1.203212in}{2.458239in}}%
\pgfpathlineto{\pgfqpoint{1.204058in}{2.035355in}}%
\pgfpathlineto{\pgfqpoint{1.204903in}{2.293942in}}%
\pgfpathlineto{\pgfqpoint{1.206595in}{2.001604in}}%
\pgfpathlineto{\pgfqpoint{1.207440in}{2.181769in}}%
\pgfpathlineto{\pgfqpoint{1.208286in}{2.070586in}}%
\pgfpathlineto{\pgfqpoint{1.209131in}{2.005361in}}%
\pgfpathlineto{\pgfqpoint{1.209977in}{2.310669in}}%
\pgfpathlineto{\pgfqpoint{1.210823in}{2.049931in}}%
\pgfpathlineto{\pgfqpoint{1.211668in}{2.272596in}}%
\pgfpathlineto{\pgfqpoint{1.212514in}{2.210834in}}%
\pgfpathlineto{\pgfqpoint{1.213359in}{1.907448in}}%
\pgfpathlineto{\pgfqpoint{1.214205in}{2.084827in}}%
\pgfpathlineto{\pgfqpoint{1.216742in}{2.267168in}}%
\pgfpathlineto{\pgfqpoint{1.217588in}{2.245563in}}%
\pgfpathlineto{\pgfqpoint{1.219279in}{2.027385in}}%
\pgfpathlineto{\pgfqpoint{1.221816in}{2.377561in}}%
\pgfpathlineto{\pgfqpoint{1.222661in}{2.059260in}}%
\pgfpathlineto{\pgfqpoint{1.223507in}{2.746106in}}%
\pgfpathlineto{\pgfqpoint{1.226044in}{1.977559in}}%
\pgfpathlineto{\pgfqpoint{1.228581in}{2.336797in}}%
\pgfpathlineto{\pgfqpoint{1.229426in}{2.014722in}}%
\pgfpathlineto{\pgfqpoint{1.230272in}{2.253323in}}%
\pgfpathlineto{\pgfqpoint{1.231118in}{2.086711in}}%
\pgfpathlineto{\pgfqpoint{1.231963in}{2.217927in}}%
\pgfpathlineto{\pgfqpoint{1.232809in}{2.434597in}}%
\pgfpathlineto{\pgfqpoint{1.233654in}{2.067158in}}%
\pgfpathlineto{\pgfqpoint{1.234500in}{2.240341in}}%
\pgfpathlineto{\pgfqpoint{1.235346in}{2.113457in}}%
\pgfpathlineto{\pgfqpoint{1.236191in}{2.342114in}}%
\pgfpathlineto{\pgfqpoint{1.237037in}{2.294010in}}%
\pgfpathlineto{\pgfqpoint{1.238728in}{1.846388in}}%
\pgfpathlineto{\pgfqpoint{1.239574in}{2.331404in}}%
\pgfpathlineto{\pgfqpoint{1.240419in}{2.021881in}}%
\pgfpathlineto{\pgfqpoint{1.241265in}{1.923751in}}%
\pgfpathlineto{\pgfqpoint{1.242111in}{2.153314in}}%
\pgfpathlineto{\pgfqpoint{1.242956in}{1.753400in}}%
\pgfpathlineto{\pgfqpoint{1.243802in}{2.116800in}}%
\pgfpathlineto{\pgfqpoint{1.244648in}{2.033587in}}%
\pgfpathlineto{\pgfqpoint{1.245493in}{2.004781in}}%
\pgfpathlineto{\pgfqpoint{1.248030in}{2.385606in}}%
\pgfpathlineto{\pgfqpoint{1.249721in}{2.088054in}}%
\pgfpathlineto{\pgfqpoint{1.250567in}{2.342836in}}%
\pgfpathlineto{\pgfqpoint{1.251413in}{2.054455in}}%
\pgfpathlineto{\pgfqpoint{1.253104in}{2.778536in}}%
\pgfpathlineto{\pgfqpoint{1.254795in}{1.937408in}}%
\pgfpathlineto{\pgfqpoint{1.255641in}{1.957463in}}%
\pgfpathlineto{\pgfqpoint{1.256486in}{2.267161in}}%
\pgfpathlineto{\pgfqpoint{1.257332in}{1.856368in}}%
\pgfpathlineto{\pgfqpoint{1.258178in}{2.362221in}}%
\pgfpathlineto{\pgfqpoint{1.259023in}{2.162710in}}%
\pgfpathlineto{\pgfqpoint{1.259869in}{2.271018in}}%
\pgfpathlineto{\pgfqpoint{1.260714in}{2.195231in}}%
\pgfpathlineto{\pgfqpoint{1.261560in}{2.025946in}}%
\pgfpathlineto{\pgfqpoint{1.264097in}{2.382622in}}%
\pgfpathlineto{\pgfqpoint{1.266634in}{1.991777in}}%
\pgfpathlineto{\pgfqpoint{1.267479in}{2.013350in}}%
\pgfpathlineto{\pgfqpoint{1.268325in}{2.254415in}}%
\pgfpathlineto{\pgfqpoint{1.269171in}{2.025119in}}%
\pgfpathlineto{\pgfqpoint{1.270016in}{2.322764in}}%
\pgfpathlineto{\pgfqpoint{1.270862in}{2.035633in}}%
\pgfpathlineto{\pgfqpoint{1.271708in}{2.163021in}}%
\pgfpathlineto{\pgfqpoint{1.272553in}{2.090684in}}%
\pgfpathlineto{\pgfqpoint{1.273399in}{2.098522in}}%
\pgfpathlineto{\pgfqpoint{1.274244in}{2.919513in}}%
\pgfpathlineto{\pgfqpoint{1.275090in}{2.027329in}}%
\pgfpathlineto{\pgfqpoint{1.275936in}{2.608765in}}%
\pgfpathlineto{\pgfqpoint{1.280164in}{2.112633in}}%
\pgfpathlineto{\pgfqpoint{1.282701in}{2.478677in}}%
\pgfpathlineto{\pgfqpoint{1.283546in}{2.056408in}}%
\pgfpathlineto{\pgfqpoint{1.284392in}{2.344917in}}%
\pgfpathlineto{\pgfqpoint{1.285238in}{2.295269in}}%
\pgfpathlineto{\pgfqpoint{1.286083in}{2.469610in}}%
\pgfpathlineto{\pgfqpoint{1.288620in}{2.081690in}}%
\pgfpathlineto{\pgfqpoint{1.289466in}{2.032877in}}%
\pgfpathlineto{\pgfqpoint{1.290311in}{2.494403in}}%
\pgfpathlineto{\pgfqpoint{1.292002in}{2.034620in}}%
\pgfpathlineto{\pgfqpoint{1.292848in}{2.193606in}}%
\pgfpathlineto{\pgfqpoint{1.293694in}{1.987776in}}%
\pgfpathlineto{\pgfqpoint{1.294539in}{2.430387in}}%
\pgfpathlineto{\pgfqpoint{1.295385in}{2.288085in}}%
\pgfpathlineto{\pgfqpoint{1.296231in}{2.243612in}}%
\pgfpathlineto{\pgfqpoint{1.297076in}{1.929637in}}%
\pgfpathlineto{\pgfqpoint{1.297922in}{2.041251in}}%
\pgfpathlineto{\pgfqpoint{1.298767in}{2.362178in}}%
\pgfpathlineto{\pgfqpoint{1.299613in}{2.163382in}}%
\pgfpathlineto{\pgfqpoint{1.302150in}{1.975806in}}%
\pgfpathlineto{\pgfqpoint{1.302996in}{2.303867in}}%
\pgfpathlineto{\pgfqpoint{1.303841in}{2.108422in}}%
\pgfpathlineto{\pgfqpoint{1.305532in}{2.250515in}}%
\pgfpathlineto{\pgfqpoint{1.306378in}{2.247336in}}%
\pgfpathlineto{\pgfqpoint{1.307224in}{2.052704in}}%
\pgfpathlineto{\pgfqpoint{1.308069in}{2.160372in}}%
\pgfpathlineto{\pgfqpoint{1.308915in}{2.196535in}}%
\pgfpathlineto{\pgfqpoint{1.309761in}{2.100957in}}%
\pgfpathlineto{\pgfqpoint{1.311452in}{2.298462in}}%
\pgfpathlineto{\pgfqpoint{1.312297in}{2.140765in}}%
\pgfpathlineto{\pgfqpoint{1.313989in}{2.368749in}}%
\pgfpathlineto{\pgfqpoint{1.315680in}{2.058473in}}%
\pgfpathlineto{\pgfqpoint{1.316526in}{2.081746in}}%
\pgfpathlineto{\pgfqpoint{1.317371in}{2.346910in}}%
\pgfpathlineto{\pgfqpoint{1.318217in}{2.190358in}}%
\pgfpathlineto{\pgfqpoint{1.319908in}{2.384023in}}%
\pgfpathlineto{\pgfqpoint{1.322445in}{1.999766in}}%
\pgfpathlineto{\pgfqpoint{1.323291in}{2.088079in}}%
\pgfpathlineto{\pgfqpoint{1.324136in}{2.339449in}}%
\pgfpathlineto{\pgfqpoint{1.326673in}{1.980928in}}%
\pgfpathlineto{\pgfqpoint{1.327519in}{2.554024in}}%
\pgfpathlineto{\pgfqpoint{1.328364in}{2.046395in}}%
\pgfpathlineto{\pgfqpoint{1.329210in}{2.246120in}}%
\pgfpathlineto{\pgfqpoint{1.330056in}{2.336771in}}%
\pgfpathlineto{\pgfqpoint{1.330901in}{2.073423in}}%
\pgfpathlineto{\pgfqpoint{1.331747in}{2.348612in}}%
\pgfpathlineto{\pgfqpoint{1.332592in}{2.188396in}}%
\pgfpathlineto{\pgfqpoint{1.333438in}{2.003779in}}%
\pgfpathlineto{\pgfqpoint{1.334284in}{2.068286in}}%
\pgfpathlineto{\pgfqpoint{1.335129in}{2.206828in}}%
\pgfpathlineto{\pgfqpoint{1.335975in}{1.924458in}}%
\pgfpathlineto{\pgfqpoint{1.336821in}{2.292586in}}%
\pgfpathlineto{\pgfqpoint{1.337666in}{1.894929in}}%
\pgfpathlineto{\pgfqpoint{1.339357in}{2.403035in}}%
\pgfpathlineto{\pgfqpoint{1.340203in}{2.025014in}}%
\pgfpathlineto{\pgfqpoint{1.341049in}{2.352374in}}%
\pgfpathlineto{\pgfqpoint{1.341894in}{2.146631in}}%
\pgfpathlineto{\pgfqpoint{1.342740in}{2.261051in}}%
\pgfpathlineto{\pgfqpoint{1.343586in}{2.212609in}}%
\pgfpathlineto{\pgfqpoint{1.344431in}{2.227626in}}%
\pgfpathlineto{\pgfqpoint{1.345277in}{1.968282in}}%
\pgfpathlineto{\pgfqpoint{1.346122in}{2.165250in}}%
\pgfpathlineto{\pgfqpoint{1.346968in}{2.009060in}}%
\pgfpathlineto{\pgfqpoint{1.347814in}{2.341259in}}%
\pgfpathlineto{\pgfqpoint{1.348659in}{2.338206in}}%
\pgfpathlineto{\pgfqpoint{1.349505in}{2.277568in}}%
\pgfpathlineto{\pgfqpoint{1.350351in}{2.322086in}}%
\pgfpathlineto{\pgfqpoint{1.351196in}{2.301360in}}%
\pgfpathlineto{\pgfqpoint{1.352042in}{2.227758in}}%
\pgfpathlineto{\pgfqpoint{1.352887in}{2.315106in}}%
\pgfpathlineto{\pgfqpoint{1.353733in}{2.235966in}}%
\pgfpathlineto{\pgfqpoint{1.354579in}{1.863967in}}%
\pgfpathlineto{\pgfqpoint{1.355424in}{2.060454in}}%
\pgfpathlineto{\pgfqpoint{1.356270in}{2.412792in}}%
\pgfpathlineto{\pgfqpoint{1.357116in}{2.054848in}}%
\pgfpathlineto{\pgfqpoint{1.357961in}{2.468015in}}%
\pgfpathlineto{\pgfqpoint{1.358807in}{2.213238in}}%
\pgfpathlineto{\pgfqpoint{1.359652in}{1.825377in}}%
\pgfpathlineto{\pgfqpoint{1.360498in}{1.874981in}}%
\pgfpathlineto{\pgfqpoint{1.361344in}{1.845153in}}%
\pgfpathlineto{\pgfqpoint{1.362189in}{2.217311in}}%
\pgfpathlineto{\pgfqpoint{1.363035in}{2.118573in}}%
\pgfpathlineto{\pgfqpoint{1.363881in}{2.197429in}}%
\pgfpathlineto{\pgfqpoint{1.364726in}{2.143722in}}%
\pgfpathlineto{\pgfqpoint{1.365572in}{2.153007in}}%
\pgfpathlineto{\pgfqpoint{1.366417in}{2.386564in}}%
\pgfpathlineto{\pgfqpoint{1.367263in}{2.278271in}}%
\pgfpathlineto{\pgfqpoint{1.368109in}{2.082935in}}%
\pgfpathlineto{\pgfqpoint{1.370645in}{2.440794in}}%
\pgfpathlineto{\pgfqpoint{1.371491in}{2.109757in}}%
\pgfpathlineto{\pgfqpoint{1.372337in}{2.404764in}}%
\pgfpathlineto{\pgfqpoint{1.373182in}{2.165957in}}%
\pgfpathlineto{\pgfqpoint{1.374028in}{2.185115in}}%
\pgfpathlineto{\pgfqpoint{1.374874in}{2.123625in}}%
\pgfpathlineto{\pgfqpoint{1.375719in}{2.492554in}}%
\pgfpathlineto{\pgfqpoint{1.376565in}{2.304601in}}%
\pgfpathlineto{\pgfqpoint{1.378256in}{2.141103in}}%
\pgfpathlineto{\pgfqpoint{1.379102in}{2.537057in}}%
\pgfpathlineto{\pgfqpoint{1.379947in}{2.022437in}}%
\pgfpathlineto{\pgfqpoint{1.380793in}{2.170033in}}%
\pgfpathlineto{\pgfqpoint{1.381639in}{2.140288in}}%
\pgfpathlineto{\pgfqpoint{1.383330in}{1.533204in}}%
\pgfpathlineto{\pgfqpoint{1.385021in}{2.165790in}}%
\pgfpathlineto{\pgfqpoint{1.385867in}{2.143414in}}%
\pgfpathlineto{\pgfqpoint{1.386712in}{2.165996in}}%
\pgfpathlineto{\pgfqpoint{1.387558in}{1.918128in}}%
\pgfpathlineto{\pgfqpoint{1.388404in}{2.604881in}}%
\pgfpathlineto{\pgfqpoint{1.389249in}{2.000379in}}%
\pgfpathlineto{\pgfqpoint{1.390095in}{2.142567in}}%
\pgfpathlineto{\pgfqpoint{1.391786in}{2.690306in}}%
\pgfpathlineto{\pgfqpoint{1.393477in}{2.194289in}}%
\pgfpathlineto{\pgfqpoint{1.394323in}{2.550678in}}%
\pgfpathlineto{\pgfqpoint{1.395169in}{2.362377in}}%
\pgfpathlineto{\pgfqpoint{1.396860in}{1.993107in}}%
\pgfpathlineto{\pgfqpoint{1.398551in}{2.410216in}}%
\pgfpathlineto{\pgfqpoint{1.399397in}{2.286795in}}%
\pgfpathlineto{\pgfqpoint{1.400242in}{2.551385in}}%
\pgfpathlineto{\pgfqpoint{1.401088in}{1.978757in}}%
\pgfpathlineto{\pgfqpoint{1.401934in}{2.547740in}}%
\pgfpathlineto{\pgfqpoint{1.402779in}{1.826870in}}%
\pgfpathlineto{\pgfqpoint{1.403625in}{2.331527in}}%
\pgfpathlineto{\pgfqpoint{1.404470in}{2.103073in}}%
\pgfpathlineto{\pgfqpoint{1.405316in}{2.553735in}}%
\pgfpathlineto{\pgfqpoint{1.407853in}{1.906615in}}%
\pgfpathlineto{\pgfqpoint{1.408699in}{2.214274in}}%
\pgfpathlineto{\pgfqpoint{1.409544in}{1.997680in}}%
\pgfpathlineto{\pgfqpoint{1.410390in}{2.050912in}}%
\pgfpathlineto{\pgfqpoint{1.412927in}{2.264356in}}%
\pgfpathlineto{\pgfqpoint{1.413772in}{2.257185in}}%
\pgfpathlineto{\pgfqpoint{1.414618in}{2.055887in}}%
\pgfpathlineto{\pgfqpoint{1.415464in}{2.445286in}}%
\pgfpathlineto{\pgfqpoint{1.416309in}{1.976691in}}%
\pgfpathlineto{\pgfqpoint{1.417155in}{2.309981in}}%
\pgfpathlineto{\pgfqpoint{1.418000in}{2.185366in}}%
\pgfpathlineto{\pgfqpoint{1.418846in}{1.717516in}}%
\pgfpathlineto{\pgfqpoint{1.420537in}{2.792106in}}%
\pgfpathlineto{\pgfqpoint{1.421383in}{1.795094in}}%
\pgfpathlineto{\pgfqpoint{1.422229in}{2.201930in}}%
\pgfpathlineto{\pgfqpoint{1.423920in}{1.755889in}}%
\pgfpathlineto{\pgfqpoint{1.424765in}{1.978485in}}%
\pgfpathlineto{\pgfqpoint{1.425611in}{1.793054in}}%
\pgfpathlineto{\pgfqpoint{1.426457in}{2.157403in}}%
\pgfpathlineto{\pgfqpoint{1.427302in}{2.031401in}}%
\pgfpathlineto{\pgfqpoint{1.428148in}{2.120777in}}%
\pgfpathlineto{\pgfqpoint{1.428994in}{2.645494in}}%
\pgfpathlineto{\pgfqpoint{1.429839in}{2.139109in}}%
\pgfpathlineto{\pgfqpoint{1.430685in}{2.445501in}}%
\pgfpathlineto{\pgfqpoint{1.434067in}{1.935084in}}%
\pgfpathlineto{\pgfqpoint{1.434913in}{2.328699in}}%
\pgfpathlineto{\pgfqpoint{1.435759in}{2.286809in}}%
\pgfpathlineto{\pgfqpoint{1.436604in}{2.364369in}}%
\pgfpathlineto{\pgfqpoint{1.437450in}{2.118647in}}%
\pgfpathlineto{\pgfqpoint{1.439987in}{2.540933in}}%
\pgfpathlineto{\pgfqpoint{1.440832in}{2.190494in}}%
\pgfpathlineto{\pgfqpoint{1.441678in}{2.496780in}}%
\pgfpathlineto{\pgfqpoint{1.443369in}{2.148703in}}%
\pgfpathlineto{\pgfqpoint{1.445060in}{2.535962in}}%
\pgfpathlineto{\pgfqpoint{1.445906in}{2.257996in}}%
\pgfpathlineto{\pgfqpoint{1.446752in}{2.347632in}}%
\pgfpathlineto{\pgfqpoint{1.448443in}{2.197352in}}%
\pgfpathlineto{\pgfqpoint{1.449288in}{1.689206in}}%
\pgfpathlineto{\pgfqpoint{1.450134in}{2.246195in}}%
\pgfpathlineto{\pgfqpoint{1.450980in}{2.236655in}}%
\pgfpathlineto{\pgfqpoint{1.453517in}{1.559859in}}%
\pgfpathlineto{\pgfqpoint{1.455208in}{2.138567in}}%
\pgfpathlineto{\pgfqpoint{1.456053in}{2.031585in}}%
\pgfpathlineto{\pgfqpoint{1.456899in}{2.495257in}}%
\pgfpathlineto{\pgfqpoint{1.457745in}{2.296887in}}%
\pgfpathlineto{\pgfqpoint{1.458590in}{2.244175in}}%
\pgfpathlineto{\pgfqpoint{1.459436in}{2.062174in}}%
\pgfpathlineto{\pgfqpoint{1.460282in}{2.706655in}}%
\pgfpathlineto{\pgfqpoint{1.461973in}{1.788797in}}%
\pgfpathlineto{\pgfqpoint{1.462818in}{1.986903in}}%
\pgfpathlineto{\pgfqpoint{1.463664in}{2.483815in}}%
\pgfpathlineto{\pgfqpoint{1.464510in}{1.793582in}}%
\pgfpathlineto{\pgfqpoint{1.465355in}{1.978347in}}%
\pgfpathlineto{\pgfqpoint{1.466201in}{1.711745in}}%
\pgfpathlineto{\pgfqpoint{1.467047in}{2.394738in}}%
\pgfpathlineto{\pgfqpoint{1.467892in}{2.075087in}}%
\pgfpathlineto{\pgfqpoint{1.469583in}{2.347970in}}%
\pgfpathlineto{\pgfqpoint{1.472120in}{1.953949in}}%
\pgfpathlineto{\pgfqpoint{1.472966in}{2.423589in}}%
\pgfpathlineto{\pgfqpoint{1.473812in}{1.894044in}}%
\pgfpathlineto{\pgfqpoint{1.474657in}{1.931691in}}%
\pgfpathlineto{\pgfqpoint{1.475503in}{2.392755in}}%
\pgfpathlineto{\pgfqpoint{1.476348in}{2.036400in}}%
\pgfpathlineto{\pgfqpoint{1.479731in}{2.576862in}}%
\pgfpathlineto{\pgfqpoint{1.482268in}{1.921481in}}%
\pgfpathlineto{\pgfqpoint{1.483113in}{2.467453in}}%
\pgfpathlineto{\pgfqpoint{1.483959in}{2.209423in}}%
\pgfpathlineto{\pgfqpoint{1.484805in}{2.385709in}}%
\pgfpathlineto{\pgfqpoint{1.485650in}{2.040871in}}%
\pgfpathlineto{\pgfqpoint{1.486496in}{2.534730in}}%
\pgfpathlineto{\pgfqpoint{1.487342in}{2.306647in}}%
\pgfpathlineto{\pgfqpoint{1.488187in}{2.221744in}}%
\pgfpathlineto{\pgfqpoint{1.489033in}{2.460847in}}%
\pgfpathlineto{\pgfqpoint{1.491570in}{1.763814in}}%
\pgfpathlineto{\pgfqpoint{1.493261in}{2.721592in}}%
\pgfpathlineto{\pgfqpoint{1.494107in}{2.202902in}}%
\pgfpathlineto{\pgfqpoint{1.494952in}{2.431395in}}%
\pgfpathlineto{\pgfqpoint{1.497489in}{2.105299in}}%
\pgfpathlineto{\pgfqpoint{1.498335in}{2.237504in}}%
\pgfpathlineto{\pgfqpoint{1.499180in}{2.060009in}}%
\pgfpathlineto{\pgfqpoint{1.500026in}{2.303541in}}%
\pgfpathlineto{\pgfqpoint{1.500872in}{2.203591in}}%
\pgfpathlineto{\pgfqpoint{1.502563in}{1.867817in}}%
\pgfpathlineto{\pgfqpoint{1.504254in}{2.261595in}}%
\pgfpathlineto{\pgfqpoint{1.505100in}{2.653321in}}%
\pgfpathlineto{\pgfqpoint{1.505945in}{2.030811in}}%
\pgfpathlineto{\pgfqpoint{1.506791in}{2.252664in}}%
\pgfpathlineto{\pgfqpoint{1.507637in}{2.614927in}}%
\pgfpathlineto{\pgfqpoint{1.508482in}{2.080535in}}%
\pgfpathlineto{\pgfqpoint{1.509328in}{2.089914in}}%
\pgfpathlineto{\pgfqpoint{1.510173in}{2.441127in}}%
\pgfpathlineto{\pgfqpoint{1.511019in}{2.049712in}}%
\pgfpathlineto{\pgfqpoint{1.511865in}{2.080944in}}%
\pgfpathlineto{\pgfqpoint{1.513556in}{2.397930in}}%
\pgfpathlineto{\pgfqpoint{1.514402in}{2.099175in}}%
\pgfpathlineto{\pgfqpoint{1.515247in}{2.368116in}}%
\pgfpathlineto{\pgfqpoint{1.516093in}{2.194170in}}%
\pgfpathlineto{\pgfqpoint{1.516938in}{1.883754in}}%
\pgfpathlineto{\pgfqpoint{1.517784in}{2.015609in}}%
\pgfpathlineto{\pgfqpoint{1.518630in}{2.051268in}}%
\pgfpathlineto{\pgfqpoint{1.519475in}{1.971488in}}%
\pgfpathlineto{\pgfqpoint{1.520321in}{2.645935in}}%
\pgfpathlineto{\pgfqpoint{1.522858in}{1.874988in}}%
\pgfpathlineto{\pgfqpoint{1.524549in}{2.486364in}}%
\pgfpathlineto{\pgfqpoint{1.525395in}{2.074272in}}%
\pgfpathlineto{\pgfqpoint{1.526240in}{2.378478in}}%
\pgfpathlineto{\pgfqpoint{1.528777in}{2.076538in}}%
\pgfpathlineto{\pgfqpoint{1.529623in}{2.953308in}}%
\pgfpathlineto{\pgfqpoint{1.530468in}{1.501166in}}%
\pgfpathlineto{\pgfqpoint{1.531314in}{1.858020in}}%
\pgfpathlineto{\pgfqpoint{1.532160in}{2.719619in}}%
\pgfpathlineto{\pgfqpoint{1.533005in}{2.289381in}}%
\pgfpathlineto{\pgfqpoint{1.533851in}{1.909703in}}%
\pgfpathlineto{\pgfqpoint{1.534696in}{1.910764in}}%
\pgfpathlineto{\pgfqpoint{1.536388in}{2.535077in}}%
\pgfpathlineto{\pgfqpoint{1.537233in}{2.095326in}}%
\pgfpathlineto{\pgfqpoint{1.538079in}{2.252801in}}%
\pgfpathlineto{\pgfqpoint{1.539770in}{1.993691in}}%
\pgfpathlineto{\pgfqpoint{1.540616in}{2.438665in}}%
\pgfpathlineto{\pgfqpoint{1.541461in}{2.070875in}}%
\pgfpathlineto{\pgfqpoint{1.542307in}{2.363734in}}%
\pgfpathlineto{\pgfqpoint{1.543153in}{2.326379in}}%
\pgfpathlineto{\pgfqpoint{1.543998in}{2.065590in}}%
\pgfpathlineto{\pgfqpoint{1.544844in}{2.189036in}}%
\pgfpathlineto{\pgfqpoint{1.545690in}{2.325659in}}%
\pgfpathlineto{\pgfqpoint{1.546535in}{2.068652in}}%
\pgfpathlineto{\pgfqpoint{1.547381in}{2.934152in}}%
\pgfpathlineto{\pgfqpoint{1.548226in}{2.101799in}}%
\pgfpathlineto{\pgfqpoint{1.549072in}{2.188272in}}%
\pgfpathlineto{\pgfqpoint{1.549918in}{2.650355in}}%
\pgfpathlineto{\pgfqpoint{1.550763in}{1.939990in}}%
\pgfpathlineto{\pgfqpoint{1.551609in}{2.189012in}}%
\pgfpathlineto{\pgfqpoint{1.552455in}{1.899245in}}%
\pgfpathlineto{\pgfqpoint{1.553300in}{2.543771in}}%
\pgfpathlineto{\pgfqpoint{1.554146in}{2.336928in}}%
\pgfpathlineto{\pgfqpoint{1.554991in}{2.392179in}}%
\pgfpathlineto{\pgfqpoint{1.555837in}{1.808632in}}%
\pgfpathlineto{\pgfqpoint{1.556683in}{2.212594in}}%
\pgfpathlineto{\pgfqpoint{1.557528in}{1.912555in}}%
\pgfpathlineto{\pgfqpoint{1.558374in}{2.278200in}}%
\pgfpathlineto{\pgfqpoint{1.559220in}{2.046324in}}%
\pgfpathlineto{\pgfqpoint{1.560065in}{2.054320in}}%
\pgfpathlineto{\pgfqpoint{1.560911in}{2.338794in}}%
\pgfpathlineto{\pgfqpoint{1.561756in}{1.698553in}}%
\pgfpathlineto{\pgfqpoint{1.562602in}{2.027527in}}%
\pgfpathlineto{\pgfqpoint{1.563448in}{1.985926in}}%
\pgfpathlineto{\pgfqpoint{1.564293in}{2.351152in}}%
\pgfpathlineto{\pgfqpoint{1.565139in}{2.062952in}}%
\pgfpathlineto{\pgfqpoint{1.566830in}{2.139323in}}%
\pgfpathlineto{\pgfqpoint{1.568521in}{2.405837in}}%
\pgfpathlineto{\pgfqpoint{1.569367in}{1.834237in}}%
\pgfpathlineto{\pgfqpoint{1.570213in}{2.128575in}}%
\pgfpathlineto{\pgfqpoint{1.571058in}{1.770325in}}%
\pgfpathlineto{\pgfqpoint{1.572750in}{2.381751in}}%
\pgfpathlineto{\pgfqpoint{1.574441in}{2.452668in}}%
\pgfpathlineto{\pgfqpoint{1.575286in}{1.722862in}}%
\pgfpathlineto{\pgfqpoint{1.576132in}{2.093206in}}%
\pgfpathlineto{\pgfqpoint{1.577823in}{1.969668in}}%
\pgfpathlineto{\pgfqpoint{1.578669in}{2.564007in}}%
\pgfpathlineto{\pgfqpoint{1.579515in}{2.330727in}}%
\pgfpathlineto{\pgfqpoint{1.580360in}{1.966921in}}%
\pgfpathlineto{\pgfqpoint{1.581206in}{1.978400in}}%
\pgfpathlineto{\pgfqpoint{1.583743in}{2.182997in}}%
\pgfpathlineto{\pgfqpoint{1.584588in}{1.821431in}}%
\pgfpathlineto{\pgfqpoint{1.585434in}{2.190783in}}%
\pgfpathlineto{\pgfqpoint{1.586280in}{1.678895in}}%
\pgfpathlineto{\pgfqpoint{1.587971in}{2.640737in}}%
\pgfpathlineto{\pgfqpoint{1.590508in}{1.813206in}}%
\pgfpathlineto{\pgfqpoint{1.593045in}{2.232215in}}%
\pgfpathlineto{\pgfqpoint{1.593890in}{2.215214in}}%
\pgfpathlineto{\pgfqpoint{1.594736in}{2.580483in}}%
\pgfpathlineto{\pgfqpoint{1.595581in}{2.119707in}}%
\pgfpathlineto{\pgfqpoint{1.596427in}{2.493975in}}%
\pgfpathlineto{\pgfqpoint{1.597273in}{2.930045in}}%
\pgfpathlineto{\pgfqpoint{1.598964in}{1.948731in}}%
\pgfpathlineto{\pgfqpoint{1.599810in}{2.161367in}}%
\pgfpathlineto{\pgfqpoint{1.600655in}{1.849383in}}%
\pgfpathlineto{\pgfqpoint{1.601501in}{2.297569in}}%
\pgfpathlineto{\pgfqpoint{1.602346in}{2.107523in}}%
\pgfpathlineto{\pgfqpoint{1.603192in}{1.712771in}}%
\pgfpathlineto{\pgfqpoint{1.604038in}{2.364760in}}%
\pgfpathlineto{\pgfqpoint{1.604883in}{2.115043in}}%
\pgfpathlineto{\pgfqpoint{1.605729in}{1.960148in}}%
\pgfpathlineto{\pgfqpoint{1.606574in}{2.157049in}}%
\pgfpathlineto{\pgfqpoint{1.607420in}{2.011060in}}%
\pgfpathlineto{\pgfqpoint{1.608266in}{2.084471in}}%
\pgfpathlineto{\pgfqpoint{1.609957in}{1.631789in}}%
\pgfpathlineto{\pgfqpoint{1.611648in}{2.805218in}}%
\pgfpathlineto{\pgfqpoint{1.613339in}{1.974318in}}%
\pgfpathlineto{\pgfqpoint{1.614185in}{2.142477in}}%
\pgfpathlineto{\pgfqpoint{1.615031in}{2.258322in}}%
\pgfpathlineto{\pgfqpoint{1.615876in}{2.210387in}}%
\pgfpathlineto{\pgfqpoint{1.616722in}{2.095607in}}%
\pgfpathlineto{\pgfqpoint{1.617568in}{2.373319in}}%
\pgfpathlineto{\pgfqpoint{1.619259in}{1.741593in}}%
\pgfpathlineto{\pgfqpoint{1.620950in}{2.382813in}}%
\pgfpathlineto{\pgfqpoint{1.621796in}{2.370514in}}%
\pgfpathlineto{\pgfqpoint{1.622641in}{2.073553in}}%
\pgfpathlineto{\pgfqpoint{1.623487in}{2.321410in}}%
\pgfpathlineto{\pgfqpoint{1.624333in}{1.965906in}}%
\pgfpathlineto{\pgfqpoint{1.625178in}{2.239518in}}%
\pgfpathlineto{\pgfqpoint{1.626024in}{2.570598in}}%
\pgfpathlineto{\pgfqpoint{1.626869in}{1.913060in}}%
\pgfpathlineto{\pgfqpoint{1.627715in}{2.126294in}}%
\pgfpathlineto{\pgfqpoint{1.628561in}{1.889096in}}%
\pgfpathlineto{\pgfqpoint{1.629406in}{1.959030in}}%
\pgfpathlineto{\pgfqpoint{1.630252in}{2.698697in}}%
\pgfpathlineto{\pgfqpoint{1.631098in}{2.281581in}}%
\pgfpathlineto{\pgfqpoint{1.631943in}{2.588123in}}%
\pgfpathlineto{\pgfqpoint{1.632789in}{2.335937in}}%
\pgfpathlineto{\pgfqpoint{1.633634in}{2.185138in}}%
\pgfpathlineto{\pgfqpoint{1.634480in}{2.509917in}}%
\pgfpathlineto{\pgfqpoint{1.635326in}{2.009140in}}%
\pgfpathlineto{\pgfqpoint{1.636171in}{2.275182in}}%
\pgfpathlineto{\pgfqpoint{1.637017in}{2.053731in}}%
\pgfpathlineto{\pgfqpoint{1.637863in}{2.406771in}}%
\pgfpathlineto{\pgfqpoint{1.640399in}{1.480160in}}%
\pgfpathlineto{\pgfqpoint{1.643782in}{2.860447in}}%
\pgfpathlineto{\pgfqpoint{1.646319in}{2.092055in}}%
\pgfpathlineto{\pgfqpoint{1.648856in}{1.764919in}}%
\pgfpathlineto{\pgfqpoint{1.649701in}{2.287020in}}%
\pgfpathlineto{\pgfqpoint{1.650547in}{1.908336in}}%
\pgfpathlineto{\pgfqpoint{1.651393in}{1.699386in}}%
\pgfpathlineto{\pgfqpoint{1.652238in}{2.563204in}}%
\pgfpathlineto{\pgfqpoint{1.653084in}{2.180778in}}%
\pgfpathlineto{\pgfqpoint{1.653929in}{1.688973in}}%
\pgfpathlineto{\pgfqpoint{1.654775in}{2.973125in}}%
\pgfpathlineto{\pgfqpoint{1.655621in}{2.536134in}}%
\pgfpathlineto{\pgfqpoint{1.656466in}{2.804581in}}%
\pgfpathlineto{\pgfqpoint{1.659003in}{2.088818in}}%
\pgfpathlineto{\pgfqpoint{1.659849in}{2.475760in}}%
\pgfpathlineto{\pgfqpoint{1.660694in}{1.842883in}}%
\pgfpathlineto{\pgfqpoint{1.661540in}{2.347284in}}%
\pgfpathlineto{\pgfqpoint{1.662386in}{2.652307in}}%
\pgfpathlineto{\pgfqpoint{1.663231in}{2.506110in}}%
\pgfpathlineto{\pgfqpoint{1.664077in}{2.536867in}}%
\pgfpathlineto{\pgfqpoint{1.664923in}{2.190579in}}%
\pgfpathlineto{\pgfqpoint{1.665768in}{2.891662in}}%
\pgfpathlineto{\pgfqpoint{1.666614in}{2.437342in}}%
\pgfpathlineto{\pgfqpoint{1.667459in}{2.507069in}}%
\pgfpathlineto{\pgfqpoint{1.669151in}{1.779787in}}%
\pgfpathlineto{\pgfqpoint{1.669996in}{2.363120in}}%
\pgfpathlineto{\pgfqpoint{1.670842in}{1.729478in}}%
\pgfpathlineto{\pgfqpoint{1.673379in}{2.682132in}}%
\pgfpathlineto{\pgfqpoint{1.675916in}{1.768742in}}%
\pgfpathlineto{\pgfqpoint{1.676761in}{2.497584in}}%
\pgfpathlineto{\pgfqpoint{1.677607in}{1.636977in}}%
\pgfpathlineto{\pgfqpoint{1.678453in}{2.276828in}}%
\pgfpathlineto{\pgfqpoint{1.679298in}{1.790051in}}%
\pgfpathlineto{\pgfqpoint{1.680144in}{1.940647in}}%
\pgfpathlineto{\pgfqpoint{1.680989in}{1.891426in}}%
\pgfpathlineto{\pgfqpoint{1.682681in}{2.592772in}}%
\pgfpathlineto{\pgfqpoint{1.683526in}{2.526257in}}%
\pgfpathlineto{\pgfqpoint{1.684372in}{1.744340in}}%
\pgfpathlineto{\pgfqpoint{1.685217in}{1.834974in}}%
\pgfpathlineto{\pgfqpoint{1.686063in}{2.482349in}}%
\pgfpathlineto{\pgfqpoint{1.686909in}{1.464770in}}%
\pgfpathlineto{\pgfqpoint{1.687754in}{2.331371in}}%
\pgfpathlineto{\pgfqpoint{1.688600in}{1.980461in}}%
\pgfpathlineto{\pgfqpoint{1.689446in}{2.278246in}}%
\pgfpathlineto{\pgfqpoint{1.690291in}{1.959584in}}%
\pgfpathlineto{\pgfqpoint{1.691137in}{2.321537in}}%
\pgfpathlineto{\pgfqpoint{1.691982in}{2.297496in}}%
\pgfpathlineto{\pgfqpoint{1.692828in}{1.916684in}}%
\pgfpathlineto{\pgfqpoint{1.694519in}{2.473471in}}%
\pgfpathlineto{\pgfqpoint{1.697056in}{1.715569in}}%
\pgfpathlineto{\pgfqpoint{1.697902in}{2.384922in}}%
\pgfpathlineto{\pgfqpoint{1.698747in}{1.995904in}}%
\pgfpathlineto{\pgfqpoint{1.699593in}{2.422140in}}%
\pgfpathlineto{\pgfqpoint{1.700439in}{1.625955in}}%
\pgfpathlineto{\pgfqpoint{1.701284in}{2.255110in}}%
\pgfpathlineto{\pgfqpoint{1.702130in}{2.638516in}}%
\pgfpathlineto{\pgfqpoint{1.703821in}{1.905099in}}%
\pgfpathlineto{\pgfqpoint{1.705512in}{2.470008in}}%
\pgfpathlineto{\pgfqpoint{1.706358in}{1.287594in}}%
\pgfpathlineto{\pgfqpoint{1.707204in}{2.740066in}}%
\pgfpathlineto{\pgfqpoint{1.708049in}{2.163457in}}%
\pgfpathlineto{\pgfqpoint{1.709741in}{1.919582in}}%
\pgfpathlineto{\pgfqpoint{1.712277in}{2.671757in}}%
\pgfpathlineto{\pgfqpoint{1.714814in}{1.880185in}}%
\pgfpathlineto{\pgfqpoint{1.716506in}{2.150187in}}%
\pgfpathlineto{\pgfqpoint{1.717351in}{1.949320in}}%
\pgfpathlineto{\pgfqpoint{1.718197in}{2.108042in}}%
\pgfpathlineto{\pgfqpoint{1.719042in}{2.506615in}}%
\pgfpathlineto{\pgfqpoint{1.721579in}{1.850277in}}%
\pgfpathlineto{\pgfqpoint{1.723271in}{2.400271in}}%
\pgfpathlineto{\pgfqpoint{1.724116in}{2.302659in}}%
\pgfpathlineto{\pgfqpoint{1.724962in}{2.102128in}}%
\pgfpathlineto{\pgfqpoint{1.725807in}{2.602305in}}%
\pgfpathlineto{\pgfqpoint{1.726653in}{2.142636in}}%
\pgfpathlineto{\pgfqpoint{1.727499in}{2.767175in}}%
\pgfpathlineto{\pgfqpoint{1.728344in}{1.901297in}}%
\pgfpathlineto{\pgfqpoint{1.729190in}{2.356779in}}%
\pgfpathlineto{\pgfqpoint{1.730036in}{2.510481in}}%
\pgfpathlineto{\pgfqpoint{1.730881in}{1.691057in}}%
\pgfpathlineto{\pgfqpoint{1.731727in}{2.037218in}}%
\pgfpathlineto{\pgfqpoint{1.732572in}{2.749700in}}%
\pgfpathlineto{\pgfqpoint{1.733418in}{2.267518in}}%
\pgfpathlineto{\pgfqpoint{1.734264in}{2.741993in}}%
\pgfpathlineto{\pgfqpoint{1.735955in}{1.658523in}}%
\pgfpathlineto{\pgfqpoint{1.736801in}{1.856299in}}%
\pgfpathlineto{\pgfqpoint{1.739337in}{2.414914in}}%
\pgfpathlineto{\pgfqpoint{1.741029in}{1.793214in}}%
\pgfpathlineto{\pgfqpoint{1.741874in}{2.166687in}}%
\pgfpathlineto{\pgfqpoint{1.742720in}{1.787351in}}%
\pgfpathlineto{\pgfqpoint{1.743566in}{2.285004in}}%
\pgfpathlineto{\pgfqpoint{1.744411in}{1.759938in}}%
\pgfpathlineto{\pgfqpoint{1.745257in}{2.205134in}}%
\pgfpathlineto{\pgfqpoint{1.746102in}{1.827552in}}%
\pgfpathlineto{\pgfqpoint{1.747794in}{2.451843in}}%
\pgfpathlineto{\pgfqpoint{1.748639in}{1.648544in}}%
\pgfpathlineto{\pgfqpoint{1.749485in}{2.652587in}}%
\pgfpathlineto{\pgfqpoint{1.750331in}{2.341939in}}%
\pgfpathlineto{\pgfqpoint{1.752867in}{1.823623in}}%
\pgfpathlineto{\pgfqpoint{1.753713in}{2.512137in}}%
\pgfpathlineto{\pgfqpoint{1.756250in}{1.740118in}}%
\pgfpathlineto{\pgfqpoint{1.757941in}{2.361225in}}%
\pgfpathlineto{\pgfqpoint{1.758787in}{2.173353in}}%
\pgfpathlineto{\pgfqpoint{1.759632in}{2.476012in}}%
\pgfpathlineto{\pgfqpoint{1.760478in}{1.604592in}}%
\pgfpathlineto{\pgfqpoint{1.761324in}{2.507174in}}%
\pgfpathlineto{\pgfqpoint{1.762169in}{2.124841in}}%
\pgfpathlineto{\pgfqpoint{1.763015in}{2.086396in}}%
\pgfpathlineto{\pgfqpoint{1.763860in}{1.742562in}}%
\pgfpathlineto{\pgfqpoint{1.764706in}{2.312160in}}%
\pgfpathlineto{\pgfqpoint{1.765552in}{2.184063in}}%
\pgfpathlineto{\pgfqpoint{1.766397in}{2.187141in}}%
\pgfpathlineto{\pgfqpoint{1.767243in}{2.305612in}}%
\pgfpathlineto{\pgfqpoint{1.768089in}{1.825980in}}%
\pgfpathlineto{\pgfqpoint{1.768934in}{2.189715in}}%
\pgfpathlineto{\pgfqpoint{1.770625in}{1.933839in}}%
\pgfpathlineto{\pgfqpoint{1.772317in}{2.724044in}}%
\pgfpathlineto{\pgfqpoint{1.773162in}{1.720532in}}%
\pgfpathlineto{\pgfqpoint{1.774008in}{2.307315in}}%
\pgfpathlineto{\pgfqpoint{1.774854in}{1.764100in}}%
\pgfpathlineto{\pgfqpoint{1.775699in}{2.680405in}}%
\pgfpathlineto{\pgfqpoint{1.776545in}{2.324821in}}%
\pgfpathlineto{\pgfqpoint{1.777390in}{2.096043in}}%
\pgfpathlineto{\pgfqpoint{1.779082in}{2.571908in}}%
\pgfpathlineto{\pgfqpoint{1.779927in}{1.811399in}}%
\pgfpathlineto{\pgfqpoint{1.780773in}{2.628505in}}%
\pgfpathlineto{\pgfqpoint{1.781619in}{1.587508in}}%
\pgfpathlineto{\pgfqpoint{1.782464in}{2.301096in}}%
\pgfpathlineto{\pgfqpoint{1.783310in}{2.942245in}}%
\pgfpathlineto{\pgfqpoint{1.784155in}{2.221472in}}%
\pgfpathlineto{\pgfqpoint{1.785001in}{2.248143in}}%
\pgfpathlineto{\pgfqpoint{1.785847in}{2.241100in}}%
\pgfpathlineto{\pgfqpoint{1.786692in}{1.609901in}}%
\pgfpathlineto{\pgfqpoint{1.787538in}{2.375273in}}%
\pgfpathlineto{\pgfqpoint{1.788384in}{2.089926in}}%
\pgfpathlineto{\pgfqpoint{1.789229in}{2.081040in}}%
\pgfpathlineto{\pgfqpoint{1.790075in}{1.755421in}}%
\pgfpathlineto{\pgfqpoint{1.791766in}{2.367198in}}%
\pgfpathlineto{\pgfqpoint{1.792612in}{1.921334in}}%
\pgfpathlineto{\pgfqpoint{1.793457in}{2.125183in}}%
\pgfpathlineto{\pgfqpoint{1.794303in}{3.046485in}}%
\pgfpathlineto{\pgfqpoint{1.795149in}{1.841909in}}%
\pgfpathlineto{\pgfqpoint{1.795994in}{2.444046in}}%
\pgfpathlineto{\pgfqpoint{1.797685in}{1.763186in}}%
\pgfpathlineto{\pgfqpoint{1.798531in}{1.868191in}}%
\pgfpathlineto{\pgfqpoint{1.800222in}{2.558068in}}%
\pgfpathlineto{\pgfqpoint{1.801068in}{2.099467in}}%
\pgfpathlineto{\pgfqpoint{1.801914in}{2.421920in}}%
\pgfpathlineto{\pgfqpoint{1.802759in}{2.641156in}}%
\pgfpathlineto{\pgfqpoint{1.805296in}{1.658834in}}%
\pgfpathlineto{\pgfqpoint{1.806142in}{2.627987in}}%
\pgfpathlineto{\pgfqpoint{1.806987in}{1.781857in}}%
\pgfpathlineto{\pgfqpoint{1.807833in}{1.815617in}}%
\pgfpathlineto{\pgfqpoint{1.809524in}{2.591213in}}%
\pgfpathlineto{\pgfqpoint{1.810370in}{1.991735in}}%
\pgfpathlineto{\pgfqpoint{1.811215in}{2.245494in}}%
\pgfpathlineto{\pgfqpoint{1.812061in}{2.022923in}}%
\pgfpathlineto{\pgfqpoint{1.815444in}{2.941016in}}%
\pgfpathlineto{\pgfqpoint{1.816289in}{2.088467in}}%
\pgfpathlineto{\pgfqpoint{1.817135in}{2.354279in}}%
\pgfpathlineto{\pgfqpoint{1.817980in}{2.581434in}}%
\pgfpathlineto{\pgfqpoint{1.818826in}{2.358310in}}%
\pgfpathlineto{\pgfqpoint{1.820517in}{2.637399in}}%
\pgfpathlineto{\pgfqpoint{1.823900in}{1.861248in}}%
\pgfpathlineto{\pgfqpoint{1.825591in}{2.501375in}}%
\pgfpathlineto{\pgfqpoint{1.826437in}{2.331195in}}%
\pgfpathlineto{\pgfqpoint{1.827282in}{2.584246in}}%
\pgfpathlineto{\pgfqpoint{1.828974in}{1.742840in}}%
\pgfpathlineto{\pgfqpoint{1.830665in}{2.423604in}}%
\pgfpathlineto{\pgfqpoint{1.831510in}{1.673324in}}%
\pgfpathlineto{\pgfqpoint{1.832356in}{1.856272in}}%
\pgfpathlineto{\pgfqpoint{1.834047in}{2.546725in}}%
\pgfpathlineto{\pgfqpoint{1.836584in}{2.056456in}}%
\pgfpathlineto{\pgfqpoint{1.837430in}{2.720001in}}%
\pgfpathlineto{\pgfqpoint{1.838275in}{2.096074in}}%
\pgfpathlineto{\pgfqpoint{1.839121in}{2.205169in}}%
\pgfpathlineto{\pgfqpoint{1.839967in}{2.351308in}}%
\pgfpathlineto{\pgfqpoint{1.840812in}{1.839505in}}%
\pgfpathlineto{\pgfqpoint{1.841658in}{1.942843in}}%
\pgfpathlineto{\pgfqpoint{1.844195in}{2.151715in}}%
\pgfpathlineto{\pgfqpoint{1.845040in}{2.016926in}}%
\pgfpathlineto{\pgfqpoint{1.848423in}{2.820537in}}%
\pgfpathlineto{\pgfqpoint{1.849268in}{3.020411in}}%
\pgfpathlineto{\pgfqpoint{1.850960in}{1.909443in}}%
\pgfpathlineto{\pgfqpoint{1.851805in}{2.748866in}}%
\pgfpathlineto{\pgfqpoint{1.852651in}{1.842477in}}%
\pgfpathlineto{\pgfqpoint{1.853497in}{2.436140in}}%
\pgfpathlineto{\pgfqpoint{1.855188in}{1.792676in}}%
\pgfpathlineto{\pgfqpoint{1.856033in}{2.073426in}}%
\pgfpathlineto{\pgfqpoint{1.857725in}{2.674146in}}%
\pgfpathlineto{\pgfqpoint{1.859416in}{1.692989in}}%
\pgfpathlineto{\pgfqpoint{1.862798in}{2.564430in}}%
\pgfpathlineto{\pgfqpoint{1.863644in}{1.305662in}}%
\pgfpathlineto{\pgfqpoint{1.864490in}{2.752934in}}%
\pgfpathlineto{\pgfqpoint{1.865335in}{1.997222in}}%
\pgfpathlineto{\pgfqpoint{1.867872in}{2.722618in}}%
\pgfpathlineto{\pgfqpoint{1.868718in}{2.131519in}}%
\pgfpathlineto{\pgfqpoint{1.869563in}{2.420107in}}%
\pgfpathlineto{\pgfqpoint{1.871255in}{1.907153in}}%
\pgfpathlineto{\pgfqpoint{1.872100in}{2.411452in}}%
\pgfpathlineto{\pgfqpoint{1.872946in}{1.657549in}}%
\pgfpathlineto{\pgfqpoint{1.874637in}{2.559007in}}%
\pgfpathlineto{\pgfqpoint{1.875483in}{2.025545in}}%
\pgfpathlineto{\pgfqpoint{1.876328in}{2.091082in}}%
\pgfpathlineto{\pgfqpoint{1.877174in}{2.125328in}}%
\pgfpathlineto{\pgfqpoint{1.878020in}{1.983157in}}%
\pgfpathlineto{\pgfqpoint{1.880557in}{2.729531in}}%
\pgfpathlineto{\pgfqpoint{1.883093in}{1.754903in}}%
\pgfpathlineto{\pgfqpoint{1.883939in}{1.838019in}}%
\pgfpathlineto{\pgfqpoint{1.884785in}{2.268775in}}%
\pgfpathlineto{\pgfqpoint{1.885630in}{1.968286in}}%
\pgfpathlineto{\pgfqpoint{1.888167in}{1.599290in}}%
\pgfpathlineto{\pgfqpoint{1.889858in}{2.252549in}}%
\pgfpathlineto{\pgfqpoint{1.890704in}{2.053678in}}%
\pgfpathlineto{\pgfqpoint{1.892395in}{2.672139in}}%
\pgfpathlineto{\pgfqpoint{1.895778in}{2.029038in}}%
\pgfpathlineto{\pgfqpoint{1.896623in}{2.396540in}}%
\pgfpathlineto{\pgfqpoint{1.897469in}{2.112086in}}%
\pgfpathlineto{\pgfqpoint{1.899160in}{2.388674in}}%
\pgfpathlineto{\pgfqpoint{1.900006in}{2.293652in}}%
\pgfpathlineto{\pgfqpoint{1.900852in}{2.543780in}}%
\pgfpathlineto{\pgfqpoint{1.903388in}{2.018299in}}%
\pgfpathlineto{\pgfqpoint{1.904234in}{1.927516in}}%
\pgfpathlineto{\pgfqpoint{1.905925in}{2.568224in}}%
\pgfpathlineto{\pgfqpoint{1.906771in}{2.538064in}}%
\pgfpathlineto{\pgfqpoint{1.907617in}{1.551578in}}%
\pgfpathlineto{\pgfqpoint{1.908462in}{2.092944in}}%
\pgfpathlineto{\pgfqpoint{1.910153in}{2.832312in}}%
\pgfpathlineto{\pgfqpoint{1.911845in}{1.883459in}}%
\pgfpathlineto{\pgfqpoint{1.912690in}{2.129102in}}%
\pgfpathlineto{\pgfqpoint{1.913536in}{2.018115in}}%
\pgfpathlineto{\pgfqpoint{1.914382in}{2.729621in}}%
\pgfpathlineto{\pgfqpoint{1.916073in}{1.765003in}}%
\pgfpathlineto{\pgfqpoint{1.917764in}{2.391329in}}%
\pgfpathlineto{\pgfqpoint{1.918610in}{2.388962in}}%
\pgfpathlineto{\pgfqpoint{1.920301in}{1.830339in}}%
\pgfpathlineto{\pgfqpoint{1.921147in}{2.424387in}}%
\pgfpathlineto{\pgfqpoint{1.922838in}{1.603123in}}%
\pgfpathlineto{\pgfqpoint{1.923683in}{1.606346in}}%
\pgfpathlineto{\pgfqpoint{1.924529in}{2.571167in}}%
\pgfpathlineto{\pgfqpoint{1.925375in}{2.268010in}}%
\pgfpathlineto{\pgfqpoint{1.926220in}{1.189513in}}%
\pgfpathlineto{\pgfqpoint{1.928757in}{2.636162in}}%
\pgfpathlineto{\pgfqpoint{1.929603in}{1.885282in}}%
\pgfpathlineto{\pgfqpoint{1.930448in}{1.972045in}}%
\pgfpathlineto{\pgfqpoint{1.931294in}{2.193395in}}%
\pgfpathlineto{\pgfqpoint{1.932140in}{1.836863in}}%
\pgfpathlineto{\pgfqpoint{1.932985in}{1.967310in}}%
\pgfpathlineto{\pgfqpoint{1.933831in}{2.115155in}}%
\pgfpathlineto{\pgfqpoint{1.934676in}{1.941734in}}%
\pgfpathlineto{\pgfqpoint{1.935522in}{2.322671in}}%
\pgfpathlineto{\pgfqpoint{1.936368in}{2.089087in}}%
\pgfpathlineto{\pgfqpoint{1.937213in}{2.365284in}}%
\pgfpathlineto{\pgfqpoint{1.939750in}{1.872946in}}%
\pgfpathlineto{\pgfqpoint{1.942287in}{2.508205in}}%
\pgfpathlineto{\pgfqpoint{1.943133in}{2.298612in}}%
\pgfpathlineto{\pgfqpoint{1.943978in}{2.150701in}}%
\pgfpathlineto{\pgfqpoint{1.944824in}{1.448096in}}%
\pgfpathlineto{\pgfqpoint{1.945670in}{2.005419in}}%
\pgfpathlineto{\pgfqpoint{1.946515in}{3.123674in}}%
\pgfpathlineto{\pgfqpoint{1.947361in}{2.182750in}}%
\pgfpathlineto{\pgfqpoint{1.948206in}{2.549152in}}%
\pgfpathlineto{\pgfqpoint{1.949052in}{3.133330in}}%
\pgfpathlineto{\pgfqpoint{1.949898in}{2.713229in}}%
\pgfpathlineto{\pgfqpoint{1.950743in}{1.648745in}}%
\pgfpathlineto{\pgfqpoint{1.951589in}{2.347526in}}%
\pgfpathlineto{\pgfqpoint{1.952435in}{2.387789in}}%
\pgfpathlineto{\pgfqpoint{1.953280in}{2.279525in}}%
\pgfpathlineto{\pgfqpoint{1.954971in}{2.762912in}}%
\pgfpathlineto{\pgfqpoint{1.956663in}{1.747375in}}%
\pgfpathlineto{\pgfqpoint{1.958354in}{2.499254in}}%
\pgfpathlineto{\pgfqpoint{1.959200in}{1.465103in}}%
\pgfpathlineto{\pgfqpoint{1.960045in}{2.150833in}}%
\pgfpathlineto{\pgfqpoint{1.960891in}{1.753991in}}%
\pgfpathlineto{\pgfqpoint{1.962582in}{2.531575in}}%
\pgfpathlineto{\pgfqpoint{1.964273in}{1.887974in}}%
\pgfpathlineto{\pgfqpoint{1.965119in}{2.450205in}}%
\pgfpathlineto{\pgfqpoint{1.965965in}{2.133580in}}%
\pgfpathlineto{\pgfqpoint{1.966810in}{1.828980in}}%
\pgfpathlineto{\pgfqpoint{1.967656in}{2.371047in}}%
\pgfpathlineto{\pgfqpoint{1.968501in}{2.137890in}}%
\pgfpathlineto{\pgfqpoint{1.969347in}{1.777585in}}%
\pgfpathlineto{\pgfqpoint{1.970193in}{2.510745in}}%
\pgfpathlineto{\pgfqpoint{1.971038in}{1.601730in}}%
\pgfpathlineto{\pgfqpoint{1.971884in}{1.840629in}}%
\pgfpathlineto{\pgfqpoint{1.973575in}{2.596814in}}%
\pgfpathlineto{\pgfqpoint{1.975266in}{1.734747in}}%
\pgfpathlineto{\pgfqpoint{1.976112in}{2.056315in}}%
\pgfpathlineto{\pgfqpoint{1.976958in}{1.784797in}}%
\pgfpathlineto{\pgfqpoint{1.978649in}{2.597147in}}%
\pgfpathlineto{\pgfqpoint{1.979495in}{2.472608in}}%
\pgfpathlineto{\pgfqpoint{1.980340in}{1.913631in}}%
\pgfpathlineto{\pgfqpoint{1.981186in}{2.239297in}}%
\pgfpathlineto{\pgfqpoint{1.982031in}{1.867470in}}%
\pgfpathlineto{\pgfqpoint{1.982877in}{1.904102in}}%
\pgfpathlineto{\pgfqpoint{1.983723in}{2.542215in}}%
\pgfpathlineto{\pgfqpoint{1.984568in}{2.121181in}}%
\pgfpathlineto{\pgfqpoint{1.986260in}{2.845141in}}%
\pgfpathlineto{\pgfqpoint{1.988796in}{1.692232in}}%
\pgfpathlineto{\pgfqpoint{1.989642in}{2.043471in}}%
\pgfpathlineto{\pgfqpoint{1.990488in}{1.494207in}}%
\pgfpathlineto{\pgfqpoint{1.991333in}{1.883704in}}%
\pgfpathlineto{\pgfqpoint{1.992179in}{1.922295in}}%
\pgfpathlineto{\pgfqpoint{1.993025in}{2.820278in}}%
\pgfpathlineto{\pgfqpoint{1.995561in}{1.737018in}}%
\pgfpathlineto{\pgfqpoint{1.998098in}{1.357711in}}%
\pgfpathlineto{\pgfqpoint{1.998944in}{2.368662in}}%
\pgfpathlineto{\pgfqpoint{1.999790in}{2.304991in}}%
\pgfpathlineto{\pgfqpoint{2.000635in}{1.460494in}}%
\pgfpathlineto{\pgfqpoint{2.001481in}{2.007931in}}%
\pgfpathlineto{\pgfqpoint{2.004018in}{2.496862in}}%
\pgfpathlineto{\pgfqpoint{2.004863in}{2.323561in}}%
\pgfpathlineto{\pgfqpoint{2.005709in}{2.752890in}}%
\pgfpathlineto{\pgfqpoint{2.008246in}{1.674816in}}%
\pgfpathlineto{\pgfqpoint{2.009091in}{1.742847in}}%
\pgfpathlineto{\pgfqpoint{2.009937in}{2.553846in}}%
\pgfpathlineto{\pgfqpoint{2.010783in}{2.269036in}}%
\pgfpathlineto{\pgfqpoint{2.011628in}{2.253614in}}%
\pgfpathlineto{\pgfqpoint{2.012474in}{1.416425in}}%
\pgfpathlineto{\pgfqpoint{2.013319in}{1.954455in}}%
\pgfpathlineto{\pgfqpoint{2.014165in}{2.006089in}}%
\pgfpathlineto{\pgfqpoint{2.015011in}{1.463570in}}%
\pgfpathlineto{\pgfqpoint{2.016702in}{2.858612in}}%
\pgfpathlineto{\pgfqpoint{2.019239in}{1.155103in}}%
\pgfpathlineto{\pgfqpoint{2.020084in}{2.185292in}}%
\pgfpathlineto{\pgfqpoint{2.020930in}{2.019642in}}%
\pgfpathlineto{\pgfqpoint{2.021776in}{1.880201in}}%
\pgfpathlineto{\pgfqpoint{2.022621in}{2.777440in}}%
\pgfpathlineto{\pgfqpoint{2.023467in}{1.713116in}}%
\pgfpathlineto{\pgfqpoint{2.024313in}{2.027328in}}%
\pgfpathlineto{\pgfqpoint{2.026849in}{2.584521in}}%
\pgfpathlineto{\pgfqpoint{2.030232in}{1.210933in}}%
\pgfpathlineto{\pgfqpoint{2.031078in}{2.436713in}}%
\pgfpathlineto{\pgfqpoint{2.031923in}{2.137913in}}%
\pgfpathlineto{\pgfqpoint{2.032769in}{2.515005in}}%
\pgfpathlineto{\pgfqpoint{2.033614in}{2.028470in}}%
\pgfpathlineto{\pgfqpoint{2.034460in}{2.879962in}}%
\pgfpathlineto{\pgfqpoint{2.035306in}{1.642860in}}%
\pgfpathlineto{\pgfqpoint{2.036151in}{1.776572in}}%
\pgfpathlineto{\pgfqpoint{2.036997in}{1.597392in}}%
\pgfpathlineto{\pgfqpoint{2.038688in}{2.672068in}}%
\pgfpathlineto{\pgfqpoint{2.039534in}{1.644394in}}%
\pgfpathlineto{\pgfqpoint{2.040379in}{2.822333in}}%
\pgfpathlineto{\pgfqpoint{2.041225in}{2.092905in}}%
\pgfpathlineto{\pgfqpoint{2.042071in}{2.408408in}}%
\pgfpathlineto{\pgfqpoint{2.043762in}{1.685426in}}%
\pgfpathlineto{\pgfqpoint{2.045453in}{2.545867in}}%
\pgfpathlineto{\pgfqpoint{2.047990in}{1.906517in}}%
\pgfpathlineto{\pgfqpoint{2.048836in}{2.058707in}}%
\pgfpathlineto{\pgfqpoint{2.049681in}{2.839563in}}%
\pgfpathlineto{\pgfqpoint{2.050527in}{2.116429in}}%
\pgfpathlineto{\pgfqpoint{2.051373in}{2.244816in}}%
\pgfpathlineto{\pgfqpoint{2.052218in}{2.125857in}}%
\pgfpathlineto{\pgfqpoint{2.053064in}{2.299627in}}%
\pgfpathlineto{\pgfqpoint{2.054755in}{1.823981in}}%
\pgfpathlineto{\pgfqpoint{2.055601in}{2.417210in}}%
\pgfpathlineto{\pgfqpoint{2.056446in}{1.650274in}}%
\pgfpathlineto{\pgfqpoint{2.057292in}{2.128570in}}%
\pgfpathlineto{\pgfqpoint{2.058138in}{1.804632in}}%
\pgfpathlineto{\pgfqpoint{2.058983in}{2.005126in}}%
\pgfpathlineto{\pgfqpoint{2.059829in}{1.941749in}}%
\pgfpathlineto{\pgfqpoint{2.060674in}{2.153011in}}%
\pgfpathlineto{\pgfqpoint{2.061520in}{1.906242in}}%
\pgfpathlineto{\pgfqpoint{2.062366in}{2.069293in}}%
\pgfpathlineto{\pgfqpoint{2.063211in}{2.042437in}}%
\pgfpathlineto{\pgfqpoint{2.064057in}{1.936708in}}%
\pgfpathlineto{\pgfqpoint{2.064903in}{2.160097in}}%
\pgfpathlineto{\pgfqpoint{2.066594in}{1.544321in}}%
\pgfpathlineto{\pgfqpoint{2.067439in}{2.553251in}}%
\pgfpathlineto{\pgfqpoint{2.068285in}{2.264595in}}%
\pgfpathlineto{\pgfqpoint{2.069131in}{2.115628in}}%
\pgfpathlineto{\pgfqpoint{2.069976in}{1.328891in}}%
\pgfpathlineto{\pgfqpoint{2.070822in}{2.737243in}}%
\pgfpathlineto{\pgfqpoint{2.071668in}{1.940999in}}%
\pgfpathlineto{\pgfqpoint{2.074204in}{2.658348in}}%
\pgfpathlineto{\pgfqpoint{2.075050in}{2.226188in}}%
\pgfpathlineto{\pgfqpoint{2.075896in}{2.852315in}}%
\pgfpathlineto{\pgfqpoint{2.076741in}{1.621961in}}%
\pgfpathlineto{\pgfqpoint{2.077587in}{1.939543in}}%
\pgfpathlineto{\pgfqpoint{2.079278in}{2.090209in}}%
\pgfpathlineto{\pgfqpoint{2.080124in}{2.050697in}}%
\pgfpathlineto{\pgfqpoint{2.080969in}{1.559297in}}%
\pgfpathlineto{\pgfqpoint{2.081815in}{2.441506in}}%
\pgfpathlineto{\pgfqpoint{2.082661in}{2.426255in}}%
\pgfpathlineto{\pgfqpoint{2.083506in}{1.841694in}}%
\pgfpathlineto{\pgfqpoint{2.084352in}{2.556613in}}%
\pgfpathlineto{\pgfqpoint{2.085197in}{2.161538in}}%
\pgfpathlineto{\pgfqpoint{2.086889in}{1.827706in}}%
\pgfpathlineto{\pgfqpoint{2.088580in}{2.719470in}}%
\pgfpathlineto{\pgfqpoint{2.089426in}{1.998193in}}%
\pgfpathlineto{\pgfqpoint{2.090271in}{2.262505in}}%
\pgfpathlineto{\pgfqpoint{2.091117in}{2.324450in}}%
\pgfpathlineto{\pgfqpoint{2.091962in}{1.821879in}}%
\pgfpathlineto{\pgfqpoint{2.092808in}{2.091472in}}%
\pgfpathlineto{\pgfqpoint{2.093654in}{2.600181in}}%
\pgfpathlineto{\pgfqpoint{2.096191in}{2.013557in}}%
\pgfpathlineto{\pgfqpoint{2.097882in}{1.684349in}}%
\pgfpathlineto{\pgfqpoint{2.098727in}{2.576814in}}%
\pgfpathlineto{\pgfqpoint{2.099573in}{1.708891in}}%
\pgfpathlineto{\pgfqpoint{2.100419in}{2.198136in}}%
\pgfpathlineto{\pgfqpoint{2.101264in}{2.483745in}}%
\pgfpathlineto{\pgfqpoint{2.102110in}{1.788491in}}%
\pgfpathlineto{\pgfqpoint{2.103801in}{2.621736in}}%
\pgfpathlineto{\pgfqpoint{2.104647in}{2.083077in}}%
\pgfpathlineto{\pgfqpoint{2.105492in}{2.303209in}}%
\pgfpathlineto{\pgfqpoint{2.106338in}{2.519284in}}%
\pgfpathlineto{\pgfqpoint{2.108029in}{2.082216in}}%
\pgfpathlineto{\pgfqpoint{2.108875in}{2.122989in}}%
\pgfpathlineto{\pgfqpoint{2.109721in}{2.127966in}}%
\pgfpathlineto{\pgfqpoint{2.111412in}{2.503700in}}%
\pgfpathlineto{\pgfqpoint{2.112257in}{1.466873in}}%
\pgfpathlineto{\pgfqpoint{2.113103in}{2.686213in}}%
\pgfpathlineto{\pgfqpoint{2.113949in}{2.177329in}}%
\pgfpathlineto{\pgfqpoint{2.115640in}{1.886779in}}%
\pgfpathlineto{\pgfqpoint{2.116486in}{2.076051in}}%
\pgfpathlineto{\pgfqpoint{2.117331in}{1.415462in}}%
\pgfpathlineto{\pgfqpoint{2.118177in}{1.616003in}}%
\pgfpathlineto{\pgfqpoint{2.119022in}{1.776349in}}%
\pgfpathlineto{\pgfqpoint{2.119868in}{2.452730in}}%
\pgfpathlineto{\pgfqpoint{2.120714in}{2.110331in}}%
\pgfpathlineto{\pgfqpoint{2.121559in}{2.530022in}}%
\pgfpathlineto{\pgfqpoint{2.122405in}{1.446217in}}%
\pgfpathlineto{\pgfqpoint{2.123251in}{1.891465in}}%
\pgfpathlineto{\pgfqpoint{2.125787in}{2.506584in}}%
\pgfpathlineto{\pgfqpoint{2.126633in}{1.786236in}}%
\pgfpathlineto{\pgfqpoint{2.127479in}{2.360352in}}%
\pgfpathlineto{\pgfqpoint{2.128324in}{2.578897in}}%
\pgfpathlineto{\pgfqpoint{2.130016in}{1.739890in}}%
\pgfpathlineto{\pgfqpoint{2.130861in}{2.356058in}}%
\pgfpathlineto{\pgfqpoint{2.131707in}{2.071277in}}%
\pgfpathlineto{\pgfqpoint{2.133398in}{2.226175in}}%
\pgfpathlineto{\pgfqpoint{2.134244in}{1.859939in}}%
\pgfpathlineto{\pgfqpoint{2.137626in}{3.270512in}}%
\pgfpathlineto{\pgfqpoint{2.139317in}{1.787329in}}%
\pgfpathlineto{\pgfqpoint{2.140163in}{1.851135in}}%
\pgfpathlineto{\pgfqpoint{2.141009in}{2.541953in}}%
\pgfpathlineto{\pgfqpoint{2.141854in}{2.280731in}}%
\pgfpathlineto{\pgfqpoint{2.142700in}{2.352966in}}%
\pgfpathlineto{\pgfqpoint{2.143546in}{2.043823in}}%
\pgfpathlineto{\pgfqpoint{2.144391in}{2.263743in}}%
\pgfpathlineto{\pgfqpoint{2.145237in}{2.307823in}}%
\pgfpathlineto{\pgfqpoint{2.146082in}{2.262209in}}%
\pgfpathlineto{\pgfqpoint{2.146928in}{1.946138in}}%
\pgfpathlineto{\pgfqpoint{2.147774in}{2.627599in}}%
\pgfpathlineto{\pgfqpoint{2.148619in}{1.960559in}}%
\pgfpathlineto{\pgfqpoint{2.149465in}{2.445176in}}%
\pgfpathlineto{\pgfqpoint{2.150311in}{1.914487in}}%
\pgfpathlineto{\pgfqpoint{2.151156in}{2.081068in}}%
\pgfpathlineto{\pgfqpoint{2.152002in}{2.370615in}}%
\pgfpathlineto{\pgfqpoint{2.152847in}{1.824191in}}%
\pgfpathlineto{\pgfqpoint{2.153693in}{2.049599in}}%
\pgfpathlineto{\pgfqpoint{2.154539in}{2.028838in}}%
\pgfpathlineto{\pgfqpoint{2.157076in}{2.871582in}}%
\pgfpathlineto{\pgfqpoint{2.159612in}{1.733549in}}%
\pgfpathlineto{\pgfqpoint{2.162149in}{2.534968in}}%
\pgfpathlineto{\pgfqpoint{2.162995in}{2.166031in}}%
\pgfpathlineto{\pgfqpoint{2.163840in}{2.216870in}}%
\pgfpathlineto{\pgfqpoint{2.165532in}{2.622330in}}%
\pgfpathlineto{\pgfqpoint{2.167223in}{1.989746in}}%
\pgfpathlineto{\pgfqpoint{2.168069in}{2.436091in}}%
\pgfpathlineto{\pgfqpoint{2.168914in}{1.329128in}}%
\pgfpathlineto{\pgfqpoint{2.169760in}{2.439442in}}%
\pgfpathlineto{\pgfqpoint{2.170605in}{2.152273in}}%
\pgfpathlineto{\pgfqpoint{2.172297in}{2.505102in}}%
\pgfpathlineto{\pgfqpoint{2.173142in}{2.066108in}}%
\pgfpathlineto{\pgfqpoint{2.173988in}{2.523841in}}%
\pgfpathlineto{\pgfqpoint{2.174834in}{1.690567in}}%
\pgfpathlineto{\pgfqpoint{2.175679in}{2.239359in}}%
\pgfpathlineto{\pgfqpoint{2.176525in}{2.679762in}}%
\pgfpathlineto{\pgfqpoint{2.177370in}{2.058442in}}%
\pgfpathlineto{\pgfqpoint{2.178216in}{2.273994in}}%
\pgfpathlineto{\pgfqpoint{2.179062in}{1.966385in}}%
\pgfpathlineto{\pgfqpoint{2.180753in}{2.509057in}}%
\pgfpathlineto{\pgfqpoint{2.181599in}{2.440484in}}%
\pgfpathlineto{\pgfqpoint{2.183290in}{1.998205in}}%
\pgfpathlineto{\pgfqpoint{2.184135in}{2.379133in}}%
\pgfpathlineto{\pgfqpoint{2.184981in}{2.048727in}}%
\pgfpathlineto{\pgfqpoint{2.185827in}{2.280236in}}%
\pgfpathlineto{\pgfqpoint{2.187518in}{2.774620in}}%
\pgfpathlineto{\pgfqpoint{2.188364in}{2.263364in}}%
\pgfpathlineto{\pgfqpoint{2.189209in}{2.445578in}}%
\pgfpathlineto{\pgfqpoint{2.190055in}{2.341293in}}%
\pgfpathlineto{\pgfqpoint{2.190900in}{2.661883in}}%
\pgfpathlineto{\pgfqpoint{2.191746in}{2.362865in}}%
\pgfpathlineto{\pgfqpoint{2.192592in}{2.390724in}}%
\pgfpathlineto{\pgfqpoint{2.193437in}{2.393450in}}%
\pgfpathlineto{\pgfqpoint{2.194283in}{2.205256in}}%
\pgfpathlineto{\pgfqpoint{2.195129in}{2.616372in}}%
\pgfpathlineto{\pgfqpoint{2.195974in}{2.150186in}}%
\pgfpathlineto{\pgfqpoint{2.196820in}{2.605240in}}%
\pgfpathlineto{\pgfqpoint{2.197665in}{2.584959in}}%
\pgfpathlineto{\pgfqpoint{2.198511in}{1.899521in}}%
\pgfpathlineto{\pgfqpoint{2.199357in}{2.004169in}}%
\pgfpathlineto{\pgfqpoint{2.200202in}{2.130083in}}%
\pgfpathlineto{\pgfqpoint{2.201048in}{1.581321in}}%
\pgfpathlineto{\pgfqpoint{2.201894in}{1.881070in}}%
\pgfpathlineto{\pgfqpoint{2.202739in}{1.896205in}}%
\pgfpathlineto{\pgfqpoint{2.203585in}{2.410933in}}%
\pgfpathlineto{\pgfqpoint{2.204430in}{1.539249in}}%
\pgfpathlineto{\pgfqpoint{2.205276in}{2.587751in}}%
\pgfpathlineto{\pgfqpoint{2.206122in}{2.180327in}}%
\pgfpathlineto{\pgfqpoint{2.206967in}{2.367205in}}%
\pgfpathlineto{\pgfqpoint{2.210350in}{1.360722in}}%
\pgfpathlineto{\pgfqpoint{2.212041in}{1.985938in}}%
\pgfpathlineto{\pgfqpoint{2.212887in}{2.606589in}}%
\pgfpathlineto{\pgfqpoint{2.213732in}{1.856233in}}%
\pgfpathlineto{\pgfqpoint{2.215424in}{2.793250in}}%
\pgfpathlineto{\pgfqpoint{2.217115in}{1.806533in}}%
\pgfpathlineto{\pgfqpoint{2.217960in}{2.061832in}}%
\pgfpathlineto{\pgfqpoint{2.218806in}{1.759872in}}%
\pgfpathlineto{\pgfqpoint{2.219652in}{1.888377in}}%
\pgfpathlineto{\pgfqpoint{2.220497in}{2.374766in}}%
\pgfpathlineto{\pgfqpoint{2.221343in}{1.795763in}}%
\pgfpathlineto{\pgfqpoint{2.222189in}{2.699923in}}%
\pgfpathlineto{\pgfqpoint{2.223034in}{2.195716in}}%
\pgfpathlineto{\pgfqpoint{2.223880in}{2.264369in}}%
\pgfpathlineto{\pgfqpoint{2.224725in}{2.122239in}}%
\pgfpathlineto{\pgfqpoint{2.225571in}{2.436589in}}%
\pgfpathlineto{\pgfqpoint{2.226417in}{1.857050in}}%
\pgfpathlineto{\pgfqpoint{2.228954in}{2.783393in}}%
\pgfpathlineto{\pgfqpoint{2.230645in}{2.049291in}}%
\pgfpathlineto{\pgfqpoint{2.231490in}{2.610181in}}%
\pgfpathlineto{\pgfqpoint{2.232336in}{2.582349in}}%
\pgfpathlineto{\pgfqpoint{2.233182in}{1.746658in}}%
\pgfpathlineto{\pgfqpoint{2.234027in}{2.395671in}}%
\pgfpathlineto{\pgfqpoint{2.235719in}{1.918890in}}%
\pgfpathlineto{\pgfqpoint{2.237410in}{2.049331in}}%
\pgfpathlineto{\pgfqpoint{2.238255in}{0.906709in}}%
\pgfpathlineto{\pgfqpoint{2.239101in}{2.573283in}}%
\pgfpathlineto{\pgfqpoint{2.239947in}{2.219401in}}%
\pgfpathlineto{\pgfqpoint{2.240792in}{1.420643in}}%
\pgfpathlineto{\pgfqpoint{2.241638in}{2.386716in}}%
\pgfpathlineto{\pgfqpoint{2.242483in}{1.997911in}}%
\pgfpathlineto{\pgfqpoint{2.243329in}{2.158610in}}%
\pgfpathlineto{\pgfqpoint{2.244175in}{1.783848in}}%
\pgfpathlineto{\pgfqpoint{2.245020in}{2.427718in}}%
\pgfpathlineto{\pgfqpoint{2.245866in}{2.171970in}}%
\pgfpathlineto{\pgfqpoint{2.246712in}{1.743315in}}%
\pgfpathlineto{\pgfqpoint{2.247557in}{1.929643in}}%
\pgfpathlineto{\pgfqpoint{2.248403in}{2.406211in}}%
\pgfpathlineto{\pgfqpoint{2.249248in}{1.372703in}}%
\pgfpathlineto{\pgfqpoint{2.250094in}{2.736638in}}%
\pgfpathlineto{\pgfqpoint{2.250940in}{2.288045in}}%
\pgfpathlineto{\pgfqpoint{2.251785in}{2.876005in}}%
\pgfpathlineto{\pgfqpoint{2.252631in}{2.241146in}}%
\pgfpathlineto{\pgfqpoint{2.253477in}{2.403363in}}%
\pgfpathlineto{\pgfqpoint{2.254322in}{2.959918in}}%
\pgfpathlineto{\pgfqpoint{2.255168in}{1.861547in}}%
\pgfpathlineto{\pgfqpoint{2.256013in}{2.560940in}}%
\pgfpathlineto{\pgfqpoint{2.256859in}{2.158282in}}%
\pgfpathlineto{\pgfqpoint{2.257705in}{2.217645in}}%
\pgfpathlineto{\pgfqpoint{2.258550in}{2.268305in}}%
\pgfpathlineto{\pgfqpoint{2.259396in}{2.913504in}}%
\pgfpathlineto{\pgfqpoint{2.260242in}{2.389433in}}%
\pgfpathlineto{\pgfqpoint{2.261087in}{2.243678in}}%
\pgfpathlineto{\pgfqpoint{2.261933in}{2.258312in}}%
\pgfpathlineto{\pgfqpoint{2.262778in}{2.557171in}}%
\pgfpathlineto{\pgfqpoint{2.264470in}{1.663017in}}%
\pgfpathlineto{\pgfqpoint{2.267007in}{2.753077in}}%
\pgfpathlineto{\pgfqpoint{2.269543in}{2.000521in}}%
\pgfpathlineto{\pgfqpoint{2.270389in}{1.832367in}}%
\pgfpathlineto{\pgfqpoint{2.271235in}{2.347774in}}%
\pgfpathlineto{\pgfqpoint{2.272080in}{2.077166in}}%
\pgfpathlineto{\pgfqpoint{2.272926in}{2.430371in}}%
\pgfpathlineto{\pgfqpoint{2.273772in}{1.899127in}}%
\pgfpathlineto{\pgfqpoint{2.275463in}{2.630974in}}%
\pgfpathlineto{\pgfqpoint{2.278845in}{1.425766in}}%
\pgfpathlineto{\pgfqpoint{2.281382in}{2.634545in}}%
\pgfpathlineto{\pgfqpoint{2.283073in}{2.666860in}}%
\pgfpathlineto{\pgfqpoint{2.283919in}{1.813240in}}%
\pgfpathlineto{\pgfqpoint{2.284765in}{2.144618in}}%
\pgfpathlineto{\pgfqpoint{2.285610in}{2.660225in}}%
\pgfpathlineto{\pgfqpoint{2.286456in}{2.197947in}}%
\pgfpathlineto{\pgfqpoint{2.287302in}{2.575596in}}%
\pgfpathlineto{\pgfqpoint{2.288147in}{2.281591in}}%
\pgfpathlineto{\pgfqpoint{2.288993in}{1.307474in}}%
\pgfpathlineto{\pgfqpoint{2.289838in}{2.170715in}}%
\pgfpathlineto{\pgfqpoint{2.290684in}{1.892406in}}%
\pgfpathlineto{\pgfqpoint{2.292375in}{2.768143in}}%
\pgfpathlineto{\pgfqpoint{2.294067in}{1.926230in}}%
\pgfpathlineto{\pgfqpoint{2.294912in}{2.304318in}}%
\pgfpathlineto{\pgfqpoint{2.295758in}{2.270980in}}%
\pgfpathlineto{\pgfqpoint{2.296603in}{1.571316in}}%
\pgfpathlineto{\pgfqpoint{2.297449in}{2.521852in}}%
\pgfpathlineto{\pgfqpoint{2.298295in}{2.041150in}}%
\pgfpathlineto{\pgfqpoint{2.299140in}{1.844390in}}%
\pgfpathlineto{\pgfqpoint{2.299986in}{1.914898in}}%
\pgfpathlineto{\pgfqpoint{2.300832in}{2.050902in}}%
\pgfpathlineto{\pgfqpoint{2.301677in}{2.800453in}}%
\pgfpathlineto{\pgfqpoint{2.302523in}{2.560445in}}%
\pgfpathlineto{\pgfqpoint{2.303368in}{1.789649in}}%
\pgfpathlineto{\pgfqpoint{2.304214in}{1.959590in}}%
\pgfpathlineto{\pgfqpoint{2.305060in}{2.330307in}}%
\pgfpathlineto{\pgfqpoint{2.306751in}{1.617465in}}%
\pgfpathlineto{\pgfqpoint{2.307597in}{2.829706in}}%
\pgfpathlineto{\pgfqpoint{2.308442in}{2.621261in}}%
\pgfpathlineto{\pgfqpoint{2.309288in}{2.700126in}}%
\pgfpathlineto{\pgfqpoint{2.310979in}{2.115546in}}%
\pgfpathlineto{\pgfqpoint{2.312670in}{2.426934in}}%
\pgfpathlineto{\pgfqpoint{2.313516in}{1.893645in}}%
\pgfpathlineto{\pgfqpoint{2.314362in}{2.089081in}}%
\pgfpathlineto{\pgfqpoint{2.315207in}{2.706435in}}%
\pgfpathlineto{\pgfqpoint{2.316053in}{2.100579in}}%
\pgfpathlineto{\pgfqpoint{2.316898in}{2.291910in}}%
\pgfpathlineto{\pgfqpoint{2.317744in}{2.445751in}}%
\pgfpathlineto{\pgfqpoint{2.319435in}{1.858084in}}%
\pgfpathlineto{\pgfqpoint{2.320281in}{2.083245in}}%
\pgfpathlineto{\pgfqpoint{2.321972in}{1.699696in}}%
\pgfpathlineto{\pgfqpoint{2.322818in}{2.061689in}}%
\pgfpathlineto{\pgfqpoint{2.323663in}{1.801331in}}%
\pgfpathlineto{\pgfqpoint{2.325355in}{2.359103in}}%
\pgfpathlineto{\pgfqpoint{2.326200in}{2.282491in}}%
\pgfpathlineto{\pgfqpoint{2.327046in}{2.138383in}}%
\pgfpathlineto{\pgfqpoint{2.327891in}{1.382586in}}%
\pgfpathlineto{\pgfqpoint{2.330428in}{2.458357in}}%
\pgfpathlineto{\pgfqpoint{2.331274in}{2.372059in}}%
\pgfpathlineto{\pgfqpoint{2.332120in}{2.400975in}}%
\pgfpathlineto{\pgfqpoint{2.332965in}{1.756303in}}%
\pgfpathlineto{\pgfqpoint{2.333811in}{2.803357in}}%
\pgfpathlineto{\pgfqpoint{2.334656in}{1.460746in}}%
\pgfpathlineto{\pgfqpoint{2.335502in}{2.684468in}}%
\pgfpathlineto{\pgfqpoint{2.336348in}{2.298607in}}%
\pgfpathlineto{\pgfqpoint{2.337193in}{2.779073in}}%
\pgfpathlineto{\pgfqpoint{2.338885in}{1.589122in}}%
\pgfpathlineto{\pgfqpoint{2.340576in}{2.239504in}}%
\pgfpathlineto{\pgfqpoint{2.341421in}{2.925242in}}%
\pgfpathlineto{\pgfqpoint{2.342267in}{2.481695in}}%
\pgfpathlineto{\pgfqpoint{2.343113in}{1.985809in}}%
\pgfpathlineto{\pgfqpoint{2.343958in}{2.175751in}}%
\pgfpathlineto{\pgfqpoint{2.344804in}{2.742804in}}%
\pgfpathlineto{\pgfqpoint{2.345650in}{2.413326in}}%
\pgfpathlineto{\pgfqpoint{2.346495in}{2.382876in}}%
\pgfpathlineto{\pgfqpoint{2.347341in}{2.404476in}}%
\pgfpathlineto{\pgfqpoint{2.348186in}{2.802724in}}%
\pgfpathlineto{\pgfqpoint{2.349032in}{1.868828in}}%
\pgfpathlineto{\pgfqpoint{2.349878in}{2.341311in}}%
\pgfpathlineto{\pgfqpoint{2.350723in}{2.352088in}}%
\pgfpathlineto{\pgfqpoint{2.351569in}{2.291668in}}%
\pgfpathlineto{\pgfqpoint{2.352415in}{2.363885in}}%
\pgfpathlineto{\pgfqpoint{2.353260in}{2.186297in}}%
\pgfpathlineto{\pgfqpoint{2.354106in}{2.488965in}}%
\pgfpathlineto{\pgfqpoint{2.356643in}{1.377389in}}%
\pgfpathlineto{\pgfqpoint{2.357488in}{2.438464in}}%
\pgfpathlineto{\pgfqpoint{2.358334in}{2.386908in}}%
\pgfpathlineto{\pgfqpoint{2.359180in}{2.215603in}}%
\pgfpathlineto{\pgfqpoint{2.360025in}{2.635500in}}%
\pgfpathlineto{\pgfqpoint{2.362562in}{1.521895in}}%
\pgfpathlineto{\pgfqpoint{2.365099in}{2.520590in}}%
\pgfpathlineto{\pgfqpoint{2.365945in}{2.914759in}}%
\pgfpathlineto{\pgfqpoint{2.366790in}{1.711916in}}%
\pgfpathlineto{\pgfqpoint{2.367636in}{1.865562in}}%
\pgfpathlineto{\pgfqpoint{2.368481in}{1.995441in}}%
\pgfpathlineto{\pgfqpoint{2.370173in}{2.843875in}}%
\pgfpathlineto{\pgfqpoint{2.371864in}{1.904903in}}%
\pgfpathlineto{\pgfqpoint{2.372710in}{2.339339in}}%
\pgfpathlineto{\pgfqpoint{2.373555in}{1.840868in}}%
\pgfpathlineto{\pgfqpoint{2.374401in}{2.342208in}}%
\pgfpathlineto{\pgfqpoint{2.375246in}{2.030569in}}%
\pgfpathlineto{\pgfqpoint{2.376092in}{2.086392in}}%
\pgfpathlineto{\pgfqpoint{2.376938in}{2.481090in}}%
\pgfpathlineto{\pgfqpoint{2.377783in}{2.318500in}}%
\pgfpathlineto{\pgfqpoint{2.378629in}{2.472960in}}%
\pgfpathlineto{\pgfqpoint{2.380320in}{1.715129in}}%
\pgfpathlineto{\pgfqpoint{2.382011in}{2.577304in}}%
\pgfpathlineto{\pgfqpoint{2.384548in}{1.790858in}}%
\pgfpathlineto{\pgfqpoint{2.385394in}{2.756937in}}%
\pgfpathlineto{\pgfqpoint{2.386240in}{2.368683in}}%
\pgfpathlineto{\pgfqpoint{2.387931in}{2.428960in}}%
\pgfpathlineto{\pgfqpoint{2.388776in}{2.344844in}}%
\pgfpathlineto{\pgfqpoint{2.389622in}{2.542006in}}%
\pgfpathlineto{\pgfqpoint{2.391313in}{2.027861in}}%
\pgfpathlineto{\pgfqpoint{2.393850in}{2.466862in}}%
\pgfpathlineto{\pgfqpoint{2.394696in}{1.634216in}}%
\pgfpathlineto{\pgfqpoint{2.395541in}{2.308714in}}%
\pgfpathlineto{\pgfqpoint{2.398078in}{1.740156in}}%
\pgfpathlineto{\pgfqpoint{2.398924in}{1.812067in}}%
\pgfpathlineto{\pgfqpoint{2.400615in}{2.738317in}}%
\pgfpathlineto{\pgfqpoint{2.401461in}{2.543075in}}%
\pgfpathlineto{\pgfqpoint{2.403152in}{2.384986in}}%
\pgfpathlineto{\pgfqpoint{2.403998in}{2.650832in}}%
\pgfpathlineto{\pgfqpoint{2.404843in}{2.080312in}}%
\pgfpathlineto{\pgfqpoint{2.405689in}{2.420767in}}%
\pgfpathlineto{\pgfqpoint{2.408226in}{1.675551in}}%
\pgfpathlineto{\pgfqpoint{2.410763in}{2.473276in}}%
\pgfpathlineto{\pgfqpoint{2.411608in}{1.581794in}}%
\pgfpathlineto{\pgfqpoint{2.412454in}{1.889495in}}%
\pgfpathlineto{\pgfqpoint{2.414991in}{2.508546in}}%
\pgfpathlineto{\pgfqpoint{2.415836in}{2.215343in}}%
\pgfpathlineto{\pgfqpoint{2.416682in}{2.937944in}}%
\pgfpathlineto{\pgfqpoint{2.417528in}{2.599423in}}%
\pgfpathlineto{\pgfqpoint{2.418373in}{2.549472in}}%
\pgfpathlineto{\pgfqpoint{2.420064in}{2.751059in}}%
\pgfpathlineto{\pgfqpoint{2.421756in}{1.740521in}}%
\pgfpathlineto{\pgfqpoint{2.423447in}{1.941362in}}%
\pgfpathlineto{\pgfqpoint{2.424293in}{2.492023in}}%
\pgfpathlineto{\pgfqpoint{2.425138in}{2.426864in}}%
\pgfpathlineto{\pgfqpoint{2.425984in}{2.190349in}}%
\pgfpathlineto{\pgfqpoint{2.426829in}{2.454729in}}%
\pgfpathlineto{\pgfqpoint{2.428521in}{1.880397in}}%
\pgfpathlineto{\pgfqpoint{2.430212in}{2.344577in}}%
\pgfpathlineto{\pgfqpoint{2.431058in}{2.015766in}}%
\pgfpathlineto{\pgfqpoint{2.431903in}{2.236077in}}%
\pgfpathlineto{\pgfqpoint{2.433594in}{3.140746in}}%
\pgfpathlineto{\pgfqpoint{2.434440in}{1.642218in}}%
\pgfpathlineto{\pgfqpoint{2.435286in}{2.031372in}}%
\pgfpathlineto{\pgfqpoint{2.436131in}{2.304814in}}%
\pgfpathlineto{\pgfqpoint{2.436977in}{1.760731in}}%
\pgfpathlineto{\pgfqpoint{2.437823in}{2.069728in}}%
\pgfpathlineto{\pgfqpoint{2.440359in}{2.807978in}}%
\pgfpathlineto{\pgfqpoint{2.441205in}{2.108263in}}%
\pgfpathlineto{\pgfqpoint{2.442051in}{2.472395in}}%
\pgfpathlineto{\pgfqpoint{2.442896in}{3.065478in}}%
\pgfpathlineto{\pgfqpoint{2.445433in}{1.950374in}}%
\pgfpathlineto{\pgfqpoint{2.446279in}{1.841490in}}%
\pgfpathlineto{\pgfqpoint{2.447970in}{2.377765in}}%
\pgfpathlineto{\pgfqpoint{2.449661in}{3.007574in}}%
\pgfpathlineto{\pgfqpoint{2.452198in}{1.792394in}}%
\pgfpathlineto{\pgfqpoint{2.453044in}{2.240949in}}%
\pgfpathlineto{\pgfqpoint{2.453889in}{1.770479in}}%
\pgfpathlineto{\pgfqpoint{2.454735in}{2.470308in}}%
\pgfpathlineto{\pgfqpoint{2.455581in}{2.082873in}}%
\pgfpathlineto{\pgfqpoint{2.456426in}{1.928778in}}%
\pgfpathlineto{\pgfqpoint{2.458118in}{2.884204in}}%
\pgfpathlineto{\pgfqpoint{2.461500in}{1.817805in}}%
\pgfpathlineto{\pgfqpoint{2.463191in}{2.726545in}}%
\pgfpathlineto{\pgfqpoint{2.464037in}{1.443020in}}%
\pgfpathlineto{\pgfqpoint{2.464883in}{1.900383in}}%
\pgfpathlineto{\pgfqpoint{2.466574in}{2.427146in}}%
\pgfpathlineto{\pgfqpoint{2.467419in}{1.945102in}}%
\pgfpathlineto{\pgfqpoint{2.468265in}{2.008893in}}%
\pgfpathlineto{\pgfqpoint{2.469111in}{2.316180in}}%
\pgfpathlineto{\pgfqpoint{2.469956in}{1.977983in}}%
\pgfpathlineto{\pgfqpoint{2.470802in}{2.578745in}}%
\pgfpathlineto{\pgfqpoint{2.471648in}{2.510011in}}%
\pgfpathlineto{\pgfqpoint{2.473339in}{1.814158in}}%
\pgfpathlineto{\pgfqpoint{2.474184in}{2.211698in}}%
\pgfpathlineto{\pgfqpoint{2.476721in}{2.594762in}}%
\pgfpathlineto{\pgfqpoint{2.477567in}{2.345928in}}%
\pgfpathlineto{\pgfqpoint{2.478413in}{2.662942in}}%
\pgfpathlineto{\pgfqpoint{2.479258in}{2.516826in}}%
\pgfpathlineto{\pgfqpoint{2.480104in}{2.094375in}}%
\pgfpathlineto{\pgfqpoint{2.480949in}{2.574753in}}%
\pgfpathlineto{\pgfqpoint{2.482641in}{1.888420in}}%
\pgfpathlineto{\pgfqpoint{2.484332in}{2.339079in}}%
\pgfpathlineto{\pgfqpoint{2.485177in}{1.654377in}}%
\pgfpathlineto{\pgfqpoint{2.486023in}{1.856253in}}%
\pgfpathlineto{\pgfqpoint{2.486869in}{2.613166in}}%
\pgfpathlineto{\pgfqpoint{2.487714in}{2.406914in}}%
\pgfpathlineto{\pgfqpoint{2.488560in}{1.870951in}}%
\pgfpathlineto{\pgfqpoint{2.489406in}{2.538116in}}%
\pgfpathlineto{\pgfqpoint{2.490251in}{1.956831in}}%
\pgfpathlineto{\pgfqpoint{2.491097in}{2.526397in}}%
\pgfpathlineto{\pgfqpoint{2.491942in}{2.163552in}}%
\pgfpathlineto{\pgfqpoint{2.492788in}{2.440312in}}%
\pgfpathlineto{\pgfqpoint{2.493634in}{2.137830in}}%
\pgfpathlineto{\pgfqpoint{2.494479in}{2.555063in}}%
\pgfpathlineto{\pgfqpoint{2.497016in}{1.812594in}}%
\pgfpathlineto{\pgfqpoint{2.498707in}{2.264774in}}%
\pgfpathlineto{\pgfqpoint{2.499553in}{1.868131in}}%
\pgfpathlineto{\pgfqpoint{2.500399in}{2.666825in}}%
\pgfpathlineto{\pgfqpoint{2.501244in}{2.142236in}}%
\pgfpathlineto{\pgfqpoint{2.502936in}{2.718072in}}%
\pgfpathlineto{\pgfqpoint{2.506318in}{2.276091in}}%
\pgfpathlineto{\pgfqpoint{2.508009in}{2.382891in}}%
\pgfpathlineto{\pgfqpoint{2.508855in}{2.229862in}}%
\pgfpathlineto{\pgfqpoint{2.509701in}{2.351266in}}%
\pgfpathlineto{\pgfqpoint{2.510546in}{1.623795in}}%
\pgfpathlineto{\pgfqpoint{2.511392in}{2.251233in}}%
\pgfpathlineto{\pgfqpoint{2.512237in}{2.202348in}}%
\pgfpathlineto{\pgfqpoint{2.513929in}{2.052579in}}%
\pgfpathlineto{\pgfqpoint{2.515620in}{2.558803in}}%
\pgfpathlineto{\pgfqpoint{2.516466in}{2.542337in}}%
\pgfpathlineto{\pgfqpoint{2.518157in}{1.886787in}}%
\pgfpathlineto{\pgfqpoint{2.519848in}{1.483163in}}%
\pgfpathlineto{\pgfqpoint{2.521539in}{2.783881in}}%
\pgfpathlineto{\pgfqpoint{2.522385in}{1.664547in}}%
\pgfpathlineto{\pgfqpoint{2.523231in}{2.042510in}}%
\pgfpathlineto{\pgfqpoint{2.524076in}{2.397580in}}%
\pgfpathlineto{\pgfqpoint{2.524922in}{2.116675in}}%
\pgfpathlineto{\pgfqpoint{2.525767in}{2.090906in}}%
\pgfpathlineto{\pgfqpoint{2.526613in}{2.936408in}}%
\pgfpathlineto{\pgfqpoint{2.527459in}{2.738294in}}%
\pgfpathlineto{\pgfqpoint{2.528304in}{1.855466in}}%
\pgfpathlineto{\pgfqpoint{2.529150in}{2.678766in}}%
\pgfpathlineto{\pgfqpoint{2.529996in}{2.337199in}}%
\pgfpathlineto{\pgfqpoint{2.530841in}{2.336379in}}%
\pgfpathlineto{\pgfqpoint{2.531687in}{2.158646in}}%
\pgfpathlineto{\pgfqpoint{2.532532in}{2.554352in}}%
\pgfpathlineto{\pgfqpoint{2.533378in}{2.014088in}}%
\pgfpathlineto{\pgfqpoint{2.534224in}{2.632123in}}%
\pgfpathlineto{\pgfqpoint{2.535069in}{1.537980in}}%
\pgfpathlineto{\pgfqpoint{2.535915in}{1.732566in}}%
\pgfpathlineto{\pgfqpoint{2.536761in}{1.901238in}}%
\pgfpathlineto{\pgfqpoint{2.538452in}{2.993031in}}%
\pgfpathlineto{\pgfqpoint{2.539297in}{2.794392in}}%
\pgfpathlineto{\pgfqpoint{2.540143in}{1.594021in}}%
\pgfpathlineto{\pgfqpoint{2.540989in}{2.209525in}}%
\pgfpathlineto{\pgfqpoint{2.541834in}{1.545933in}}%
\pgfpathlineto{\pgfqpoint{2.542680in}{2.270394in}}%
\pgfpathlineto{\pgfqpoint{2.543526in}{2.203322in}}%
\pgfpathlineto{\pgfqpoint{2.544371in}{1.691469in}}%
\pgfpathlineto{\pgfqpoint{2.545217in}{2.066524in}}%
\pgfpathlineto{\pgfqpoint{2.547754in}{2.461613in}}%
\pgfpathlineto{\pgfqpoint{2.549445in}{2.264151in}}%
\pgfpathlineto{\pgfqpoint{2.551136in}{1.339497in}}%
\pgfpathlineto{\pgfqpoint{2.551982in}{2.630257in}}%
\pgfpathlineto{\pgfqpoint{2.552827in}{2.415599in}}%
\pgfpathlineto{\pgfqpoint{2.553673in}{1.873790in}}%
\pgfpathlineto{\pgfqpoint{2.554519in}{2.354491in}}%
\pgfpathlineto{\pgfqpoint{2.555364in}{2.276736in}}%
\pgfpathlineto{\pgfqpoint{2.556210in}{1.868058in}}%
\pgfpathlineto{\pgfqpoint{2.557056in}{1.868846in}}%
\pgfpathlineto{\pgfqpoint{2.557901in}{1.917317in}}%
\pgfpathlineto{\pgfqpoint{2.558747in}{2.759955in}}%
\pgfpathlineto{\pgfqpoint{2.559592in}{1.834548in}}%
\pgfpathlineto{\pgfqpoint{2.560438in}{2.149941in}}%
\pgfpathlineto{\pgfqpoint{2.561284in}{2.271122in}}%
\pgfpathlineto{\pgfqpoint{2.562129in}{2.171984in}}%
\pgfpathlineto{\pgfqpoint{2.564666in}{2.393824in}}%
\pgfpathlineto{\pgfqpoint{2.565512in}{2.747551in}}%
\pgfpathlineto{\pgfqpoint{2.566357in}{1.884780in}}%
\pgfpathlineto{\pgfqpoint{2.567203in}{2.346387in}}%
\pgfpathlineto{\pgfqpoint{2.568049in}{2.363098in}}%
\pgfpathlineto{\pgfqpoint{2.568894in}{2.170433in}}%
\pgfpathlineto{\pgfqpoint{2.569740in}{2.346937in}}%
\pgfpathlineto{\pgfqpoint{2.571431in}{1.736395in}}%
\pgfpathlineto{\pgfqpoint{2.572277in}{2.008161in}}%
\pgfpathlineto{\pgfqpoint{2.573122in}{2.789320in}}%
\pgfpathlineto{\pgfqpoint{2.573968in}{1.800291in}}%
\pgfpathlineto{\pgfqpoint{2.574814in}{1.913231in}}%
\pgfpathlineto{\pgfqpoint{2.575659in}{2.621487in}}%
\pgfpathlineto{\pgfqpoint{2.576505in}{2.127010in}}%
\pgfpathlineto{\pgfqpoint{2.578196in}{1.552584in}}%
\pgfpathlineto{\pgfqpoint{2.579887in}{2.873044in}}%
\pgfpathlineto{\pgfqpoint{2.580733in}{2.397505in}}%
\pgfpathlineto{\pgfqpoint{2.583270in}{2.674505in}}%
\pgfpathlineto{\pgfqpoint{2.584961in}{1.912275in}}%
\pgfpathlineto{\pgfqpoint{2.585807in}{2.322350in}}%
\pgfpathlineto{\pgfqpoint{2.586652in}{2.059161in}}%
\pgfpathlineto{\pgfqpoint{2.587498in}{1.903876in}}%
\pgfpathlineto{\pgfqpoint{2.588344in}{2.603103in}}%
\pgfpathlineto{\pgfqpoint{2.589189in}{2.062277in}}%
\pgfpathlineto{\pgfqpoint{2.590035in}{2.510142in}}%
\pgfpathlineto{\pgfqpoint{2.591726in}{1.918623in}}%
\pgfpathlineto{\pgfqpoint{2.592572in}{2.411020in}}%
\pgfpathlineto{\pgfqpoint{2.595109in}{1.793071in}}%
\pgfpathlineto{\pgfqpoint{2.597645in}{2.072258in}}%
\pgfpathlineto{\pgfqpoint{2.598491in}{2.029384in}}%
\pgfpathlineto{\pgfqpoint{2.599337in}{2.980927in}}%
\pgfpathlineto{\pgfqpoint{2.600182in}{2.247960in}}%
\pgfpathlineto{\pgfqpoint{2.602719in}{1.811406in}}%
\pgfpathlineto{\pgfqpoint{2.603565in}{2.559372in}}%
\pgfpathlineto{\pgfqpoint{2.604410in}{2.306489in}}%
\pgfpathlineto{\pgfqpoint{2.605256in}{1.607696in}}%
\pgfpathlineto{\pgfqpoint{2.606947in}{2.318882in}}%
\pgfpathlineto{\pgfqpoint{2.607793in}{1.798130in}}%
\pgfpathlineto{\pgfqpoint{2.610330in}{2.879806in}}%
\pgfpathlineto{\pgfqpoint{2.611175in}{2.035959in}}%
\pgfpathlineto{\pgfqpoint{2.612021in}{2.573857in}}%
\pgfpathlineto{\pgfqpoint{2.613712in}{1.702964in}}%
\pgfpathlineto{\pgfqpoint{2.614558in}{1.942502in}}%
\pgfpathlineto{\pgfqpoint{2.615404in}{2.318697in}}%
\pgfpathlineto{\pgfqpoint{2.617940in}{1.047260in}}%
\pgfpathlineto{\pgfqpoint{2.619632in}{2.781306in}}%
\pgfpathlineto{\pgfqpoint{2.622169in}{1.832842in}}%
\pgfpathlineto{\pgfqpoint{2.623014in}{1.916448in}}%
\pgfpathlineto{\pgfqpoint{2.623860in}{1.804775in}}%
\pgfpathlineto{\pgfqpoint{2.627242in}{2.929443in}}%
\pgfpathlineto{\pgfqpoint{2.628088in}{1.930322in}}%
\pgfpathlineto{\pgfqpoint{2.628934in}{2.753862in}}%
\pgfpathlineto{\pgfqpoint{2.630625in}{1.767298in}}%
\pgfpathlineto{\pgfqpoint{2.631470in}{2.748083in}}%
\pgfpathlineto{\pgfqpoint{2.632316in}{2.469408in}}%
\pgfpathlineto{\pgfqpoint{2.633162in}{2.555094in}}%
\pgfpathlineto{\pgfqpoint{2.634853in}{1.942729in}}%
\pgfpathlineto{\pgfqpoint{2.635699in}{2.074802in}}%
\pgfpathlineto{\pgfqpoint{2.636544in}{1.957482in}}%
\pgfpathlineto{\pgfqpoint{2.639081in}{2.922969in}}%
\pgfpathlineto{\pgfqpoint{2.639927in}{1.815952in}}%
\pgfpathlineto{\pgfqpoint{2.640772in}{2.628823in}}%
\pgfpathlineto{\pgfqpoint{2.641618in}{3.017236in}}%
\pgfpathlineto{\pgfqpoint{2.642463in}{1.749640in}}%
\pgfpathlineto{\pgfqpoint{2.643309in}{2.015928in}}%
\pgfpathlineto{\pgfqpoint{2.645000in}{2.042775in}}%
\pgfpathlineto{\pgfqpoint{2.645846in}{2.685992in}}%
\pgfpathlineto{\pgfqpoint{2.646692in}{1.749547in}}%
\pgfpathlineto{\pgfqpoint{2.647537in}{1.939582in}}%
\pgfpathlineto{\pgfqpoint{2.649228in}{2.215757in}}%
\pgfpathlineto{\pgfqpoint{2.650074in}{1.665621in}}%
\pgfpathlineto{\pgfqpoint{2.651765in}{2.388006in}}%
\pgfpathlineto{\pgfqpoint{2.654302in}{2.200469in}}%
\pgfpathlineto{\pgfqpoint{2.655148in}{2.257807in}}%
\pgfpathlineto{\pgfqpoint{2.655993in}{2.483055in}}%
\pgfpathlineto{\pgfqpoint{2.656839in}{2.262402in}}%
\pgfpathlineto{\pgfqpoint{2.657685in}{2.726887in}}%
\pgfpathlineto{\pgfqpoint{2.659376in}{2.008051in}}%
\pgfpathlineto{\pgfqpoint{2.660222in}{2.428397in}}%
\pgfpathlineto{\pgfqpoint{2.661067in}{2.406446in}}%
\pgfpathlineto{\pgfqpoint{2.662758in}{1.699377in}}%
\pgfpathlineto{\pgfqpoint{2.664450in}{1.945068in}}%
\pgfpathlineto{\pgfqpoint{2.666141in}{1.704816in}}%
\pgfpathlineto{\pgfqpoint{2.666987in}{2.216145in}}%
\pgfpathlineto{\pgfqpoint{2.667832in}{1.788619in}}%
\pgfpathlineto{\pgfqpoint{2.668678in}{2.297496in}}%
\pgfpathlineto{\pgfqpoint{2.669523in}{1.981435in}}%
\pgfpathlineto{\pgfqpoint{2.670369in}{2.088132in}}%
\pgfpathlineto{\pgfqpoint{2.671215in}{1.621072in}}%
\pgfpathlineto{\pgfqpoint{2.673752in}{2.513200in}}%
\pgfpathlineto{\pgfqpoint{2.675443in}{2.336886in}}%
\pgfpathlineto{\pgfqpoint{2.677134in}{2.535664in}}%
\pgfpathlineto{\pgfqpoint{2.678825in}{1.731280in}}%
\pgfpathlineto{\pgfqpoint{2.679671in}{2.521706in}}%
\pgfpathlineto{\pgfqpoint{2.680517in}{2.216717in}}%
\pgfpathlineto{\pgfqpoint{2.682208in}{2.090946in}}%
\pgfpathlineto{\pgfqpoint{2.683053in}{2.180073in}}%
\pgfpathlineto{\pgfqpoint{2.683899in}{1.945414in}}%
\pgfpathlineto{\pgfqpoint{2.684745in}{1.952993in}}%
\pgfpathlineto{\pgfqpoint{2.686436in}{2.712359in}}%
\pgfpathlineto{\pgfqpoint{2.688127in}{1.950079in}}%
\pgfpathlineto{\pgfqpoint{2.688973in}{2.374421in}}%
\pgfpathlineto{\pgfqpoint{2.689818in}{2.169516in}}%
\pgfpathlineto{\pgfqpoint{2.690664in}{1.788520in}}%
\pgfpathlineto{\pgfqpoint{2.691510in}{2.449363in}}%
\pgfpathlineto{\pgfqpoint{2.692355in}{2.256019in}}%
\pgfpathlineto{\pgfqpoint{2.693201in}{2.532466in}}%
\pgfpathlineto{\pgfqpoint{2.694047in}{1.995378in}}%
\pgfpathlineto{\pgfqpoint{2.694892in}{3.024951in}}%
\pgfpathlineto{\pgfqpoint{2.695738in}{2.342122in}}%
\pgfpathlineto{\pgfqpoint{2.696583in}{1.640754in}}%
\pgfpathlineto{\pgfqpoint{2.697429in}{3.229707in}}%
\pgfpathlineto{\pgfqpoint{2.698275in}{1.401451in}}%
\pgfpathlineto{\pgfqpoint{2.699120in}{2.631596in}}%
\pgfpathlineto{\pgfqpoint{2.699966in}{2.506736in}}%
\pgfpathlineto{\pgfqpoint{2.700812in}{2.615909in}}%
\pgfpathlineto{\pgfqpoint{2.703348in}{2.144857in}}%
\pgfpathlineto{\pgfqpoint{2.704194in}{2.577492in}}%
\pgfpathlineto{\pgfqpoint{2.705040in}{2.331631in}}%
\pgfpathlineto{\pgfqpoint{2.705885in}{2.640412in}}%
\pgfpathlineto{\pgfqpoint{2.707577in}{1.900625in}}%
\pgfpathlineto{\pgfqpoint{2.708422in}{2.680478in}}%
\pgfpathlineto{\pgfqpoint{2.709268in}{1.588966in}}%
\pgfpathlineto{\pgfqpoint{2.710113in}{2.252853in}}%
\pgfpathlineto{\pgfqpoint{2.710959in}{1.953817in}}%
\pgfpathlineto{\pgfqpoint{2.714342in}{2.903250in}}%
\pgfpathlineto{\pgfqpoint{2.715187in}{1.981987in}}%
\pgfpathlineto{\pgfqpoint{2.716033in}{2.291277in}}%
\pgfpathlineto{\pgfqpoint{2.716878in}{2.251696in}}%
\pgfpathlineto{\pgfqpoint{2.717724in}{1.665178in}}%
\pgfpathlineto{\pgfqpoint{2.718570in}{2.865184in}}%
\pgfpathlineto{\pgfqpoint{2.719415in}{2.175228in}}%
\pgfpathlineto{\pgfqpoint{2.721106in}{1.537804in}}%
\pgfpathlineto{\pgfqpoint{2.721952in}{2.501034in}}%
\pgfpathlineto{\pgfqpoint{2.722798in}{2.306744in}}%
\pgfpathlineto{\pgfqpoint{2.724489in}{2.043458in}}%
\pgfpathlineto{\pgfqpoint{2.727871in}{2.601587in}}%
\pgfpathlineto{\pgfqpoint{2.728717in}{2.710607in}}%
\pgfpathlineto{\pgfqpoint{2.730408in}{1.779607in}}%
\pgfpathlineto{\pgfqpoint{2.731254in}{2.163718in}}%
\pgfpathlineto{\pgfqpoint{2.732100in}{2.190389in}}%
\pgfpathlineto{\pgfqpoint{2.733791in}{1.990772in}}%
\pgfpathlineto{\pgfqpoint{2.734636in}{2.236494in}}%
\pgfpathlineto{\pgfqpoint{2.735482in}{2.176324in}}%
\pgfpathlineto{\pgfqpoint{2.736328in}{2.211016in}}%
\pgfpathlineto{\pgfqpoint{2.737173in}{2.896105in}}%
\pgfpathlineto{\pgfqpoint{2.738865in}{2.131817in}}%
\pgfpathlineto{\pgfqpoint{2.739710in}{2.352204in}}%
\pgfpathlineto{\pgfqpoint{2.741401in}{2.133364in}}%
\pgfpathlineto{\pgfqpoint{2.742247in}{2.132299in}}%
\pgfpathlineto{\pgfqpoint{2.743093in}{2.428525in}}%
\pgfpathlineto{\pgfqpoint{2.745630in}{1.725853in}}%
\pgfpathlineto{\pgfqpoint{2.748166in}{2.371247in}}%
\pgfpathlineto{\pgfqpoint{2.749012in}{2.657433in}}%
\pgfpathlineto{\pgfqpoint{2.750703in}{2.012341in}}%
\pgfpathlineto{\pgfqpoint{2.751549in}{2.403681in}}%
\pgfpathlineto{\pgfqpoint{2.752395in}{2.214140in}}%
\pgfpathlineto{\pgfqpoint{2.753240in}{2.181190in}}%
\pgfpathlineto{\pgfqpoint{2.754086in}{1.930454in}}%
\pgfpathlineto{\pgfqpoint{2.754931in}{2.513003in}}%
\pgfpathlineto{\pgfqpoint{2.755777in}{2.367480in}}%
\pgfpathlineto{\pgfqpoint{2.758314in}{2.237067in}}%
\pgfpathlineto{\pgfqpoint{2.759160in}{2.359065in}}%
\pgfpathlineto{\pgfqpoint{2.760005in}{2.058373in}}%
\pgfpathlineto{\pgfqpoint{2.760851in}{2.298977in}}%
\pgfpathlineto{\pgfqpoint{2.761696in}{2.252246in}}%
\pgfpathlineto{\pgfqpoint{2.762542in}{1.932776in}}%
\pgfpathlineto{\pgfqpoint{2.763388in}{1.999866in}}%
\pgfpathlineto{\pgfqpoint{2.764233in}{2.561358in}}%
\pgfpathlineto{\pgfqpoint{2.765079in}{1.532835in}}%
\pgfpathlineto{\pgfqpoint{2.765925in}{2.290314in}}%
\pgfpathlineto{\pgfqpoint{2.766770in}{2.430844in}}%
\pgfpathlineto{\pgfqpoint{2.768461in}{2.168813in}}%
\pgfpathlineto{\pgfqpoint{2.769307in}{2.429406in}}%
\pgfpathlineto{\pgfqpoint{2.770153in}{2.240176in}}%
\pgfpathlineto{\pgfqpoint{2.770998in}{2.186731in}}%
\pgfpathlineto{\pgfqpoint{2.771844in}{1.889731in}}%
\pgfpathlineto{\pgfqpoint{2.774381in}{2.651099in}}%
\pgfpathlineto{\pgfqpoint{2.776072in}{1.887942in}}%
\pgfpathlineto{\pgfqpoint{2.776918in}{1.887870in}}%
\pgfpathlineto{\pgfqpoint{2.778609in}{2.391507in}}%
\pgfpathlineto{\pgfqpoint{2.780300in}{1.828790in}}%
\pgfpathlineto{\pgfqpoint{2.781146in}{1.924099in}}%
\pgfpathlineto{\pgfqpoint{2.782837in}{2.203466in}}%
\pgfpathlineto{\pgfqpoint{2.783683in}{1.832646in}}%
\pgfpathlineto{\pgfqpoint{2.786220in}{2.395443in}}%
\pgfpathlineto{\pgfqpoint{2.787065in}{2.442320in}}%
\pgfpathlineto{\pgfqpoint{2.789602in}{1.908180in}}%
\pgfpathlineto{\pgfqpoint{2.790448in}{2.434293in}}%
\pgfpathlineto{\pgfqpoint{2.791293in}{2.251796in}}%
\pgfpathlineto{\pgfqpoint{2.792139in}{2.248085in}}%
\pgfpathlineto{\pgfqpoint{2.792985in}{2.030530in}}%
\pgfpathlineto{\pgfqpoint{2.794676in}{2.729123in}}%
\pgfpathlineto{\pgfqpoint{2.795521in}{1.879614in}}%
\pgfpathlineto{\pgfqpoint{2.796367in}{1.940370in}}%
\pgfpathlineto{\pgfqpoint{2.797213in}{2.001438in}}%
\pgfpathlineto{\pgfqpoint{2.798058in}{2.183526in}}%
\pgfpathlineto{\pgfqpoint{2.798904in}{1.725218in}}%
\pgfpathlineto{\pgfqpoint{2.799749in}{2.615306in}}%
\pgfpathlineto{\pgfqpoint{2.800595in}{2.437204in}}%
\pgfpathlineto{\pgfqpoint{2.801441in}{2.575156in}}%
\pgfpathlineto{\pgfqpoint{2.803132in}{1.553925in}}%
\pgfpathlineto{\pgfqpoint{2.804823in}{2.203728in}}%
\pgfpathlineto{\pgfqpoint{2.805669in}{1.767852in}}%
\pgfpathlineto{\pgfqpoint{2.806514in}{2.046380in}}%
\pgfpathlineto{\pgfqpoint{2.807360in}{1.983001in}}%
\pgfpathlineto{\pgfqpoint{2.808206in}{2.484206in}}%
\pgfpathlineto{\pgfqpoint{2.809051in}{2.222426in}}%
\pgfpathlineto{\pgfqpoint{2.809897in}{1.641846in}}%
\pgfpathlineto{\pgfqpoint{2.812434in}{2.436378in}}%
\pgfpathlineto{\pgfqpoint{2.813279in}{2.874935in}}%
\pgfpathlineto{\pgfqpoint{2.814125in}{2.088144in}}%
\pgfpathlineto{\pgfqpoint{2.814971in}{2.205121in}}%
\pgfpathlineto{\pgfqpoint{2.815816in}{2.366400in}}%
\pgfpathlineto{\pgfqpoint{2.817508in}{1.624136in}}%
\pgfpathlineto{\pgfqpoint{2.818353in}{2.742605in}}%
\pgfpathlineto{\pgfqpoint{2.819199in}{2.346962in}}%
\pgfpathlineto{\pgfqpoint{2.820044in}{2.095630in}}%
\pgfpathlineto{\pgfqpoint{2.820890in}{2.426573in}}%
\pgfpathlineto{\pgfqpoint{2.821736in}{2.416537in}}%
\pgfpathlineto{\pgfqpoint{2.822581in}{1.742897in}}%
\pgfpathlineto{\pgfqpoint{2.823427in}{1.917971in}}%
\pgfpathlineto{\pgfqpoint{2.825964in}{2.266306in}}%
\pgfpathlineto{\pgfqpoint{2.826809in}{2.217124in}}%
\pgfpathlineto{\pgfqpoint{2.827655in}{2.857759in}}%
\pgfpathlineto{\pgfqpoint{2.828501in}{1.919750in}}%
\pgfpathlineto{\pgfqpoint{2.829346in}{2.357215in}}%
\pgfpathlineto{\pgfqpoint{2.830192in}{2.400937in}}%
\pgfpathlineto{\pgfqpoint{2.831038in}{2.397536in}}%
\pgfpathlineto{\pgfqpoint{2.831883in}{1.280001in}}%
\pgfpathlineto{\pgfqpoint{2.832729in}{2.116656in}}%
\pgfpathlineto{\pgfqpoint{2.833574in}{1.880778in}}%
\pgfpathlineto{\pgfqpoint{2.834420in}{2.563770in}}%
\pgfpathlineto{\pgfqpoint{2.835266in}{2.084913in}}%
\pgfpathlineto{\pgfqpoint{2.836111in}{2.532579in}}%
\pgfpathlineto{\pgfqpoint{2.836957in}{2.203532in}}%
\pgfpathlineto{\pgfqpoint{2.837803in}{2.296726in}}%
\pgfpathlineto{\pgfqpoint{2.839494in}{2.798827in}}%
\pgfpathlineto{\pgfqpoint{2.840339in}{1.948909in}}%
\pgfpathlineto{\pgfqpoint{2.841185in}{2.490449in}}%
\pgfpathlineto{\pgfqpoint{2.842031in}{2.468061in}}%
\pgfpathlineto{\pgfqpoint{2.842876in}{1.940505in}}%
\pgfpathlineto{\pgfqpoint{2.843722in}{2.207678in}}%
\pgfpathlineto{\pgfqpoint{2.845413in}{2.372112in}}%
\pgfpathlineto{\pgfqpoint{2.847104in}{1.744060in}}%
\pgfpathlineto{\pgfqpoint{2.847950in}{2.205494in}}%
\pgfpathlineto{\pgfqpoint{2.848796in}{1.855879in}}%
\pgfpathlineto{\pgfqpoint{2.849641in}{1.913998in}}%
\pgfpathlineto{\pgfqpoint{2.850487in}{2.389142in}}%
\pgfpathlineto{\pgfqpoint{2.851333in}{1.994944in}}%
\pgfpathlineto{\pgfqpoint{2.852178in}{2.366043in}}%
\pgfpathlineto{\pgfqpoint{2.853024in}{2.217429in}}%
\pgfpathlineto{\pgfqpoint{2.853869in}{1.858358in}}%
\pgfpathlineto{\pgfqpoint{2.857252in}{2.981142in}}%
\pgfpathlineto{\pgfqpoint{2.858098in}{1.888295in}}%
\pgfpathlineto{\pgfqpoint{2.858943in}{2.524837in}}%
\pgfpathlineto{\pgfqpoint{2.859789in}{1.712219in}}%
\pgfpathlineto{\pgfqpoint{2.860634in}{1.891669in}}%
\pgfpathlineto{\pgfqpoint{2.861480in}{1.823936in}}%
\pgfpathlineto{\pgfqpoint{2.862326in}{2.248465in}}%
\pgfpathlineto{\pgfqpoint{2.863171in}{2.169740in}}%
\pgfpathlineto{\pgfqpoint{2.864017in}{2.022802in}}%
\pgfpathlineto{\pgfqpoint{2.865708in}{2.930743in}}%
\pgfpathlineto{\pgfqpoint{2.867399in}{1.736733in}}%
\pgfpathlineto{\pgfqpoint{2.868245in}{1.816733in}}%
\pgfpathlineto{\pgfqpoint{2.869091in}{2.337459in}}%
\pgfpathlineto{\pgfqpoint{2.869936in}{1.834596in}}%
\pgfpathlineto{\pgfqpoint{2.870782in}{2.524245in}}%
\pgfpathlineto{\pgfqpoint{2.871628in}{2.346959in}}%
\pgfpathlineto{\pgfqpoint{2.872473in}{1.959553in}}%
\pgfpathlineto{\pgfqpoint{2.873319in}{2.226033in}}%
\pgfpathlineto{\pgfqpoint{2.874164in}{2.293364in}}%
\pgfpathlineto{\pgfqpoint{2.875010in}{1.769536in}}%
\pgfpathlineto{\pgfqpoint{2.875856in}{2.584184in}}%
\pgfpathlineto{\pgfqpoint{2.876701in}{2.534133in}}%
\pgfpathlineto{\pgfqpoint{2.877547in}{2.276405in}}%
\pgfpathlineto{\pgfqpoint{2.878392in}{2.704496in}}%
\pgfpathlineto{\pgfqpoint{2.880084in}{1.797029in}}%
\pgfpathlineto{\pgfqpoint{2.880929in}{2.617876in}}%
\pgfpathlineto{\pgfqpoint{2.881775in}{2.202321in}}%
\pgfpathlineto{\pgfqpoint{2.882621in}{1.548046in}}%
\pgfpathlineto{\pgfqpoint{2.883466in}{2.321943in}}%
\pgfpathlineto{\pgfqpoint{2.884312in}{1.971588in}}%
\pgfpathlineto{\pgfqpoint{2.886003in}{1.899324in}}%
\pgfpathlineto{\pgfqpoint{2.887694in}{2.231541in}}%
\pgfpathlineto{\pgfqpoint{2.889386in}{1.861190in}}%
\pgfpathlineto{\pgfqpoint{2.890231in}{2.215546in}}%
\pgfpathlineto{\pgfqpoint{2.891077in}{1.794650in}}%
\pgfpathlineto{\pgfqpoint{2.891922in}{2.593806in}}%
\pgfpathlineto{\pgfqpoint{2.892768in}{2.337806in}}%
\pgfpathlineto{\pgfqpoint{2.893614in}{2.297210in}}%
\pgfpathlineto{\pgfqpoint{2.895305in}{1.876163in}}%
\pgfpathlineto{\pgfqpoint{2.896151in}{2.067389in}}%
\pgfpathlineto{\pgfqpoint{2.896996in}{2.042922in}}%
\pgfpathlineto{\pgfqpoint{2.897842in}{2.024129in}}%
\pgfpathlineto{\pgfqpoint{2.899533in}{1.593790in}}%
\pgfpathlineto{\pgfqpoint{2.900379in}{2.648177in}}%
\pgfpathlineto{\pgfqpoint{2.901224in}{2.139144in}}%
\pgfpathlineto{\pgfqpoint{2.902070in}{2.144519in}}%
\pgfpathlineto{\pgfqpoint{2.902916in}{2.572729in}}%
\pgfpathlineto{\pgfqpoint{2.903761in}{2.248716in}}%
\pgfpathlineto{\pgfqpoint{2.904607in}{2.400233in}}%
\pgfpathlineto{\pgfqpoint{2.905452in}{2.028978in}}%
\pgfpathlineto{\pgfqpoint{2.906298in}{2.098848in}}%
\pgfpathlineto{\pgfqpoint{2.907144in}{2.902821in}}%
\pgfpathlineto{\pgfqpoint{2.907989in}{2.004254in}}%
\pgfpathlineto{\pgfqpoint{2.908835in}{2.134402in}}%
\pgfpathlineto{\pgfqpoint{2.909681in}{3.106334in}}%
\pgfpathlineto{\pgfqpoint{2.911372in}{2.284015in}}%
\pgfpathlineto{\pgfqpoint{2.912217in}{2.654156in}}%
\pgfpathlineto{\pgfqpoint{2.913909in}{1.703156in}}%
\pgfpathlineto{\pgfqpoint{2.914754in}{1.808301in}}%
\pgfpathlineto{\pgfqpoint{2.916446in}{2.520629in}}%
\pgfpathlineto{\pgfqpoint{2.917291in}{2.419751in}}%
\pgfpathlineto{\pgfqpoint{2.918137in}{2.357842in}}%
\pgfpathlineto{\pgfqpoint{2.918982in}{2.024745in}}%
\pgfpathlineto{\pgfqpoint{2.919828in}{2.393821in}}%
\pgfpathlineto{\pgfqpoint{2.921519in}{1.401690in}}%
\pgfpathlineto{\pgfqpoint{2.922365in}{2.263168in}}%
\pgfpathlineto{\pgfqpoint{2.923211in}{2.239340in}}%
\pgfpathlineto{\pgfqpoint{2.924056in}{1.671003in}}%
\pgfpathlineto{\pgfqpoint{2.924902in}{2.135170in}}%
\pgfpathlineto{\pgfqpoint{2.925747in}{2.139635in}}%
\pgfpathlineto{\pgfqpoint{2.926593in}{2.761946in}}%
\pgfpathlineto{\pgfqpoint{2.928284in}{2.014170in}}%
\pgfpathlineto{\pgfqpoint{2.929130in}{2.321731in}}%
\pgfpathlineto{\pgfqpoint{2.930821in}{1.794286in}}%
\pgfpathlineto{\pgfqpoint{2.931667in}{2.211109in}}%
\pgfpathlineto{\pgfqpoint{2.932512in}{2.149705in}}%
\pgfpathlineto{\pgfqpoint{2.933358in}{2.159218in}}%
\pgfpathlineto{\pgfqpoint{2.934204in}{2.730821in}}%
\pgfpathlineto{\pgfqpoint{2.935049in}{2.093188in}}%
\pgfpathlineto{\pgfqpoint{2.935895in}{2.474203in}}%
\pgfpathlineto{\pgfqpoint{2.936741in}{2.258197in}}%
\pgfpathlineto{\pgfqpoint{2.937586in}{2.330427in}}%
\pgfpathlineto{\pgfqpoint{2.938432in}{2.280043in}}%
\pgfpathlineto{\pgfqpoint{2.939277in}{2.940454in}}%
\pgfpathlineto{\pgfqpoint{2.940123in}{2.436699in}}%
\pgfpathlineto{\pgfqpoint{2.940969in}{1.803259in}}%
\pgfpathlineto{\pgfqpoint{2.941814in}{2.557922in}}%
\pgfpathlineto{\pgfqpoint{2.942660in}{2.437081in}}%
\pgfpathlineto{\pgfqpoint{2.943506in}{1.885987in}}%
\pgfpathlineto{\pgfqpoint{2.944351in}{3.032779in}}%
\pgfpathlineto{\pgfqpoint{2.946042in}{1.770938in}}%
\pgfpathlineto{\pgfqpoint{2.946888in}{2.915426in}}%
\pgfpathlineto{\pgfqpoint{2.947734in}{2.491042in}}%
\pgfpathlineto{\pgfqpoint{2.948579in}{1.764109in}}%
\pgfpathlineto{\pgfqpoint{2.949425in}{1.803068in}}%
\pgfpathlineto{\pgfqpoint{2.951116in}{2.746299in}}%
\pgfpathlineto{\pgfqpoint{2.952807in}{1.446885in}}%
\pgfpathlineto{\pgfqpoint{2.955344in}{2.537227in}}%
\pgfpathlineto{\pgfqpoint{2.957881in}{2.089671in}}%
\pgfpathlineto{\pgfqpoint{2.958727in}{2.549903in}}%
\pgfpathlineto{\pgfqpoint{2.961264in}{1.314569in}}%
\pgfpathlineto{\pgfqpoint{2.963800in}{2.187718in}}%
\pgfpathlineto{\pgfqpoint{2.964646in}{2.091659in}}%
\pgfpathlineto{\pgfqpoint{2.965492in}{3.296457in}}%
\pgfpathlineto{\pgfqpoint{2.966337in}{2.158839in}}%
\pgfpathlineto{\pgfqpoint{2.967183in}{2.789820in}}%
\pgfpathlineto{\pgfqpoint{2.968029in}{2.050125in}}%
\pgfpathlineto{\pgfqpoint{2.968874in}{2.107493in}}%
\pgfpathlineto{\pgfqpoint{2.970565in}{2.533406in}}%
\pgfpathlineto{\pgfqpoint{2.971411in}{2.462986in}}%
\pgfpathlineto{\pgfqpoint{2.973102in}{1.855827in}}%
\pgfpathlineto{\pgfqpoint{2.973948in}{2.811345in}}%
\pgfpathlineto{\pgfqpoint{2.974794in}{2.282718in}}%
\pgfpathlineto{\pgfqpoint{2.975639in}{2.352187in}}%
\pgfpathlineto{\pgfqpoint{2.976485in}{1.644677in}}%
\pgfpathlineto{\pgfqpoint{2.977330in}{2.222322in}}%
\pgfpathlineto{\pgfqpoint{2.978176in}{1.723810in}}%
\pgfpathlineto{\pgfqpoint{2.979022in}{2.256103in}}%
\pgfpathlineto{\pgfqpoint{2.979867in}{1.815151in}}%
\pgfpathlineto{\pgfqpoint{2.982404in}{2.804242in}}%
\pgfpathlineto{\pgfqpoint{2.983250in}{1.686455in}}%
\pgfpathlineto{\pgfqpoint{2.984095in}{2.344779in}}%
\pgfpathlineto{\pgfqpoint{2.984941in}{2.289333in}}%
\pgfpathlineto{\pgfqpoint{2.986632in}{2.914496in}}%
\pgfpathlineto{\pgfqpoint{2.987478in}{2.556832in}}%
\pgfpathlineto{\pgfqpoint{2.989169in}{2.400371in}}%
\pgfpathlineto{\pgfqpoint{2.990015in}{2.472943in}}%
\pgfpathlineto{\pgfqpoint{2.990860in}{1.832609in}}%
\pgfpathlineto{\pgfqpoint{2.991706in}{2.225965in}}%
\pgfpathlineto{\pgfqpoint{2.993397in}{2.450775in}}%
\pgfpathlineto{\pgfqpoint{2.994243in}{2.224596in}}%
\pgfpathlineto{\pgfqpoint{2.995089in}{2.398594in}}%
\pgfpathlineto{\pgfqpoint{2.995934in}{2.227750in}}%
\pgfpathlineto{\pgfqpoint{2.996780in}{2.751703in}}%
\pgfpathlineto{\pgfqpoint{2.997625in}{2.042215in}}%
\pgfpathlineto{\pgfqpoint{2.998471in}{2.241967in}}%
\pgfpathlineto{\pgfqpoint{3.000162in}{2.747315in}}%
\pgfpathlineto{\pgfqpoint{3.001854in}{2.002342in}}%
\pgfpathlineto{\pgfqpoint{3.002699in}{2.787959in}}%
\pgfpathlineto{\pgfqpoint{3.003545in}{2.404106in}}%
\pgfpathlineto{\pgfqpoint{3.005236in}{2.116330in}}%
\pgfpathlineto{\pgfqpoint{3.006082in}{2.152992in}}%
\pgfpathlineto{\pgfqpoint{3.006927in}{2.271459in}}%
\pgfpathlineto{\pgfqpoint{3.007773in}{1.730395in}}%
\pgfpathlineto{\pgfqpoint{3.008619in}{2.587893in}}%
\pgfpathlineto{\pgfqpoint{3.009464in}{1.774432in}}%
\pgfpathlineto{\pgfqpoint{3.010310in}{2.003267in}}%
\pgfpathlineto{\pgfqpoint{3.011155in}{2.319648in}}%
\pgfpathlineto{\pgfqpoint{3.012001in}{2.282856in}}%
\pgfpathlineto{\pgfqpoint{3.012847in}{2.338951in}}%
\pgfpathlineto{\pgfqpoint{3.013692in}{2.823811in}}%
\pgfpathlineto{\pgfqpoint{3.015384in}{1.688230in}}%
\pgfpathlineto{\pgfqpoint{3.016229in}{2.484459in}}%
\pgfpathlineto{\pgfqpoint{3.017075in}{2.258844in}}%
\pgfpathlineto{\pgfqpoint{3.018766in}{2.447857in}}%
\pgfpathlineto{\pgfqpoint{3.020457in}{1.876844in}}%
\pgfpathlineto{\pgfqpoint{3.022149in}{2.729344in}}%
\pgfpathlineto{\pgfqpoint{3.023840in}{1.871486in}}%
\pgfpathlineto{\pgfqpoint{3.025531in}{2.648518in}}%
\pgfpathlineto{\pgfqpoint{3.027222in}{2.000245in}}%
\pgfpathlineto{\pgfqpoint{3.028068in}{2.003834in}}%
\pgfpathlineto{\pgfqpoint{3.028914in}{2.495507in}}%
\pgfpathlineto{\pgfqpoint{3.029759in}{1.655963in}}%
\pgfpathlineto{\pgfqpoint{3.030605in}{2.265704in}}%
\pgfpathlineto{\pgfqpoint{3.031450in}{2.629747in}}%
\pgfpathlineto{\pgfqpoint{3.032296in}{1.763442in}}%
\pgfpathlineto{\pgfqpoint{3.033142in}{2.641384in}}%
\pgfpathlineto{\pgfqpoint{3.033987in}{2.612238in}}%
\pgfpathlineto{\pgfqpoint{3.034833in}{2.063874in}}%
\pgfpathlineto{\pgfqpoint{3.037370in}{2.718640in}}%
\pgfpathlineto{\pgfqpoint{3.039061in}{1.741025in}}%
\pgfpathlineto{\pgfqpoint{3.039907in}{2.345841in}}%
\pgfpathlineto{\pgfqpoint{3.040752in}{2.009488in}}%
\pgfpathlineto{\pgfqpoint{3.041598in}{1.832915in}}%
\pgfpathlineto{\pgfqpoint{3.045826in}{2.521469in}}%
\pgfpathlineto{\pgfqpoint{3.046672in}{1.982716in}}%
\pgfpathlineto{\pgfqpoint{3.047517in}{2.083923in}}%
\pgfpathlineto{\pgfqpoint{3.050054in}{2.510829in}}%
\pgfpathlineto{\pgfqpoint{3.050900in}{2.465451in}}%
\pgfpathlineto{\pgfqpoint{3.051745in}{1.934563in}}%
\pgfpathlineto{\pgfqpoint{3.052591in}{2.308974in}}%
\pgfpathlineto{\pgfqpoint{3.054282in}{1.955329in}}%
\pgfpathlineto{\pgfqpoint{3.055973in}{2.239156in}}%
\pgfpathlineto{\pgfqpoint{3.056819in}{2.154789in}}%
\pgfpathlineto{\pgfqpoint{3.057665in}{2.419418in}}%
\pgfpathlineto{\pgfqpoint{3.058510in}{2.227084in}}%
\pgfpathlineto{\pgfqpoint{3.060202in}{1.740149in}}%
\pgfpathlineto{\pgfqpoint{3.062738in}{2.528393in}}%
\pgfpathlineto{\pgfqpoint{3.063584in}{1.767335in}}%
\pgfpathlineto{\pgfqpoint{3.064430in}{2.175025in}}%
\pgfpathlineto{\pgfqpoint{3.065275in}{2.245760in}}%
\pgfpathlineto{\pgfqpoint{3.066121in}{1.726416in}}%
\pgfpathlineto{\pgfqpoint{3.066967in}{2.543395in}}%
\pgfpathlineto{\pgfqpoint{3.067812in}{2.375593in}}%
\pgfpathlineto{\pgfqpoint{3.068658in}{1.827347in}}%
\pgfpathlineto{\pgfqpoint{3.069503in}{2.807639in}}%
\pgfpathlineto{\pgfqpoint{3.070349in}{1.775440in}}%
\pgfpathlineto{\pgfqpoint{3.071195in}{2.001445in}}%
\pgfpathlineto{\pgfqpoint{3.072040in}{2.040084in}}%
\pgfpathlineto{\pgfqpoint{3.072886in}{2.812021in}}%
\pgfpathlineto{\pgfqpoint{3.073732in}{2.432987in}}%
\pgfpathlineto{\pgfqpoint{3.076268in}{1.936832in}}%
\pgfpathlineto{\pgfqpoint{3.077960in}{2.244005in}}%
\pgfpathlineto{\pgfqpoint{3.078805in}{1.820200in}}%
\pgfpathlineto{\pgfqpoint{3.079651in}{2.307276in}}%
\pgfpathlineto{\pgfqpoint{3.080497in}{2.088402in}}%
\pgfpathlineto{\pgfqpoint{3.081342in}{1.785218in}}%
\pgfpathlineto{\pgfqpoint{3.083879in}{2.409324in}}%
\pgfpathlineto{\pgfqpoint{3.084725in}{2.213689in}}%
\pgfpathlineto{\pgfqpoint{3.085570in}{1.563817in}}%
\pgfpathlineto{\pgfqpoint{3.086416in}{1.928540in}}%
\pgfpathlineto{\pgfqpoint{3.088107in}{2.524787in}}%
\pgfpathlineto{\pgfqpoint{3.088953in}{1.675717in}}%
\pgfpathlineto{\pgfqpoint{3.090644in}{2.628222in}}%
\pgfpathlineto{\pgfqpoint{3.091490in}{1.612474in}}%
\pgfpathlineto{\pgfqpoint{3.092335in}{2.335183in}}%
\pgfpathlineto{\pgfqpoint{3.093181in}{1.606592in}}%
\pgfpathlineto{\pgfqpoint{3.094027in}{1.880768in}}%
\pgfpathlineto{\pgfqpoint{3.094872in}{1.965189in}}%
\pgfpathlineto{\pgfqpoint{3.095718in}{2.599418in}}%
\pgfpathlineto{\pgfqpoint{3.096563in}{1.942822in}}%
\pgfpathlineto{\pgfqpoint{3.097409in}{2.803030in}}%
\pgfpathlineto{\pgfqpoint{3.098255in}{2.036259in}}%
\pgfpathlineto{\pgfqpoint{3.099100in}{2.304414in}}%
\pgfpathlineto{\pgfqpoint{3.100792in}{1.895151in}}%
\pgfpathlineto{\pgfqpoint{3.101637in}{2.010996in}}%
\pgfpathlineto{\pgfqpoint{3.102483in}{1.972362in}}%
\pgfpathlineto{\pgfqpoint{3.103328in}{1.703736in}}%
\pgfpathlineto{\pgfqpoint{3.105020in}{2.722431in}}%
\pgfpathlineto{\pgfqpoint{3.105865in}{2.642177in}}%
\pgfpathlineto{\pgfqpoint{3.106711in}{1.996774in}}%
\pgfpathlineto{\pgfqpoint{3.107557in}{2.076198in}}%
\pgfpathlineto{\pgfqpoint{3.108402in}{2.169972in}}%
\pgfpathlineto{\pgfqpoint{3.109248in}{1.818158in}}%
\pgfpathlineto{\pgfqpoint{3.111785in}{2.218920in}}%
\pgfpathlineto{\pgfqpoint{3.113476in}{2.494672in}}%
\pgfpathlineto{\pgfqpoint{3.116013in}{1.913399in}}%
\pgfpathlineto{\pgfqpoint{3.116858in}{2.739688in}}%
\pgfpathlineto{\pgfqpoint{3.118550in}{1.988273in}}%
\pgfpathlineto{\pgfqpoint{3.120241in}{1.930952in}}%
\pgfpathlineto{\pgfqpoint{3.121086in}{1.947908in}}%
\pgfpathlineto{\pgfqpoint{3.121932in}{2.615136in}}%
\pgfpathlineto{\pgfqpoint{3.122778in}{2.154613in}}%
\pgfpathlineto{\pgfqpoint{3.123623in}{2.660431in}}%
\pgfpathlineto{\pgfqpoint{3.124469in}{1.715298in}}%
\pgfpathlineto{\pgfqpoint{3.125315in}{2.856963in}}%
\pgfpathlineto{\pgfqpoint{3.126160in}{2.488316in}}%
\pgfpathlineto{\pgfqpoint{3.127006in}{2.464065in}}%
\pgfpathlineto{\pgfqpoint{3.127851in}{2.118242in}}%
\pgfpathlineto{\pgfqpoint{3.128697in}{2.720659in}}%
\pgfpathlineto{\pgfqpoint{3.129543in}{1.979565in}}%
\pgfpathlineto{\pgfqpoint{3.130388in}{2.229217in}}%
\pgfpathlineto{\pgfqpoint{3.132080in}{1.931763in}}%
\pgfpathlineto{\pgfqpoint{3.132925in}{2.841623in}}%
\pgfpathlineto{\pgfqpoint{3.133771in}{1.886032in}}%
\pgfpathlineto{\pgfqpoint{3.134616in}{1.965741in}}%
\pgfpathlineto{\pgfqpoint{3.135462in}{2.369175in}}%
\pgfpathlineto{\pgfqpoint{3.136308in}{2.135123in}}%
\pgfpathlineto{\pgfqpoint{3.137153in}{2.201258in}}%
\pgfpathlineto{\pgfqpoint{3.137999in}{1.338780in}}%
\pgfpathlineto{\pgfqpoint{3.139690in}{2.757446in}}%
\pgfpathlineto{\pgfqpoint{3.140536in}{2.466732in}}%
\pgfpathlineto{\pgfqpoint{3.141381in}{2.630697in}}%
\pgfpathlineto{\pgfqpoint{3.143918in}{1.944640in}}%
\pgfpathlineto{\pgfqpoint{3.144764in}{2.721397in}}%
\pgfpathlineto{\pgfqpoint{3.145610in}{2.294276in}}%
\pgfpathlineto{\pgfqpoint{3.147301in}{2.671182in}}%
\pgfpathlineto{\pgfqpoint{3.148146in}{1.914920in}}%
\pgfpathlineto{\pgfqpoint{3.148992in}{2.024509in}}%
\pgfpathlineto{\pgfqpoint{3.149838in}{2.540035in}}%
\pgfpathlineto{\pgfqpoint{3.150683in}{1.805983in}}%
\pgfpathlineto{\pgfqpoint{3.151529in}{2.165131in}}%
\pgfpathlineto{\pgfqpoint{3.153220in}{2.601778in}}%
\pgfpathlineto{\pgfqpoint{3.154066in}{1.933004in}}%
\pgfpathlineto{\pgfqpoint{3.154911in}{2.497449in}}%
\pgfpathlineto{\pgfqpoint{3.155757in}{1.859121in}}%
\pgfpathlineto{\pgfqpoint{3.156603in}{2.132820in}}%
\pgfpathlineto{\pgfqpoint{3.158294in}{2.445964in}}%
\pgfpathlineto{\pgfqpoint{3.159140in}{2.421363in}}%
\pgfpathlineto{\pgfqpoint{3.159985in}{1.963993in}}%
\pgfpathlineto{\pgfqpoint{3.160831in}{2.446374in}}%
\pgfpathlineto{\pgfqpoint{3.161676in}{2.147278in}}%
\pgfpathlineto{\pgfqpoint{3.162522in}{1.992486in}}%
\pgfpathlineto{\pgfqpoint{3.163368in}{2.003320in}}%
\pgfpathlineto{\pgfqpoint{3.164213in}{2.514627in}}%
\pgfpathlineto{\pgfqpoint{3.165059in}{1.821276in}}%
\pgfpathlineto{\pgfqpoint{3.165905in}{2.242550in}}%
\pgfpathlineto{\pgfqpoint{3.167596in}{2.299018in}}%
\pgfpathlineto{\pgfqpoint{3.168441in}{2.160222in}}%
\pgfpathlineto{\pgfqpoint{3.170133in}{2.626784in}}%
\pgfpathlineto{\pgfqpoint{3.171824in}{1.919590in}}%
\pgfpathlineto{\pgfqpoint{3.172670in}{2.415644in}}%
\pgfpathlineto{\pgfqpoint{3.173515in}{1.988776in}}%
\pgfpathlineto{\pgfqpoint{3.174361in}{2.157315in}}%
\pgfpathlineto{\pgfqpoint{3.175206in}{2.631431in}}%
\pgfpathlineto{\pgfqpoint{3.176898in}{1.784079in}}%
\pgfpathlineto{\pgfqpoint{3.177743in}{2.513339in}}%
\pgfpathlineto{\pgfqpoint{3.178589in}{2.127550in}}%
\pgfpathlineto{\pgfqpoint{3.179435in}{2.431371in}}%
\pgfpathlineto{\pgfqpoint{3.181126in}{1.904006in}}%
\pgfpathlineto{\pgfqpoint{3.182817in}{2.767919in}}%
\pgfpathlineto{\pgfqpoint{3.184508in}{1.829455in}}%
\pgfpathlineto{\pgfqpoint{3.186200in}{2.517517in}}%
\pgfpathlineto{\pgfqpoint{3.187045in}{1.746526in}}%
\pgfpathlineto{\pgfqpoint{3.187891in}{2.444373in}}%
\pgfpathlineto{\pgfqpoint{3.188736in}{1.940626in}}%
\pgfpathlineto{\pgfqpoint{3.189582in}{1.911165in}}%
\pgfpathlineto{\pgfqpoint{3.192119in}{2.759019in}}%
\pgfpathlineto{\pgfqpoint{3.193810in}{1.863600in}}%
\pgfpathlineto{\pgfqpoint{3.194656in}{1.918562in}}%
\pgfpathlineto{\pgfqpoint{3.195501in}{2.466969in}}%
\pgfpathlineto{\pgfqpoint{3.196347in}{2.191102in}}%
\pgfpathlineto{\pgfqpoint{3.197193in}{2.028577in}}%
\pgfpathlineto{\pgfqpoint{3.198038in}{2.712846in}}%
\pgfpathlineto{\pgfqpoint{3.198884in}{2.603769in}}%
\pgfpathlineto{\pgfqpoint{3.200575in}{2.180010in}}%
\pgfpathlineto{\pgfqpoint{3.201421in}{2.511658in}}%
\pgfpathlineto{\pgfqpoint{3.202266in}{2.358183in}}%
\pgfpathlineto{\pgfqpoint{3.203112in}{2.000517in}}%
\pgfpathlineto{\pgfqpoint{3.205649in}{2.742138in}}%
\pgfpathlineto{\pgfqpoint{3.207340in}{2.216741in}}%
\pgfpathlineto{\pgfqpoint{3.208186in}{2.713907in}}%
\pgfpathlineto{\pgfqpoint{3.209031in}{1.547692in}}%
\pgfpathlineto{\pgfqpoint{3.209877in}{2.138468in}}%
\pgfpathlineto{\pgfqpoint{3.210723in}{2.299297in}}%
\pgfpathlineto{\pgfqpoint{3.211568in}{2.168470in}}%
\pgfpathlineto{\pgfqpoint{3.212414in}{2.205193in}}%
\pgfpathlineto{\pgfqpoint{3.214105in}{1.463376in}}%
\pgfpathlineto{\pgfqpoint{3.215796in}{2.315794in}}%
\pgfpathlineto{\pgfqpoint{3.217488in}{1.870738in}}%
\pgfpathlineto{\pgfqpoint{3.218333in}{1.903752in}}%
\pgfpathlineto{\pgfqpoint{3.219179in}{1.922433in}}%
\pgfpathlineto{\pgfqpoint{3.220870in}{2.363096in}}%
\pgfpathlineto{\pgfqpoint{3.221716in}{1.871439in}}%
\pgfpathlineto{\pgfqpoint{3.222561in}{2.502489in}}%
\pgfpathlineto{\pgfqpoint{3.223407in}{1.955927in}}%
\pgfpathlineto{\pgfqpoint{3.224253in}{2.341255in}}%
\pgfpathlineto{\pgfqpoint{3.225098in}{2.179058in}}%
\pgfpathlineto{\pgfqpoint{3.225944in}{2.216034in}}%
\pgfpathlineto{\pgfqpoint{3.226789in}{2.307336in}}%
\pgfpathlineto{\pgfqpoint{3.227635in}{2.191248in}}%
\pgfpathlineto{\pgfqpoint{3.228481in}{2.221798in}}%
\pgfpathlineto{\pgfqpoint{3.229326in}{2.378120in}}%
\pgfpathlineto{\pgfqpoint{3.230172in}{1.967771in}}%
\pgfpathlineto{\pgfqpoint{3.231018in}{2.003832in}}%
\pgfpathlineto{\pgfqpoint{3.231863in}{2.720940in}}%
\pgfpathlineto{\pgfqpoint{3.232709in}{1.650367in}}%
\pgfpathlineto{\pgfqpoint{3.233554in}{2.068700in}}%
\pgfpathlineto{\pgfqpoint{3.234400in}{2.281221in}}%
\pgfpathlineto{\pgfqpoint{3.235246in}{1.736779in}}%
\pgfpathlineto{\pgfqpoint{3.236091in}{2.035371in}}%
\pgfpathlineto{\pgfqpoint{3.236937in}{2.340762in}}%
\pgfpathlineto{\pgfqpoint{3.237783in}{1.984869in}}%
\pgfpathlineto{\pgfqpoint{3.238628in}{2.036598in}}%
\pgfpathlineto{\pgfqpoint{3.239474in}{1.958878in}}%
\pgfpathlineto{\pgfqpoint{3.240319in}{2.274140in}}%
\pgfpathlineto{\pgfqpoint{3.241165in}{1.766999in}}%
\pgfpathlineto{\pgfqpoint{3.242856in}{2.412471in}}%
\pgfpathlineto{\pgfqpoint{3.243702in}{1.964661in}}%
\pgfpathlineto{\pgfqpoint{3.244548in}{2.042602in}}%
\pgfpathlineto{\pgfqpoint{3.245393in}{2.566749in}}%
\pgfpathlineto{\pgfqpoint{3.246239in}{2.333012in}}%
\pgfpathlineto{\pgfqpoint{3.247930in}{1.999542in}}%
\pgfpathlineto{\pgfqpoint{3.248776in}{2.500615in}}%
\pgfpathlineto{\pgfqpoint{3.249621in}{2.368106in}}%
\pgfpathlineto{\pgfqpoint{3.251313in}{1.944266in}}%
\pgfpathlineto{\pgfqpoint{3.252158in}{2.454201in}}%
\pgfpathlineto{\pgfqpoint{3.253004in}{2.109843in}}%
\pgfpathlineto{\pgfqpoint{3.253849in}{2.140502in}}%
\pgfpathlineto{\pgfqpoint{3.254695in}{2.243113in}}%
\pgfpathlineto{\pgfqpoint{3.255541in}{2.665460in}}%
\pgfpathlineto{\pgfqpoint{3.258078in}{1.897680in}}%
\pgfpathlineto{\pgfqpoint{3.258923in}{2.689659in}}%
\pgfpathlineto{\pgfqpoint{3.259769in}{2.156155in}}%
\pgfpathlineto{\pgfqpoint{3.260614in}{1.924467in}}%
\pgfpathlineto{\pgfqpoint{3.261460in}{2.531299in}}%
\pgfpathlineto{\pgfqpoint{3.262306in}{1.571622in}}%
\pgfpathlineto{\pgfqpoint{3.263997in}{2.515785in}}%
\pgfpathlineto{\pgfqpoint{3.264843in}{2.366874in}}%
\pgfpathlineto{\pgfqpoint{3.266534in}{1.741984in}}%
\pgfpathlineto{\pgfqpoint{3.268225in}{2.234948in}}%
\pgfpathlineto{\pgfqpoint{3.269071in}{1.720683in}}%
\pgfpathlineto{\pgfqpoint{3.271608in}{2.577561in}}%
\pgfpathlineto{\pgfqpoint{3.272453in}{1.959646in}}%
\pgfpathlineto{\pgfqpoint{3.273299in}{2.642990in}}%
\pgfpathlineto{\pgfqpoint{3.274144in}{2.259930in}}%
\pgfpathlineto{\pgfqpoint{3.274990in}{1.615390in}}%
\pgfpathlineto{\pgfqpoint{3.275836in}{2.105408in}}%
\pgfpathlineto{\pgfqpoint{3.276681in}{2.726655in}}%
\pgfpathlineto{\pgfqpoint{3.277527in}{2.012688in}}%
\pgfpathlineto{\pgfqpoint{3.278372in}{2.135713in}}%
\pgfpathlineto{\pgfqpoint{3.279218in}{1.940436in}}%
\pgfpathlineto{\pgfqpoint{3.280064in}{2.355436in}}%
\pgfpathlineto{\pgfqpoint{3.280909in}{2.072559in}}%
\pgfpathlineto{\pgfqpoint{3.281755in}{2.158502in}}%
\pgfpathlineto{\pgfqpoint{3.282601in}{2.628884in}}%
\pgfpathlineto{\pgfqpoint{3.283446in}{2.302211in}}%
\pgfpathlineto{\pgfqpoint{3.284292in}{2.459043in}}%
\pgfpathlineto{\pgfqpoint{3.285137in}{1.440540in}}%
\pgfpathlineto{\pgfqpoint{3.285983in}{2.355465in}}%
\pgfpathlineto{\pgfqpoint{3.286829in}{1.875734in}}%
\pgfpathlineto{\pgfqpoint{3.287674in}{2.466777in}}%
\pgfpathlineto{\pgfqpoint{3.288520in}{2.172665in}}%
\pgfpathlineto{\pgfqpoint{3.289366in}{2.031904in}}%
\pgfpathlineto{\pgfqpoint{3.290211in}{2.807089in}}%
\pgfpathlineto{\pgfqpoint{3.291057in}{2.508192in}}%
\pgfpathlineto{\pgfqpoint{3.291902in}{1.838212in}}%
\pgfpathlineto{\pgfqpoint{3.292748in}{1.840079in}}%
\pgfpathlineto{\pgfqpoint{3.293594in}{2.207211in}}%
\pgfpathlineto{\pgfqpoint{3.294439in}{2.070332in}}%
\pgfpathlineto{\pgfqpoint{3.295285in}{1.758501in}}%
\pgfpathlineto{\pgfqpoint{3.296976in}{2.740800in}}%
\pgfpathlineto{\pgfqpoint{3.297822in}{2.275906in}}%
\pgfpathlineto{\pgfqpoint{3.298667in}{2.318128in}}%
\pgfpathlineto{\pgfqpoint{3.299513in}{1.959140in}}%
\pgfpathlineto{\pgfqpoint{3.302050in}{2.766410in}}%
\pgfpathlineto{\pgfqpoint{3.303741in}{1.753017in}}%
\pgfpathlineto{\pgfqpoint{3.304587in}{2.404287in}}%
\pgfpathlineto{\pgfqpoint{3.305432in}{1.995111in}}%
\pgfpathlineto{\pgfqpoint{3.306278in}{2.295224in}}%
\pgfpathlineto{\pgfqpoint{3.307124in}{1.947016in}}%
\pgfpathlineto{\pgfqpoint{3.307969in}{2.177563in}}%
\pgfpathlineto{\pgfqpoint{3.308815in}{2.060352in}}%
\pgfpathlineto{\pgfqpoint{3.310506in}{3.014861in}}%
\pgfpathlineto{\pgfqpoint{3.312197in}{1.825476in}}%
\pgfpathlineto{\pgfqpoint{3.314734in}{2.326305in}}%
\pgfpathlineto{\pgfqpoint{3.316426in}{2.670991in}}%
\pgfpathlineto{\pgfqpoint{3.317271in}{1.673271in}}%
\pgfpathlineto{\pgfqpoint{3.318117in}{2.451380in}}%
\pgfpathlineto{\pgfqpoint{3.318962in}{1.927617in}}%
\pgfpathlineto{\pgfqpoint{3.319808in}{2.397552in}}%
\pgfpathlineto{\pgfqpoint{3.320654in}{2.332298in}}%
\pgfpathlineto{\pgfqpoint{3.322345in}{2.027135in}}%
\pgfpathlineto{\pgfqpoint{3.323191in}{2.208250in}}%
\pgfpathlineto{\pgfqpoint{3.324036in}{2.803938in}}%
\pgfpathlineto{\pgfqpoint{3.325727in}{1.779349in}}%
\pgfpathlineto{\pgfqpoint{3.326573in}{2.318261in}}%
\pgfpathlineto{\pgfqpoint{3.327419in}{1.855423in}}%
\pgfpathlineto{\pgfqpoint{3.328264in}{2.017380in}}%
\pgfpathlineto{\pgfqpoint{3.329956in}{2.023758in}}%
\pgfpathlineto{\pgfqpoint{3.331647in}{2.228242in}}%
\pgfpathlineto{\pgfqpoint{3.332492in}{1.830438in}}%
\pgfpathlineto{\pgfqpoint{3.334184in}{2.491353in}}%
\pgfpathlineto{\pgfqpoint{3.335875in}{1.932706in}}%
\pgfpathlineto{\pgfqpoint{3.336721in}{2.842938in}}%
\pgfpathlineto{\pgfqpoint{3.337566in}{2.297087in}}%
\pgfpathlineto{\pgfqpoint{3.338412in}{1.923503in}}%
\pgfpathlineto{\pgfqpoint{3.339257in}{2.376982in}}%
\pgfpathlineto{\pgfqpoint{3.340103in}{2.308611in}}%
\pgfpathlineto{\pgfqpoint{3.340949in}{2.162146in}}%
\pgfpathlineto{\pgfqpoint{3.341794in}{1.691924in}}%
\pgfpathlineto{\pgfqpoint{3.344331in}{2.498912in}}%
\pgfpathlineto{\pgfqpoint{3.345177in}{1.936522in}}%
\pgfpathlineto{\pgfqpoint{3.346022in}{2.199330in}}%
\pgfpathlineto{\pgfqpoint{3.346868in}{2.106619in}}%
\pgfpathlineto{\pgfqpoint{3.347714in}{2.540103in}}%
\pgfpathlineto{\pgfqpoint{3.348559in}{2.044270in}}%
\pgfpathlineto{\pgfqpoint{3.349405in}{2.471704in}}%
\pgfpathlineto{\pgfqpoint{3.350251in}{2.249499in}}%
\pgfpathlineto{\pgfqpoint{3.351096in}{2.168125in}}%
\pgfpathlineto{\pgfqpoint{3.351942in}{1.617971in}}%
\pgfpathlineto{\pgfqpoint{3.354479in}{2.711720in}}%
\pgfpathlineto{\pgfqpoint{3.356170in}{1.858866in}}%
\pgfpathlineto{\pgfqpoint{3.357861in}{2.317775in}}%
\pgfpathlineto{\pgfqpoint{3.358707in}{2.203861in}}%
\pgfpathlineto{\pgfqpoint{3.359552in}{2.318176in}}%
\pgfpathlineto{\pgfqpoint{3.360398in}{1.728054in}}%
\pgfpathlineto{\pgfqpoint{3.361244in}{2.121188in}}%
\pgfpathlineto{\pgfqpoint{3.362089in}{2.112129in}}%
\pgfpathlineto{\pgfqpoint{3.362935in}{1.875775in}}%
\pgfpathlineto{\pgfqpoint{3.363780in}{2.070065in}}%
\pgfpathlineto{\pgfqpoint{3.364626in}{1.930314in}}%
\pgfpathlineto{\pgfqpoint{3.365472in}{2.591381in}}%
\pgfpathlineto{\pgfqpoint{3.366317in}{1.940653in}}%
\pgfpathlineto{\pgfqpoint{3.367163in}{2.109546in}}%
\pgfpathlineto{\pgfqpoint{3.368009in}{2.248389in}}%
\pgfpathlineto{\pgfqpoint{3.369700in}{1.883833in}}%
\pgfpathlineto{\pgfqpoint{3.370545in}{2.326153in}}%
\pgfpathlineto{\pgfqpoint{3.371391in}{1.943035in}}%
\pgfpathlineto{\pgfqpoint{3.373082in}{2.417601in}}%
\pgfpathlineto{\pgfqpoint{3.373928in}{2.417422in}}%
\pgfpathlineto{\pgfqpoint{3.374774in}{2.063708in}}%
\pgfpathlineto{\pgfqpoint{3.375619in}{2.331016in}}%
\pgfpathlineto{\pgfqpoint{3.378156in}{1.976456in}}%
\pgfpathlineto{\pgfqpoint{3.379002in}{1.930811in}}%
\pgfpathlineto{\pgfqpoint{3.379847in}{2.270794in}}%
\pgfpathlineto{\pgfqpoint{3.380693in}{1.868234in}}%
\pgfpathlineto{\pgfqpoint{3.381539in}{2.135981in}}%
\pgfpathlineto{\pgfqpoint{3.383230in}{1.864059in}}%
\pgfpathlineto{\pgfqpoint{3.384921in}{2.488579in}}%
\pgfpathlineto{\pgfqpoint{3.388304in}{1.906459in}}%
\pgfpathlineto{\pgfqpoint{3.389149in}{2.252890in}}%
\pgfpathlineto{\pgfqpoint{3.389995in}{2.001364in}}%
\pgfpathlineto{\pgfqpoint{3.390840in}{1.416587in}}%
\pgfpathlineto{\pgfqpoint{3.392532in}{2.232318in}}%
\pgfpathlineto{\pgfqpoint{3.393377in}{2.206279in}}%
\pgfpathlineto{\pgfqpoint{3.394223in}{1.904760in}}%
\pgfpathlineto{\pgfqpoint{3.395069in}{2.181541in}}%
\pgfpathlineto{\pgfqpoint{3.395914in}{1.666895in}}%
\pgfpathlineto{\pgfqpoint{3.397605in}{2.814530in}}%
\pgfpathlineto{\pgfqpoint{3.398451in}{1.502614in}}%
\pgfpathlineto{\pgfqpoint{3.399297in}{2.542456in}}%
\pgfpathlineto{\pgfqpoint{3.400142in}{2.010237in}}%
\pgfpathlineto{\pgfqpoint{3.400988in}{2.297220in}}%
\pgfpathlineto{\pgfqpoint{3.401834in}{1.938385in}}%
\pgfpathlineto{\pgfqpoint{3.402679in}{2.656173in}}%
\pgfpathlineto{\pgfqpoint{3.403525in}{1.843052in}}%
\pgfpathlineto{\pgfqpoint{3.404370in}{2.331745in}}%
\pgfpathlineto{\pgfqpoint{3.405216in}{2.184970in}}%
\pgfpathlineto{\pgfqpoint{3.406062in}{2.696444in}}%
\pgfpathlineto{\pgfqpoint{3.408599in}{1.308917in}}%
\pgfpathlineto{\pgfqpoint{3.409444in}{1.977999in}}%
\pgfpathlineto{\pgfqpoint{3.410290in}{1.964191in}}%
\pgfpathlineto{\pgfqpoint{3.411135in}{2.013687in}}%
\pgfpathlineto{\pgfqpoint{3.412827in}{2.419940in}}%
\pgfpathlineto{\pgfqpoint{3.415364in}{2.187770in}}%
\pgfpathlineto{\pgfqpoint{3.416209in}{2.655138in}}%
\pgfpathlineto{\pgfqpoint{3.417055in}{1.954837in}}%
\pgfpathlineto{\pgfqpoint{3.417900in}{2.775953in}}%
\pgfpathlineto{\pgfqpoint{3.418746in}{1.893026in}}%
\pgfpathlineto{\pgfqpoint{3.419592in}{2.570382in}}%
\pgfpathlineto{\pgfqpoint{3.422129in}{1.852125in}}%
\pgfpathlineto{\pgfqpoint{3.422974in}{2.729705in}}%
\pgfpathlineto{\pgfqpoint{3.423820in}{2.026784in}}%
\pgfpathlineto{\pgfqpoint{3.424665in}{2.504534in}}%
\pgfpathlineto{\pgfqpoint{3.425511in}{2.304161in}}%
\pgfpathlineto{\pgfqpoint{3.427202in}{1.656827in}}%
\pgfpathlineto{\pgfqpoint{3.428048in}{2.410449in}}%
\pgfpathlineto{\pgfqpoint{3.428894in}{2.183948in}}%
\pgfpathlineto{\pgfqpoint{3.429739in}{1.982695in}}%
\pgfpathlineto{\pgfqpoint{3.430585in}{2.369336in}}%
\pgfpathlineto{\pgfqpoint{3.431430in}{2.208640in}}%
\pgfpathlineto{\pgfqpoint{3.432276in}{2.244494in}}%
\pgfpathlineto{\pgfqpoint{3.433122in}{2.128554in}}%
\pgfpathlineto{\pgfqpoint{3.433967in}{1.497152in}}%
\pgfpathlineto{\pgfqpoint{3.434813in}{1.738785in}}%
\pgfpathlineto{\pgfqpoint{3.435658in}{2.537274in}}%
\pgfpathlineto{\pgfqpoint{3.436504in}{2.041255in}}%
\pgfpathlineto{\pgfqpoint{3.439041in}{3.042024in}}%
\pgfpathlineto{\pgfqpoint{3.441578in}{1.869217in}}%
\pgfpathlineto{\pgfqpoint{3.443269in}{2.212591in}}%
\pgfpathlineto{\pgfqpoint{3.444115in}{2.137488in}}%
\pgfpathlineto{\pgfqpoint{3.444960in}{2.203558in}}%
\pgfpathlineto{\pgfqpoint{3.446652in}{1.972992in}}%
\pgfpathlineto{\pgfqpoint{3.447497in}{2.640006in}}%
\pgfpathlineto{\pgfqpoint{3.448343in}{1.913335in}}%
\pgfpathlineto{\pgfqpoint{3.449188in}{2.337763in}}%
\pgfpathlineto{\pgfqpoint{3.450034in}{2.314641in}}%
\pgfpathlineto{\pgfqpoint{3.450880in}{2.516422in}}%
\pgfpathlineto{\pgfqpoint{3.451725in}{2.418181in}}%
\pgfpathlineto{\pgfqpoint{3.452571in}{2.219157in}}%
\pgfpathlineto{\pgfqpoint{3.453417in}{2.617798in}}%
\pgfpathlineto{\pgfqpoint{3.455953in}{1.699471in}}%
\pgfpathlineto{\pgfqpoint{3.456799in}{2.322521in}}%
\pgfpathlineto{\pgfqpoint{3.457645in}{2.191422in}}%
\pgfpathlineto{\pgfqpoint{3.458490in}{2.226835in}}%
\pgfpathlineto{\pgfqpoint{3.459336in}{2.390551in}}%
\pgfpathlineto{\pgfqpoint{3.461027in}{1.846409in}}%
\pgfpathlineto{\pgfqpoint{3.461873in}{2.450342in}}%
\pgfpathlineto{\pgfqpoint{3.462718in}{2.262138in}}%
\pgfpathlineto{\pgfqpoint{3.463564in}{2.461498in}}%
\pgfpathlineto{\pgfqpoint{3.464410in}{1.629290in}}%
\pgfpathlineto{\pgfqpoint{3.466947in}{2.739072in}}%
\pgfpathlineto{\pgfqpoint{3.467792in}{2.085192in}}%
\pgfpathlineto{\pgfqpoint{3.468638in}{2.660351in}}%
\pgfpathlineto{\pgfqpoint{3.470329in}{1.652667in}}%
\pgfpathlineto{\pgfqpoint{3.471175in}{2.625899in}}%
\pgfpathlineto{\pgfqpoint{3.472020in}{2.025103in}}%
\pgfpathlineto{\pgfqpoint{3.473712in}{2.676867in}}%
\pgfpathlineto{\pgfqpoint{3.474557in}{2.629287in}}%
\pgfpathlineto{\pgfqpoint{3.475403in}{2.486666in}}%
\pgfpathlineto{\pgfqpoint{3.476248in}{1.994022in}}%
\pgfpathlineto{\pgfqpoint{3.477094in}{2.929359in}}%
\pgfpathlineto{\pgfqpoint{3.477940in}{2.174301in}}%
\pgfpathlineto{\pgfqpoint{3.478785in}{1.858337in}}%
\pgfpathlineto{\pgfqpoint{3.479631in}{2.105955in}}%
\pgfpathlineto{\pgfqpoint{3.480477in}{2.019418in}}%
\pgfpathlineto{\pgfqpoint{3.481322in}{2.424572in}}%
\pgfpathlineto{\pgfqpoint{3.482168in}{2.212872in}}%
\pgfpathlineto{\pgfqpoint{3.483013in}{2.188749in}}%
\pgfpathlineto{\pgfqpoint{3.483859in}{2.053612in}}%
\pgfpathlineto{\pgfqpoint{3.484705in}{2.238032in}}%
\pgfpathlineto{\pgfqpoint{3.485550in}{1.952281in}}%
\pgfpathlineto{\pgfqpoint{3.486396in}{2.182811in}}%
\pgfpathlineto{\pgfqpoint{3.488087in}{2.536170in}}%
\pgfpathlineto{\pgfqpoint{3.489778in}{2.026894in}}%
\pgfpathlineto{\pgfqpoint{3.490624in}{2.610243in}}%
\pgfpathlineto{\pgfqpoint{3.491470in}{2.195216in}}%
\pgfpathlineto{\pgfqpoint{3.493161in}{2.688743in}}%
\pgfpathlineto{\pgfqpoint{3.495698in}{1.670774in}}%
\pgfpathlineto{\pgfqpoint{3.496543in}{1.521784in}}%
\pgfpathlineto{\pgfqpoint{3.497389in}{2.656722in}}%
\pgfpathlineto{\pgfqpoint{3.498235in}{2.092609in}}%
\pgfpathlineto{\pgfqpoint{3.499080in}{1.824890in}}%
\pgfpathlineto{\pgfqpoint{3.499926in}{2.458716in}}%
\pgfpathlineto{\pgfqpoint{3.500772in}{2.062370in}}%
\pgfpathlineto{\pgfqpoint{3.501617in}{1.925626in}}%
\pgfpathlineto{\pgfqpoint{3.502463in}{1.489194in}}%
\pgfpathlineto{\pgfqpoint{3.503308in}{2.304070in}}%
\pgfpathlineto{\pgfqpoint{3.504154in}{2.105115in}}%
\pgfpathlineto{\pgfqpoint{3.505845in}{1.824672in}}%
\pgfpathlineto{\pgfqpoint{3.506691in}{1.827398in}}%
\pgfpathlineto{\pgfqpoint{3.507537in}{1.724941in}}%
\pgfpathlineto{\pgfqpoint{3.510073in}{2.325763in}}%
\pgfpathlineto{\pgfqpoint{3.510919in}{2.537589in}}%
\pgfpathlineto{\pgfqpoint{3.511765in}{1.650097in}}%
\pgfpathlineto{\pgfqpoint{3.514301in}{2.692531in}}%
\pgfpathlineto{\pgfqpoint{3.516838in}{1.854497in}}%
\pgfpathlineto{\pgfqpoint{3.519375in}{2.535690in}}%
\pgfpathlineto{\pgfqpoint{3.520221in}{1.919055in}}%
\pgfpathlineto{\pgfqpoint{3.521066in}{2.122632in}}%
\pgfpathlineto{\pgfqpoint{3.521912in}{2.916851in}}%
\pgfpathlineto{\pgfqpoint{3.522758in}{2.290383in}}%
\pgfpathlineto{\pgfqpoint{3.523603in}{2.452800in}}%
\pgfpathlineto{\pgfqpoint{3.524449in}{1.995305in}}%
\pgfpathlineto{\pgfqpoint{3.525295in}{2.168814in}}%
\pgfpathlineto{\pgfqpoint{3.526986in}{2.067574in}}%
\pgfpathlineto{\pgfqpoint{3.527831in}{2.566558in}}%
\pgfpathlineto{\pgfqpoint{3.528677in}{2.447165in}}%
\pgfpathlineto{\pgfqpoint{3.529523in}{2.356952in}}%
\pgfpathlineto{\pgfqpoint{3.530368in}{1.766219in}}%
\pgfpathlineto{\pgfqpoint{3.531214in}{2.250255in}}%
\pgfpathlineto{\pgfqpoint{3.532060in}{2.416461in}}%
\pgfpathlineto{\pgfqpoint{3.532905in}{1.777671in}}%
\pgfpathlineto{\pgfqpoint{3.533751in}{2.645847in}}%
\pgfpathlineto{\pgfqpoint{3.534596in}{1.914379in}}%
\pgfpathlineto{\pgfqpoint{3.535442in}{2.331722in}}%
\pgfpathlineto{\pgfqpoint{3.536288in}{2.480717in}}%
\pgfpathlineto{\pgfqpoint{3.537133in}{2.059451in}}%
\pgfpathlineto{\pgfqpoint{3.537979in}{2.679597in}}%
\pgfpathlineto{\pgfqpoint{3.538825in}{2.668888in}}%
\pgfpathlineto{\pgfqpoint{3.540516in}{1.755877in}}%
\pgfpathlineto{\pgfqpoint{3.541361in}{1.841397in}}%
\pgfpathlineto{\pgfqpoint{3.542207in}{2.335080in}}%
\pgfpathlineto{\pgfqpoint{3.543053in}{2.248543in}}%
\pgfpathlineto{\pgfqpoint{3.543898in}{2.404470in}}%
\pgfpathlineto{\pgfqpoint{3.544744in}{1.560781in}}%
\pgfpathlineto{\pgfqpoint{3.547281in}{2.548557in}}%
\pgfpathlineto{\pgfqpoint{3.548972in}{1.700090in}}%
\pgfpathlineto{\pgfqpoint{3.550663in}{2.220797in}}%
\pgfpathlineto{\pgfqpoint{3.551509in}{2.076951in}}%
\pgfpathlineto{\pgfqpoint{3.552355in}{2.379196in}}%
\pgfpathlineto{\pgfqpoint{3.553200in}{1.890885in}}%
\pgfpathlineto{\pgfqpoint{3.554046in}{2.028365in}}%
\pgfpathlineto{\pgfqpoint{3.554891in}{2.528329in}}%
\pgfpathlineto{\pgfqpoint{3.555737in}{2.450948in}}%
\pgfpathlineto{\pgfqpoint{3.556583in}{2.514838in}}%
\pgfpathlineto{\pgfqpoint{3.557428in}{2.195791in}}%
\pgfpathlineto{\pgfqpoint{3.558274in}{2.666599in}}%
\pgfpathlineto{\pgfqpoint{3.559965in}{1.707160in}}%
\pgfpathlineto{\pgfqpoint{3.561656in}{2.139296in}}%
\pgfpathlineto{\pgfqpoint{3.562502in}{1.705401in}}%
\pgfpathlineto{\pgfqpoint{3.564193in}{2.249305in}}%
\pgfpathlineto{\pgfqpoint{3.565885in}{1.793325in}}%
\pgfpathlineto{\pgfqpoint{3.568421in}{2.679005in}}%
\pgfpathlineto{\pgfqpoint{3.569267in}{2.094048in}}%
\pgfpathlineto{\pgfqpoint{3.570113in}{2.690556in}}%
\pgfpathlineto{\pgfqpoint{3.570958in}{2.293485in}}%
\pgfpathlineto{\pgfqpoint{3.571804in}{1.597413in}}%
\pgfpathlineto{\pgfqpoint{3.572650in}{2.625850in}}%
\pgfpathlineto{\pgfqpoint{3.573495in}{2.234409in}}%
\pgfpathlineto{\pgfqpoint{3.574341in}{2.304280in}}%
\pgfpathlineto{\pgfqpoint{3.575186in}{2.812412in}}%
\pgfpathlineto{\pgfqpoint{3.576032in}{2.606710in}}%
\pgfpathlineto{\pgfqpoint{3.576878in}{1.629186in}}%
\pgfpathlineto{\pgfqpoint{3.577723in}{2.183667in}}%
\pgfpathlineto{\pgfqpoint{3.578569in}{2.688215in}}%
\pgfpathlineto{\pgfqpoint{3.579415in}{2.298585in}}%
\pgfpathlineto{\pgfqpoint{3.580260in}{1.990007in}}%
\pgfpathlineto{\pgfqpoint{3.581106in}{2.579583in}}%
\pgfpathlineto{\pgfqpoint{3.581951in}{1.626705in}}%
\pgfpathlineto{\pgfqpoint{3.582797in}{1.752654in}}%
\pgfpathlineto{\pgfqpoint{3.584488in}{2.544635in}}%
\pgfpathlineto{\pgfqpoint{3.586180in}{1.775943in}}%
\pgfpathlineto{\pgfqpoint{3.587871in}{2.888663in}}%
\pgfpathlineto{\pgfqpoint{3.589562in}{2.068088in}}%
\pgfpathlineto{\pgfqpoint{3.590408in}{2.137601in}}%
\pgfpathlineto{\pgfqpoint{3.591253in}{1.724682in}}%
\pgfpathlineto{\pgfqpoint{3.592099in}{2.799676in}}%
\pgfpathlineto{\pgfqpoint{3.592944in}{1.699022in}}%
\pgfpathlineto{\pgfqpoint{3.593790in}{2.282465in}}%
\pgfpathlineto{\pgfqpoint{3.594636in}{2.268080in}}%
\pgfpathlineto{\pgfqpoint{3.595481in}{2.202795in}}%
\pgfpathlineto{\pgfqpoint{3.596327in}{2.690748in}}%
\pgfpathlineto{\pgfqpoint{3.597173in}{2.069693in}}%
\pgfpathlineto{\pgfqpoint{3.598018in}{2.259481in}}%
\pgfpathlineto{\pgfqpoint{3.598864in}{1.991869in}}%
\pgfpathlineto{\pgfqpoint{3.599709in}{2.300311in}}%
\pgfpathlineto{\pgfqpoint{3.600555in}{1.990982in}}%
\pgfpathlineto{\pgfqpoint{3.601401in}{2.233224in}}%
\pgfpathlineto{\pgfqpoint{3.602246in}{1.940430in}}%
\pgfpathlineto{\pgfqpoint{3.603092in}{2.385670in}}%
\pgfpathlineto{\pgfqpoint{3.603938in}{1.750185in}}%
\pgfpathlineto{\pgfqpoint{3.604783in}{2.066161in}}%
\pgfpathlineto{\pgfqpoint{3.605629in}{2.379735in}}%
\pgfpathlineto{\pgfqpoint{3.606474in}{2.006438in}}%
\pgfpathlineto{\pgfqpoint{3.608166in}{2.520634in}}%
\pgfpathlineto{\pgfqpoint{3.609857in}{1.850433in}}%
\pgfpathlineto{\pgfqpoint{3.610703in}{2.137065in}}%
\pgfpathlineto{\pgfqpoint{3.611548in}{2.546650in}}%
\pgfpathlineto{\pgfqpoint{3.612394in}{2.497813in}}%
\pgfpathlineto{\pgfqpoint{3.615776in}{1.916960in}}%
\pgfpathlineto{\pgfqpoint{3.616622in}{2.293245in}}%
\pgfpathlineto{\pgfqpoint{3.617468in}{2.250127in}}%
\pgfpathlineto{\pgfqpoint{3.618313in}{1.577672in}}%
\pgfpathlineto{\pgfqpoint{3.619159in}{2.031270in}}%
\pgfpathlineto{\pgfqpoint{3.620004in}{2.151296in}}%
\pgfpathlineto{\pgfqpoint{3.620850in}{2.060004in}}%
\pgfpathlineto{\pgfqpoint{3.621696in}{2.610643in}}%
\pgfpathlineto{\pgfqpoint{3.622541in}{2.110014in}}%
\pgfpathlineto{\pgfqpoint{3.623387in}{2.150525in}}%
\pgfpathlineto{\pgfqpoint{3.626769in}{2.620376in}}%
\pgfpathlineto{\pgfqpoint{3.627615in}{1.778660in}}%
\pgfpathlineto{\pgfqpoint{3.628461in}{2.255879in}}%
\pgfpathlineto{\pgfqpoint{3.629306in}{2.254048in}}%
\pgfpathlineto{\pgfqpoint{3.630152in}{2.602724in}}%
\pgfpathlineto{\pgfqpoint{3.630998in}{2.237685in}}%
\pgfpathlineto{\pgfqpoint{3.631843in}{2.759730in}}%
\pgfpathlineto{\pgfqpoint{3.633534in}{2.191849in}}%
\pgfpathlineto{\pgfqpoint{3.635226in}{2.517362in}}%
\pgfpathlineto{\pgfqpoint{3.637763in}{1.640989in}}%
\pgfpathlineto{\pgfqpoint{3.638608in}{1.755120in}}%
\pgfpathlineto{\pgfqpoint{3.639454in}{1.740886in}}%
\pgfpathlineto{\pgfqpoint{3.640299in}{2.114753in}}%
\pgfpathlineto{\pgfqpoint{3.641145in}{2.033800in}}%
\pgfpathlineto{\pgfqpoint{3.641991in}{2.083027in}}%
\pgfpathlineto{\pgfqpoint{3.642836in}{2.621821in}}%
\pgfpathlineto{\pgfqpoint{3.643682in}{2.333761in}}%
\pgfpathlineto{\pgfqpoint{3.645373in}{2.190194in}}%
\pgfpathlineto{\pgfqpoint{3.646219in}{2.700650in}}%
\pgfpathlineto{\pgfqpoint{3.647064in}{2.678644in}}%
\pgfpathlineto{\pgfqpoint{3.647910in}{2.199780in}}%
\pgfpathlineto{\pgfqpoint{3.648756in}{2.918119in}}%
\pgfpathlineto{\pgfqpoint{3.649601in}{2.645258in}}%
\pgfpathlineto{\pgfqpoint{3.650447in}{2.431947in}}%
\pgfpathlineto{\pgfqpoint{3.651293in}{2.758219in}}%
\pgfpathlineto{\pgfqpoint{3.652984in}{1.563979in}}%
\pgfpathlineto{\pgfqpoint{3.653829in}{2.419477in}}%
\pgfpathlineto{\pgfqpoint{3.654675in}{2.091787in}}%
\pgfpathlineto{\pgfqpoint{3.655521in}{2.376773in}}%
\pgfpathlineto{\pgfqpoint{3.657212in}{1.917730in}}%
\pgfpathlineto{\pgfqpoint{3.659749in}{2.351622in}}%
\pgfpathlineto{\pgfqpoint{3.660594in}{1.845093in}}%
\pgfpathlineto{\pgfqpoint{3.661440in}{2.259988in}}%
\pgfpathlineto{\pgfqpoint{3.662286in}{1.970884in}}%
\pgfpathlineto{\pgfqpoint{3.663131in}{2.177035in}}%
\pgfpathlineto{\pgfqpoint{3.664823in}{2.505232in}}%
\pgfpathlineto{\pgfqpoint{3.666514in}{1.846058in}}%
\pgfpathlineto{\pgfqpoint{3.667359in}{2.814374in}}%
\pgfpathlineto{\pgfqpoint{3.668205in}{2.558055in}}%
\pgfpathlineto{\pgfqpoint{3.669051in}{2.004349in}}%
\pgfpathlineto{\pgfqpoint{3.669896in}{2.607892in}}%
\pgfpathlineto{\pgfqpoint{3.670742in}{2.234044in}}%
\pgfpathlineto{\pgfqpoint{3.671587in}{2.431264in}}%
\pgfpathlineto{\pgfqpoint{3.672433in}{2.370472in}}%
\pgfpathlineto{\pgfqpoint{3.673279in}{2.026545in}}%
\pgfpathlineto{\pgfqpoint{3.674124in}{2.230611in}}%
\pgfpathlineto{\pgfqpoint{3.674970in}{2.620702in}}%
\pgfpathlineto{\pgfqpoint{3.675816in}{1.957670in}}%
\pgfpathlineto{\pgfqpoint{3.676661in}{2.361068in}}%
\pgfpathlineto{\pgfqpoint{3.677507in}{2.209613in}}%
\pgfpathlineto{\pgfqpoint{3.678352in}{2.241322in}}%
\pgfpathlineto{\pgfqpoint{3.679198in}{2.267663in}}%
\pgfpathlineto{\pgfqpoint{3.680044in}{2.499071in}}%
\pgfpathlineto{\pgfqpoint{3.680889in}{1.896559in}}%
\pgfpathlineto{\pgfqpoint{3.681735in}{2.109843in}}%
\pgfpathlineto{\pgfqpoint{3.683426in}{2.276071in}}%
\pgfpathlineto{\pgfqpoint{3.684272in}{2.005245in}}%
\pgfpathlineto{\pgfqpoint{3.685117in}{2.190507in}}%
\pgfpathlineto{\pgfqpoint{3.686809in}{2.108612in}}%
\pgfpathlineto{\pgfqpoint{3.687654in}{1.997409in}}%
\pgfpathlineto{\pgfqpoint{3.688500in}{2.388443in}}%
\pgfpathlineto{\pgfqpoint{3.689346in}{1.741730in}}%
\pgfpathlineto{\pgfqpoint{3.690191in}{2.038517in}}%
\pgfpathlineto{\pgfqpoint{3.691882in}{3.051218in}}%
\pgfpathlineto{\pgfqpoint{3.692728in}{2.773830in}}%
\pgfpathlineto{\pgfqpoint{3.695265in}{1.651545in}}%
\pgfpathlineto{\pgfqpoint{3.696111in}{2.562751in}}%
\pgfpathlineto{\pgfqpoint{3.696956in}{1.839351in}}%
\pgfpathlineto{\pgfqpoint{3.699493in}{2.454511in}}%
\pgfpathlineto{\pgfqpoint{3.700339in}{2.104432in}}%
\pgfpathlineto{\pgfqpoint{3.701184in}{2.193986in}}%
\pgfpathlineto{\pgfqpoint{3.702876in}{1.839334in}}%
\pgfpathlineto{\pgfqpoint{3.703721in}{2.702781in}}%
\pgfpathlineto{\pgfqpoint{3.704567in}{2.049878in}}%
\pgfpathlineto{\pgfqpoint{3.705412in}{2.276851in}}%
\pgfpathlineto{\pgfqpoint{3.706258in}{1.888568in}}%
\pgfpathlineto{\pgfqpoint{3.708795in}{2.741099in}}%
\pgfpathlineto{\pgfqpoint{3.709641in}{2.874921in}}%
\pgfpathlineto{\pgfqpoint{3.711332in}{2.104455in}}%
\pgfpathlineto{\pgfqpoint{3.713023in}{2.509251in}}%
\pgfpathlineto{\pgfqpoint{3.715560in}{2.224387in}}%
\pgfpathlineto{\pgfqpoint{3.716406in}{2.123271in}}%
\pgfpathlineto{\pgfqpoint{3.717251in}{1.692944in}}%
\pgfpathlineto{\pgfqpoint{3.718097in}{2.418583in}}%
\pgfpathlineto{\pgfqpoint{3.718942in}{1.690626in}}%
\pgfpathlineto{\pgfqpoint{3.719788in}{2.239764in}}%
\pgfpathlineto{\pgfqpoint{3.721479in}{1.870460in}}%
\pgfpathlineto{\pgfqpoint{3.722325in}{2.562512in}}%
\pgfpathlineto{\pgfqpoint{3.723171in}{2.412047in}}%
\pgfpathlineto{\pgfqpoint{3.724016in}{2.243739in}}%
\pgfpathlineto{\pgfqpoint{3.725707in}{2.897842in}}%
\pgfpathlineto{\pgfqpoint{3.726553in}{2.862496in}}%
\pgfpathlineto{\pgfqpoint{3.727399in}{1.822257in}}%
\pgfpathlineto{\pgfqpoint{3.728244in}{2.475293in}}%
\pgfpathlineto{\pgfqpoint{3.729090in}{1.729324in}}%
\pgfpathlineto{\pgfqpoint{3.729936in}{2.035056in}}%
\pgfpathlineto{\pgfqpoint{3.730781in}{2.524125in}}%
\pgfpathlineto{\pgfqpoint{3.731627in}{2.343182in}}%
\pgfpathlineto{\pgfqpoint{3.732472in}{2.286948in}}%
\pgfpathlineto{\pgfqpoint{3.734164in}{2.613641in}}%
\pgfpathlineto{\pgfqpoint{3.735855in}{2.082659in}}%
\pgfpathlineto{\pgfqpoint{3.736701in}{2.154225in}}%
\pgfpathlineto{\pgfqpoint{3.737546in}{2.013124in}}%
\pgfpathlineto{\pgfqpoint{3.738392in}{2.500141in}}%
\pgfpathlineto{\pgfqpoint{3.739237in}{2.447423in}}%
\pgfpathlineto{\pgfqpoint{3.740083in}{2.374511in}}%
\pgfpathlineto{\pgfqpoint{3.741774in}{2.094990in}}%
\pgfpathlineto{\pgfqpoint{3.742620in}{2.166880in}}%
\pgfpathlineto{\pgfqpoint{3.744311in}{1.524632in}}%
\pgfpathlineto{\pgfqpoint{3.745157in}{2.305056in}}%
\pgfpathlineto{\pgfqpoint{3.746002in}{1.886015in}}%
\pgfpathlineto{\pgfqpoint{3.746848in}{1.759003in}}%
\pgfpathlineto{\pgfqpoint{3.749385in}{2.478022in}}%
\pgfpathlineto{\pgfqpoint{3.750231in}{1.779979in}}%
\pgfpathlineto{\pgfqpoint{3.751076in}{2.536471in}}%
\pgfpathlineto{\pgfqpoint{3.751922in}{1.997474in}}%
\pgfpathlineto{\pgfqpoint{3.752767in}{2.229599in}}%
\pgfpathlineto{\pgfqpoint{3.753613in}{2.038263in}}%
\pgfpathlineto{\pgfqpoint{3.754459in}{1.597268in}}%
\pgfpathlineto{\pgfqpoint{3.755304in}{1.915195in}}%
\pgfpathlineto{\pgfqpoint{3.758687in}{2.655830in}}%
\pgfpathlineto{\pgfqpoint{3.760378in}{2.128351in}}%
\pgfpathlineto{\pgfqpoint{3.761224in}{2.470338in}}%
\pgfpathlineto{\pgfqpoint{3.762069in}{1.412479in}}%
\pgfpathlineto{\pgfqpoint{3.764606in}{2.627051in}}%
\pgfpathlineto{\pgfqpoint{3.765452in}{2.845465in}}%
\pgfpathlineto{\pgfqpoint{3.766297in}{1.782081in}}%
\pgfpathlineto{\pgfqpoint{3.767143in}{2.148801in}}%
\pgfpathlineto{\pgfqpoint{3.767989in}{2.177735in}}%
\pgfpathlineto{\pgfqpoint{3.769680in}{2.321326in}}%
\pgfpathlineto{\pgfqpoint{3.770525in}{2.316486in}}%
\pgfpathlineto{\pgfqpoint{3.771371in}{1.712500in}}%
\pgfpathlineto{\pgfqpoint{3.772217in}{2.965492in}}%
\pgfpathlineto{\pgfqpoint{3.773062in}{2.143282in}}%
\pgfpathlineto{\pgfqpoint{3.773908in}{2.115554in}}%
\pgfpathlineto{\pgfqpoint{3.774754in}{2.388453in}}%
\pgfpathlineto{\pgfqpoint{3.775599in}{2.150854in}}%
\pgfpathlineto{\pgfqpoint{3.776445in}{2.275793in}}%
\pgfpathlineto{\pgfqpoint{3.777290in}{2.283997in}}%
\pgfpathlineto{\pgfqpoint{3.778136in}{2.009068in}}%
\pgfpathlineto{\pgfqpoint{3.778982in}{2.647857in}}%
\pgfpathlineto{\pgfqpoint{3.779827in}{2.169516in}}%
\pgfpathlineto{\pgfqpoint{3.781519in}{2.000497in}}%
\pgfpathlineto{\pgfqpoint{3.784055in}{2.500095in}}%
\pgfpathlineto{\pgfqpoint{3.786592in}{1.696794in}}%
\pgfpathlineto{\pgfqpoint{3.788284in}{2.553606in}}%
\pgfpathlineto{\pgfqpoint{3.789129in}{1.801674in}}%
\pgfpathlineto{\pgfqpoint{3.790820in}{2.699932in}}%
\pgfpathlineto{\pgfqpoint{3.791666in}{2.647824in}}%
\pgfpathlineto{\pgfqpoint{3.792512in}{2.496923in}}%
\pgfpathlineto{\pgfqpoint{3.793357in}{1.814443in}}%
\pgfpathlineto{\pgfqpoint{3.794203in}{1.942614in}}%
\pgfpathlineto{\pgfqpoint{3.795894in}{2.657599in}}%
\pgfpathlineto{\pgfqpoint{3.797585in}{1.936136in}}%
\pgfpathlineto{\pgfqpoint{3.799277in}{2.352375in}}%
\pgfpathlineto{\pgfqpoint{3.800122in}{2.321002in}}%
\pgfpathlineto{\pgfqpoint{3.800968in}{1.930158in}}%
\pgfpathlineto{\pgfqpoint{3.801814in}{2.238416in}}%
\pgfpathlineto{\pgfqpoint{3.802659in}{2.011680in}}%
\pgfpathlineto{\pgfqpoint{3.803505in}{2.712702in}}%
\pgfpathlineto{\pgfqpoint{3.804350in}{1.869217in}}%
\pgfpathlineto{\pgfqpoint{3.805196in}{1.904002in}}%
\pgfpathlineto{\pgfqpoint{3.806042in}{2.386803in}}%
\pgfpathlineto{\pgfqpoint{3.806887in}{2.332557in}}%
\pgfpathlineto{\pgfqpoint{3.807733in}{1.879522in}}%
\pgfpathlineto{\pgfqpoint{3.808579in}{2.518049in}}%
\pgfpathlineto{\pgfqpoint{3.809424in}{2.489431in}}%
\pgfpathlineto{\pgfqpoint{3.810270in}{1.683968in}}%
\pgfpathlineto{\pgfqpoint{3.812807in}{2.650212in}}%
\pgfpathlineto{\pgfqpoint{3.813652in}{2.168808in}}%
\pgfpathlineto{\pgfqpoint{3.814498in}{2.364617in}}%
\pgfpathlineto{\pgfqpoint{3.816189in}{2.132942in}}%
\pgfpathlineto{\pgfqpoint{3.817035in}{2.379206in}}%
\pgfpathlineto{\pgfqpoint{3.817880in}{2.184473in}}%
\pgfpathlineto{\pgfqpoint{3.818726in}{2.401717in}}%
\pgfpathlineto{\pgfqpoint{3.819572in}{2.381164in}}%
\pgfpathlineto{\pgfqpoint{3.820417in}{2.191589in}}%
\pgfpathlineto{\pgfqpoint{3.821263in}{2.362012in}}%
\pgfpathlineto{\pgfqpoint{3.822109in}{1.873737in}}%
\pgfpathlineto{\pgfqpoint{3.822954in}{2.092657in}}%
\pgfpathlineto{\pgfqpoint{3.823800in}{2.068894in}}%
\pgfpathlineto{\pgfqpoint{3.824645in}{1.955637in}}%
\pgfpathlineto{\pgfqpoint{3.826337in}{2.459419in}}%
\pgfpathlineto{\pgfqpoint{3.827182in}{2.455574in}}%
\pgfpathlineto{\pgfqpoint{3.828028in}{1.919014in}}%
\pgfpathlineto{\pgfqpoint{3.828874in}{2.235284in}}%
\pgfpathlineto{\pgfqpoint{3.829719in}{2.220299in}}%
\pgfpathlineto{\pgfqpoint{3.831410in}{1.770602in}}%
\pgfpathlineto{\pgfqpoint{3.832256in}{2.108183in}}%
\pgfpathlineto{\pgfqpoint{3.833102in}{1.975139in}}%
\pgfpathlineto{\pgfqpoint{3.833947in}{2.027253in}}%
\pgfpathlineto{\pgfqpoint{3.834793in}{2.942116in}}%
\pgfpathlineto{\pgfqpoint{3.835638in}{1.820960in}}%
\pgfpathlineto{\pgfqpoint{3.836484in}{2.166345in}}%
\pgfpathlineto{\pgfqpoint{3.837330in}{2.204518in}}%
\pgfpathlineto{\pgfqpoint{3.838175in}{1.907955in}}%
\pgfpathlineto{\pgfqpoint{3.839021in}{2.290653in}}%
\pgfpathlineto{\pgfqpoint{3.839867in}{1.722414in}}%
\pgfpathlineto{\pgfqpoint{3.840712in}{2.307167in}}%
\pgfpathlineto{\pgfqpoint{3.841558in}{2.015548in}}%
\pgfpathlineto{\pgfqpoint{3.842403in}{2.366826in}}%
\pgfpathlineto{\pgfqpoint{3.843249in}{2.106408in}}%
\pgfpathlineto{\pgfqpoint{3.844095in}{2.273110in}}%
\pgfpathlineto{\pgfqpoint{3.845786in}{3.095288in}}%
\pgfpathlineto{\pgfqpoint{3.848323in}{2.221630in}}%
\pgfpathlineto{\pgfqpoint{3.850014in}{1.902913in}}%
\pgfpathlineto{\pgfqpoint{3.852551in}{2.864220in}}%
\pgfpathlineto{\pgfqpoint{3.854242in}{2.278598in}}%
\pgfpathlineto{\pgfqpoint{3.855088in}{2.652585in}}%
\pgfpathlineto{\pgfqpoint{3.856779in}{1.854556in}}%
\pgfpathlineto{\pgfqpoint{3.857625in}{1.872832in}}%
\pgfpathlineto{\pgfqpoint{3.858470in}{1.895501in}}%
\pgfpathlineto{\pgfqpoint{3.859316in}{2.428741in}}%
\pgfpathlineto{\pgfqpoint{3.861007in}{1.692878in}}%
\pgfpathlineto{\pgfqpoint{3.861853in}{2.255361in}}%
\pgfpathlineto{\pgfqpoint{3.862698in}{1.576993in}}%
\pgfpathlineto{\pgfqpoint{3.863544in}{2.267638in}}%
\pgfpathlineto{\pgfqpoint{3.864390in}{1.888423in}}%
\pgfpathlineto{\pgfqpoint{3.866081in}{2.611529in}}%
\pgfpathlineto{\pgfqpoint{3.866927in}{1.899838in}}%
\pgfpathlineto{\pgfqpoint{3.867772in}{2.310094in}}%
\pgfpathlineto{\pgfqpoint{3.870309in}{1.789141in}}%
\pgfpathlineto{\pgfqpoint{3.872000in}{2.622429in}}%
\pgfpathlineto{\pgfqpoint{3.872846in}{3.119490in}}%
\pgfpathlineto{\pgfqpoint{3.875383in}{2.289654in}}%
\pgfpathlineto{\pgfqpoint{3.876228in}{2.283611in}}%
\pgfpathlineto{\pgfqpoint{3.877074in}{2.051390in}}%
\pgfpathlineto{\pgfqpoint{3.878765in}{2.840701in}}%
\pgfpathlineto{\pgfqpoint{3.880457in}{2.083091in}}%
\pgfpathlineto{\pgfqpoint{3.881302in}{2.951227in}}%
\pgfpathlineto{\pgfqpoint{3.882148in}{1.749851in}}%
\pgfpathlineto{\pgfqpoint{3.882993in}{2.349609in}}%
\pgfpathlineto{\pgfqpoint{3.883839in}{2.079384in}}%
\pgfpathlineto{\pgfqpoint{3.884685in}{2.440856in}}%
\pgfpathlineto{\pgfqpoint{3.885530in}{2.382317in}}%
\pgfpathlineto{\pgfqpoint{3.886376in}{2.094208in}}%
\pgfpathlineto{\pgfqpoint{3.887222in}{2.208481in}}%
\pgfpathlineto{\pgfqpoint{3.888067in}{2.283734in}}%
\pgfpathlineto{\pgfqpoint{3.888913in}{2.257153in}}%
\pgfpathlineto{\pgfqpoint{3.889758in}{1.766878in}}%
\pgfpathlineto{\pgfqpoint{3.890604in}{2.348106in}}%
\pgfpathlineto{\pgfqpoint{3.891450in}{2.117712in}}%
\pgfpathlineto{\pgfqpoint{3.893141in}{1.820818in}}%
\pgfpathlineto{\pgfqpoint{3.895678in}{2.590579in}}%
\pgfpathlineto{\pgfqpoint{3.896523in}{2.054543in}}%
\pgfpathlineto{\pgfqpoint{3.897369in}{2.430241in}}%
\pgfpathlineto{\pgfqpoint{3.898215in}{2.000291in}}%
\pgfpathlineto{\pgfqpoint{3.899060in}{2.467967in}}%
\pgfpathlineto{\pgfqpoint{3.899906in}{1.768085in}}%
\pgfpathlineto{\pgfqpoint{3.900752in}{2.108989in}}%
\pgfpathlineto{\pgfqpoint{3.901597in}{2.007923in}}%
\pgfpathlineto{\pgfqpoint{3.902443in}{2.420965in}}%
\pgfpathlineto{\pgfqpoint{3.904134in}{1.742532in}}%
\pgfpathlineto{\pgfqpoint{3.906671in}{2.289090in}}%
\pgfpathlineto{\pgfqpoint{3.907517in}{2.333109in}}%
\pgfpathlineto{\pgfqpoint{3.908362in}{2.640044in}}%
\pgfpathlineto{\pgfqpoint{3.909208in}{1.801066in}}%
\pgfpathlineto{\pgfqpoint{3.910053in}{2.371735in}}%
\pgfpathlineto{\pgfqpoint{3.910899in}{2.662927in}}%
\pgfpathlineto{\pgfqpoint{3.911745in}{2.157839in}}%
\pgfpathlineto{\pgfqpoint{3.912590in}{2.488590in}}%
\pgfpathlineto{\pgfqpoint{3.913436in}{2.351528in}}%
\pgfpathlineto{\pgfqpoint{3.914281in}{1.824889in}}%
\pgfpathlineto{\pgfqpoint{3.915973in}{2.654805in}}%
\pgfpathlineto{\pgfqpoint{3.918510in}{1.820830in}}%
\pgfpathlineto{\pgfqpoint{3.919355in}{2.355308in}}%
\pgfpathlineto{\pgfqpoint{3.920201in}{2.256272in}}%
\pgfpathlineto{\pgfqpoint{3.921046in}{2.291598in}}%
\pgfpathlineto{\pgfqpoint{3.921892in}{1.886100in}}%
\pgfpathlineto{\pgfqpoint{3.922738in}{2.001173in}}%
\pgfpathlineto{\pgfqpoint{3.925275in}{2.845744in}}%
\pgfpathlineto{\pgfqpoint{3.926120in}{2.230514in}}%
\pgfpathlineto{\pgfqpoint{3.926966in}{2.411254in}}%
\pgfpathlineto{\pgfqpoint{3.929503in}{1.999346in}}%
\pgfpathlineto{\pgfqpoint{3.931194in}{2.802201in}}%
\pgfpathlineto{\pgfqpoint{3.934576in}{1.845953in}}%
\pgfpathlineto{\pgfqpoint{3.937113in}{2.606633in}}%
\pgfpathlineto{\pgfqpoint{3.939650in}{1.665325in}}%
\pgfpathlineto{\pgfqpoint{3.940496in}{2.596589in}}%
\pgfpathlineto{\pgfqpoint{3.941341in}{2.199786in}}%
\pgfpathlineto{\pgfqpoint{3.942187in}{2.352898in}}%
\pgfpathlineto{\pgfqpoint{3.943033in}{1.553026in}}%
\pgfpathlineto{\pgfqpoint{3.943878in}{1.896457in}}%
\pgfpathlineto{\pgfqpoint{3.944724in}{1.776883in}}%
\pgfpathlineto{\pgfqpoint{3.947261in}{2.460347in}}%
\pgfpathlineto{\pgfqpoint{3.948106in}{1.583222in}}%
\pgfpathlineto{\pgfqpoint{3.948952in}{2.073459in}}%
\pgfpathlineto{\pgfqpoint{3.949798in}{2.733805in}}%
\pgfpathlineto{\pgfqpoint{3.950643in}{2.514236in}}%
\pgfpathlineto{\pgfqpoint{3.953180in}{1.649438in}}%
\pgfpathlineto{\pgfqpoint{3.954871in}{2.587511in}}%
\pgfpathlineto{\pgfqpoint{3.955717in}{1.888763in}}%
\pgfpathlineto{\pgfqpoint{3.956563in}{2.678047in}}%
\pgfpathlineto{\pgfqpoint{3.957408in}{2.124237in}}%
\pgfpathlineto{\pgfqpoint{3.958254in}{2.180358in}}%
\pgfpathlineto{\pgfqpoint{3.959100in}{2.387707in}}%
\pgfpathlineto{\pgfqpoint{3.959945in}{1.957959in}}%
\pgfpathlineto{\pgfqpoint{3.961636in}{2.547796in}}%
\pgfpathlineto{\pgfqpoint{3.964173in}{1.625714in}}%
\pgfpathlineto{\pgfqpoint{3.965865in}{1.830729in}}%
\pgfpathlineto{\pgfqpoint{3.966710in}{1.809181in}}%
\pgfpathlineto{\pgfqpoint{3.969247in}{2.537273in}}%
\pgfpathlineto{\pgfqpoint{3.970938in}{2.019433in}}%
\pgfpathlineto{\pgfqpoint{3.972630in}{2.565911in}}%
\pgfpathlineto{\pgfqpoint{3.973475in}{1.773036in}}%
\pgfpathlineto{\pgfqpoint{3.974321in}{2.208323in}}%
\pgfpathlineto{\pgfqpoint{3.975166in}{2.227206in}}%
\pgfpathlineto{\pgfqpoint{3.976012in}{1.959641in}}%
\pgfpathlineto{\pgfqpoint{3.976858in}{2.697051in}}%
\pgfpathlineto{\pgfqpoint{3.977703in}{2.030839in}}%
\pgfpathlineto{\pgfqpoint{3.978549in}{2.267890in}}%
\pgfpathlineto{\pgfqpoint{3.981086in}{2.640501in}}%
\pgfpathlineto{\pgfqpoint{3.981931in}{2.797649in}}%
\pgfpathlineto{\pgfqpoint{3.982777in}{2.565142in}}%
\pgfpathlineto{\pgfqpoint{3.983623in}{2.628560in}}%
\pgfpathlineto{\pgfqpoint{3.984468in}{1.506154in}}%
\pgfpathlineto{\pgfqpoint{3.985314in}{2.124104in}}%
\pgfpathlineto{\pgfqpoint{3.986160in}{2.160735in}}%
\pgfpathlineto{\pgfqpoint{3.987005in}{1.778925in}}%
\pgfpathlineto{\pgfqpoint{3.987851in}{2.168843in}}%
\pgfpathlineto{\pgfqpoint{3.988696in}{2.167516in}}%
\pgfpathlineto{\pgfqpoint{3.989542in}{2.138213in}}%
\pgfpathlineto{\pgfqpoint{3.992079in}{3.159647in}}%
\pgfpathlineto{\pgfqpoint{3.993770in}{2.134244in}}%
\pgfpathlineto{\pgfqpoint{3.995461in}{2.493847in}}%
\pgfpathlineto{\pgfqpoint{3.996307in}{1.839093in}}%
\pgfpathlineto{\pgfqpoint{3.997998in}{2.466753in}}%
\pgfpathlineto{\pgfqpoint{3.998844in}{2.035715in}}%
\pgfpathlineto{\pgfqpoint{3.999689in}{2.701498in}}%
\pgfpathlineto{\pgfqpoint{4.000535in}{1.965937in}}%
\pgfpathlineto{\pgfqpoint{4.001381in}{2.317407in}}%
\pgfpathlineto{\pgfqpoint{4.002226in}{2.309001in}}%
\pgfpathlineto{\pgfqpoint{4.003918in}{2.571618in}}%
\pgfpathlineto{\pgfqpoint{4.004763in}{2.998300in}}%
\pgfpathlineto{\pgfqpoint{4.005609in}{2.227843in}}%
\pgfpathlineto{\pgfqpoint{4.006454in}{2.615814in}}%
\pgfpathlineto{\pgfqpoint{4.007300in}{2.840868in}}%
\pgfpathlineto{\pgfqpoint{4.008991in}{2.010614in}}%
\pgfpathlineto{\pgfqpoint{4.009837in}{2.614725in}}%
\pgfpathlineto{\pgfqpoint{4.012374in}{1.629080in}}%
\pgfpathlineto{\pgfqpoint{4.014065in}{2.078448in}}%
\pgfpathlineto{\pgfqpoint{4.014911in}{1.774541in}}%
\pgfpathlineto{\pgfqpoint{4.015756in}{1.859312in}}%
\pgfpathlineto{\pgfqpoint{4.016602in}{2.745640in}}%
\pgfpathlineto{\pgfqpoint{4.017448in}{1.767313in}}%
\pgfpathlineto{\pgfqpoint{4.018293in}{2.078912in}}%
\pgfpathlineto{\pgfqpoint{4.019139in}{2.536310in}}%
\pgfpathlineto{\pgfqpoint{4.019984in}{2.470312in}}%
\pgfpathlineto{\pgfqpoint{4.020830in}{2.499266in}}%
\pgfpathlineto{\pgfqpoint{4.021676in}{2.388197in}}%
\pgfpathlineto{\pgfqpoint{4.023367in}{1.434504in}}%
\pgfpathlineto{\pgfqpoint{4.024213in}{2.153567in}}%
\pgfpathlineto{\pgfqpoint{4.025058in}{2.098560in}}%
\pgfpathlineto{\pgfqpoint{4.025904in}{2.219584in}}%
\pgfpathlineto{\pgfqpoint{4.026749in}{2.163592in}}%
\pgfpathlineto{\pgfqpoint{4.027595in}{1.795359in}}%
\pgfpathlineto{\pgfqpoint{4.028441in}{1.809854in}}%
\pgfpathlineto{\pgfqpoint{4.030132in}{2.229228in}}%
\pgfpathlineto{\pgfqpoint{4.030978in}{1.834944in}}%
\pgfpathlineto{\pgfqpoint{4.031823in}{1.864975in}}%
\pgfpathlineto{\pgfqpoint{4.032669in}{2.231733in}}%
\pgfpathlineto{\pgfqpoint{4.033514in}{1.689404in}}%
\pgfpathlineto{\pgfqpoint{4.035206in}{2.313306in}}%
\pgfpathlineto{\pgfqpoint{4.036051in}{2.228972in}}%
\pgfpathlineto{\pgfqpoint{4.036897in}{2.625857in}}%
\pgfpathlineto{\pgfqpoint{4.037743in}{2.448523in}}%
\pgfpathlineto{\pgfqpoint{4.039434in}{2.489119in}}%
\pgfpathlineto{\pgfqpoint{4.040279in}{2.021571in}}%
\pgfpathlineto{\pgfqpoint{4.041125in}{2.517998in}}%
\pgfpathlineto{\pgfqpoint{4.041971in}{1.559117in}}%
\pgfpathlineto{\pgfqpoint{4.042816in}{2.130493in}}%
\pgfpathlineto{\pgfqpoint{4.043662in}{2.557273in}}%
\pgfpathlineto{\pgfqpoint{4.044508in}{2.419877in}}%
\pgfpathlineto{\pgfqpoint{4.045353in}{1.774312in}}%
\pgfpathlineto{\pgfqpoint{4.046199in}{1.775348in}}%
\pgfpathlineto{\pgfqpoint{4.047890in}{2.612194in}}%
\pgfpathlineto{\pgfqpoint{4.048736in}{2.475810in}}%
\pgfpathlineto{\pgfqpoint{4.049581in}{1.923358in}}%
\pgfpathlineto{\pgfqpoint{4.050427in}{2.431407in}}%
\pgfpathlineto{\pgfqpoint{4.051273in}{2.170862in}}%
\pgfpathlineto{\pgfqpoint{4.052118in}{1.900598in}}%
\pgfpathlineto{\pgfqpoint{4.052964in}{1.984261in}}%
\pgfpathlineto{\pgfqpoint{4.055501in}{2.583194in}}%
\pgfpathlineto{\pgfqpoint{4.057192in}{1.876148in}}%
\pgfpathlineto{\pgfqpoint{4.058038in}{2.536085in}}%
\pgfpathlineto{\pgfqpoint{4.058883in}{2.519200in}}%
\pgfpathlineto{\pgfqpoint{4.059729in}{1.852376in}}%
\pgfpathlineto{\pgfqpoint{4.060574in}{2.024224in}}%
\pgfpathlineto{\pgfqpoint{4.063111in}{2.561765in}}%
\pgfpathlineto{\pgfqpoint{4.067339in}{2.035792in}}%
\pgfpathlineto{\pgfqpoint{4.068185in}{2.202302in}}%
\pgfpathlineto{\pgfqpoint{4.069031in}{2.522252in}}%
\pgfpathlineto{\pgfqpoint{4.071567in}{1.912073in}}%
\pgfpathlineto{\pgfqpoint{4.072413in}{2.202707in}}%
\pgfpathlineto{\pgfqpoint{4.073259in}{2.010182in}}%
\pgfpathlineto{\pgfqpoint{4.074104in}{2.023144in}}%
\pgfpathlineto{\pgfqpoint{4.075796in}{2.875346in}}%
\pgfpathlineto{\pgfqpoint{4.078332in}{1.872931in}}%
\pgfpathlineto{\pgfqpoint{4.080024in}{2.167236in}}%
\pgfpathlineto{\pgfqpoint{4.080869in}{2.639928in}}%
\pgfpathlineto{\pgfqpoint{4.081715in}{2.351758in}}%
\pgfpathlineto{\pgfqpoint{4.082561in}{2.099584in}}%
\pgfpathlineto{\pgfqpoint{4.083406in}{2.524152in}}%
\pgfpathlineto{\pgfqpoint{4.084252in}{2.183327in}}%
\pgfpathlineto{\pgfqpoint{4.085097in}{2.411151in}}%
\pgfpathlineto{\pgfqpoint{4.085943in}{2.132762in}}%
\pgfpathlineto{\pgfqpoint{4.086789in}{2.616500in}}%
\pgfpathlineto{\pgfqpoint{4.087634in}{2.232255in}}%
\pgfpathlineto{\pgfqpoint{4.088480in}{1.741343in}}%
\pgfpathlineto{\pgfqpoint{4.089326in}{2.432386in}}%
\pgfpathlineto{\pgfqpoint{4.090171in}{2.303849in}}%
\pgfpathlineto{\pgfqpoint{4.091017in}{2.036336in}}%
\pgfpathlineto{\pgfqpoint{4.092708in}{2.609121in}}%
\pgfpathlineto{\pgfqpoint{4.094399in}{2.391994in}}%
\pgfpathlineto{\pgfqpoint{4.096936in}{1.888027in}}%
\pgfpathlineto{\pgfqpoint{4.099473in}{2.714259in}}%
\pgfpathlineto{\pgfqpoint{4.101164in}{1.771082in}}%
\pgfpathlineto{\pgfqpoint{4.102856in}{2.567291in}}%
\pgfpathlineto{\pgfqpoint{4.103701in}{1.748969in}}%
\pgfpathlineto{\pgfqpoint{4.104547in}{2.404513in}}%
\pgfpathlineto{\pgfqpoint{4.105392in}{2.646673in}}%
\pgfpathlineto{\pgfqpoint{4.106238in}{2.091053in}}%
\pgfpathlineto{\pgfqpoint{4.107084in}{2.973090in}}%
\pgfpathlineto{\pgfqpoint{4.107929in}{2.428150in}}%
\pgfpathlineto{\pgfqpoint{4.108775in}{2.327009in}}%
\pgfpathlineto{\pgfqpoint{4.109621in}{1.671524in}}%
\pgfpathlineto{\pgfqpoint{4.110466in}{2.245886in}}%
\pgfpathlineto{\pgfqpoint{4.111312in}{1.797579in}}%
\pgfpathlineto{\pgfqpoint{4.113849in}{2.321463in}}%
\pgfpathlineto{\pgfqpoint{4.114694in}{2.389631in}}%
\pgfpathlineto{\pgfqpoint{4.115540in}{1.372145in}}%
\pgfpathlineto{\pgfqpoint{4.116386in}{2.477412in}}%
\pgfpathlineto{\pgfqpoint{4.117231in}{2.171942in}}%
\pgfpathlineto{\pgfqpoint{4.118077in}{2.051979in}}%
\pgfpathlineto{\pgfqpoint{4.118922in}{2.496710in}}%
\pgfpathlineto{\pgfqpoint{4.119768in}{2.383531in}}%
\pgfpathlineto{\pgfqpoint{4.120614in}{1.621815in}}%
\pgfpathlineto{\pgfqpoint{4.121459in}{1.954180in}}%
\pgfpathlineto{\pgfqpoint{4.122305in}{2.593018in}}%
\pgfpathlineto{\pgfqpoint{4.124842in}{1.451510in}}%
\pgfpathlineto{\pgfqpoint{4.125687in}{2.852450in}}%
\pgfpathlineto{\pgfqpoint{4.126533in}{1.921385in}}%
\pgfpathlineto{\pgfqpoint{4.127379in}{2.305269in}}%
\pgfpathlineto{\pgfqpoint{4.128224in}{1.587381in}}%
\pgfpathlineto{\pgfqpoint{4.129916in}{2.759917in}}%
\pgfpathlineto{\pgfqpoint{4.130761in}{2.709142in}}%
\pgfpathlineto{\pgfqpoint{4.131607in}{1.894392in}}%
\pgfpathlineto{\pgfqpoint{4.132452in}{2.278777in}}%
\pgfpathlineto{\pgfqpoint{4.133298in}{2.639536in}}%
\pgfpathlineto{\pgfqpoint{4.134144in}{2.453134in}}%
\pgfpathlineto{\pgfqpoint{4.134989in}{1.968216in}}%
\pgfpathlineto{\pgfqpoint{4.135835in}{2.166694in}}%
\pgfpathlineto{\pgfqpoint{4.137526in}{2.607881in}}%
\pgfpathlineto{\pgfqpoint{4.138372in}{2.296694in}}%
\pgfpathlineto{\pgfqpoint{4.139217in}{2.422053in}}%
\pgfpathlineto{\pgfqpoint{4.140063in}{2.859167in}}%
\pgfpathlineto{\pgfqpoint{4.140909in}{2.025426in}}%
\pgfpathlineto{\pgfqpoint{4.141754in}{2.081331in}}%
\pgfpathlineto{\pgfqpoint{4.142600in}{2.152487in}}%
\pgfpathlineto{\pgfqpoint{4.143446in}{1.954268in}}%
\pgfpathlineto{\pgfqpoint{4.144291in}{2.513727in}}%
\pgfpathlineto{\pgfqpoint{4.145137in}{2.336150in}}%
\pgfpathlineto{\pgfqpoint{4.145982in}{2.457691in}}%
\pgfpathlineto{\pgfqpoint{4.147674in}{1.975264in}}%
\pgfpathlineto{\pgfqpoint{4.148519in}{2.110834in}}%
\pgfpathlineto{\pgfqpoint{4.149365in}{1.922407in}}%
\pgfpathlineto{\pgfqpoint{4.150210in}{2.421330in}}%
\pgfpathlineto{\pgfqpoint{4.151056in}{1.764158in}}%
\pgfpathlineto{\pgfqpoint{4.151902in}{2.157224in}}%
\pgfpathlineto{\pgfqpoint{4.152747in}{1.954882in}}%
\pgfpathlineto{\pgfqpoint{4.153593in}{2.732998in}}%
\pgfpathlineto{\pgfqpoint{4.154439in}{2.697382in}}%
\pgfpathlineto{\pgfqpoint{4.156130in}{1.624772in}}%
\pgfpathlineto{\pgfqpoint{4.157821in}{2.614766in}}%
\pgfpathlineto{\pgfqpoint{4.158667in}{2.572820in}}%
\pgfpathlineto{\pgfqpoint{4.159512in}{2.345016in}}%
\pgfpathlineto{\pgfqpoint{4.160358in}{2.503110in}}%
\pgfpathlineto{\pgfqpoint{4.162049in}{2.351536in}}%
\pgfpathlineto{\pgfqpoint{4.162895in}{2.468980in}}%
\pgfpathlineto{\pgfqpoint{4.163740in}{1.966840in}}%
\pgfpathlineto{\pgfqpoint{4.164586in}{2.151355in}}%
\pgfpathlineto{\pgfqpoint{4.165432in}{2.634567in}}%
\pgfpathlineto{\pgfqpoint{4.167969in}{1.919892in}}%
\pgfpathlineto{\pgfqpoint{4.168814in}{2.086425in}}%
\pgfpathlineto{\pgfqpoint{4.169660in}{1.676454in}}%
\pgfpathlineto{\pgfqpoint{4.170505in}{1.698881in}}%
\pgfpathlineto{\pgfqpoint{4.173042in}{2.622063in}}%
\pgfpathlineto{\pgfqpoint{4.174734in}{1.927492in}}%
\pgfpathlineto{\pgfqpoint{4.175579in}{1.968493in}}%
\pgfpathlineto{\pgfqpoint{4.178116in}{2.352119in}}%
\pgfpathlineto{\pgfqpoint{4.178962in}{1.968238in}}%
\pgfpathlineto{\pgfqpoint{4.179807in}{2.320417in}}%
\pgfpathlineto{\pgfqpoint{4.180653in}{2.305120in}}%
\pgfpathlineto{\pgfqpoint{4.181499in}{1.813127in}}%
\pgfpathlineto{\pgfqpoint{4.182344in}{2.283174in}}%
\pgfpathlineto{\pgfqpoint{4.183190in}{1.481806in}}%
\pgfpathlineto{\pgfqpoint{4.184035in}{2.404353in}}%
\pgfpathlineto{\pgfqpoint{4.184881in}{2.363191in}}%
\pgfpathlineto{\pgfqpoint{4.185727in}{1.002928in}}%
\pgfpathlineto{\pgfqpoint{4.188264in}{2.858199in}}%
\pgfpathlineto{\pgfqpoint{4.189955in}{3.187986in}}%
\pgfpathlineto{\pgfqpoint{4.191646in}{1.982826in}}%
\pgfpathlineto{\pgfqpoint{4.192492in}{2.820058in}}%
\pgfpathlineto{\pgfqpoint{4.193337in}{2.236852in}}%
\pgfpathlineto{\pgfqpoint{4.195029in}{1.946100in}}%
\pgfpathlineto{\pgfqpoint{4.195874in}{2.107380in}}%
\pgfpathlineto{\pgfqpoint{4.196720in}{2.670211in}}%
\pgfpathlineto{\pgfqpoint{4.197565in}{1.593532in}}%
\pgfpathlineto{\pgfqpoint{4.198411in}{1.969743in}}%
\pgfpathlineto{\pgfqpoint{4.199257in}{2.231882in}}%
\pgfpathlineto{\pgfqpoint{4.200102in}{2.199736in}}%
\pgfpathlineto{\pgfqpoint{4.201794in}{1.933251in}}%
\pgfpathlineto{\pgfqpoint{4.202639in}{2.054278in}}%
\pgfpathlineto{\pgfqpoint{4.203485in}{2.594142in}}%
\pgfpathlineto{\pgfqpoint{4.204330in}{1.968873in}}%
\pgfpathlineto{\pgfqpoint{4.205176in}{2.191073in}}%
\pgfpathlineto{\pgfqpoint{4.206022in}{1.869532in}}%
\pgfpathlineto{\pgfqpoint{4.206867in}{2.593596in}}%
\pgfpathlineto{\pgfqpoint{4.207713in}{1.838686in}}%
\pgfpathlineto{\pgfqpoint{4.208559in}{2.285097in}}%
\pgfpathlineto{\pgfqpoint{4.209404in}{2.322009in}}%
\pgfpathlineto{\pgfqpoint{4.211095in}{1.821761in}}%
\pgfpathlineto{\pgfqpoint{4.211941in}{1.887277in}}%
\pgfpathlineto{\pgfqpoint{4.213632in}{2.620992in}}%
\pgfpathlineto{\pgfqpoint{4.214478in}{1.992449in}}%
\pgfpathlineto{\pgfqpoint{4.215324in}{2.333822in}}%
\pgfpathlineto{\pgfqpoint{4.216169in}{2.305833in}}%
\pgfpathlineto{\pgfqpoint{4.217860in}{1.871365in}}%
\pgfpathlineto{\pgfqpoint{4.218706in}{2.591586in}}%
\pgfpathlineto{\pgfqpoint{4.219552in}{2.581952in}}%
\pgfpathlineto{\pgfqpoint{4.220397in}{1.737206in}}%
\pgfpathlineto{\pgfqpoint{4.221243in}{2.146856in}}%
\pgfpathlineto{\pgfqpoint{4.222089in}{1.953418in}}%
\pgfpathlineto{\pgfqpoint{4.223780in}{2.554744in}}%
\pgfpathlineto{\pgfqpoint{4.225471in}{2.202623in}}%
\pgfpathlineto{\pgfqpoint{4.226317in}{2.876359in}}%
\pgfpathlineto{\pgfqpoint{4.227162in}{1.709184in}}%
\pgfpathlineto{\pgfqpoint{4.228008in}{3.065903in}}%
\pgfpathlineto{\pgfqpoint{4.228853in}{1.856972in}}%
\pgfpathlineto{\pgfqpoint{4.229699in}{2.160737in}}%
\pgfpathlineto{\pgfqpoint{4.232236in}{2.565518in}}%
\pgfpathlineto{\pgfqpoint{4.233082in}{2.850329in}}%
\pgfpathlineto{\pgfqpoint{4.234773in}{1.900992in}}%
\pgfpathlineto{\pgfqpoint{4.236464in}{2.775267in}}%
\pgfpathlineto{\pgfqpoint{4.238155in}{1.828858in}}%
\pgfpathlineto{\pgfqpoint{4.239847in}{2.113588in}}%
\pgfpathlineto{\pgfqpoint{4.240692in}{1.533322in}}%
\pgfpathlineto{\pgfqpoint{4.241538in}{1.750492in}}%
\pgfpathlineto{\pgfqpoint{4.242383in}{2.393811in}}%
\pgfpathlineto{\pgfqpoint{4.243229in}{2.257702in}}%
\pgfpathlineto{\pgfqpoint{4.244075in}{2.118220in}}%
\pgfpathlineto{\pgfqpoint{4.244920in}{2.685965in}}%
\pgfpathlineto{\pgfqpoint{4.245766in}{2.346512in}}%
\pgfpathlineto{\pgfqpoint{4.246612in}{2.195317in}}%
\pgfpathlineto{\pgfqpoint{4.247457in}{2.642609in}}%
\pgfpathlineto{\pgfqpoint{4.248303in}{2.183508in}}%
\pgfpathlineto{\pgfqpoint{4.249148in}{2.516908in}}%
\pgfpathlineto{\pgfqpoint{4.251685in}{1.885628in}}%
\pgfpathlineto{\pgfqpoint{4.254222in}{2.705891in}}%
\pgfpathlineto{\pgfqpoint{4.255913in}{1.776794in}}%
\pgfpathlineto{\pgfqpoint{4.256759in}{3.102785in}}%
\pgfpathlineto{\pgfqpoint{4.257605in}{2.375954in}}%
\pgfpathlineto{\pgfqpoint{4.259296in}{2.533941in}}%
\pgfpathlineto{\pgfqpoint{4.260142in}{2.389510in}}%
\pgfpathlineto{\pgfqpoint{4.260987in}{1.809016in}}%
\pgfpathlineto{\pgfqpoint{4.261833in}{1.957636in}}%
\pgfpathlineto{\pgfqpoint{4.264370in}{2.647679in}}%
\pgfpathlineto{\pgfqpoint{4.265215in}{2.646331in}}%
\pgfpathlineto{\pgfqpoint{4.266061in}{2.074594in}}%
\pgfpathlineto{\pgfqpoint{4.266907in}{2.367996in}}%
\pgfpathlineto{\pgfqpoint{4.267752in}{2.288137in}}%
\pgfpathlineto{\pgfqpoint{4.269443in}{1.420831in}}%
\pgfpathlineto{\pgfqpoint{4.270289in}{2.445026in}}%
\pgfpathlineto{\pgfqpoint{4.271135in}{1.682069in}}%
\pgfpathlineto{\pgfqpoint{4.272826in}{2.215482in}}%
\pgfpathlineto{\pgfqpoint{4.273672in}{1.979006in}}%
\pgfpathlineto{\pgfqpoint{4.274517in}{2.055140in}}%
\pgfpathlineto{\pgfqpoint{4.276208in}{1.917361in}}%
\pgfpathlineto{\pgfqpoint{4.277054in}{2.085000in}}%
\pgfpathlineto{\pgfqpoint{4.277900in}{1.868406in}}%
\pgfpathlineto{\pgfqpoint{4.280437in}{2.684702in}}%
\pgfpathlineto{\pgfqpoint{4.282128in}{1.702575in}}%
\pgfpathlineto{\pgfqpoint{4.283819in}{1.941728in}}%
\pgfpathlineto{\pgfqpoint{4.284665in}{2.496009in}}%
\pgfpathlineto{\pgfqpoint{4.285510in}{2.271181in}}%
\pgfpathlineto{\pgfqpoint{4.286356in}{2.444714in}}%
\pgfpathlineto{\pgfqpoint{4.287202in}{2.068612in}}%
\pgfpathlineto{\pgfqpoint{4.288047in}{2.129409in}}%
\pgfpathlineto{\pgfqpoint{4.289738in}{2.608454in}}%
\pgfpathlineto{\pgfqpoint{4.290584in}{1.526439in}}%
\pgfpathlineto{\pgfqpoint{4.291430in}{2.414001in}}%
\pgfpathlineto{\pgfqpoint{4.293967in}{1.691412in}}%
\pgfpathlineto{\pgfqpoint{4.294812in}{1.630007in}}%
\pgfpathlineto{\pgfqpoint{4.295658in}{2.609353in}}%
\pgfpathlineto{\pgfqpoint{4.296503in}{1.856646in}}%
\pgfpathlineto{\pgfqpoint{4.297349in}{2.232124in}}%
\pgfpathlineto{\pgfqpoint{4.299040in}{1.667991in}}%
\pgfpathlineto{\pgfqpoint{4.300732in}{2.874305in}}%
\pgfpathlineto{\pgfqpoint{4.302423in}{2.311719in}}%
\pgfpathlineto{\pgfqpoint{4.303268in}{2.245422in}}%
\pgfpathlineto{\pgfqpoint{4.304114in}{1.855150in}}%
\pgfpathlineto{\pgfqpoint{4.304960in}{2.452447in}}%
\pgfpathlineto{\pgfqpoint{4.305805in}{2.114365in}}%
\pgfpathlineto{\pgfqpoint{4.306651in}{1.893546in}}%
\pgfpathlineto{\pgfqpoint{4.307496in}{2.019957in}}%
\pgfpathlineto{\pgfqpoint{4.310033in}{2.293668in}}%
\pgfpathlineto{\pgfqpoint{4.310879in}{2.164113in}}%
\pgfpathlineto{\pgfqpoint{4.311725in}{2.198905in}}%
\pgfpathlineto{\pgfqpoint{4.312570in}{1.951722in}}%
\pgfpathlineto{\pgfqpoint{4.313416in}{2.688842in}}%
\pgfpathlineto{\pgfqpoint{4.314261in}{2.007783in}}%
\pgfpathlineto{\pgfqpoint{4.315107in}{2.068515in}}%
\pgfpathlineto{\pgfqpoint{4.315953in}{2.347185in}}%
\pgfpathlineto{\pgfqpoint{4.316798in}{2.275978in}}%
\pgfpathlineto{\pgfqpoint{4.317644in}{1.840168in}}%
\pgfpathlineto{\pgfqpoint{4.318490in}{2.430661in}}%
\pgfpathlineto{\pgfqpoint{4.319335in}{2.044075in}}%
\pgfpathlineto{\pgfqpoint{4.320181in}{2.080705in}}%
\pgfpathlineto{\pgfqpoint{4.321026in}{2.342015in}}%
\pgfpathlineto{\pgfqpoint{4.321872in}{1.730462in}}%
\pgfpathlineto{\pgfqpoint{4.322718in}{2.111095in}}%
\pgfpathlineto{\pgfqpoint{4.324409in}{2.432247in}}%
\pgfpathlineto{\pgfqpoint{4.326100in}{1.571166in}}%
\pgfpathlineto{\pgfqpoint{4.328637in}{2.376961in}}%
\pgfpathlineto{\pgfqpoint{4.329483in}{2.392412in}}%
\pgfpathlineto{\pgfqpoint{4.330328in}{2.495536in}}%
\pgfpathlineto{\pgfqpoint{4.331174in}{2.075262in}}%
\pgfpathlineto{\pgfqpoint{4.332020in}{2.514009in}}%
\pgfpathlineto{\pgfqpoint{4.332865in}{2.304613in}}%
\pgfpathlineto{\pgfqpoint{4.333711in}{2.401599in}}%
\pgfpathlineto{\pgfqpoint{4.337093in}{1.866094in}}%
\pgfpathlineto{\pgfqpoint{4.337939in}{2.409414in}}%
\pgfpathlineto{\pgfqpoint{4.338785in}{2.179190in}}%
\pgfpathlineto{\pgfqpoint{4.339630in}{2.204052in}}%
\pgfpathlineto{\pgfqpoint{4.340476in}{2.460128in}}%
\pgfpathlineto{\pgfqpoint{4.341321in}{1.923841in}}%
\pgfpathlineto{\pgfqpoint{4.342167in}{2.813866in}}%
\pgfpathlineto{\pgfqpoint{4.343013in}{2.451935in}}%
\pgfpathlineto{\pgfqpoint{4.343858in}{1.836997in}}%
\pgfpathlineto{\pgfqpoint{4.345550in}{2.627579in}}%
\pgfpathlineto{\pgfqpoint{4.347241in}{2.112839in}}%
\pgfpathlineto{\pgfqpoint{4.349778in}{2.958274in}}%
\pgfpathlineto{\pgfqpoint{4.352315in}{1.929628in}}%
\pgfpathlineto{\pgfqpoint{4.353160in}{1.782580in}}%
\pgfpathlineto{\pgfqpoint{4.354006in}{2.394842in}}%
\pgfpathlineto{\pgfqpoint{4.354851in}{2.374643in}}%
\pgfpathlineto{\pgfqpoint{4.355697in}{1.836319in}}%
\pgfpathlineto{\pgfqpoint{4.356543in}{2.444245in}}%
\pgfpathlineto{\pgfqpoint{4.357388in}{2.097643in}}%
\pgfpathlineto{\pgfqpoint{4.358234in}{2.307255in}}%
\pgfpathlineto{\pgfqpoint{4.359080in}{2.071574in}}%
\pgfpathlineto{\pgfqpoint{4.359925in}{2.203803in}}%
\pgfpathlineto{\pgfqpoint{4.360771in}{2.201246in}}%
\pgfpathlineto{\pgfqpoint{4.361616in}{1.877640in}}%
\pgfpathlineto{\pgfqpoint{4.362462in}{2.741471in}}%
\pgfpathlineto{\pgfqpoint{4.363308in}{2.330374in}}%
\pgfpathlineto{\pgfqpoint{4.364153in}{1.764168in}}%
\pgfpathlineto{\pgfqpoint{4.364999in}{2.114465in}}%
\pgfpathlineto{\pgfqpoint{4.365845in}{2.394812in}}%
\pgfpathlineto{\pgfqpoint{4.366690in}{1.823568in}}%
\pgfpathlineto{\pgfqpoint{4.367536in}{2.621831in}}%
\pgfpathlineto{\pgfqpoint{4.368381in}{2.520589in}}%
\pgfpathlineto{\pgfqpoint{4.369227in}{1.542866in}}%
\pgfpathlineto{\pgfqpoint{4.370073in}{2.060468in}}%
\pgfpathlineto{\pgfqpoint{4.370918in}{1.772794in}}%
\pgfpathlineto{\pgfqpoint{4.371764in}{2.666544in}}%
\pgfpathlineto{\pgfqpoint{4.372610in}{2.470084in}}%
\pgfpathlineto{\pgfqpoint{4.373455in}{1.710907in}}%
\pgfpathlineto{\pgfqpoint{4.374301in}{1.950526in}}%
\pgfpathlineto{\pgfqpoint{4.375146in}{2.905628in}}%
\pgfpathlineto{\pgfqpoint{4.375992in}{2.341146in}}%
\pgfpathlineto{\pgfqpoint{4.376838in}{2.069459in}}%
\pgfpathlineto{\pgfqpoint{4.377683in}{2.403213in}}%
\pgfpathlineto{\pgfqpoint{4.378529in}{1.728709in}}%
\pgfpathlineto{\pgfqpoint{4.381066in}{3.066008in}}%
\pgfpathlineto{\pgfqpoint{4.381911in}{1.871790in}}%
\pgfpathlineto{\pgfqpoint{4.382757in}{2.207883in}}%
\pgfpathlineto{\pgfqpoint{4.383603in}{2.553538in}}%
\pgfpathlineto{\pgfqpoint{4.384448in}{2.537992in}}%
\pgfpathlineto{\pgfqpoint{4.385294in}{1.806794in}}%
\pgfpathlineto{\pgfqpoint{4.386140in}{2.739149in}}%
\pgfpathlineto{\pgfqpoint{4.386985in}{2.062102in}}%
\pgfpathlineto{\pgfqpoint{4.387831in}{2.539158in}}%
\pgfpathlineto{\pgfqpoint{4.388676in}{1.920061in}}%
\pgfpathlineto{\pgfqpoint{4.389522in}{2.053869in}}%
\pgfpathlineto{\pgfqpoint{4.390368in}{2.080744in}}%
\pgfpathlineto{\pgfqpoint{4.391213in}{2.692021in}}%
\pgfpathlineto{\pgfqpoint{4.392059in}{2.497384in}}%
\pgfpathlineto{\pgfqpoint{4.392904in}{1.768603in}}%
\pgfpathlineto{\pgfqpoint{4.393750in}{1.771812in}}%
\pgfpathlineto{\pgfqpoint{4.394596in}{2.490690in}}%
\pgfpathlineto{\pgfqpoint{4.395441in}{2.237964in}}%
\pgfpathlineto{\pgfqpoint{4.396287in}{1.851481in}}%
\pgfpathlineto{\pgfqpoint{4.397133in}{2.645135in}}%
\pgfpathlineto{\pgfqpoint{4.397978in}{2.160260in}}%
\pgfpathlineto{\pgfqpoint{4.399669in}{1.777817in}}%
\pgfpathlineto{\pgfqpoint{4.400515in}{2.189801in}}%
\pgfpathlineto{\pgfqpoint{4.401361in}{1.634609in}}%
\pgfpathlineto{\pgfqpoint{4.403052in}{2.481335in}}%
\pgfpathlineto{\pgfqpoint{4.403898in}{2.010255in}}%
\pgfpathlineto{\pgfqpoint{4.404743in}{2.567293in}}%
\pgfpathlineto{\pgfqpoint{4.405589in}{1.791476in}}%
\pgfpathlineto{\pgfqpoint{4.406434in}{1.823410in}}%
\pgfpathlineto{\pgfqpoint{4.408971in}{2.729280in}}%
\pgfpathlineto{\pgfqpoint{4.409817in}{2.674612in}}%
\pgfpathlineto{\pgfqpoint{4.411508in}{1.490798in}}%
\pgfpathlineto{\pgfqpoint{4.414045in}{2.505054in}}%
\pgfpathlineto{\pgfqpoint{4.414891in}{1.728318in}}%
\pgfpathlineto{\pgfqpoint{4.415736in}{2.142387in}}%
\pgfpathlineto{\pgfqpoint{4.416582in}{1.776105in}}%
\pgfpathlineto{\pgfqpoint{4.417428in}{2.363982in}}%
\pgfpathlineto{\pgfqpoint{4.418273in}{1.943461in}}%
\pgfpathlineto{\pgfqpoint{4.419119in}{2.269541in}}%
\pgfpathlineto{\pgfqpoint{4.419964in}{1.935132in}}%
\pgfpathlineto{\pgfqpoint{4.421656in}{2.503257in}}%
\pgfpathlineto{\pgfqpoint{4.422501in}{1.698272in}}%
\pgfpathlineto{\pgfqpoint{4.423347in}{1.903719in}}%
\pgfpathlineto{\pgfqpoint{4.425038in}{2.426396in}}%
\pgfpathlineto{\pgfqpoint{4.425884in}{2.061303in}}%
\pgfpathlineto{\pgfqpoint{4.426729in}{2.862770in}}%
\pgfpathlineto{\pgfqpoint{4.427575in}{2.010204in}}%
\pgfpathlineto{\pgfqpoint{4.428421in}{2.260644in}}%
\pgfpathlineto{\pgfqpoint{4.429266in}{2.551337in}}%
\pgfpathlineto{\pgfqpoint{4.430112in}{2.314588in}}%
\pgfpathlineto{\pgfqpoint{4.430958in}{1.257285in}}%
\pgfpathlineto{\pgfqpoint{4.431803in}{1.966083in}}%
\pgfpathlineto{\pgfqpoint{4.432649in}{3.157879in}}%
\pgfpathlineto{\pgfqpoint{4.433494in}{2.494952in}}%
\pgfpathlineto{\pgfqpoint{4.434340in}{2.726252in}}%
\pgfpathlineto{\pgfqpoint{4.435186in}{2.126097in}}%
\pgfpathlineto{\pgfqpoint{4.436031in}{2.525050in}}%
\pgfpathlineto{\pgfqpoint{4.437723in}{1.435023in}}%
\pgfpathlineto{\pgfqpoint{4.439414in}{2.352888in}}%
\pgfpathlineto{\pgfqpoint{4.440259in}{1.534030in}}%
\pgfpathlineto{\pgfqpoint{4.441105in}{2.024408in}}%
\pgfpathlineto{\pgfqpoint{4.441951in}{1.806062in}}%
\pgfpathlineto{\pgfqpoint{4.442796in}{1.878909in}}%
\pgfpathlineto{\pgfqpoint{4.443642in}{2.569187in}}%
\pgfpathlineto{\pgfqpoint{4.444488in}{1.801397in}}%
\pgfpathlineto{\pgfqpoint{4.445333in}{2.451549in}}%
\pgfpathlineto{\pgfqpoint{4.446179in}{2.346328in}}%
\pgfpathlineto{\pgfqpoint{4.447870in}{1.833227in}}%
\pgfpathlineto{\pgfqpoint{4.448716in}{2.210628in}}%
\pgfpathlineto{\pgfqpoint{4.449561in}{1.802850in}}%
\pgfpathlineto{\pgfqpoint{4.450407in}{1.832304in}}%
\pgfpathlineto{\pgfqpoint{4.452098in}{2.708215in}}%
\pgfpathlineto{\pgfqpoint{4.453789in}{2.138058in}}%
\pgfpathlineto{\pgfqpoint{4.454635in}{3.011177in}}%
\pgfpathlineto{\pgfqpoint{4.455481in}{2.524754in}}%
\pgfpathlineto{\pgfqpoint{4.456326in}{2.049497in}}%
\pgfpathlineto{\pgfqpoint{4.458018in}{2.472850in}}%
\pgfpathlineto{\pgfqpoint{4.458863in}{2.010409in}}%
\pgfpathlineto{\pgfqpoint{4.459709in}{2.465646in}}%
\pgfpathlineto{\pgfqpoint{4.460554in}{2.116457in}}%
\pgfpathlineto{\pgfqpoint{4.461400in}{2.191070in}}%
\pgfpathlineto{\pgfqpoint{4.463091in}{2.594315in}}%
\pgfpathlineto{\pgfqpoint{4.463937in}{2.133110in}}%
\pgfpathlineto{\pgfqpoint{4.464783in}{2.282375in}}%
\pgfpathlineto{\pgfqpoint{4.465628in}{2.517736in}}%
\pgfpathlineto{\pgfqpoint{4.467319in}{1.932140in}}%
\pgfpathlineto{\pgfqpoint{4.468165in}{2.681768in}}%
\pgfpathlineto{\pgfqpoint{4.469856in}{1.696801in}}%
\pgfpathlineto{\pgfqpoint{4.471547in}{2.447049in}}%
\pgfpathlineto{\pgfqpoint{4.472393in}{1.998137in}}%
\pgfpathlineto{\pgfqpoint{4.473239in}{2.210126in}}%
\pgfpathlineto{\pgfqpoint{4.474084in}{1.956506in}}%
\pgfpathlineto{\pgfqpoint{4.476621in}{2.310811in}}%
\pgfpathlineto{\pgfqpoint{4.477467in}{2.240846in}}%
\pgfpathlineto{\pgfqpoint{4.478312in}{1.799634in}}%
\pgfpathlineto{\pgfqpoint{4.479158in}{1.921917in}}%
\pgfpathlineto{\pgfqpoint{4.480004in}{1.907517in}}%
\pgfpathlineto{\pgfqpoint{4.480849in}{1.846079in}}%
\pgfpathlineto{\pgfqpoint{4.483386in}{2.164333in}}%
\pgfpathlineto{\pgfqpoint{4.484232in}{3.125315in}}%
\pgfpathlineto{\pgfqpoint{4.485923in}{1.936013in}}%
\pgfpathlineto{\pgfqpoint{4.486769in}{2.002208in}}%
\pgfpathlineto{\pgfqpoint{4.487614in}{2.358371in}}%
\pgfpathlineto{\pgfqpoint{4.488460in}{1.039808in}}%
\pgfpathlineto{\pgfqpoint{4.490151in}{2.719165in}}%
\pgfpathlineto{\pgfqpoint{4.491842in}{2.052262in}}%
\pgfpathlineto{\pgfqpoint{4.492688in}{2.800797in}}%
\pgfpathlineto{\pgfqpoint{4.493534in}{2.057993in}}%
\pgfpathlineto{\pgfqpoint{4.494379in}{2.969110in}}%
\pgfpathlineto{\pgfqpoint{4.496071in}{1.877265in}}%
\pgfpathlineto{\pgfqpoint{4.497762in}{2.632496in}}%
\pgfpathlineto{\pgfqpoint{4.499453in}{1.805534in}}%
\pgfpathlineto{\pgfqpoint{4.500299in}{1.983600in}}%
\pgfpathlineto{\pgfqpoint{4.501144in}{2.816842in}}%
\pgfpathlineto{\pgfqpoint{4.501990in}{2.422633in}}%
\pgfpathlineto{\pgfqpoint{4.502836in}{2.300765in}}%
\pgfpathlineto{\pgfqpoint{4.503681in}{1.485691in}}%
\pgfpathlineto{\pgfqpoint{4.505372in}{3.262657in}}%
\pgfpathlineto{\pgfqpoint{4.507909in}{2.163236in}}%
\pgfpathlineto{\pgfqpoint{4.510446in}{1.738391in}}%
\pgfpathlineto{\pgfqpoint{4.511292in}{2.207288in}}%
\pgfpathlineto{\pgfqpoint{4.512137in}{1.652794in}}%
\pgfpathlineto{\pgfqpoint{4.512983in}{2.809179in}}%
\pgfpathlineto{\pgfqpoint{4.513829in}{2.074162in}}%
\pgfpathlineto{\pgfqpoint{4.514674in}{2.011369in}}%
\pgfpathlineto{\pgfqpoint{4.515520in}{2.530310in}}%
\pgfpathlineto{\pgfqpoint{4.516366in}{2.301389in}}%
\pgfpathlineto{\pgfqpoint{4.517211in}{2.156182in}}%
\pgfpathlineto{\pgfqpoint{4.518057in}{2.167292in}}%
\pgfpathlineto{\pgfqpoint{4.518902in}{1.476202in}}%
\pgfpathlineto{\pgfqpoint{4.519748in}{2.482587in}}%
\pgfpathlineto{\pgfqpoint{4.520594in}{2.481787in}}%
\pgfpathlineto{\pgfqpoint{4.523131in}{1.878461in}}%
\pgfpathlineto{\pgfqpoint{4.523976in}{2.881171in}}%
\pgfpathlineto{\pgfqpoint{4.524822in}{2.238314in}}%
\pgfpathlineto{\pgfqpoint{4.525667in}{1.844067in}}%
\pgfpathlineto{\pgfqpoint{4.526513in}{2.363626in}}%
\pgfpathlineto{\pgfqpoint{4.527359in}{1.775231in}}%
\pgfpathlineto{\pgfqpoint{4.528204in}{2.590528in}}%
\pgfpathlineto{\pgfqpoint{4.529050in}{1.880551in}}%
\pgfpathlineto{\pgfqpoint{4.529896in}{2.075827in}}%
\pgfpathlineto{\pgfqpoint{4.530741in}{2.066225in}}%
\pgfpathlineto{\pgfqpoint{4.531587in}{2.778822in}}%
\pgfpathlineto{\pgfqpoint{4.532432in}{2.279496in}}%
\pgfpathlineto{\pgfqpoint{4.533278in}{2.383785in}}%
\pgfpathlineto{\pgfqpoint{4.534124in}{2.812636in}}%
\pgfpathlineto{\pgfqpoint{4.534969in}{1.904932in}}%
\pgfpathlineto{\pgfqpoint{4.535815in}{2.171051in}}%
\pgfpathlineto{\pgfqpoint{4.537506in}{1.864261in}}%
\pgfpathlineto{\pgfqpoint{4.540043in}{2.878401in}}%
\pgfpathlineto{\pgfqpoint{4.541734in}{2.164106in}}%
\pgfpathlineto{\pgfqpoint{4.542580in}{2.310491in}}%
\pgfpathlineto{\pgfqpoint{4.543426in}{2.414295in}}%
\pgfpathlineto{\pgfqpoint{4.544271in}{2.333342in}}%
\pgfpathlineto{\pgfqpoint{4.546808in}{1.916239in}}%
\pgfpathlineto{\pgfqpoint{4.548499in}{2.481462in}}%
\pgfpathlineto{\pgfqpoint{4.550190in}{1.914118in}}%
\pgfpathlineto{\pgfqpoint{4.551036in}{2.706289in}}%
\pgfpathlineto{\pgfqpoint{4.551882in}{1.881600in}}%
\pgfpathlineto{\pgfqpoint{4.552727in}{2.309392in}}%
\pgfpathlineto{\pgfqpoint{4.553573in}{2.455319in}}%
\pgfpathlineto{\pgfqpoint{4.556110in}{1.827314in}}%
\pgfpathlineto{\pgfqpoint{4.558647in}{2.252292in}}%
\pgfpathlineto{\pgfqpoint{4.559492in}{2.478796in}}%
\pgfpathlineto{\pgfqpoint{4.560338in}{1.976847in}}%
\pgfpathlineto{\pgfqpoint{4.561184in}{2.370407in}}%
\pgfpathlineto{\pgfqpoint{4.562029in}{2.147811in}}%
\pgfpathlineto{\pgfqpoint{4.562875in}{2.222129in}}%
\pgfpathlineto{\pgfqpoint{4.563720in}{0.637273in}}%
\pgfpathlineto{\pgfqpoint{4.566257in}{2.517443in}}%
\pgfpathlineto{\pgfqpoint{4.567949in}{1.938362in}}%
\pgfpathlineto{\pgfqpoint{4.568794in}{2.249695in}}%
\pgfpathlineto{\pgfqpoint{4.571331in}{1.733944in}}%
\pgfpathlineto{\pgfqpoint{4.574714in}{2.904618in}}%
\pgfpathlineto{\pgfqpoint{4.578096in}{1.123854in}}%
\pgfpathlineto{\pgfqpoint{4.578942in}{1.807698in}}%
\pgfpathlineto{\pgfqpoint{4.579787in}{2.000096in}}%
\pgfpathlineto{\pgfqpoint{4.580633in}{1.744787in}}%
\pgfpathlineto{\pgfqpoint{4.581479in}{2.508014in}}%
\pgfpathlineto{\pgfqpoint{4.582324in}{2.257319in}}%
\pgfpathlineto{\pgfqpoint{4.583170in}{1.580873in}}%
\pgfpathlineto{\pgfqpoint{4.584015in}{2.499577in}}%
\pgfpathlineto{\pgfqpoint{4.584861in}{2.104691in}}%
\pgfpathlineto{\pgfqpoint{4.585707in}{2.419069in}}%
\pgfpathlineto{\pgfqpoint{4.586552in}{2.205234in}}%
\pgfpathlineto{\pgfqpoint{4.587398in}{2.418827in}}%
\pgfpathlineto{\pgfqpoint{4.588244in}{1.913400in}}%
\pgfpathlineto{\pgfqpoint{4.589089in}{2.275216in}}%
\pgfpathlineto{\pgfqpoint{4.589935in}{2.545624in}}%
\pgfpathlineto{\pgfqpoint{4.590780in}{2.390173in}}%
\pgfpathlineto{\pgfqpoint{4.591626in}{2.510642in}}%
\pgfpathlineto{\pgfqpoint{4.592472in}{2.001438in}}%
\pgfpathlineto{\pgfqpoint{4.593317in}{2.078713in}}%
\pgfpathlineto{\pgfqpoint{4.595009in}{2.683575in}}%
\pgfpathlineto{\pgfqpoint{4.596700in}{1.802882in}}%
\pgfpathlineto{\pgfqpoint{4.597545in}{2.469206in}}%
\pgfpathlineto{\pgfqpoint{4.598391in}{2.317843in}}%
\pgfpathlineto{\pgfqpoint{4.600082in}{1.379120in}}%
\pgfpathlineto{\pgfqpoint{4.600928in}{2.220680in}}%
\pgfpathlineto{\pgfqpoint{4.601774in}{1.476398in}}%
\pgfpathlineto{\pgfqpoint{4.602619in}{2.252984in}}%
\pgfpathlineto{\pgfqpoint{4.603465in}{1.709060in}}%
\pgfpathlineto{\pgfqpoint{4.605156in}{2.447483in}}%
\pgfpathlineto{\pgfqpoint{4.606002in}{2.021001in}}%
\pgfpathlineto{\pgfqpoint{4.606847in}{2.384278in}}%
\pgfpathlineto{\pgfqpoint{4.607693in}{2.366759in}}%
\pgfpathlineto{\pgfqpoint{4.611075in}{1.809665in}}%
\pgfpathlineto{\pgfqpoint{4.613612in}{2.869422in}}%
\pgfpathlineto{\pgfqpoint{4.614458in}{1.892157in}}%
\pgfpathlineto{\pgfqpoint{4.615304in}{2.028474in}}%
\pgfpathlineto{\pgfqpoint{4.616149in}{1.795504in}}%
\pgfpathlineto{\pgfqpoint{4.616995in}{2.338498in}}%
\pgfpathlineto{\pgfqpoint{4.617840in}{2.223688in}}%
\pgfpathlineto{\pgfqpoint{4.618686in}{1.627192in}}%
\pgfpathlineto{\pgfqpoint{4.619532in}{2.707790in}}%
\pgfpathlineto{\pgfqpoint{4.620377in}{2.084695in}}%
\pgfpathlineto{\pgfqpoint{4.621223in}{2.098296in}}%
\pgfpathlineto{\pgfqpoint{4.622069in}{1.163685in}}%
\pgfpathlineto{\pgfqpoint{4.622914in}{2.782475in}}%
\pgfpathlineto{\pgfqpoint{4.623760in}{2.291520in}}%
\pgfpathlineto{\pgfqpoint{4.624605in}{2.155593in}}%
\pgfpathlineto{\pgfqpoint{4.625451in}{1.751550in}}%
\pgfpathlineto{\pgfqpoint{4.627988in}{2.833651in}}%
\pgfpathlineto{\pgfqpoint{4.628833in}{1.607408in}}%
\pgfpathlineto{\pgfqpoint{4.629679in}{2.138643in}}%
\pgfpathlineto{\pgfqpoint{4.631370in}{2.244799in}}%
\pgfpathlineto{\pgfqpoint{4.632216in}{1.407261in}}%
\pgfpathlineto{\pgfqpoint{4.633062in}{1.768890in}}%
\pgfpathlineto{\pgfqpoint{4.634753in}{2.494570in}}%
\pgfpathlineto{\pgfqpoint{4.635598in}{2.158195in}}%
\pgfpathlineto{\pgfqpoint{4.637290in}{1.544267in}}%
\pgfpathlineto{\pgfqpoint{4.638135in}{2.479849in}}%
\pgfpathlineto{\pgfqpoint{4.638981in}{2.250404in}}%
\pgfpathlineto{\pgfqpoint{4.640672in}{1.786587in}}%
\pgfpathlineto{\pgfqpoint{4.641518in}{1.466707in}}%
\pgfpathlineto{\pgfqpoint{4.643209in}{2.544820in}}%
\pgfpathlineto{\pgfqpoint{4.644900in}{2.023103in}}%
\pgfpathlineto{\pgfqpoint{4.645746in}{2.707280in}}%
\pgfpathlineto{\pgfqpoint{4.647437in}{1.625838in}}%
\pgfpathlineto{\pgfqpoint{4.648283in}{1.733343in}}%
\pgfpathlineto{\pgfqpoint{4.649128in}{2.662263in}}%
\pgfpathlineto{\pgfqpoint{4.649974in}{2.141527in}}%
\pgfpathlineto{\pgfqpoint{4.650820in}{2.432747in}}%
\pgfpathlineto{\pgfqpoint{4.651665in}{2.265956in}}%
\pgfpathlineto{\pgfqpoint{4.652511in}{2.103470in}}%
\pgfpathlineto{\pgfqpoint{4.654202in}{2.585791in}}%
\pgfpathlineto{\pgfqpoint{4.655048in}{1.755032in}}%
\pgfpathlineto{\pgfqpoint{4.655893in}{2.860907in}}%
\pgfpathlineto{\pgfqpoint{4.656739in}{2.123777in}}%
\pgfpathlineto{\pgfqpoint{4.658430in}{2.825785in}}%
\pgfpathlineto{\pgfqpoint{4.659276in}{2.169545in}}%
\pgfpathlineto{\pgfqpoint{4.660122in}{2.584648in}}%
\pgfpathlineto{\pgfqpoint{4.660967in}{2.462682in}}%
\pgfpathlineto{\pgfqpoint{4.661813in}{2.489782in}}%
\pgfpathlineto{\pgfqpoint{4.662658in}{1.752373in}}%
\pgfpathlineto{\pgfqpoint{4.663504in}{2.405030in}}%
\pgfpathlineto{\pgfqpoint{4.664350in}{1.896340in}}%
\pgfpathlineto{\pgfqpoint{4.665195in}{1.659426in}}%
\pgfpathlineto{\pgfqpoint{4.666041in}{2.484756in}}%
\pgfpathlineto{\pgfqpoint{4.666887in}{2.402534in}}%
\pgfpathlineto{\pgfqpoint{4.668578in}{2.228065in}}%
\pgfpathlineto{\pgfqpoint{4.669423in}{2.370718in}}%
\pgfpathlineto{\pgfqpoint{4.670269in}{1.594563in}}%
\pgfpathlineto{\pgfqpoint{4.671115in}{2.113778in}}%
\pgfpathlineto{\pgfqpoint{4.671960in}{1.733574in}}%
\pgfpathlineto{\pgfqpoint{4.673652in}{2.474614in}}%
\pgfpathlineto{\pgfqpoint{4.674497in}{2.261771in}}%
\pgfpathlineto{\pgfqpoint{4.675343in}{2.740134in}}%
\pgfpathlineto{\pgfqpoint{4.676188in}{1.905872in}}%
\pgfpathlineto{\pgfqpoint{4.677034in}{2.194413in}}%
\pgfpathlineto{\pgfqpoint{4.677880in}{2.061870in}}%
\pgfpathlineto{\pgfqpoint{4.678725in}{2.428189in}}%
\pgfpathlineto{\pgfqpoint{4.679571in}{2.050766in}}%
\pgfpathlineto{\pgfqpoint{4.680417in}{2.229134in}}%
\pgfpathlineto{\pgfqpoint{4.681262in}{2.667176in}}%
\pgfpathlineto{\pgfqpoint{4.682953in}{1.714186in}}%
\pgfpathlineto{\pgfqpoint{4.683799in}{2.278360in}}%
\pgfpathlineto{\pgfqpoint{4.684645in}{2.103528in}}%
\pgfpathlineto{\pgfqpoint{4.685490in}{1.701259in}}%
\pgfpathlineto{\pgfqpoint{4.686336in}{2.103123in}}%
\pgfpathlineto{\pgfqpoint{4.687182in}{2.091932in}}%
\pgfpathlineto{\pgfqpoint{4.688027in}{1.807520in}}%
\pgfpathlineto{\pgfqpoint{4.689718in}{2.477422in}}%
\pgfpathlineto{\pgfqpoint{4.691410in}{2.158821in}}%
\pgfpathlineto{\pgfqpoint{4.692255in}{2.347727in}}%
\pgfpathlineto{\pgfqpoint{4.693101in}{1.469743in}}%
\pgfpathlineto{\pgfqpoint{4.693947in}{1.869236in}}%
\pgfpathlineto{\pgfqpoint{4.694792in}{2.455933in}}%
\pgfpathlineto{\pgfqpoint{4.695638in}{1.694527in}}%
\pgfpathlineto{\pgfqpoint{4.697329in}{2.570011in}}%
\pgfpathlineto{\pgfqpoint{4.699020in}{1.817693in}}%
\pgfpathlineto{\pgfqpoint{4.699866in}{2.277388in}}%
\pgfpathlineto{\pgfqpoint{4.700712in}{1.884978in}}%
\pgfpathlineto{\pgfqpoint{4.702403in}{2.426600in}}%
\pgfpathlineto{\pgfqpoint{4.703248in}{1.519549in}}%
\pgfpathlineto{\pgfqpoint{4.704094in}{2.315491in}}%
\pgfpathlineto{\pgfqpoint{4.704940in}{2.138588in}}%
\pgfpathlineto{\pgfqpoint{4.705785in}{2.077553in}}%
\pgfpathlineto{\pgfqpoint{4.709168in}{2.997993in}}%
\pgfpathlineto{\pgfqpoint{4.710013in}{1.445162in}}%
\pgfpathlineto{\pgfqpoint{4.710859in}{2.240685in}}%
\pgfpathlineto{\pgfqpoint{4.711705in}{2.163373in}}%
\pgfpathlineto{\pgfqpoint{4.712550in}{2.560640in}}%
\pgfpathlineto{\pgfqpoint{4.713396in}{2.341675in}}%
\pgfpathlineto{\pgfqpoint{4.715087in}{1.827699in}}%
\pgfpathlineto{\pgfqpoint{4.715933in}{2.031352in}}%
\pgfpathlineto{\pgfqpoint{4.716778in}{2.161360in}}%
\pgfpathlineto{\pgfqpoint{4.717624in}{2.038610in}}%
\pgfpathlineto{\pgfqpoint{4.718470in}{1.442229in}}%
\pgfpathlineto{\pgfqpoint{4.719315in}{1.907975in}}%
\pgfpathlineto{\pgfqpoint{4.721852in}{2.811698in}}%
\pgfpathlineto{\pgfqpoint{4.722698in}{1.623757in}}%
\pgfpathlineto{\pgfqpoint{4.723543in}{1.898274in}}%
\pgfpathlineto{\pgfqpoint{4.724389in}{2.357318in}}%
\pgfpathlineto{\pgfqpoint{4.725235in}{1.637551in}}%
\pgfpathlineto{\pgfqpoint{4.726080in}{2.229846in}}%
\pgfpathlineto{\pgfqpoint{4.726926in}{2.143772in}}%
\pgfpathlineto{\pgfqpoint{4.727771in}{1.527308in}}%
\pgfpathlineto{\pgfqpoint{4.728617in}{2.478406in}}%
\pgfpathlineto{\pgfqpoint{4.729463in}{1.650806in}}%
\pgfpathlineto{\pgfqpoint{4.730308in}{2.916984in}}%
\pgfpathlineto{\pgfqpoint{4.731154in}{2.261700in}}%
\pgfpathlineto{\pgfqpoint{4.732000in}{1.792473in}}%
\pgfpathlineto{\pgfqpoint{4.732845in}{2.670367in}}%
\pgfpathlineto{\pgfqpoint{4.733691in}{1.922955in}}%
\pgfpathlineto{\pgfqpoint{4.734536in}{2.274951in}}%
\pgfpathlineto{\pgfqpoint{4.735382in}{2.072811in}}%
\pgfpathlineto{\pgfqpoint{4.736228in}{2.084478in}}%
\pgfpathlineto{\pgfqpoint{4.737073in}{2.063962in}}%
\pgfpathlineto{\pgfqpoint{4.737919in}{2.606392in}}%
\pgfpathlineto{\pgfqpoint{4.738765in}{2.244244in}}%
\pgfpathlineto{\pgfqpoint{4.739610in}{2.418616in}}%
\pgfpathlineto{\pgfqpoint{4.740456in}{1.981291in}}%
\pgfpathlineto{\pgfqpoint{4.741301in}{2.547668in}}%
\pgfpathlineto{\pgfqpoint{4.742147in}{2.115748in}}%
\pgfpathlineto{\pgfqpoint{4.742993in}{2.562158in}}%
\pgfpathlineto{\pgfqpoint{4.744684in}{1.881601in}}%
\pgfpathlineto{\pgfqpoint{4.745530in}{2.684919in}}%
\pgfpathlineto{\pgfqpoint{4.746375in}{2.113206in}}%
\pgfpathlineto{\pgfqpoint{4.747221in}{2.510363in}}%
\pgfpathlineto{\pgfqpoint{4.748066in}{1.680955in}}%
\pgfpathlineto{\pgfqpoint{4.748912in}{1.808335in}}%
\pgfpathlineto{\pgfqpoint{4.749758in}{1.989910in}}%
\pgfpathlineto{\pgfqpoint{4.750603in}{3.154451in}}%
\pgfpathlineto{\pgfqpoint{4.751449in}{1.864109in}}%
\pgfpathlineto{\pgfqpoint{4.752295in}{2.863564in}}%
\pgfpathlineto{\pgfqpoint{4.753140in}{2.250599in}}%
\pgfpathlineto{\pgfqpoint{4.753986in}{2.710461in}}%
\pgfpathlineto{\pgfqpoint{4.755677in}{2.008590in}}%
\pgfpathlineto{\pgfqpoint{4.756523in}{2.535651in}}%
\pgfpathlineto{\pgfqpoint{4.757368in}{2.209400in}}%
\pgfpathlineto{\pgfqpoint{4.758214in}{2.224960in}}%
\pgfpathlineto{\pgfqpoint{4.759060in}{2.564542in}}%
\pgfpathlineto{\pgfqpoint{4.759905in}{1.891769in}}%
\pgfpathlineto{\pgfqpoint{4.760751in}{1.897125in}}%
\pgfpathlineto{\pgfqpoint{4.761596in}{2.272571in}}%
\pgfpathlineto{\pgfqpoint{4.762442in}{2.218652in}}%
\pgfpathlineto{\pgfqpoint{4.763288in}{1.566921in}}%
\pgfpathlineto{\pgfqpoint{4.764133in}{2.481576in}}%
\pgfpathlineto{\pgfqpoint{4.764979in}{2.381097in}}%
\pgfpathlineto{\pgfqpoint{4.765825in}{2.354250in}}%
\pgfpathlineto{\pgfqpoint{4.766670in}{2.108504in}}%
\pgfpathlineto{\pgfqpoint{4.767516in}{2.399839in}}%
\pgfpathlineto{\pgfqpoint{4.768361in}{2.314977in}}%
\pgfpathlineto{\pgfqpoint{4.769207in}{1.723088in}}%
\pgfpathlineto{\pgfqpoint{4.770053in}{2.092485in}}%
\pgfpathlineto{\pgfqpoint{4.771744in}{1.791137in}}%
\pgfpathlineto{\pgfqpoint{4.772590in}{2.697795in}}%
\pgfpathlineto{\pgfqpoint{4.773435in}{2.435579in}}%
\pgfpathlineto{\pgfqpoint{4.774281in}{2.741097in}}%
\pgfpathlineto{\pgfqpoint{4.775972in}{1.842325in}}%
\pgfpathlineto{\pgfqpoint{4.776818in}{2.038100in}}%
\pgfpathlineto{\pgfqpoint{4.777663in}{1.869937in}}%
\pgfpathlineto{\pgfqpoint{4.778509in}{2.406595in}}%
\pgfpathlineto{\pgfqpoint{4.779355in}{1.799999in}}%
\pgfpathlineto{\pgfqpoint{4.780200in}{2.144308in}}%
\pgfpathlineto{\pgfqpoint{4.781891in}{2.506407in}}%
\pgfpathlineto{\pgfqpoint{4.783583in}{2.025942in}}%
\pgfpathlineto{\pgfqpoint{4.784428in}{2.141988in}}%
\pgfpathlineto{\pgfqpoint{4.786119in}{2.508658in}}%
\pgfpathlineto{\pgfqpoint{4.788656in}{1.931586in}}%
\pgfpathlineto{\pgfqpoint{4.790348in}{2.159890in}}%
\pgfpathlineto{\pgfqpoint{4.791193in}{2.083484in}}%
\pgfpathlineto{\pgfqpoint{4.792039in}{2.406942in}}%
\pgfpathlineto{\pgfqpoint{4.792884in}{1.821408in}}%
\pgfpathlineto{\pgfqpoint{4.793730in}{2.348043in}}%
\pgfpathlineto{\pgfqpoint{4.794576in}{2.135322in}}%
\pgfpathlineto{\pgfqpoint{4.795421in}{1.921059in}}%
\pgfpathlineto{\pgfqpoint{4.797113in}{2.473971in}}%
\pgfpathlineto{\pgfqpoint{4.799649in}{1.886558in}}%
\pgfpathlineto{\pgfqpoint{4.802186in}{2.332132in}}%
\pgfpathlineto{\pgfqpoint{4.803032in}{2.377298in}}%
\pgfpathlineto{\pgfqpoint{4.805569in}{1.743253in}}%
\pgfpathlineto{\pgfqpoint{4.806414in}{2.359369in}}%
\pgfpathlineto{\pgfqpoint{4.807260in}{1.943625in}}%
\pgfpathlineto{\pgfqpoint{4.808106in}{1.481293in}}%
\pgfpathlineto{\pgfqpoint{4.808951in}{2.275939in}}%
\pgfpathlineto{\pgfqpoint{4.809797in}{2.010702in}}%
\pgfpathlineto{\pgfqpoint{4.810643in}{2.001151in}}%
\pgfpathlineto{\pgfqpoint{4.812334in}{2.914720in}}%
\pgfpathlineto{\pgfqpoint{4.813179in}{2.140740in}}%
\pgfpathlineto{\pgfqpoint{4.814025in}{2.392523in}}%
\pgfpathlineto{\pgfqpoint{4.815716in}{2.955412in}}%
\pgfpathlineto{\pgfqpoint{4.817408in}{2.005804in}}%
\pgfpathlineto{\pgfqpoint{4.818253in}{2.664332in}}%
\pgfpathlineto{\pgfqpoint{4.819099in}{1.746030in}}%
\pgfpathlineto{\pgfqpoint{4.820790in}{3.042284in}}%
\pgfpathlineto{\pgfqpoint{4.822481in}{2.172340in}}%
\pgfpathlineto{\pgfqpoint{4.823327in}{2.915689in}}%
\pgfpathlineto{\pgfqpoint{4.824173in}{2.328497in}}%
\pgfpathlineto{\pgfqpoint{4.825018in}{2.639314in}}%
\pgfpathlineto{\pgfqpoint{4.827555in}{1.935171in}}%
\pgfpathlineto{\pgfqpoint{4.828401in}{2.382731in}}%
\pgfpathlineto{\pgfqpoint{4.829246in}{2.375266in}}%
\pgfpathlineto{\pgfqpoint{4.830092in}{2.392493in}}%
\pgfpathlineto{\pgfqpoint{4.832629in}{1.813565in}}%
\pgfpathlineto{\pgfqpoint{4.834320in}{2.266726in}}%
\pgfpathlineto{\pgfqpoint{4.835166in}{1.728146in}}%
\pgfpathlineto{\pgfqpoint{4.837703in}{3.382727in}}%
\pgfpathlineto{\pgfqpoint{4.840239in}{2.014612in}}%
\pgfpathlineto{\pgfqpoint{4.841085in}{1.289024in}}%
\pgfpathlineto{\pgfqpoint{4.843622in}{2.113215in}}%
\pgfpathlineto{\pgfqpoint{4.844468in}{2.420573in}}%
\pgfpathlineto{\pgfqpoint{4.845313in}{1.999955in}}%
\pgfpathlineto{\pgfqpoint{4.847004in}{2.717454in}}%
\pgfpathlineto{\pgfqpoint{4.848696in}{1.995734in}}%
\pgfpathlineto{\pgfqpoint{4.849541in}{2.647637in}}%
\pgfpathlineto{\pgfqpoint{4.850387in}{2.256959in}}%
\pgfpathlineto{\pgfqpoint{4.851233in}{2.650934in}}%
\pgfpathlineto{\pgfqpoint{4.852078in}{2.241089in}}%
\pgfpathlineto{\pgfqpoint{4.852924in}{2.327571in}}%
\pgfpathlineto{\pgfqpoint{4.853769in}{2.651492in}}%
\pgfpathlineto{\pgfqpoint{4.854615in}{2.389024in}}%
\pgfpathlineto{\pgfqpoint{4.856306in}{1.343118in}}%
\pgfpathlineto{\pgfqpoint{4.857998in}{2.392370in}}%
\pgfpathlineto{\pgfqpoint{4.858843in}{2.417115in}}%
\pgfpathlineto{\pgfqpoint{4.859689in}{2.899461in}}%
\pgfpathlineto{\pgfqpoint{4.861380in}{1.742472in}}%
\pgfpathlineto{\pgfqpoint{4.863917in}{2.198385in}}%
\pgfpathlineto{\pgfqpoint{4.864762in}{3.013147in}}%
\pgfpathlineto{\pgfqpoint{4.865608in}{1.927646in}}%
\pgfpathlineto{\pgfqpoint{4.866454in}{2.316958in}}%
\pgfpathlineto{\pgfqpoint{4.867299in}{1.988937in}}%
\pgfpathlineto{\pgfqpoint{4.868145in}{2.207854in}}%
\pgfpathlineto{\pgfqpoint{4.869836in}{1.894613in}}%
\pgfpathlineto{\pgfqpoint{4.870682in}{1.989100in}}%
\pgfpathlineto{\pgfqpoint{4.871527in}{1.771871in}}%
\pgfpathlineto{\pgfqpoint{4.873219in}{2.137024in}}%
\pgfpathlineto{\pgfqpoint{4.874064in}{2.109147in}}%
\pgfpathlineto{\pgfqpoint{4.874910in}{1.738269in}}%
\pgfpathlineto{\pgfqpoint{4.876601in}{2.337563in}}%
\pgfpathlineto{\pgfqpoint{4.877447in}{1.821872in}}%
\pgfpathlineto{\pgfqpoint{4.878292in}{2.959349in}}%
\pgfpathlineto{\pgfqpoint{4.879138in}{2.441358in}}%
\pgfpathlineto{\pgfqpoint{4.879984in}{2.629791in}}%
\pgfpathlineto{\pgfqpoint{4.881675in}{2.007274in}}%
\pgfpathlineto{\pgfqpoint{4.882521in}{2.510473in}}%
\pgfpathlineto{\pgfqpoint{4.884212in}{1.686353in}}%
\pgfpathlineto{\pgfqpoint{4.885057in}{2.521723in}}%
\pgfpathlineto{\pgfqpoint{4.885903in}{2.449857in}}%
\pgfpathlineto{\pgfqpoint{4.886749in}{2.371205in}}%
\pgfpathlineto{\pgfqpoint{4.889286in}{1.987320in}}%
\pgfpathlineto{\pgfqpoint{4.890131in}{2.771314in}}%
\pgfpathlineto{\pgfqpoint{4.892668in}{1.634870in}}%
\pgfpathlineto{\pgfqpoint{4.893514in}{2.402970in}}%
\pgfpathlineto{\pgfqpoint{4.894359in}{2.086178in}}%
\pgfpathlineto{\pgfqpoint{4.895205in}{2.331980in}}%
\pgfpathlineto{\pgfqpoint{4.896051in}{2.162481in}}%
\pgfpathlineto{\pgfqpoint{4.896896in}{2.224068in}}%
\pgfpathlineto{\pgfqpoint{4.897742in}{2.011277in}}%
\pgfpathlineto{\pgfqpoint{4.898587in}{2.459574in}}%
\pgfpathlineto{\pgfqpoint{4.900279in}{1.713936in}}%
\pgfpathlineto{\pgfqpoint{4.901124in}{1.759380in}}%
\pgfpathlineto{\pgfqpoint{4.901970in}{1.911202in}}%
\pgfpathlineto{\pgfqpoint{4.903661in}{1.727307in}}%
\pgfpathlineto{\pgfqpoint{4.907044in}{2.581465in}}%
\pgfpathlineto{\pgfqpoint{4.907889in}{1.923621in}}%
\pgfpathlineto{\pgfqpoint{4.908735in}{2.249155in}}%
\pgfpathlineto{\pgfqpoint{4.909581in}{2.272955in}}%
\pgfpathlineto{\pgfqpoint{4.911272in}{2.572541in}}%
\pgfpathlineto{\pgfqpoint{4.912117in}{2.950945in}}%
\pgfpathlineto{\pgfqpoint{4.912963in}{1.407420in}}%
\pgfpathlineto{\pgfqpoint{4.913809in}{1.744950in}}%
\pgfpathlineto{\pgfqpoint{4.916346in}{2.174118in}}%
\pgfpathlineto{\pgfqpoint{4.917191in}{1.423932in}}%
\pgfpathlineto{\pgfqpoint{4.918037in}{2.084158in}}%
\pgfpathlineto{\pgfqpoint{4.918882in}{1.802891in}}%
\pgfpathlineto{\pgfqpoint{4.921419in}{2.325424in}}%
\pgfpathlineto{\pgfqpoint{4.922265in}{2.442776in}}%
\pgfpathlineto{\pgfqpoint{4.923956in}{1.546997in}}%
\pgfpathlineto{\pgfqpoint{4.924802in}{2.515619in}}%
\pgfpathlineto{\pgfqpoint{4.925647in}{2.059005in}}%
\pgfpathlineto{\pgfqpoint{4.926493in}{1.819222in}}%
\pgfpathlineto{\pgfqpoint{4.927339in}{2.382928in}}%
\pgfpathlineto{\pgfqpoint{4.928184in}{2.053829in}}%
\pgfpathlineto{\pgfqpoint{4.929030in}{1.923624in}}%
\pgfpathlineto{\pgfqpoint{4.929876in}{3.349668in}}%
\pgfpathlineto{\pgfqpoint{4.930721in}{2.249950in}}%
\pgfpathlineto{\pgfqpoint{4.931567in}{2.446262in}}%
\pgfpathlineto{\pgfqpoint{4.932412in}{1.941141in}}%
\pgfpathlineto{\pgfqpoint{4.933258in}{2.545330in}}%
\pgfpathlineto{\pgfqpoint{4.934104in}{2.450511in}}%
\pgfpathlineto{\pgfqpoint{4.937486in}{1.913855in}}%
\pgfpathlineto{\pgfqpoint{4.940023in}{2.369337in}}%
\pgfpathlineto{\pgfqpoint{4.943405in}{1.699430in}}%
\pgfpathlineto{\pgfqpoint{4.944251in}{2.590969in}}%
\pgfpathlineto{\pgfqpoint{4.945097in}{2.014115in}}%
\pgfpathlineto{\pgfqpoint{4.945942in}{1.950048in}}%
\pgfpathlineto{\pgfqpoint{4.946788in}{1.966205in}}%
\pgfpathlineto{\pgfqpoint{4.948479in}{2.250308in}}%
\pgfpathlineto{\pgfqpoint{4.949325in}{1.875047in}}%
\pgfpathlineto{\pgfqpoint{4.951016in}{2.793825in}}%
\pgfpathlineto{\pgfqpoint{4.952707in}{2.136080in}}%
\pgfpathlineto{\pgfqpoint{4.953553in}{2.269163in}}%
\pgfpathlineto{\pgfqpoint{4.954399in}{1.653809in}}%
\pgfpathlineto{\pgfqpoint{4.955244in}{2.708707in}}%
\pgfpathlineto{\pgfqpoint{4.956090in}{1.740341in}}%
\pgfpathlineto{\pgfqpoint{4.956935in}{2.538978in}}%
\pgfpathlineto{\pgfqpoint{4.957781in}{1.791195in}}%
\pgfpathlineto{\pgfqpoint{4.960318in}{3.047166in}}%
\pgfpathlineto{\pgfqpoint{4.961164in}{2.007457in}}%
\pgfpathlineto{\pgfqpoint{4.962009in}{2.265991in}}%
\pgfpathlineto{\pgfqpoint{4.963700in}{2.924542in}}%
\pgfpathlineto{\pgfqpoint{4.965392in}{1.733126in}}%
\pgfpathlineto{\pgfqpoint{4.966237in}{2.014816in}}%
\pgfpathlineto{\pgfqpoint{4.967083in}{2.662402in}}%
\pgfpathlineto{\pgfqpoint{4.967929in}{2.550748in}}%
\pgfpathlineto{\pgfqpoint{4.968774in}{1.729964in}}%
\pgfpathlineto{\pgfqpoint{4.969620in}{2.000246in}}%
\pgfpathlineto{\pgfqpoint{4.970465in}{1.740305in}}%
\pgfpathlineto{\pgfqpoint{4.971311in}{1.851427in}}%
\pgfpathlineto{\pgfqpoint{4.973002in}{1.657070in}}%
\pgfpathlineto{\pgfqpoint{4.973848in}{2.423297in}}%
\pgfpathlineto{\pgfqpoint{4.974694in}{1.763310in}}%
\pgfpathlineto{\pgfqpoint{4.975539in}{2.190379in}}%
\pgfpathlineto{\pgfqpoint{4.976385in}{2.548827in}}%
\pgfpathlineto{\pgfqpoint{4.978076in}{2.136264in}}%
\pgfpathlineto{\pgfqpoint{4.979767in}{2.666956in}}%
\pgfpathlineto{\pgfqpoint{4.980613in}{2.372328in}}%
\pgfpathlineto{\pgfqpoint{4.981459in}{2.387356in}}%
\pgfpathlineto{\pgfqpoint{4.983150in}{1.682510in}}%
\pgfpathlineto{\pgfqpoint{4.984841in}{2.426000in}}%
\pgfpathlineto{\pgfqpoint{4.985687in}{2.248068in}}%
\pgfpathlineto{\pgfqpoint{4.986532in}{2.023499in}}%
\pgfpathlineto{\pgfqpoint{4.988224in}{2.785550in}}%
\pgfpathlineto{\pgfqpoint{4.991606in}{1.541745in}}%
\pgfpathlineto{\pgfqpoint{4.992452in}{3.109854in}}%
\pgfpathlineto{\pgfqpoint{4.993297in}{2.315550in}}%
\pgfpathlineto{\pgfqpoint{4.994989in}{1.677373in}}%
\pgfpathlineto{\pgfqpoint{4.997525in}{2.132278in}}%
\pgfpathlineto{\pgfqpoint{4.998371in}{2.076622in}}%
\pgfpathlineto{\pgfqpoint{5.000062in}{2.965875in}}%
\pgfpathlineto{\pgfqpoint{5.001754in}{1.995420in}}%
\pgfpathlineto{\pgfqpoint{5.002599in}{2.070584in}}%
\pgfpathlineto{\pgfqpoint{5.003445in}{2.563852in}}%
\pgfpathlineto{\pgfqpoint{5.004290in}{2.126508in}}%
\pgfpathlineto{\pgfqpoint{5.005136in}{2.249660in}}%
\pgfpathlineto{\pgfqpoint{5.005982in}{2.841706in}}%
\pgfpathlineto{\pgfqpoint{5.006827in}{2.124108in}}%
\pgfpathlineto{\pgfqpoint{5.007673in}{2.838012in}}%
\pgfpathlineto{\pgfqpoint{5.009364in}{1.915865in}}%
\pgfpathlineto{\pgfqpoint{5.010210in}{2.909932in}}%
\pgfpathlineto{\pgfqpoint{5.011055in}{1.997192in}}%
\pgfpathlineto{\pgfqpoint{5.011901in}{2.059988in}}%
\pgfpathlineto{\pgfqpoint{5.013592in}{2.644968in}}%
\pgfpathlineto{\pgfqpoint{5.014438in}{1.860890in}}%
\pgfpathlineto{\pgfqpoint{5.015284in}{2.050467in}}%
\pgfpathlineto{\pgfqpoint{5.016129in}{1.906097in}}%
\pgfpathlineto{\pgfqpoint{5.018666in}{2.663016in}}%
\pgfpathlineto{\pgfqpoint{5.019512in}{2.460184in}}%
\pgfpathlineto{\pgfqpoint{5.020357in}{2.179288in}}%
\pgfpathlineto{\pgfqpoint{5.021203in}{2.522436in}}%
\pgfpathlineto{\pgfqpoint{5.022049in}{2.255558in}}%
\pgfpathlineto{\pgfqpoint{5.023740in}{2.457606in}}%
\pgfpathlineto{\pgfqpoint{5.024585in}{1.887752in}}%
\pgfpathlineto{\pgfqpoint{5.025431in}{1.933306in}}%
\pgfpathlineto{\pgfqpoint{5.027122in}{2.572331in}}%
\pgfpathlineto{\pgfqpoint{5.028813in}{2.008789in}}%
\pgfpathlineto{\pgfqpoint{5.029659in}{2.724453in}}%
\pgfpathlineto{\pgfqpoint{5.030505in}{2.203824in}}%
\pgfpathlineto{\pgfqpoint{5.031350in}{2.249786in}}%
\pgfpathlineto{\pgfqpoint{5.032196in}{2.228097in}}%
\pgfpathlineto{\pgfqpoint{5.033042in}{2.675638in}}%
\pgfpathlineto{\pgfqpoint{5.033887in}{2.427190in}}%
\pgfpathlineto{\pgfqpoint{5.034733in}{1.907513in}}%
\pgfpathlineto{\pgfqpoint{5.035578in}{1.939135in}}%
\pgfpathlineto{\pgfqpoint{5.037270in}{2.530756in}}%
\pgfpathlineto{\pgfqpoint{5.038115in}{1.796027in}}%
\pgfpathlineto{\pgfqpoint{5.039807in}{3.167098in}}%
\pgfpathlineto{\pgfqpoint{5.040652in}{1.937568in}}%
\pgfpathlineto{\pgfqpoint{5.041498in}{2.392238in}}%
\pgfpathlineto{\pgfqpoint{5.042343in}{2.388153in}}%
\pgfpathlineto{\pgfqpoint{5.043189in}{2.203625in}}%
\pgfpathlineto{\pgfqpoint{5.044035in}{2.253247in}}%
\pgfpathlineto{\pgfqpoint{5.044880in}{2.252820in}}%
\pgfpathlineto{\pgfqpoint{5.045726in}{2.594435in}}%
\pgfpathlineto{\pgfqpoint{5.046572in}{2.371999in}}%
\pgfpathlineto{\pgfqpoint{5.048263in}{1.722963in}}%
\pgfpathlineto{\pgfqpoint{5.049954in}{2.581747in}}%
\pgfpathlineto{\pgfqpoint{5.050800in}{2.000780in}}%
\pgfpathlineto{\pgfqpoint{5.051645in}{2.385589in}}%
\pgfpathlineto{\pgfqpoint{5.052491in}{2.583058in}}%
\pgfpathlineto{\pgfqpoint{5.053337in}{1.780333in}}%
\pgfpathlineto{\pgfqpoint{5.054182in}{1.936099in}}%
\pgfpathlineto{\pgfqpoint{5.056719in}{2.194045in}}%
\pgfpathlineto{\pgfqpoint{5.057565in}{1.945733in}}%
\pgfpathlineto{\pgfqpoint{5.059256in}{2.916420in}}%
\pgfpathlineto{\pgfqpoint{5.060102in}{1.728681in}}%
\pgfpathlineto{\pgfqpoint{5.060947in}{2.410642in}}%
\pgfpathlineto{\pgfqpoint{5.061793in}{1.812472in}}%
\pgfpathlineto{\pgfqpoint{5.062638in}{2.845865in}}%
\pgfpathlineto{\pgfqpoint{5.064330in}{1.146101in}}%
\pgfpathlineto{\pgfqpoint{5.065175in}{2.460138in}}%
\pgfpathlineto{\pgfqpoint{5.066021in}{1.726296in}}%
\pgfpathlineto{\pgfqpoint{5.066867in}{2.438466in}}%
\pgfpathlineto{\pgfqpoint{5.067712in}{2.188963in}}%
\pgfpathlineto{\pgfqpoint{5.068558in}{1.859790in}}%
\pgfpathlineto{\pgfqpoint{5.069403in}{2.013680in}}%
\pgfpathlineto{\pgfqpoint{5.070249in}{2.338153in}}%
\pgfpathlineto{\pgfqpoint{5.071095in}{2.261515in}}%
\pgfpathlineto{\pgfqpoint{5.071940in}{2.171341in}}%
\pgfpathlineto{\pgfqpoint{5.073632in}{2.594627in}}%
\pgfpathlineto{\pgfqpoint{5.074477in}{2.529270in}}%
\pgfpathlineto{\pgfqpoint{5.075323in}{2.001497in}}%
\pgfpathlineto{\pgfqpoint{5.076168in}{2.388937in}}%
\pgfpathlineto{\pgfqpoint{5.078705in}{1.810005in}}%
\pgfpathlineto{\pgfqpoint{5.080397in}{2.763530in}}%
\pgfpathlineto{\pgfqpoint{5.081242in}{2.387761in}}%
\pgfpathlineto{\pgfqpoint{5.082088in}{3.174285in}}%
\pgfpathlineto{\pgfqpoint{5.083779in}{2.063099in}}%
\pgfpathlineto{\pgfqpoint{5.084625in}{2.592147in}}%
\pgfpathlineto{\pgfqpoint{5.085470in}{2.005807in}}%
\pgfpathlineto{\pgfqpoint{5.086316in}{2.120876in}}%
\pgfpathlineto{\pgfqpoint{5.087162in}{2.511572in}}%
\pgfpathlineto{\pgfqpoint{5.088007in}{1.938779in}}%
\pgfpathlineto{\pgfqpoint{5.088853in}{2.265840in}}%
\pgfpathlineto{\pgfqpoint{5.089698in}{3.178699in}}%
\pgfpathlineto{\pgfqpoint{5.090544in}{2.459201in}}%
\pgfpathlineto{\pgfqpoint{5.091390in}{2.646001in}}%
\pgfpathlineto{\pgfqpoint{5.093081in}{2.049486in}}%
\pgfpathlineto{\pgfqpoint{5.093927in}{2.779054in}}%
\pgfpathlineto{\pgfqpoint{5.095618in}{1.548424in}}%
\pgfpathlineto{\pgfqpoint{5.096463in}{1.750488in}}%
\pgfpathlineto{\pgfqpoint{5.099000in}{2.644259in}}%
\pgfpathlineto{\pgfqpoint{5.099846in}{2.642835in}}%
\pgfpathlineto{\pgfqpoint{5.101537in}{1.800703in}}%
\pgfpathlineto{\pgfqpoint{5.103228in}{2.840304in}}%
\pgfpathlineto{\pgfqpoint{5.104074in}{2.379280in}}%
\pgfpathlineto{\pgfqpoint{5.106611in}{1.953870in}}%
\pgfpathlineto{\pgfqpoint{5.107456in}{2.077251in}}%
\pgfpathlineto{\pgfqpoint{5.108302in}{2.656302in}}%
\pgfpathlineto{\pgfqpoint{5.109148in}{2.649615in}}%
\pgfpathlineto{\pgfqpoint{5.109993in}{2.583413in}}%
\pgfpathlineto{\pgfqpoint{5.110839in}{3.110465in}}%
\pgfpathlineto{\pgfqpoint{5.112530in}{1.803427in}}%
\pgfpathlineto{\pgfqpoint{5.113376in}{2.593901in}}%
\pgfpathlineto{\pgfqpoint{5.114221in}{1.800303in}}%
\pgfpathlineto{\pgfqpoint{5.115067in}{2.446441in}}%
\pgfpathlineto{\pgfqpoint{5.115913in}{2.523408in}}%
\pgfpathlineto{\pgfqpoint{5.116758in}{1.935602in}}%
\pgfpathlineto{\pgfqpoint{5.117604in}{1.940720in}}%
\pgfpathlineto{\pgfqpoint{5.121832in}{2.585940in}}%
\pgfpathlineto{\pgfqpoint{5.122678in}{2.185549in}}%
\pgfpathlineto{\pgfqpoint{5.123523in}{2.636361in}}%
\pgfpathlineto{\pgfqpoint{5.126060in}{1.747439in}}%
\pgfpathlineto{\pgfqpoint{5.126906in}{1.810081in}}%
\pgfpathlineto{\pgfqpoint{5.127751in}{1.730820in}}%
\pgfpathlineto{\pgfqpoint{5.128597in}{2.259784in}}%
\pgfpathlineto{\pgfqpoint{5.129443in}{2.007356in}}%
\pgfpathlineto{\pgfqpoint{5.130288in}{2.317981in}}%
\pgfpathlineto{\pgfqpoint{5.131134in}{1.805837in}}%
\pgfpathlineto{\pgfqpoint{5.131980in}{2.192378in}}%
\pgfpathlineto{\pgfqpoint{5.132825in}{1.987011in}}%
\pgfpathlineto{\pgfqpoint{5.133671in}{2.647569in}}%
\pgfpathlineto{\pgfqpoint{5.134516in}{2.338339in}}%
\pgfpathlineto{\pgfqpoint{5.135362in}{1.958636in}}%
\pgfpathlineto{\pgfqpoint{5.136208in}{2.379400in}}%
\pgfpathlineto{\pgfqpoint{5.137053in}{1.479770in}}%
\pgfpathlineto{\pgfqpoint{5.137899in}{2.136131in}}%
\pgfpathlineto{\pgfqpoint{5.138745in}{2.255633in}}%
\pgfpathlineto{\pgfqpoint{5.139590in}{2.184910in}}%
\pgfpathlineto{\pgfqpoint{5.142127in}{1.649963in}}%
\pgfpathlineto{\pgfqpoint{5.142973in}{2.520505in}}%
\pgfpathlineto{\pgfqpoint{5.143818in}{2.051537in}}%
\pgfpathlineto{\pgfqpoint{5.144664in}{2.227648in}}%
\pgfpathlineto{\pgfqpoint{5.145510in}{1.933398in}}%
\pgfpathlineto{\pgfqpoint{5.146355in}{2.856199in}}%
\pgfpathlineto{\pgfqpoint{5.147201in}{1.866496in}}%
\pgfpathlineto{\pgfqpoint{5.148046in}{2.364310in}}%
\pgfpathlineto{\pgfqpoint{5.148892in}{1.867099in}}%
\pgfpathlineto{\pgfqpoint{5.149738in}{2.395471in}}%
\pgfpathlineto{\pgfqpoint{5.152275in}{1.491378in}}%
\pgfpathlineto{\pgfqpoint{5.154811in}{2.451478in}}%
\pgfpathlineto{\pgfqpoint{5.156503in}{2.062207in}}%
\pgfpathlineto{\pgfqpoint{5.157348in}{2.694968in}}%
\pgfpathlineto{\pgfqpoint{5.158194in}{2.336939in}}%
\pgfpathlineto{\pgfqpoint{5.159040in}{2.657746in}}%
\pgfpathlineto{\pgfqpoint{5.159885in}{1.690747in}}%
\pgfpathlineto{\pgfqpoint{5.160731in}{2.722170in}}%
\pgfpathlineto{\pgfqpoint{5.161576in}{1.946218in}}%
\pgfpathlineto{\pgfqpoint{5.163268in}{2.266570in}}%
\pgfpathlineto{\pgfqpoint{5.164959in}{2.260185in}}%
\pgfpathlineto{\pgfqpoint{5.165805in}{2.142728in}}%
\pgfpathlineto{\pgfqpoint{5.166650in}{2.200899in}}%
\pgfpathlineto{\pgfqpoint{5.167496in}{2.517628in}}%
\pgfpathlineto{\pgfqpoint{5.168341in}{2.268778in}}%
\pgfpathlineto{\pgfqpoint{5.169187in}{1.997256in}}%
\pgfpathlineto{\pgfqpoint{5.170033in}{2.461007in}}%
\pgfpathlineto{\pgfqpoint{5.170878in}{1.549542in}}%
\pgfpathlineto{\pgfqpoint{5.171724in}{2.249926in}}%
\pgfpathlineto{\pgfqpoint{5.173415in}{1.402794in}}%
\pgfpathlineto{\pgfqpoint{5.175952in}{2.525631in}}%
\pgfpathlineto{\pgfqpoint{5.177643in}{1.804531in}}%
\pgfpathlineto{\pgfqpoint{5.178489in}{2.192240in}}%
\pgfpathlineto{\pgfqpoint{5.179335in}{1.423467in}}%
\pgfpathlineto{\pgfqpoint{5.180180in}{2.008651in}}%
\pgfpathlineto{\pgfqpoint{5.181026in}{2.123785in}}%
\pgfpathlineto{\pgfqpoint{5.181871in}{1.869778in}}%
\pgfpathlineto{\pgfqpoint{5.182717in}{2.003574in}}%
\pgfpathlineto{\pgfqpoint{5.184408in}{2.647892in}}%
\pgfpathlineto{\pgfqpoint{5.185254in}{2.328360in}}%
\pgfpathlineto{\pgfqpoint{5.186099in}{2.031914in}}%
\pgfpathlineto{\pgfqpoint{5.187791in}{2.517076in}}%
\pgfpathlineto{\pgfqpoint{5.188636in}{1.937737in}}%
\pgfpathlineto{\pgfqpoint{5.188636in}{1.937737in}}%
\pgfusepath{stroke}%
\end{pgfscope}%
\begin{pgfscope}%
\pgfsetrectcap%
\pgfsetmiterjoin%
\pgfsetlinewidth{0.803000pt}%
\definecolor{currentstroke}{rgb}{0.000000,0.000000,0.000000}%
\pgfsetstrokecolor{currentstroke}%
\pgfsetdash{}{0pt}%
\pgfpathmoveto{\pgfqpoint{0.750000in}{0.500000in}}%
\pgfpathlineto{\pgfqpoint{0.750000in}{3.520000in}}%
\pgfusepath{stroke}%
\end{pgfscope}%
\begin{pgfscope}%
\pgfsetrectcap%
\pgfsetmiterjoin%
\pgfsetlinewidth{0.803000pt}%
\definecolor{currentstroke}{rgb}{0.000000,0.000000,0.000000}%
\pgfsetstrokecolor{currentstroke}%
\pgfsetdash{}{0pt}%
\pgfpathmoveto{\pgfqpoint{5.400000in}{0.500000in}}%
\pgfpathlineto{\pgfqpoint{5.400000in}{3.520000in}}%
\pgfusepath{stroke}%
\end{pgfscope}%
\begin{pgfscope}%
\pgfsetrectcap%
\pgfsetmiterjoin%
\pgfsetlinewidth{0.803000pt}%
\definecolor{currentstroke}{rgb}{0.000000,0.000000,0.000000}%
\pgfsetstrokecolor{currentstroke}%
\pgfsetdash{}{0pt}%
\pgfpathmoveto{\pgfqpoint{0.750000in}{0.500000in}}%
\pgfpathlineto{\pgfqpoint{5.400000in}{0.500000in}}%
\pgfusepath{stroke}%
\end{pgfscope}%
\begin{pgfscope}%
\pgfsetrectcap%
\pgfsetmiterjoin%
\pgfsetlinewidth{0.803000pt}%
\definecolor{currentstroke}{rgb}{0.000000,0.000000,0.000000}%
\pgfsetstrokecolor{currentstroke}%
\pgfsetdash{}{0pt}%
\pgfpathmoveto{\pgfqpoint{0.750000in}{3.520000in}}%
\pgfpathlineto{\pgfqpoint{5.400000in}{3.520000in}}%
\pgfusepath{stroke}%
\end{pgfscope}%
\begin{pgfscope}%
\pgfsetbuttcap%
\pgfsetmiterjoin%
\definecolor{currentfill}{rgb}{1.000000,1.000000,1.000000}%
\pgfsetfillcolor{currentfill}%
\pgfsetfillopacity{0.800000}%
\pgfsetlinewidth{1.003750pt}%
\definecolor{currentstroke}{rgb}{0.800000,0.800000,0.800000}%
\pgfsetstrokecolor{currentstroke}%
\pgfsetstrokeopacity{0.800000}%
\pgfsetdash{}{0pt}%
\pgfpathmoveto{\pgfqpoint{0.847222in}{3.001174in}}%
\pgfpathlineto{\pgfqpoint{1.587619in}{3.001174in}}%
\pgfpathquadraticcurveto{\pgfqpoint{1.615397in}{3.001174in}}{\pgfqpoint{1.615397in}{3.028952in}}%
\pgfpathlineto{\pgfqpoint{1.615397in}{3.422778in}}%
\pgfpathquadraticcurveto{\pgfqpoint{1.615397in}{3.450556in}}{\pgfqpoint{1.587619in}{3.450556in}}%
\pgfpathlineto{\pgfqpoint{0.847222in}{3.450556in}}%
\pgfpathquadraticcurveto{\pgfqpoint{0.819444in}{3.450556in}}{\pgfqpoint{0.819444in}{3.422778in}}%
\pgfpathlineto{\pgfqpoint{0.819444in}{3.028952in}}%
\pgfpathquadraticcurveto{\pgfqpoint{0.819444in}{3.001174in}}{\pgfqpoint{0.847222in}{3.001174in}}%
\pgfpathlineto{\pgfqpoint{0.847222in}{3.001174in}}%
\pgfpathclose%
\pgfusepath{stroke,fill}%
\end{pgfscope}%
\begin{pgfscope}%
\pgfsetrectcap%
\pgfsetroundjoin%
\pgfsetlinewidth{1.505625pt}%
\definecolor{currentstroke}{rgb}{1.000000,0.000000,0.000000}%
\pgfsetstrokecolor{currentstroke}%
\pgfsetdash{}{0pt}%
\pgfpathmoveto{\pgfqpoint{0.875000in}{3.338088in}}%
\pgfpathlineto{\pgfqpoint{1.013889in}{3.338088in}}%
\pgfpathlineto{\pgfqpoint{1.152778in}{3.338088in}}%
\pgfusepath{stroke}%
\end{pgfscope}%
\begin{pgfscope}%
\definecolor{textcolor}{rgb}{0.000000,0.000000,0.000000}%
\pgfsetstrokecolor{textcolor}%
\pgfsetfillcolor{textcolor}%
\pgftext[x=1.263889in,y=3.289477in,left,base]{\color{textcolor}\sffamily\fontsize{10.000000}{12.000000}\selectfont SNN}%
\end{pgfscope}%
\begin{pgfscope}%
\pgfsetrectcap%
\pgfsetroundjoin%
\pgfsetlinewidth{1.505625pt}%
\definecolor{currentstroke}{rgb}{0.000000,0.500000,0.000000}%
\pgfsetstrokecolor{currentstroke}%
\pgfsetdash{}{0pt}%
\pgfpathmoveto{\pgfqpoint{0.875000in}{3.134231in}}%
\pgfpathlineto{\pgfqpoint{1.013889in}{3.134231in}}%
\pgfpathlineto{\pgfqpoint{1.152778in}{3.134231in}}%
\pgfusepath{stroke}%
\end{pgfscope}%
\begin{pgfscope}%
\definecolor{textcolor}{rgb}{0.000000,0.000000,0.000000}%
\pgfsetstrokecolor{textcolor}%
\pgfsetfillcolor{textcolor}%
\pgftext[x=1.263889in,y=3.085620in,left,base]{\color{textcolor}\sffamily\fontsize{10.000000}{12.000000}\selectfont NN}%
\end{pgfscope}%
\end{pgfpicture}%
\makeatother%
\endgroup%

    \caption{Caption}
    \label{fig:my_label}
\end{figure}

\begin{figure}
%% Creator: Matplotlib, PGF backend
%%
%% To include the figure in your LaTeX document, write
%%   \input{<filename>.pgf}
%%
%% Make sure the required packages are loaded in your preamble
%%   \usepackage{pgf}
%%
%% Also ensure that all the required font packages are loaded; for instance,
%% the lmodern package is sometimes necessary when using math font.
%%   \usepackage{lmodern}
%%
%% Figures using additional raster images can only be included by \input if
%% they are in the same directory as the main LaTeX file. For loading figures
%% from other directories you can use the `import` package
%%   \usepackage{import}
%%
%% and then include the figures with
%%   \import{<path to file>}{<filename>.pgf}
%%
%% Matplotlib used the following preamble
%%   \usepackage{fontspec}
%%   \setmainfont{DejaVuSerif.ttf}[Path=\detokenize{C:/I/python38/Lib/site-packages/matplotlib/mpl-data/fonts/ttf/}]
%%   \setsansfont{DejaVuSans.ttf}[Path=\detokenize{C:/I/python38/Lib/site-packages/matplotlib/mpl-data/fonts/ttf/}]
%%   \setmonofont{DejaVuSansMono.ttf}[Path=\detokenize{C:/I/python38/Lib/site-packages/matplotlib/mpl-data/fonts/ttf/}]
%%
\begingroup%
\makeatletter%
\begin{pgfpicture}%
\pgfpathrectangle{\pgfpointorigin}{\pgfqpoint{6.000000in}{4.000000in}}%
\pgfusepath{use as bounding box, clip}%
\begin{pgfscope}%
\pgfsetbuttcap%
\pgfsetmiterjoin%
\pgfsetlinewidth{0.000000pt}%
\definecolor{currentstroke}{rgb}{1.000000,1.000000,1.000000}%
\pgfsetstrokecolor{currentstroke}%
\pgfsetstrokeopacity{0.000000}%
\pgfsetdash{}{0pt}%
\pgfpathmoveto{\pgfqpoint{0.000000in}{0.000000in}}%
\pgfpathlineto{\pgfqpoint{6.000000in}{0.000000in}}%
\pgfpathlineto{\pgfqpoint{6.000000in}{4.000000in}}%
\pgfpathlineto{\pgfqpoint{0.000000in}{4.000000in}}%
\pgfpathlineto{\pgfqpoint{0.000000in}{0.000000in}}%
\pgfpathclose%
\pgfusepath{}%
\end{pgfscope}%
\begin{pgfscope}%
\pgfsetbuttcap%
\pgfsetmiterjoin%
\definecolor{currentfill}{rgb}{1.000000,1.000000,1.000000}%
\pgfsetfillcolor{currentfill}%
\pgfsetlinewidth{0.000000pt}%
\definecolor{currentstroke}{rgb}{0.000000,0.000000,0.000000}%
\pgfsetstrokecolor{currentstroke}%
\pgfsetstrokeopacity{0.000000}%
\pgfsetdash{}{0pt}%
\pgfpathmoveto{\pgfqpoint{0.750000in}{0.500000in}}%
\pgfpathlineto{\pgfqpoint{5.400000in}{0.500000in}}%
\pgfpathlineto{\pgfqpoint{5.400000in}{3.520000in}}%
\pgfpathlineto{\pgfqpoint{0.750000in}{3.520000in}}%
\pgfpathlineto{\pgfqpoint{0.750000in}{0.500000in}}%
\pgfpathclose%
\pgfusepath{fill}%
\end{pgfscope}%
\begin{pgfscope}%
\pgfpathrectangle{\pgfqpoint{0.750000in}{0.500000in}}{\pgfqpoint{4.650000in}{3.020000in}}%
\pgfusepath{clip}%
\pgfsetbuttcap%
\pgfsetmiterjoin%
\definecolor{currentfill}{rgb}{1.000000,0.000000,0.000000}%
\pgfsetfillcolor{currentfill}%
\pgfsetlinewidth{0.000000pt}%
\definecolor{currentstroke}{rgb}{0.000000,0.000000,0.000000}%
\pgfsetstrokecolor{currentstroke}%
\pgfsetstrokeopacity{0.000000}%
\pgfsetdash{}{0pt}%
\pgfpathmoveto{\pgfqpoint{0.961364in}{0.500000in}}%
\pgfpathlineto{\pgfqpoint{0.994389in}{0.500000in}}%
\pgfpathlineto{\pgfqpoint{0.994389in}{0.510459in}}%
\pgfpathlineto{\pgfqpoint{0.961364in}{0.510459in}}%
\pgfpathlineto{\pgfqpoint{0.961364in}{0.500000in}}%
\pgfpathclose%
\pgfusepath{fill}%
\end{pgfscope}%
\begin{pgfscope}%
\pgfpathrectangle{\pgfqpoint{0.750000in}{0.500000in}}{\pgfqpoint{4.650000in}{3.020000in}}%
\pgfusepath{clip}%
\pgfsetbuttcap%
\pgfsetmiterjoin%
\definecolor{currentfill}{rgb}{1.000000,0.000000,0.000000}%
\pgfsetfillcolor{currentfill}%
\pgfsetlinewidth{0.000000pt}%
\definecolor{currentstroke}{rgb}{0.000000,0.000000,0.000000}%
\pgfsetstrokecolor{currentstroke}%
\pgfsetstrokeopacity{0.000000}%
\pgfsetdash{}{0pt}%
\pgfpathmoveto{\pgfqpoint{0.994389in}{0.500000in}}%
\pgfpathlineto{\pgfqpoint{1.027415in}{0.500000in}}%
\pgfpathlineto{\pgfqpoint{1.027415in}{0.500000in}}%
\pgfpathlineto{\pgfqpoint{0.994389in}{0.500000in}}%
\pgfpathlineto{\pgfqpoint{0.994389in}{0.500000in}}%
\pgfpathclose%
\pgfusepath{fill}%
\end{pgfscope}%
\begin{pgfscope}%
\pgfpathrectangle{\pgfqpoint{0.750000in}{0.500000in}}{\pgfqpoint{4.650000in}{3.020000in}}%
\pgfusepath{clip}%
\pgfsetbuttcap%
\pgfsetmiterjoin%
\definecolor{currentfill}{rgb}{1.000000,0.000000,0.000000}%
\pgfsetfillcolor{currentfill}%
\pgfsetlinewidth{0.000000pt}%
\definecolor{currentstroke}{rgb}{0.000000,0.000000,0.000000}%
\pgfsetstrokecolor{currentstroke}%
\pgfsetstrokeopacity{0.000000}%
\pgfsetdash{}{0pt}%
\pgfpathmoveto{\pgfqpoint{1.027415in}{0.500000in}}%
\pgfpathlineto{\pgfqpoint{1.060440in}{0.500000in}}%
\pgfpathlineto{\pgfqpoint{1.060440in}{0.500000in}}%
\pgfpathlineto{\pgfqpoint{1.027415in}{0.500000in}}%
\pgfpathlineto{\pgfqpoint{1.027415in}{0.500000in}}%
\pgfpathclose%
\pgfusepath{fill}%
\end{pgfscope}%
\begin{pgfscope}%
\pgfpathrectangle{\pgfqpoint{0.750000in}{0.500000in}}{\pgfqpoint{4.650000in}{3.020000in}}%
\pgfusepath{clip}%
\pgfsetbuttcap%
\pgfsetmiterjoin%
\definecolor{currentfill}{rgb}{1.000000,0.000000,0.000000}%
\pgfsetfillcolor{currentfill}%
\pgfsetlinewidth{0.000000pt}%
\definecolor{currentstroke}{rgb}{0.000000,0.000000,0.000000}%
\pgfsetstrokecolor{currentstroke}%
\pgfsetstrokeopacity{0.000000}%
\pgfsetdash{}{0pt}%
\pgfpathmoveto{\pgfqpoint{1.060440in}{0.500000in}}%
\pgfpathlineto{\pgfqpoint{1.093466in}{0.500000in}}%
\pgfpathlineto{\pgfqpoint{1.093466in}{0.500000in}}%
\pgfpathlineto{\pgfqpoint{1.060440in}{0.500000in}}%
\pgfpathlineto{\pgfqpoint{1.060440in}{0.500000in}}%
\pgfpathclose%
\pgfusepath{fill}%
\end{pgfscope}%
\begin{pgfscope}%
\pgfpathrectangle{\pgfqpoint{0.750000in}{0.500000in}}{\pgfqpoint{4.650000in}{3.020000in}}%
\pgfusepath{clip}%
\pgfsetbuttcap%
\pgfsetmiterjoin%
\definecolor{currentfill}{rgb}{1.000000,0.000000,0.000000}%
\pgfsetfillcolor{currentfill}%
\pgfsetlinewidth{0.000000pt}%
\definecolor{currentstroke}{rgb}{0.000000,0.000000,0.000000}%
\pgfsetstrokecolor{currentstroke}%
\pgfsetstrokeopacity{0.000000}%
\pgfsetdash{}{0pt}%
\pgfpathmoveto{\pgfqpoint{1.093466in}{0.500000in}}%
\pgfpathlineto{\pgfqpoint{1.126491in}{0.500000in}}%
\pgfpathlineto{\pgfqpoint{1.126491in}{0.500000in}}%
\pgfpathlineto{\pgfqpoint{1.093466in}{0.500000in}}%
\pgfpathlineto{\pgfqpoint{1.093466in}{0.500000in}}%
\pgfpathclose%
\pgfusepath{fill}%
\end{pgfscope}%
\begin{pgfscope}%
\pgfpathrectangle{\pgfqpoint{0.750000in}{0.500000in}}{\pgfqpoint{4.650000in}{3.020000in}}%
\pgfusepath{clip}%
\pgfsetbuttcap%
\pgfsetmiterjoin%
\definecolor{currentfill}{rgb}{1.000000,0.000000,0.000000}%
\pgfsetfillcolor{currentfill}%
\pgfsetlinewidth{0.000000pt}%
\definecolor{currentstroke}{rgb}{0.000000,0.000000,0.000000}%
\pgfsetstrokecolor{currentstroke}%
\pgfsetstrokeopacity{0.000000}%
\pgfsetdash{}{0pt}%
\pgfpathmoveto{\pgfqpoint{1.126491in}{0.500000in}}%
\pgfpathlineto{\pgfqpoint{1.159517in}{0.500000in}}%
\pgfpathlineto{\pgfqpoint{1.159517in}{0.500000in}}%
\pgfpathlineto{\pgfqpoint{1.126491in}{0.500000in}}%
\pgfpathlineto{\pgfqpoint{1.126491in}{0.500000in}}%
\pgfpathclose%
\pgfusepath{fill}%
\end{pgfscope}%
\begin{pgfscope}%
\pgfpathrectangle{\pgfqpoint{0.750000in}{0.500000in}}{\pgfqpoint{4.650000in}{3.020000in}}%
\pgfusepath{clip}%
\pgfsetbuttcap%
\pgfsetmiterjoin%
\definecolor{currentfill}{rgb}{1.000000,0.000000,0.000000}%
\pgfsetfillcolor{currentfill}%
\pgfsetlinewidth{0.000000pt}%
\definecolor{currentstroke}{rgb}{0.000000,0.000000,0.000000}%
\pgfsetstrokecolor{currentstroke}%
\pgfsetstrokeopacity{0.000000}%
\pgfsetdash{}{0pt}%
\pgfpathmoveto{\pgfqpoint{1.159517in}{0.500000in}}%
\pgfpathlineto{\pgfqpoint{1.192543in}{0.500000in}}%
\pgfpathlineto{\pgfqpoint{1.192543in}{0.500000in}}%
\pgfpathlineto{\pgfqpoint{1.159517in}{0.500000in}}%
\pgfpathlineto{\pgfqpoint{1.159517in}{0.500000in}}%
\pgfpathclose%
\pgfusepath{fill}%
\end{pgfscope}%
\begin{pgfscope}%
\pgfpathrectangle{\pgfqpoint{0.750000in}{0.500000in}}{\pgfqpoint{4.650000in}{3.020000in}}%
\pgfusepath{clip}%
\pgfsetbuttcap%
\pgfsetmiterjoin%
\definecolor{currentfill}{rgb}{1.000000,0.000000,0.000000}%
\pgfsetfillcolor{currentfill}%
\pgfsetlinewidth{0.000000pt}%
\definecolor{currentstroke}{rgb}{0.000000,0.000000,0.000000}%
\pgfsetstrokecolor{currentstroke}%
\pgfsetstrokeopacity{0.000000}%
\pgfsetdash{}{0pt}%
\pgfpathmoveto{\pgfqpoint{1.192543in}{0.500000in}}%
\pgfpathlineto{\pgfqpoint{1.225568in}{0.500000in}}%
\pgfpathlineto{\pgfqpoint{1.225568in}{0.510459in}}%
\pgfpathlineto{\pgfqpoint{1.192543in}{0.510459in}}%
\pgfpathlineto{\pgfqpoint{1.192543in}{0.500000in}}%
\pgfpathclose%
\pgfusepath{fill}%
\end{pgfscope}%
\begin{pgfscope}%
\pgfpathrectangle{\pgfqpoint{0.750000in}{0.500000in}}{\pgfqpoint{4.650000in}{3.020000in}}%
\pgfusepath{clip}%
\pgfsetbuttcap%
\pgfsetmiterjoin%
\definecolor{currentfill}{rgb}{1.000000,0.000000,0.000000}%
\pgfsetfillcolor{currentfill}%
\pgfsetlinewidth{0.000000pt}%
\definecolor{currentstroke}{rgb}{0.000000,0.000000,0.000000}%
\pgfsetstrokecolor{currentstroke}%
\pgfsetstrokeopacity{0.000000}%
\pgfsetdash{}{0pt}%
\pgfpathmoveto{\pgfqpoint{1.225568in}{0.500000in}}%
\pgfpathlineto{\pgfqpoint{1.258594in}{0.500000in}}%
\pgfpathlineto{\pgfqpoint{1.258594in}{0.500000in}}%
\pgfpathlineto{\pgfqpoint{1.225568in}{0.500000in}}%
\pgfpathlineto{\pgfqpoint{1.225568in}{0.500000in}}%
\pgfpathclose%
\pgfusepath{fill}%
\end{pgfscope}%
\begin{pgfscope}%
\pgfpathrectangle{\pgfqpoint{0.750000in}{0.500000in}}{\pgfqpoint{4.650000in}{3.020000in}}%
\pgfusepath{clip}%
\pgfsetbuttcap%
\pgfsetmiterjoin%
\definecolor{currentfill}{rgb}{1.000000,0.000000,0.000000}%
\pgfsetfillcolor{currentfill}%
\pgfsetlinewidth{0.000000pt}%
\definecolor{currentstroke}{rgb}{0.000000,0.000000,0.000000}%
\pgfsetstrokecolor{currentstroke}%
\pgfsetstrokeopacity{0.000000}%
\pgfsetdash{}{0pt}%
\pgfpathmoveto{\pgfqpoint{1.258594in}{0.500000in}}%
\pgfpathlineto{\pgfqpoint{1.291619in}{0.500000in}}%
\pgfpathlineto{\pgfqpoint{1.291619in}{0.510459in}}%
\pgfpathlineto{\pgfqpoint{1.258594in}{0.510459in}}%
\pgfpathlineto{\pgfqpoint{1.258594in}{0.500000in}}%
\pgfpathclose%
\pgfusepath{fill}%
\end{pgfscope}%
\begin{pgfscope}%
\pgfpathrectangle{\pgfqpoint{0.750000in}{0.500000in}}{\pgfqpoint{4.650000in}{3.020000in}}%
\pgfusepath{clip}%
\pgfsetbuttcap%
\pgfsetmiterjoin%
\definecolor{currentfill}{rgb}{1.000000,0.000000,0.000000}%
\pgfsetfillcolor{currentfill}%
\pgfsetlinewidth{0.000000pt}%
\definecolor{currentstroke}{rgb}{0.000000,0.000000,0.000000}%
\pgfsetstrokecolor{currentstroke}%
\pgfsetstrokeopacity{0.000000}%
\pgfsetdash{}{0pt}%
\pgfpathmoveto{\pgfqpoint{1.291619in}{0.500000in}}%
\pgfpathlineto{\pgfqpoint{1.324645in}{0.500000in}}%
\pgfpathlineto{\pgfqpoint{1.324645in}{0.500000in}}%
\pgfpathlineto{\pgfqpoint{1.291619in}{0.500000in}}%
\pgfpathlineto{\pgfqpoint{1.291619in}{0.500000in}}%
\pgfpathclose%
\pgfusepath{fill}%
\end{pgfscope}%
\begin{pgfscope}%
\pgfpathrectangle{\pgfqpoint{0.750000in}{0.500000in}}{\pgfqpoint{4.650000in}{3.020000in}}%
\pgfusepath{clip}%
\pgfsetbuttcap%
\pgfsetmiterjoin%
\definecolor{currentfill}{rgb}{1.000000,0.000000,0.000000}%
\pgfsetfillcolor{currentfill}%
\pgfsetlinewidth{0.000000pt}%
\definecolor{currentstroke}{rgb}{0.000000,0.000000,0.000000}%
\pgfsetstrokecolor{currentstroke}%
\pgfsetstrokeopacity{0.000000}%
\pgfsetdash{}{0pt}%
\pgfpathmoveto{\pgfqpoint{1.324645in}{0.500000in}}%
\pgfpathlineto{\pgfqpoint{1.357670in}{0.500000in}}%
\pgfpathlineto{\pgfqpoint{1.357670in}{0.510459in}}%
\pgfpathlineto{\pgfqpoint{1.324645in}{0.510459in}}%
\pgfpathlineto{\pgfqpoint{1.324645in}{0.500000in}}%
\pgfpathclose%
\pgfusepath{fill}%
\end{pgfscope}%
\begin{pgfscope}%
\pgfpathrectangle{\pgfqpoint{0.750000in}{0.500000in}}{\pgfqpoint{4.650000in}{3.020000in}}%
\pgfusepath{clip}%
\pgfsetbuttcap%
\pgfsetmiterjoin%
\definecolor{currentfill}{rgb}{1.000000,0.000000,0.000000}%
\pgfsetfillcolor{currentfill}%
\pgfsetlinewidth{0.000000pt}%
\definecolor{currentstroke}{rgb}{0.000000,0.000000,0.000000}%
\pgfsetstrokecolor{currentstroke}%
\pgfsetstrokeopacity{0.000000}%
\pgfsetdash{}{0pt}%
\pgfpathmoveto{\pgfqpoint{1.357670in}{0.500000in}}%
\pgfpathlineto{\pgfqpoint{1.390696in}{0.500000in}}%
\pgfpathlineto{\pgfqpoint{1.390696in}{0.531377in}}%
\pgfpathlineto{\pgfqpoint{1.357670in}{0.531377in}}%
\pgfpathlineto{\pgfqpoint{1.357670in}{0.500000in}}%
\pgfpathclose%
\pgfusepath{fill}%
\end{pgfscope}%
\begin{pgfscope}%
\pgfpathrectangle{\pgfqpoint{0.750000in}{0.500000in}}{\pgfqpoint{4.650000in}{3.020000in}}%
\pgfusepath{clip}%
\pgfsetbuttcap%
\pgfsetmiterjoin%
\definecolor{currentfill}{rgb}{1.000000,0.000000,0.000000}%
\pgfsetfillcolor{currentfill}%
\pgfsetlinewidth{0.000000pt}%
\definecolor{currentstroke}{rgb}{0.000000,0.000000,0.000000}%
\pgfsetstrokecolor{currentstroke}%
\pgfsetstrokeopacity{0.000000}%
\pgfsetdash{}{0pt}%
\pgfpathmoveto{\pgfqpoint{1.390696in}{0.500000in}}%
\pgfpathlineto{\pgfqpoint{1.423722in}{0.500000in}}%
\pgfpathlineto{\pgfqpoint{1.423722in}{0.500000in}}%
\pgfpathlineto{\pgfqpoint{1.390696in}{0.500000in}}%
\pgfpathlineto{\pgfqpoint{1.390696in}{0.500000in}}%
\pgfpathclose%
\pgfusepath{fill}%
\end{pgfscope}%
\begin{pgfscope}%
\pgfpathrectangle{\pgfqpoint{0.750000in}{0.500000in}}{\pgfqpoint{4.650000in}{3.020000in}}%
\pgfusepath{clip}%
\pgfsetbuttcap%
\pgfsetmiterjoin%
\definecolor{currentfill}{rgb}{1.000000,0.000000,0.000000}%
\pgfsetfillcolor{currentfill}%
\pgfsetlinewidth{0.000000pt}%
\definecolor{currentstroke}{rgb}{0.000000,0.000000,0.000000}%
\pgfsetstrokecolor{currentstroke}%
\pgfsetstrokeopacity{0.000000}%
\pgfsetdash{}{0pt}%
\pgfpathmoveto{\pgfqpoint{1.423722in}{0.500000in}}%
\pgfpathlineto{\pgfqpoint{1.456747in}{0.500000in}}%
\pgfpathlineto{\pgfqpoint{1.456747in}{0.510459in}}%
\pgfpathlineto{\pgfqpoint{1.423722in}{0.510459in}}%
\pgfpathlineto{\pgfqpoint{1.423722in}{0.500000in}}%
\pgfpathclose%
\pgfusepath{fill}%
\end{pgfscope}%
\begin{pgfscope}%
\pgfpathrectangle{\pgfqpoint{0.750000in}{0.500000in}}{\pgfqpoint{4.650000in}{3.020000in}}%
\pgfusepath{clip}%
\pgfsetbuttcap%
\pgfsetmiterjoin%
\definecolor{currentfill}{rgb}{1.000000,0.000000,0.000000}%
\pgfsetfillcolor{currentfill}%
\pgfsetlinewidth{0.000000pt}%
\definecolor{currentstroke}{rgb}{0.000000,0.000000,0.000000}%
\pgfsetstrokecolor{currentstroke}%
\pgfsetstrokeopacity{0.000000}%
\pgfsetdash{}{0pt}%
\pgfpathmoveto{\pgfqpoint{1.456747in}{0.500000in}}%
\pgfpathlineto{\pgfqpoint{1.489773in}{0.500000in}}%
\pgfpathlineto{\pgfqpoint{1.489773in}{0.500000in}}%
\pgfpathlineto{\pgfqpoint{1.456747in}{0.500000in}}%
\pgfpathlineto{\pgfqpoint{1.456747in}{0.500000in}}%
\pgfpathclose%
\pgfusepath{fill}%
\end{pgfscope}%
\begin{pgfscope}%
\pgfpathrectangle{\pgfqpoint{0.750000in}{0.500000in}}{\pgfqpoint{4.650000in}{3.020000in}}%
\pgfusepath{clip}%
\pgfsetbuttcap%
\pgfsetmiterjoin%
\definecolor{currentfill}{rgb}{1.000000,0.000000,0.000000}%
\pgfsetfillcolor{currentfill}%
\pgfsetlinewidth{0.000000pt}%
\definecolor{currentstroke}{rgb}{0.000000,0.000000,0.000000}%
\pgfsetstrokecolor{currentstroke}%
\pgfsetstrokeopacity{0.000000}%
\pgfsetdash{}{0pt}%
\pgfpathmoveto{\pgfqpoint{1.489773in}{0.500000in}}%
\pgfpathlineto{\pgfqpoint{1.522798in}{0.500000in}}%
\pgfpathlineto{\pgfqpoint{1.522798in}{0.510459in}}%
\pgfpathlineto{\pgfqpoint{1.489773in}{0.510459in}}%
\pgfpathlineto{\pgfqpoint{1.489773in}{0.500000in}}%
\pgfpathclose%
\pgfusepath{fill}%
\end{pgfscope}%
\begin{pgfscope}%
\pgfpathrectangle{\pgfqpoint{0.750000in}{0.500000in}}{\pgfqpoint{4.650000in}{3.020000in}}%
\pgfusepath{clip}%
\pgfsetbuttcap%
\pgfsetmiterjoin%
\definecolor{currentfill}{rgb}{1.000000,0.000000,0.000000}%
\pgfsetfillcolor{currentfill}%
\pgfsetlinewidth{0.000000pt}%
\definecolor{currentstroke}{rgb}{0.000000,0.000000,0.000000}%
\pgfsetstrokecolor{currentstroke}%
\pgfsetstrokeopacity{0.000000}%
\pgfsetdash{}{0pt}%
\pgfpathmoveto{\pgfqpoint{1.522798in}{0.500000in}}%
\pgfpathlineto{\pgfqpoint{1.555824in}{0.500000in}}%
\pgfpathlineto{\pgfqpoint{1.555824in}{0.541835in}}%
\pgfpathlineto{\pgfqpoint{1.522798in}{0.541835in}}%
\pgfpathlineto{\pgfqpoint{1.522798in}{0.500000in}}%
\pgfpathclose%
\pgfusepath{fill}%
\end{pgfscope}%
\begin{pgfscope}%
\pgfpathrectangle{\pgfqpoint{0.750000in}{0.500000in}}{\pgfqpoint{4.650000in}{3.020000in}}%
\pgfusepath{clip}%
\pgfsetbuttcap%
\pgfsetmiterjoin%
\definecolor{currentfill}{rgb}{1.000000,0.000000,0.000000}%
\pgfsetfillcolor{currentfill}%
\pgfsetlinewidth{0.000000pt}%
\definecolor{currentstroke}{rgb}{0.000000,0.000000,0.000000}%
\pgfsetstrokecolor{currentstroke}%
\pgfsetstrokeopacity{0.000000}%
\pgfsetdash{}{0pt}%
\pgfpathmoveto{\pgfqpoint{1.555824in}{0.500000in}}%
\pgfpathlineto{\pgfqpoint{1.588849in}{0.500000in}}%
\pgfpathlineto{\pgfqpoint{1.588849in}{0.510459in}}%
\pgfpathlineto{\pgfqpoint{1.555824in}{0.510459in}}%
\pgfpathlineto{\pgfqpoint{1.555824in}{0.500000in}}%
\pgfpathclose%
\pgfusepath{fill}%
\end{pgfscope}%
\begin{pgfscope}%
\pgfpathrectangle{\pgfqpoint{0.750000in}{0.500000in}}{\pgfqpoint{4.650000in}{3.020000in}}%
\pgfusepath{clip}%
\pgfsetbuttcap%
\pgfsetmiterjoin%
\definecolor{currentfill}{rgb}{1.000000,0.000000,0.000000}%
\pgfsetfillcolor{currentfill}%
\pgfsetlinewidth{0.000000pt}%
\definecolor{currentstroke}{rgb}{0.000000,0.000000,0.000000}%
\pgfsetstrokecolor{currentstroke}%
\pgfsetstrokeopacity{0.000000}%
\pgfsetdash{}{0pt}%
\pgfpathmoveto{\pgfqpoint{1.588849in}{0.500000in}}%
\pgfpathlineto{\pgfqpoint{1.621875in}{0.500000in}}%
\pgfpathlineto{\pgfqpoint{1.621875in}{0.635965in}}%
\pgfpathlineto{\pgfqpoint{1.588849in}{0.635965in}}%
\pgfpathlineto{\pgfqpoint{1.588849in}{0.500000in}}%
\pgfpathclose%
\pgfusepath{fill}%
\end{pgfscope}%
\begin{pgfscope}%
\pgfpathrectangle{\pgfqpoint{0.750000in}{0.500000in}}{\pgfqpoint{4.650000in}{3.020000in}}%
\pgfusepath{clip}%
\pgfsetbuttcap%
\pgfsetmiterjoin%
\definecolor{currentfill}{rgb}{1.000000,0.000000,0.000000}%
\pgfsetfillcolor{currentfill}%
\pgfsetlinewidth{0.000000pt}%
\definecolor{currentstroke}{rgb}{0.000000,0.000000,0.000000}%
\pgfsetstrokecolor{currentstroke}%
\pgfsetstrokeopacity{0.000000}%
\pgfsetdash{}{0pt}%
\pgfpathmoveto{\pgfqpoint{1.621875in}{0.500000in}}%
\pgfpathlineto{\pgfqpoint{1.654901in}{0.500000in}}%
\pgfpathlineto{\pgfqpoint{1.654901in}{0.510459in}}%
\pgfpathlineto{\pgfqpoint{1.621875in}{0.510459in}}%
\pgfpathlineto{\pgfqpoint{1.621875in}{0.500000in}}%
\pgfpathclose%
\pgfusepath{fill}%
\end{pgfscope}%
\begin{pgfscope}%
\pgfpathrectangle{\pgfqpoint{0.750000in}{0.500000in}}{\pgfqpoint{4.650000in}{3.020000in}}%
\pgfusepath{clip}%
\pgfsetbuttcap%
\pgfsetmiterjoin%
\definecolor{currentfill}{rgb}{1.000000,0.000000,0.000000}%
\pgfsetfillcolor{currentfill}%
\pgfsetlinewidth{0.000000pt}%
\definecolor{currentstroke}{rgb}{0.000000,0.000000,0.000000}%
\pgfsetstrokecolor{currentstroke}%
\pgfsetstrokeopacity{0.000000}%
\pgfsetdash{}{0pt}%
\pgfpathmoveto{\pgfqpoint{1.654901in}{0.500000in}}%
\pgfpathlineto{\pgfqpoint{1.687926in}{0.500000in}}%
\pgfpathlineto{\pgfqpoint{1.687926in}{0.520918in}}%
\pgfpathlineto{\pgfqpoint{1.654901in}{0.520918in}}%
\pgfpathlineto{\pgfqpoint{1.654901in}{0.500000in}}%
\pgfpathclose%
\pgfusepath{fill}%
\end{pgfscope}%
\begin{pgfscope}%
\pgfpathrectangle{\pgfqpoint{0.750000in}{0.500000in}}{\pgfqpoint{4.650000in}{3.020000in}}%
\pgfusepath{clip}%
\pgfsetbuttcap%
\pgfsetmiterjoin%
\definecolor{currentfill}{rgb}{1.000000,0.000000,0.000000}%
\pgfsetfillcolor{currentfill}%
\pgfsetlinewidth{0.000000pt}%
\definecolor{currentstroke}{rgb}{0.000000,0.000000,0.000000}%
\pgfsetstrokecolor{currentstroke}%
\pgfsetstrokeopacity{0.000000}%
\pgfsetdash{}{0pt}%
\pgfpathmoveto{\pgfqpoint{1.687926in}{0.500000in}}%
\pgfpathlineto{\pgfqpoint{1.720952in}{0.500000in}}%
\pgfpathlineto{\pgfqpoint{1.720952in}{0.615048in}}%
\pgfpathlineto{\pgfqpoint{1.687926in}{0.615048in}}%
\pgfpathlineto{\pgfqpoint{1.687926in}{0.500000in}}%
\pgfpathclose%
\pgfusepath{fill}%
\end{pgfscope}%
\begin{pgfscope}%
\pgfpathrectangle{\pgfqpoint{0.750000in}{0.500000in}}{\pgfqpoint{4.650000in}{3.020000in}}%
\pgfusepath{clip}%
\pgfsetbuttcap%
\pgfsetmiterjoin%
\definecolor{currentfill}{rgb}{1.000000,0.000000,0.000000}%
\pgfsetfillcolor{currentfill}%
\pgfsetlinewidth{0.000000pt}%
\definecolor{currentstroke}{rgb}{0.000000,0.000000,0.000000}%
\pgfsetstrokecolor{currentstroke}%
\pgfsetstrokeopacity{0.000000}%
\pgfsetdash{}{0pt}%
\pgfpathmoveto{\pgfqpoint{1.720952in}{0.500000in}}%
\pgfpathlineto{\pgfqpoint{1.753977in}{0.500000in}}%
\pgfpathlineto{\pgfqpoint{1.753977in}{0.500000in}}%
\pgfpathlineto{\pgfqpoint{1.720952in}{0.500000in}}%
\pgfpathlineto{\pgfqpoint{1.720952in}{0.500000in}}%
\pgfpathclose%
\pgfusepath{fill}%
\end{pgfscope}%
\begin{pgfscope}%
\pgfpathrectangle{\pgfqpoint{0.750000in}{0.500000in}}{\pgfqpoint{4.650000in}{3.020000in}}%
\pgfusepath{clip}%
\pgfsetbuttcap%
\pgfsetmiterjoin%
\definecolor{currentfill}{rgb}{1.000000,0.000000,0.000000}%
\pgfsetfillcolor{currentfill}%
\pgfsetlinewidth{0.000000pt}%
\definecolor{currentstroke}{rgb}{0.000000,0.000000,0.000000}%
\pgfsetstrokecolor{currentstroke}%
\pgfsetstrokeopacity{0.000000}%
\pgfsetdash{}{0pt}%
\pgfpathmoveto{\pgfqpoint{1.753977in}{0.500000in}}%
\pgfpathlineto{\pgfqpoint{1.787003in}{0.500000in}}%
\pgfpathlineto{\pgfqpoint{1.787003in}{0.656883in}}%
\pgfpathlineto{\pgfqpoint{1.753977in}{0.656883in}}%
\pgfpathlineto{\pgfqpoint{1.753977in}{0.500000in}}%
\pgfpathclose%
\pgfusepath{fill}%
\end{pgfscope}%
\begin{pgfscope}%
\pgfpathrectangle{\pgfqpoint{0.750000in}{0.500000in}}{\pgfqpoint{4.650000in}{3.020000in}}%
\pgfusepath{clip}%
\pgfsetbuttcap%
\pgfsetmiterjoin%
\definecolor{currentfill}{rgb}{1.000000,0.000000,0.000000}%
\pgfsetfillcolor{currentfill}%
\pgfsetlinewidth{0.000000pt}%
\definecolor{currentstroke}{rgb}{0.000000,0.000000,0.000000}%
\pgfsetstrokecolor{currentstroke}%
\pgfsetstrokeopacity{0.000000}%
\pgfsetdash{}{0pt}%
\pgfpathmoveto{\pgfqpoint{1.787003in}{0.500000in}}%
\pgfpathlineto{\pgfqpoint{1.820028in}{0.500000in}}%
\pgfpathlineto{\pgfqpoint{1.820028in}{0.510459in}}%
\pgfpathlineto{\pgfqpoint{1.787003in}{0.510459in}}%
\pgfpathlineto{\pgfqpoint{1.787003in}{0.500000in}}%
\pgfpathclose%
\pgfusepath{fill}%
\end{pgfscope}%
\begin{pgfscope}%
\pgfpathrectangle{\pgfqpoint{0.750000in}{0.500000in}}{\pgfqpoint{4.650000in}{3.020000in}}%
\pgfusepath{clip}%
\pgfsetbuttcap%
\pgfsetmiterjoin%
\definecolor{currentfill}{rgb}{1.000000,0.000000,0.000000}%
\pgfsetfillcolor{currentfill}%
\pgfsetlinewidth{0.000000pt}%
\definecolor{currentstroke}{rgb}{0.000000,0.000000,0.000000}%
\pgfsetstrokecolor{currentstroke}%
\pgfsetstrokeopacity{0.000000}%
\pgfsetdash{}{0pt}%
\pgfpathmoveto{\pgfqpoint{1.820028in}{0.500000in}}%
\pgfpathlineto{\pgfqpoint{1.853054in}{0.500000in}}%
\pgfpathlineto{\pgfqpoint{1.853054in}{0.677801in}}%
\pgfpathlineto{\pgfqpoint{1.820028in}{0.677801in}}%
\pgfpathlineto{\pgfqpoint{1.820028in}{0.500000in}}%
\pgfpathclose%
\pgfusepath{fill}%
\end{pgfscope}%
\begin{pgfscope}%
\pgfpathrectangle{\pgfqpoint{0.750000in}{0.500000in}}{\pgfqpoint{4.650000in}{3.020000in}}%
\pgfusepath{clip}%
\pgfsetbuttcap%
\pgfsetmiterjoin%
\definecolor{currentfill}{rgb}{1.000000,0.000000,0.000000}%
\pgfsetfillcolor{currentfill}%
\pgfsetlinewidth{0.000000pt}%
\definecolor{currentstroke}{rgb}{0.000000,0.000000,0.000000}%
\pgfsetstrokecolor{currentstroke}%
\pgfsetstrokeopacity{0.000000}%
\pgfsetdash{}{0pt}%
\pgfpathmoveto{\pgfqpoint{1.853054in}{0.500000in}}%
\pgfpathlineto{\pgfqpoint{1.886080in}{0.500000in}}%
\pgfpathlineto{\pgfqpoint{1.886080in}{0.541835in}}%
\pgfpathlineto{\pgfqpoint{1.853054in}{0.541835in}}%
\pgfpathlineto{\pgfqpoint{1.853054in}{0.500000in}}%
\pgfpathclose%
\pgfusepath{fill}%
\end{pgfscope}%
\begin{pgfscope}%
\pgfpathrectangle{\pgfqpoint{0.750000in}{0.500000in}}{\pgfqpoint{4.650000in}{3.020000in}}%
\pgfusepath{clip}%
\pgfsetbuttcap%
\pgfsetmiterjoin%
\definecolor{currentfill}{rgb}{1.000000,0.000000,0.000000}%
\pgfsetfillcolor{currentfill}%
\pgfsetlinewidth{0.000000pt}%
\definecolor{currentstroke}{rgb}{0.000000,0.000000,0.000000}%
\pgfsetstrokecolor{currentstroke}%
\pgfsetstrokeopacity{0.000000}%
\pgfsetdash{}{0pt}%
\pgfpathmoveto{\pgfqpoint{1.886080in}{0.500000in}}%
\pgfpathlineto{\pgfqpoint{1.919105in}{0.500000in}}%
\pgfpathlineto{\pgfqpoint{1.919105in}{0.541835in}}%
\pgfpathlineto{\pgfqpoint{1.886080in}{0.541835in}}%
\pgfpathlineto{\pgfqpoint{1.886080in}{0.500000in}}%
\pgfpathclose%
\pgfusepath{fill}%
\end{pgfscope}%
\begin{pgfscope}%
\pgfpathrectangle{\pgfqpoint{0.750000in}{0.500000in}}{\pgfqpoint{4.650000in}{3.020000in}}%
\pgfusepath{clip}%
\pgfsetbuttcap%
\pgfsetmiterjoin%
\definecolor{currentfill}{rgb}{1.000000,0.000000,0.000000}%
\pgfsetfillcolor{currentfill}%
\pgfsetlinewidth{0.000000pt}%
\definecolor{currentstroke}{rgb}{0.000000,0.000000,0.000000}%
\pgfsetstrokecolor{currentstroke}%
\pgfsetstrokeopacity{0.000000}%
\pgfsetdash{}{0pt}%
\pgfpathmoveto{\pgfqpoint{1.919105in}{0.500000in}}%
\pgfpathlineto{\pgfqpoint{1.952131in}{0.500000in}}%
\pgfpathlineto{\pgfqpoint{1.952131in}{0.918355in}}%
\pgfpathlineto{\pgfqpoint{1.919105in}{0.918355in}}%
\pgfpathlineto{\pgfqpoint{1.919105in}{0.500000in}}%
\pgfpathclose%
\pgfusepath{fill}%
\end{pgfscope}%
\begin{pgfscope}%
\pgfpathrectangle{\pgfqpoint{0.750000in}{0.500000in}}{\pgfqpoint{4.650000in}{3.020000in}}%
\pgfusepath{clip}%
\pgfsetbuttcap%
\pgfsetmiterjoin%
\definecolor{currentfill}{rgb}{1.000000,0.000000,0.000000}%
\pgfsetfillcolor{currentfill}%
\pgfsetlinewidth{0.000000pt}%
\definecolor{currentstroke}{rgb}{0.000000,0.000000,0.000000}%
\pgfsetstrokecolor{currentstroke}%
\pgfsetstrokeopacity{0.000000}%
\pgfsetdash{}{0pt}%
\pgfpathmoveto{\pgfqpoint{1.952131in}{0.500000in}}%
\pgfpathlineto{\pgfqpoint{1.985156in}{0.500000in}}%
\pgfpathlineto{\pgfqpoint{1.985156in}{0.541835in}}%
\pgfpathlineto{\pgfqpoint{1.952131in}{0.541835in}}%
\pgfpathlineto{\pgfqpoint{1.952131in}{0.500000in}}%
\pgfpathclose%
\pgfusepath{fill}%
\end{pgfscope}%
\begin{pgfscope}%
\pgfpathrectangle{\pgfqpoint{0.750000in}{0.500000in}}{\pgfqpoint{4.650000in}{3.020000in}}%
\pgfusepath{clip}%
\pgfsetbuttcap%
\pgfsetmiterjoin%
\definecolor{currentfill}{rgb}{1.000000,0.000000,0.000000}%
\pgfsetfillcolor{currentfill}%
\pgfsetlinewidth{0.000000pt}%
\definecolor{currentstroke}{rgb}{0.000000,0.000000,0.000000}%
\pgfsetstrokecolor{currentstroke}%
\pgfsetstrokeopacity{0.000000}%
\pgfsetdash{}{0pt}%
\pgfpathmoveto{\pgfqpoint{1.985156in}{0.500000in}}%
\pgfpathlineto{\pgfqpoint{2.018182in}{0.500000in}}%
\pgfpathlineto{\pgfqpoint{2.018182in}{0.907896in}}%
\pgfpathlineto{\pgfqpoint{1.985156in}{0.907896in}}%
\pgfpathlineto{\pgfqpoint{1.985156in}{0.500000in}}%
\pgfpathclose%
\pgfusepath{fill}%
\end{pgfscope}%
\begin{pgfscope}%
\pgfpathrectangle{\pgfqpoint{0.750000in}{0.500000in}}{\pgfqpoint{4.650000in}{3.020000in}}%
\pgfusepath{clip}%
\pgfsetbuttcap%
\pgfsetmiterjoin%
\definecolor{currentfill}{rgb}{1.000000,0.000000,0.000000}%
\pgfsetfillcolor{currentfill}%
\pgfsetlinewidth{0.000000pt}%
\definecolor{currentstroke}{rgb}{0.000000,0.000000,0.000000}%
\pgfsetstrokecolor{currentstroke}%
\pgfsetstrokeopacity{0.000000}%
\pgfsetdash{}{0pt}%
\pgfpathmoveto{\pgfqpoint{2.018182in}{0.500000in}}%
\pgfpathlineto{\pgfqpoint{2.051207in}{0.500000in}}%
\pgfpathlineto{\pgfqpoint{2.051207in}{0.520918in}}%
\pgfpathlineto{\pgfqpoint{2.018182in}{0.520918in}}%
\pgfpathlineto{\pgfqpoint{2.018182in}{0.500000in}}%
\pgfpathclose%
\pgfusepath{fill}%
\end{pgfscope}%
\begin{pgfscope}%
\pgfpathrectangle{\pgfqpoint{0.750000in}{0.500000in}}{\pgfqpoint{4.650000in}{3.020000in}}%
\pgfusepath{clip}%
\pgfsetbuttcap%
\pgfsetmiterjoin%
\definecolor{currentfill}{rgb}{1.000000,0.000000,0.000000}%
\pgfsetfillcolor{currentfill}%
\pgfsetlinewidth{0.000000pt}%
\definecolor{currentstroke}{rgb}{0.000000,0.000000,0.000000}%
\pgfsetstrokecolor{currentstroke}%
\pgfsetstrokeopacity{0.000000}%
\pgfsetdash{}{0pt}%
\pgfpathmoveto{\pgfqpoint{2.051207in}{0.500000in}}%
\pgfpathlineto{\pgfqpoint{2.084233in}{0.500000in}}%
\pgfpathlineto{\pgfqpoint{2.084233in}{0.604589in}}%
\pgfpathlineto{\pgfqpoint{2.051207in}{0.604589in}}%
\pgfpathlineto{\pgfqpoint{2.051207in}{0.500000in}}%
\pgfpathclose%
\pgfusepath{fill}%
\end{pgfscope}%
\begin{pgfscope}%
\pgfpathrectangle{\pgfqpoint{0.750000in}{0.500000in}}{\pgfqpoint{4.650000in}{3.020000in}}%
\pgfusepath{clip}%
\pgfsetbuttcap%
\pgfsetmiterjoin%
\definecolor{currentfill}{rgb}{1.000000,0.000000,0.000000}%
\pgfsetfillcolor{currentfill}%
\pgfsetlinewidth{0.000000pt}%
\definecolor{currentstroke}{rgb}{0.000000,0.000000,0.000000}%
\pgfsetstrokecolor{currentstroke}%
\pgfsetstrokeopacity{0.000000}%
\pgfsetdash{}{0pt}%
\pgfpathmoveto{\pgfqpoint{2.084233in}{0.500000in}}%
\pgfpathlineto{\pgfqpoint{2.117259in}{0.500000in}}%
\pgfpathlineto{\pgfqpoint{2.117259in}{1.012485in}}%
\pgfpathlineto{\pgfqpoint{2.084233in}{1.012485in}}%
\pgfpathlineto{\pgfqpoint{2.084233in}{0.500000in}}%
\pgfpathclose%
\pgfusepath{fill}%
\end{pgfscope}%
\begin{pgfscope}%
\pgfpathrectangle{\pgfqpoint{0.750000in}{0.500000in}}{\pgfqpoint{4.650000in}{3.020000in}}%
\pgfusepath{clip}%
\pgfsetbuttcap%
\pgfsetmiterjoin%
\definecolor{currentfill}{rgb}{1.000000,0.000000,0.000000}%
\pgfsetfillcolor{currentfill}%
\pgfsetlinewidth{0.000000pt}%
\definecolor{currentstroke}{rgb}{0.000000,0.000000,0.000000}%
\pgfsetstrokecolor{currentstroke}%
\pgfsetstrokeopacity{0.000000}%
\pgfsetdash{}{0pt}%
\pgfpathmoveto{\pgfqpoint{2.117259in}{0.500000in}}%
\pgfpathlineto{\pgfqpoint{2.150284in}{0.500000in}}%
\pgfpathlineto{\pgfqpoint{2.150284in}{0.573212in}}%
\pgfpathlineto{\pgfqpoint{2.117259in}{0.573212in}}%
\pgfpathlineto{\pgfqpoint{2.117259in}{0.500000in}}%
\pgfpathclose%
\pgfusepath{fill}%
\end{pgfscope}%
\begin{pgfscope}%
\pgfpathrectangle{\pgfqpoint{0.750000in}{0.500000in}}{\pgfqpoint{4.650000in}{3.020000in}}%
\pgfusepath{clip}%
\pgfsetbuttcap%
\pgfsetmiterjoin%
\definecolor{currentfill}{rgb}{1.000000,0.000000,0.000000}%
\pgfsetfillcolor{currentfill}%
\pgfsetlinewidth{0.000000pt}%
\definecolor{currentstroke}{rgb}{0.000000,0.000000,0.000000}%
\pgfsetstrokecolor{currentstroke}%
\pgfsetstrokeopacity{0.000000}%
\pgfsetdash{}{0pt}%
\pgfpathmoveto{\pgfqpoint{2.150284in}{0.500000in}}%
\pgfpathlineto{\pgfqpoint{2.183310in}{0.500000in}}%
\pgfpathlineto{\pgfqpoint{2.183310in}{1.002026in}}%
\pgfpathlineto{\pgfqpoint{2.150284in}{1.002026in}}%
\pgfpathlineto{\pgfqpoint{2.150284in}{0.500000in}}%
\pgfpathclose%
\pgfusepath{fill}%
\end{pgfscope}%
\begin{pgfscope}%
\pgfpathrectangle{\pgfqpoint{0.750000in}{0.500000in}}{\pgfqpoint{4.650000in}{3.020000in}}%
\pgfusepath{clip}%
\pgfsetbuttcap%
\pgfsetmiterjoin%
\definecolor{currentfill}{rgb}{1.000000,0.000000,0.000000}%
\pgfsetfillcolor{currentfill}%
\pgfsetlinewidth{0.000000pt}%
\definecolor{currentstroke}{rgb}{0.000000,0.000000,0.000000}%
\pgfsetstrokecolor{currentstroke}%
\pgfsetstrokeopacity{0.000000}%
\pgfsetdash{}{0pt}%
\pgfpathmoveto{\pgfqpoint{2.183310in}{0.500000in}}%
\pgfpathlineto{\pgfqpoint{2.216335in}{0.500000in}}%
\pgfpathlineto{\pgfqpoint{2.216335in}{0.625506in}}%
\pgfpathlineto{\pgfqpoint{2.183310in}{0.625506in}}%
\pgfpathlineto{\pgfqpoint{2.183310in}{0.500000in}}%
\pgfpathclose%
\pgfusepath{fill}%
\end{pgfscope}%
\begin{pgfscope}%
\pgfpathrectangle{\pgfqpoint{0.750000in}{0.500000in}}{\pgfqpoint{4.650000in}{3.020000in}}%
\pgfusepath{clip}%
\pgfsetbuttcap%
\pgfsetmiterjoin%
\definecolor{currentfill}{rgb}{1.000000,0.000000,0.000000}%
\pgfsetfillcolor{currentfill}%
\pgfsetlinewidth{0.000000pt}%
\definecolor{currentstroke}{rgb}{0.000000,0.000000,0.000000}%
\pgfsetstrokecolor{currentstroke}%
\pgfsetstrokeopacity{0.000000}%
\pgfsetdash{}{0pt}%
\pgfpathmoveto{\pgfqpoint{2.216335in}{0.500000in}}%
\pgfpathlineto{\pgfqpoint{2.249361in}{0.500000in}}%
\pgfpathlineto{\pgfqpoint{2.249361in}{0.615048in}}%
\pgfpathlineto{\pgfqpoint{2.216335in}{0.615048in}}%
\pgfpathlineto{\pgfqpoint{2.216335in}{0.500000in}}%
\pgfpathclose%
\pgfusepath{fill}%
\end{pgfscope}%
\begin{pgfscope}%
\pgfpathrectangle{\pgfqpoint{0.750000in}{0.500000in}}{\pgfqpoint{4.650000in}{3.020000in}}%
\pgfusepath{clip}%
\pgfsetbuttcap%
\pgfsetmiterjoin%
\definecolor{currentfill}{rgb}{1.000000,0.000000,0.000000}%
\pgfsetfillcolor{currentfill}%
\pgfsetlinewidth{0.000000pt}%
\definecolor{currentstroke}{rgb}{0.000000,0.000000,0.000000}%
\pgfsetstrokecolor{currentstroke}%
\pgfsetstrokeopacity{0.000000}%
\pgfsetdash{}{0pt}%
\pgfpathmoveto{\pgfqpoint{2.249361in}{0.500000in}}%
\pgfpathlineto{\pgfqpoint{2.282386in}{0.500000in}}%
\pgfpathlineto{\pgfqpoint{2.282386in}{1.284416in}}%
\pgfpathlineto{\pgfqpoint{2.249361in}{1.284416in}}%
\pgfpathlineto{\pgfqpoint{2.249361in}{0.500000in}}%
\pgfpathclose%
\pgfusepath{fill}%
\end{pgfscope}%
\begin{pgfscope}%
\pgfpathrectangle{\pgfqpoint{0.750000in}{0.500000in}}{\pgfqpoint{4.650000in}{3.020000in}}%
\pgfusepath{clip}%
\pgfsetbuttcap%
\pgfsetmiterjoin%
\definecolor{currentfill}{rgb}{1.000000,0.000000,0.000000}%
\pgfsetfillcolor{currentfill}%
\pgfsetlinewidth{0.000000pt}%
\definecolor{currentstroke}{rgb}{0.000000,0.000000,0.000000}%
\pgfsetstrokecolor{currentstroke}%
\pgfsetstrokeopacity{0.000000}%
\pgfsetdash{}{0pt}%
\pgfpathmoveto{\pgfqpoint{2.282386in}{0.500000in}}%
\pgfpathlineto{\pgfqpoint{2.315412in}{0.500000in}}%
\pgfpathlineto{\pgfqpoint{2.315412in}{0.667342in}}%
\pgfpathlineto{\pgfqpoint{2.282386in}{0.667342in}}%
\pgfpathlineto{\pgfqpoint{2.282386in}{0.500000in}}%
\pgfpathclose%
\pgfusepath{fill}%
\end{pgfscope}%
\begin{pgfscope}%
\pgfpathrectangle{\pgfqpoint{0.750000in}{0.500000in}}{\pgfqpoint{4.650000in}{3.020000in}}%
\pgfusepath{clip}%
\pgfsetbuttcap%
\pgfsetmiterjoin%
\definecolor{currentfill}{rgb}{1.000000,0.000000,0.000000}%
\pgfsetfillcolor{currentfill}%
\pgfsetlinewidth{0.000000pt}%
\definecolor{currentstroke}{rgb}{0.000000,0.000000,0.000000}%
\pgfsetstrokecolor{currentstroke}%
\pgfsetstrokeopacity{0.000000}%
\pgfsetdash{}{0pt}%
\pgfpathmoveto{\pgfqpoint{2.315412in}{0.500000in}}%
\pgfpathlineto{\pgfqpoint{2.348437in}{0.500000in}}%
\pgfpathlineto{\pgfqpoint{2.348437in}{1.357628in}}%
\pgfpathlineto{\pgfqpoint{2.315412in}{1.357628in}}%
\pgfpathlineto{\pgfqpoint{2.315412in}{0.500000in}}%
\pgfpathclose%
\pgfusepath{fill}%
\end{pgfscope}%
\begin{pgfscope}%
\pgfpathrectangle{\pgfqpoint{0.750000in}{0.500000in}}{\pgfqpoint{4.650000in}{3.020000in}}%
\pgfusepath{clip}%
\pgfsetbuttcap%
\pgfsetmiterjoin%
\definecolor{currentfill}{rgb}{1.000000,0.000000,0.000000}%
\pgfsetfillcolor{currentfill}%
\pgfsetlinewidth{0.000000pt}%
\definecolor{currentstroke}{rgb}{0.000000,0.000000,0.000000}%
\pgfsetstrokecolor{currentstroke}%
\pgfsetstrokeopacity{0.000000}%
\pgfsetdash{}{0pt}%
\pgfpathmoveto{\pgfqpoint{2.348437in}{0.500000in}}%
\pgfpathlineto{\pgfqpoint{2.381463in}{0.500000in}}%
\pgfpathlineto{\pgfqpoint{2.381463in}{0.730095in}}%
\pgfpathlineto{\pgfqpoint{2.348437in}{0.730095in}}%
\pgfpathlineto{\pgfqpoint{2.348437in}{0.500000in}}%
\pgfpathclose%
\pgfusepath{fill}%
\end{pgfscope}%
\begin{pgfscope}%
\pgfpathrectangle{\pgfqpoint{0.750000in}{0.500000in}}{\pgfqpoint{4.650000in}{3.020000in}}%
\pgfusepath{clip}%
\pgfsetbuttcap%
\pgfsetmiterjoin%
\definecolor{currentfill}{rgb}{1.000000,0.000000,0.000000}%
\pgfsetfillcolor{currentfill}%
\pgfsetlinewidth{0.000000pt}%
\definecolor{currentstroke}{rgb}{0.000000,0.000000,0.000000}%
\pgfsetstrokecolor{currentstroke}%
\pgfsetstrokeopacity{0.000000}%
\pgfsetdash{}{0pt}%
\pgfpathmoveto{\pgfqpoint{2.381463in}{0.500000in}}%
\pgfpathlineto{\pgfqpoint{2.414489in}{0.500000in}}%
\pgfpathlineto{\pgfqpoint{2.414489in}{0.730095in}}%
\pgfpathlineto{\pgfqpoint{2.381463in}{0.730095in}}%
\pgfpathlineto{\pgfqpoint{2.381463in}{0.500000in}}%
\pgfpathclose%
\pgfusepath{fill}%
\end{pgfscope}%
\begin{pgfscope}%
\pgfpathrectangle{\pgfqpoint{0.750000in}{0.500000in}}{\pgfqpoint{4.650000in}{3.020000in}}%
\pgfusepath{clip}%
\pgfsetbuttcap%
\pgfsetmiterjoin%
\definecolor{currentfill}{rgb}{1.000000,0.000000,0.000000}%
\pgfsetfillcolor{currentfill}%
\pgfsetlinewidth{0.000000pt}%
\definecolor{currentstroke}{rgb}{0.000000,0.000000,0.000000}%
\pgfsetstrokecolor{currentstroke}%
\pgfsetstrokeopacity{0.000000}%
\pgfsetdash{}{0pt}%
\pgfpathmoveto{\pgfqpoint{2.414489in}{0.500000in}}%
\pgfpathlineto{\pgfqpoint{2.447514in}{0.500000in}}%
\pgfpathlineto{\pgfqpoint{2.447514in}{1.796900in}}%
\pgfpathlineto{\pgfqpoint{2.414489in}{1.796900in}}%
\pgfpathlineto{\pgfqpoint{2.414489in}{0.500000in}}%
\pgfpathclose%
\pgfusepath{fill}%
\end{pgfscope}%
\begin{pgfscope}%
\pgfpathrectangle{\pgfqpoint{0.750000in}{0.500000in}}{\pgfqpoint{4.650000in}{3.020000in}}%
\pgfusepath{clip}%
\pgfsetbuttcap%
\pgfsetmiterjoin%
\definecolor{currentfill}{rgb}{1.000000,0.000000,0.000000}%
\pgfsetfillcolor{currentfill}%
\pgfsetlinewidth{0.000000pt}%
\definecolor{currentstroke}{rgb}{0.000000,0.000000,0.000000}%
\pgfsetstrokecolor{currentstroke}%
\pgfsetstrokeopacity{0.000000}%
\pgfsetdash{}{0pt}%
\pgfpathmoveto{\pgfqpoint{2.447514in}{0.500000in}}%
\pgfpathlineto{\pgfqpoint{2.480540in}{0.500000in}}%
\pgfpathlineto{\pgfqpoint{2.480540in}{0.751013in}}%
\pgfpathlineto{\pgfqpoint{2.447514in}{0.751013in}}%
\pgfpathlineto{\pgfqpoint{2.447514in}{0.500000in}}%
\pgfpathclose%
\pgfusepath{fill}%
\end{pgfscope}%
\begin{pgfscope}%
\pgfpathrectangle{\pgfqpoint{0.750000in}{0.500000in}}{\pgfqpoint{4.650000in}{3.020000in}}%
\pgfusepath{clip}%
\pgfsetbuttcap%
\pgfsetmiterjoin%
\definecolor{currentfill}{rgb}{1.000000,0.000000,0.000000}%
\pgfsetfillcolor{currentfill}%
\pgfsetlinewidth{0.000000pt}%
\definecolor{currentstroke}{rgb}{0.000000,0.000000,0.000000}%
\pgfsetstrokecolor{currentstroke}%
\pgfsetstrokeopacity{0.000000}%
\pgfsetdash{}{0pt}%
\pgfpathmoveto{\pgfqpoint{2.480540in}{0.500000in}}%
\pgfpathlineto{\pgfqpoint{2.513565in}{0.500000in}}%
\pgfpathlineto{\pgfqpoint{2.513565in}{1.932866in}}%
\pgfpathlineto{\pgfqpoint{2.480540in}{1.932866in}}%
\pgfpathlineto{\pgfqpoint{2.480540in}{0.500000in}}%
\pgfpathclose%
\pgfusepath{fill}%
\end{pgfscope}%
\begin{pgfscope}%
\pgfpathrectangle{\pgfqpoint{0.750000in}{0.500000in}}{\pgfqpoint{4.650000in}{3.020000in}}%
\pgfusepath{clip}%
\pgfsetbuttcap%
\pgfsetmiterjoin%
\definecolor{currentfill}{rgb}{1.000000,0.000000,0.000000}%
\pgfsetfillcolor{currentfill}%
\pgfsetlinewidth{0.000000pt}%
\definecolor{currentstroke}{rgb}{0.000000,0.000000,0.000000}%
\pgfsetstrokecolor{currentstroke}%
\pgfsetstrokeopacity{0.000000}%
\pgfsetdash{}{0pt}%
\pgfpathmoveto{\pgfqpoint{2.513565in}{0.500000in}}%
\pgfpathlineto{\pgfqpoint{2.546591in}{0.500000in}}%
\pgfpathlineto{\pgfqpoint{2.546591in}{0.751013in}}%
\pgfpathlineto{\pgfqpoint{2.513565in}{0.751013in}}%
\pgfpathlineto{\pgfqpoint{2.513565in}{0.500000in}}%
\pgfpathclose%
\pgfusepath{fill}%
\end{pgfscope}%
\begin{pgfscope}%
\pgfpathrectangle{\pgfqpoint{0.750000in}{0.500000in}}{\pgfqpoint{4.650000in}{3.020000in}}%
\pgfusepath{clip}%
\pgfsetbuttcap%
\pgfsetmiterjoin%
\definecolor{currentfill}{rgb}{1.000000,0.000000,0.000000}%
\pgfsetfillcolor{currentfill}%
\pgfsetlinewidth{0.000000pt}%
\definecolor{currentstroke}{rgb}{0.000000,0.000000,0.000000}%
\pgfsetstrokecolor{currentstroke}%
\pgfsetstrokeopacity{0.000000}%
\pgfsetdash{}{0pt}%
\pgfpathmoveto{\pgfqpoint{2.546591in}{0.500000in}}%
\pgfpathlineto{\pgfqpoint{2.579616in}{0.500000in}}%
\pgfpathlineto{\pgfqpoint{2.579616in}{0.792848in}}%
\pgfpathlineto{\pgfqpoint{2.546591in}{0.792848in}}%
\pgfpathlineto{\pgfqpoint{2.546591in}{0.500000in}}%
\pgfpathclose%
\pgfusepath{fill}%
\end{pgfscope}%
\begin{pgfscope}%
\pgfpathrectangle{\pgfqpoint{0.750000in}{0.500000in}}{\pgfqpoint{4.650000in}{3.020000in}}%
\pgfusepath{clip}%
\pgfsetbuttcap%
\pgfsetmiterjoin%
\definecolor{currentfill}{rgb}{1.000000,0.000000,0.000000}%
\pgfsetfillcolor{currentfill}%
\pgfsetlinewidth{0.000000pt}%
\definecolor{currentstroke}{rgb}{0.000000,0.000000,0.000000}%
\pgfsetstrokecolor{currentstroke}%
\pgfsetstrokeopacity{0.000000}%
\pgfsetdash{}{0pt}%
\pgfpathmoveto{\pgfqpoint{2.579616in}{0.500000in}}%
\pgfpathlineto{\pgfqpoint{2.612642in}{0.500000in}}%
\pgfpathlineto{\pgfqpoint{2.612642in}{2.288468in}}%
\pgfpathlineto{\pgfqpoint{2.579616in}{2.288468in}}%
\pgfpathlineto{\pgfqpoint{2.579616in}{0.500000in}}%
\pgfpathclose%
\pgfusepath{fill}%
\end{pgfscope}%
\begin{pgfscope}%
\pgfpathrectangle{\pgfqpoint{0.750000in}{0.500000in}}{\pgfqpoint{4.650000in}{3.020000in}}%
\pgfusepath{clip}%
\pgfsetbuttcap%
\pgfsetmiterjoin%
\definecolor{currentfill}{rgb}{1.000000,0.000000,0.000000}%
\pgfsetfillcolor{currentfill}%
\pgfsetlinewidth{0.000000pt}%
\definecolor{currentstroke}{rgb}{0.000000,0.000000,0.000000}%
\pgfsetstrokecolor{currentstroke}%
\pgfsetstrokeopacity{0.000000}%
\pgfsetdash{}{0pt}%
\pgfpathmoveto{\pgfqpoint{2.612642in}{0.500000in}}%
\pgfpathlineto{\pgfqpoint{2.645668in}{0.500000in}}%
\pgfpathlineto{\pgfqpoint{2.645668in}{0.782390in}}%
\pgfpathlineto{\pgfqpoint{2.612642in}{0.782390in}}%
\pgfpathlineto{\pgfqpoint{2.612642in}{0.500000in}}%
\pgfpathclose%
\pgfusepath{fill}%
\end{pgfscope}%
\begin{pgfscope}%
\pgfpathrectangle{\pgfqpoint{0.750000in}{0.500000in}}{\pgfqpoint{4.650000in}{3.020000in}}%
\pgfusepath{clip}%
\pgfsetbuttcap%
\pgfsetmiterjoin%
\definecolor{currentfill}{rgb}{1.000000,0.000000,0.000000}%
\pgfsetfillcolor{currentfill}%
\pgfsetlinewidth{0.000000pt}%
\definecolor{currentstroke}{rgb}{0.000000,0.000000,0.000000}%
\pgfsetstrokecolor{currentstroke}%
\pgfsetstrokeopacity{0.000000}%
\pgfsetdash{}{0pt}%
\pgfpathmoveto{\pgfqpoint{2.645668in}{0.500000in}}%
\pgfpathlineto{\pgfqpoint{2.678693in}{0.500000in}}%
\pgfpathlineto{\pgfqpoint{2.678693in}{2.319844in}}%
\pgfpathlineto{\pgfqpoint{2.645668in}{2.319844in}}%
\pgfpathlineto{\pgfqpoint{2.645668in}{0.500000in}}%
\pgfpathclose%
\pgfusepath{fill}%
\end{pgfscope}%
\begin{pgfscope}%
\pgfpathrectangle{\pgfqpoint{0.750000in}{0.500000in}}{\pgfqpoint{4.650000in}{3.020000in}}%
\pgfusepath{clip}%
\pgfsetbuttcap%
\pgfsetmiterjoin%
\definecolor{currentfill}{rgb}{1.000000,0.000000,0.000000}%
\pgfsetfillcolor{currentfill}%
\pgfsetlinewidth{0.000000pt}%
\definecolor{currentstroke}{rgb}{0.000000,0.000000,0.000000}%
\pgfsetstrokecolor{currentstroke}%
\pgfsetstrokeopacity{0.000000}%
\pgfsetdash{}{0pt}%
\pgfpathmoveto{\pgfqpoint{2.678693in}{0.500000in}}%
\pgfpathlineto{\pgfqpoint{2.711719in}{0.500000in}}%
\pgfpathlineto{\pgfqpoint{2.711719in}{0.813766in}}%
\pgfpathlineto{\pgfqpoint{2.678693in}{0.813766in}}%
\pgfpathlineto{\pgfqpoint{2.678693in}{0.500000in}}%
\pgfpathclose%
\pgfusepath{fill}%
\end{pgfscope}%
\begin{pgfscope}%
\pgfpathrectangle{\pgfqpoint{0.750000in}{0.500000in}}{\pgfqpoint{4.650000in}{3.020000in}}%
\pgfusepath{clip}%
\pgfsetbuttcap%
\pgfsetmiterjoin%
\definecolor{currentfill}{rgb}{1.000000,0.000000,0.000000}%
\pgfsetfillcolor{currentfill}%
\pgfsetlinewidth{0.000000pt}%
\definecolor{currentstroke}{rgb}{0.000000,0.000000,0.000000}%
\pgfsetstrokecolor{currentstroke}%
\pgfsetstrokeopacity{0.000000}%
\pgfsetdash{}{0pt}%
\pgfpathmoveto{\pgfqpoint{2.711719in}{0.500000in}}%
\pgfpathlineto{\pgfqpoint{2.744744in}{0.500000in}}%
\pgfpathlineto{\pgfqpoint{2.744744in}{0.949732in}}%
\pgfpathlineto{\pgfqpoint{2.711719in}{0.949732in}}%
\pgfpathlineto{\pgfqpoint{2.711719in}{0.500000in}}%
\pgfpathclose%
\pgfusepath{fill}%
\end{pgfscope}%
\begin{pgfscope}%
\pgfpathrectangle{\pgfqpoint{0.750000in}{0.500000in}}{\pgfqpoint{4.650000in}{3.020000in}}%
\pgfusepath{clip}%
\pgfsetbuttcap%
\pgfsetmiterjoin%
\definecolor{currentfill}{rgb}{1.000000,0.000000,0.000000}%
\pgfsetfillcolor{currentfill}%
\pgfsetlinewidth{0.000000pt}%
\definecolor{currentstroke}{rgb}{0.000000,0.000000,0.000000}%
\pgfsetstrokecolor{currentstroke}%
\pgfsetstrokeopacity{0.000000}%
\pgfsetdash{}{0pt}%
\pgfpathmoveto{\pgfqpoint{2.744744in}{0.500000in}}%
\pgfpathlineto{\pgfqpoint{2.777770in}{0.500000in}}%
\pgfpathlineto{\pgfqpoint{2.777770in}{2.403515in}}%
\pgfpathlineto{\pgfqpoint{2.744744in}{2.403515in}}%
\pgfpathlineto{\pgfqpoint{2.744744in}{0.500000in}}%
\pgfpathclose%
\pgfusepath{fill}%
\end{pgfscope}%
\begin{pgfscope}%
\pgfpathrectangle{\pgfqpoint{0.750000in}{0.500000in}}{\pgfqpoint{4.650000in}{3.020000in}}%
\pgfusepath{clip}%
\pgfsetbuttcap%
\pgfsetmiterjoin%
\definecolor{currentfill}{rgb}{1.000000,0.000000,0.000000}%
\pgfsetfillcolor{currentfill}%
\pgfsetlinewidth{0.000000pt}%
\definecolor{currentstroke}{rgb}{0.000000,0.000000,0.000000}%
\pgfsetstrokecolor{currentstroke}%
\pgfsetstrokeopacity{0.000000}%
\pgfsetdash{}{0pt}%
\pgfpathmoveto{\pgfqpoint{2.777770in}{0.500000in}}%
\pgfpathlineto{\pgfqpoint{2.810795in}{0.500000in}}%
\pgfpathlineto{\pgfqpoint{2.810795in}{0.939273in}}%
\pgfpathlineto{\pgfqpoint{2.777770in}{0.939273in}}%
\pgfpathlineto{\pgfqpoint{2.777770in}{0.500000in}}%
\pgfpathclose%
\pgfusepath{fill}%
\end{pgfscope}%
\begin{pgfscope}%
\pgfpathrectangle{\pgfqpoint{0.750000in}{0.500000in}}{\pgfqpoint{4.650000in}{3.020000in}}%
\pgfusepath{clip}%
\pgfsetbuttcap%
\pgfsetmiterjoin%
\definecolor{currentfill}{rgb}{1.000000,0.000000,0.000000}%
\pgfsetfillcolor{currentfill}%
\pgfsetlinewidth{0.000000pt}%
\definecolor{currentstroke}{rgb}{0.000000,0.000000,0.000000}%
\pgfsetstrokecolor{currentstroke}%
\pgfsetstrokeopacity{0.000000}%
\pgfsetdash{}{0pt}%
\pgfpathmoveto{\pgfqpoint{2.810795in}{0.500000in}}%
\pgfpathlineto{\pgfqpoint{2.843821in}{0.500000in}}%
\pgfpathlineto{\pgfqpoint{2.843821in}{2.351221in}}%
\pgfpathlineto{\pgfqpoint{2.810795in}{2.351221in}}%
\pgfpathlineto{\pgfqpoint{2.810795in}{0.500000in}}%
\pgfpathclose%
\pgfusepath{fill}%
\end{pgfscope}%
\begin{pgfscope}%
\pgfpathrectangle{\pgfqpoint{0.750000in}{0.500000in}}{\pgfqpoint{4.650000in}{3.020000in}}%
\pgfusepath{clip}%
\pgfsetbuttcap%
\pgfsetmiterjoin%
\definecolor{currentfill}{rgb}{1.000000,0.000000,0.000000}%
\pgfsetfillcolor{currentfill}%
\pgfsetlinewidth{0.000000pt}%
\definecolor{currentstroke}{rgb}{0.000000,0.000000,0.000000}%
\pgfsetstrokecolor{currentstroke}%
\pgfsetstrokeopacity{0.000000}%
\pgfsetdash{}{0pt}%
\pgfpathmoveto{\pgfqpoint{2.843821in}{0.500000in}}%
\pgfpathlineto{\pgfqpoint{2.876847in}{0.500000in}}%
\pgfpathlineto{\pgfqpoint{2.876847in}{0.907896in}}%
\pgfpathlineto{\pgfqpoint{2.843821in}{0.907896in}}%
\pgfpathlineto{\pgfqpoint{2.843821in}{0.500000in}}%
\pgfpathclose%
\pgfusepath{fill}%
\end{pgfscope}%
\begin{pgfscope}%
\pgfpathrectangle{\pgfqpoint{0.750000in}{0.500000in}}{\pgfqpoint{4.650000in}{3.020000in}}%
\pgfusepath{clip}%
\pgfsetbuttcap%
\pgfsetmiterjoin%
\definecolor{currentfill}{rgb}{1.000000,0.000000,0.000000}%
\pgfsetfillcolor{currentfill}%
\pgfsetlinewidth{0.000000pt}%
\definecolor{currentstroke}{rgb}{0.000000,0.000000,0.000000}%
\pgfsetstrokecolor{currentstroke}%
\pgfsetstrokeopacity{0.000000}%
\pgfsetdash{}{0pt}%
\pgfpathmoveto{\pgfqpoint{2.876847in}{0.500000in}}%
\pgfpathlineto{\pgfqpoint{2.909872in}{0.500000in}}%
\pgfpathlineto{\pgfqpoint{2.909872in}{2.769576in}}%
\pgfpathlineto{\pgfqpoint{2.876847in}{2.769576in}}%
\pgfpathlineto{\pgfqpoint{2.876847in}{0.500000in}}%
\pgfpathclose%
\pgfusepath{fill}%
\end{pgfscope}%
\begin{pgfscope}%
\pgfpathrectangle{\pgfqpoint{0.750000in}{0.500000in}}{\pgfqpoint{4.650000in}{3.020000in}}%
\pgfusepath{clip}%
\pgfsetbuttcap%
\pgfsetmiterjoin%
\definecolor{currentfill}{rgb}{1.000000,0.000000,0.000000}%
\pgfsetfillcolor{currentfill}%
\pgfsetlinewidth{0.000000pt}%
\definecolor{currentstroke}{rgb}{0.000000,0.000000,0.000000}%
\pgfsetstrokecolor{currentstroke}%
\pgfsetstrokeopacity{0.000000}%
\pgfsetdash{}{0pt}%
\pgfpathmoveto{\pgfqpoint{2.909872in}{0.500000in}}%
\pgfpathlineto{\pgfqpoint{2.942898in}{0.500000in}}%
\pgfpathlineto{\pgfqpoint{2.942898in}{1.294874in}}%
\pgfpathlineto{\pgfqpoint{2.909872in}{1.294874in}}%
\pgfpathlineto{\pgfqpoint{2.909872in}{0.500000in}}%
\pgfpathclose%
\pgfusepath{fill}%
\end{pgfscope}%
\begin{pgfscope}%
\pgfpathrectangle{\pgfqpoint{0.750000in}{0.500000in}}{\pgfqpoint{4.650000in}{3.020000in}}%
\pgfusepath{clip}%
\pgfsetbuttcap%
\pgfsetmiterjoin%
\definecolor{currentfill}{rgb}{1.000000,0.000000,0.000000}%
\pgfsetfillcolor{currentfill}%
\pgfsetlinewidth{0.000000pt}%
\definecolor{currentstroke}{rgb}{0.000000,0.000000,0.000000}%
\pgfsetstrokecolor{currentstroke}%
\pgfsetstrokeopacity{0.000000}%
\pgfsetdash{}{0pt}%
\pgfpathmoveto{\pgfqpoint{2.942898in}{0.500000in}}%
\pgfpathlineto{\pgfqpoint{2.975923in}{0.500000in}}%
\pgfpathlineto{\pgfqpoint{2.975923in}{0.897437in}}%
\pgfpathlineto{\pgfqpoint{2.942898in}{0.897437in}}%
\pgfpathlineto{\pgfqpoint{2.942898in}{0.500000in}}%
\pgfpathclose%
\pgfusepath{fill}%
\end{pgfscope}%
\begin{pgfscope}%
\pgfpathrectangle{\pgfqpoint{0.750000in}{0.500000in}}{\pgfqpoint{4.650000in}{3.020000in}}%
\pgfusepath{clip}%
\pgfsetbuttcap%
\pgfsetmiterjoin%
\definecolor{currentfill}{rgb}{1.000000,0.000000,0.000000}%
\pgfsetfillcolor{currentfill}%
\pgfsetlinewidth{0.000000pt}%
\definecolor{currentstroke}{rgb}{0.000000,0.000000,0.000000}%
\pgfsetstrokecolor{currentstroke}%
\pgfsetstrokeopacity{0.000000}%
\pgfsetdash{}{0pt}%
\pgfpathmoveto{\pgfqpoint{2.975923in}{0.500000in}}%
\pgfpathlineto{\pgfqpoint{3.008949in}{0.500000in}}%
\pgfpathlineto{\pgfqpoint{3.008949in}{3.020589in}}%
\pgfpathlineto{\pgfqpoint{2.975923in}{3.020589in}}%
\pgfpathlineto{\pgfqpoint{2.975923in}{0.500000in}}%
\pgfpathclose%
\pgfusepath{fill}%
\end{pgfscope}%
\begin{pgfscope}%
\pgfpathrectangle{\pgfqpoint{0.750000in}{0.500000in}}{\pgfqpoint{4.650000in}{3.020000in}}%
\pgfusepath{clip}%
\pgfsetbuttcap%
\pgfsetmiterjoin%
\definecolor{currentfill}{rgb}{1.000000,0.000000,0.000000}%
\pgfsetfillcolor{currentfill}%
\pgfsetlinewidth{0.000000pt}%
\definecolor{currentstroke}{rgb}{0.000000,0.000000,0.000000}%
\pgfsetstrokecolor{currentstroke}%
\pgfsetstrokeopacity{0.000000}%
\pgfsetdash{}{0pt}%
\pgfpathmoveto{\pgfqpoint{3.008949in}{0.500000in}}%
\pgfpathlineto{\pgfqpoint{3.041974in}{0.500000in}}%
\pgfpathlineto{\pgfqpoint{3.041974in}{1.002026in}}%
\pgfpathlineto{\pgfqpoint{3.008949in}{1.002026in}}%
\pgfpathlineto{\pgfqpoint{3.008949in}{0.500000in}}%
\pgfpathclose%
\pgfusepath{fill}%
\end{pgfscope}%
\begin{pgfscope}%
\pgfpathrectangle{\pgfqpoint{0.750000in}{0.500000in}}{\pgfqpoint{4.650000in}{3.020000in}}%
\pgfusepath{clip}%
\pgfsetbuttcap%
\pgfsetmiterjoin%
\definecolor{currentfill}{rgb}{1.000000,0.000000,0.000000}%
\pgfsetfillcolor{currentfill}%
\pgfsetlinewidth{0.000000pt}%
\definecolor{currentstroke}{rgb}{0.000000,0.000000,0.000000}%
\pgfsetstrokecolor{currentstroke}%
\pgfsetstrokeopacity{0.000000}%
\pgfsetdash{}{0pt}%
\pgfpathmoveto{\pgfqpoint{3.041974in}{0.500000in}}%
\pgfpathlineto{\pgfqpoint{3.075000in}{0.500000in}}%
\pgfpathlineto{\pgfqpoint{3.075000in}{3.376190in}}%
\pgfpathlineto{\pgfqpoint{3.041974in}{3.376190in}}%
\pgfpathlineto{\pgfqpoint{3.041974in}{0.500000in}}%
\pgfpathclose%
\pgfusepath{fill}%
\end{pgfscope}%
\begin{pgfscope}%
\pgfpathrectangle{\pgfqpoint{0.750000in}{0.500000in}}{\pgfqpoint{4.650000in}{3.020000in}}%
\pgfusepath{clip}%
\pgfsetbuttcap%
\pgfsetmiterjoin%
\definecolor{currentfill}{rgb}{1.000000,0.000000,0.000000}%
\pgfsetfillcolor{currentfill}%
\pgfsetlinewidth{0.000000pt}%
\definecolor{currentstroke}{rgb}{0.000000,0.000000,0.000000}%
\pgfsetstrokecolor{currentstroke}%
\pgfsetstrokeopacity{0.000000}%
\pgfsetdash{}{0pt}%
\pgfpathmoveto{\pgfqpoint{3.075000in}{0.500000in}}%
\pgfpathlineto{\pgfqpoint{3.108026in}{0.500000in}}%
\pgfpathlineto{\pgfqpoint{3.108026in}{1.148450in}}%
\pgfpathlineto{\pgfqpoint{3.075000in}{1.148450in}}%
\pgfpathlineto{\pgfqpoint{3.075000in}{0.500000in}}%
\pgfpathclose%
\pgfusepath{fill}%
\end{pgfscope}%
\begin{pgfscope}%
\pgfpathrectangle{\pgfqpoint{0.750000in}{0.500000in}}{\pgfqpoint{4.650000in}{3.020000in}}%
\pgfusepath{clip}%
\pgfsetbuttcap%
\pgfsetmiterjoin%
\definecolor{currentfill}{rgb}{1.000000,0.000000,0.000000}%
\pgfsetfillcolor{currentfill}%
\pgfsetlinewidth{0.000000pt}%
\definecolor{currentstroke}{rgb}{0.000000,0.000000,0.000000}%
\pgfsetstrokecolor{currentstroke}%
\pgfsetstrokeopacity{0.000000}%
\pgfsetdash{}{0pt}%
\pgfpathmoveto{\pgfqpoint{3.108026in}{0.500000in}}%
\pgfpathlineto{\pgfqpoint{3.141051in}{0.500000in}}%
\pgfpathlineto{\pgfqpoint{3.141051in}{0.981108in}}%
\pgfpathlineto{\pgfqpoint{3.108026in}{0.981108in}}%
\pgfpathlineto{\pgfqpoint{3.108026in}{0.500000in}}%
\pgfpathclose%
\pgfusepath{fill}%
\end{pgfscope}%
\begin{pgfscope}%
\pgfpathrectangle{\pgfqpoint{0.750000in}{0.500000in}}{\pgfqpoint{4.650000in}{3.020000in}}%
\pgfusepath{clip}%
\pgfsetbuttcap%
\pgfsetmiterjoin%
\definecolor{currentfill}{rgb}{1.000000,0.000000,0.000000}%
\pgfsetfillcolor{currentfill}%
\pgfsetlinewidth{0.000000pt}%
\definecolor{currentstroke}{rgb}{0.000000,0.000000,0.000000}%
\pgfsetstrokecolor{currentstroke}%
\pgfsetstrokeopacity{0.000000}%
\pgfsetdash{}{0pt}%
\pgfpathmoveto{\pgfqpoint{3.141051in}{0.500000in}}%
\pgfpathlineto{\pgfqpoint{3.174077in}{0.500000in}}%
\pgfpathlineto{\pgfqpoint{3.174077in}{2.664987in}}%
\pgfpathlineto{\pgfqpoint{3.141051in}{2.664987in}}%
\pgfpathlineto{\pgfqpoint{3.141051in}{0.500000in}}%
\pgfpathclose%
\pgfusepath{fill}%
\end{pgfscope}%
\begin{pgfscope}%
\pgfpathrectangle{\pgfqpoint{0.750000in}{0.500000in}}{\pgfqpoint{4.650000in}{3.020000in}}%
\pgfusepath{clip}%
\pgfsetbuttcap%
\pgfsetmiterjoin%
\definecolor{currentfill}{rgb}{1.000000,0.000000,0.000000}%
\pgfsetfillcolor{currentfill}%
\pgfsetlinewidth{0.000000pt}%
\definecolor{currentstroke}{rgb}{0.000000,0.000000,0.000000}%
\pgfsetstrokecolor{currentstroke}%
\pgfsetstrokeopacity{0.000000}%
\pgfsetdash{}{0pt}%
\pgfpathmoveto{\pgfqpoint{3.174077in}{0.500000in}}%
\pgfpathlineto{\pgfqpoint{3.207102in}{0.500000in}}%
\pgfpathlineto{\pgfqpoint{3.207102in}{0.949732in}}%
\pgfpathlineto{\pgfqpoint{3.174077in}{0.949732in}}%
\pgfpathlineto{\pgfqpoint{3.174077in}{0.500000in}}%
\pgfpathclose%
\pgfusepath{fill}%
\end{pgfscope}%
\begin{pgfscope}%
\pgfpathrectangle{\pgfqpoint{0.750000in}{0.500000in}}{\pgfqpoint{4.650000in}{3.020000in}}%
\pgfusepath{clip}%
\pgfsetbuttcap%
\pgfsetmiterjoin%
\definecolor{currentfill}{rgb}{1.000000,0.000000,0.000000}%
\pgfsetfillcolor{currentfill}%
\pgfsetlinewidth{0.000000pt}%
\definecolor{currentstroke}{rgb}{0.000000,0.000000,0.000000}%
\pgfsetstrokecolor{currentstroke}%
\pgfsetstrokeopacity{0.000000}%
\pgfsetdash{}{0pt}%
\pgfpathmoveto{\pgfqpoint{3.207102in}{0.500000in}}%
\pgfpathlineto{\pgfqpoint{3.240128in}{0.500000in}}%
\pgfpathlineto{\pgfqpoint{3.240128in}{3.072883in}}%
\pgfpathlineto{\pgfqpoint{3.207102in}{3.072883in}}%
\pgfpathlineto{\pgfqpoint{3.207102in}{0.500000in}}%
\pgfpathclose%
\pgfusepath{fill}%
\end{pgfscope}%
\begin{pgfscope}%
\pgfpathrectangle{\pgfqpoint{0.750000in}{0.500000in}}{\pgfqpoint{4.650000in}{3.020000in}}%
\pgfusepath{clip}%
\pgfsetbuttcap%
\pgfsetmiterjoin%
\definecolor{currentfill}{rgb}{1.000000,0.000000,0.000000}%
\pgfsetfillcolor{currentfill}%
\pgfsetlinewidth{0.000000pt}%
\definecolor{currentstroke}{rgb}{0.000000,0.000000,0.000000}%
\pgfsetstrokecolor{currentstroke}%
\pgfsetstrokeopacity{0.000000}%
\pgfsetdash{}{0pt}%
\pgfpathmoveto{\pgfqpoint{3.240128in}{0.500000in}}%
\pgfpathlineto{\pgfqpoint{3.273153in}{0.500000in}}%
\pgfpathlineto{\pgfqpoint{3.273153in}{0.949732in}}%
\pgfpathlineto{\pgfqpoint{3.240128in}{0.949732in}}%
\pgfpathlineto{\pgfqpoint{3.240128in}{0.500000in}}%
\pgfpathclose%
\pgfusepath{fill}%
\end{pgfscope}%
\begin{pgfscope}%
\pgfpathrectangle{\pgfqpoint{0.750000in}{0.500000in}}{\pgfqpoint{4.650000in}{3.020000in}}%
\pgfusepath{clip}%
\pgfsetbuttcap%
\pgfsetmiterjoin%
\definecolor{currentfill}{rgb}{1.000000,0.000000,0.000000}%
\pgfsetfillcolor{currentfill}%
\pgfsetlinewidth{0.000000pt}%
\definecolor{currentstroke}{rgb}{0.000000,0.000000,0.000000}%
\pgfsetstrokecolor{currentstroke}%
\pgfsetstrokeopacity{0.000000}%
\pgfsetdash{}{0pt}%
\pgfpathmoveto{\pgfqpoint{3.273153in}{0.500000in}}%
\pgfpathlineto{\pgfqpoint{3.306179in}{0.500000in}}%
\pgfpathlineto{\pgfqpoint{3.306179in}{0.970649in}}%
\pgfpathlineto{\pgfqpoint{3.273153in}{0.970649in}}%
\pgfpathlineto{\pgfqpoint{3.273153in}{0.500000in}}%
\pgfpathclose%
\pgfusepath{fill}%
\end{pgfscope}%
\begin{pgfscope}%
\pgfpathrectangle{\pgfqpoint{0.750000in}{0.500000in}}{\pgfqpoint{4.650000in}{3.020000in}}%
\pgfusepath{clip}%
\pgfsetbuttcap%
\pgfsetmiterjoin%
\definecolor{currentfill}{rgb}{1.000000,0.000000,0.000000}%
\pgfsetfillcolor{currentfill}%
\pgfsetlinewidth{0.000000pt}%
\definecolor{currentstroke}{rgb}{0.000000,0.000000,0.000000}%
\pgfsetstrokecolor{currentstroke}%
\pgfsetstrokeopacity{0.000000}%
\pgfsetdash{}{0pt}%
\pgfpathmoveto{\pgfqpoint{3.306179in}{0.500000in}}%
\pgfpathlineto{\pgfqpoint{3.339205in}{0.500000in}}%
\pgfpathlineto{\pgfqpoint{3.339205in}{2.476727in}}%
\pgfpathlineto{\pgfqpoint{3.306179in}{2.476727in}}%
\pgfpathlineto{\pgfqpoint{3.306179in}{0.500000in}}%
\pgfpathclose%
\pgfusepath{fill}%
\end{pgfscope}%
\begin{pgfscope}%
\pgfpathrectangle{\pgfqpoint{0.750000in}{0.500000in}}{\pgfqpoint{4.650000in}{3.020000in}}%
\pgfusepath{clip}%
\pgfsetbuttcap%
\pgfsetmiterjoin%
\definecolor{currentfill}{rgb}{1.000000,0.000000,0.000000}%
\pgfsetfillcolor{currentfill}%
\pgfsetlinewidth{0.000000pt}%
\definecolor{currentstroke}{rgb}{0.000000,0.000000,0.000000}%
\pgfsetstrokecolor{currentstroke}%
\pgfsetstrokeopacity{0.000000}%
\pgfsetdash{}{0pt}%
\pgfpathmoveto{\pgfqpoint{3.339205in}{0.500000in}}%
\pgfpathlineto{\pgfqpoint{3.372230in}{0.500000in}}%
\pgfpathlineto{\pgfqpoint{3.372230in}{0.886978in}}%
\pgfpathlineto{\pgfqpoint{3.339205in}{0.886978in}}%
\pgfpathlineto{\pgfqpoint{3.339205in}{0.500000in}}%
\pgfpathclose%
\pgfusepath{fill}%
\end{pgfscope}%
\begin{pgfscope}%
\pgfpathrectangle{\pgfqpoint{0.750000in}{0.500000in}}{\pgfqpoint{4.650000in}{3.020000in}}%
\pgfusepath{clip}%
\pgfsetbuttcap%
\pgfsetmiterjoin%
\definecolor{currentfill}{rgb}{1.000000,0.000000,0.000000}%
\pgfsetfillcolor{currentfill}%
\pgfsetlinewidth{0.000000pt}%
\definecolor{currentstroke}{rgb}{0.000000,0.000000,0.000000}%
\pgfsetstrokecolor{currentstroke}%
\pgfsetstrokeopacity{0.000000}%
\pgfsetdash{}{0pt}%
\pgfpathmoveto{\pgfqpoint{3.372230in}{0.500000in}}%
\pgfpathlineto{\pgfqpoint{3.405256in}{0.500000in}}%
\pgfpathlineto{\pgfqpoint{3.405256in}{2.518563in}}%
\pgfpathlineto{\pgfqpoint{3.372230in}{2.518563in}}%
\pgfpathlineto{\pgfqpoint{3.372230in}{0.500000in}}%
\pgfpathclose%
\pgfusepath{fill}%
\end{pgfscope}%
\begin{pgfscope}%
\pgfpathrectangle{\pgfqpoint{0.750000in}{0.500000in}}{\pgfqpoint{4.650000in}{3.020000in}}%
\pgfusepath{clip}%
\pgfsetbuttcap%
\pgfsetmiterjoin%
\definecolor{currentfill}{rgb}{1.000000,0.000000,0.000000}%
\pgfsetfillcolor{currentfill}%
\pgfsetlinewidth{0.000000pt}%
\definecolor{currentstroke}{rgb}{0.000000,0.000000,0.000000}%
\pgfsetstrokecolor{currentstroke}%
\pgfsetstrokeopacity{0.000000}%
\pgfsetdash{}{0pt}%
\pgfpathmoveto{\pgfqpoint{3.405256in}{0.500000in}}%
\pgfpathlineto{\pgfqpoint{3.438281in}{0.500000in}}%
\pgfpathlineto{\pgfqpoint{3.438281in}{0.845143in}}%
\pgfpathlineto{\pgfqpoint{3.405256in}{0.845143in}}%
\pgfpathlineto{\pgfqpoint{3.405256in}{0.500000in}}%
\pgfpathclose%
\pgfusepath{fill}%
\end{pgfscope}%
\begin{pgfscope}%
\pgfpathrectangle{\pgfqpoint{0.750000in}{0.500000in}}{\pgfqpoint{4.650000in}{3.020000in}}%
\pgfusepath{clip}%
\pgfsetbuttcap%
\pgfsetmiterjoin%
\definecolor{currentfill}{rgb}{1.000000,0.000000,0.000000}%
\pgfsetfillcolor{currentfill}%
\pgfsetlinewidth{0.000000pt}%
\definecolor{currentstroke}{rgb}{0.000000,0.000000,0.000000}%
\pgfsetstrokecolor{currentstroke}%
\pgfsetstrokeopacity{0.000000}%
\pgfsetdash{}{0pt}%
\pgfpathmoveto{\pgfqpoint{3.438281in}{0.500000in}}%
\pgfpathlineto{\pgfqpoint{3.471307in}{0.500000in}}%
\pgfpathlineto{\pgfqpoint{3.471307in}{0.813766in}}%
\pgfpathlineto{\pgfqpoint{3.438281in}{0.813766in}}%
\pgfpathlineto{\pgfqpoint{3.438281in}{0.500000in}}%
\pgfpathclose%
\pgfusepath{fill}%
\end{pgfscope}%
\begin{pgfscope}%
\pgfpathrectangle{\pgfqpoint{0.750000in}{0.500000in}}{\pgfqpoint{4.650000in}{3.020000in}}%
\pgfusepath{clip}%
\pgfsetbuttcap%
\pgfsetmiterjoin%
\definecolor{currentfill}{rgb}{1.000000,0.000000,0.000000}%
\pgfsetfillcolor{currentfill}%
\pgfsetlinewidth{0.000000pt}%
\definecolor{currentstroke}{rgb}{0.000000,0.000000,0.000000}%
\pgfsetstrokecolor{currentstroke}%
\pgfsetstrokeopacity{0.000000}%
\pgfsetdash{}{0pt}%
\pgfpathmoveto{\pgfqpoint{3.471307in}{0.500000in}}%
\pgfpathlineto{\pgfqpoint{3.504332in}{0.500000in}}%
\pgfpathlineto{\pgfqpoint{3.504332in}{2.142043in}}%
\pgfpathlineto{\pgfqpoint{3.471307in}{2.142043in}}%
\pgfpathlineto{\pgfqpoint{3.471307in}{0.500000in}}%
\pgfpathclose%
\pgfusepath{fill}%
\end{pgfscope}%
\begin{pgfscope}%
\pgfpathrectangle{\pgfqpoint{0.750000in}{0.500000in}}{\pgfqpoint{4.650000in}{3.020000in}}%
\pgfusepath{clip}%
\pgfsetbuttcap%
\pgfsetmiterjoin%
\definecolor{currentfill}{rgb}{1.000000,0.000000,0.000000}%
\pgfsetfillcolor{currentfill}%
\pgfsetlinewidth{0.000000pt}%
\definecolor{currentstroke}{rgb}{0.000000,0.000000,0.000000}%
\pgfsetstrokecolor{currentstroke}%
\pgfsetstrokeopacity{0.000000}%
\pgfsetdash{}{0pt}%
\pgfpathmoveto{\pgfqpoint{3.504332in}{0.500000in}}%
\pgfpathlineto{\pgfqpoint{3.537358in}{0.500000in}}%
\pgfpathlineto{\pgfqpoint{3.537358in}{0.803307in}}%
\pgfpathlineto{\pgfqpoint{3.504332in}{0.803307in}}%
\pgfpathlineto{\pgfqpoint{3.504332in}{0.500000in}}%
\pgfpathclose%
\pgfusepath{fill}%
\end{pgfscope}%
\begin{pgfscope}%
\pgfpathrectangle{\pgfqpoint{0.750000in}{0.500000in}}{\pgfqpoint{4.650000in}{3.020000in}}%
\pgfusepath{clip}%
\pgfsetbuttcap%
\pgfsetmiterjoin%
\definecolor{currentfill}{rgb}{1.000000,0.000000,0.000000}%
\pgfsetfillcolor{currentfill}%
\pgfsetlinewidth{0.000000pt}%
\definecolor{currentstroke}{rgb}{0.000000,0.000000,0.000000}%
\pgfsetstrokecolor{currentstroke}%
\pgfsetstrokeopacity{0.000000}%
\pgfsetdash{}{0pt}%
\pgfpathmoveto{\pgfqpoint{3.537358in}{0.500000in}}%
\pgfpathlineto{\pgfqpoint{3.570384in}{0.500000in}}%
\pgfpathlineto{\pgfqpoint{3.570384in}{2.298926in}}%
\pgfpathlineto{\pgfqpoint{3.537358in}{2.298926in}}%
\pgfpathlineto{\pgfqpoint{3.537358in}{0.500000in}}%
\pgfpathclose%
\pgfusepath{fill}%
\end{pgfscope}%
\begin{pgfscope}%
\pgfpathrectangle{\pgfqpoint{0.750000in}{0.500000in}}{\pgfqpoint{4.650000in}{3.020000in}}%
\pgfusepath{clip}%
\pgfsetbuttcap%
\pgfsetmiterjoin%
\definecolor{currentfill}{rgb}{1.000000,0.000000,0.000000}%
\pgfsetfillcolor{currentfill}%
\pgfsetlinewidth{0.000000pt}%
\definecolor{currentstroke}{rgb}{0.000000,0.000000,0.000000}%
\pgfsetstrokecolor{currentstroke}%
\pgfsetstrokeopacity{0.000000}%
\pgfsetdash{}{0pt}%
\pgfpathmoveto{\pgfqpoint{3.570384in}{0.500000in}}%
\pgfpathlineto{\pgfqpoint{3.603409in}{0.500000in}}%
\pgfpathlineto{\pgfqpoint{3.603409in}{0.792848in}}%
\pgfpathlineto{\pgfqpoint{3.570384in}{0.792848in}}%
\pgfpathlineto{\pgfqpoint{3.570384in}{0.500000in}}%
\pgfpathclose%
\pgfusepath{fill}%
\end{pgfscope}%
\begin{pgfscope}%
\pgfpathrectangle{\pgfqpoint{0.750000in}{0.500000in}}{\pgfqpoint{4.650000in}{3.020000in}}%
\pgfusepath{clip}%
\pgfsetbuttcap%
\pgfsetmiterjoin%
\definecolor{currentfill}{rgb}{1.000000,0.000000,0.000000}%
\pgfsetfillcolor{currentfill}%
\pgfsetlinewidth{0.000000pt}%
\definecolor{currentstroke}{rgb}{0.000000,0.000000,0.000000}%
\pgfsetstrokecolor{currentstroke}%
\pgfsetstrokeopacity{0.000000}%
\pgfsetdash{}{0pt}%
\pgfpathmoveto{\pgfqpoint{3.603409in}{0.500000in}}%
\pgfpathlineto{\pgfqpoint{3.636435in}{0.500000in}}%
\pgfpathlineto{\pgfqpoint{3.636435in}{0.730095in}}%
\pgfpathlineto{\pgfqpoint{3.603409in}{0.730095in}}%
\pgfpathlineto{\pgfqpoint{3.603409in}{0.500000in}}%
\pgfpathclose%
\pgfusepath{fill}%
\end{pgfscope}%
\begin{pgfscope}%
\pgfpathrectangle{\pgfqpoint{0.750000in}{0.500000in}}{\pgfqpoint{4.650000in}{3.020000in}}%
\pgfusepath{clip}%
\pgfsetbuttcap%
\pgfsetmiterjoin%
\definecolor{currentfill}{rgb}{1.000000,0.000000,0.000000}%
\pgfsetfillcolor{currentfill}%
\pgfsetlinewidth{0.000000pt}%
\definecolor{currentstroke}{rgb}{0.000000,0.000000,0.000000}%
\pgfsetstrokecolor{currentstroke}%
\pgfsetstrokeopacity{0.000000}%
\pgfsetdash{}{0pt}%
\pgfpathmoveto{\pgfqpoint{3.636435in}{0.500000in}}%
\pgfpathlineto{\pgfqpoint{3.669460in}{0.500000in}}%
\pgfpathlineto{\pgfqpoint{3.669460in}{1.838736in}}%
\pgfpathlineto{\pgfqpoint{3.636435in}{1.838736in}}%
\pgfpathlineto{\pgfqpoint{3.636435in}{0.500000in}}%
\pgfpathclose%
\pgfusepath{fill}%
\end{pgfscope}%
\begin{pgfscope}%
\pgfpathrectangle{\pgfqpoint{0.750000in}{0.500000in}}{\pgfqpoint{4.650000in}{3.020000in}}%
\pgfusepath{clip}%
\pgfsetbuttcap%
\pgfsetmiterjoin%
\definecolor{currentfill}{rgb}{1.000000,0.000000,0.000000}%
\pgfsetfillcolor{currentfill}%
\pgfsetlinewidth{0.000000pt}%
\definecolor{currentstroke}{rgb}{0.000000,0.000000,0.000000}%
\pgfsetstrokecolor{currentstroke}%
\pgfsetstrokeopacity{0.000000}%
\pgfsetdash{}{0pt}%
\pgfpathmoveto{\pgfqpoint{3.669460in}{0.500000in}}%
\pgfpathlineto{\pgfqpoint{3.702486in}{0.500000in}}%
\pgfpathlineto{\pgfqpoint{3.702486in}{0.625506in}}%
\pgfpathlineto{\pgfqpoint{3.669460in}{0.625506in}}%
\pgfpathlineto{\pgfqpoint{3.669460in}{0.500000in}}%
\pgfpathclose%
\pgfusepath{fill}%
\end{pgfscope}%
\begin{pgfscope}%
\pgfpathrectangle{\pgfqpoint{0.750000in}{0.500000in}}{\pgfqpoint{4.650000in}{3.020000in}}%
\pgfusepath{clip}%
\pgfsetbuttcap%
\pgfsetmiterjoin%
\definecolor{currentfill}{rgb}{1.000000,0.000000,0.000000}%
\pgfsetfillcolor{currentfill}%
\pgfsetlinewidth{0.000000pt}%
\definecolor{currentstroke}{rgb}{0.000000,0.000000,0.000000}%
\pgfsetstrokecolor{currentstroke}%
\pgfsetstrokeopacity{0.000000}%
\pgfsetdash{}{0pt}%
\pgfpathmoveto{\pgfqpoint{3.702486in}{0.500000in}}%
\pgfpathlineto{\pgfqpoint{3.735511in}{0.500000in}}%
\pgfpathlineto{\pgfqpoint{3.735511in}{1.671394in}}%
\pgfpathlineto{\pgfqpoint{3.702486in}{1.671394in}}%
\pgfpathlineto{\pgfqpoint{3.702486in}{0.500000in}}%
\pgfpathclose%
\pgfusepath{fill}%
\end{pgfscope}%
\begin{pgfscope}%
\pgfpathrectangle{\pgfqpoint{0.750000in}{0.500000in}}{\pgfqpoint{4.650000in}{3.020000in}}%
\pgfusepath{clip}%
\pgfsetbuttcap%
\pgfsetmiterjoin%
\definecolor{currentfill}{rgb}{1.000000,0.000000,0.000000}%
\pgfsetfillcolor{currentfill}%
\pgfsetlinewidth{0.000000pt}%
\definecolor{currentstroke}{rgb}{0.000000,0.000000,0.000000}%
\pgfsetstrokecolor{currentstroke}%
\pgfsetstrokeopacity{0.000000}%
\pgfsetdash{}{0pt}%
\pgfpathmoveto{\pgfqpoint{3.735511in}{0.500000in}}%
\pgfpathlineto{\pgfqpoint{3.768537in}{0.500000in}}%
\pgfpathlineto{\pgfqpoint{3.768537in}{0.635965in}}%
\pgfpathlineto{\pgfqpoint{3.735511in}{0.635965in}}%
\pgfpathlineto{\pgfqpoint{3.735511in}{0.500000in}}%
\pgfpathclose%
\pgfusepath{fill}%
\end{pgfscope}%
\begin{pgfscope}%
\pgfpathrectangle{\pgfqpoint{0.750000in}{0.500000in}}{\pgfqpoint{4.650000in}{3.020000in}}%
\pgfusepath{clip}%
\pgfsetbuttcap%
\pgfsetmiterjoin%
\definecolor{currentfill}{rgb}{1.000000,0.000000,0.000000}%
\pgfsetfillcolor{currentfill}%
\pgfsetlinewidth{0.000000pt}%
\definecolor{currentstroke}{rgb}{0.000000,0.000000,0.000000}%
\pgfsetstrokecolor{currentstroke}%
\pgfsetstrokeopacity{0.000000}%
\pgfsetdash{}{0pt}%
\pgfpathmoveto{\pgfqpoint{3.768537in}{0.500000in}}%
\pgfpathlineto{\pgfqpoint{3.801563in}{0.500000in}}%
\pgfpathlineto{\pgfqpoint{3.801563in}{0.656883in}}%
\pgfpathlineto{\pgfqpoint{3.768537in}{0.656883in}}%
\pgfpathlineto{\pgfqpoint{3.768537in}{0.500000in}}%
\pgfpathclose%
\pgfusepath{fill}%
\end{pgfscope}%
\begin{pgfscope}%
\pgfpathrectangle{\pgfqpoint{0.750000in}{0.500000in}}{\pgfqpoint{4.650000in}{3.020000in}}%
\pgfusepath{clip}%
\pgfsetbuttcap%
\pgfsetmiterjoin%
\definecolor{currentfill}{rgb}{1.000000,0.000000,0.000000}%
\pgfsetfillcolor{currentfill}%
\pgfsetlinewidth{0.000000pt}%
\definecolor{currentstroke}{rgb}{0.000000,0.000000,0.000000}%
\pgfsetstrokecolor{currentstroke}%
\pgfsetstrokeopacity{0.000000}%
\pgfsetdash{}{0pt}%
\pgfpathmoveto{\pgfqpoint{3.801562in}{0.500000in}}%
\pgfpathlineto{\pgfqpoint{3.834588in}{0.500000in}}%
\pgfpathlineto{\pgfqpoint{3.834588in}{1.472675in}}%
\pgfpathlineto{\pgfqpoint{3.801562in}{1.472675in}}%
\pgfpathlineto{\pgfqpoint{3.801562in}{0.500000in}}%
\pgfpathclose%
\pgfusepath{fill}%
\end{pgfscope}%
\begin{pgfscope}%
\pgfpathrectangle{\pgfqpoint{0.750000in}{0.500000in}}{\pgfqpoint{4.650000in}{3.020000in}}%
\pgfusepath{clip}%
\pgfsetbuttcap%
\pgfsetmiterjoin%
\definecolor{currentfill}{rgb}{1.000000,0.000000,0.000000}%
\pgfsetfillcolor{currentfill}%
\pgfsetlinewidth{0.000000pt}%
\definecolor{currentstroke}{rgb}{0.000000,0.000000,0.000000}%
\pgfsetstrokecolor{currentstroke}%
\pgfsetstrokeopacity{0.000000}%
\pgfsetdash{}{0pt}%
\pgfpathmoveto{\pgfqpoint{3.834588in}{0.500000in}}%
\pgfpathlineto{\pgfqpoint{3.867614in}{0.500000in}}%
\pgfpathlineto{\pgfqpoint{3.867614in}{0.615048in}}%
\pgfpathlineto{\pgfqpoint{3.834588in}{0.615048in}}%
\pgfpathlineto{\pgfqpoint{3.834588in}{0.500000in}}%
\pgfpathclose%
\pgfusepath{fill}%
\end{pgfscope}%
\begin{pgfscope}%
\pgfpathrectangle{\pgfqpoint{0.750000in}{0.500000in}}{\pgfqpoint{4.650000in}{3.020000in}}%
\pgfusepath{clip}%
\pgfsetbuttcap%
\pgfsetmiterjoin%
\definecolor{currentfill}{rgb}{1.000000,0.000000,0.000000}%
\pgfsetfillcolor{currentfill}%
\pgfsetlinewidth{0.000000pt}%
\definecolor{currentstroke}{rgb}{0.000000,0.000000,0.000000}%
\pgfsetstrokecolor{currentstroke}%
\pgfsetstrokeopacity{0.000000}%
\pgfsetdash{}{0pt}%
\pgfpathmoveto{\pgfqpoint{3.867614in}{0.500000in}}%
\pgfpathlineto{\pgfqpoint{3.900639in}{0.500000in}}%
\pgfpathlineto{\pgfqpoint{3.900639in}{1.420381in}}%
\pgfpathlineto{\pgfqpoint{3.867614in}{1.420381in}}%
\pgfpathlineto{\pgfqpoint{3.867614in}{0.500000in}}%
\pgfpathclose%
\pgfusepath{fill}%
\end{pgfscope}%
\begin{pgfscope}%
\pgfpathrectangle{\pgfqpoint{0.750000in}{0.500000in}}{\pgfqpoint{4.650000in}{3.020000in}}%
\pgfusepath{clip}%
\pgfsetbuttcap%
\pgfsetmiterjoin%
\definecolor{currentfill}{rgb}{1.000000,0.000000,0.000000}%
\pgfsetfillcolor{currentfill}%
\pgfsetlinewidth{0.000000pt}%
\definecolor{currentstroke}{rgb}{0.000000,0.000000,0.000000}%
\pgfsetstrokecolor{currentstroke}%
\pgfsetstrokeopacity{0.000000}%
\pgfsetdash{}{0pt}%
\pgfpathmoveto{\pgfqpoint{3.900639in}{0.500000in}}%
\pgfpathlineto{\pgfqpoint{3.933665in}{0.500000in}}%
\pgfpathlineto{\pgfqpoint{3.933665in}{0.552294in}}%
\pgfpathlineto{\pgfqpoint{3.900639in}{0.552294in}}%
\pgfpathlineto{\pgfqpoint{3.900639in}{0.500000in}}%
\pgfpathclose%
\pgfusepath{fill}%
\end{pgfscope}%
\begin{pgfscope}%
\pgfpathrectangle{\pgfqpoint{0.750000in}{0.500000in}}{\pgfqpoint{4.650000in}{3.020000in}}%
\pgfusepath{clip}%
\pgfsetbuttcap%
\pgfsetmiterjoin%
\definecolor{currentfill}{rgb}{1.000000,0.000000,0.000000}%
\pgfsetfillcolor{currentfill}%
\pgfsetlinewidth{0.000000pt}%
\definecolor{currentstroke}{rgb}{0.000000,0.000000,0.000000}%
\pgfsetstrokecolor{currentstroke}%
\pgfsetstrokeopacity{0.000000}%
\pgfsetdash{}{0pt}%
\pgfpathmoveto{\pgfqpoint{3.933665in}{0.500000in}}%
\pgfpathlineto{\pgfqpoint{3.966690in}{0.500000in}}%
\pgfpathlineto{\pgfqpoint{3.966690in}{0.719636in}}%
\pgfpathlineto{\pgfqpoint{3.933665in}{0.719636in}}%
\pgfpathlineto{\pgfqpoint{3.933665in}{0.500000in}}%
\pgfpathclose%
\pgfusepath{fill}%
\end{pgfscope}%
\begin{pgfscope}%
\pgfpathrectangle{\pgfqpoint{0.750000in}{0.500000in}}{\pgfqpoint{4.650000in}{3.020000in}}%
\pgfusepath{clip}%
\pgfsetbuttcap%
\pgfsetmiterjoin%
\definecolor{currentfill}{rgb}{1.000000,0.000000,0.000000}%
\pgfsetfillcolor{currentfill}%
\pgfsetlinewidth{0.000000pt}%
\definecolor{currentstroke}{rgb}{0.000000,0.000000,0.000000}%
\pgfsetstrokecolor{currentstroke}%
\pgfsetstrokeopacity{0.000000}%
\pgfsetdash{}{0pt}%
\pgfpathmoveto{\pgfqpoint{3.966690in}{0.500000in}}%
\pgfpathlineto{\pgfqpoint{3.999716in}{0.500000in}}%
\pgfpathlineto{\pgfqpoint{3.999716in}{0.918355in}}%
\pgfpathlineto{\pgfqpoint{3.966690in}{0.918355in}}%
\pgfpathlineto{\pgfqpoint{3.966690in}{0.500000in}}%
\pgfpathclose%
\pgfusepath{fill}%
\end{pgfscope}%
\begin{pgfscope}%
\pgfpathrectangle{\pgfqpoint{0.750000in}{0.500000in}}{\pgfqpoint{4.650000in}{3.020000in}}%
\pgfusepath{clip}%
\pgfsetbuttcap%
\pgfsetmiterjoin%
\definecolor{currentfill}{rgb}{1.000000,0.000000,0.000000}%
\pgfsetfillcolor{currentfill}%
\pgfsetlinewidth{0.000000pt}%
\definecolor{currentstroke}{rgb}{0.000000,0.000000,0.000000}%
\pgfsetstrokecolor{currentstroke}%
\pgfsetstrokeopacity{0.000000}%
\pgfsetdash{}{0pt}%
\pgfpathmoveto{\pgfqpoint{3.999716in}{0.500000in}}%
\pgfpathlineto{\pgfqpoint{4.032741in}{0.500000in}}%
\pgfpathlineto{\pgfqpoint{4.032741in}{0.541835in}}%
\pgfpathlineto{\pgfqpoint{3.999716in}{0.541835in}}%
\pgfpathlineto{\pgfqpoint{3.999716in}{0.500000in}}%
\pgfpathclose%
\pgfusepath{fill}%
\end{pgfscope}%
\begin{pgfscope}%
\pgfpathrectangle{\pgfqpoint{0.750000in}{0.500000in}}{\pgfqpoint{4.650000in}{3.020000in}}%
\pgfusepath{clip}%
\pgfsetbuttcap%
\pgfsetmiterjoin%
\definecolor{currentfill}{rgb}{1.000000,0.000000,0.000000}%
\pgfsetfillcolor{currentfill}%
\pgfsetlinewidth{0.000000pt}%
\definecolor{currentstroke}{rgb}{0.000000,0.000000,0.000000}%
\pgfsetstrokecolor{currentstroke}%
\pgfsetstrokeopacity{0.000000}%
\pgfsetdash{}{0pt}%
\pgfpathmoveto{\pgfqpoint{4.032741in}{0.500000in}}%
\pgfpathlineto{\pgfqpoint{4.065767in}{0.500000in}}%
\pgfpathlineto{\pgfqpoint{4.065767in}{1.158909in}}%
\pgfpathlineto{\pgfqpoint{4.032741in}{1.158909in}}%
\pgfpathlineto{\pgfqpoint{4.032741in}{0.500000in}}%
\pgfpathclose%
\pgfusepath{fill}%
\end{pgfscope}%
\begin{pgfscope}%
\pgfpathrectangle{\pgfqpoint{0.750000in}{0.500000in}}{\pgfqpoint{4.650000in}{3.020000in}}%
\pgfusepath{clip}%
\pgfsetbuttcap%
\pgfsetmiterjoin%
\definecolor{currentfill}{rgb}{1.000000,0.000000,0.000000}%
\pgfsetfillcolor{currentfill}%
\pgfsetlinewidth{0.000000pt}%
\definecolor{currentstroke}{rgb}{0.000000,0.000000,0.000000}%
\pgfsetstrokecolor{currentstroke}%
\pgfsetstrokeopacity{0.000000}%
\pgfsetdash{}{0pt}%
\pgfpathmoveto{\pgfqpoint{4.065767in}{0.500000in}}%
\pgfpathlineto{\pgfqpoint{4.098793in}{0.500000in}}%
\pgfpathlineto{\pgfqpoint{4.098793in}{0.562753in}}%
\pgfpathlineto{\pgfqpoint{4.065767in}{0.562753in}}%
\pgfpathlineto{\pgfqpoint{4.065767in}{0.500000in}}%
\pgfpathclose%
\pgfusepath{fill}%
\end{pgfscope}%
\begin{pgfscope}%
\pgfpathrectangle{\pgfqpoint{0.750000in}{0.500000in}}{\pgfqpoint{4.650000in}{3.020000in}}%
\pgfusepath{clip}%
\pgfsetbuttcap%
\pgfsetmiterjoin%
\definecolor{currentfill}{rgb}{1.000000,0.000000,0.000000}%
\pgfsetfillcolor{currentfill}%
\pgfsetlinewidth{0.000000pt}%
\definecolor{currentstroke}{rgb}{0.000000,0.000000,0.000000}%
\pgfsetstrokecolor{currentstroke}%
\pgfsetstrokeopacity{0.000000}%
\pgfsetdash{}{0pt}%
\pgfpathmoveto{\pgfqpoint{4.098793in}{0.500000in}}%
\pgfpathlineto{\pgfqpoint{4.131818in}{0.500000in}}%
\pgfpathlineto{\pgfqpoint{4.131818in}{0.897437in}}%
\pgfpathlineto{\pgfqpoint{4.098793in}{0.897437in}}%
\pgfpathlineto{\pgfqpoint{4.098793in}{0.500000in}}%
\pgfpathclose%
\pgfusepath{fill}%
\end{pgfscope}%
\begin{pgfscope}%
\pgfpathrectangle{\pgfqpoint{0.750000in}{0.500000in}}{\pgfqpoint{4.650000in}{3.020000in}}%
\pgfusepath{clip}%
\pgfsetbuttcap%
\pgfsetmiterjoin%
\definecolor{currentfill}{rgb}{1.000000,0.000000,0.000000}%
\pgfsetfillcolor{currentfill}%
\pgfsetlinewidth{0.000000pt}%
\definecolor{currentstroke}{rgb}{0.000000,0.000000,0.000000}%
\pgfsetstrokecolor{currentstroke}%
\pgfsetstrokeopacity{0.000000}%
\pgfsetdash{}{0pt}%
\pgfpathmoveto{\pgfqpoint{4.131818in}{0.500000in}}%
\pgfpathlineto{\pgfqpoint{4.164844in}{0.500000in}}%
\pgfpathlineto{\pgfqpoint{4.164844in}{0.520918in}}%
\pgfpathlineto{\pgfqpoint{4.131818in}{0.520918in}}%
\pgfpathlineto{\pgfqpoint{4.131818in}{0.500000in}}%
\pgfpathclose%
\pgfusepath{fill}%
\end{pgfscope}%
\begin{pgfscope}%
\pgfpathrectangle{\pgfqpoint{0.750000in}{0.500000in}}{\pgfqpoint{4.650000in}{3.020000in}}%
\pgfusepath{clip}%
\pgfsetbuttcap%
\pgfsetmiterjoin%
\definecolor{currentfill}{rgb}{1.000000,0.000000,0.000000}%
\pgfsetfillcolor{currentfill}%
\pgfsetlinewidth{0.000000pt}%
\definecolor{currentstroke}{rgb}{0.000000,0.000000,0.000000}%
\pgfsetstrokecolor{currentstroke}%
\pgfsetstrokeopacity{0.000000}%
\pgfsetdash{}{0pt}%
\pgfpathmoveto{\pgfqpoint{4.164844in}{0.500000in}}%
\pgfpathlineto{\pgfqpoint{4.197869in}{0.500000in}}%
\pgfpathlineto{\pgfqpoint{4.197869in}{0.541835in}}%
\pgfpathlineto{\pgfqpoint{4.164844in}{0.541835in}}%
\pgfpathlineto{\pgfqpoint{4.164844in}{0.500000in}}%
\pgfpathclose%
\pgfusepath{fill}%
\end{pgfscope}%
\begin{pgfscope}%
\pgfpathrectangle{\pgfqpoint{0.750000in}{0.500000in}}{\pgfqpoint{4.650000in}{3.020000in}}%
\pgfusepath{clip}%
\pgfsetbuttcap%
\pgfsetmiterjoin%
\definecolor{currentfill}{rgb}{1.000000,0.000000,0.000000}%
\pgfsetfillcolor{currentfill}%
\pgfsetlinewidth{0.000000pt}%
\definecolor{currentstroke}{rgb}{0.000000,0.000000,0.000000}%
\pgfsetstrokecolor{currentstroke}%
\pgfsetstrokeopacity{0.000000}%
\pgfsetdash{}{0pt}%
\pgfpathmoveto{\pgfqpoint{4.197869in}{0.500000in}}%
\pgfpathlineto{\pgfqpoint{4.230895in}{0.500000in}}%
\pgfpathlineto{\pgfqpoint{4.230895in}{0.939273in}}%
\pgfpathlineto{\pgfqpoint{4.197869in}{0.939273in}}%
\pgfpathlineto{\pgfqpoint{4.197869in}{0.500000in}}%
\pgfpathclose%
\pgfusepath{fill}%
\end{pgfscope}%
\begin{pgfscope}%
\pgfpathrectangle{\pgfqpoint{0.750000in}{0.500000in}}{\pgfqpoint{4.650000in}{3.020000in}}%
\pgfusepath{clip}%
\pgfsetbuttcap%
\pgfsetmiterjoin%
\definecolor{currentfill}{rgb}{1.000000,0.000000,0.000000}%
\pgfsetfillcolor{currentfill}%
\pgfsetlinewidth{0.000000pt}%
\definecolor{currentstroke}{rgb}{0.000000,0.000000,0.000000}%
\pgfsetstrokecolor{currentstroke}%
\pgfsetstrokeopacity{0.000000}%
\pgfsetdash{}{0pt}%
\pgfpathmoveto{\pgfqpoint{4.230895in}{0.500000in}}%
\pgfpathlineto{\pgfqpoint{4.263920in}{0.500000in}}%
\pgfpathlineto{\pgfqpoint{4.263920in}{0.531377in}}%
\pgfpathlineto{\pgfqpoint{4.230895in}{0.531377in}}%
\pgfpathlineto{\pgfqpoint{4.230895in}{0.500000in}}%
\pgfpathclose%
\pgfusepath{fill}%
\end{pgfscope}%
\begin{pgfscope}%
\pgfpathrectangle{\pgfqpoint{0.750000in}{0.500000in}}{\pgfqpoint{4.650000in}{3.020000in}}%
\pgfusepath{clip}%
\pgfsetbuttcap%
\pgfsetmiterjoin%
\definecolor{currentfill}{rgb}{1.000000,0.000000,0.000000}%
\pgfsetfillcolor{currentfill}%
\pgfsetlinewidth{0.000000pt}%
\definecolor{currentstroke}{rgb}{0.000000,0.000000,0.000000}%
\pgfsetstrokecolor{currentstroke}%
\pgfsetstrokeopacity{0.000000}%
\pgfsetdash{}{0pt}%
\pgfpathmoveto{\pgfqpoint{4.263920in}{0.500000in}}%
\pgfpathlineto{\pgfqpoint{4.296946in}{0.500000in}}%
\pgfpathlineto{\pgfqpoint{4.296946in}{0.646424in}}%
\pgfpathlineto{\pgfqpoint{4.263920in}{0.646424in}}%
\pgfpathlineto{\pgfqpoint{4.263920in}{0.500000in}}%
\pgfpathclose%
\pgfusepath{fill}%
\end{pgfscope}%
\begin{pgfscope}%
\pgfpathrectangle{\pgfqpoint{0.750000in}{0.500000in}}{\pgfqpoint{4.650000in}{3.020000in}}%
\pgfusepath{clip}%
\pgfsetbuttcap%
\pgfsetmiterjoin%
\definecolor{currentfill}{rgb}{1.000000,0.000000,0.000000}%
\pgfsetfillcolor{currentfill}%
\pgfsetlinewidth{0.000000pt}%
\definecolor{currentstroke}{rgb}{0.000000,0.000000,0.000000}%
\pgfsetstrokecolor{currentstroke}%
\pgfsetstrokeopacity{0.000000}%
\pgfsetdash{}{0pt}%
\pgfpathmoveto{\pgfqpoint{4.296946in}{0.500000in}}%
\pgfpathlineto{\pgfqpoint{4.329972in}{0.500000in}}%
\pgfpathlineto{\pgfqpoint{4.329972in}{0.520918in}}%
\pgfpathlineto{\pgfqpoint{4.296946in}{0.520918in}}%
\pgfpathlineto{\pgfqpoint{4.296946in}{0.500000in}}%
\pgfpathclose%
\pgfusepath{fill}%
\end{pgfscope}%
\begin{pgfscope}%
\pgfpathrectangle{\pgfqpoint{0.750000in}{0.500000in}}{\pgfqpoint{4.650000in}{3.020000in}}%
\pgfusepath{clip}%
\pgfsetbuttcap%
\pgfsetmiterjoin%
\definecolor{currentfill}{rgb}{1.000000,0.000000,0.000000}%
\pgfsetfillcolor{currentfill}%
\pgfsetlinewidth{0.000000pt}%
\definecolor{currentstroke}{rgb}{0.000000,0.000000,0.000000}%
\pgfsetstrokecolor{currentstroke}%
\pgfsetstrokeopacity{0.000000}%
\pgfsetdash{}{0pt}%
\pgfpathmoveto{\pgfqpoint{4.329972in}{0.500000in}}%
\pgfpathlineto{\pgfqpoint{4.362997in}{0.500000in}}%
\pgfpathlineto{\pgfqpoint{4.362997in}{0.500000in}}%
\pgfpathlineto{\pgfqpoint{4.329972in}{0.500000in}}%
\pgfpathlineto{\pgfqpoint{4.329972in}{0.500000in}}%
\pgfpathclose%
\pgfusepath{fill}%
\end{pgfscope}%
\begin{pgfscope}%
\pgfpathrectangle{\pgfqpoint{0.750000in}{0.500000in}}{\pgfqpoint{4.650000in}{3.020000in}}%
\pgfusepath{clip}%
\pgfsetbuttcap%
\pgfsetmiterjoin%
\definecolor{currentfill}{rgb}{1.000000,0.000000,0.000000}%
\pgfsetfillcolor{currentfill}%
\pgfsetlinewidth{0.000000pt}%
\definecolor{currentstroke}{rgb}{0.000000,0.000000,0.000000}%
\pgfsetstrokecolor{currentstroke}%
\pgfsetstrokeopacity{0.000000}%
\pgfsetdash{}{0pt}%
\pgfpathmoveto{\pgfqpoint{4.362997in}{0.500000in}}%
\pgfpathlineto{\pgfqpoint{4.396023in}{0.500000in}}%
\pgfpathlineto{\pgfqpoint{4.396023in}{0.677801in}}%
\pgfpathlineto{\pgfqpoint{4.362997in}{0.677801in}}%
\pgfpathlineto{\pgfqpoint{4.362997in}{0.500000in}}%
\pgfpathclose%
\pgfusepath{fill}%
\end{pgfscope}%
\begin{pgfscope}%
\pgfpathrectangle{\pgfqpoint{0.750000in}{0.500000in}}{\pgfqpoint{4.650000in}{3.020000in}}%
\pgfusepath{clip}%
\pgfsetbuttcap%
\pgfsetmiterjoin%
\definecolor{currentfill}{rgb}{1.000000,0.000000,0.000000}%
\pgfsetfillcolor{currentfill}%
\pgfsetlinewidth{0.000000pt}%
\definecolor{currentstroke}{rgb}{0.000000,0.000000,0.000000}%
\pgfsetstrokecolor{currentstroke}%
\pgfsetstrokeopacity{0.000000}%
\pgfsetdash{}{0pt}%
\pgfpathmoveto{\pgfqpoint{4.396023in}{0.500000in}}%
\pgfpathlineto{\pgfqpoint{4.429048in}{0.500000in}}%
\pgfpathlineto{\pgfqpoint{4.429048in}{0.510459in}}%
\pgfpathlineto{\pgfqpoint{4.396023in}{0.510459in}}%
\pgfpathlineto{\pgfqpoint{4.396023in}{0.500000in}}%
\pgfpathclose%
\pgfusepath{fill}%
\end{pgfscope}%
\begin{pgfscope}%
\pgfpathrectangle{\pgfqpoint{0.750000in}{0.500000in}}{\pgfqpoint{4.650000in}{3.020000in}}%
\pgfusepath{clip}%
\pgfsetbuttcap%
\pgfsetmiterjoin%
\definecolor{currentfill}{rgb}{1.000000,0.000000,0.000000}%
\pgfsetfillcolor{currentfill}%
\pgfsetlinewidth{0.000000pt}%
\definecolor{currentstroke}{rgb}{0.000000,0.000000,0.000000}%
\pgfsetstrokecolor{currentstroke}%
\pgfsetstrokeopacity{0.000000}%
\pgfsetdash{}{0pt}%
\pgfpathmoveto{\pgfqpoint{4.429048in}{0.500000in}}%
\pgfpathlineto{\pgfqpoint{4.462074in}{0.500000in}}%
\pgfpathlineto{\pgfqpoint{4.462074in}{0.656883in}}%
\pgfpathlineto{\pgfqpoint{4.429048in}{0.656883in}}%
\pgfpathlineto{\pgfqpoint{4.429048in}{0.500000in}}%
\pgfpathclose%
\pgfusepath{fill}%
\end{pgfscope}%
\begin{pgfscope}%
\pgfpathrectangle{\pgfqpoint{0.750000in}{0.500000in}}{\pgfqpoint{4.650000in}{3.020000in}}%
\pgfusepath{clip}%
\pgfsetbuttcap%
\pgfsetmiterjoin%
\definecolor{currentfill}{rgb}{1.000000,0.000000,0.000000}%
\pgfsetfillcolor{currentfill}%
\pgfsetlinewidth{0.000000pt}%
\definecolor{currentstroke}{rgb}{0.000000,0.000000,0.000000}%
\pgfsetstrokecolor{currentstroke}%
\pgfsetstrokeopacity{0.000000}%
\pgfsetdash{}{0pt}%
\pgfpathmoveto{\pgfqpoint{4.462074in}{0.500000in}}%
\pgfpathlineto{\pgfqpoint{4.495099in}{0.500000in}}%
\pgfpathlineto{\pgfqpoint{4.495099in}{0.520918in}}%
\pgfpathlineto{\pgfqpoint{4.462074in}{0.520918in}}%
\pgfpathlineto{\pgfqpoint{4.462074in}{0.500000in}}%
\pgfpathclose%
\pgfusepath{fill}%
\end{pgfscope}%
\begin{pgfscope}%
\pgfpathrectangle{\pgfqpoint{0.750000in}{0.500000in}}{\pgfqpoint{4.650000in}{3.020000in}}%
\pgfusepath{clip}%
\pgfsetbuttcap%
\pgfsetmiterjoin%
\definecolor{currentfill}{rgb}{1.000000,0.000000,0.000000}%
\pgfsetfillcolor{currentfill}%
\pgfsetlinewidth{0.000000pt}%
\definecolor{currentstroke}{rgb}{0.000000,0.000000,0.000000}%
\pgfsetstrokecolor{currentstroke}%
\pgfsetstrokeopacity{0.000000}%
\pgfsetdash{}{0pt}%
\pgfpathmoveto{\pgfqpoint{4.495099in}{0.500000in}}%
\pgfpathlineto{\pgfqpoint{4.528125in}{0.500000in}}%
\pgfpathlineto{\pgfqpoint{4.528125in}{0.520918in}}%
\pgfpathlineto{\pgfqpoint{4.495099in}{0.520918in}}%
\pgfpathlineto{\pgfqpoint{4.495099in}{0.500000in}}%
\pgfpathclose%
\pgfusepath{fill}%
\end{pgfscope}%
\begin{pgfscope}%
\pgfpathrectangle{\pgfqpoint{0.750000in}{0.500000in}}{\pgfqpoint{4.650000in}{3.020000in}}%
\pgfusepath{clip}%
\pgfsetbuttcap%
\pgfsetmiterjoin%
\definecolor{currentfill}{rgb}{1.000000,0.000000,0.000000}%
\pgfsetfillcolor{currentfill}%
\pgfsetlinewidth{0.000000pt}%
\definecolor{currentstroke}{rgb}{0.000000,0.000000,0.000000}%
\pgfsetstrokecolor{currentstroke}%
\pgfsetstrokeopacity{0.000000}%
\pgfsetdash{}{0pt}%
\pgfpathmoveto{\pgfqpoint{4.528125in}{0.500000in}}%
\pgfpathlineto{\pgfqpoint{4.561151in}{0.500000in}}%
\pgfpathlineto{\pgfqpoint{4.561151in}{0.615048in}}%
\pgfpathlineto{\pgfqpoint{4.528125in}{0.615048in}}%
\pgfpathlineto{\pgfqpoint{4.528125in}{0.500000in}}%
\pgfpathclose%
\pgfusepath{fill}%
\end{pgfscope}%
\begin{pgfscope}%
\pgfpathrectangle{\pgfqpoint{0.750000in}{0.500000in}}{\pgfqpoint{4.650000in}{3.020000in}}%
\pgfusepath{clip}%
\pgfsetbuttcap%
\pgfsetmiterjoin%
\definecolor{currentfill}{rgb}{1.000000,0.000000,0.000000}%
\pgfsetfillcolor{currentfill}%
\pgfsetlinewidth{0.000000pt}%
\definecolor{currentstroke}{rgb}{0.000000,0.000000,0.000000}%
\pgfsetstrokecolor{currentstroke}%
\pgfsetstrokeopacity{0.000000}%
\pgfsetdash{}{0pt}%
\pgfpathmoveto{\pgfqpoint{4.561151in}{0.500000in}}%
\pgfpathlineto{\pgfqpoint{4.594176in}{0.500000in}}%
\pgfpathlineto{\pgfqpoint{4.594176in}{0.500000in}}%
\pgfpathlineto{\pgfqpoint{4.561151in}{0.500000in}}%
\pgfpathlineto{\pgfqpoint{4.561151in}{0.500000in}}%
\pgfpathclose%
\pgfusepath{fill}%
\end{pgfscope}%
\begin{pgfscope}%
\pgfpathrectangle{\pgfqpoint{0.750000in}{0.500000in}}{\pgfqpoint{4.650000in}{3.020000in}}%
\pgfusepath{clip}%
\pgfsetbuttcap%
\pgfsetmiterjoin%
\definecolor{currentfill}{rgb}{1.000000,0.000000,0.000000}%
\pgfsetfillcolor{currentfill}%
\pgfsetlinewidth{0.000000pt}%
\definecolor{currentstroke}{rgb}{0.000000,0.000000,0.000000}%
\pgfsetstrokecolor{currentstroke}%
\pgfsetstrokeopacity{0.000000}%
\pgfsetdash{}{0pt}%
\pgfpathmoveto{\pgfqpoint{4.594176in}{0.500000in}}%
\pgfpathlineto{\pgfqpoint{4.627202in}{0.500000in}}%
\pgfpathlineto{\pgfqpoint{4.627202in}{0.520918in}}%
\pgfpathlineto{\pgfqpoint{4.594176in}{0.520918in}}%
\pgfpathlineto{\pgfqpoint{4.594176in}{0.500000in}}%
\pgfpathclose%
\pgfusepath{fill}%
\end{pgfscope}%
\begin{pgfscope}%
\pgfpathrectangle{\pgfqpoint{0.750000in}{0.500000in}}{\pgfqpoint{4.650000in}{3.020000in}}%
\pgfusepath{clip}%
\pgfsetbuttcap%
\pgfsetmiterjoin%
\definecolor{currentfill}{rgb}{1.000000,0.000000,0.000000}%
\pgfsetfillcolor{currentfill}%
\pgfsetlinewidth{0.000000pt}%
\definecolor{currentstroke}{rgb}{0.000000,0.000000,0.000000}%
\pgfsetstrokecolor{currentstroke}%
\pgfsetstrokeopacity{0.000000}%
\pgfsetdash{}{0pt}%
\pgfpathmoveto{\pgfqpoint{4.627202in}{0.500000in}}%
\pgfpathlineto{\pgfqpoint{4.660227in}{0.500000in}}%
\pgfpathlineto{\pgfqpoint{4.660227in}{0.510459in}}%
\pgfpathlineto{\pgfqpoint{4.627202in}{0.510459in}}%
\pgfpathlineto{\pgfqpoint{4.627202in}{0.500000in}}%
\pgfpathclose%
\pgfusepath{fill}%
\end{pgfscope}%
\begin{pgfscope}%
\pgfpathrectangle{\pgfqpoint{0.750000in}{0.500000in}}{\pgfqpoint{4.650000in}{3.020000in}}%
\pgfusepath{clip}%
\pgfsetbuttcap%
\pgfsetmiterjoin%
\definecolor{currentfill}{rgb}{1.000000,0.000000,0.000000}%
\pgfsetfillcolor{currentfill}%
\pgfsetlinewidth{0.000000pt}%
\definecolor{currentstroke}{rgb}{0.000000,0.000000,0.000000}%
\pgfsetstrokecolor{currentstroke}%
\pgfsetstrokeopacity{0.000000}%
\pgfsetdash{}{0pt}%
\pgfpathmoveto{\pgfqpoint{4.660227in}{0.500000in}}%
\pgfpathlineto{\pgfqpoint{4.693253in}{0.500000in}}%
\pgfpathlineto{\pgfqpoint{4.693253in}{0.500000in}}%
\pgfpathlineto{\pgfqpoint{4.660227in}{0.500000in}}%
\pgfpathlineto{\pgfqpoint{4.660227in}{0.500000in}}%
\pgfpathclose%
\pgfusepath{fill}%
\end{pgfscope}%
\begin{pgfscope}%
\pgfpathrectangle{\pgfqpoint{0.750000in}{0.500000in}}{\pgfqpoint{4.650000in}{3.020000in}}%
\pgfusepath{clip}%
\pgfsetbuttcap%
\pgfsetmiterjoin%
\definecolor{currentfill}{rgb}{1.000000,0.000000,0.000000}%
\pgfsetfillcolor{currentfill}%
\pgfsetlinewidth{0.000000pt}%
\definecolor{currentstroke}{rgb}{0.000000,0.000000,0.000000}%
\pgfsetstrokecolor{currentstroke}%
\pgfsetstrokeopacity{0.000000}%
\pgfsetdash{}{0pt}%
\pgfpathmoveto{\pgfqpoint{4.693253in}{0.500000in}}%
\pgfpathlineto{\pgfqpoint{4.726278in}{0.500000in}}%
\pgfpathlineto{\pgfqpoint{4.726278in}{0.520918in}}%
\pgfpathlineto{\pgfqpoint{4.693253in}{0.520918in}}%
\pgfpathlineto{\pgfqpoint{4.693253in}{0.500000in}}%
\pgfpathclose%
\pgfusepath{fill}%
\end{pgfscope}%
\begin{pgfscope}%
\pgfpathrectangle{\pgfqpoint{0.750000in}{0.500000in}}{\pgfqpoint{4.650000in}{3.020000in}}%
\pgfusepath{clip}%
\pgfsetbuttcap%
\pgfsetmiterjoin%
\definecolor{currentfill}{rgb}{1.000000,0.000000,0.000000}%
\pgfsetfillcolor{currentfill}%
\pgfsetlinewidth{0.000000pt}%
\definecolor{currentstroke}{rgb}{0.000000,0.000000,0.000000}%
\pgfsetstrokecolor{currentstroke}%
\pgfsetstrokeopacity{0.000000}%
\pgfsetdash{}{0pt}%
\pgfpathmoveto{\pgfqpoint{4.726278in}{0.500000in}}%
\pgfpathlineto{\pgfqpoint{4.759304in}{0.500000in}}%
\pgfpathlineto{\pgfqpoint{4.759304in}{0.500000in}}%
\pgfpathlineto{\pgfqpoint{4.726278in}{0.500000in}}%
\pgfpathlineto{\pgfqpoint{4.726278in}{0.500000in}}%
\pgfpathclose%
\pgfusepath{fill}%
\end{pgfscope}%
\begin{pgfscope}%
\pgfpathrectangle{\pgfqpoint{0.750000in}{0.500000in}}{\pgfqpoint{4.650000in}{3.020000in}}%
\pgfusepath{clip}%
\pgfsetbuttcap%
\pgfsetmiterjoin%
\definecolor{currentfill}{rgb}{1.000000,0.000000,0.000000}%
\pgfsetfillcolor{currentfill}%
\pgfsetlinewidth{0.000000pt}%
\definecolor{currentstroke}{rgb}{0.000000,0.000000,0.000000}%
\pgfsetstrokecolor{currentstroke}%
\pgfsetstrokeopacity{0.000000}%
\pgfsetdash{}{0pt}%
\pgfpathmoveto{\pgfqpoint{4.759304in}{0.500000in}}%
\pgfpathlineto{\pgfqpoint{4.792330in}{0.500000in}}%
\pgfpathlineto{\pgfqpoint{4.792330in}{0.510459in}}%
\pgfpathlineto{\pgfqpoint{4.759304in}{0.510459in}}%
\pgfpathlineto{\pgfqpoint{4.759304in}{0.500000in}}%
\pgfpathclose%
\pgfusepath{fill}%
\end{pgfscope}%
\begin{pgfscope}%
\pgfpathrectangle{\pgfqpoint{0.750000in}{0.500000in}}{\pgfqpoint{4.650000in}{3.020000in}}%
\pgfusepath{clip}%
\pgfsetbuttcap%
\pgfsetmiterjoin%
\definecolor{currentfill}{rgb}{1.000000,0.000000,0.000000}%
\pgfsetfillcolor{currentfill}%
\pgfsetlinewidth{0.000000pt}%
\definecolor{currentstroke}{rgb}{0.000000,0.000000,0.000000}%
\pgfsetstrokecolor{currentstroke}%
\pgfsetstrokeopacity{0.000000}%
\pgfsetdash{}{0pt}%
\pgfpathmoveto{\pgfqpoint{4.792330in}{0.500000in}}%
\pgfpathlineto{\pgfqpoint{4.825355in}{0.500000in}}%
\pgfpathlineto{\pgfqpoint{4.825355in}{0.500000in}}%
\pgfpathlineto{\pgfqpoint{4.792330in}{0.500000in}}%
\pgfpathlineto{\pgfqpoint{4.792330in}{0.500000in}}%
\pgfpathclose%
\pgfusepath{fill}%
\end{pgfscope}%
\begin{pgfscope}%
\pgfpathrectangle{\pgfqpoint{0.750000in}{0.500000in}}{\pgfqpoint{4.650000in}{3.020000in}}%
\pgfusepath{clip}%
\pgfsetbuttcap%
\pgfsetmiterjoin%
\definecolor{currentfill}{rgb}{1.000000,0.000000,0.000000}%
\pgfsetfillcolor{currentfill}%
\pgfsetlinewidth{0.000000pt}%
\definecolor{currentstroke}{rgb}{0.000000,0.000000,0.000000}%
\pgfsetstrokecolor{currentstroke}%
\pgfsetstrokeopacity{0.000000}%
\pgfsetdash{}{0pt}%
\pgfpathmoveto{\pgfqpoint{4.825355in}{0.500000in}}%
\pgfpathlineto{\pgfqpoint{4.858381in}{0.500000in}}%
\pgfpathlineto{\pgfqpoint{4.858381in}{0.500000in}}%
\pgfpathlineto{\pgfqpoint{4.825355in}{0.500000in}}%
\pgfpathlineto{\pgfqpoint{4.825355in}{0.500000in}}%
\pgfpathclose%
\pgfusepath{fill}%
\end{pgfscope}%
\begin{pgfscope}%
\pgfpathrectangle{\pgfqpoint{0.750000in}{0.500000in}}{\pgfqpoint{4.650000in}{3.020000in}}%
\pgfusepath{clip}%
\pgfsetbuttcap%
\pgfsetmiterjoin%
\definecolor{currentfill}{rgb}{1.000000,0.000000,0.000000}%
\pgfsetfillcolor{currentfill}%
\pgfsetlinewidth{0.000000pt}%
\definecolor{currentstroke}{rgb}{0.000000,0.000000,0.000000}%
\pgfsetstrokecolor{currentstroke}%
\pgfsetstrokeopacity{0.000000}%
\pgfsetdash{}{0pt}%
\pgfpathmoveto{\pgfqpoint{4.858381in}{0.500000in}}%
\pgfpathlineto{\pgfqpoint{4.891406in}{0.500000in}}%
\pgfpathlineto{\pgfqpoint{4.891406in}{0.520918in}}%
\pgfpathlineto{\pgfqpoint{4.858381in}{0.520918in}}%
\pgfpathlineto{\pgfqpoint{4.858381in}{0.500000in}}%
\pgfpathclose%
\pgfusepath{fill}%
\end{pgfscope}%
\begin{pgfscope}%
\pgfpathrectangle{\pgfqpoint{0.750000in}{0.500000in}}{\pgfqpoint{4.650000in}{3.020000in}}%
\pgfusepath{clip}%
\pgfsetbuttcap%
\pgfsetmiterjoin%
\definecolor{currentfill}{rgb}{1.000000,0.000000,0.000000}%
\pgfsetfillcolor{currentfill}%
\pgfsetlinewidth{0.000000pt}%
\definecolor{currentstroke}{rgb}{0.000000,0.000000,0.000000}%
\pgfsetstrokecolor{currentstroke}%
\pgfsetstrokeopacity{0.000000}%
\pgfsetdash{}{0pt}%
\pgfpathmoveto{\pgfqpoint{4.891406in}{0.500000in}}%
\pgfpathlineto{\pgfqpoint{4.924432in}{0.500000in}}%
\pgfpathlineto{\pgfqpoint{4.924432in}{0.500000in}}%
\pgfpathlineto{\pgfqpoint{4.891406in}{0.500000in}}%
\pgfpathlineto{\pgfqpoint{4.891406in}{0.500000in}}%
\pgfpathclose%
\pgfusepath{fill}%
\end{pgfscope}%
\begin{pgfscope}%
\pgfpathrectangle{\pgfqpoint{0.750000in}{0.500000in}}{\pgfqpoint{4.650000in}{3.020000in}}%
\pgfusepath{clip}%
\pgfsetbuttcap%
\pgfsetmiterjoin%
\definecolor{currentfill}{rgb}{1.000000,0.000000,0.000000}%
\pgfsetfillcolor{currentfill}%
\pgfsetlinewidth{0.000000pt}%
\definecolor{currentstroke}{rgb}{0.000000,0.000000,0.000000}%
\pgfsetstrokecolor{currentstroke}%
\pgfsetstrokeopacity{0.000000}%
\pgfsetdash{}{0pt}%
\pgfpathmoveto{\pgfqpoint{4.924432in}{0.500000in}}%
\pgfpathlineto{\pgfqpoint{4.957457in}{0.500000in}}%
\pgfpathlineto{\pgfqpoint{4.957457in}{0.531377in}}%
\pgfpathlineto{\pgfqpoint{4.924432in}{0.531377in}}%
\pgfpathlineto{\pgfqpoint{4.924432in}{0.500000in}}%
\pgfpathclose%
\pgfusepath{fill}%
\end{pgfscope}%
\begin{pgfscope}%
\pgfpathrectangle{\pgfqpoint{0.750000in}{0.500000in}}{\pgfqpoint{4.650000in}{3.020000in}}%
\pgfusepath{clip}%
\pgfsetbuttcap%
\pgfsetmiterjoin%
\definecolor{currentfill}{rgb}{1.000000,0.000000,0.000000}%
\pgfsetfillcolor{currentfill}%
\pgfsetlinewidth{0.000000pt}%
\definecolor{currentstroke}{rgb}{0.000000,0.000000,0.000000}%
\pgfsetstrokecolor{currentstroke}%
\pgfsetstrokeopacity{0.000000}%
\pgfsetdash{}{0pt}%
\pgfpathmoveto{\pgfqpoint{4.957457in}{0.500000in}}%
\pgfpathlineto{\pgfqpoint{4.990483in}{0.500000in}}%
\pgfpathlineto{\pgfqpoint{4.990483in}{0.500000in}}%
\pgfpathlineto{\pgfqpoint{4.957457in}{0.500000in}}%
\pgfpathlineto{\pgfqpoint{4.957457in}{0.500000in}}%
\pgfpathclose%
\pgfusepath{fill}%
\end{pgfscope}%
\begin{pgfscope}%
\pgfpathrectangle{\pgfqpoint{0.750000in}{0.500000in}}{\pgfqpoint{4.650000in}{3.020000in}}%
\pgfusepath{clip}%
\pgfsetbuttcap%
\pgfsetmiterjoin%
\definecolor{currentfill}{rgb}{1.000000,0.000000,0.000000}%
\pgfsetfillcolor{currentfill}%
\pgfsetlinewidth{0.000000pt}%
\definecolor{currentstroke}{rgb}{0.000000,0.000000,0.000000}%
\pgfsetstrokecolor{currentstroke}%
\pgfsetstrokeopacity{0.000000}%
\pgfsetdash{}{0pt}%
\pgfpathmoveto{\pgfqpoint{4.990483in}{0.500000in}}%
\pgfpathlineto{\pgfqpoint{5.023509in}{0.500000in}}%
\pgfpathlineto{\pgfqpoint{5.023509in}{0.500000in}}%
\pgfpathlineto{\pgfqpoint{4.990483in}{0.500000in}}%
\pgfpathlineto{\pgfqpoint{4.990483in}{0.500000in}}%
\pgfpathclose%
\pgfusepath{fill}%
\end{pgfscope}%
\begin{pgfscope}%
\pgfpathrectangle{\pgfqpoint{0.750000in}{0.500000in}}{\pgfqpoint{4.650000in}{3.020000in}}%
\pgfusepath{clip}%
\pgfsetbuttcap%
\pgfsetmiterjoin%
\definecolor{currentfill}{rgb}{1.000000,0.000000,0.000000}%
\pgfsetfillcolor{currentfill}%
\pgfsetlinewidth{0.000000pt}%
\definecolor{currentstroke}{rgb}{0.000000,0.000000,0.000000}%
\pgfsetstrokecolor{currentstroke}%
\pgfsetstrokeopacity{0.000000}%
\pgfsetdash{}{0pt}%
\pgfpathmoveto{\pgfqpoint{5.023509in}{0.500000in}}%
\pgfpathlineto{\pgfqpoint{5.056534in}{0.500000in}}%
\pgfpathlineto{\pgfqpoint{5.056534in}{0.520918in}}%
\pgfpathlineto{\pgfqpoint{5.023509in}{0.520918in}}%
\pgfpathlineto{\pgfqpoint{5.023509in}{0.500000in}}%
\pgfpathclose%
\pgfusepath{fill}%
\end{pgfscope}%
\begin{pgfscope}%
\pgfpathrectangle{\pgfqpoint{0.750000in}{0.500000in}}{\pgfqpoint{4.650000in}{3.020000in}}%
\pgfusepath{clip}%
\pgfsetbuttcap%
\pgfsetmiterjoin%
\definecolor{currentfill}{rgb}{1.000000,0.000000,0.000000}%
\pgfsetfillcolor{currentfill}%
\pgfsetlinewidth{0.000000pt}%
\definecolor{currentstroke}{rgb}{0.000000,0.000000,0.000000}%
\pgfsetstrokecolor{currentstroke}%
\pgfsetstrokeopacity{0.000000}%
\pgfsetdash{}{0pt}%
\pgfpathmoveto{\pgfqpoint{5.056534in}{0.500000in}}%
\pgfpathlineto{\pgfqpoint{5.089560in}{0.500000in}}%
\pgfpathlineto{\pgfqpoint{5.089560in}{0.500000in}}%
\pgfpathlineto{\pgfqpoint{5.056534in}{0.500000in}}%
\pgfpathlineto{\pgfqpoint{5.056534in}{0.500000in}}%
\pgfpathclose%
\pgfusepath{fill}%
\end{pgfscope}%
\begin{pgfscope}%
\pgfpathrectangle{\pgfqpoint{0.750000in}{0.500000in}}{\pgfqpoint{4.650000in}{3.020000in}}%
\pgfusepath{clip}%
\pgfsetbuttcap%
\pgfsetmiterjoin%
\definecolor{currentfill}{rgb}{1.000000,0.000000,0.000000}%
\pgfsetfillcolor{currentfill}%
\pgfsetlinewidth{0.000000pt}%
\definecolor{currentstroke}{rgb}{0.000000,0.000000,0.000000}%
\pgfsetstrokecolor{currentstroke}%
\pgfsetstrokeopacity{0.000000}%
\pgfsetdash{}{0pt}%
\pgfpathmoveto{\pgfqpoint{5.089560in}{0.500000in}}%
\pgfpathlineto{\pgfqpoint{5.122585in}{0.500000in}}%
\pgfpathlineto{\pgfqpoint{5.122585in}{0.520918in}}%
\pgfpathlineto{\pgfqpoint{5.089560in}{0.520918in}}%
\pgfpathlineto{\pgfqpoint{5.089560in}{0.500000in}}%
\pgfpathclose%
\pgfusepath{fill}%
\end{pgfscope}%
\begin{pgfscope}%
\pgfpathrectangle{\pgfqpoint{0.750000in}{0.500000in}}{\pgfqpoint{4.650000in}{3.020000in}}%
\pgfusepath{clip}%
\pgfsetbuttcap%
\pgfsetmiterjoin%
\definecolor{currentfill}{rgb}{1.000000,0.000000,0.000000}%
\pgfsetfillcolor{currentfill}%
\pgfsetlinewidth{0.000000pt}%
\definecolor{currentstroke}{rgb}{0.000000,0.000000,0.000000}%
\pgfsetstrokecolor{currentstroke}%
\pgfsetstrokeopacity{0.000000}%
\pgfsetdash{}{0pt}%
\pgfpathmoveto{\pgfqpoint{5.122585in}{0.500000in}}%
\pgfpathlineto{\pgfqpoint{5.155611in}{0.500000in}}%
\pgfpathlineto{\pgfqpoint{5.155611in}{0.500000in}}%
\pgfpathlineto{\pgfqpoint{5.122585in}{0.500000in}}%
\pgfpathlineto{\pgfqpoint{5.122585in}{0.500000in}}%
\pgfpathclose%
\pgfusepath{fill}%
\end{pgfscope}%
\begin{pgfscope}%
\pgfpathrectangle{\pgfqpoint{0.750000in}{0.500000in}}{\pgfqpoint{4.650000in}{3.020000in}}%
\pgfusepath{clip}%
\pgfsetbuttcap%
\pgfsetmiterjoin%
\definecolor{currentfill}{rgb}{1.000000,0.000000,0.000000}%
\pgfsetfillcolor{currentfill}%
\pgfsetlinewidth{0.000000pt}%
\definecolor{currentstroke}{rgb}{0.000000,0.000000,0.000000}%
\pgfsetstrokecolor{currentstroke}%
\pgfsetstrokeopacity{0.000000}%
\pgfsetdash{}{0pt}%
\pgfpathmoveto{\pgfqpoint{5.155611in}{0.500000in}}%
\pgfpathlineto{\pgfqpoint{5.188636in}{0.500000in}}%
\pgfpathlineto{\pgfqpoint{5.188636in}{0.510459in}}%
\pgfpathlineto{\pgfqpoint{5.155611in}{0.510459in}}%
\pgfpathlineto{\pgfqpoint{5.155611in}{0.500000in}}%
\pgfpathclose%
\pgfusepath{fill}%
\end{pgfscope}%
\begin{pgfscope}%
\pgfsetbuttcap%
\pgfsetroundjoin%
\definecolor{currentfill}{rgb}{0.000000,0.000000,0.000000}%
\pgfsetfillcolor{currentfill}%
\pgfsetlinewidth{0.803000pt}%
\definecolor{currentstroke}{rgb}{0.000000,0.000000,0.000000}%
\pgfsetstrokecolor{currentstroke}%
\pgfsetdash{}{0pt}%
\pgfsys@defobject{currentmarker}{\pgfqpoint{0.000000in}{-0.048611in}}{\pgfqpoint{0.000000in}{0.000000in}}{%
\pgfpathmoveto{\pgfqpoint{0.000000in}{0.000000in}}%
\pgfpathlineto{\pgfqpoint{0.000000in}{-0.048611in}}%
\pgfusepath{stroke,fill}%
}%
\begin{pgfscope}%
\pgfsys@transformshift{0.988234in}{0.500000in}%
\pgfsys@useobject{currentmarker}{}%
\end{pgfscope}%
\end{pgfscope}%
\begin{pgfscope}%
\definecolor{textcolor}{rgb}{0.000000,0.000000,0.000000}%
\pgfsetstrokecolor{textcolor}%
\pgfsetfillcolor{textcolor}%
\pgftext[x=0.988234in,y=0.402778in,,top]{\color{textcolor}\sffamily\fontsize{10.000000}{12.000000}\selectfont 0.400}%
\end{pgfscope}%
\begin{pgfscope}%
\pgfsetbuttcap%
\pgfsetroundjoin%
\definecolor{currentfill}{rgb}{0.000000,0.000000,0.000000}%
\pgfsetfillcolor{currentfill}%
\pgfsetlinewidth{0.803000pt}%
\definecolor{currentstroke}{rgb}{0.000000,0.000000,0.000000}%
\pgfsetstrokecolor{currentstroke}%
\pgfsetdash{}{0pt}%
\pgfsys@defobject{currentmarker}{\pgfqpoint{0.000000in}{-0.048611in}}{\pgfqpoint{0.000000in}{0.000000in}}{%
\pgfpathmoveto{\pgfqpoint{0.000000in}{0.000000in}}%
\pgfpathlineto{\pgfqpoint{0.000000in}{-0.048611in}}%
\pgfusepath{stroke,fill}%
}%
\begin{pgfscope}%
\pgfsys@transformshift{1.509215in}{0.500000in}%
\pgfsys@useobject{currentmarker}{}%
\end{pgfscope}%
\end{pgfscope}%
\begin{pgfscope}%
\definecolor{textcolor}{rgb}{0.000000,0.000000,0.000000}%
\pgfsetstrokecolor{textcolor}%
\pgfsetfillcolor{textcolor}%
\pgftext[x=1.509215in,y=0.402778in,,top]{\color{textcolor}\sffamily\fontsize{10.000000}{12.000000}\selectfont 0.425}%
\end{pgfscope}%
\begin{pgfscope}%
\pgfsetbuttcap%
\pgfsetroundjoin%
\definecolor{currentfill}{rgb}{0.000000,0.000000,0.000000}%
\pgfsetfillcolor{currentfill}%
\pgfsetlinewidth{0.803000pt}%
\definecolor{currentstroke}{rgb}{0.000000,0.000000,0.000000}%
\pgfsetstrokecolor{currentstroke}%
\pgfsetdash{}{0pt}%
\pgfsys@defobject{currentmarker}{\pgfqpoint{0.000000in}{-0.048611in}}{\pgfqpoint{0.000000in}{0.000000in}}{%
\pgfpathmoveto{\pgfqpoint{0.000000in}{0.000000in}}%
\pgfpathlineto{\pgfqpoint{0.000000in}{-0.048611in}}%
\pgfusepath{stroke,fill}%
}%
\begin{pgfscope}%
\pgfsys@transformshift{2.030197in}{0.500000in}%
\pgfsys@useobject{currentmarker}{}%
\end{pgfscope}%
\end{pgfscope}%
\begin{pgfscope}%
\definecolor{textcolor}{rgb}{0.000000,0.000000,0.000000}%
\pgfsetstrokecolor{textcolor}%
\pgfsetfillcolor{textcolor}%
\pgftext[x=2.030197in,y=0.402778in,,top]{\color{textcolor}\sffamily\fontsize{10.000000}{12.000000}\selectfont 0.450}%
\end{pgfscope}%
\begin{pgfscope}%
\pgfsetbuttcap%
\pgfsetroundjoin%
\definecolor{currentfill}{rgb}{0.000000,0.000000,0.000000}%
\pgfsetfillcolor{currentfill}%
\pgfsetlinewidth{0.803000pt}%
\definecolor{currentstroke}{rgb}{0.000000,0.000000,0.000000}%
\pgfsetstrokecolor{currentstroke}%
\pgfsetdash{}{0pt}%
\pgfsys@defobject{currentmarker}{\pgfqpoint{0.000000in}{-0.048611in}}{\pgfqpoint{0.000000in}{0.000000in}}{%
\pgfpathmoveto{\pgfqpoint{0.000000in}{0.000000in}}%
\pgfpathlineto{\pgfqpoint{0.000000in}{-0.048611in}}%
\pgfusepath{stroke,fill}%
}%
\begin{pgfscope}%
\pgfsys@transformshift{2.551178in}{0.500000in}%
\pgfsys@useobject{currentmarker}{}%
\end{pgfscope}%
\end{pgfscope}%
\begin{pgfscope}%
\definecolor{textcolor}{rgb}{0.000000,0.000000,0.000000}%
\pgfsetstrokecolor{textcolor}%
\pgfsetfillcolor{textcolor}%
\pgftext[x=2.551178in,y=0.402778in,,top]{\color{textcolor}\sffamily\fontsize{10.000000}{12.000000}\selectfont 0.475}%
\end{pgfscope}%
\begin{pgfscope}%
\pgfsetbuttcap%
\pgfsetroundjoin%
\definecolor{currentfill}{rgb}{0.000000,0.000000,0.000000}%
\pgfsetfillcolor{currentfill}%
\pgfsetlinewidth{0.803000pt}%
\definecolor{currentstroke}{rgb}{0.000000,0.000000,0.000000}%
\pgfsetstrokecolor{currentstroke}%
\pgfsetdash{}{0pt}%
\pgfsys@defobject{currentmarker}{\pgfqpoint{0.000000in}{-0.048611in}}{\pgfqpoint{0.000000in}{0.000000in}}{%
\pgfpathmoveto{\pgfqpoint{0.000000in}{0.000000in}}%
\pgfpathlineto{\pgfqpoint{0.000000in}{-0.048611in}}%
\pgfusepath{stroke,fill}%
}%
\begin{pgfscope}%
\pgfsys@transformshift{3.072159in}{0.500000in}%
\pgfsys@useobject{currentmarker}{}%
\end{pgfscope}%
\end{pgfscope}%
\begin{pgfscope}%
\definecolor{textcolor}{rgb}{0.000000,0.000000,0.000000}%
\pgfsetstrokecolor{textcolor}%
\pgfsetfillcolor{textcolor}%
\pgftext[x=3.072159in,y=0.402778in,,top]{\color{textcolor}\sffamily\fontsize{10.000000}{12.000000}\selectfont 0.500}%
\end{pgfscope}%
\begin{pgfscope}%
\pgfsetbuttcap%
\pgfsetroundjoin%
\definecolor{currentfill}{rgb}{0.000000,0.000000,0.000000}%
\pgfsetfillcolor{currentfill}%
\pgfsetlinewidth{0.803000pt}%
\definecolor{currentstroke}{rgb}{0.000000,0.000000,0.000000}%
\pgfsetstrokecolor{currentstroke}%
\pgfsetdash{}{0pt}%
\pgfsys@defobject{currentmarker}{\pgfqpoint{0.000000in}{-0.048611in}}{\pgfqpoint{0.000000in}{0.000000in}}{%
\pgfpathmoveto{\pgfqpoint{0.000000in}{0.000000in}}%
\pgfpathlineto{\pgfqpoint{0.000000in}{-0.048611in}}%
\pgfusepath{stroke,fill}%
}%
\begin{pgfscope}%
\pgfsys@transformshift{3.593140in}{0.500000in}%
\pgfsys@useobject{currentmarker}{}%
\end{pgfscope}%
\end{pgfscope}%
\begin{pgfscope}%
\definecolor{textcolor}{rgb}{0.000000,0.000000,0.000000}%
\pgfsetstrokecolor{textcolor}%
\pgfsetfillcolor{textcolor}%
\pgftext[x=3.593140in,y=0.402778in,,top]{\color{textcolor}\sffamily\fontsize{10.000000}{12.000000}\selectfont 0.525}%
\end{pgfscope}%
\begin{pgfscope}%
\pgfsetbuttcap%
\pgfsetroundjoin%
\definecolor{currentfill}{rgb}{0.000000,0.000000,0.000000}%
\pgfsetfillcolor{currentfill}%
\pgfsetlinewidth{0.803000pt}%
\definecolor{currentstroke}{rgb}{0.000000,0.000000,0.000000}%
\pgfsetstrokecolor{currentstroke}%
\pgfsetdash{}{0pt}%
\pgfsys@defobject{currentmarker}{\pgfqpoint{0.000000in}{-0.048611in}}{\pgfqpoint{0.000000in}{0.000000in}}{%
\pgfpathmoveto{\pgfqpoint{0.000000in}{0.000000in}}%
\pgfpathlineto{\pgfqpoint{0.000000in}{-0.048611in}}%
\pgfusepath{stroke,fill}%
}%
\begin{pgfscope}%
\pgfsys@transformshift{4.114121in}{0.500000in}%
\pgfsys@useobject{currentmarker}{}%
\end{pgfscope}%
\end{pgfscope}%
\begin{pgfscope}%
\definecolor{textcolor}{rgb}{0.000000,0.000000,0.000000}%
\pgfsetstrokecolor{textcolor}%
\pgfsetfillcolor{textcolor}%
\pgftext[x=4.114121in,y=0.402778in,,top]{\color{textcolor}\sffamily\fontsize{10.000000}{12.000000}\selectfont 0.550}%
\end{pgfscope}%
\begin{pgfscope}%
\pgfsetbuttcap%
\pgfsetroundjoin%
\definecolor{currentfill}{rgb}{0.000000,0.000000,0.000000}%
\pgfsetfillcolor{currentfill}%
\pgfsetlinewidth{0.803000pt}%
\definecolor{currentstroke}{rgb}{0.000000,0.000000,0.000000}%
\pgfsetstrokecolor{currentstroke}%
\pgfsetdash{}{0pt}%
\pgfsys@defobject{currentmarker}{\pgfqpoint{0.000000in}{-0.048611in}}{\pgfqpoint{0.000000in}{0.000000in}}{%
\pgfpathmoveto{\pgfqpoint{0.000000in}{0.000000in}}%
\pgfpathlineto{\pgfqpoint{0.000000in}{-0.048611in}}%
\pgfusepath{stroke,fill}%
}%
\begin{pgfscope}%
\pgfsys@transformshift{4.635103in}{0.500000in}%
\pgfsys@useobject{currentmarker}{}%
\end{pgfscope}%
\end{pgfscope}%
\begin{pgfscope}%
\definecolor{textcolor}{rgb}{0.000000,0.000000,0.000000}%
\pgfsetstrokecolor{textcolor}%
\pgfsetfillcolor{textcolor}%
\pgftext[x=4.635103in,y=0.402778in,,top]{\color{textcolor}\sffamily\fontsize{10.000000}{12.000000}\selectfont 0.575}%
\end{pgfscope}%
\begin{pgfscope}%
\pgfsetbuttcap%
\pgfsetroundjoin%
\definecolor{currentfill}{rgb}{0.000000,0.000000,0.000000}%
\pgfsetfillcolor{currentfill}%
\pgfsetlinewidth{0.803000pt}%
\definecolor{currentstroke}{rgb}{0.000000,0.000000,0.000000}%
\pgfsetstrokecolor{currentstroke}%
\pgfsetdash{}{0pt}%
\pgfsys@defobject{currentmarker}{\pgfqpoint{0.000000in}{-0.048611in}}{\pgfqpoint{0.000000in}{0.000000in}}{%
\pgfpathmoveto{\pgfqpoint{0.000000in}{0.000000in}}%
\pgfpathlineto{\pgfqpoint{0.000000in}{-0.048611in}}%
\pgfusepath{stroke,fill}%
}%
\begin{pgfscope}%
\pgfsys@transformshift{5.156084in}{0.500000in}%
\pgfsys@useobject{currentmarker}{}%
\end{pgfscope}%
\end{pgfscope}%
\begin{pgfscope}%
\definecolor{textcolor}{rgb}{0.000000,0.000000,0.000000}%
\pgfsetstrokecolor{textcolor}%
\pgfsetfillcolor{textcolor}%
\pgftext[x=5.156084in,y=0.402778in,,top]{\color{textcolor}\sffamily\fontsize{10.000000}{12.000000}\selectfont 0.600}%
\end{pgfscope}%
\begin{pgfscope}%
\definecolor{textcolor}{rgb}{0.000000,0.000000,0.000000}%
\pgfsetstrokecolor{textcolor}%
\pgfsetfillcolor{textcolor}%
\pgftext[x=3.075000in,y=0.212809in,,top]{\color{textcolor}\sffamily\fontsize{10.000000}{12.000000}\selectfont loss}%
\end{pgfscope}%
\begin{pgfscope}%
\pgfsetbuttcap%
\pgfsetroundjoin%
\definecolor{currentfill}{rgb}{0.000000,0.000000,0.000000}%
\pgfsetfillcolor{currentfill}%
\pgfsetlinewidth{0.803000pt}%
\definecolor{currentstroke}{rgb}{0.000000,0.000000,0.000000}%
\pgfsetstrokecolor{currentstroke}%
\pgfsetdash{}{0pt}%
\pgfsys@defobject{currentmarker}{\pgfqpoint{-0.048611in}{0.000000in}}{\pgfqpoint{-0.000000in}{0.000000in}}{%
\pgfpathmoveto{\pgfqpoint{-0.000000in}{0.000000in}}%
\pgfpathlineto{\pgfqpoint{-0.048611in}{0.000000in}}%
\pgfusepath{stroke,fill}%
}%
\begin{pgfscope}%
\pgfsys@transformshift{0.750000in}{0.500000in}%
\pgfsys@useobject{currentmarker}{}%
\end{pgfscope}%
\end{pgfscope}%
\begin{pgfscope}%
\definecolor{textcolor}{rgb}{0.000000,0.000000,0.000000}%
\pgfsetstrokecolor{textcolor}%
\pgfsetfillcolor{textcolor}%
\pgftext[x=0.564412in, y=0.447238in, left, base]{\color{textcolor}\sffamily\fontsize{10.000000}{12.000000}\selectfont 0}%
\end{pgfscope}%
\begin{pgfscope}%
\pgfsetbuttcap%
\pgfsetroundjoin%
\definecolor{currentfill}{rgb}{0.000000,0.000000,0.000000}%
\pgfsetfillcolor{currentfill}%
\pgfsetlinewidth{0.803000pt}%
\definecolor{currentstroke}{rgb}{0.000000,0.000000,0.000000}%
\pgfsetstrokecolor{currentstroke}%
\pgfsetdash{}{0pt}%
\pgfsys@defobject{currentmarker}{\pgfqpoint{-0.048611in}{0.000000in}}{\pgfqpoint{-0.000000in}{0.000000in}}{%
\pgfpathmoveto{\pgfqpoint{-0.000000in}{0.000000in}}%
\pgfpathlineto{\pgfqpoint{-0.048611in}{0.000000in}}%
\pgfusepath{stroke,fill}%
}%
\begin{pgfscope}%
\pgfsys@transformshift{0.750000in}{1.022944in}%
\pgfsys@useobject{currentmarker}{}%
\end{pgfscope}%
\end{pgfscope}%
\begin{pgfscope}%
\definecolor{textcolor}{rgb}{0.000000,0.000000,0.000000}%
\pgfsetstrokecolor{textcolor}%
\pgfsetfillcolor{textcolor}%
\pgftext[x=0.476047in, y=0.970182in, left, base]{\color{textcolor}\sffamily\fontsize{10.000000}{12.000000}\selectfont 50}%
\end{pgfscope}%
\begin{pgfscope}%
\pgfsetbuttcap%
\pgfsetroundjoin%
\definecolor{currentfill}{rgb}{0.000000,0.000000,0.000000}%
\pgfsetfillcolor{currentfill}%
\pgfsetlinewidth{0.803000pt}%
\definecolor{currentstroke}{rgb}{0.000000,0.000000,0.000000}%
\pgfsetstrokecolor{currentstroke}%
\pgfsetdash{}{0pt}%
\pgfsys@defobject{currentmarker}{\pgfqpoint{-0.048611in}{0.000000in}}{\pgfqpoint{-0.000000in}{0.000000in}}{%
\pgfpathmoveto{\pgfqpoint{-0.000000in}{0.000000in}}%
\pgfpathlineto{\pgfqpoint{-0.048611in}{0.000000in}}%
\pgfusepath{stroke,fill}%
}%
\begin{pgfscope}%
\pgfsys@transformshift{0.750000in}{1.545887in}%
\pgfsys@useobject{currentmarker}{}%
\end{pgfscope}%
\end{pgfscope}%
\begin{pgfscope}%
\definecolor{textcolor}{rgb}{0.000000,0.000000,0.000000}%
\pgfsetstrokecolor{textcolor}%
\pgfsetfillcolor{textcolor}%
\pgftext[x=0.387682in, y=1.493126in, left, base]{\color{textcolor}\sffamily\fontsize{10.000000}{12.000000}\selectfont 100}%
\end{pgfscope}%
\begin{pgfscope}%
\pgfsetbuttcap%
\pgfsetroundjoin%
\definecolor{currentfill}{rgb}{0.000000,0.000000,0.000000}%
\pgfsetfillcolor{currentfill}%
\pgfsetlinewidth{0.803000pt}%
\definecolor{currentstroke}{rgb}{0.000000,0.000000,0.000000}%
\pgfsetstrokecolor{currentstroke}%
\pgfsetdash{}{0pt}%
\pgfsys@defobject{currentmarker}{\pgfqpoint{-0.048611in}{0.000000in}}{\pgfqpoint{-0.000000in}{0.000000in}}{%
\pgfpathmoveto{\pgfqpoint{-0.000000in}{0.000000in}}%
\pgfpathlineto{\pgfqpoint{-0.048611in}{0.000000in}}%
\pgfusepath{stroke,fill}%
}%
\begin{pgfscope}%
\pgfsys@transformshift{0.750000in}{2.068831in}%
\pgfsys@useobject{currentmarker}{}%
\end{pgfscope}%
\end{pgfscope}%
\begin{pgfscope}%
\definecolor{textcolor}{rgb}{0.000000,0.000000,0.000000}%
\pgfsetstrokecolor{textcolor}%
\pgfsetfillcolor{textcolor}%
\pgftext[x=0.387682in, y=2.016070in, left, base]{\color{textcolor}\sffamily\fontsize{10.000000}{12.000000}\selectfont 150}%
\end{pgfscope}%
\begin{pgfscope}%
\pgfsetbuttcap%
\pgfsetroundjoin%
\definecolor{currentfill}{rgb}{0.000000,0.000000,0.000000}%
\pgfsetfillcolor{currentfill}%
\pgfsetlinewidth{0.803000pt}%
\definecolor{currentstroke}{rgb}{0.000000,0.000000,0.000000}%
\pgfsetstrokecolor{currentstroke}%
\pgfsetdash{}{0pt}%
\pgfsys@defobject{currentmarker}{\pgfqpoint{-0.048611in}{0.000000in}}{\pgfqpoint{-0.000000in}{0.000000in}}{%
\pgfpathmoveto{\pgfqpoint{-0.000000in}{0.000000in}}%
\pgfpathlineto{\pgfqpoint{-0.048611in}{0.000000in}}%
\pgfusepath{stroke,fill}%
}%
\begin{pgfscope}%
\pgfsys@transformshift{0.750000in}{2.591775in}%
\pgfsys@useobject{currentmarker}{}%
\end{pgfscope}%
\end{pgfscope}%
\begin{pgfscope}%
\definecolor{textcolor}{rgb}{0.000000,0.000000,0.000000}%
\pgfsetstrokecolor{textcolor}%
\pgfsetfillcolor{textcolor}%
\pgftext[x=0.387682in, y=2.539013in, left, base]{\color{textcolor}\sffamily\fontsize{10.000000}{12.000000}\selectfont 200}%
\end{pgfscope}%
\begin{pgfscope}%
\pgfsetbuttcap%
\pgfsetroundjoin%
\definecolor{currentfill}{rgb}{0.000000,0.000000,0.000000}%
\pgfsetfillcolor{currentfill}%
\pgfsetlinewidth{0.803000pt}%
\definecolor{currentstroke}{rgb}{0.000000,0.000000,0.000000}%
\pgfsetstrokecolor{currentstroke}%
\pgfsetdash{}{0pt}%
\pgfsys@defobject{currentmarker}{\pgfqpoint{-0.048611in}{0.000000in}}{\pgfqpoint{-0.000000in}{0.000000in}}{%
\pgfpathmoveto{\pgfqpoint{-0.000000in}{0.000000in}}%
\pgfpathlineto{\pgfqpoint{-0.048611in}{0.000000in}}%
\pgfusepath{stroke,fill}%
}%
\begin{pgfscope}%
\pgfsys@transformshift{0.750000in}{3.114719in}%
\pgfsys@useobject{currentmarker}{}%
\end{pgfscope}%
\end{pgfscope}%
\begin{pgfscope}%
\definecolor{textcolor}{rgb}{0.000000,0.000000,0.000000}%
\pgfsetstrokecolor{textcolor}%
\pgfsetfillcolor{textcolor}%
\pgftext[x=0.387682in, y=3.061957in, left, base]{\color{textcolor}\sffamily\fontsize{10.000000}{12.000000}\selectfont 250}%
\end{pgfscope}%
\begin{pgfscope}%
\definecolor{textcolor}{rgb}{0.000000,0.000000,0.000000}%
\pgfsetstrokecolor{textcolor}%
\pgfsetfillcolor{textcolor}%
\pgftext[x=0.332126in,y=2.010000in,,bottom,rotate=90.000000]{\color{textcolor}\sffamily\fontsize{10.000000}{12.000000}\selectfont count}%
\end{pgfscope}%
\begin{pgfscope}%
\pgfsetrectcap%
\pgfsetmiterjoin%
\pgfsetlinewidth{0.803000pt}%
\definecolor{currentstroke}{rgb}{0.000000,0.000000,0.000000}%
\pgfsetstrokecolor{currentstroke}%
\pgfsetdash{}{0pt}%
\pgfpathmoveto{\pgfqpoint{0.750000in}{0.500000in}}%
\pgfpathlineto{\pgfqpoint{0.750000in}{3.520000in}}%
\pgfusepath{stroke}%
\end{pgfscope}%
\begin{pgfscope}%
\pgfsetrectcap%
\pgfsetmiterjoin%
\pgfsetlinewidth{0.803000pt}%
\definecolor{currentstroke}{rgb}{0.000000,0.000000,0.000000}%
\pgfsetstrokecolor{currentstroke}%
\pgfsetdash{}{0pt}%
\pgfpathmoveto{\pgfqpoint{5.400000in}{0.500000in}}%
\pgfpathlineto{\pgfqpoint{5.400000in}{3.520000in}}%
\pgfusepath{stroke}%
\end{pgfscope}%
\begin{pgfscope}%
\pgfsetrectcap%
\pgfsetmiterjoin%
\pgfsetlinewidth{0.803000pt}%
\definecolor{currentstroke}{rgb}{0.000000,0.000000,0.000000}%
\pgfsetstrokecolor{currentstroke}%
\pgfsetdash{}{0pt}%
\pgfpathmoveto{\pgfqpoint{0.750000in}{0.500000in}}%
\pgfpathlineto{\pgfqpoint{5.400000in}{0.500000in}}%
\pgfusepath{stroke}%
\end{pgfscope}%
\begin{pgfscope}%
\pgfsetrectcap%
\pgfsetmiterjoin%
\pgfsetlinewidth{0.803000pt}%
\definecolor{currentstroke}{rgb}{0.000000,0.000000,0.000000}%
\pgfsetstrokecolor{currentstroke}%
\pgfsetdash{}{0pt}%
\pgfpathmoveto{\pgfqpoint{0.750000in}{3.520000in}}%
\pgfpathlineto{\pgfqpoint{5.400000in}{3.520000in}}%
\pgfusepath{stroke}%
\end{pgfscope}%
\begin{pgfscope}%
\definecolor{textcolor}{rgb}{0.000000,0.000000,0.000000}%
\pgfsetstrokecolor{textcolor}%
\pgfsetfillcolor{textcolor}%
\pgftext[x=3.075000in,y=3.603333in,,base]{\color{textcolor}\sffamily\fontsize{12.000000}{14.400000}\selectfont loss over time for Rijndael's MixColumns function}%
\end{pgfscope}%
\begin{pgfscope}%
\pgfsetbuttcap%
\pgfsetmiterjoin%
\definecolor{currentfill}{rgb}{1.000000,1.000000,1.000000}%
\pgfsetfillcolor{currentfill}%
\pgfsetfillopacity{0.800000}%
\pgfsetlinewidth{1.003750pt}%
\definecolor{currentstroke}{rgb}{0.800000,0.800000,0.800000}%
\pgfsetstrokecolor{currentstroke}%
\pgfsetstrokeopacity{0.800000}%
\pgfsetdash{}{0pt}%
\pgfpathmoveto{\pgfqpoint{4.562381in}{3.205032in}}%
\pgfpathlineto{\pgfqpoint{5.302778in}{3.205032in}}%
\pgfpathquadraticcurveto{\pgfqpoint{5.330556in}{3.205032in}}{\pgfqpoint{5.330556in}{3.232809in}}%
\pgfpathlineto{\pgfqpoint{5.330556in}{3.422778in}}%
\pgfpathquadraticcurveto{\pgfqpoint{5.330556in}{3.450556in}}{\pgfqpoint{5.302778in}{3.450556in}}%
\pgfpathlineto{\pgfqpoint{4.562381in}{3.450556in}}%
\pgfpathquadraticcurveto{\pgfqpoint{4.534603in}{3.450556in}}{\pgfqpoint{4.534603in}{3.422778in}}%
\pgfpathlineto{\pgfqpoint{4.534603in}{3.232809in}}%
\pgfpathquadraticcurveto{\pgfqpoint{4.534603in}{3.205032in}}{\pgfqpoint{4.562381in}{3.205032in}}%
\pgfpathlineto{\pgfqpoint{4.562381in}{3.205032in}}%
\pgfpathclose%
\pgfusepath{stroke,fill}%
\end{pgfscope}%
\begin{pgfscope}%
\pgfsetbuttcap%
\pgfsetmiterjoin%
\definecolor{currentfill}{rgb}{1.000000,0.000000,0.000000}%
\pgfsetfillcolor{currentfill}%
\pgfsetlinewidth{0.000000pt}%
\definecolor{currentstroke}{rgb}{0.000000,0.000000,0.000000}%
\pgfsetstrokecolor{currentstroke}%
\pgfsetstrokeopacity{0.000000}%
\pgfsetdash{}{0pt}%
\pgfpathmoveto{\pgfqpoint{4.590158in}{3.289477in}}%
\pgfpathlineto{\pgfqpoint{4.867936in}{3.289477in}}%
\pgfpathlineto{\pgfqpoint{4.867936in}{3.386699in}}%
\pgfpathlineto{\pgfqpoint{4.590158in}{3.386699in}}%
\pgfpathlineto{\pgfqpoint{4.590158in}{3.289477in}}%
\pgfpathclose%
\pgfusepath{fill}%
\end{pgfscope}%
\begin{pgfscope}%
\definecolor{textcolor}{rgb}{0.000000,0.000000,0.000000}%
\pgfsetstrokecolor{textcolor}%
\pgfsetfillcolor{textcolor}%
\pgftext[x=4.979047in,y=3.289477in,left,base]{\color{textcolor}\sffamily\fontsize{10.000000}{12.000000}\selectfont SNN}%
\end{pgfscope}%
\end{pgfpicture}%
\makeatother%
\endgroup%

    \caption{Caption}
    \label{fig:my_label}
\end{figure}

\begin{figure}
%% Creator: Matplotlib, PGF backend
%%
%% To include the figure in your LaTeX document, write
%%   \input{<filename>.pgf}
%%
%% Make sure the required packages are loaded in your preamble
%%   \usepackage{pgf}
%%
%% Also ensure that all the required font packages are loaded; for instance,
%% the lmodern package is sometimes necessary when using math font.
%%   \usepackage{lmodern}
%%
%% Figures using additional raster images can only be included by \input if
%% they are in the same directory as the main LaTeX file. For loading figures
%% from other directories you can use the `import` package
%%   \usepackage{import}
%%
%% and then include the figures with
%%   \import{<path to file>}{<filename>.pgf}
%%
%% Matplotlib used the following preamble
%%   \usepackage{fontspec}
%%   \setmainfont{DejaVuSerif.ttf}[Path=\detokenize{C:/I/python38/Lib/site-packages/matplotlib/mpl-data/fonts/ttf/}]
%%   \setsansfont{DejaVuSans.ttf}[Path=\detokenize{C:/I/python38/Lib/site-packages/matplotlib/mpl-data/fonts/ttf/}]
%%   \setmonofont{DejaVuSansMono.ttf}[Path=\detokenize{C:/I/python38/Lib/site-packages/matplotlib/mpl-data/fonts/ttf/}]
%%
\begingroup%
\makeatletter%
\begin{pgfpicture}%
\pgfpathrectangle{\pgfpointorigin}{\pgfqpoint{6.000000in}{4.000000in}}%
\pgfusepath{use as bounding box, clip}%
\begin{pgfscope}%
\pgfsetbuttcap%
\pgfsetmiterjoin%
\pgfsetlinewidth{0.000000pt}%
\definecolor{currentstroke}{rgb}{1.000000,1.000000,1.000000}%
\pgfsetstrokecolor{currentstroke}%
\pgfsetstrokeopacity{0.000000}%
\pgfsetdash{}{0pt}%
\pgfpathmoveto{\pgfqpoint{0.000000in}{0.000000in}}%
\pgfpathlineto{\pgfqpoint{6.000000in}{0.000000in}}%
\pgfpathlineto{\pgfqpoint{6.000000in}{4.000000in}}%
\pgfpathlineto{\pgfqpoint{0.000000in}{4.000000in}}%
\pgfpathlineto{\pgfqpoint{0.000000in}{0.000000in}}%
\pgfpathclose%
\pgfusepath{}%
\end{pgfscope}%
\begin{pgfscope}%
\pgfsetbuttcap%
\pgfsetmiterjoin%
\definecolor{currentfill}{rgb}{1.000000,1.000000,1.000000}%
\pgfsetfillcolor{currentfill}%
\pgfsetlinewidth{0.000000pt}%
\definecolor{currentstroke}{rgb}{0.000000,0.000000,0.000000}%
\pgfsetstrokecolor{currentstroke}%
\pgfsetstrokeopacity{0.000000}%
\pgfsetdash{}{0pt}%
\pgfpathmoveto{\pgfqpoint{0.750000in}{0.500000in}}%
\pgfpathlineto{\pgfqpoint{5.400000in}{0.500000in}}%
\pgfpathlineto{\pgfqpoint{5.400000in}{3.520000in}}%
\pgfpathlineto{\pgfqpoint{0.750000in}{3.520000in}}%
\pgfpathlineto{\pgfqpoint{0.750000in}{0.500000in}}%
\pgfpathclose%
\pgfusepath{fill}%
\end{pgfscope}%
\begin{pgfscope}%
\pgfpathrectangle{\pgfqpoint{0.750000in}{0.500000in}}{\pgfqpoint{4.650000in}{3.020000in}}%
\pgfusepath{clip}%
\pgfsetbuttcap%
\pgfsetmiterjoin%
\definecolor{currentfill}{rgb}{0.000000,0.500000,0.000000}%
\pgfsetfillcolor{currentfill}%
\pgfsetlinewidth{0.000000pt}%
\definecolor{currentstroke}{rgb}{0.000000,0.000000,0.000000}%
\pgfsetstrokecolor{currentstroke}%
\pgfsetstrokeopacity{0.000000}%
\pgfsetdash{}{0pt}%
\pgfpathmoveto{\pgfqpoint{0.961364in}{0.500000in}}%
\pgfpathlineto{\pgfqpoint{0.994389in}{0.500000in}}%
\pgfpathlineto{\pgfqpoint{0.994389in}{0.515979in}}%
\pgfpathlineto{\pgfqpoint{0.961364in}{0.515979in}}%
\pgfpathlineto{\pgfqpoint{0.961364in}{0.500000in}}%
\pgfpathclose%
\pgfusepath{fill}%
\end{pgfscope}%
\begin{pgfscope}%
\pgfpathrectangle{\pgfqpoint{0.750000in}{0.500000in}}{\pgfqpoint{4.650000in}{3.020000in}}%
\pgfusepath{clip}%
\pgfsetbuttcap%
\pgfsetmiterjoin%
\definecolor{currentfill}{rgb}{0.000000,0.500000,0.000000}%
\pgfsetfillcolor{currentfill}%
\pgfsetlinewidth{0.000000pt}%
\definecolor{currentstroke}{rgb}{0.000000,0.000000,0.000000}%
\pgfsetstrokecolor{currentstroke}%
\pgfsetstrokeopacity{0.000000}%
\pgfsetdash{}{0pt}%
\pgfpathmoveto{\pgfqpoint{0.994389in}{0.500000in}}%
\pgfpathlineto{\pgfqpoint{1.027415in}{0.500000in}}%
\pgfpathlineto{\pgfqpoint{1.027415in}{0.500000in}}%
\pgfpathlineto{\pgfqpoint{0.994389in}{0.500000in}}%
\pgfpathlineto{\pgfqpoint{0.994389in}{0.500000in}}%
\pgfpathclose%
\pgfusepath{fill}%
\end{pgfscope}%
\begin{pgfscope}%
\pgfpathrectangle{\pgfqpoint{0.750000in}{0.500000in}}{\pgfqpoint{4.650000in}{3.020000in}}%
\pgfusepath{clip}%
\pgfsetbuttcap%
\pgfsetmiterjoin%
\definecolor{currentfill}{rgb}{0.000000,0.500000,0.000000}%
\pgfsetfillcolor{currentfill}%
\pgfsetlinewidth{0.000000pt}%
\definecolor{currentstroke}{rgb}{0.000000,0.000000,0.000000}%
\pgfsetstrokecolor{currentstroke}%
\pgfsetstrokeopacity{0.000000}%
\pgfsetdash{}{0pt}%
\pgfpathmoveto{\pgfqpoint{1.027415in}{0.500000in}}%
\pgfpathlineto{\pgfqpoint{1.060440in}{0.500000in}}%
\pgfpathlineto{\pgfqpoint{1.060440in}{0.500000in}}%
\pgfpathlineto{\pgfqpoint{1.027415in}{0.500000in}}%
\pgfpathlineto{\pgfqpoint{1.027415in}{0.500000in}}%
\pgfpathclose%
\pgfusepath{fill}%
\end{pgfscope}%
\begin{pgfscope}%
\pgfpathrectangle{\pgfqpoint{0.750000in}{0.500000in}}{\pgfqpoint{4.650000in}{3.020000in}}%
\pgfusepath{clip}%
\pgfsetbuttcap%
\pgfsetmiterjoin%
\definecolor{currentfill}{rgb}{0.000000,0.500000,0.000000}%
\pgfsetfillcolor{currentfill}%
\pgfsetlinewidth{0.000000pt}%
\definecolor{currentstroke}{rgb}{0.000000,0.000000,0.000000}%
\pgfsetstrokecolor{currentstroke}%
\pgfsetstrokeopacity{0.000000}%
\pgfsetdash{}{0pt}%
\pgfpathmoveto{\pgfqpoint{1.060440in}{0.500000in}}%
\pgfpathlineto{\pgfqpoint{1.093466in}{0.500000in}}%
\pgfpathlineto{\pgfqpoint{1.093466in}{0.500000in}}%
\pgfpathlineto{\pgfqpoint{1.060440in}{0.500000in}}%
\pgfpathlineto{\pgfqpoint{1.060440in}{0.500000in}}%
\pgfpathclose%
\pgfusepath{fill}%
\end{pgfscope}%
\begin{pgfscope}%
\pgfpathrectangle{\pgfqpoint{0.750000in}{0.500000in}}{\pgfqpoint{4.650000in}{3.020000in}}%
\pgfusepath{clip}%
\pgfsetbuttcap%
\pgfsetmiterjoin%
\definecolor{currentfill}{rgb}{0.000000,0.500000,0.000000}%
\pgfsetfillcolor{currentfill}%
\pgfsetlinewidth{0.000000pt}%
\definecolor{currentstroke}{rgb}{0.000000,0.000000,0.000000}%
\pgfsetstrokecolor{currentstroke}%
\pgfsetstrokeopacity{0.000000}%
\pgfsetdash{}{0pt}%
\pgfpathmoveto{\pgfqpoint{1.093466in}{0.500000in}}%
\pgfpathlineto{\pgfqpoint{1.126491in}{0.500000in}}%
\pgfpathlineto{\pgfqpoint{1.126491in}{0.500000in}}%
\pgfpathlineto{\pgfqpoint{1.093466in}{0.500000in}}%
\pgfpathlineto{\pgfqpoint{1.093466in}{0.500000in}}%
\pgfpathclose%
\pgfusepath{fill}%
\end{pgfscope}%
\begin{pgfscope}%
\pgfpathrectangle{\pgfqpoint{0.750000in}{0.500000in}}{\pgfqpoint{4.650000in}{3.020000in}}%
\pgfusepath{clip}%
\pgfsetbuttcap%
\pgfsetmiterjoin%
\definecolor{currentfill}{rgb}{0.000000,0.500000,0.000000}%
\pgfsetfillcolor{currentfill}%
\pgfsetlinewidth{0.000000pt}%
\definecolor{currentstroke}{rgb}{0.000000,0.000000,0.000000}%
\pgfsetstrokecolor{currentstroke}%
\pgfsetstrokeopacity{0.000000}%
\pgfsetdash{}{0pt}%
\pgfpathmoveto{\pgfqpoint{1.126491in}{0.500000in}}%
\pgfpathlineto{\pgfqpoint{1.159517in}{0.500000in}}%
\pgfpathlineto{\pgfqpoint{1.159517in}{0.500000in}}%
\pgfpathlineto{\pgfqpoint{1.126491in}{0.500000in}}%
\pgfpathlineto{\pgfqpoint{1.126491in}{0.500000in}}%
\pgfpathclose%
\pgfusepath{fill}%
\end{pgfscope}%
\begin{pgfscope}%
\pgfpathrectangle{\pgfqpoint{0.750000in}{0.500000in}}{\pgfqpoint{4.650000in}{3.020000in}}%
\pgfusepath{clip}%
\pgfsetbuttcap%
\pgfsetmiterjoin%
\definecolor{currentfill}{rgb}{0.000000,0.500000,0.000000}%
\pgfsetfillcolor{currentfill}%
\pgfsetlinewidth{0.000000pt}%
\definecolor{currentstroke}{rgb}{0.000000,0.000000,0.000000}%
\pgfsetstrokecolor{currentstroke}%
\pgfsetstrokeopacity{0.000000}%
\pgfsetdash{}{0pt}%
\pgfpathmoveto{\pgfqpoint{1.159517in}{0.500000in}}%
\pgfpathlineto{\pgfqpoint{1.192543in}{0.500000in}}%
\pgfpathlineto{\pgfqpoint{1.192543in}{0.500000in}}%
\pgfpathlineto{\pgfqpoint{1.159517in}{0.500000in}}%
\pgfpathlineto{\pgfqpoint{1.159517in}{0.500000in}}%
\pgfpathclose%
\pgfusepath{fill}%
\end{pgfscope}%
\begin{pgfscope}%
\pgfpathrectangle{\pgfqpoint{0.750000in}{0.500000in}}{\pgfqpoint{4.650000in}{3.020000in}}%
\pgfusepath{clip}%
\pgfsetbuttcap%
\pgfsetmiterjoin%
\definecolor{currentfill}{rgb}{0.000000,0.500000,0.000000}%
\pgfsetfillcolor{currentfill}%
\pgfsetlinewidth{0.000000pt}%
\definecolor{currentstroke}{rgb}{0.000000,0.000000,0.000000}%
\pgfsetstrokecolor{currentstroke}%
\pgfsetstrokeopacity{0.000000}%
\pgfsetdash{}{0pt}%
\pgfpathmoveto{\pgfqpoint{1.192543in}{0.500000in}}%
\pgfpathlineto{\pgfqpoint{1.225568in}{0.500000in}}%
\pgfpathlineto{\pgfqpoint{1.225568in}{0.500000in}}%
\pgfpathlineto{\pgfqpoint{1.192543in}{0.500000in}}%
\pgfpathlineto{\pgfqpoint{1.192543in}{0.500000in}}%
\pgfpathclose%
\pgfusepath{fill}%
\end{pgfscope}%
\begin{pgfscope}%
\pgfpathrectangle{\pgfqpoint{0.750000in}{0.500000in}}{\pgfqpoint{4.650000in}{3.020000in}}%
\pgfusepath{clip}%
\pgfsetbuttcap%
\pgfsetmiterjoin%
\definecolor{currentfill}{rgb}{0.000000,0.500000,0.000000}%
\pgfsetfillcolor{currentfill}%
\pgfsetlinewidth{0.000000pt}%
\definecolor{currentstroke}{rgb}{0.000000,0.000000,0.000000}%
\pgfsetstrokecolor{currentstroke}%
\pgfsetstrokeopacity{0.000000}%
\pgfsetdash{}{0pt}%
\pgfpathmoveto{\pgfqpoint{1.225568in}{0.500000in}}%
\pgfpathlineto{\pgfqpoint{1.258594in}{0.500000in}}%
\pgfpathlineto{\pgfqpoint{1.258594in}{0.500000in}}%
\pgfpathlineto{\pgfqpoint{1.225568in}{0.500000in}}%
\pgfpathlineto{\pgfqpoint{1.225568in}{0.500000in}}%
\pgfpathclose%
\pgfusepath{fill}%
\end{pgfscope}%
\begin{pgfscope}%
\pgfpathrectangle{\pgfqpoint{0.750000in}{0.500000in}}{\pgfqpoint{4.650000in}{3.020000in}}%
\pgfusepath{clip}%
\pgfsetbuttcap%
\pgfsetmiterjoin%
\definecolor{currentfill}{rgb}{0.000000,0.500000,0.000000}%
\pgfsetfillcolor{currentfill}%
\pgfsetlinewidth{0.000000pt}%
\definecolor{currentstroke}{rgb}{0.000000,0.000000,0.000000}%
\pgfsetstrokecolor{currentstroke}%
\pgfsetstrokeopacity{0.000000}%
\pgfsetdash{}{0pt}%
\pgfpathmoveto{\pgfqpoint{1.258594in}{0.500000in}}%
\pgfpathlineto{\pgfqpoint{1.291619in}{0.500000in}}%
\pgfpathlineto{\pgfqpoint{1.291619in}{0.500000in}}%
\pgfpathlineto{\pgfqpoint{1.258594in}{0.500000in}}%
\pgfpathlineto{\pgfqpoint{1.258594in}{0.500000in}}%
\pgfpathclose%
\pgfusepath{fill}%
\end{pgfscope}%
\begin{pgfscope}%
\pgfpathrectangle{\pgfqpoint{0.750000in}{0.500000in}}{\pgfqpoint{4.650000in}{3.020000in}}%
\pgfusepath{clip}%
\pgfsetbuttcap%
\pgfsetmiterjoin%
\definecolor{currentfill}{rgb}{0.000000,0.500000,0.000000}%
\pgfsetfillcolor{currentfill}%
\pgfsetlinewidth{0.000000pt}%
\definecolor{currentstroke}{rgb}{0.000000,0.000000,0.000000}%
\pgfsetstrokecolor{currentstroke}%
\pgfsetstrokeopacity{0.000000}%
\pgfsetdash{}{0pt}%
\pgfpathmoveto{\pgfqpoint{1.291619in}{0.500000in}}%
\pgfpathlineto{\pgfqpoint{1.324645in}{0.500000in}}%
\pgfpathlineto{\pgfqpoint{1.324645in}{0.500000in}}%
\pgfpathlineto{\pgfqpoint{1.291619in}{0.500000in}}%
\pgfpathlineto{\pgfqpoint{1.291619in}{0.500000in}}%
\pgfpathclose%
\pgfusepath{fill}%
\end{pgfscope}%
\begin{pgfscope}%
\pgfpathrectangle{\pgfqpoint{0.750000in}{0.500000in}}{\pgfqpoint{4.650000in}{3.020000in}}%
\pgfusepath{clip}%
\pgfsetbuttcap%
\pgfsetmiterjoin%
\definecolor{currentfill}{rgb}{0.000000,0.500000,0.000000}%
\pgfsetfillcolor{currentfill}%
\pgfsetlinewidth{0.000000pt}%
\definecolor{currentstroke}{rgb}{0.000000,0.000000,0.000000}%
\pgfsetstrokecolor{currentstroke}%
\pgfsetstrokeopacity{0.000000}%
\pgfsetdash{}{0pt}%
\pgfpathmoveto{\pgfqpoint{1.324645in}{0.500000in}}%
\pgfpathlineto{\pgfqpoint{1.357670in}{0.500000in}}%
\pgfpathlineto{\pgfqpoint{1.357670in}{0.500000in}}%
\pgfpathlineto{\pgfqpoint{1.324645in}{0.500000in}}%
\pgfpathlineto{\pgfqpoint{1.324645in}{0.500000in}}%
\pgfpathclose%
\pgfusepath{fill}%
\end{pgfscope}%
\begin{pgfscope}%
\pgfpathrectangle{\pgfqpoint{0.750000in}{0.500000in}}{\pgfqpoint{4.650000in}{3.020000in}}%
\pgfusepath{clip}%
\pgfsetbuttcap%
\pgfsetmiterjoin%
\definecolor{currentfill}{rgb}{0.000000,0.500000,0.000000}%
\pgfsetfillcolor{currentfill}%
\pgfsetlinewidth{0.000000pt}%
\definecolor{currentstroke}{rgb}{0.000000,0.000000,0.000000}%
\pgfsetstrokecolor{currentstroke}%
\pgfsetstrokeopacity{0.000000}%
\pgfsetdash{}{0pt}%
\pgfpathmoveto{\pgfqpoint{1.357670in}{0.500000in}}%
\pgfpathlineto{\pgfqpoint{1.390696in}{0.500000in}}%
\pgfpathlineto{\pgfqpoint{1.390696in}{0.515979in}}%
\pgfpathlineto{\pgfqpoint{1.357670in}{0.515979in}}%
\pgfpathlineto{\pgfqpoint{1.357670in}{0.500000in}}%
\pgfpathclose%
\pgfusepath{fill}%
\end{pgfscope}%
\begin{pgfscope}%
\pgfpathrectangle{\pgfqpoint{0.750000in}{0.500000in}}{\pgfqpoint{4.650000in}{3.020000in}}%
\pgfusepath{clip}%
\pgfsetbuttcap%
\pgfsetmiterjoin%
\definecolor{currentfill}{rgb}{0.000000,0.500000,0.000000}%
\pgfsetfillcolor{currentfill}%
\pgfsetlinewidth{0.000000pt}%
\definecolor{currentstroke}{rgb}{0.000000,0.000000,0.000000}%
\pgfsetstrokecolor{currentstroke}%
\pgfsetstrokeopacity{0.000000}%
\pgfsetdash{}{0pt}%
\pgfpathmoveto{\pgfqpoint{1.390696in}{0.500000in}}%
\pgfpathlineto{\pgfqpoint{1.423722in}{0.500000in}}%
\pgfpathlineto{\pgfqpoint{1.423722in}{0.500000in}}%
\pgfpathlineto{\pgfqpoint{1.390696in}{0.500000in}}%
\pgfpathlineto{\pgfqpoint{1.390696in}{0.500000in}}%
\pgfpathclose%
\pgfusepath{fill}%
\end{pgfscope}%
\begin{pgfscope}%
\pgfpathrectangle{\pgfqpoint{0.750000in}{0.500000in}}{\pgfqpoint{4.650000in}{3.020000in}}%
\pgfusepath{clip}%
\pgfsetbuttcap%
\pgfsetmiterjoin%
\definecolor{currentfill}{rgb}{0.000000,0.500000,0.000000}%
\pgfsetfillcolor{currentfill}%
\pgfsetlinewidth{0.000000pt}%
\definecolor{currentstroke}{rgb}{0.000000,0.000000,0.000000}%
\pgfsetstrokecolor{currentstroke}%
\pgfsetstrokeopacity{0.000000}%
\pgfsetdash{}{0pt}%
\pgfpathmoveto{\pgfqpoint{1.423722in}{0.500000in}}%
\pgfpathlineto{\pgfqpoint{1.456747in}{0.500000in}}%
\pgfpathlineto{\pgfqpoint{1.456747in}{0.500000in}}%
\pgfpathlineto{\pgfqpoint{1.423722in}{0.500000in}}%
\pgfpathlineto{\pgfqpoint{1.423722in}{0.500000in}}%
\pgfpathclose%
\pgfusepath{fill}%
\end{pgfscope}%
\begin{pgfscope}%
\pgfpathrectangle{\pgfqpoint{0.750000in}{0.500000in}}{\pgfqpoint{4.650000in}{3.020000in}}%
\pgfusepath{clip}%
\pgfsetbuttcap%
\pgfsetmiterjoin%
\definecolor{currentfill}{rgb}{0.000000,0.500000,0.000000}%
\pgfsetfillcolor{currentfill}%
\pgfsetlinewidth{0.000000pt}%
\definecolor{currentstroke}{rgb}{0.000000,0.000000,0.000000}%
\pgfsetstrokecolor{currentstroke}%
\pgfsetstrokeopacity{0.000000}%
\pgfsetdash{}{0pt}%
\pgfpathmoveto{\pgfqpoint{1.456747in}{0.500000in}}%
\pgfpathlineto{\pgfqpoint{1.489773in}{0.500000in}}%
\pgfpathlineto{\pgfqpoint{1.489773in}{0.500000in}}%
\pgfpathlineto{\pgfqpoint{1.456747in}{0.500000in}}%
\pgfpathlineto{\pgfqpoint{1.456747in}{0.500000in}}%
\pgfpathclose%
\pgfusepath{fill}%
\end{pgfscope}%
\begin{pgfscope}%
\pgfpathrectangle{\pgfqpoint{0.750000in}{0.500000in}}{\pgfqpoint{4.650000in}{3.020000in}}%
\pgfusepath{clip}%
\pgfsetbuttcap%
\pgfsetmiterjoin%
\definecolor{currentfill}{rgb}{0.000000,0.500000,0.000000}%
\pgfsetfillcolor{currentfill}%
\pgfsetlinewidth{0.000000pt}%
\definecolor{currentstroke}{rgb}{0.000000,0.000000,0.000000}%
\pgfsetstrokecolor{currentstroke}%
\pgfsetstrokeopacity{0.000000}%
\pgfsetdash{}{0pt}%
\pgfpathmoveto{\pgfqpoint{1.489773in}{0.500000in}}%
\pgfpathlineto{\pgfqpoint{1.522798in}{0.500000in}}%
\pgfpathlineto{\pgfqpoint{1.522798in}{0.500000in}}%
\pgfpathlineto{\pgfqpoint{1.489773in}{0.500000in}}%
\pgfpathlineto{\pgfqpoint{1.489773in}{0.500000in}}%
\pgfpathclose%
\pgfusepath{fill}%
\end{pgfscope}%
\begin{pgfscope}%
\pgfpathrectangle{\pgfqpoint{0.750000in}{0.500000in}}{\pgfqpoint{4.650000in}{3.020000in}}%
\pgfusepath{clip}%
\pgfsetbuttcap%
\pgfsetmiterjoin%
\definecolor{currentfill}{rgb}{0.000000,0.500000,0.000000}%
\pgfsetfillcolor{currentfill}%
\pgfsetlinewidth{0.000000pt}%
\definecolor{currentstroke}{rgb}{0.000000,0.000000,0.000000}%
\pgfsetstrokecolor{currentstroke}%
\pgfsetstrokeopacity{0.000000}%
\pgfsetdash{}{0pt}%
\pgfpathmoveto{\pgfqpoint{1.522798in}{0.500000in}}%
\pgfpathlineto{\pgfqpoint{1.555824in}{0.500000in}}%
\pgfpathlineto{\pgfqpoint{1.555824in}{0.515979in}}%
\pgfpathlineto{\pgfqpoint{1.522798in}{0.515979in}}%
\pgfpathlineto{\pgfqpoint{1.522798in}{0.500000in}}%
\pgfpathclose%
\pgfusepath{fill}%
\end{pgfscope}%
\begin{pgfscope}%
\pgfpathrectangle{\pgfqpoint{0.750000in}{0.500000in}}{\pgfqpoint{4.650000in}{3.020000in}}%
\pgfusepath{clip}%
\pgfsetbuttcap%
\pgfsetmiterjoin%
\definecolor{currentfill}{rgb}{0.000000,0.500000,0.000000}%
\pgfsetfillcolor{currentfill}%
\pgfsetlinewidth{0.000000pt}%
\definecolor{currentstroke}{rgb}{0.000000,0.000000,0.000000}%
\pgfsetstrokecolor{currentstroke}%
\pgfsetstrokeopacity{0.000000}%
\pgfsetdash{}{0pt}%
\pgfpathmoveto{\pgfqpoint{1.555824in}{0.500000in}}%
\pgfpathlineto{\pgfqpoint{1.588849in}{0.500000in}}%
\pgfpathlineto{\pgfqpoint{1.588849in}{0.515979in}}%
\pgfpathlineto{\pgfqpoint{1.555824in}{0.515979in}}%
\pgfpathlineto{\pgfqpoint{1.555824in}{0.500000in}}%
\pgfpathclose%
\pgfusepath{fill}%
\end{pgfscope}%
\begin{pgfscope}%
\pgfpathrectangle{\pgfqpoint{0.750000in}{0.500000in}}{\pgfqpoint{4.650000in}{3.020000in}}%
\pgfusepath{clip}%
\pgfsetbuttcap%
\pgfsetmiterjoin%
\definecolor{currentfill}{rgb}{0.000000,0.500000,0.000000}%
\pgfsetfillcolor{currentfill}%
\pgfsetlinewidth{0.000000pt}%
\definecolor{currentstroke}{rgb}{0.000000,0.000000,0.000000}%
\pgfsetstrokecolor{currentstroke}%
\pgfsetstrokeopacity{0.000000}%
\pgfsetdash{}{0pt}%
\pgfpathmoveto{\pgfqpoint{1.588849in}{0.500000in}}%
\pgfpathlineto{\pgfqpoint{1.621875in}{0.500000in}}%
\pgfpathlineto{\pgfqpoint{1.621875in}{0.515979in}}%
\pgfpathlineto{\pgfqpoint{1.588849in}{0.515979in}}%
\pgfpathlineto{\pgfqpoint{1.588849in}{0.500000in}}%
\pgfpathclose%
\pgfusepath{fill}%
\end{pgfscope}%
\begin{pgfscope}%
\pgfpathrectangle{\pgfqpoint{0.750000in}{0.500000in}}{\pgfqpoint{4.650000in}{3.020000in}}%
\pgfusepath{clip}%
\pgfsetbuttcap%
\pgfsetmiterjoin%
\definecolor{currentfill}{rgb}{0.000000,0.500000,0.000000}%
\pgfsetfillcolor{currentfill}%
\pgfsetlinewidth{0.000000pt}%
\definecolor{currentstroke}{rgb}{0.000000,0.000000,0.000000}%
\pgfsetstrokecolor{currentstroke}%
\pgfsetstrokeopacity{0.000000}%
\pgfsetdash{}{0pt}%
\pgfpathmoveto{\pgfqpoint{1.621875in}{0.500000in}}%
\pgfpathlineto{\pgfqpoint{1.654901in}{0.500000in}}%
\pgfpathlineto{\pgfqpoint{1.654901in}{0.500000in}}%
\pgfpathlineto{\pgfqpoint{1.621875in}{0.500000in}}%
\pgfpathlineto{\pgfqpoint{1.621875in}{0.500000in}}%
\pgfpathclose%
\pgfusepath{fill}%
\end{pgfscope}%
\begin{pgfscope}%
\pgfpathrectangle{\pgfqpoint{0.750000in}{0.500000in}}{\pgfqpoint{4.650000in}{3.020000in}}%
\pgfusepath{clip}%
\pgfsetbuttcap%
\pgfsetmiterjoin%
\definecolor{currentfill}{rgb}{0.000000,0.500000,0.000000}%
\pgfsetfillcolor{currentfill}%
\pgfsetlinewidth{0.000000pt}%
\definecolor{currentstroke}{rgb}{0.000000,0.000000,0.000000}%
\pgfsetstrokecolor{currentstroke}%
\pgfsetstrokeopacity{0.000000}%
\pgfsetdash{}{0pt}%
\pgfpathmoveto{\pgfqpoint{1.654901in}{0.500000in}}%
\pgfpathlineto{\pgfqpoint{1.687926in}{0.500000in}}%
\pgfpathlineto{\pgfqpoint{1.687926in}{0.500000in}}%
\pgfpathlineto{\pgfqpoint{1.654901in}{0.500000in}}%
\pgfpathlineto{\pgfqpoint{1.654901in}{0.500000in}}%
\pgfpathclose%
\pgfusepath{fill}%
\end{pgfscope}%
\begin{pgfscope}%
\pgfpathrectangle{\pgfqpoint{0.750000in}{0.500000in}}{\pgfqpoint{4.650000in}{3.020000in}}%
\pgfusepath{clip}%
\pgfsetbuttcap%
\pgfsetmiterjoin%
\definecolor{currentfill}{rgb}{0.000000,0.500000,0.000000}%
\pgfsetfillcolor{currentfill}%
\pgfsetlinewidth{0.000000pt}%
\definecolor{currentstroke}{rgb}{0.000000,0.000000,0.000000}%
\pgfsetstrokecolor{currentstroke}%
\pgfsetstrokeopacity{0.000000}%
\pgfsetdash{}{0pt}%
\pgfpathmoveto{\pgfqpoint{1.687926in}{0.500000in}}%
\pgfpathlineto{\pgfqpoint{1.720952in}{0.500000in}}%
\pgfpathlineto{\pgfqpoint{1.720952in}{0.515979in}}%
\pgfpathlineto{\pgfqpoint{1.687926in}{0.515979in}}%
\pgfpathlineto{\pgfqpoint{1.687926in}{0.500000in}}%
\pgfpathclose%
\pgfusepath{fill}%
\end{pgfscope}%
\begin{pgfscope}%
\pgfpathrectangle{\pgfqpoint{0.750000in}{0.500000in}}{\pgfqpoint{4.650000in}{3.020000in}}%
\pgfusepath{clip}%
\pgfsetbuttcap%
\pgfsetmiterjoin%
\definecolor{currentfill}{rgb}{0.000000,0.500000,0.000000}%
\pgfsetfillcolor{currentfill}%
\pgfsetlinewidth{0.000000pt}%
\definecolor{currentstroke}{rgb}{0.000000,0.000000,0.000000}%
\pgfsetstrokecolor{currentstroke}%
\pgfsetstrokeopacity{0.000000}%
\pgfsetdash{}{0pt}%
\pgfpathmoveto{\pgfqpoint{1.720952in}{0.500000in}}%
\pgfpathlineto{\pgfqpoint{1.753977in}{0.500000in}}%
\pgfpathlineto{\pgfqpoint{1.753977in}{0.515979in}}%
\pgfpathlineto{\pgfqpoint{1.720952in}{0.515979in}}%
\pgfpathlineto{\pgfqpoint{1.720952in}{0.500000in}}%
\pgfpathclose%
\pgfusepath{fill}%
\end{pgfscope}%
\begin{pgfscope}%
\pgfpathrectangle{\pgfqpoint{0.750000in}{0.500000in}}{\pgfqpoint{4.650000in}{3.020000in}}%
\pgfusepath{clip}%
\pgfsetbuttcap%
\pgfsetmiterjoin%
\definecolor{currentfill}{rgb}{0.000000,0.500000,0.000000}%
\pgfsetfillcolor{currentfill}%
\pgfsetlinewidth{0.000000pt}%
\definecolor{currentstroke}{rgb}{0.000000,0.000000,0.000000}%
\pgfsetstrokecolor{currentstroke}%
\pgfsetstrokeopacity{0.000000}%
\pgfsetdash{}{0pt}%
\pgfpathmoveto{\pgfqpoint{1.753977in}{0.500000in}}%
\pgfpathlineto{\pgfqpoint{1.787003in}{0.500000in}}%
\pgfpathlineto{\pgfqpoint{1.787003in}{0.531958in}}%
\pgfpathlineto{\pgfqpoint{1.753977in}{0.531958in}}%
\pgfpathlineto{\pgfqpoint{1.753977in}{0.500000in}}%
\pgfpathclose%
\pgfusepath{fill}%
\end{pgfscope}%
\begin{pgfscope}%
\pgfpathrectangle{\pgfqpoint{0.750000in}{0.500000in}}{\pgfqpoint{4.650000in}{3.020000in}}%
\pgfusepath{clip}%
\pgfsetbuttcap%
\pgfsetmiterjoin%
\definecolor{currentfill}{rgb}{0.000000,0.500000,0.000000}%
\pgfsetfillcolor{currentfill}%
\pgfsetlinewidth{0.000000pt}%
\definecolor{currentstroke}{rgb}{0.000000,0.000000,0.000000}%
\pgfsetstrokecolor{currentstroke}%
\pgfsetstrokeopacity{0.000000}%
\pgfsetdash{}{0pt}%
\pgfpathmoveto{\pgfqpoint{1.787003in}{0.500000in}}%
\pgfpathlineto{\pgfqpoint{1.820028in}{0.500000in}}%
\pgfpathlineto{\pgfqpoint{1.820028in}{0.515979in}}%
\pgfpathlineto{\pgfqpoint{1.787003in}{0.515979in}}%
\pgfpathlineto{\pgfqpoint{1.787003in}{0.500000in}}%
\pgfpathclose%
\pgfusepath{fill}%
\end{pgfscope}%
\begin{pgfscope}%
\pgfpathrectangle{\pgfqpoint{0.750000in}{0.500000in}}{\pgfqpoint{4.650000in}{3.020000in}}%
\pgfusepath{clip}%
\pgfsetbuttcap%
\pgfsetmiterjoin%
\definecolor{currentfill}{rgb}{0.000000,0.500000,0.000000}%
\pgfsetfillcolor{currentfill}%
\pgfsetlinewidth{0.000000pt}%
\definecolor{currentstroke}{rgb}{0.000000,0.000000,0.000000}%
\pgfsetstrokecolor{currentstroke}%
\pgfsetstrokeopacity{0.000000}%
\pgfsetdash{}{0pt}%
\pgfpathmoveto{\pgfqpoint{1.820028in}{0.500000in}}%
\pgfpathlineto{\pgfqpoint{1.853054in}{0.500000in}}%
\pgfpathlineto{\pgfqpoint{1.853054in}{0.515979in}}%
\pgfpathlineto{\pgfqpoint{1.820028in}{0.515979in}}%
\pgfpathlineto{\pgfqpoint{1.820028in}{0.500000in}}%
\pgfpathclose%
\pgfusepath{fill}%
\end{pgfscope}%
\begin{pgfscope}%
\pgfpathrectangle{\pgfqpoint{0.750000in}{0.500000in}}{\pgfqpoint{4.650000in}{3.020000in}}%
\pgfusepath{clip}%
\pgfsetbuttcap%
\pgfsetmiterjoin%
\definecolor{currentfill}{rgb}{0.000000,0.500000,0.000000}%
\pgfsetfillcolor{currentfill}%
\pgfsetlinewidth{0.000000pt}%
\definecolor{currentstroke}{rgb}{0.000000,0.000000,0.000000}%
\pgfsetstrokecolor{currentstroke}%
\pgfsetstrokeopacity{0.000000}%
\pgfsetdash{}{0pt}%
\pgfpathmoveto{\pgfqpoint{1.853054in}{0.500000in}}%
\pgfpathlineto{\pgfqpoint{1.886080in}{0.500000in}}%
\pgfpathlineto{\pgfqpoint{1.886080in}{0.500000in}}%
\pgfpathlineto{\pgfqpoint{1.853054in}{0.500000in}}%
\pgfpathlineto{\pgfqpoint{1.853054in}{0.500000in}}%
\pgfpathclose%
\pgfusepath{fill}%
\end{pgfscope}%
\begin{pgfscope}%
\pgfpathrectangle{\pgfqpoint{0.750000in}{0.500000in}}{\pgfqpoint{4.650000in}{3.020000in}}%
\pgfusepath{clip}%
\pgfsetbuttcap%
\pgfsetmiterjoin%
\definecolor{currentfill}{rgb}{0.000000,0.500000,0.000000}%
\pgfsetfillcolor{currentfill}%
\pgfsetlinewidth{0.000000pt}%
\definecolor{currentstroke}{rgb}{0.000000,0.000000,0.000000}%
\pgfsetstrokecolor{currentstroke}%
\pgfsetstrokeopacity{0.000000}%
\pgfsetdash{}{0pt}%
\pgfpathmoveto{\pgfqpoint{1.886080in}{0.500000in}}%
\pgfpathlineto{\pgfqpoint{1.919105in}{0.500000in}}%
\pgfpathlineto{\pgfqpoint{1.919105in}{0.515979in}}%
\pgfpathlineto{\pgfqpoint{1.886080in}{0.515979in}}%
\pgfpathlineto{\pgfqpoint{1.886080in}{0.500000in}}%
\pgfpathclose%
\pgfusepath{fill}%
\end{pgfscope}%
\begin{pgfscope}%
\pgfpathrectangle{\pgfqpoint{0.750000in}{0.500000in}}{\pgfqpoint{4.650000in}{3.020000in}}%
\pgfusepath{clip}%
\pgfsetbuttcap%
\pgfsetmiterjoin%
\definecolor{currentfill}{rgb}{0.000000,0.500000,0.000000}%
\pgfsetfillcolor{currentfill}%
\pgfsetlinewidth{0.000000pt}%
\definecolor{currentstroke}{rgb}{0.000000,0.000000,0.000000}%
\pgfsetstrokecolor{currentstroke}%
\pgfsetstrokeopacity{0.000000}%
\pgfsetdash{}{0pt}%
\pgfpathmoveto{\pgfqpoint{1.919105in}{0.500000in}}%
\pgfpathlineto{\pgfqpoint{1.952131in}{0.500000in}}%
\pgfpathlineto{\pgfqpoint{1.952131in}{0.515979in}}%
\pgfpathlineto{\pgfqpoint{1.919105in}{0.515979in}}%
\pgfpathlineto{\pgfqpoint{1.919105in}{0.500000in}}%
\pgfpathclose%
\pgfusepath{fill}%
\end{pgfscope}%
\begin{pgfscope}%
\pgfpathrectangle{\pgfqpoint{0.750000in}{0.500000in}}{\pgfqpoint{4.650000in}{3.020000in}}%
\pgfusepath{clip}%
\pgfsetbuttcap%
\pgfsetmiterjoin%
\definecolor{currentfill}{rgb}{0.000000,0.500000,0.000000}%
\pgfsetfillcolor{currentfill}%
\pgfsetlinewidth{0.000000pt}%
\definecolor{currentstroke}{rgb}{0.000000,0.000000,0.000000}%
\pgfsetstrokecolor{currentstroke}%
\pgfsetstrokeopacity{0.000000}%
\pgfsetdash{}{0pt}%
\pgfpathmoveto{\pgfqpoint{1.952131in}{0.500000in}}%
\pgfpathlineto{\pgfqpoint{1.985156in}{0.500000in}}%
\pgfpathlineto{\pgfqpoint{1.985156in}{0.531958in}}%
\pgfpathlineto{\pgfqpoint{1.952131in}{0.531958in}}%
\pgfpathlineto{\pgfqpoint{1.952131in}{0.500000in}}%
\pgfpathclose%
\pgfusepath{fill}%
\end{pgfscope}%
\begin{pgfscope}%
\pgfpathrectangle{\pgfqpoint{0.750000in}{0.500000in}}{\pgfqpoint{4.650000in}{3.020000in}}%
\pgfusepath{clip}%
\pgfsetbuttcap%
\pgfsetmiterjoin%
\definecolor{currentfill}{rgb}{0.000000,0.500000,0.000000}%
\pgfsetfillcolor{currentfill}%
\pgfsetlinewidth{0.000000pt}%
\definecolor{currentstroke}{rgb}{0.000000,0.000000,0.000000}%
\pgfsetstrokecolor{currentstroke}%
\pgfsetstrokeopacity{0.000000}%
\pgfsetdash{}{0pt}%
\pgfpathmoveto{\pgfqpoint{1.985156in}{0.500000in}}%
\pgfpathlineto{\pgfqpoint{2.018182in}{0.500000in}}%
\pgfpathlineto{\pgfqpoint{2.018182in}{0.563915in}}%
\pgfpathlineto{\pgfqpoint{1.985156in}{0.563915in}}%
\pgfpathlineto{\pgfqpoint{1.985156in}{0.500000in}}%
\pgfpathclose%
\pgfusepath{fill}%
\end{pgfscope}%
\begin{pgfscope}%
\pgfpathrectangle{\pgfqpoint{0.750000in}{0.500000in}}{\pgfqpoint{4.650000in}{3.020000in}}%
\pgfusepath{clip}%
\pgfsetbuttcap%
\pgfsetmiterjoin%
\definecolor{currentfill}{rgb}{0.000000,0.500000,0.000000}%
\pgfsetfillcolor{currentfill}%
\pgfsetlinewidth{0.000000pt}%
\definecolor{currentstroke}{rgb}{0.000000,0.000000,0.000000}%
\pgfsetstrokecolor{currentstroke}%
\pgfsetstrokeopacity{0.000000}%
\pgfsetdash{}{0pt}%
\pgfpathmoveto{\pgfqpoint{2.018182in}{0.500000in}}%
\pgfpathlineto{\pgfqpoint{2.051207in}{0.500000in}}%
\pgfpathlineto{\pgfqpoint{2.051207in}{0.579894in}}%
\pgfpathlineto{\pgfqpoint{2.018182in}{0.579894in}}%
\pgfpathlineto{\pgfqpoint{2.018182in}{0.500000in}}%
\pgfpathclose%
\pgfusepath{fill}%
\end{pgfscope}%
\begin{pgfscope}%
\pgfpathrectangle{\pgfqpoint{0.750000in}{0.500000in}}{\pgfqpoint{4.650000in}{3.020000in}}%
\pgfusepath{clip}%
\pgfsetbuttcap%
\pgfsetmiterjoin%
\definecolor{currentfill}{rgb}{0.000000,0.500000,0.000000}%
\pgfsetfillcolor{currentfill}%
\pgfsetlinewidth{0.000000pt}%
\definecolor{currentstroke}{rgb}{0.000000,0.000000,0.000000}%
\pgfsetstrokecolor{currentstroke}%
\pgfsetstrokeopacity{0.000000}%
\pgfsetdash{}{0pt}%
\pgfpathmoveto{\pgfqpoint{2.051207in}{0.500000in}}%
\pgfpathlineto{\pgfqpoint{2.084233in}{0.500000in}}%
\pgfpathlineto{\pgfqpoint{2.084233in}{0.531958in}}%
\pgfpathlineto{\pgfqpoint{2.051207in}{0.531958in}}%
\pgfpathlineto{\pgfqpoint{2.051207in}{0.500000in}}%
\pgfpathclose%
\pgfusepath{fill}%
\end{pgfscope}%
\begin{pgfscope}%
\pgfpathrectangle{\pgfqpoint{0.750000in}{0.500000in}}{\pgfqpoint{4.650000in}{3.020000in}}%
\pgfusepath{clip}%
\pgfsetbuttcap%
\pgfsetmiterjoin%
\definecolor{currentfill}{rgb}{0.000000,0.500000,0.000000}%
\pgfsetfillcolor{currentfill}%
\pgfsetlinewidth{0.000000pt}%
\definecolor{currentstroke}{rgb}{0.000000,0.000000,0.000000}%
\pgfsetstrokecolor{currentstroke}%
\pgfsetstrokeopacity{0.000000}%
\pgfsetdash{}{0pt}%
\pgfpathmoveto{\pgfqpoint{2.084233in}{0.500000in}}%
\pgfpathlineto{\pgfqpoint{2.117259in}{0.500000in}}%
\pgfpathlineto{\pgfqpoint{2.117259in}{0.579894in}}%
\pgfpathlineto{\pgfqpoint{2.084233in}{0.579894in}}%
\pgfpathlineto{\pgfqpoint{2.084233in}{0.500000in}}%
\pgfpathclose%
\pgfusepath{fill}%
\end{pgfscope}%
\begin{pgfscope}%
\pgfpathrectangle{\pgfqpoint{0.750000in}{0.500000in}}{\pgfqpoint{4.650000in}{3.020000in}}%
\pgfusepath{clip}%
\pgfsetbuttcap%
\pgfsetmiterjoin%
\definecolor{currentfill}{rgb}{0.000000,0.500000,0.000000}%
\pgfsetfillcolor{currentfill}%
\pgfsetlinewidth{0.000000pt}%
\definecolor{currentstroke}{rgb}{0.000000,0.000000,0.000000}%
\pgfsetstrokecolor{currentstroke}%
\pgfsetstrokeopacity{0.000000}%
\pgfsetdash{}{0pt}%
\pgfpathmoveto{\pgfqpoint{2.117259in}{0.500000in}}%
\pgfpathlineto{\pgfqpoint{2.150284in}{0.500000in}}%
\pgfpathlineto{\pgfqpoint{2.150284in}{0.579894in}}%
\pgfpathlineto{\pgfqpoint{2.117259in}{0.579894in}}%
\pgfpathlineto{\pgfqpoint{2.117259in}{0.500000in}}%
\pgfpathclose%
\pgfusepath{fill}%
\end{pgfscope}%
\begin{pgfscope}%
\pgfpathrectangle{\pgfqpoint{0.750000in}{0.500000in}}{\pgfqpoint{4.650000in}{3.020000in}}%
\pgfusepath{clip}%
\pgfsetbuttcap%
\pgfsetmiterjoin%
\definecolor{currentfill}{rgb}{0.000000,0.500000,0.000000}%
\pgfsetfillcolor{currentfill}%
\pgfsetlinewidth{0.000000pt}%
\definecolor{currentstroke}{rgb}{0.000000,0.000000,0.000000}%
\pgfsetstrokecolor{currentstroke}%
\pgfsetstrokeopacity{0.000000}%
\pgfsetdash{}{0pt}%
\pgfpathmoveto{\pgfqpoint{2.150284in}{0.500000in}}%
\pgfpathlineto{\pgfqpoint{2.183310in}{0.500000in}}%
\pgfpathlineto{\pgfqpoint{2.183310in}{0.643810in}}%
\pgfpathlineto{\pgfqpoint{2.150284in}{0.643810in}}%
\pgfpathlineto{\pgfqpoint{2.150284in}{0.500000in}}%
\pgfpathclose%
\pgfusepath{fill}%
\end{pgfscope}%
\begin{pgfscope}%
\pgfpathrectangle{\pgfqpoint{0.750000in}{0.500000in}}{\pgfqpoint{4.650000in}{3.020000in}}%
\pgfusepath{clip}%
\pgfsetbuttcap%
\pgfsetmiterjoin%
\definecolor{currentfill}{rgb}{0.000000,0.500000,0.000000}%
\pgfsetfillcolor{currentfill}%
\pgfsetlinewidth{0.000000pt}%
\definecolor{currentstroke}{rgb}{0.000000,0.000000,0.000000}%
\pgfsetstrokecolor{currentstroke}%
\pgfsetstrokeopacity{0.000000}%
\pgfsetdash{}{0pt}%
\pgfpathmoveto{\pgfqpoint{2.183310in}{0.500000in}}%
\pgfpathlineto{\pgfqpoint{2.216335in}{0.500000in}}%
\pgfpathlineto{\pgfqpoint{2.216335in}{0.659788in}}%
\pgfpathlineto{\pgfqpoint{2.183310in}{0.659788in}}%
\pgfpathlineto{\pgfqpoint{2.183310in}{0.500000in}}%
\pgfpathclose%
\pgfusepath{fill}%
\end{pgfscope}%
\begin{pgfscope}%
\pgfpathrectangle{\pgfqpoint{0.750000in}{0.500000in}}{\pgfqpoint{4.650000in}{3.020000in}}%
\pgfusepath{clip}%
\pgfsetbuttcap%
\pgfsetmiterjoin%
\definecolor{currentfill}{rgb}{0.000000,0.500000,0.000000}%
\pgfsetfillcolor{currentfill}%
\pgfsetlinewidth{0.000000pt}%
\definecolor{currentstroke}{rgb}{0.000000,0.000000,0.000000}%
\pgfsetstrokecolor{currentstroke}%
\pgfsetstrokeopacity{0.000000}%
\pgfsetdash{}{0pt}%
\pgfpathmoveto{\pgfqpoint{2.216335in}{0.500000in}}%
\pgfpathlineto{\pgfqpoint{2.249361in}{0.500000in}}%
\pgfpathlineto{\pgfqpoint{2.249361in}{0.643810in}}%
\pgfpathlineto{\pgfqpoint{2.216335in}{0.643810in}}%
\pgfpathlineto{\pgfqpoint{2.216335in}{0.500000in}}%
\pgfpathclose%
\pgfusepath{fill}%
\end{pgfscope}%
\begin{pgfscope}%
\pgfpathrectangle{\pgfqpoint{0.750000in}{0.500000in}}{\pgfqpoint{4.650000in}{3.020000in}}%
\pgfusepath{clip}%
\pgfsetbuttcap%
\pgfsetmiterjoin%
\definecolor{currentfill}{rgb}{0.000000,0.500000,0.000000}%
\pgfsetfillcolor{currentfill}%
\pgfsetlinewidth{0.000000pt}%
\definecolor{currentstroke}{rgb}{0.000000,0.000000,0.000000}%
\pgfsetstrokecolor{currentstroke}%
\pgfsetstrokeopacity{0.000000}%
\pgfsetdash{}{0pt}%
\pgfpathmoveto{\pgfqpoint{2.249361in}{0.500000in}}%
\pgfpathlineto{\pgfqpoint{2.282386in}{0.500000in}}%
\pgfpathlineto{\pgfqpoint{2.282386in}{0.707725in}}%
\pgfpathlineto{\pgfqpoint{2.249361in}{0.707725in}}%
\pgfpathlineto{\pgfqpoint{2.249361in}{0.500000in}}%
\pgfpathclose%
\pgfusepath{fill}%
\end{pgfscope}%
\begin{pgfscope}%
\pgfpathrectangle{\pgfqpoint{0.750000in}{0.500000in}}{\pgfqpoint{4.650000in}{3.020000in}}%
\pgfusepath{clip}%
\pgfsetbuttcap%
\pgfsetmiterjoin%
\definecolor{currentfill}{rgb}{0.000000,0.500000,0.000000}%
\pgfsetfillcolor{currentfill}%
\pgfsetlinewidth{0.000000pt}%
\definecolor{currentstroke}{rgb}{0.000000,0.000000,0.000000}%
\pgfsetstrokecolor{currentstroke}%
\pgfsetstrokeopacity{0.000000}%
\pgfsetdash{}{0pt}%
\pgfpathmoveto{\pgfqpoint{2.282386in}{0.500000in}}%
\pgfpathlineto{\pgfqpoint{2.315412in}{0.500000in}}%
\pgfpathlineto{\pgfqpoint{2.315412in}{0.563915in}}%
\pgfpathlineto{\pgfqpoint{2.282386in}{0.563915in}}%
\pgfpathlineto{\pgfqpoint{2.282386in}{0.500000in}}%
\pgfpathclose%
\pgfusepath{fill}%
\end{pgfscope}%
\begin{pgfscope}%
\pgfpathrectangle{\pgfqpoint{0.750000in}{0.500000in}}{\pgfqpoint{4.650000in}{3.020000in}}%
\pgfusepath{clip}%
\pgfsetbuttcap%
\pgfsetmiterjoin%
\definecolor{currentfill}{rgb}{0.000000,0.500000,0.000000}%
\pgfsetfillcolor{currentfill}%
\pgfsetlinewidth{0.000000pt}%
\definecolor{currentstroke}{rgb}{0.000000,0.000000,0.000000}%
\pgfsetstrokecolor{currentstroke}%
\pgfsetstrokeopacity{0.000000}%
\pgfsetdash{}{0pt}%
\pgfpathmoveto{\pgfqpoint{2.315412in}{0.500000in}}%
\pgfpathlineto{\pgfqpoint{2.348437in}{0.500000in}}%
\pgfpathlineto{\pgfqpoint{2.348437in}{0.691746in}}%
\pgfpathlineto{\pgfqpoint{2.315412in}{0.691746in}}%
\pgfpathlineto{\pgfqpoint{2.315412in}{0.500000in}}%
\pgfpathclose%
\pgfusepath{fill}%
\end{pgfscope}%
\begin{pgfscope}%
\pgfpathrectangle{\pgfqpoint{0.750000in}{0.500000in}}{\pgfqpoint{4.650000in}{3.020000in}}%
\pgfusepath{clip}%
\pgfsetbuttcap%
\pgfsetmiterjoin%
\definecolor{currentfill}{rgb}{0.000000,0.500000,0.000000}%
\pgfsetfillcolor{currentfill}%
\pgfsetlinewidth{0.000000pt}%
\definecolor{currentstroke}{rgb}{0.000000,0.000000,0.000000}%
\pgfsetstrokecolor{currentstroke}%
\pgfsetstrokeopacity{0.000000}%
\pgfsetdash{}{0pt}%
\pgfpathmoveto{\pgfqpoint{2.348438in}{0.500000in}}%
\pgfpathlineto{\pgfqpoint{2.381463in}{0.500000in}}%
\pgfpathlineto{\pgfqpoint{2.381463in}{0.787619in}}%
\pgfpathlineto{\pgfqpoint{2.348438in}{0.787619in}}%
\pgfpathlineto{\pgfqpoint{2.348438in}{0.500000in}}%
\pgfpathclose%
\pgfusepath{fill}%
\end{pgfscope}%
\begin{pgfscope}%
\pgfpathrectangle{\pgfqpoint{0.750000in}{0.500000in}}{\pgfqpoint{4.650000in}{3.020000in}}%
\pgfusepath{clip}%
\pgfsetbuttcap%
\pgfsetmiterjoin%
\definecolor{currentfill}{rgb}{0.000000,0.500000,0.000000}%
\pgfsetfillcolor{currentfill}%
\pgfsetlinewidth{0.000000pt}%
\definecolor{currentstroke}{rgb}{0.000000,0.000000,0.000000}%
\pgfsetstrokecolor{currentstroke}%
\pgfsetstrokeopacity{0.000000}%
\pgfsetdash{}{0pt}%
\pgfpathmoveto{\pgfqpoint{2.381463in}{0.500000in}}%
\pgfpathlineto{\pgfqpoint{2.414489in}{0.500000in}}%
\pgfpathlineto{\pgfqpoint{2.414489in}{0.691746in}}%
\pgfpathlineto{\pgfqpoint{2.381463in}{0.691746in}}%
\pgfpathlineto{\pgfqpoint{2.381463in}{0.500000in}}%
\pgfpathclose%
\pgfusepath{fill}%
\end{pgfscope}%
\begin{pgfscope}%
\pgfpathrectangle{\pgfqpoint{0.750000in}{0.500000in}}{\pgfqpoint{4.650000in}{3.020000in}}%
\pgfusepath{clip}%
\pgfsetbuttcap%
\pgfsetmiterjoin%
\definecolor{currentfill}{rgb}{0.000000,0.500000,0.000000}%
\pgfsetfillcolor{currentfill}%
\pgfsetlinewidth{0.000000pt}%
\definecolor{currentstroke}{rgb}{0.000000,0.000000,0.000000}%
\pgfsetstrokecolor{currentstroke}%
\pgfsetstrokeopacity{0.000000}%
\pgfsetdash{}{0pt}%
\pgfpathmoveto{\pgfqpoint{2.414489in}{0.500000in}}%
\pgfpathlineto{\pgfqpoint{2.447514in}{0.500000in}}%
\pgfpathlineto{\pgfqpoint{2.447514in}{0.787619in}}%
\pgfpathlineto{\pgfqpoint{2.414489in}{0.787619in}}%
\pgfpathlineto{\pgfqpoint{2.414489in}{0.500000in}}%
\pgfpathclose%
\pgfusepath{fill}%
\end{pgfscope}%
\begin{pgfscope}%
\pgfpathrectangle{\pgfqpoint{0.750000in}{0.500000in}}{\pgfqpoint{4.650000in}{3.020000in}}%
\pgfusepath{clip}%
\pgfsetbuttcap%
\pgfsetmiterjoin%
\definecolor{currentfill}{rgb}{0.000000,0.500000,0.000000}%
\pgfsetfillcolor{currentfill}%
\pgfsetlinewidth{0.000000pt}%
\definecolor{currentstroke}{rgb}{0.000000,0.000000,0.000000}%
\pgfsetstrokecolor{currentstroke}%
\pgfsetstrokeopacity{0.000000}%
\pgfsetdash{}{0pt}%
\pgfpathmoveto{\pgfqpoint{2.447514in}{0.500000in}}%
\pgfpathlineto{\pgfqpoint{2.480540in}{0.500000in}}%
\pgfpathlineto{\pgfqpoint{2.480540in}{0.771640in}}%
\pgfpathlineto{\pgfqpoint{2.447514in}{0.771640in}}%
\pgfpathlineto{\pgfqpoint{2.447514in}{0.500000in}}%
\pgfpathclose%
\pgfusepath{fill}%
\end{pgfscope}%
\begin{pgfscope}%
\pgfpathrectangle{\pgfqpoint{0.750000in}{0.500000in}}{\pgfqpoint{4.650000in}{3.020000in}}%
\pgfusepath{clip}%
\pgfsetbuttcap%
\pgfsetmiterjoin%
\definecolor{currentfill}{rgb}{0.000000,0.500000,0.000000}%
\pgfsetfillcolor{currentfill}%
\pgfsetlinewidth{0.000000pt}%
\definecolor{currentstroke}{rgb}{0.000000,0.000000,0.000000}%
\pgfsetstrokecolor{currentstroke}%
\pgfsetstrokeopacity{0.000000}%
\pgfsetdash{}{0pt}%
\pgfpathmoveto{\pgfqpoint{2.480540in}{0.500000in}}%
\pgfpathlineto{\pgfqpoint{2.513565in}{0.500000in}}%
\pgfpathlineto{\pgfqpoint{2.513565in}{0.899471in}}%
\pgfpathlineto{\pgfqpoint{2.480540in}{0.899471in}}%
\pgfpathlineto{\pgfqpoint{2.480540in}{0.500000in}}%
\pgfpathclose%
\pgfusepath{fill}%
\end{pgfscope}%
\begin{pgfscope}%
\pgfpathrectangle{\pgfqpoint{0.750000in}{0.500000in}}{\pgfqpoint{4.650000in}{3.020000in}}%
\pgfusepath{clip}%
\pgfsetbuttcap%
\pgfsetmiterjoin%
\definecolor{currentfill}{rgb}{0.000000,0.500000,0.000000}%
\pgfsetfillcolor{currentfill}%
\pgfsetlinewidth{0.000000pt}%
\definecolor{currentstroke}{rgb}{0.000000,0.000000,0.000000}%
\pgfsetstrokecolor{currentstroke}%
\pgfsetstrokeopacity{0.000000}%
\pgfsetdash{}{0pt}%
\pgfpathmoveto{\pgfqpoint{2.513565in}{0.500000in}}%
\pgfpathlineto{\pgfqpoint{2.546591in}{0.500000in}}%
\pgfpathlineto{\pgfqpoint{2.546591in}{0.915450in}}%
\pgfpathlineto{\pgfqpoint{2.513565in}{0.915450in}}%
\pgfpathlineto{\pgfqpoint{2.513565in}{0.500000in}}%
\pgfpathclose%
\pgfusepath{fill}%
\end{pgfscope}%
\begin{pgfscope}%
\pgfpathrectangle{\pgfqpoint{0.750000in}{0.500000in}}{\pgfqpoint{4.650000in}{3.020000in}}%
\pgfusepath{clip}%
\pgfsetbuttcap%
\pgfsetmiterjoin%
\definecolor{currentfill}{rgb}{0.000000,0.500000,0.000000}%
\pgfsetfillcolor{currentfill}%
\pgfsetlinewidth{0.000000pt}%
\definecolor{currentstroke}{rgb}{0.000000,0.000000,0.000000}%
\pgfsetstrokecolor{currentstroke}%
\pgfsetstrokeopacity{0.000000}%
\pgfsetdash{}{0pt}%
\pgfpathmoveto{\pgfqpoint{2.546591in}{0.500000in}}%
\pgfpathlineto{\pgfqpoint{2.579616in}{0.500000in}}%
\pgfpathlineto{\pgfqpoint{2.579616in}{0.899471in}}%
\pgfpathlineto{\pgfqpoint{2.546591in}{0.899471in}}%
\pgfpathlineto{\pgfqpoint{2.546591in}{0.500000in}}%
\pgfpathclose%
\pgfusepath{fill}%
\end{pgfscope}%
\begin{pgfscope}%
\pgfpathrectangle{\pgfqpoint{0.750000in}{0.500000in}}{\pgfqpoint{4.650000in}{3.020000in}}%
\pgfusepath{clip}%
\pgfsetbuttcap%
\pgfsetmiterjoin%
\definecolor{currentfill}{rgb}{0.000000,0.500000,0.000000}%
\pgfsetfillcolor{currentfill}%
\pgfsetlinewidth{0.000000pt}%
\definecolor{currentstroke}{rgb}{0.000000,0.000000,0.000000}%
\pgfsetstrokecolor{currentstroke}%
\pgfsetstrokeopacity{0.000000}%
\pgfsetdash{}{0pt}%
\pgfpathmoveto{\pgfqpoint{2.579616in}{0.500000in}}%
\pgfpathlineto{\pgfqpoint{2.612642in}{0.500000in}}%
\pgfpathlineto{\pgfqpoint{2.612642in}{1.123175in}}%
\pgfpathlineto{\pgfqpoint{2.579616in}{1.123175in}}%
\pgfpathlineto{\pgfqpoint{2.579616in}{0.500000in}}%
\pgfpathclose%
\pgfusepath{fill}%
\end{pgfscope}%
\begin{pgfscope}%
\pgfpathrectangle{\pgfqpoint{0.750000in}{0.500000in}}{\pgfqpoint{4.650000in}{3.020000in}}%
\pgfusepath{clip}%
\pgfsetbuttcap%
\pgfsetmiterjoin%
\definecolor{currentfill}{rgb}{0.000000,0.500000,0.000000}%
\pgfsetfillcolor{currentfill}%
\pgfsetlinewidth{0.000000pt}%
\definecolor{currentstroke}{rgb}{0.000000,0.000000,0.000000}%
\pgfsetstrokecolor{currentstroke}%
\pgfsetstrokeopacity{0.000000}%
\pgfsetdash{}{0pt}%
\pgfpathmoveto{\pgfqpoint{2.612642in}{0.500000in}}%
\pgfpathlineto{\pgfqpoint{2.645668in}{0.500000in}}%
\pgfpathlineto{\pgfqpoint{2.645668in}{1.155132in}}%
\pgfpathlineto{\pgfqpoint{2.612642in}{1.155132in}}%
\pgfpathlineto{\pgfqpoint{2.612642in}{0.500000in}}%
\pgfpathclose%
\pgfusepath{fill}%
\end{pgfscope}%
\begin{pgfscope}%
\pgfpathrectangle{\pgfqpoint{0.750000in}{0.500000in}}{\pgfqpoint{4.650000in}{3.020000in}}%
\pgfusepath{clip}%
\pgfsetbuttcap%
\pgfsetmiterjoin%
\definecolor{currentfill}{rgb}{0.000000,0.500000,0.000000}%
\pgfsetfillcolor{currentfill}%
\pgfsetlinewidth{0.000000pt}%
\definecolor{currentstroke}{rgb}{0.000000,0.000000,0.000000}%
\pgfsetstrokecolor{currentstroke}%
\pgfsetstrokeopacity{0.000000}%
\pgfsetdash{}{0pt}%
\pgfpathmoveto{\pgfqpoint{2.645668in}{0.500000in}}%
\pgfpathlineto{\pgfqpoint{2.678693in}{0.500000in}}%
\pgfpathlineto{\pgfqpoint{2.678693in}{1.458730in}}%
\pgfpathlineto{\pgfqpoint{2.645668in}{1.458730in}}%
\pgfpathlineto{\pgfqpoint{2.645668in}{0.500000in}}%
\pgfpathclose%
\pgfusepath{fill}%
\end{pgfscope}%
\begin{pgfscope}%
\pgfpathrectangle{\pgfqpoint{0.750000in}{0.500000in}}{\pgfqpoint{4.650000in}{3.020000in}}%
\pgfusepath{clip}%
\pgfsetbuttcap%
\pgfsetmiterjoin%
\definecolor{currentfill}{rgb}{0.000000,0.500000,0.000000}%
\pgfsetfillcolor{currentfill}%
\pgfsetlinewidth{0.000000pt}%
\definecolor{currentstroke}{rgb}{0.000000,0.000000,0.000000}%
\pgfsetstrokecolor{currentstroke}%
\pgfsetstrokeopacity{0.000000}%
\pgfsetdash{}{0pt}%
\pgfpathmoveto{\pgfqpoint{2.678693in}{0.500000in}}%
\pgfpathlineto{\pgfqpoint{2.711719in}{0.500000in}}%
\pgfpathlineto{\pgfqpoint{2.711719in}{1.330899in}}%
\pgfpathlineto{\pgfqpoint{2.678693in}{1.330899in}}%
\pgfpathlineto{\pgfqpoint{2.678693in}{0.500000in}}%
\pgfpathclose%
\pgfusepath{fill}%
\end{pgfscope}%
\begin{pgfscope}%
\pgfpathrectangle{\pgfqpoint{0.750000in}{0.500000in}}{\pgfqpoint{4.650000in}{3.020000in}}%
\pgfusepath{clip}%
\pgfsetbuttcap%
\pgfsetmiterjoin%
\definecolor{currentfill}{rgb}{0.000000,0.500000,0.000000}%
\pgfsetfillcolor{currentfill}%
\pgfsetlinewidth{0.000000pt}%
\definecolor{currentstroke}{rgb}{0.000000,0.000000,0.000000}%
\pgfsetstrokecolor{currentstroke}%
\pgfsetstrokeopacity{0.000000}%
\pgfsetdash{}{0pt}%
\pgfpathmoveto{\pgfqpoint{2.711719in}{0.500000in}}%
\pgfpathlineto{\pgfqpoint{2.744744in}{0.500000in}}%
\pgfpathlineto{\pgfqpoint{2.744744in}{1.522646in}}%
\pgfpathlineto{\pgfqpoint{2.711719in}{1.522646in}}%
\pgfpathlineto{\pgfqpoint{2.711719in}{0.500000in}}%
\pgfpathclose%
\pgfusepath{fill}%
\end{pgfscope}%
\begin{pgfscope}%
\pgfpathrectangle{\pgfqpoint{0.750000in}{0.500000in}}{\pgfqpoint{4.650000in}{3.020000in}}%
\pgfusepath{clip}%
\pgfsetbuttcap%
\pgfsetmiterjoin%
\definecolor{currentfill}{rgb}{0.000000,0.500000,0.000000}%
\pgfsetfillcolor{currentfill}%
\pgfsetlinewidth{0.000000pt}%
\definecolor{currentstroke}{rgb}{0.000000,0.000000,0.000000}%
\pgfsetstrokecolor{currentstroke}%
\pgfsetstrokeopacity{0.000000}%
\pgfsetdash{}{0pt}%
\pgfpathmoveto{\pgfqpoint{2.744744in}{0.500000in}}%
\pgfpathlineto{\pgfqpoint{2.777770in}{0.500000in}}%
\pgfpathlineto{\pgfqpoint{2.777770in}{1.490688in}}%
\pgfpathlineto{\pgfqpoint{2.744744in}{1.490688in}}%
\pgfpathlineto{\pgfqpoint{2.744744in}{0.500000in}}%
\pgfpathclose%
\pgfusepath{fill}%
\end{pgfscope}%
\begin{pgfscope}%
\pgfpathrectangle{\pgfqpoint{0.750000in}{0.500000in}}{\pgfqpoint{4.650000in}{3.020000in}}%
\pgfusepath{clip}%
\pgfsetbuttcap%
\pgfsetmiterjoin%
\definecolor{currentfill}{rgb}{0.000000,0.500000,0.000000}%
\pgfsetfillcolor{currentfill}%
\pgfsetlinewidth{0.000000pt}%
\definecolor{currentstroke}{rgb}{0.000000,0.000000,0.000000}%
\pgfsetstrokecolor{currentstroke}%
\pgfsetstrokeopacity{0.000000}%
\pgfsetdash{}{0pt}%
\pgfpathmoveto{\pgfqpoint{2.777770in}{0.500000in}}%
\pgfpathlineto{\pgfqpoint{2.810795in}{0.500000in}}%
\pgfpathlineto{\pgfqpoint{2.810795in}{1.522646in}}%
\pgfpathlineto{\pgfqpoint{2.777770in}{1.522646in}}%
\pgfpathlineto{\pgfqpoint{2.777770in}{0.500000in}}%
\pgfpathclose%
\pgfusepath{fill}%
\end{pgfscope}%
\begin{pgfscope}%
\pgfpathrectangle{\pgfqpoint{0.750000in}{0.500000in}}{\pgfqpoint{4.650000in}{3.020000in}}%
\pgfusepath{clip}%
\pgfsetbuttcap%
\pgfsetmiterjoin%
\definecolor{currentfill}{rgb}{0.000000,0.500000,0.000000}%
\pgfsetfillcolor{currentfill}%
\pgfsetlinewidth{0.000000pt}%
\definecolor{currentstroke}{rgb}{0.000000,0.000000,0.000000}%
\pgfsetstrokecolor{currentstroke}%
\pgfsetstrokeopacity{0.000000}%
\pgfsetdash{}{0pt}%
\pgfpathmoveto{\pgfqpoint{2.810795in}{0.500000in}}%
\pgfpathlineto{\pgfqpoint{2.843821in}{0.500000in}}%
\pgfpathlineto{\pgfqpoint{2.843821in}{1.570582in}}%
\pgfpathlineto{\pgfqpoint{2.810795in}{1.570582in}}%
\pgfpathlineto{\pgfqpoint{2.810795in}{0.500000in}}%
\pgfpathclose%
\pgfusepath{fill}%
\end{pgfscope}%
\begin{pgfscope}%
\pgfpathrectangle{\pgfqpoint{0.750000in}{0.500000in}}{\pgfqpoint{4.650000in}{3.020000in}}%
\pgfusepath{clip}%
\pgfsetbuttcap%
\pgfsetmiterjoin%
\definecolor{currentfill}{rgb}{0.000000,0.500000,0.000000}%
\pgfsetfillcolor{currentfill}%
\pgfsetlinewidth{0.000000pt}%
\definecolor{currentstroke}{rgb}{0.000000,0.000000,0.000000}%
\pgfsetstrokecolor{currentstroke}%
\pgfsetstrokeopacity{0.000000}%
\pgfsetdash{}{0pt}%
\pgfpathmoveto{\pgfqpoint{2.843821in}{0.500000in}}%
\pgfpathlineto{\pgfqpoint{2.876847in}{0.500000in}}%
\pgfpathlineto{\pgfqpoint{2.876847in}{1.810265in}}%
\pgfpathlineto{\pgfqpoint{2.843821in}{1.810265in}}%
\pgfpathlineto{\pgfqpoint{2.843821in}{0.500000in}}%
\pgfpathclose%
\pgfusepath{fill}%
\end{pgfscope}%
\begin{pgfscope}%
\pgfpathrectangle{\pgfqpoint{0.750000in}{0.500000in}}{\pgfqpoint{4.650000in}{3.020000in}}%
\pgfusepath{clip}%
\pgfsetbuttcap%
\pgfsetmiterjoin%
\definecolor{currentfill}{rgb}{0.000000,0.500000,0.000000}%
\pgfsetfillcolor{currentfill}%
\pgfsetlinewidth{0.000000pt}%
\definecolor{currentstroke}{rgb}{0.000000,0.000000,0.000000}%
\pgfsetstrokecolor{currentstroke}%
\pgfsetstrokeopacity{0.000000}%
\pgfsetdash{}{0pt}%
\pgfpathmoveto{\pgfqpoint{2.876847in}{0.500000in}}%
\pgfpathlineto{\pgfqpoint{2.909872in}{0.500000in}}%
\pgfpathlineto{\pgfqpoint{2.909872in}{1.794286in}}%
\pgfpathlineto{\pgfqpoint{2.876847in}{1.794286in}}%
\pgfpathlineto{\pgfqpoint{2.876847in}{0.500000in}}%
\pgfpathclose%
\pgfusepath{fill}%
\end{pgfscope}%
\begin{pgfscope}%
\pgfpathrectangle{\pgfqpoint{0.750000in}{0.500000in}}{\pgfqpoint{4.650000in}{3.020000in}}%
\pgfusepath{clip}%
\pgfsetbuttcap%
\pgfsetmiterjoin%
\definecolor{currentfill}{rgb}{0.000000,0.500000,0.000000}%
\pgfsetfillcolor{currentfill}%
\pgfsetlinewidth{0.000000pt}%
\definecolor{currentstroke}{rgb}{0.000000,0.000000,0.000000}%
\pgfsetstrokecolor{currentstroke}%
\pgfsetstrokeopacity{0.000000}%
\pgfsetdash{}{0pt}%
\pgfpathmoveto{\pgfqpoint{2.909872in}{0.500000in}}%
\pgfpathlineto{\pgfqpoint{2.942898in}{0.500000in}}%
\pgfpathlineto{\pgfqpoint{2.942898in}{2.225714in}}%
\pgfpathlineto{\pgfqpoint{2.909872in}{2.225714in}}%
\pgfpathlineto{\pgfqpoint{2.909872in}{0.500000in}}%
\pgfpathclose%
\pgfusepath{fill}%
\end{pgfscope}%
\begin{pgfscope}%
\pgfpathrectangle{\pgfqpoint{0.750000in}{0.500000in}}{\pgfqpoint{4.650000in}{3.020000in}}%
\pgfusepath{clip}%
\pgfsetbuttcap%
\pgfsetmiterjoin%
\definecolor{currentfill}{rgb}{0.000000,0.500000,0.000000}%
\pgfsetfillcolor{currentfill}%
\pgfsetlinewidth{0.000000pt}%
\definecolor{currentstroke}{rgb}{0.000000,0.000000,0.000000}%
\pgfsetstrokecolor{currentstroke}%
\pgfsetstrokeopacity{0.000000}%
\pgfsetdash{}{0pt}%
\pgfpathmoveto{\pgfqpoint{2.942898in}{0.500000in}}%
\pgfpathlineto{\pgfqpoint{2.975923in}{0.500000in}}%
\pgfpathlineto{\pgfqpoint{2.975923in}{2.097884in}}%
\pgfpathlineto{\pgfqpoint{2.942898in}{2.097884in}}%
\pgfpathlineto{\pgfqpoint{2.942898in}{0.500000in}}%
\pgfpathclose%
\pgfusepath{fill}%
\end{pgfscope}%
\begin{pgfscope}%
\pgfpathrectangle{\pgfqpoint{0.750000in}{0.500000in}}{\pgfqpoint{4.650000in}{3.020000in}}%
\pgfusepath{clip}%
\pgfsetbuttcap%
\pgfsetmiterjoin%
\definecolor{currentfill}{rgb}{0.000000,0.500000,0.000000}%
\pgfsetfillcolor{currentfill}%
\pgfsetlinewidth{0.000000pt}%
\definecolor{currentstroke}{rgb}{0.000000,0.000000,0.000000}%
\pgfsetstrokecolor{currentstroke}%
\pgfsetstrokeopacity{0.000000}%
\pgfsetdash{}{0pt}%
\pgfpathmoveto{\pgfqpoint{2.975923in}{0.500000in}}%
\pgfpathlineto{\pgfqpoint{3.008949in}{0.500000in}}%
\pgfpathlineto{\pgfqpoint{3.008949in}{1.698413in}}%
\pgfpathlineto{\pgfqpoint{2.975923in}{1.698413in}}%
\pgfpathlineto{\pgfqpoint{2.975923in}{0.500000in}}%
\pgfpathclose%
\pgfusepath{fill}%
\end{pgfscope}%
\begin{pgfscope}%
\pgfpathrectangle{\pgfqpoint{0.750000in}{0.500000in}}{\pgfqpoint{4.650000in}{3.020000in}}%
\pgfusepath{clip}%
\pgfsetbuttcap%
\pgfsetmiterjoin%
\definecolor{currentfill}{rgb}{0.000000,0.500000,0.000000}%
\pgfsetfillcolor{currentfill}%
\pgfsetlinewidth{0.000000pt}%
\definecolor{currentstroke}{rgb}{0.000000,0.000000,0.000000}%
\pgfsetstrokecolor{currentstroke}%
\pgfsetstrokeopacity{0.000000}%
\pgfsetdash{}{0pt}%
\pgfpathmoveto{\pgfqpoint{3.008949in}{0.500000in}}%
\pgfpathlineto{\pgfqpoint{3.041974in}{0.500000in}}%
\pgfpathlineto{\pgfqpoint{3.041974in}{1.730370in}}%
\pgfpathlineto{\pgfqpoint{3.008949in}{1.730370in}}%
\pgfpathlineto{\pgfqpoint{3.008949in}{0.500000in}}%
\pgfpathclose%
\pgfusepath{fill}%
\end{pgfscope}%
\begin{pgfscope}%
\pgfpathrectangle{\pgfqpoint{0.750000in}{0.500000in}}{\pgfqpoint{4.650000in}{3.020000in}}%
\pgfusepath{clip}%
\pgfsetbuttcap%
\pgfsetmiterjoin%
\definecolor{currentfill}{rgb}{0.000000,0.500000,0.000000}%
\pgfsetfillcolor{currentfill}%
\pgfsetlinewidth{0.000000pt}%
\definecolor{currentstroke}{rgb}{0.000000,0.000000,0.000000}%
\pgfsetstrokecolor{currentstroke}%
\pgfsetstrokeopacity{0.000000}%
\pgfsetdash{}{0pt}%
\pgfpathmoveto{\pgfqpoint{3.041974in}{0.500000in}}%
\pgfpathlineto{\pgfqpoint{3.075000in}{0.500000in}}%
\pgfpathlineto{\pgfqpoint{3.075000in}{2.529312in}}%
\pgfpathlineto{\pgfqpoint{3.041974in}{2.529312in}}%
\pgfpathlineto{\pgfqpoint{3.041974in}{0.500000in}}%
\pgfpathclose%
\pgfusepath{fill}%
\end{pgfscope}%
\begin{pgfscope}%
\pgfpathrectangle{\pgfqpoint{0.750000in}{0.500000in}}{\pgfqpoint{4.650000in}{3.020000in}}%
\pgfusepath{clip}%
\pgfsetbuttcap%
\pgfsetmiterjoin%
\definecolor{currentfill}{rgb}{0.000000,0.500000,0.000000}%
\pgfsetfillcolor{currentfill}%
\pgfsetlinewidth{0.000000pt}%
\definecolor{currentstroke}{rgb}{0.000000,0.000000,0.000000}%
\pgfsetstrokecolor{currentstroke}%
\pgfsetstrokeopacity{0.000000}%
\pgfsetdash{}{0pt}%
\pgfpathmoveto{\pgfqpoint{3.075000in}{0.500000in}}%
\pgfpathlineto{\pgfqpoint{3.108026in}{0.500000in}}%
\pgfpathlineto{\pgfqpoint{3.108026in}{2.321587in}}%
\pgfpathlineto{\pgfqpoint{3.075000in}{2.321587in}}%
\pgfpathlineto{\pgfqpoint{3.075000in}{0.500000in}}%
\pgfpathclose%
\pgfusepath{fill}%
\end{pgfscope}%
\begin{pgfscope}%
\pgfpathrectangle{\pgfqpoint{0.750000in}{0.500000in}}{\pgfqpoint{4.650000in}{3.020000in}}%
\pgfusepath{clip}%
\pgfsetbuttcap%
\pgfsetmiterjoin%
\definecolor{currentfill}{rgb}{0.000000,0.500000,0.000000}%
\pgfsetfillcolor{currentfill}%
\pgfsetlinewidth{0.000000pt}%
\definecolor{currentstroke}{rgb}{0.000000,0.000000,0.000000}%
\pgfsetstrokecolor{currentstroke}%
\pgfsetstrokeopacity{0.000000}%
\pgfsetdash{}{0pt}%
\pgfpathmoveto{\pgfqpoint{3.108026in}{0.500000in}}%
\pgfpathlineto{\pgfqpoint{3.141051in}{0.500000in}}%
\pgfpathlineto{\pgfqpoint{3.141051in}{2.097884in}}%
\pgfpathlineto{\pgfqpoint{3.108026in}{2.097884in}}%
\pgfpathlineto{\pgfqpoint{3.108026in}{0.500000in}}%
\pgfpathclose%
\pgfusepath{fill}%
\end{pgfscope}%
\begin{pgfscope}%
\pgfpathrectangle{\pgfqpoint{0.750000in}{0.500000in}}{\pgfqpoint{4.650000in}{3.020000in}}%
\pgfusepath{clip}%
\pgfsetbuttcap%
\pgfsetmiterjoin%
\definecolor{currentfill}{rgb}{0.000000,0.500000,0.000000}%
\pgfsetfillcolor{currentfill}%
\pgfsetlinewidth{0.000000pt}%
\definecolor{currentstroke}{rgb}{0.000000,0.000000,0.000000}%
\pgfsetstrokecolor{currentstroke}%
\pgfsetstrokeopacity{0.000000}%
\pgfsetdash{}{0pt}%
\pgfpathmoveto{\pgfqpoint{3.141051in}{0.500000in}}%
\pgfpathlineto{\pgfqpoint{3.174077in}{0.500000in}}%
\pgfpathlineto{\pgfqpoint{3.174077in}{2.497354in}}%
\pgfpathlineto{\pgfqpoint{3.141051in}{2.497354in}}%
\pgfpathlineto{\pgfqpoint{3.141051in}{0.500000in}}%
\pgfpathclose%
\pgfusepath{fill}%
\end{pgfscope}%
\begin{pgfscope}%
\pgfpathrectangle{\pgfqpoint{0.750000in}{0.500000in}}{\pgfqpoint{4.650000in}{3.020000in}}%
\pgfusepath{clip}%
\pgfsetbuttcap%
\pgfsetmiterjoin%
\definecolor{currentfill}{rgb}{0.000000,0.500000,0.000000}%
\pgfsetfillcolor{currentfill}%
\pgfsetlinewidth{0.000000pt}%
\definecolor{currentstroke}{rgb}{0.000000,0.000000,0.000000}%
\pgfsetstrokecolor{currentstroke}%
\pgfsetstrokeopacity{0.000000}%
\pgfsetdash{}{0pt}%
\pgfpathmoveto{\pgfqpoint{3.174077in}{0.500000in}}%
\pgfpathlineto{\pgfqpoint{3.207102in}{0.500000in}}%
\pgfpathlineto{\pgfqpoint{3.207102in}{2.609206in}}%
\pgfpathlineto{\pgfqpoint{3.174077in}{2.609206in}}%
\pgfpathlineto{\pgfqpoint{3.174077in}{0.500000in}}%
\pgfpathclose%
\pgfusepath{fill}%
\end{pgfscope}%
\begin{pgfscope}%
\pgfpathrectangle{\pgfqpoint{0.750000in}{0.500000in}}{\pgfqpoint{4.650000in}{3.020000in}}%
\pgfusepath{clip}%
\pgfsetbuttcap%
\pgfsetmiterjoin%
\definecolor{currentfill}{rgb}{0.000000,0.500000,0.000000}%
\pgfsetfillcolor{currentfill}%
\pgfsetlinewidth{0.000000pt}%
\definecolor{currentstroke}{rgb}{0.000000,0.000000,0.000000}%
\pgfsetstrokecolor{currentstroke}%
\pgfsetstrokeopacity{0.000000}%
\pgfsetdash{}{0pt}%
\pgfpathmoveto{\pgfqpoint{3.207102in}{0.500000in}}%
\pgfpathlineto{\pgfqpoint{3.240128in}{0.500000in}}%
\pgfpathlineto{\pgfqpoint{3.240128in}{2.481376in}}%
\pgfpathlineto{\pgfqpoint{3.207102in}{2.481376in}}%
\pgfpathlineto{\pgfqpoint{3.207102in}{0.500000in}}%
\pgfpathclose%
\pgfusepath{fill}%
\end{pgfscope}%
\begin{pgfscope}%
\pgfpathrectangle{\pgfqpoint{0.750000in}{0.500000in}}{\pgfqpoint{4.650000in}{3.020000in}}%
\pgfusepath{clip}%
\pgfsetbuttcap%
\pgfsetmiterjoin%
\definecolor{currentfill}{rgb}{0.000000,0.500000,0.000000}%
\pgfsetfillcolor{currentfill}%
\pgfsetlinewidth{0.000000pt}%
\definecolor{currentstroke}{rgb}{0.000000,0.000000,0.000000}%
\pgfsetstrokecolor{currentstroke}%
\pgfsetstrokeopacity{0.000000}%
\pgfsetdash{}{0pt}%
\pgfpathmoveto{\pgfqpoint{3.240128in}{0.500000in}}%
\pgfpathlineto{\pgfqpoint{3.273153in}{0.500000in}}%
\pgfpathlineto{\pgfqpoint{3.273153in}{2.577249in}}%
\pgfpathlineto{\pgfqpoint{3.240128in}{2.577249in}}%
\pgfpathlineto{\pgfqpoint{3.240128in}{0.500000in}}%
\pgfpathclose%
\pgfusepath{fill}%
\end{pgfscope}%
\begin{pgfscope}%
\pgfpathrectangle{\pgfqpoint{0.750000in}{0.500000in}}{\pgfqpoint{4.650000in}{3.020000in}}%
\pgfusepath{clip}%
\pgfsetbuttcap%
\pgfsetmiterjoin%
\definecolor{currentfill}{rgb}{0.000000,0.500000,0.000000}%
\pgfsetfillcolor{currentfill}%
\pgfsetlinewidth{0.000000pt}%
\definecolor{currentstroke}{rgb}{0.000000,0.000000,0.000000}%
\pgfsetstrokecolor{currentstroke}%
\pgfsetstrokeopacity{0.000000}%
\pgfsetdash{}{0pt}%
\pgfpathmoveto{\pgfqpoint{3.273153in}{0.500000in}}%
\pgfpathlineto{\pgfqpoint{3.306179in}{0.500000in}}%
\pgfpathlineto{\pgfqpoint{3.306179in}{2.816931in}}%
\pgfpathlineto{\pgfqpoint{3.273153in}{2.816931in}}%
\pgfpathlineto{\pgfqpoint{3.273153in}{0.500000in}}%
\pgfpathclose%
\pgfusepath{fill}%
\end{pgfscope}%
\begin{pgfscope}%
\pgfpathrectangle{\pgfqpoint{0.750000in}{0.500000in}}{\pgfqpoint{4.650000in}{3.020000in}}%
\pgfusepath{clip}%
\pgfsetbuttcap%
\pgfsetmiterjoin%
\definecolor{currentfill}{rgb}{0.000000,0.500000,0.000000}%
\pgfsetfillcolor{currentfill}%
\pgfsetlinewidth{0.000000pt}%
\definecolor{currentstroke}{rgb}{0.000000,0.000000,0.000000}%
\pgfsetstrokecolor{currentstroke}%
\pgfsetstrokeopacity{0.000000}%
\pgfsetdash{}{0pt}%
\pgfpathmoveto{\pgfqpoint{3.306179in}{0.500000in}}%
\pgfpathlineto{\pgfqpoint{3.339205in}{0.500000in}}%
\pgfpathlineto{\pgfqpoint{3.339205in}{3.008677in}}%
\pgfpathlineto{\pgfqpoint{3.306179in}{3.008677in}}%
\pgfpathlineto{\pgfqpoint{3.306179in}{0.500000in}}%
\pgfpathclose%
\pgfusepath{fill}%
\end{pgfscope}%
\begin{pgfscope}%
\pgfpathrectangle{\pgfqpoint{0.750000in}{0.500000in}}{\pgfqpoint{4.650000in}{3.020000in}}%
\pgfusepath{clip}%
\pgfsetbuttcap%
\pgfsetmiterjoin%
\definecolor{currentfill}{rgb}{0.000000,0.500000,0.000000}%
\pgfsetfillcolor{currentfill}%
\pgfsetlinewidth{0.000000pt}%
\definecolor{currentstroke}{rgb}{0.000000,0.000000,0.000000}%
\pgfsetstrokecolor{currentstroke}%
\pgfsetstrokeopacity{0.000000}%
\pgfsetdash{}{0pt}%
\pgfpathmoveto{\pgfqpoint{3.339205in}{0.500000in}}%
\pgfpathlineto{\pgfqpoint{3.372230in}{0.500000in}}%
\pgfpathlineto{\pgfqpoint{3.372230in}{3.296296in}}%
\pgfpathlineto{\pgfqpoint{3.339205in}{3.296296in}}%
\pgfpathlineto{\pgfqpoint{3.339205in}{0.500000in}}%
\pgfpathclose%
\pgfusepath{fill}%
\end{pgfscope}%
\begin{pgfscope}%
\pgfpathrectangle{\pgfqpoint{0.750000in}{0.500000in}}{\pgfqpoint{4.650000in}{3.020000in}}%
\pgfusepath{clip}%
\pgfsetbuttcap%
\pgfsetmiterjoin%
\definecolor{currentfill}{rgb}{0.000000,0.500000,0.000000}%
\pgfsetfillcolor{currentfill}%
\pgfsetlinewidth{0.000000pt}%
\definecolor{currentstroke}{rgb}{0.000000,0.000000,0.000000}%
\pgfsetstrokecolor{currentstroke}%
\pgfsetstrokeopacity{0.000000}%
\pgfsetdash{}{0pt}%
\pgfpathmoveto{\pgfqpoint{3.372230in}{0.500000in}}%
\pgfpathlineto{\pgfqpoint{3.405256in}{0.500000in}}%
\pgfpathlineto{\pgfqpoint{3.405256in}{3.376190in}}%
\pgfpathlineto{\pgfqpoint{3.372230in}{3.376190in}}%
\pgfpathlineto{\pgfqpoint{3.372230in}{0.500000in}}%
\pgfpathclose%
\pgfusepath{fill}%
\end{pgfscope}%
\begin{pgfscope}%
\pgfpathrectangle{\pgfqpoint{0.750000in}{0.500000in}}{\pgfqpoint{4.650000in}{3.020000in}}%
\pgfusepath{clip}%
\pgfsetbuttcap%
\pgfsetmiterjoin%
\definecolor{currentfill}{rgb}{0.000000,0.500000,0.000000}%
\pgfsetfillcolor{currentfill}%
\pgfsetlinewidth{0.000000pt}%
\definecolor{currentstroke}{rgb}{0.000000,0.000000,0.000000}%
\pgfsetstrokecolor{currentstroke}%
\pgfsetstrokeopacity{0.000000}%
\pgfsetdash{}{0pt}%
\pgfpathmoveto{\pgfqpoint{3.405256in}{0.500000in}}%
\pgfpathlineto{\pgfqpoint{3.438281in}{0.500000in}}%
\pgfpathlineto{\pgfqpoint{3.438281in}{2.992698in}}%
\pgfpathlineto{\pgfqpoint{3.405256in}{2.992698in}}%
\pgfpathlineto{\pgfqpoint{3.405256in}{0.500000in}}%
\pgfpathclose%
\pgfusepath{fill}%
\end{pgfscope}%
\begin{pgfscope}%
\pgfpathrectangle{\pgfqpoint{0.750000in}{0.500000in}}{\pgfqpoint{4.650000in}{3.020000in}}%
\pgfusepath{clip}%
\pgfsetbuttcap%
\pgfsetmiterjoin%
\definecolor{currentfill}{rgb}{0.000000,0.500000,0.000000}%
\pgfsetfillcolor{currentfill}%
\pgfsetlinewidth{0.000000pt}%
\definecolor{currentstroke}{rgb}{0.000000,0.000000,0.000000}%
\pgfsetstrokecolor{currentstroke}%
\pgfsetstrokeopacity{0.000000}%
\pgfsetdash{}{0pt}%
\pgfpathmoveto{\pgfqpoint{3.438281in}{0.500000in}}%
\pgfpathlineto{\pgfqpoint{3.471307in}{0.500000in}}%
\pgfpathlineto{\pgfqpoint{3.471307in}{2.912804in}}%
\pgfpathlineto{\pgfqpoint{3.438281in}{2.912804in}}%
\pgfpathlineto{\pgfqpoint{3.438281in}{0.500000in}}%
\pgfpathclose%
\pgfusepath{fill}%
\end{pgfscope}%
\begin{pgfscope}%
\pgfpathrectangle{\pgfqpoint{0.750000in}{0.500000in}}{\pgfqpoint{4.650000in}{3.020000in}}%
\pgfusepath{clip}%
\pgfsetbuttcap%
\pgfsetmiterjoin%
\definecolor{currentfill}{rgb}{0.000000,0.500000,0.000000}%
\pgfsetfillcolor{currentfill}%
\pgfsetlinewidth{0.000000pt}%
\definecolor{currentstroke}{rgb}{0.000000,0.000000,0.000000}%
\pgfsetstrokecolor{currentstroke}%
\pgfsetstrokeopacity{0.000000}%
\pgfsetdash{}{0pt}%
\pgfpathmoveto{\pgfqpoint{3.471307in}{0.500000in}}%
\pgfpathlineto{\pgfqpoint{3.504332in}{0.500000in}}%
\pgfpathlineto{\pgfqpoint{3.504332in}{2.465397in}}%
\pgfpathlineto{\pgfqpoint{3.471307in}{2.465397in}}%
\pgfpathlineto{\pgfqpoint{3.471307in}{0.500000in}}%
\pgfpathclose%
\pgfusepath{fill}%
\end{pgfscope}%
\begin{pgfscope}%
\pgfpathrectangle{\pgfqpoint{0.750000in}{0.500000in}}{\pgfqpoint{4.650000in}{3.020000in}}%
\pgfusepath{clip}%
\pgfsetbuttcap%
\pgfsetmiterjoin%
\definecolor{currentfill}{rgb}{0.000000,0.500000,0.000000}%
\pgfsetfillcolor{currentfill}%
\pgfsetlinewidth{0.000000pt}%
\definecolor{currentstroke}{rgb}{0.000000,0.000000,0.000000}%
\pgfsetstrokecolor{currentstroke}%
\pgfsetstrokeopacity{0.000000}%
\pgfsetdash{}{0pt}%
\pgfpathmoveto{\pgfqpoint{3.504332in}{0.500000in}}%
\pgfpathlineto{\pgfqpoint{3.537358in}{0.500000in}}%
\pgfpathlineto{\pgfqpoint{3.537358in}{2.465397in}}%
\pgfpathlineto{\pgfqpoint{3.504332in}{2.465397in}}%
\pgfpathlineto{\pgfqpoint{3.504332in}{0.500000in}}%
\pgfpathclose%
\pgfusepath{fill}%
\end{pgfscope}%
\begin{pgfscope}%
\pgfpathrectangle{\pgfqpoint{0.750000in}{0.500000in}}{\pgfqpoint{4.650000in}{3.020000in}}%
\pgfusepath{clip}%
\pgfsetbuttcap%
\pgfsetmiterjoin%
\definecolor{currentfill}{rgb}{0.000000,0.500000,0.000000}%
\pgfsetfillcolor{currentfill}%
\pgfsetlinewidth{0.000000pt}%
\definecolor{currentstroke}{rgb}{0.000000,0.000000,0.000000}%
\pgfsetstrokecolor{currentstroke}%
\pgfsetstrokeopacity{0.000000}%
\pgfsetdash{}{0pt}%
\pgfpathmoveto{\pgfqpoint{3.537358in}{0.500000in}}%
\pgfpathlineto{\pgfqpoint{3.570384in}{0.500000in}}%
\pgfpathlineto{\pgfqpoint{3.570384in}{2.497354in}}%
\pgfpathlineto{\pgfqpoint{3.537358in}{2.497354in}}%
\pgfpathlineto{\pgfqpoint{3.537358in}{0.500000in}}%
\pgfpathclose%
\pgfusepath{fill}%
\end{pgfscope}%
\begin{pgfscope}%
\pgfpathrectangle{\pgfqpoint{0.750000in}{0.500000in}}{\pgfqpoint{4.650000in}{3.020000in}}%
\pgfusepath{clip}%
\pgfsetbuttcap%
\pgfsetmiterjoin%
\definecolor{currentfill}{rgb}{0.000000,0.500000,0.000000}%
\pgfsetfillcolor{currentfill}%
\pgfsetlinewidth{0.000000pt}%
\definecolor{currentstroke}{rgb}{0.000000,0.000000,0.000000}%
\pgfsetstrokecolor{currentstroke}%
\pgfsetstrokeopacity{0.000000}%
\pgfsetdash{}{0pt}%
\pgfpathmoveto{\pgfqpoint{3.570384in}{0.500000in}}%
\pgfpathlineto{\pgfqpoint{3.603409in}{0.500000in}}%
\pgfpathlineto{\pgfqpoint{3.603409in}{2.545291in}}%
\pgfpathlineto{\pgfqpoint{3.570384in}{2.545291in}}%
\pgfpathlineto{\pgfqpoint{3.570384in}{0.500000in}}%
\pgfpathclose%
\pgfusepath{fill}%
\end{pgfscope}%
\begin{pgfscope}%
\pgfpathrectangle{\pgfqpoint{0.750000in}{0.500000in}}{\pgfqpoint{4.650000in}{3.020000in}}%
\pgfusepath{clip}%
\pgfsetbuttcap%
\pgfsetmiterjoin%
\definecolor{currentfill}{rgb}{0.000000,0.500000,0.000000}%
\pgfsetfillcolor{currentfill}%
\pgfsetlinewidth{0.000000pt}%
\definecolor{currentstroke}{rgb}{0.000000,0.000000,0.000000}%
\pgfsetstrokecolor{currentstroke}%
\pgfsetstrokeopacity{0.000000}%
\pgfsetdash{}{0pt}%
\pgfpathmoveto{\pgfqpoint{3.603409in}{0.500000in}}%
\pgfpathlineto{\pgfqpoint{3.636435in}{0.500000in}}%
\pgfpathlineto{\pgfqpoint{3.636435in}{2.289630in}}%
\pgfpathlineto{\pgfqpoint{3.603409in}{2.289630in}}%
\pgfpathlineto{\pgfqpoint{3.603409in}{0.500000in}}%
\pgfpathclose%
\pgfusepath{fill}%
\end{pgfscope}%
\begin{pgfscope}%
\pgfpathrectangle{\pgfqpoint{0.750000in}{0.500000in}}{\pgfqpoint{4.650000in}{3.020000in}}%
\pgfusepath{clip}%
\pgfsetbuttcap%
\pgfsetmiterjoin%
\definecolor{currentfill}{rgb}{0.000000,0.500000,0.000000}%
\pgfsetfillcolor{currentfill}%
\pgfsetlinewidth{0.000000pt}%
\definecolor{currentstroke}{rgb}{0.000000,0.000000,0.000000}%
\pgfsetstrokecolor{currentstroke}%
\pgfsetstrokeopacity{0.000000}%
\pgfsetdash{}{0pt}%
\pgfpathmoveto{\pgfqpoint{3.636435in}{0.500000in}}%
\pgfpathlineto{\pgfqpoint{3.669460in}{0.500000in}}%
\pgfpathlineto{\pgfqpoint{3.669460in}{2.289630in}}%
\pgfpathlineto{\pgfqpoint{3.636435in}{2.289630in}}%
\pgfpathlineto{\pgfqpoint{3.636435in}{0.500000in}}%
\pgfpathclose%
\pgfusepath{fill}%
\end{pgfscope}%
\begin{pgfscope}%
\pgfpathrectangle{\pgfqpoint{0.750000in}{0.500000in}}{\pgfqpoint{4.650000in}{3.020000in}}%
\pgfusepath{clip}%
\pgfsetbuttcap%
\pgfsetmiterjoin%
\definecolor{currentfill}{rgb}{0.000000,0.500000,0.000000}%
\pgfsetfillcolor{currentfill}%
\pgfsetlinewidth{0.000000pt}%
\definecolor{currentstroke}{rgb}{0.000000,0.000000,0.000000}%
\pgfsetstrokecolor{currentstroke}%
\pgfsetstrokeopacity{0.000000}%
\pgfsetdash{}{0pt}%
\pgfpathmoveto{\pgfqpoint{3.669460in}{0.500000in}}%
\pgfpathlineto{\pgfqpoint{3.702486in}{0.500000in}}%
\pgfpathlineto{\pgfqpoint{3.702486in}{1.810265in}}%
\pgfpathlineto{\pgfqpoint{3.669460in}{1.810265in}}%
\pgfpathlineto{\pgfqpoint{3.669460in}{0.500000in}}%
\pgfpathclose%
\pgfusepath{fill}%
\end{pgfscope}%
\begin{pgfscope}%
\pgfpathrectangle{\pgfqpoint{0.750000in}{0.500000in}}{\pgfqpoint{4.650000in}{3.020000in}}%
\pgfusepath{clip}%
\pgfsetbuttcap%
\pgfsetmiterjoin%
\definecolor{currentfill}{rgb}{0.000000,0.500000,0.000000}%
\pgfsetfillcolor{currentfill}%
\pgfsetlinewidth{0.000000pt}%
\definecolor{currentstroke}{rgb}{0.000000,0.000000,0.000000}%
\pgfsetstrokecolor{currentstroke}%
\pgfsetstrokeopacity{0.000000}%
\pgfsetdash{}{0pt}%
\pgfpathmoveto{\pgfqpoint{3.702486in}{0.500000in}}%
\pgfpathlineto{\pgfqpoint{3.735511in}{0.500000in}}%
\pgfpathlineto{\pgfqpoint{3.735511in}{2.033968in}}%
\pgfpathlineto{\pgfqpoint{3.702486in}{2.033968in}}%
\pgfpathlineto{\pgfqpoint{3.702486in}{0.500000in}}%
\pgfpathclose%
\pgfusepath{fill}%
\end{pgfscope}%
\begin{pgfscope}%
\pgfpathrectangle{\pgfqpoint{0.750000in}{0.500000in}}{\pgfqpoint{4.650000in}{3.020000in}}%
\pgfusepath{clip}%
\pgfsetbuttcap%
\pgfsetmiterjoin%
\definecolor{currentfill}{rgb}{0.000000,0.500000,0.000000}%
\pgfsetfillcolor{currentfill}%
\pgfsetlinewidth{0.000000pt}%
\definecolor{currentstroke}{rgb}{0.000000,0.000000,0.000000}%
\pgfsetstrokecolor{currentstroke}%
\pgfsetstrokeopacity{0.000000}%
\pgfsetdash{}{0pt}%
\pgfpathmoveto{\pgfqpoint{3.735511in}{0.500000in}}%
\pgfpathlineto{\pgfqpoint{3.768537in}{0.500000in}}%
\pgfpathlineto{\pgfqpoint{3.768537in}{2.065926in}}%
\pgfpathlineto{\pgfqpoint{3.735511in}{2.065926in}}%
\pgfpathlineto{\pgfqpoint{3.735511in}{0.500000in}}%
\pgfpathclose%
\pgfusepath{fill}%
\end{pgfscope}%
\begin{pgfscope}%
\pgfpathrectangle{\pgfqpoint{0.750000in}{0.500000in}}{\pgfqpoint{4.650000in}{3.020000in}}%
\pgfusepath{clip}%
\pgfsetbuttcap%
\pgfsetmiterjoin%
\definecolor{currentfill}{rgb}{0.000000,0.500000,0.000000}%
\pgfsetfillcolor{currentfill}%
\pgfsetlinewidth{0.000000pt}%
\definecolor{currentstroke}{rgb}{0.000000,0.000000,0.000000}%
\pgfsetstrokecolor{currentstroke}%
\pgfsetstrokeopacity{0.000000}%
\pgfsetdash{}{0pt}%
\pgfpathmoveto{\pgfqpoint{3.768537in}{0.500000in}}%
\pgfpathlineto{\pgfqpoint{3.801563in}{0.500000in}}%
\pgfpathlineto{\pgfqpoint{3.801563in}{1.826243in}}%
\pgfpathlineto{\pgfqpoint{3.768537in}{1.826243in}}%
\pgfpathlineto{\pgfqpoint{3.768537in}{0.500000in}}%
\pgfpathclose%
\pgfusepath{fill}%
\end{pgfscope}%
\begin{pgfscope}%
\pgfpathrectangle{\pgfqpoint{0.750000in}{0.500000in}}{\pgfqpoint{4.650000in}{3.020000in}}%
\pgfusepath{clip}%
\pgfsetbuttcap%
\pgfsetmiterjoin%
\definecolor{currentfill}{rgb}{0.000000,0.500000,0.000000}%
\pgfsetfillcolor{currentfill}%
\pgfsetlinewidth{0.000000pt}%
\definecolor{currentstroke}{rgb}{0.000000,0.000000,0.000000}%
\pgfsetstrokecolor{currentstroke}%
\pgfsetstrokeopacity{0.000000}%
\pgfsetdash{}{0pt}%
\pgfpathmoveto{\pgfqpoint{3.801563in}{0.500000in}}%
\pgfpathlineto{\pgfqpoint{3.834588in}{0.500000in}}%
\pgfpathlineto{\pgfqpoint{3.834588in}{1.618519in}}%
\pgfpathlineto{\pgfqpoint{3.801563in}{1.618519in}}%
\pgfpathlineto{\pgfqpoint{3.801563in}{0.500000in}}%
\pgfpathclose%
\pgfusepath{fill}%
\end{pgfscope}%
\begin{pgfscope}%
\pgfpathrectangle{\pgfqpoint{0.750000in}{0.500000in}}{\pgfqpoint{4.650000in}{3.020000in}}%
\pgfusepath{clip}%
\pgfsetbuttcap%
\pgfsetmiterjoin%
\definecolor{currentfill}{rgb}{0.000000,0.500000,0.000000}%
\pgfsetfillcolor{currentfill}%
\pgfsetlinewidth{0.000000pt}%
\definecolor{currentstroke}{rgb}{0.000000,0.000000,0.000000}%
\pgfsetstrokecolor{currentstroke}%
\pgfsetstrokeopacity{0.000000}%
\pgfsetdash{}{0pt}%
\pgfpathmoveto{\pgfqpoint{3.834588in}{0.500000in}}%
\pgfpathlineto{\pgfqpoint{3.867614in}{0.500000in}}%
\pgfpathlineto{\pgfqpoint{3.867614in}{1.874180in}}%
\pgfpathlineto{\pgfqpoint{3.834588in}{1.874180in}}%
\pgfpathlineto{\pgfqpoint{3.834588in}{0.500000in}}%
\pgfpathclose%
\pgfusepath{fill}%
\end{pgfscope}%
\begin{pgfscope}%
\pgfpathrectangle{\pgfqpoint{0.750000in}{0.500000in}}{\pgfqpoint{4.650000in}{3.020000in}}%
\pgfusepath{clip}%
\pgfsetbuttcap%
\pgfsetmiterjoin%
\definecolor{currentfill}{rgb}{0.000000,0.500000,0.000000}%
\pgfsetfillcolor{currentfill}%
\pgfsetlinewidth{0.000000pt}%
\definecolor{currentstroke}{rgb}{0.000000,0.000000,0.000000}%
\pgfsetstrokecolor{currentstroke}%
\pgfsetstrokeopacity{0.000000}%
\pgfsetdash{}{0pt}%
\pgfpathmoveto{\pgfqpoint{3.867614in}{0.500000in}}%
\pgfpathlineto{\pgfqpoint{3.900639in}{0.500000in}}%
\pgfpathlineto{\pgfqpoint{3.900639in}{1.554603in}}%
\pgfpathlineto{\pgfqpoint{3.867614in}{1.554603in}}%
\pgfpathlineto{\pgfqpoint{3.867614in}{0.500000in}}%
\pgfpathclose%
\pgfusepath{fill}%
\end{pgfscope}%
\begin{pgfscope}%
\pgfpathrectangle{\pgfqpoint{0.750000in}{0.500000in}}{\pgfqpoint{4.650000in}{3.020000in}}%
\pgfusepath{clip}%
\pgfsetbuttcap%
\pgfsetmiterjoin%
\definecolor{currentfill}{rgb}{0.000000,0.500000,0.000000}%
\pgfsetfillcolor{currentfill}%
\pgfsetlinewidth{0.000000pt}%
\definecolor{currentstroke}{rgb}{0.000000,0.000000,0.000000}%
\pgfsetstrokecolor{currentstroke}%
\pgfsetstrokeopacity{0.000000}%
\pgfsetdash{}{0pt}%
\pgfpathmoveto{\pgfqpoint{3.900639in}{0.500000in}}%
\pgfpathlineto{\pgfqpoint{3.933665in}{0.500000in}}%
\pgfpathlineto{\pgfqpoint{3.933665in}{1.554603in}}%
\pgfpathlineto{\pgfqpoint{3.900639in}{1.554603in}}%
\pgfpathlineto{\pgfqpoint{3.900639in}{0.500000in}}%
\pgfpathclose%
\pgfusepath{fill}%
\end{pgfscope}%
\begin{pgfscope}%
\pgfpathrectangle{\pgfqpoint{0.750000in}{0.500000in}}{\pgfqpoint{4.650000in}{3.020000in}}%
\pgfusepath{clip}%
\pgfsetbuttcap%
\pgfsetmiterjoin%
\definecolor{currentfill}{rgb}{0.000000,0.500000,0.000000}%
\pgfsetfillcolor{currentfill}%
\pgfsetlinewidth{0.000000pt}%
\definecolor{currentstroke}{rgb}{0.000000,0.000000,0.000000}%
\pgfsetstrokecolor{currentstroke}%
\pgfsetstrokeopacity{0.000000}%
\pgfsetdash{}{0pt}%
\pgfpathmoveto{\pgfqpoint{3.933665in}{0.500000in}}%
\pgfpathlineto{\pgfqpoint{3.966690in}{0.500000in}}%
\pgfpathlineto{\pgfqpoint{3.966690in}{1.410794in}}%
\pgfpathlineto{\pgfqpoint{3.933665in}{1.410794in}}%
\pgfpathlineto{\pgfqpoint{3.933665in}{0.500000in}}%
\pgfpathclose%
\pgfusepath{fill}%
\end{pgfscope}%
\begin{pgfscope}%
\pgfpathrectangle{\pgfqpoint{0.750000in}{0.500000in}}{\pgfqpoint{4.650000in}{3.020000in}}%
\pgfusepath{clip}%
\pgfsetbuttcap%
\pgfsetmiterjoin%
\definecolor{currentfill}{rgb}{0.000000,0.500000,0.000000}%
\pgfsetfillcolor{currentfill}%
\pgfsetlinewidth{0.000000pt}%
\definecolor{currentstroke}{rgb}{0.000000,0.000000,0.000000}%
\pgfsetstrokecolor{currentstroke}%
\pgfsetstrokeopacity{0.000000}%
\pgfsetdash{}{0pt}%
\pgfpathmoveto{\pgfqpoint{3.966690in}{0.500000in}}%
\pgfpathlineto{\pgfqpoint{3.999716in}{0.500000in}}%
\pgfpathlineto{\pgfqpoint{3.999716in}{1.314921in}}%
\pgfpathlineto{\pgfqpoint{3.966690in}{1.314921in}}%
\pgfpathlineto{\pgfqpoint{3.966690in}{0.500000in}}%
\pgfpathclose%
\pgfusepath{fill}%
\end{pgfscope}%
\begin{pgfscope}%
\pgfpathrectangle{\pgfqpoint{0.750000in}{0.500000in}}{\pgfqpoint{4.650000in}{3.020000in}}%
\pgfusepath{clip}%
\pgfsetbuttcap%
\pgfsetmiterjoin%
\definecolor{currentfill}{rgb}{0.000000,0.500000,0.000000}%
\pgfsetfillcolor{currentfill}%
\pgfsetlinewidth{0.000000pt}%
\definecolor{currentstroke}{rgb}{0.000000,0.000000,0.000000}%
\pgfsetstrokecolor{currentstroke}%
\pgfsetstrokeopacity{0.000000}%
\pgfsetdash{}{0pt}%
\pgfpathmoveto{\pgfqpoint{3.999716in}{0.500000in}}%
\pgfpathlineto{\pgfqpoint{4.032741in}{0.500000in}}%
\pgfpathlineto{\pgfqpoint{4.032741in}{1.378836in}}%
\pgfpathlineto{\pgfqpoint{3.999716in}{1.378836in}}%
\pgfpathlineto{\pgfqpoint{3.999716in}{0.500000in}}%
\pgfpathclose%
\pgfusepath{fill}%
\end{pgfscope}%
\begin{pgfscope}%
\pgfpathrectangle{\pgfqpoint{0.750000in}{0.500000in}}{\pgfqpoint{4.650000in}{3.020000in}}%
\pgfusepath{clip}%
\pgfsetbuttcap%
\pgfsetmiterjoin%
\definecolor{currentfill}{rgb}{0.000000,0.500000,0.000000}%
\pgfsetfillcolor{currentfill}%
\pgfsetlinewidth{0.000000pt}%
\definecolor{currentstroke}{rgb}{0.000000,0.000000,0.000000}%
\pgfsetstrokecolor{currentstroke}%
\pgfsetstrokeopacity{0.000000}%
\pgfsetdash{}{0pt}%
\pgfpathmoveto{\pgfqpoint{4.032741in}{0.500000in}}%
\pgfpathlineto{\pgfqpoint{4.065767in}{0.500000in}}%
\pgfpathlineto{\pgfqpoint{4.065767in}{1.426772in}}%
\pgfpathlineto{\pgfqpoint{4.032741in}{1.426772in}}%
\pgfpathlineto{\pgfqpoint{4.032741in}{0.500000in}}%
\pgfpathclose%
\pgfusepath{fill}%
\end{pgfscope}%
\begin{pgfscope}%
\pgfpathrectangle{\pgfqpoint{0.750000in}{0.500000in}}{\pgfqpoint{4.650000in}{3.020000in}}%
\pgfusepath{clip}%
\pgfsetbuttcap%
\pgfsetmiterjoin%
\definecolor{currentfill}{rgb}{0.000000,0.500000,0.000000}%
\pgfsetfillcolor{currentfill}%
\pgfsetlinewidth{0.000000pt}%
\definecolor{currentstroke}{rgb}{0.000000,0.000000,0.000000}%
\pgfsetstrokecolor{currentstroke}%
\pgfsetstrokeopacity{0.000000}%
\pgfsetdash{}{0pt}%
\pgfpathmoveto{\pgfqpoint{4.065767in}{0.500000in}}%
\pgfpathlineto{\pgfqpoint{4.098793in}{0.500000in}}%
\pgfpathlineto{\pgfqpoint{4.098793in}{1.187090in}}%
\pgfpathlineto{\pgfqpoint{4.065767in}{1.187090in}}%
\pgfpathlineto{\pgfqpoint{4.065767in}{0.500000in}}%
\pgfpathclose%
\pgfusepath{fill}%
\end{pgfscope}%
\begin{pgfscope}%
\pgfpathrectangle{\pgfqpoint{0.750000in}{0.500000in}}{\pgfqpoint{4.650000in}{3.020000in}}%
\pgfusepath{clip}%
\pgfsetbuttcap%
\pgfsetmiterjoin%
\definecolor{currentfill}{rgb}{0.000000,0.500000,0.000000}%
\pgfsetfillcolor{currentfill}%
\pgfsetlinewidth{0.000000pt}%
\definecolor{currentstroke}{rgb}{0.000000,0.000000,0.000000}%
\pgfsetstrokecolor{currentstroke}%
\pgfsetstrokeopacity{0.000000}%
\pgfsetdash{}{0pt}%
\pgfpathmoveto{\pgfqpoint{4.098793in}{0.500000in}}%
\pgfpathlineto{\pgfqpoint{4.131818in}{0.500000in}}%
\pgfpathlineto{\pgfqpoint{4.131818in}{1.011323in}}%
\pgfpathlineto{\pgfqpoint{4.098793in}{1.011323in}}%
\pgfpathlineto{\pgfqpoint{4.098793in}{0.500000in}}%
\pgfpathclose%
\pgfusepath{fill}%
\end{pgfscope}%
\begin{pgfscope}%
\pgfpathrectangle{\pgfqpoint{0.750000in}{0.500000in}}{\pgfqpoint{4.650000in}{3.020000in}}%
\pgfusepath{clip}%
\pgfsetbuttcap%
\pgfsetmiterjoin%
\definecolor{currentfill}{rgb}{0.000000,0.500000,0.000000}%
\pgfsetfillcolor{currentfill}%
\pgfsetlinewidth{0.000000pt}%
\definecolor{currentstroke}{rgb}{0.000000,0.000000,0.000000}%
\pgfsetstrokecolor{currentstroke}%
\pgfsetstrokeopacity{0.000000}%
\pgfsetdash{}{0pt}%
\pgfpathmoveto{\pgfqpoint{4.131818in}{0.500000in}}%
\pgfpathlineto{\pgfqpoint{4.164844in}{0.500000in}}%
\pgfpathlineto{\pgfqpoint{4.164844in}{1.059259in}}%
\pgfpathlineto{\pgfqpoint{4.131818in}{1.059259in}}%
\pgfpathlineto{\pgfqpoint{4.131818in}{0.500000in}}%
\pgfpathclose%
\pgfusepath{fill}%
\end{pgfscope}%
\begin{pgfscope}%
\pgfpathrectangle{\pgfqpoint{0.750000in}{0.500000in}}{\pgfqpoint{4.650000in}{3.020000in}}%
\pgfusepath{clip}%
\pgfsetbuttcap%
\pgfsetmiterjoin%
\definecolor{currentfill}{rgb}{0.000000,0.500000,0.000000}%
\pgfsetfillcolor{currentfill}%
\pgfsetlinewidth{0.000000pt}%
\definecolor{currentstroke}{rgb}{0.000000,0.000000,0.000000}%
\pgfsetstrokecolor{currentstroke}%
\pgfsetstrokeopacity{0.000000}%
\pgfsetdash{}{0pt}%
\pgfpathmoveto{\pgfqpoint{4.164844in}{0.500000in}}%
\pgfpathlineto{\pgfqpoint{4.197869in}{0.500000in}}%
\pgfpathlineto{\pgfqpoint{4.197869in}{1.059259in}}%
\pgfpathlineto{\pgfqpoint{4.164844in}{1.059259in}}%
\pgfpathlineto{\pgfqpoint{4.164844in}{0.500000in}}%
\pgfpathclose%
\pgfusepath{fill}%
\end{pgfscope}%
\begin{pgfscope}%
\pgfpathrectangle{\pgfqpoint{0.750000in}{0.500000in}}{\pgfqpoint{4.650000in}{3.020000in}}%
\pgfusepath{clip}%
\pgfsetbuttcap%
\pgfsetmiterjoin%
\definecolor{currentfill}{rgb}{0.000000,0.500000,0.000000}%
\pgfsetfillcolor{currentfill}%
\pgfsetlinewidth{0.000000pt}%
\definecolor{currentstroke}{rgb}{0.000000,0.000000,0.000000}%
\pgfsetstrokecolor{currentstroke}%
\pgfsetstrokeopacity{0.000000}%
\pgfsetdash{}{0pt}%
\pgfpathmoveto{\pgfqpoint{4.197869in}{0.500000in}}%
\pgfpathlineto{\pgfqpoint{4.230895in}{0.500000in}}%
\pgfpathlineto{\pgfqpoint{4.230895in}{1.027302in}}%
\pgfpathlineto{\pgfqpoint{4.197869in}{1.027302in}}%
\pgfpathlineto{\pgfqpoint{4.197869in}{0.500000in}}%
\pgfpathclose%
\pgfusepath{fill}%
\end{pgfscope}%
\begin{pgfscope}%
\pgfpathrectangle{\pgfqpoint{0.750000in}{0.500000in}}{\pgfqpoint{4.650000in}{3.020000in}}%
\pgfusepath{clip}%
\pgfsetbuttcap%
\pgfsetmiterjoin%
\definecolor{currentfill}{rgb}{0.000000,0.500000,0.000000}%
\pgfsetfillcolor{currentfill}%
\pgfsetlinewidth{0.000000pt}%
\definecolor{currentstroke}{rgb}{0.000000,0.000000,0.000000}%
\pgfsetstrokecolor{currentstroke}%
\pgfsetstrokeopacity{0.000000}%
\pgfsetdash{}{0pt}%
\pgfpathmoveto{\pgfqpoint{4.230895in}{0.500000in}}%
\pgfpathlineto{\pgfqpoint{4.263920in}{0.500000in}}%
\pgfpathlineto{\pgfqpoint{4.263920in}{0.835556in}}%
\pgfpathlineto{\pgfqpoint{4.230895in}{0.835556in}}%
\pgfpathlineto{\pgfqpoint{4.230895in}{0.500000in}}%
\pgfpathclose%
\pgfusepath{fill}%
\end{pgfscope}%
\begin{pgfscope}%
\pgfpathrectangle{\pgfqpoint{0.750000in}{0.500000in}}{\pgfqpoint{4.650000in}{3.020000in}}%
\pgfusepath{clip}%
\pgfsetbuttcap%
\pgfsetmiterjoin%
\definecolor{currentfill}{rgb}{0.000000,0.500000,0.000000}%
\pgfsetfillcolor{currentfill}%
\pgfsetlinewidth{0.000000pt}%
\definecolor{currentstroke}{rgb}{0.000000,0.000000,0.000000}%
\pgfsetstrokecolor{currentstroke}%
\pgfsetstrokeopacity{0.000000}%
\pgfsetdash{}{0pt}%
\pgfpathmoveto{\pgfqpoint{4.263920in}{0.500000in}}%
\pgfpathlineto{\pgfqpoint{4.296946in}{0.500000in}}%
\pgfpathlineto{\pgfqpoint{4.296946in}{0.851534in}}%
\pgfpathlineto{\pgfqpoint{4.263920in}{0.851534in}}%
\pgfpathlineto{\pgfqpoint{4.263920in}{0.500000in}}%
\pgfpathclose%
\pgfusepath{fill}%
\end{pgfscope}%
\begin{pgfscope}%
\pgfpathrectangle{\pgfqpoint{0.750000in}{0.500000in}}{\pgfqpoint{4.650000in}{3.020000in}}%
\pgfusepath{clip}%
\pgfsetbuttcap%
\pgfsetmiterjoin%
\definecolor{currentfill}{rgb}{0.000000,0.500000,0.000000}%
\pgfsetfillcolor{currentfill}%
\pgfsetlinewidth{0.000000pt}%
\definecolor{currentstroke}{rgb}{0.000000,0.000000,0.000000}%
\pgfsetstrokecolor{currentstroke}%
\pgfsetstrokeopacity{0.000000}%
\pgfsetdash{}{0pt}%
\pgfpathmoveto{\pgfqpoint{4.296946in}{0.500000in}}%
\pgfpathlineto{\pgfqpoint{4.329972in}{0.500000in}}%
\pgfpathlineto{\pgfqpoint{4.329972in}{0.867513in}}%
\pgfpathlineto{\pgfqpoint{4.296946in}{0.867513in}}%
\pgfpathlineto{\pgfqpoint{4.296946in}{0.500000in}}%
\pgfpathclose%
\pgfusepath{fill}%
\end{pgfscope}%
\begin{pgfscope}%
\pgfpathrectangle{\pgfqpoint{0.750000in}{0.500000in}}{\pgfqpoint{4.650000in}{3.020000in}}%
\pgfusepath{clip}%
\pgfsetbuttcap%
\pgfsetmiterjoin%
\definecolor{currentfill}{rgb}{0.000000,0.500000,0.000000}%
\pgfsetfillcolor{currentfill}%
\pgfsetlinewidth{0.000000pt}%
\definecolor{currentstroke}{rgb}{0.000000,0.000000,0.000000}%
\pgfsetstrokecolor{currentstroke}%
\pgfsetstrokeopacity{0.000000}%
\pgfsetdash{}{0pt}%
\pgfpathmoveto{\pgfqpoint{4.329972in}{0.500000in}}%
\pgfpathlineto{\pgfqpoint{4.362997in}{0.500000in}}%
\pgfpathlineto{\pgfqpoint{4.362997in}{0.787619in}}%
\pgfpathlineto{\pgfqpoint{4.329972in}{0.787619in}}%
\pgfpathlineto{\pgfqpoint{4.329972in}{0.500000in}}%
\pgfpathclose%
\pgfusepath{fill}%
\end{pgfscope}%
\begin{pgfscope}%
\pgfpathrectangle{\pgfqpoint{0.750000in}{0.500000in}}{\pgfqpoint{4.650000in}{3.020000in}}%
\pgfusepath{clip}%
\pgfsetbuttcap%
\pgfsetmiterjoin%
\definecolor{currentfill}{rgb}{0.000000,0.500000,0.000000}%
\pgfsetfillcolor{currentfill}%
\pgfsetlinewidth{0.000000pt}%
\definecolor{currentstroke}{rgb}{0.000000,0.000000,0.000000}%
\pgfsetstrokecolor{currentstroke}%
\pgfsetstrokeopacity{0.000000}%
\pgfsetdash{}{0pt}%
\pgfpathmoveto{\pgfqpoint{4.362997in}{0.500000in}}%
\pgfpathlineto{\pgfqpoint{4.396023in}{0.500000in}}%
\pgfpathlineto{\pgfqpoint{4.396023in}{0.771640in}}%
\pgfpathlineto{\pgfqpoint{4.362997in}{0.771640in}}%
\pgfpathlineto{\pgfqpoint{4.362997in}{0.500000in}}%
\pgfpathclose%
\pgfusepath{fill}%
\end{pgfscope}%
\begin{pgfscope}%
\pgfpathrectangle{\pgfqpoint{0.750000in}{0.500000in}}{\pgfqpoint{4.650000in}{3.020000in}}%
\pgfusepath{clip}%
\pgfsetbuttcap%
\pgfsetmiterjoin%
\definecolor{currentfill}{rgb}{0.000000,0.500000,0.000000}%
\pgfsetfillcolor{currentfill}%
\pgfsetlinewidth{0.000000pt}%
\definecolor{currentstroke}{rgb}{0.000000,0.000000,0.000000}%
\pgfsetstrokecolor{currentstroke}%
\pgfsetstrokeopacity{0.000000}%
\pgfsetdash{}{0pt}%
\pgfpathmoveto{\pgfqpoint{4.396023in}{0.500000in}}%
\pgfpathlineto{\pgfqpoint{4.429048in}{0.500000in}}%
\pgfpathlineto{\pgfqpoint{4.429048in}{0.739683in}}%
\pgfpathlineto{\pgfqpoint{4.396023in}{0.739683in}}%
\pgfpathlineto{\pgfqpoint{4.396023in}{0.500000in}}%
\pgfpathclose%
\pgfusepath{fill}%
\end{pgfscope}%
\begin{pgfscope}%
\pgfpathrectangle{\pgfqpoint{0.750000in}{0.500000in}}{\pgfqpoint{4.650000in}{3.020000in}}%
\pgfusepath{clip}%
\pgfsetbuttcap%
\pgfsetmiterjoin%
\definecolor{currentfill}{rgb}{0.000000,0.500000,0.000000}%
\pgfsetfillcolor{currentfill}%
\pgfsetlinewidth{0.000000pt}%
\definecolor{currentstroke}{rgb}{0.000000,0.000000,0.000000}%
\pgfsetstrokecolor{currentstroke}%
\pgfsetstrokeopacity{0.000000}%
\pgfsetdash{}{0pt}%
\pgfpathmoveto{\pgfqpoint{4.429048in}{0.500000in}}%
\pgfpathlineto{\pgfqpoint{4.462074in}{0.500000in}}%
\pgfpathlineto{\pgfqpoint{4.462074in}{0.659788in}}%
\pgfpathlineto{\pgfqpoint{4.429048in}{0.659788in}}%
\pgfpathlineto{\pgfqpoint{4.429048in}{0.500000in}}%
\pgfpathclose%
\pgfusepath{fill}%
\end{pgfscope}%
\begin{pgfscope}%
\pgfpathrectangle{\pgfqpoint{0.750000in}{0.500000in}}{\pgfqpoint{4.650000in}{3.020000in}}%
\pgfusepath{clip}%
\pgfsetbuttcap%
\pgfsetmiterjoin%
\definecolor{currentfill}{rgb}{0.000000,0.500000,0.000000}%
\pgfsetfillcolor{currentfill}%
\pgfsetlinewidth{0.000000pt}%
\definecolor{currentstroke}{rgb}{0.000000,0.000000,0.000000}%
\pgfsetstrokecolor{currentstroke}%
\pgfsetstrokeopacity{0.000000}%
\pgfsetdash{}{0pt}%
\pgfpathmoveto{\pgfqpoint{4.462074in}{0.500000in}}%
\pgfpathlineto{\pgfqpoint{4.495099in}{0.500000in}}%
\pgfpathlineto{\pgfqpoint{4.495099in}{0.787619in}}%
\pgfpathlineto{\pgfqpoint{4.462074in}{0.787619in}}%
\pgfpathlineto{\pgfqpoint{4.462074in}{0.500000in}}%
\pgfpathclose%
\pgfusepath{fill}%
\end{pgfscope}%
\begin{pgfscope}%
\pgfpathrectangle{\pgfqpoint{0.750000in}{0.500000in}}{\pgfqpoint{4.650000in}{3.020000in}}%
\pgfusepath{clip}%
\pgfsetbuttcap%
\pgfsetmiterjoin%
\definecolor{currentfill}{rgb}{0.000000,0.500000,0.000000}%
\pgfsetfillcolor{currentfill}%
\pgfsetlinewidth{0.000000pt}%
\definecolor{currentstroke}{rgb}{0.000000,0.000000,0.000000}%
\pgfsetstrokecolor{currentstroke}%
\pgfsetstrokeopacity{0.000000}%
\pgfsetdash{}{0pt}%
\pgfpathmoveto{\pgfqpoint{4.495099in}{0.500000in}}%
\pgfpathlineto{\pgfqpoint{4.528125in}{0.500000in}}%
\pgfpathlineto{\pgfqpoint{4.528125in}{0.659788in}}%
\pgfpathlineto{\pgfqpoint{4.495099in}{0.659788in}}%
\pgfpathlineto{\pgfqpoint{4.495099in}{0.500000in}}%
\pgfpathclose%
\pgfusepath{fill}%
\end{pgfscope}%
\begin{pgfscope}%
\pgfpathrectangle{\pgfqpoint{0.750000in}{0.500000in}}{\pgfqpoint{4.650000in}{3.020000in}}%
\pgfusepath{clip}%
\pgfsetbuttcap%
\pgfsetmiterjoin%
\definecolor{currentfill}{rgb}{0.000000,0.500000,0.000000}%
\pgfsetfillcolor{currentfill}%
\pgfsetlinewidth{0.000000pt}%
\definecolor{currentstroke}{rgb}{0.000000,0.000000,0.000000}%
\pgfsetstrokecolor{currentstroke}%
\pgfsetstrokeopacity{0.000000}%
\pgfsetdash{}{0pt}%
\pgfpathmoveto{\pgfqpoint{4.528125in}{0.500000in}}%
\pgfpathlineto{\pgfqpoint{4.561151in}{0.500000in}}%
\pgfpathlineto{\pgfqpoint{4.561151in}{0.643810in}}%
\pgfpathlineto{\pgfqpoint{4.528125in}{0.643810in}}%
\pgfpathlineto{\pgfqpoint{4.528125in}{0.500000in}}%
\pgfpathclose%
\pgfusepath{fill}%
\end{pgfscope}%
\begin{pgfscope}%
\pgfpathrectangle{\pgfqpoint{0.750000in}{0.500000in}}{\pgfqpoint{4.650000in}{3.020000in}}%
\pgfusepath{clip}%
\pgfsetbuttcap%
\pgfsetmiterjoin%
\definecolor{currentfill}{rgb}{0.000000,0.500000,0.000000}%
\pgfsetfillcolor{currentfill}%
\pgfsetlinewidth{0.000000pt}%
\definecolor{currentstroke}{rgb}{0.000000,0.000000,0.000000}%
\pgfsetstrokecolor{currentstroke}%
\pgfsetstrokeopacity{0.000000}%
\pgfsetdash{}{0pt}%
\pgfpathmoveto{\pgfqpoint{4.561151in}{0.500000in}}%
\pgfpathlineto{\pgfqpoint{4.594176in}{0.500000in}}%
\pgfpathlineto{\pgfqpoint{4.594176in}{0.547937in}}%
\pgfpathlineto{\pgfqpoint{4.561151in}{0.547937in}}%
\pgfpathlineto{\pgfqpoint{4.561151in}{0.500000in}}%
\pgfpathclose%
\pgfusepath{fill}%
\end{pgfscope}%
\begin{pgfscope}%
\pgfpathrectangle{\pgfqpoint{0.750000in}{0.500000in}}{\pgfqpoint{4.650000in}{3.020000in}}%
\pgfusepath{clip}%
\pgfsetbuttcap%
\pgfsetmiterjoin%
\definecolor{currentfill}{rgb}{0.000000,0.500000,0.000000}%
\pgfsetfillcolor{currentfill}%
\pgfsetlinewidth{0.000000pt}%
\definecolor{currentstroke}{rgb}{0.000000,0.000000,0.000000}%
\pgfsetstrokecolor{currentstroke}%
\pgfsetstrokeopacity{0.000000}%
\pgfsetdash{}{0pt}%
\pgfpathmoveto{\pgfqpoint{4.594176in}{0.500000in}}%
\pgfpathlineto{\pgfqpoint{4.627202in}{0.500000in}}%
\pgfpathlineto{\pgfqpoint{4.627202in}{0.611852in}}%
\pgfpathlineto{\pgfqpoint{4.594176in}{0.611852in}}%
\pgfpathlineto{\pgfqpoint{4.594176in}{0.500000in}}%
\pgfpathclose%
\pgfusepath{fill}%
\end{pgfscope}%
\begin{pgfscope}%
\pgfpathrectangle{\pgfqpoint{0.750000in}{0.500000in}}{\pgfqpoint{4.650000in}{3.020000in}}%
\pgfusepath{clip}%
\pgfsetbuttcap%
\pgfsetmiterjoin%
\definecolor{currentfill}{rgb}{0.000000,0.500000,0.000000}%
\pgfsetfillcolor{currentfill}%
\pgfsetlinewidth{0.000000pt}%
\definecolor{currentstroke}{rgb}{0.000000,0.000000,0.000000}%
\pgfsetstrokecolor{currentstroke}%
\pgfsetstrokeopacity{0.000000}%
\pgfsetdash{}{0pt}%
\pgfpathmoveto{\pgfqpoint{4.627202in}{0.500000in}}%
\pgfpathlineto{\pgfqpoint{4.660227in}{0.500000in}}%
\pgfpathlineto{\pgfqpoint{4.660227in}{0.547937in}}%
\pgfpathlineto{\pgfqpoint{4.627202in}{0.547937in}}%
\pgfpathlineto{\pgfqpoint{4.627202in}{0.500000in}}%
\pgfpathclose%
\pgfusepath{fill}%
\end{pgfscope}%
\begin{pgfscope}%
\pgfpathrectangle{\pgfqpoint{0.750000in}{0.500000in}}{\pgfqpoint{4.650000in}{3.020000in}}%
\pgfusepath{clip}%
\pgfsetbuttcap%
\pgfsetmiterjoin%
\definecolor{currentfill}{rgb}{0.000000,0.500000,0.000000}%
\pgfsetfillcolor{currentfill}%
\pgfsetlinewidth{0.000000pt}%
\definecolor{currentstroke}{rgb}{0.000000,0.000000,0.000000}%
\pgfsetstrokecolor{currentstroke}%
\pgfsetstrokeopacity{0.000000}%
\pgfsetdash{}{0pt}%
\pgfpathmoveto{\pgfqpoint{4.660227in}{0.500000in}}%
\pgfpathlineto{\pgfqpoint{4.693253in}{0.500000in}}%
\pgfpathlineto{\pgfqpoint{4.693253in}{0.579894in}}%
\pgfpathlineto{\pgfqpoint{4.660227in}{0.579894in}}%
\pgfpathlineto{\pgfqpoint{4.660227in}{0.500000in}}%
\pgfpathclose%
\pgfusepath{fill}%
\end{pgfscope}%
\begin{pgfscope}%
\pgfpathrectangle{\pgfqpoint{0.750000in}{0.500000in}}{\pgfqpoint{4.650000in}{3.020000in}}%
\pgfusepath{clip}%
\pgfsetbuttcap%
\pgfsetmiterjoin%
\definecolor{currentfill}{rgb}{0.000000,0.500000,0.000000}%
\pgfsetfillcolor{currentfill}%
\pgfsetlinewidth{0.000000pt}%
\definecolor{currentstroke}{rgb}{0.000000,0.000000,0.000000}%
\pgfsetstrokecolor{currentstroke}%
\pgfsetstrokeopacity{0.000000}%
\pgfsetdash{}{0pt}%
\pgfpathmoveto{\pgfqpoint{4.693253in}{0.500000in}}%
\pgfpathlineto{\pgfqpoint{4.726278in}{0.500000in}}%
\pgfpathlineto{\pgfqpoint{4.726278in}{0.547937in}}%
\pgfpathlineto{\pgfqpoint{4.693253in}{0.547937in}}%
\pgfpathlineto{\pgfqpoint{4.693253in}{0.500000in}}%
\pgfpathclose%
\pgfusepath{fill}%
\end{pgfscope}%
\begin{pgfscope}%
\pgfpathrectangle{\pgfqpoint{0.750000in}{0.500000in}}{\pgfqpoint{4.650000in}{3.020000in}}%
\pgfusepath{clip}%
\pgfsetbuttcap%
\pgfsetmiterjoin%
\definecolor{currentfill}{rgb}{0.000000,0.500000,0.000000}%
\pgfsetfillcolor{currentfill}%
\pgfsetlinewidth{0.000000pt}%
\definecolor{currentstroke}{rgb}{0.000000,0.000000,0.000000}%
\pgfsetstrokecolor{currentstroke}%
\pgfsetstrokeopacity{0.000000}%
\pgfsetdash{}{0pt}%
\pgfpathmoveto{\pgfqpoint{4.726278in}{0.500000in}}%
\pgfpathlineto{\pgfqpoint{4.759304in}{0.500000in}}%
\pgfpathlineto{\pgfqpoint{4.759304in}{0.531958in}}%
\pgfpathlineto{\pgfqpoint{4.726278in}{0.531958in}}%
\pgfpathlineto{\pgfqpoint{4.726278in}{0.500000in}}%
\pgfpathclose%
\pgfusepath{fill}%
\end{pgfscope}%
\begin{pgfscope}%
\pgfpathrectangle{\pgfqpoint{0.750000in}{0.500000in}}{\pgfqpoint{4.650000in}{3.020000in}}%
\pgfusepath{clip}%
\pgfsetbuttcap%
\pgfsetmiterjoin%
\definecolor{currentfill}{rgb}{0.000000,0.500000,0.000000}%
\pgfsetfillcolor{currentfill}%
\pgfsetlinewidth{0.000000pt}%
\definecolor{currentstroke}{rgb}{0.000000,0.000000,0.000000}%
\pgfsetstrokecolor{currentstroke}%
\pgfsetstrokeopacity{0.000000}%
\pgfsetdash{}{0pt}%
\pgfpathmoveto{\pgfqpoint{4.759304in}{0.500000in}}%
\pgfpathlineto{\pgfqpoint{4.792330in}{0.500000in}}%
\pgfpathlineto{\pgfqpoint{4.792330in}{0.611852in}}%
\pgfpathlineto{\pgfqpoint{4.759304in}{0.611852in}}%
\pgfpathlineto{\pgfqpoint{4.759304in}{0.500000in}}%
\pgfpathclose%
\pgfusepath{fill}%
\end{pgfscope}%
\begin{pgfscope}%
\pgfpathrectangle{\pgfqpoint{0.750000in}{0.500000in}}{\pgfqpoint{4.650000in}{3.020000in}}%
\pgfusepath{clip}%
\pgfsetbuttcap%
\pgfsetmiterjoin%
\definecolor{currentfill}{rgb}{0.000000,0.500000,0.000000}%
\pgfsetfillcolor{currentfill}%
\pgfsetlinewidth{0.000000pt}%
\definecolor{currentstroke}{rgb}{0.000000,0.000000,0.000000}%
\pgfsetstrokecolor{currentstroke}%
\pgfsetstrokeopacity{0.000000}%
\pgfsetdash{}{0pt}%
\pgfpathmoveto{\pgfqpoint{4.792330in}{0.500000in}}%
\pgfpathlineto{\pgfqpoint{4.825355in}{0.500000in}}%
\pgfpathlineto{\pgfqpoint{4.825355in}{0.547937in}}%
\pgfpathlineto{\pgfqpoint{4.792330in}{0.547937in}}%
\pgfpathlineto{\pgfqpoint{4.792330in}{0.500000in}}%
\pgfpathclose%
\pgfusepath{fill}%
\end{pgfscope}%
\begin{pgfscope}%
\pgfpathrectangle{\pgfqpoint{0.750000in}{0.500000in}}{\pgfqpoint{4.650000in}{3.020000in}}%
\pgfusepath{clip}%
\pgfsetbuttcap%
\pgfsetmiterjoin%
\definecolor{currentfill}{rgb}{0.000000,0.500000,0.000000}%
\pgfsetfillcolor{currentfill}%
\pgfsetlinewidth{0.000000pt}%
\definecolor{currentstroke}{rgb}{0.000000,0.000000,0.000000}%
\pgfsetstrokecolor{currentstroke}%
\pgfsetstrokeopacity{0.000000}%
\pgfsetdash{}{0pt}%
\pgfpathmoveto{\pgfqpoint{4.825355in}{0.500000in}}%
\pgfpathlineto{\pgfqpoint{4.858381in}{0.500000in}}%
\pgfpathlineto{\pgfqpoint{4.858381in}{0.563915in}}%
\pgfpathlineto{\pgfqpoint{4.825355in}{0.563915in}}%
\pgfpathlineto{\pgfqpoint{4.825355in}{0.500000in}}%
\pgfpathclose%
\pgfusepath{fill}%
\end{pgfscope}%
\begin{pgfscope}%
\pgfpathrectangle{\pgfqpoint{0.750000in}{0.500000in}}{\pgfqpoint{4.650000in}{3.020000in}}%
\pgfusepath{clip}%
\pgfsetbuttcap%
\pgfsetmiterjoin%
\definecolor{currentfill}{rgb}{0.000000,0.500000,0.000000}%
\pgfsetfillcolor{currentfill}%
\pgfsetlinewidth{0.000000pt}%
\definecolor{currentstroke}{rgb}{0.000000,0.000000,0.000000}%
\pgfsetstrokecolor{currentstroke}%
\pgfsetstrokeopacity{0.000000}%
\pgfsetdash{}{0pt}%
\pgfpathmoveto{\pgfqpoint{4.858381in}{0.500000in}}%
\pgfpathlineto{\pgfqpoint{4.891406in}{0.500000in}}%
\pgfpathlineto{\pgfqpoint{4.891406in}{0.547937in}}%
\pgfpathlineto{\pgfqpoint{4.858381in}{0.547937in}}%
\pgfpathlineto{\pgfqpoint{4.858381in}{0.500000in}}%
\pgfpathclose%
\pgfusepath{fill}%
\end{pgfscope}%
\begin{pgfscope}%
\pgfpathrectangle{\pgfqpoint{0.750000in}{0.500000in}}{\pgfqpoint{4.650000in}{3.020000in}}%
\pgfusepath{clip}%
\pgfsetbuttcap%
\pgfsetmiterjoin%
\definecolor{currentfill}{rgb}{0.000000,0.500000,0.000000}%
\pgfsetfillcolor{currentfill}%
\pgfsetlinewidth{0.000000pt}%
\definecolor{currentstroke}{rgb}{0.000000,0.000000,0.000000}%
\pgfsetstrokecolor{currentstroke}%
\pgfsetstrokeopacity{0.000000}%
\pgfsetdash{}{0pt}%
\pgfpathmoveto{\pgfqpoint{4.891406in}{0.500000in}}%
\pgfpathlineto{\pgfqpoint{4.924432in}{0.500000in}}%
\pgfpathlineto{\pgfqpoint{4.924432in}{0.500000in}}%
\pgfpathlineto{\pgfqpoint{4.891406in}{0.500000in}}%
\pgfpathlineto{\pgfqpoint{4.891406in}{0.500000in}}%
\pgfpathclose%
\pgfusepath{fill}%
\end{pgfscope}%
\begin{pgfscope}%
\pgfpathrectangle{\pgfqpoint{0.750000in}{0.500000in}}{\pgfqpoint{4.650000in}{3.020000in}}%
\pgfusepath{clip}%
\pgfsetbuttcap%
\pgfsetmiterjoin%
\definecolor{currentfill}{rgb}{0.000000,0.500000,0.000000}%
\pgfsetfillcolor{currentfill}%
\pgfsetlinewidth{0.000000pt}%
\definecolor{currentstroke}{rgb}{0.000000,0.000000,0.000000}%
\pgfsetstrokecolor{currentstroke}%
\pgfsetstrokeopacity{0.000000}%
\pgfsetdash{}{0pt}%
\pgfpathmoveto{\pgfqpoint{4.924432in}{0.500000in}}%
\pgfpathlineto{\pgfqpoint{4.957457in}{0.500000in}}%
\pgfpathlineto{\pgfqpoint{4.957457in}{0.515979in}}%
\pgfpathlineto{\pgfqpoint{4.924432in}{0.515979in}}%
\pgfpathlineto{\pgfqpoint{4.924432in}{0.500000in}}%
\pgfpathclose%
\pgfusepath{fill}%
\end{pgfscope}%
\begin{pgfscope}%
\pgfpathrectangle{\pgfqpoint{0.750000in}{0.500000in}}{\pgfqpoint{4.650000in}{3.020000in}}%
\pgfusepath{clip}%
\pgfsetbuttcap%
\pgfsetmiterjoin%
\definecolor{currentfill}{rgb}{0.000000,0.500000,0.000000}%
\pgfsetfillcolor{currentfill}%
\pgfsetlinewidth{0.000000pt}%
\definecolor{currentstroke}{rgb}{0.000000,0.000000,0.000000}%
\pgfsetstrokecolor{currentstroke}%
\pgfsetstrokeopacity{0.000000}%
\pgfsetdash{}{0pt}%
\pgfpathmoveto{\pgfqpoint{4.957457in}{0.500000in}}%
\pgfpathlineto{\pgfqpoint{4.990483in}{0.500000in}}%
\pgfpathlineto{\pgfqpoint{4.990483in}{0.500000in}}%
\pgfpathlineto{\pgfqpoint{4.957457in}{0.500000in}}%
\pgfpathlineto{\pgfqpoint{4.957457in}{0.500000in}}%
\pgfpathclose%
\pgfusepath{fill}%
\end{pgfscope}%
\begin{pgfscope}%
\pgfpathrectangle{\pgfqpoint{0.750000in}{0.500000in}}{\pgfqpoint{4.650000in}{3.020000in}}%
\pgfusepath{clip}%
\pgfsetbuttcap%
\pgfsetmiterjoin%
\definecolor{currentfill}{rgb}{0.000000,0.500000,0.000000}%
\pgfsetfillcolor{currentfill}%
\pgfsetlinewidth{0.000000pt}%
\definecolor{currentstroke}{rgb}{0.000000,0.000000,0.000000}%
\pgfsetstrokecolor{currentstroke}%
\pgfsetstrokeopacity{0.000000}%
\pgfsetdash{}{0pt}%
\pgfpathmoveto{\pgfqpoint{4.990483in}{0.500000in}}%
\pgfpathlineto{\pgfqpoint{5.023509in}{0.500000in}}%
\pgfpathlineto{\pgfqpoint{5.023509in}{0.531958in}}%
\pgfpathlineto{\pgfqpoint{4.990483in}{0.531958in}}%
\pgfpathlineto{\pgfqpoint{4.990483in}{0.500000in}}%
\pgfpathclose%
\pgfusepath{fill}%
\end{pgfscope}%
\begin{pgfscope}%
\pgfpathrectangle{\pgfqpoint{0.750000in}{0.500000in}}{\pgfqpoint{4.650000in}{3.020000in}}%
\pgfusepath{clip}%
\pgfsetbuttcap%
\pgfsetmiterjoin%
\definecolor{currentfill}{rgb}{0.000000,0.500000,0.000000}%
\pgfsetfillcolor{currentfill}%
\pgfsetlinewidth{0.000000pt}%
\definecolor{currentstroke}{rgb}{0.000000,0.000000,0.000000}%
\pgfsetstrokecolor{currentstroke}%
\pgfsetstrokeopacity{0.000000}%
\pgfsetdash{}{0pt}%
\pgfpathmoveto{\pgfqpoint{5.023509in}{0.500000in}}%
\pgfpathlineto{\pgfqpoint{5.056534in}{0.500000in}}%
\pgfpathlineto{\pgfqpoint{5.056534in}{0.515979in}}%
\pgfpathlineto{\pgfqpoint{5.023509in}{0.515979in}}%
\pgfpathlineto{\pgfqpoint{5.023509in}{0.500000in}}%
\pgfpathclose%
\pgfusepath{fill}%
\end{pgfscope}%
\begin{pgfscope}%
\pgfpathrectangle{\pgfqpoint{0.750000in}{0.500000in}}{\pgfqpoint{4.650000in}{3.020000in}}%
\pgfusepath{clip}%
\pgfsetbuttcap%
\pgfsetmiterjoin%
\definecolor{currentfill}{rgb}{0.000000,0.500000,0.000000}%
\pgfsetfillcolor{currentfill}%
\pgfsetlinewidth{0.000000pt}%
\definecolor{currentstroke}{rgb}{0.000000,0.000000,0.000000}%
\pgfsetstrokecolor{currentstroke}%
\pgfsetstrokeopacity{0.000000}%
\pgfsetdash{}{0pt}%
\pgfpathmoveto{\pgfqpoint{5.056534in}{0.500000in}}%
\pgfpathlineto{\pgfqpoint{5.089560in}{0.500000in}}%
\pgfpathlineto{\pgfqpoint{5.089560in}{0.500000in}}%
\pgfpathlineto{\pgfqpoint{5.056534in}{0.500000in}}%
\pgfpathlineto{\pgfqpoint{5.056534in}{0.500000in}}%
\pgfpathclose%
\pgfusepath{fill}%
\end{pgfscope}%
\begin{pgfscope}%
\pgfpathrectangle{\pgfqpoint{0.750000in}{0.500000in}}{\pgfqpoint{4.650000in}{3.020000in}}%
\pgfusepath{clip}%
\pgfsetbuttcap%
\pgfsetmiterjoin%
\definecolor{currentfill}{rgb}{0.000000,0.500000,0.000000}%
\pgfsetfillcolor{currentfill}%
\pgfsetlinewidth{0.000000pt}%
\definecolor{currentstroke}{rgb}{0.000000,0.000000,0.000000}%
\pgfsetstrokecolor{currentstroke}%
\pgfsetstrokeopacity{0.000000}%
\pgfsetdash{}{0pt}%
\pgfpathmoveto{\pgfqpoint{5.089560in}{0.500000in}}%
\pgfpathlineto{\pgfqpoint{5.122585in}{0.500000in}}%
\pgfpathlineto{\pgfqpoint{5.122585in}{0.500000in}}%
\pgfpathlineto{\pgfqpoint{5.089560in}{0.500000in}}%
\pgfpathlineto{\pgfqpoint{5.089560in}{0.500000in}}%
\pgfpathclose%
\pgfusepath{fill}%
\end{pgfscope}%
\begin{pgfscope}%
\pgfpathrectangle{\pgfqpoint{0.750000in}{0.500000in}}{\pgfqpoint{4.650000in}{3.020000in}}%
\pgfusepath{clip}%
\pgfsetbuttcap%
\pgfsetmiterjoin%
\definecolor{currentfill}{rgb}{0.000000,0.500000,0.000000}%
\pgfsetfillcolor{currentfill}%
\pgfsetlinewidth{0.000000pt}%
\definecolor{currentstroke}{rgb}{0.000000,0.000000,0.000000}%
\pgfsetstrokecolor{currentstroke}%
\pgfsetstrokeopacity{0.000000}%
\pgfsetdash{}{0pt}%
\pgfpathmoveto{\pgfqpoint{5.122585in}{0.500000in}}%
\pgfpathlineto{\pgfqpoint{5.155611in}{0.500000in}}%
\pgfpathlineto{\pgfqpoint{5.155611in}{0.515979in}}%
\pgfpathlineto{\pgfqpoint{5.122585in}{0.515979in}}%
\pgfpathlineto{\pgfqpoint{5.122585in}{0.500000in}}%
\pgfpathclose%
\pgfusepath{fill}%
\end{pgfscope}%
\begin{pgfscope}%
\pgfpathrectangle{\pgfqpoint{0.750000in}{0.500000in}}{\pgfqpoint{4.650000in}{3.020000in}}%
\pgfusepath{clip}%
\pgfsetbuttcap%
\pgfsetmiterjoin%
\definecolor{currentfill}{rgb}{0.000000,0.500000,0.000000}%
\pgfsetfillcolor{currentfill}%
\pgfsetlinewidth{0.000000pt}%
\definecolor{currentstroke}{rgb}{0.000000,0.000000,0.000000}%
\pgfsetstrokecolor{currentstroke}%
\pgfsetstrokeopacity{0.000000}%
\pgfsetdash{}{0pt}%
\pgfpathmoveto{\pgfqpoint{5.155611in}{0.500000in}}%
\pgfpathlineto{\pgfqpoint{5.188636in}{0.500000in}}%
\pgfpathlineto{\pgfqpoint{5.188636in}{0.515979in}}%
\pgfpathlineto{\pgfqpoint{5.155611in}{0.515979in}}%
\pgfpathlineto{\pgfqpoint{5.155611in}{0.500000in}}%
\pgfpathclose%
\pgfusepath{fill}%
\end{pgfscope}%
\begin{pgfscope}%
\pgfsetbuttcap%
\pgfsetroundjoin%
\definecolor{currentfill}{rgb}{0.000000,0.000000,0.000000}%
\pgfsetfillcolor{currentfill}%
\pgfsetlinewidth{0.803000pt}%
\definecolor{currentstroke}{rgb}{0.000000,0.000000,0.000000}%
\pgfsetstrokecolor{currentstroke}%
\pgfsetdash{}{0pt}%
\pgfsys@defobject{currentmarker}{\pgfqpoint{0.000000in}{-0.048611in}}{\pgfqpoint{0.000000in}{0.000000in}}{%
\pgfpathmoveto{\pgfqpoint{0.000000in}{0.000000in}}%
\pgfpathlineto{\pgfqpoint{0.000000in}{-0.048611in}}%
\pgfusepath{stroke,fill}%
}%
\begin{pgfscope}%
\pgfsys@transformshift{0.782488in}{0.500000in}%
\pgfsys@useobject{currentmarker}{}%
\end{pgfscope}%
\end{pgfscope}%
\begin{pgfscope}%
\definecolor{textcolor}{rgb}{0.000000,0.000000,0.000000}%
\pgfsetstrokecolor{textcolor}%
\pgfsetfillcolor{textcolor}%
\pgftext[x=0.782488in,y=0.402778in,,top]{\color{textcolor}\sffamily\fontsize{10.000000}{12.000000}\selectfont 0.30}%
\end{pgfscope}%
\begin{pgfscope}%
\pgfsetbuttcap%
\pgfsetroundjoin%
\definecolor{currentfill}{rgb}{0.000000,0.000000,0.000000}%
\pgfsetfillcolor{currentfill}%
\pgfsetlinewidth{0.803000pt}%
\definecolor{currentstroke}{rgb}{0.000000,0.000000,0.000000}%
\pgfsetstrokecolor{currentstroke}%
\pgfsetdash{}{0pt}%
\pgfsys@defobject{currentmarker}{\pgfqpoint{0.000000in}{-0.048611in}}{\pgfqpoint{0.000000in}{0.000000in}}{%
\pgfpathmoveto{\pgfqpoint{0.000000in}{0.000000in}}%
\pgfpathlineto{\pgfqpoint{0.000000in}{-0.048611in}}%
\pgfusepath{stroke,fill}%
}%
\begin{pgfscope}%
\pgfsys@transformshift{1.426674in}{0.500000in}%
\pgfsys@useobject{currentmarker}{}%
\end{pgfscope}%
\end{pgfscope}%
\begin{pgfscope}%
\definecolor{textcolor}{rgb}{0.000000,0.000000,0.000000}%
\pgfsetstrokecolor{textcolor}%
\pgfsetfillcolor{textcolor}%
\pgftext[x=1.426674in,y=0.402778in,,top]{\color{textcolor}\sffamily\fontsize{10.000000}{12.000000}\selectfont 0.35}%
\end{pgfscope}%
\begin{pgfscope}%
\pgfsetbuttcap%
\pgfsetroundjoin%
\definecolor{currentfill}{rgb}{0.000000,0.000000,0.000000}%
\pgfsetfillcolor{currentfill}%
\pgfsetlinewidth{0.803000pt}%
\definecolor{currentstroke}{rgb}{0.000000,0.000000,0.000000}%
\pgfsetstrokecolor{currentstroke}%
\pgfsetdash{}{0pt}%
\pgfsys@defobject{currentmarker}{\pgfqpoint{0.000000in}{-0.048611in}}{\pgfqpoint{0.000000in}{0.000000in}}{%
\pgfpathmoveto{\pgfqpoint{0.000000in}{0.000000in}}%
\pgfpathlineto{\pgfqpoint{0.000000in}{-0.048611in}}%
\pgfusepath{stroke,fill}%
}%
\begin{pgfscope}%
\pgfsys@transformshift{2.070859in}{0.500000in}%
\pgfsys@useobject{currentmarker}{}%
\end{pgfscope}%
\end{pgfscope}%
\begin{pgfscope}%
\definecolor{textcolor}{rgb}{0.000000,0.000000,0.000000}%
\pgfsetstrokecolor{textcolor}%
\pgfsetfillcolor{textcolor}%
\pgftext[x=2.070859in,y=0.402778in,,top]{\color{textcolor}\sffamily\fontsize{10.000000}{12.000000}\selectfont 0.40}%
\end{pgfscope}%
\begin{pgfscope}%
\pgfsetbuttcap%
\pgfsetroundjoin%
\definecolor{currentfill}{rgb}{0.000000,0.000000,0.000000}%
\pgfsetfillcolor{currentfill}%
\pgfsetlinewidth{0.803000pt}%
\definecolor{currentstroke}{rgb}{0.000000,0.000000,0.000000}%
\pgfsetstrokecolor{currentstroke}%
\pgfsetdash{}{0pt}%
\pgfsys@defobject{currentmarker}{\pgfqpoint{0.000000in}{-0.048611in}}{\pgfqpoint{0.000000in}{0.000000in}}{%
\pgfpathmoveto{\pgfqpoint{0.000000in}{0.000000in}}%
\pgfpathlineto{\pgfqpoint{0.000000in}{-0.048611in}}%
\pgfusepath{stroke,fill}%
}%
\begin{pgfscope}%
\pgfsys@transformshift{2.715044in}{0.500000in}%
\pgfsys@useobject{currentmarker}{}%
\end{pgfscope}%
\end{pgfscope}%
\begin{pgfscope}%
\definecolor{textcolor}{rgb}{0.000000,0.000000,0.000000}%
\pgfsetstrokecolor{textcolor}%
\pgfsetfillcolor{textcolor}%
\pgftext[x=2.715044in,y=0.402778in,,top]{\color{textcolor}\sffamily\fontsize{10.000000}{12.000000}\selectfont 0.45}%
\end{pgfscope}%
\begin{pgfscope}%
\pgfsetbuttcap%
\pgfsetroundjoin%
\definecolor{currentfill}{rgb}{0.000000,0.000000,0.000000}%
\pgfsetfillcolor{currentfill}%
\pgfsetlinewidth{0.803000pt}%
\definecolor{currentstroke}{rgb}{0.000000,0.000000,0.000000}%
\pgfsetstrokecolor{currentstroke}%
\pgfsetdash{}{0pt}%
\pgfsys@defobject{currentmarker}{\pgfqpoint{0.000000in}{-0.048611in}}{\pgfqpoint{0.000000in}{0.000000in}}{%
\pgfpathmoveto{\pgfqpoint{0.000000in}{0.000000in}}%
\pgfpathlineto{\pgfqpoint{0.000000in}{-0.048611in}}%
\pgfusepath{stroke,fill}%
}%
\begin{pgfscope}%
\pgfsys@transformshift{3.359230in}{0.500000in}%
\pgfsys@useobject{currentmarker}{}%
\end{pgfscope}%
\end{pgfscope}%
\begin{pgfscope}%
\definecolor{textcolor}{rgb}{0.000000,0.000000,0.000000}%
\pgfsetstrokecolor{textcolor}%
\pgfsetfillcolor{textcolor}%
\pgftext[x=3.359230in,y=0.402778in,,top]{\color{textcolor}\sffamily\fontsize{10.000000}{12.000000}\selectfont 0.50}%
\end{pgfscope}%
\begin{pgfscope}%
\pgfsetbuttcap%
\pgfsetroundjoin%
\definecolor{currentfill}{rgb}{0.000000,0.000000,0.000000}%
\pgfsetfillcolor{currentfill}%
\pgfsetlinewidth{0.803000pt}%
\definecolor{currentstroke}{rgb}{0.000000,0.000000,0.000000}%
\pgfsetstrokecolor{currentstroke}%
\pgfsetdash{}{0pt}%
\pgfsys@defobject{currentmarker}{\pgfqpoint{0.000000in}{-0.048611in}}{\pgfqpoint{0.000000in}{0.000000in}}{%
\pgfpathmoveto{\pgfqpoint{0.000000in}{0.000000in}}%
\pgfpathlineto{\pgfqpoint{0.000000in}{-0.048611in}}%
\pgfusepath{stroke,fill}%
}%
\begin{pgfscope}%
\pgfsys@transformshift{4.003415in}{0.500000in}%
\pgfsys@useobject{currentmarker}{}%
\end{pgfscope}%
\end{pgfscope}%
\begin{pgfscope}%
\definecolor{textcolor}{rgb}{0.000000,0.000000,0.000000}%
\pgfsetstrokecolor{textcolor}%
\pgfsetfillcolor{textcolor}%
\pgftext[x=4.003415in,y=0.402778in,,top]{\color{textcolor}\sffamily\fontsize{10.000000}{12.000000}\selectfont 0.55}%
\end{pgfscope}%
\begin{pgfscope}%
\pgfsetbuttcap%
\pgfsetroundjoin%
\definecolor{currentfill}{rgb}{0.000000,0.000000,0.000000}%
\pgfsetfillcolor{currentfill}%
\pgfsetlinewidth{0.803000pt}%
\definecolor{currentstroke}{rgb}{0.000000,0.000000,0.000000}%
\pgfsetstrokecolor{currentstroke}%
\pgfsetdash{}{0pt}%
\pgfsys@defobject{currentmarker}{\pgfqpoint{0.000000in}{-0.048611in}}{\pgfqpoint{0.000000in}{0.000000in}}{%
\pgfpathmoveto{\pgfqpoint{0.000000in}{0.000000in}}%
\pgfpathlineto{\pgfqpoint{0.000000in}{-0.048611in}}%
\pgfusepath{stroke,fill}%
}%
\begin{pgfscope}%
\pgfsys@transformshift{4.647600in}{0.500000in}%
\pgfsys@useobject{currentmarker}{}%
\end{pgfscope}%
\end{pgfscope}%
\begin{pgfscope}%
\definecolor{textcolor}{rgb}{0.000000,0.000000,0.000000}%
\pgfsetstrokecolor{textcolor}%
\pgfsetfillcolor{textcolor}%
\pgftext[x=4.647600in,y=0.402778in,,top]{\color{textcolor}\sffamily\fontsize{10.000000}{12.000000}\selectfont 0.60}%
\end{pgfscope}%
\begin{pgfscope}%
\pgfsetbuttcap%
\pgfsetroundjoin%
\definecolor{currentfill}{rgb}{0.000000,0.000000,0.000000}%
\pgfsetfillcolor{currentfill}%
\pgfsetlinewidth{0.803000pt}%
\definecolor{currentstroke}{rgb}{0.000000,0.000000,0.000000}%
\pgfsetstrokecolor{currentstroke}%
\pgfsetdash{}{0pt}%
\pgfsys@defobject{currentmarker}{\pgfqpoint{0.000000in}{-0.048611in}}{\pgfqpoint{0.000000in}{0.000000in}}{%
\pgfpathmoveto{\pgfqpoint{0.000000in}{0.000000in}}%
\pgfpathlineto{\pgfqpoint{0.000000in}{-0.048611in}}%
\pgfusepath{stroke,fill}%
}%
\begin{pgfscope}%
\pgfsys@transformshift{5.291786in}{0.500000in}%
\pgfsys@useobject{currentmarker}{}%
\end{pgfscope}%
\end{pgfscope}%
\begin{pgfscope}%
\definecolor{textcolor}{rgb}{0.000000,0.000000,0.000000}%
\pgfsetstrokecolor{textcolor}%
\pgfsetfillcolor{textcolor}%
\pgftext[x=5.291786in,y=0.402778in,,top]{\color{textcolor}\sffamily\fontsize{10.000000}{12.000000}\selectfont 0.65}%
\end{pgfscope}%
\begin{pgfscope}%
\definecolor{textcolor}{rgb}{0.000000,0.000000,0.000000}%
\pgfsetstrokecolor{textcolor}%
\pgfsetfillcolor{textcolor}%
\pgftext[x=3.075000in,y=0.212809in,,top]{\color{textcolor}\sffamily\fontsize{10.000000}{12.000000}\selectfont loss}%
\end{pgfscope}%
\begin{pgfscope}%
\pgfsetbuttcap%
\pgfsetroundjoin%
\definecolor{currentfill}{rgb}{0.000000,0.000000,0.000000}%
\pgfsetfillcolor{currentfill}%
\pgfsetlinewidth{0.803000pt}%
\definecolor{currentstroke}{rgb}{0.000000,0.000000,0.000000}%
\pgfsetstrokecolor{currentstroke}%
\pgfsetdash{}{0pt}%
\pgfsys@defobject{currentmarker}{\pgfqpoint{-0.048611in}{0.000000in}}{\pgfqpoint{-0.000000in}{0.000000in}}{%
\pgfpathmoveto{\pgfqpoint{-0.000000in}{0.000000in}}%
\pgfpathlineto{\pgfqpoint{-0.048611in}{0.000000in}}%
\pgfusepath{stroke,fill}%
}%
\begin{pgfscope}%
\pgfsys@transformshift{0.750000in}{0.500000in}%
\pgfsys@useobject{currentmarker}{}%
\end{pgfscope}%
\end{pgfscope}%
\begin{pgfscope}%
\definecolor{textcolor}{rgb}{0.000000,0.000000,0.000000}%
\pgfsetstrokecolor{textcolor}%
\pgfsetfillcolor{textcolor}%
\pgftext[x=0.564412in, y=0.447238in, left, base]{\color{textcolor}\sffamily\fontsize{10.000000}{12.000000}\selectfont 0}%
\end{pgfscope}%
\begin{pgfscope}%
\pgfsetbuttcap%
\pgfsetroundjoin%
\definecolor{currentfill}{rgb}{0.000000,0.000000,0.000000}%
\pgfsetfillcolor{currentfill}%
\pgfsetlinewidth{0.803000pt}%
\definecolor{currentstroke}{rgb}{0.000000,0.000000,0.000000}%
\pgfsetstrokecolor{currentstroke}%
\pgfsetdash{}{0pt}%
\pgfsys@defobject{currentmarker}{\pgfqpoint{-0.048611in}{0.000000in}}{\pgfqpoint{-0.000000in}{0.000000in}}{%
\pgfpathmoveto{\pgfqpoint{-0.000000in}{0.000000in}}%
\pgfpathlineto{\pgfqpoint{-0.048611in}{0.000000in}}%
\pgfusepath{stroke,fill}%
}%
\begin{pgfscope}%
\pgfsys@transformshift{0.750000in}{0.899471in}%
\pgfsys@useobject{currentmarker}{}%
\end{pgfscope}%
\end{pgfscope}%
\begin{pgfscope}%
\definecolor{textcolor}{rgb}{0.000000,0.000000,0.000000}%
\pgfsetstrokecolor{textcolor}%
\pgfsetfillcolor{textcolor}%
\pgftext[x=0.476047in, y=0.846709in, left, base]{\color{textcolor}\sffamily\fontsize{10.000000}{12.000000}\selectfont 25}%
\end{pgfscope}%
\begin{pgfscope}%
\pgfsetbuttcap%
\pgfsetroundjoin%
\definecolor{currentfill}{rgb}{0.000000,0.000000,0.000000}%
\pgfsetfillcolor{currentfill}%
\pgfsetlinewidth{0.803000pt}%
\definecolor{currentstroke}{rgb}{0.000000,0.000000,0.000000}%
\pgfsetstrokecolor{currentstroke}%
\pgfsetdash{}{0pt}%
\pgfsys@defobject{currentmarker}{\pgfqpoint{-0.048611in}{0.000000in}}{\pgfqpoint{-0.000000in}{0.000000in}}{%
\pgfpathmoveto{\pgfqpoint{-0.000000in}{0.000000in}}%
\pgfpathlineto{\pgfqpoint{-0.048611in}{0.000000in}}%
\pgfusepath{stroke,fill}%
}%
\begin{pgfscope}%
\pgfsys@transformshift{0.750000in}{1.298942in}%
\pgfsys@useobject{currentmarker}{}%
\end{pgfscope}%
\end{pgfscope}%
\begin{pgfscope}%
\definecolor{textcolor}{rgb}{0.000000,0.000000,0.000000}%
\pgfsetstrokecolor{textcolor}%
\pgfsetfillcolor{textcolor}%
\pgftext[x=0.476047in, y=1.246180in, left, base]{\color{textcolor}\sffamily\fontsize{10.000000}{12.000000}\selectfont 50}%
\end{pgfscope}%
\begin{pgfscope}%
\pgfsetbuttcap%
\pgfsetroundjoin%
\definecolor{currentfill}{rgb}{0.000000,0.000000,0.000000}%
\pgfsetfillcolor{currentfill}%
\pgfsetlinewidth{0.803000pt}%
\definecolor{currentstroke}{rgb}{0.000000,0.000000,0.000000}%
\pgfsetstrokecolor{currentstroke}%
\pgfsetdash{}{0pt}%
\pgfsys@defobject{currentmarker}{\pgfqpoint{-0.048611in}{0.000000in}}{\pgfqpoint{-0.000000in}{0.000000in}}{%
\pgfpathmoveto{\pgfqpoint{-0.000000in}{0.000000in}}%
\pgfpathlineto{\pgfqpoint{-0.048611in}{0.000000in}}%
\pgfusepath{stroke,fill}%
}%
\begin{pgfscope}%
\pgfsys@transformshift{0.750000in}{1.698413in}%
\pgfsys@useobject{currentmarker}{}%
\end{pgfscope}%
\end{pgfscope}%
\begin{pgfscope}%
\definecolor{textcolor}{rgb}{0.000000,0.000000,0.000000}%
\pgfsetstrokecolor{textcolor}%
\pgfsetfillcolor{textcolor}%
\pgftext[x=0.476047in, y=1.645651in, left, base]{\color{textcolor}\sffamily\fontsize{10.000000}{12.000000}\selectfont 75}%
\end{pgfscope}%
\begin{pgfscope}%
\pgfsetbuttcap%
\pgfsetroundjoin%
\definecolor{currentfill}{rgb}{0.000000,0.000000,0.000000}%
\pgfsetfillcolor{currentfill}%
\pgfsetlinewidth{0.803000pt}%
\definecolor{currentstroke}{rgb}{0.000000,0.000000,0.000000}%
\pgfsetstrokecolor{currentstroke}%
\pgfsetdash{}{0pt}%
\pgfsys@defobject{currentmarker}{\pgfqpoint{-0.048611in}{0.000000in}}{\pgfqpoint{-0.000000in}{0.000000in}}{%
\pgfpathmoveto{\pgfqpoint{-0.000000in}{0.000000in}}%
\pgfpathlineto{\pgfqpoint{-0.048611in}{0.000000in}}%
\pgfusepath{stroke,fill}%
}%
\begin{pgfscope}%
\pgfsys@transformshift{0.750000in}{2.097884in}%
\pgfsys@useobject{currentmarker}{}%
\end{pgfscope}%
\end{pgfscope}%
\begin{pgfscope}%
\definecolor{textcolor}{rgb}{0.000000,0.000000,0.000000}%
\pgfsetstrokecolor{textcolor}%
\pgfsetfillcolor{textcolor}%
\pgftext[x=0.387682in, y=2.045122in, left, base]{\color{textcolor}\sffamily\fontsize{10.000000}{12.000000}\selectfont 100}%
\end{pgfscope}%
\begin{pgfscope}%
\pgfsetbuttcap%
\pgfsetroundjoin%
\definecolor{currentfill}{rgb}{0.000000,0.000000,0.000000}%
\pgfsetfillcolor{currentfill}%
\pgfsetlinewidth{0.803000pt}%
\definecolor{currentstroke}{rgb}{0.000000,0.000000,0.000000}%
\pgfsetstrokecolor{currentstroke}%
\pgfsetdash{}{0pt}%
\pgfsys@defobject{currentmarker}{\pgfqpoint{-0.048611in}{0.000000in}}{\pgfqpoint{-0.000000in}{0.000000in}}{%
\pgfpathmoveto{\pgfqpoint{-0.000000in}{0.000000in}}%
\pgfpathlineto{\pgfqpoint{-0.048611in}{0.000000in}}%
\pgfusepath{stroke,fill}%
}%
\begin{pgfscope}%
\pgfsys@transformshift{0.750000in}{2.497354in}%
\pgfsys@useobject{currentmarker}{}%
\end{pgfscope}%
\end{pgfscope}%
\begin{pgfscope}%
\definecolor{textcolor}{rgb}{0.000000,0.000000,0.000000}%
\pgfsetstrokecolor{textcolor}%
\pgfsetfillcolor{textcolor}%
\pgftext[x=0.387682in, y=2.444593in, left, base]{\color{textcolor}\sffamily\fontsize{10.000000}{12.000000}\selectfont 125}%
\end{pgfscope}%
\begin{pgfscope}%
\pgfsetbuttcap%
\pgfsetroundjoin%
\definecolor{currentfill}{rgb}{0.000000,0.000000,0.000000}%
\pgfsetfillcolor{currentfill}%
\pgfsetlinewidth{0.803000pt}%
\definecolor{currentstroke}{rgb}{0.000000,0.000000,0.000000}%
\pgfsetstrokecolor{currentstroke}%
\pgfsetdash{}{0pt}%
\pgfsys@defobject{currentmarker}{\pgfqpoint{-0.048611in}{0.000000in}}{\pgfqpoint{-0.000000in}{0.000000in}}{%
\pgfpathmoveto{\pgfqpoint{-0.000000in}{0.000000in}}%
\pgfpathlineto{\pgfqpoint{-0.048611in}{0.000000in}}%
\pgfusepath{stroke,fill}%
}%
\begin{pgfscope}%
\pgfsys@transformshift{0.750000in}{2.896825in}%
\pgfsys@useobject{currentmarker}{}%
\end{pgfscope}%
\end{pgfscope}%
\begin{pgfscope}%
\definecolor{textcolor}{rgb}{0.000000,0.000000,0.000000}%
\pgfsetstrokecolor{textcolor}%
\pgfsetfillcolor{textcolor}%
\pgftext[x=0.387682in, y=2.844064in, left, base]{\color{textcolor}\sffamily\fontsize{10.000000}{12.000000}\selectfont 150}%
\end{pgfscope}%
\begin{pgfscope}%
\pgfsetbuttcap%
\pgfsetroundjoin%
\definecolor{currentfill}{rgb}{0.000000,0.000000,0.000000}%
\pgfsetfillcolor{currentfill}%
\pgfsetlinewidth{0.803000pt}%
\definecolor{currentstroke}{rgb}{0.000000,0.000000,0.000000}%
\pgfsetstrokecolor{currentstroke}%
\pgfsetdash{}{0pt}%
\pgfsys@defobject{currentmarker}{\pgfqpoint{-0.048611in}{0.000000in}}{\pgfqpoint{-0.000000in}{0.000000in}}{%
\pgfpathmoveto{\pgfqpoint{-0.000000in}{0.000000in}}%
\pgfpathlineto{\pgfqpoint{-0.048611in}{0.000000in}}%
\pgfusepath{stroke,fill}%
}%
\begin{pgfscope}%
\pgfsys@transformshift{0.750000in}{3.296296in}%
\pgfsys@useobject{currentmarker}{}%
\end{pgfscope}%
\end{pgfscope}%
\begin{pgfscope}%
\definecolor{textcolor}{rgb}{0.000000,0.000000,0.000000}%
\pgfsetstrokecolor{textcolor}%
\pgfsetfillcolor{textcolor}%
\pgftext[x=0.387682in, y=3.243535in, left, base]{\color{textcolor}\sffamily\fontsize{10.000000}{12.000000}\selectfont 175}%
\end{pgfscope}%
\begin{pgfscope}%
\definecolor{textcolor}{rgb}{0.000000,0.000000,0.000000}%
\pgfsetstrokecolor{textcolor}%
\pgfsetfillcolor{textcolor}%
\pgftext[x=0.332126in,y=2.010000in,,bottom,rotate=90.000000]{\color{textcolor}\sffamily\fontsize{10.000000}{12.000000}\selectfont count}%
\end{pgfscope}%
\begin{pgfscope}%
\pgfsetrectcap%
\pgfsetmiterjoin%
\pgfsetlinewidth{0.803000pt}%
\definecolor{currentstroke}{rgb}{0.000000,0.000000,0.000000}%
\pgfsetstrokecolor{currentstroke}%
\pgfsetdash{}{0pt}%
\pgfpathmoveto{\pgfqpoint{0.750000in}{0.500000in}}%
\pgfpathlineto{\pgfqpoint{0.750000in}{3.520000in}}%
\pgfusepath{stroke}%
\end{pgfscope}%
\begin{pgfscope}%
\pgfsetrectcap%
\pgfsetmiterjoin%
\pgfsetlinewidth{0.803000pt}%
\definecolor{currentstroke}{rgb}{0.000000,0.000000,0.000000}%
\pgfsetstrokecolor{currentstroke}%
\pgfsetdash{}{0pt}%
\pgfpathmoveto{\pgfqpoint{5.400000in}{0.500000in}}%
\pgfpathlineto{\pgfqpoint{5.400000in}{3.520000in}}%
\pgfusepath{stroke}%
\end{pgfscope}%
\begin{pgfscope}%
\pgfsetrectcap%
\pgfsetmiterjoin%
\pgfsetlinewidth{0.803000pt}%
\definecolor{currentstroke}{rgb}{0.000000,0.000000,0.000000}%
\pgfsetstrokecolor{currentstroke}%
\pgfsetdash{}{0pt}%
\pgfpathmoveto{\pgfqpoint{0.750000in}{0.500000in}}%
\pgfpathlineto{\pgfqpoint{5.400000in}{0.500000in}}%
\pgfusepath{stroke}%
\end{pgfscope}%
\begin{pgfscope}%
\pgfsetrectcap%
\pgfsetmiterjoin%
\pgfsetlinewidth{0.803000pt}%
\definecolor{currentstroke}{rgb}{0.000000,0.000000,0.000000}%
\pgfsetstrokecolor{currentstroke}%
\pgfsetdash{}{0pt}%
\pgfpathmoveto{\pgfqpoint{0.750000in}{3.520000in}}%
\pgfpathlineto{\pgfqpoint{5.400000in}{3.520000in}}%
\pgfusepath{stroke}%
\end{pgfscope}%
\begin{pgfscope}%
\definecolor{textcolor}{rgb}{0.000000,0.000000,0.000000}%
\pgfsetstrokecolor{textcolor}%
\pgfsetfillcolor{textcolor}%
\pgftext[x=3.075000in,y=3.603333in,,base]{\color{textcolor}\sffamily\fontsize{12.000000}{14.400000}\selectfont loss over time for Rijndael's MixColumns function}%
\end{pgfscope}%
\begin{pgfscope}%
\pgfsetbuttcap%
\pgfsetmiterjoin%
\definecolor{currentfill}{rgb}{1.000000,1.000000,1.000000}%
\pgfsetfillcolor{currentfill}%
\pgfsetfillopacity{0.800000}%
\pgfsetlinewidth{1.003750pt}%
\definecolor{currentstroke}{rgb}{0.800000,0.800000,0.800000}%
\pgfsetstrokecolor{currentstroke}%
\pgfsetstrokeopacity{0.800000}%
\pgfsetdash{}{0pt}%
\pgfpathmoveto{\pgfqpoint{4.650543in}{3.205032in}}%
\pgfpathlineto{\pgfqpoint{5.302778in}{3.205032in}}%
\pgfpathquadraticcurveto{\pgfqpoint{5.330556in}{3.205032in}}{\pgfqpoint{5.330556in}{3.232809in}}%
\pgfpathlineto{\pgfqpoint{5.330556in}{3.422778in}}%
\pgfpathquadraticcurveto{\pgfqpoint{5.330556in}{3.450556in}}{\pgfqpoint{5.302778in}{3.450556in}}%
\pgfpathlineto{\pgfqpoint{4.650543in}{3.450556in}}%
\pgfpathquadraticcurveto{\pgfqpoint{4.622765in}{3.450556in}}{\pgfqpoint{4.622765in}{3.422778in}}%
\pgfpathlineto{\pgfqpoint{4.622765in}{3.232809in}}%
\pgfpathquadraticcurveto{\pgfqpoint{4.622765in}{3.205032in}}{\pgfqpoint{4.650543in}{3.205032in}}%
\pgfpathlineto{\pgfqpoint{4.650543in}{3.205032in}}%
\pgfpathclose%
\pgfusepath{stroke,fill}%
\end{pgfscope}%
\begin{pgfscope}%
\pgfsetbuttcap%
\pgfsetmiterjoin%
\definecolor{currentfill}{rgb}{0.000000,0.500000,0.000000}%
\pgfsetfillcolor{currentfill}%
\pgfsetlinewidth{0.000000pt}%
\definecolor{currentstroke}{rgb}{0.000000,0.000000,0.000000}%
\pgfsetstrokecolor{currentstroke}%
\pgfsetstrokeopacity{0.000000}%
\pgfsetdash{}{0pt}%
\pgfpathmoveto{\pgfqpoint{4.678320in}{3.289477in}}%
\pgfpathlineto{\pgfqpoint{4.956098in}{3.289477in}}%
\pgfpathlineto{\pgfqpoint{4.956098in}{3.386699in}}%
\pgfpathlineto{\pgfqpoint{4.678320in}{3.386699in}}%
\pgfpathlineto{\pgfqpoint{4.678320in}{3.289477in}}%
\pgfpathclose%
\pgfusepath{fill}%
\end{pgfscope}%
\begin{pgfscope}%
\definecolor{textcolor}{rgb}{0.000000,0.000000,0.000000}%
\pgfsetstrokecolor{textcolor}%
\pgfsetfillcolor{textcolor}%
\pgftext[x=5.067209in,y=3.289477in,left,base]{\color{textcolor}\sffamily\fontsize{10.000000}{12.000000}\selectfont NN}%
\end{pgfscope}%
\end{pgfpicture}%
\makeatother%
\endgroup%

    \caption{Caption}
    \label{fig:my_label}
\end{figure}

\begin{figure}
%% Creator: Matplotlib, PGF backend
%%
%% To include the figure in your LaTeX document, write
%%   \input{<filename>.pgf}
%%
%% Make sure the required packages are loaded in your preamble
%%   \usepackage{pgf}
%%
%% Also ensure that all the required font packages are loaded; for instance,
%% the lmodern package is sometimes necessary when using math font.
%%   \usepackage{lmodern}
%%
%% Figures using additional raster images can only be included by \input if
%% they are in the same directory as the main LaTeX file. For loading figures
%% from other directories you can use the `import` package
%%   \usepackage{import}
%%
%% and then include the figures with
%%   \import{<path to file>}{<filename>.pgf}
%%
%% Matplotlib used the following preamble
%%   \usepackage{fontspec}
%%   \setmainfont{DejaVuSerif.ttf}[Path=\detokenize{C:/I/python38/Lib/site-packages/matplotlib/mpl-data/fonts/ttf/}]
%%   \setsansfont{DejaVuSans.ttf}[Path=\detokenize{C:/I/python38/Lib/site-packages/matplotlib/mpl-data/fonts/ttf/}]
%%   \setmonofont{DejaVuSansMono.ttf}[Path=\detokenize{C:/I/python38/Lib/site-packages/matplotlib/mpl-data/fonts/ttf/}]
%%
\begingroup%
\makeatletter%
\begin{pgfpicture}%
\pgfpathrectangle{\pgfpointorigin}{\pgfqpoint{6.000000in}{4.000000in}}%
\pgfusepath{use as bounding box, clip}%
\begin{pgfscope}%
\pgfsetbuttcap%
\pgfsetmiterjoin%
\pgfsetlinewidth{0.000000pt}%
\definecolor{currentstroke}{rgb}{1.000000,1.000000,1.000000}%
\pgfsetstrokecolor{currentstroke}%
\pgfsetstrokeopacity{0.000000}%
\pgfsetdash{}{0pt}%
\pgfpathmoveto{\pgfqpoint{0.000000in}{0.000000in}}%
\pgfpathlineto{\pgfqpoint{6.000000in}{0.000000in}}%
\pgfpathlineto{\pgfqpoint{6.000000in}{4.000000in}}%
\pgfpathlineto{\pgfqpoint{0.000000in}{4.000000in}}%
\pgfpathlineto{\pgfqpoint{0.000000in}{0.000000in}}%
\pgfpathclose%
\pgfusepath{}%
\end{pgfscope}%
\begin{pgfscope}%
\pgfsetbuttcap%
\pgfsetmiterjoin%
\definecolor{currentfill}{rgb}{1.000000,1.000000,1.000000}%
\pgfsetfillcolor{currentfill}%
\pgfsetlinewidth{0.000000pt}%
\definecolor{currentstroke}{rgb}{0.000000,0.000000,0.000000}%
\pgfsetstrokecolor{currentstroke}%
\pgfsetstrokeopacity{0.000000}%
\pgfsetdash{}{0pt}%
\pgfpathmoveto{\pgfqpoint{0.750000in}{0.500000in}}%
\pgfpathlineto{\pgfqpoint{5.400000in}{0.500000in}}%
\pgfpathlineto{\pgfqpoint{5.400000in}{3.520000in}}%
\pgfpathlineto{\pgfqpoint{0.750000in}{3.520000in}}%
\pgfpathlineto{\pgfqpoint{0.750000in}{0.500000in}}%
\pgfpathclose%
\pgfusepath{fill}%
\end{pgfscope}%
\begin{pgfscope}%
\pgfpathrectangle{\pgfqpoint{0.750000in}{0.500000in}}{\pgfqpoint{4.650000in}{3.020000in}}%
\pgfusepath{clip}%
\pgfsetbuttcap%
\pgfsetmiterjoin%
\definecolor{currentfill}{rgb}{1.000000,0.000000,0.000000}%
\pgfsetfillcolor{currentfill}%
\pgfsetlinewidth{0.000000pt}%
\definecolor{currentstroke}{rgb}{0.000000,0.000000,0.000000}%
\pgfsetstrokecolor{currentstroke}%
\pgfsetstrokeopacity{0.000000}%
\pgfsetdash{}{0pt}%
\pgfpathmoveto{\pgfqpoint{2.054246in}{0.500000in}}%
\pgfpathlineto{\pgfqpoint{2.074664in}{0.500000in}}%
\pgfpathlineto{\pgfqpoint{2.074664in}{0.510459in}}%
\pgfpathlineto{\pgfqpoint{2.054246in}{0.510459in}}%
\pgfpathlineto{\pgfqpoint{2.054246in}{0.500000in}}%
\pgfpathclose%
\pgfusepath{fill}%
\end{pgfscope}%
\begin{pgfscope}%
\pgfpathrectangle{\pgfqpoint{0.750000in}{0.500000in}}{\pgfqpoint{4.650000in}{3.020000in}}%
\pgfusepath{clip}%
\pgfsetbuttcap%
\pgfsetmiterjoin%
\definecolor{currentfill}{rgb}{1.000000,0.000000,0.000000}%
\pgfsetfillcolor{currentfill}%
\pgfsetlinewidth{0.000000pt}%
\definecolor{currentstroke}{rgb}{0.000000,0.000000,0.000000}%
\pgfsetstrokecolor{currentstroke}%
\pgfsetstrokeopacity{0.000000}%
\pgfsetdash{}{0pt}%
\pgfpathmoveto{\pgfqpoint{2.074664in}{0.500000in}}%
\pgfpathlineto{\pgfqpoint{2.095082in}{0.500000in}}%
\pgfpathlineto{\pgfqpoint{2.095082in}{0.500000in}}%
\pgfpathlineto{\pgfqpoint{2.074664in}{0.500000in}}%
\pgfpathlineto{\pgfqpoint{2.074664in}{0.500000in}}%
\pgfpathclose%
\pgfusepath{fill}%
\end{pgfscope}%
\begin{pgfscope}%
\pgfpathrectangle{\pgfqpoint{0.750000in}{0.500000in}}{\pgfqpoint{4.650000in}{3.020000in}}%
\pgfusepath{clip}%
\pgfsetbuttcap%
\pgfsetmiterjoin%
\definecolor{currentfill}{rgb}{1.000000,0.000000,0.000000}%
\pgfsetfillcolor{currentfill}%
\pgfsetlinewidth{0.000000pt}%
\definecolor{currentstroke}{rgb}{0.000000,0.000000,0.000000}%
\pgfsetstrokecolor{currentstroke}%
\pgfsetstrokeopacity{0.000000}%
\pgfsetdash{}{0pt}%
\pgfpathmoveto{\pgfqpoint{2.095082in}{0.500000in}}%
\pgfpathlineto{\pgfqpoint{2.115500in}{0.500000in}}%
\pgfpathlineto{\pgfqpoint{2.115500in}{0.500000in}}%
\pgfpathlineto{\pgfqpoint{2.095082in}{0.500000in}}%
\pgfpathlineto{\pgfqpoint{2.095082in}{0.500000in}}%
\pgfpathclose%
\pgfusepath{fill}%
\end{pgfscope}%
\begin{pgfscope}%
\pgfpathrectangle{\pgfqpoint{0.750000in}{0.500000in}}{\pgfqpoint{4.650000in}{3.020000in}}%
\pgfusepath{clip}%
\pgfsetbuttcap%
\pgfsetmiterjoin%
\definecolor{currentfill}{rgb}{1.000000,0.000000,0.000000}%
\pgfsetfillcolor{currentfill}%
\pgfsetlinewidth{0.000000pt}%
\definecolor{currentstroke}{rgb}{0.000000,0.000000,0.000000}%
\pgfsetstrokecolor{currentstroke}%
\pgfsetstrokeopacity{0.000000}%
\pgfsetdash{}{0pt}%
\pgfpathmoveto{\pgfqpoint{2.115500in}{0.500000in}}%
\pgfpathlineto{\pgfqpoint{2.135918in}{0.500000in}}%
\pgfpathlineto{\pgfqpoint{2.135918in}{0.500000in}}%
\pgfpathlineto{\pgfqpoint{2.115500in}{0.500000in}}%
\pgfpathlineto{\pgfqpoint{2.115500in}{0.500000in}}%
\pgfpathclose%
\pgfusepath{fill}%
\end{pgfscope}%
\begin{pgfscope}%
\pgfpathrectangle{\pgfqpoint{0.750000in}{0.500000in}}{\pgfqpoint{4.650000in}{3.020000in}}%
\pgfusepath{clip}%
\pgfsetbuttcap%
\pgfsetmiterjoin%
\definecolor{currentfill}{rgb}{1.000000,0.000000,0.000000}%
\pgfsetfillcolor{currentfill}%
\pgfsetlinewidth{0.000000pt}%
\definecolor{currentstroke}{rgb}{0.000000,0.000000,0.000000}%
\pgfsetstrokecolor{currentstroke}%
\pgfsetstrokeopacity{0.000000}%
\pgfsetdash{}{0pt}%
\pgfpathmoveto{\pgfqpoint{2.135918in}{0.500000in}}%
\pgfpathlineto{\pgfqpoint{2.156335in}{0.500000in}}%
\pgfpathlineto{\pgfqpoint{2.156335in}{0.500000in}}%
\pgfpathlineto{\pgfqpoint{2.135918in}{0.500000in}}%
\pgfpathlineto{\pgfqpoint{2.135918in}{0.500000in}}%
\pgfpathclose%
\pgfusepath{fill}%
\end{pgfscope}%
\begin{pgfscope}%
\pgfpathrectangle{\pgfqpoint{0.750000in}{0.500000in}}{\pgfqpoint{4.650000in}{3.020000in}}%
\pgfusepath{clip}%
\pgfsetbuttcap%
\pgfsetmiterjoin%
\definecolor{currentfill}{rgb}{1.000000,0.000000,0.000000}%
\pgfsetfillcolor{currentfill}%
\pgfsetlinewidth{0.000000pt}%
\definecolor{currentstroke}{rgb}{0.000000,0.000000,0.000000}%
\pgfsetstrokecolor{currentstroke}%
\pgfsetstrokeopacity{0.000000}%
\pgfsetdash{}{0pt}%
\pgfpathmoveto{\pgfqpoint{2.156335in}{0.500000in}}%
\pgfpathlineto{\pgfqpoint{2.176753in}{0.500000in}}%
\pgfpathlineto{\pgfqpoint{2.176753in}{0.500000in}}%
\pgfpathlineto{\pgfqpoint{2.156335in}{0.500000in}}%
\pgfpathlineto{\pgfqpoint{2.156335in}{0.500000in}}%
\pgfpathclose%
\pgfusepath{fill}%
\end{pgfscope}%
\begin{pgfscope}%
\pgfpathrectangle{\pgfqpoint{0.750000in}{0.500000in}}{\pgfqpoint{4.650000in}{3.020000in}}%
\pgfusepath{clip}%
\pgfsetbuttcap%
\pgfsetmiterjoin%
\definecolor{currentfill}{rgb}{1.000000,0.000000,0.000000}%
\pgfsetfillcolor{currentfill}%
\pgfsetlinewidth{0.000000pt}%
\definecolor{currentstroke}{rgb}{0.000000,0.000000,0.000000}%
\pgfsetstrokecolor{currentstroke}%
\pgfsetstrokeopacity{0.000000}%
\pgfsetdash{}{0pt}%
\pgfpathmoveto{\pgfqpoint{2.176753in}{0.500000in}}%
\pgfpathlineto{\pgfqpoint{2.197171in}{0.500000in}}%
\pgfpathlineto{\pgfqpoint{2.197171in}{0.500000in}}%
\pgfpathlineto{\pgfqpoint{2.176753in}{0.500000in}}%
\pgfpathlineto{\pgfqpoint{2.176753in}{0.500000in}}%
\pgfpathclose%
\pgfusepath{fill}%
\end{pgfscope}%
\begin{pgfscope}%
\pgfpathrectangle{\pgfqpoint{0.750000in}{0.500000in}}{\pgfqpoint{4.650000in}{3.020000in}}%
\pgfusepath{clip}%
\pgfsetbuttcap%
\pgfsetmiterjoin%
\definecolor{currentfill}{rgb}{1.000000,0.000000,0.000000}%
\pgfsetfillcolor{currentfill}%
\pgfsetlinewidth{0.000000pt}%
\definecolor{currentstroke}{rgb}{0.000000,0.000000,0.000000}%
\pgfsetstrokecolor{currentstroke}%
\pgfsetstrokeopacity{0.000000}%
\pgfsetdash{}{0pt}%
\pgfpathmoveto{\pgfqpoint{2.197171in}{0.500000in}}%
\pgfpathlineto{\pgfqpoint{2.217589in}{0.500000in}}%
\pgfpathlineto{\pgfqpoint{2.217589in}{0.510459in}}%
\pgfpathlineto{\pgfqpoint{2.197171in}{0.510459in}}%
\pgfpathlineto{\pgfqpoint{2.197171in}{0.500000in}}%
\pgfpathclose%
\pgfusepath{fill}%
\end{pgfscope}%
\begin{pgfscope}%
\pgfpathrectangle{\pgfqpoint{0.750000in}{0.500000in}}{\pgfqpoint{4.650000in}{3.020000in}}%
\pgfusepath{clip}%
\pgfsetbuttcap%
\pgfsetmiterjoin%
\definecolor{currentfill}{rgb}{1.000000,0.000000,0.000000}%
\pgfsetfillcolor{currentfill}%
\pgfsetlinewidth{0.000000pt}%
\definecolor{currentstroke}{rgb}{0.000000,0.000000,0.000000}%
\pgfsetstrokecolor{currentstroke}%
\pgfsetstrokeopacity{0.000000}%
\pgfsetdash{}{0pt}%
\pgfpathmoveto{\pgfqpoint{2.217589in}{0.500000in}}%
\pgfpathlineto{\pgfqpoint{2.238007in}{0.500000in}}%
\pgfpathlineto{\pgfqpoint{2.238007in}{0.500000in}}%
\pgfpathlineto{\pgfqpoint{2.217589in}{0.500000in}}%
\pgfpathlineto{\pgfqpoint{2.217589in}{0.500000in}}%
\pgfpathclose%
\pgfusepath{fill}%
\end{pgfscope}%
\begin{pgfscope}%
\pgfpathrectangle{\pgfqpoint{0.750000in}{0.500000in}}{\pgfqpoint{4.650000in}{3.020000in}}%
\pgfusepath{clip}%
\pgfsetbuttcap%
\pgfsetmiterjoin%
\definecolor{currentfill}{rgb}{1.000000,0.000000,0.000000}%
\pgfsetfillcolor{currentfill}%
\pgfsetlinewidth{0.000000pt}%
\definecolor{currentstroke}{rgb}{0.000000,0.000000,0.000000}%
\pgfsetstrokecolor{currentstroke}%
\pgfsetstrokeopacity{0.000000}%
\pgfsetdash{}{0pt}%
\pgfpathmoveto{\pgfqpoint{2.238007in}{0.500000in}}%
\pgfpathlineto{\pgfqpoint{2.258425in}{0.500000in}}%
\pgfpathlineto{\pgfqpoint{2.258425in}{0.510459in}}%
\pgfpathlineto{\pgfqpoint{2.238007in}{0.510459in}}%
\pgfpathlineto{\pgfqpoint{2.238007in}{0.500000in}}%
\pgfpathclose%
\pgfusepath{fill}%
\end{pgfscope}%
\begin{pgfscope}%
\pgfpathrectangle{\pgfqpoint{0.750000in}{0.500000in}}{\pgfqpoint{4.650000in}{3.020000in}}%
\pgfusepath{clip}%
\pgfsetbuttcap%
\pgfsetmiterjoin%
\definecolor{currentfill}{rgb}{1.000000,0.000000,0.000000}%
\pgfsetfillcolor{currentfill}%
\pgfsetlinewidth{0.000000pt}%
\definecolor{currentstroke}{rgb}{0.000000,0.000000,0.000000}%
\pgfsetstrokecolor{currentstroke}%
\pgfsetstrokeopacity{0.000000}%
\pgfsetdash{}{0pt}%
\pgfpathmoveto{\pgfqpoint{2.258425in}{0.500000in}}%
\pgfpathlineto{\pgfqpoint{2.278842in}{0.500000in}}%
\pgfpathlineto{\pgfqpoint{2.278842in}{0.500000in}}%
\pgfpathlineto{\pgfqpoint{2.258425in}{0.500000in}}%
\pgfpathlineto{\pgfqpoint{2.258425in}{0.500000in}}%
\pgfpathclose%
\pgfusepath{fill}%
\end{pgfscope}%
\begin{pgfscope}%
\pgfpathrectangle{\pgfqpoint{0.750000in}{0.500000in}}{\pgfqpoint{4.650000in}{3.020000in}}%
\pgfusepath{clip}%
\pgfsetbuttcap%
\pgfsetmiterjoin%
\definecolor{currentfill}{rgb}{1.000000,0.000000,0.000000}%
\pgfsetfillcolor{currentfill}%
\pgfsetlinewidth{0.000000pt}%
\definecolor{currentstroke}{rgb}{0.000000,0.000000,0.000000}%
\pgfsetstrokecolor{currentstroke}%
\pgfsetstrokeopacity{0.000000}%
\pgfsetdash{}{0pt}%
\pgfpathmoveto{\pgfqpoint{2.278842in}{0.500000in}}%
\pgfpathlineto{\pgfqpoint{2.299260in}{0.500000in}}%
\pgfpathlineto{\pgfqpoint{2.299260in}{0.510459in}}%
\pgfpathlineto{\pgfqpoint{2.278842in}{0.510459in}}%
\pgfpathlineto{\pgfqpoint{2.278842in}{0.500000in}}%
\pgfpathclose%
\pgfusepath{fill}%
\end{pgfscope}%
\begin{pgfscope}%
\pgfpathrectangle{\pgfqpoint{0.750000in}{0.500000in}}{\pgfqpoint{4.650000in}{3.020000in}}%
\pgfusepath{clip}%
\pgfsetbuttcap%
\pgfsetmiterjoin%
\definecolor{currentfill}{rgb}{1.000000,0.000000,0.000000}%
\pgfsetfillcolor{currentfill}%
\pgfsetlinewidth{0.000000pt}%
\definecolor{currentstroke}{rgb}{0.000000,0.000000,0.000000}%
\pgfsetstrokecolor{currentstroke}%
\pgfsetstrokeopacity{0.000000}%
\pgfsetdash{}{0pt}%
\pgfpathmoveto{\pgfqpoint{2.299260in}{0.500000in}}%
\pgfpathlineto{\pgfqpoint{2.319678in}{0.500000in}}%
\pgfpathlineto{\pgfqpoint{2.319678in}{0.531377in}}%
\pgfpathlineto{\pgfqpoint{2.299260in}{0.531377in}}%
\pgfpathlineto{\pgfqpoint{2.299260in}{0.500000in}}%
\pgfpathclose%
\pgfusepath{fill}%
\end{pgfscope}%
\begin{pgfscope}%
\pgfpathrectangle{\pgfqpoint{0.750000in}{0.500000in}}{\pgfqpoint{4.650000in}{3.020000in}}%
\pgfusepath{clip}%
\pgfsetbuttcap%
\pgfsetmiterjoin%
\definecolor{currentfill}{rgb}{1.000000,0.000000,0.000000}%
\pgfsetfillcolor{currentfill}%
\pgfsetlinewidth{0.000000pt}%
\definecolor{currentstroke}{rgb}{0.000000,0.000000,0.000000}%
\pgfsetstrokecolor{currentstroke}%
\pgfsetstrokeopacity{0.000000}%
\pgfsetdash{}{0pt}%
\pgfpathmoveto{\pgfqpoint{2.319678in}{0.500000in}}%
\pgfpathlineto{\pgfqpoint{2.340096in}{0.500000in}}%
\pgfpathlineto{\pgfqpoint{2.340096in}{0.500000in}}%
\pgfpathlineto{\pgfqpoint{2.319678in}{0.500000in}}%
\pgfpathlineto{\pgfqpoint{2.319678in}{0.500000in}}%
\pgfpathclose%
\pgfusepath{fill}%
\end{pgfscope}%
\begin{pgfscope}%
\pgfpathrectangle{\pgfqpoint{0.750000in}{0.500000in}}{\pgfqpoint{4.650000in}{3.020000in}}%
\pgfusepath{clip}%
\pgfsetbuttcap%
\pgfsetmiterjoin%
\definecolor{currentfill}{rgb}{1.000000,0.000000,0.000000}%
\pgfsetfillcolor{currentfill}%
\pgfsetlinewidth{0.000000pt}%
\definecolor{currentstroke}{rgb}{0.000000,0.000000,0.000000}%
\pgfsetstrokecolor{currentstroke}%
\pgfsetstrokeopacity{0.000000}%
\pgfsetdash{}{0pt}%
\pgfpathmoveto{\pgfqpoint{2.340096in}{0.500000in}}%
\pgfpathlineto{\pgfqpoint{2.360514in}{0.500000in}}%
\pgfpathlineto{\pgfqpoint{2.360514in}{0.510459in}}%
\pgfpathlineto{\pgfqpoint{2.340096in}{0.510459in}}%
\pgfpathlineto{\pgfqpoint{2.340096in}{0.500000in}}%
\pgfpathclose%
\pgfusepath{fill}%
\end{pgfscope}%
\begin{pgfscope}%
\pgfpathrectangle{\pgfqpoint{0.750000in}{0.500000in}}{\pgfqpoint{4.650000in}{3.020000in}}%
\pgfusepath{clip}%
\pgfsetbuttcap%
\pgfsetmiterjoin%
\definecolor{currentfill}{rgb}{1.000000,0.000000,0.000000}%
\pgfsetfillcolor{currentfill}%
\pgfsetlinewidth{0.000000pt}%
\definecolor{currentstroke}{rgb}{0.000000,0.000000,0.000000}%
\pgfsetstrokecolor{currentstroke}%
\pgfsetstrokeopacity{0.000000}%
\pgfsetdash{}{0pt}%
\pgfpathmoveto{\pgfqpoint{2.360514in}{0.500000in}}%
\pgfpathlineto{\pgfqpoint{2.380931in}{0.500000in}}%
\pgfpathlineto{\pgfqpoint{2.380931in}{0.500000in}}%
\pgfpathlineto{\pgfqpoint{2.360514in}{0.500000in}}%
\pgfpathlineto{\pgfqpoint{2.360514in}{0.500000in}}%
\pgfpathclose%
\pgfusepath{fill}%
\end{pgfscope}%
\begin{pgfscope}%
\pgfpathrectangle{\pgfqpoint{0.750000in}{0.500000in}}{\pgfqpoint{4.650000in}{3.020000in}}%
\pgfusepath{clip}%
\pgfsetbuttcap%
\pgfsetmiterjoin%
\definecolor{currentfill}{rgb}{1.000000,0.000000,0.000000}%
\pgfsetfillcolor{currentfill}%
\pgfsetlinewidth{0.000000pt}%
\definecolor{currentstroke}{rgb}{0.000000,0.000000,0.000000}%
\pgfsetstrokecolor{currentstroke}%
\pgfsetstrokeopacity{0.000000}%
\pgfsetdash{}{0pt}%
\pgfpathmoveto{\pgfqpoint{2.380931in}{0.500000in}}%
\pgfpathlineto{\pgfqpoint{2.401349in}{0.500000in}}%
\pgfpathlineto{\pgfqpoint{2.401349in}{0.510459in}}%
\pgfpathlineto{\pgfqpoint{2.380931in}{0.510459in}}%
\pgfpathlineto{\pgfqpoint{2.380931in}{0.500000in}}%
\pgfpathclose%
\pgfusepath{fill}%
\end{pgfscope}%
\begin{pgfscope}%
\pgfpathrectangle{\pgfqpoint{0.750000in}{0.500000in}}{\pgfqpoint{4.650000in}{3.020000in}}%
\pgfusepath{clip}%
\pgfsetbuttcap%
\pgfsetmiterjoin%
\definecolor{currentfill}{rgb}{1.000000,0.000000,0.000000}%
\pgfsetfillcolor{currentfill}%
\pgfsetlinewidth{0.000000pt}%
\definecolor{currentstroke}{rgb}{0.000000,0.000000,0.000000}%
\pgfsetstrokecolor{currentstroke}%
\pgfsetstrokeopacity{0.000000}%
\pgfsetdash{}{0pt}%
\pgfpathmoveto{\pgfqpoint{2.401349in}{0.500000in}}%
\pgfpathlineto{\pgfqpoint{2.421767in}{0.500000in}}%
\pgfpathlineto{\pgfqpoint{2.421767in}{0.541835in}}%
\pgfpathlineto{\pgfqpoint{2.401349in}{0.541835in}}%
\pgfpathlineto{\pgfqpoint{2.401349in}{0.500000in}}%
\pgfpathclose%
\pgfusepath{fill}%
\end{pgfscope}%
\begin{pgfscope}%
\pgfpathrectangle{\pgfqpoint{0.750000in}{0.500000in}}{\pgfqpoint{4.650000in}{3.020000in}}%
\pgfusepath{clip}%
\pgfsetbuttcap%
\pgfsetmiterjoin%
\definecolor{currentfill}{rgb}{1.000000,0.000000,0.000000}%
\pgfsetfillcolor{currentfill}%
\pgfsetlinewidth{0.000000pt}%
\definecolor{currentstroke}{rgb}{0.000000,0.000000,0.000000}%
\pgfsetstrokecolor{currentstroke}%
\pgfsetstrokeopacity{0.000000}%
\pgfsetdash{}{0pt}%
\pgfpathmoveto{\pgfqpoint{2.421767in}{0.500000in}}%
\pgfpathlineto{\pgfqpoint{2.442185in}{0.500000in}}%
\pgfpathlineto{\pgfqpoint{2.442185in}{0.510459in}}%
\pgfpathlineto{\pgfqpoint{2.421767in}{0.510459in}}%
\pgfpathlineto{\pgfqpoint{2.421767in}{0.500000in}}%
\pgfpathclose%
\pgfusepath{fill}%
\end{pgfscope}%
\begin{pgfscope}%
\pgfpathrectangle{\pgfqpoint{0.750000in}{0.500000in}}{\pgfqpoint{4.650000in}{3.020000in}}%
\pgfusepath{clip}%
\pgfsetbuttcap%
\pgfsetmiterjoin%
\definecolor{currentfill}{rgb}{1.000000,0.000000,0.000000}%
\pgfsetfillcolor{currentfill}%
\pgfsetlinewidth{0.000000pt}%
\definecolor{currentstroke}{rgb}{0.000000,0.000000,0.000000}%
\pgfsetstrokecolor{currentstroke}%
\pgfsetstrokeopacity{0.000000}%
\pgfsetdash{}{0pt}%
\pgfpathmoveto{\pgfqpoint{2.442185in}{0.500000in}}%
\pgfpathlineto{\pgfqpoint{2.462603in}{0.500000in}}%
\pgfpathlineto{\pgfqpoint{2.462603in}{0.635965in}}%
\pgfpathlineto{\pgfqpoint{2.442185in}{0.635965in}}%
\pgfpathlineto{\pgfqpoint{2.442185in}{0.500000in}}%
\pgfpathclose%
\pgfusepath{fill}%
\end{pgfscope}%
\begin{pgfscope}%
\pgfpathrectangle{\pgfqpoint{0.750000in}{0.500000in}}{\pgfqpoint{4.650000in}{3.020000in}}%
\pgfusepath{clip}%
\pgfsetbuttcap%
\pgfsetmiterjoin%
\definecolor{currentfill}{rgb}{1.000000,0.000000,0.000000}%
\pgfsetfillcolor{currentfill}%
\pgfsetlinewidth{0.000000pt}%
\definecolor{currentstroke}{rgb}{0.000000,0.000000,0.000000}%
\pgfsetstrokecolor{currentstroke}%
\pgfsetstrokeopacity{0.000000}%
\pgfsetdash{}{0pt}%
\pgfpathmoveto{\pgfqpoint{2.462603in}{0.500000in}}%
\pgfpathlineto{\pgfqpoint{2.483020in}{0.500000in}}%
\pgfpathlineto{\pgfqpoint{2.483020in}{0.510459in}}%
\pgfpathlineto{\pgfqpoint{2.462603in}{0.510459in}}%
\pgfpathlineto{\pgfqpoint{2.462603in}{0.500000in}}%
\pgfpathclose%
\pgfusepath{fill}%
\end{pgfscope}%
\begin{pgfscope}%
\pgfpathrectangle{\pgfqpoint{0.750000in}{0.500000in}}{\pgfqpoint{4.650000in}{3.020000in}}%
\pgfusepath{clip}%
\pgfsetbuttcap%
\pgfsetmiterjoin%
\definecolor{currentfill}{rgb}{1.000000,0.000000,0.000000}%
\pgfsetfillcolor{currentfill}%
\pgfsetlinewidth{0.000000pt}%
\definecolor{currentstroke}{rgb}{0.000000,0.000000,0.000000}%
\pgfsetstrokecolor{currentstroke}%
\pgfsetstrokeopacity{0.000000}%
\pgfsetdash{}{0pt}%
\pgfpathmoveto{\pgfqpoint{2.483020in}{0.500000in}}%
\pgfpathlineto{\pgfqpoint{2.503438in}{0.500000in}}%
\pgfpathlineto{\pgfqpoint{2.503438in}{0.520918in}}%
\pgfpathlineto{\pgfqpoint{2.483020in}{0.520918in}}%
\pgfpathlineto{\pgfqpoint{2.483020in}{0.500000in}}%
\pgfpathclose%
\pgfusepath{fill}%
\end{pgfscope}%
\begin{pgfscope}%
\pgfpathrectangle{\pgfqpoint{0.750000in}{0.500000in}}{\pgfqpoint{4.650000in}{3.020000in}}%
\pgfusepath{clip}%
\pgfsetbuttcap%
\pgfsetmiterjoin%
\definecolor{currentfill}{rgb}{1.000000,0.000000,0.000000}%
\pgfsetfillcolor{currentfill}%
\pgfsetlinewidth{0.000000pt}%
\definecolor{currentstroke}{rgb}{0.000000,0.000000,0.000000}%
\pgfsetstrokecolor{currentstroke}%
\pgfsetstrokeopacity{0.000000}%
\pgfsetdash{}{0pt}%
\pgfpathmoveto{\pgfqpoint{2.503438in}{0.500000in}}%
\pgfpathlineto{\pgfqpoint{2.523856in}{0.500000in}}%
\pgfpathlineto{\pgfqpoint{2.523856in}{0.615048in}}%
\pgfpathlineto{\pgfqpoint{2.503438in}{0.615048in}}%
\pgfpathlineto{\pgfqpoint{2.503438in}{0.500000in}}%
\pgfpathclose%
\pgfusepath{fill}%
\end{pgfscope}%
\begin{pgfscope}%
\pgfpathrectangle{\pgfqpoint{0.750000in}{0.500000in}}{\pgfqpoint{4.650000in}{3.020000in}}%
\pgfusepath{clip}%
\pgfsetbuttcap%
\pgfsetmiterjoin%
\definecolor{currentfill}{rgb}{1.000000,0.000000,0.000000}%
\pgfsetfillcolor{currentfill}%
\pgfsetlinewidth{0.000000pt}%
\definecolor{currentstroke}{rgb}{0.000000,0.000000,0.000000}%
\pgfsetstrokecolor{currentstroke}%
\pgfsetstrokeopacity{0.000000}%
\pgfsetdash{}{0pt}%
\pgfpathmoveto{\pgfqpoint{2.523856in}{0.500000in}}%
\pgfpathlineto{\pgfqpoint{2.544274in}{0.500000in}}%
\pgfpathlineto{\pgfqpoint{2.544274in}{0.500000in}}%
\pgfpathlineto{\pgfqpoint{2.523856in}{0.500000in}}%
\pgfpathlineto{\pgfqpoint{2.523856in}{0.500000in}}%
\pgfpathclose%
\pgfusepath{fill}%
\end{pgfscope}%
\begin{pgfscope}%
\pgfpathrectangle{\pgfqpoint{0.750000in}{0.500000in}}{\pgfqpoint{4.650000in}{3.020000in}}%
\pgfusepath{clip}%
\pgfsetbuttcap%
\pgfsetmiterjoin%
\definecolor{currentfill}{rgb}{1.000000,0.000000,0.000000}%
\pgfsetfillcolor{currentfill}%
\pgfsetlinewidth{0.000000pt}%
\definecolor{currentstroke}{rgb}{0.000000,0.000000,0.000000}%
\pgfsetstrokecolor{currentstroke}%
\pgfsetstrokeopacity{0.000000}%
\pgfsetdash{}{0pt}%
\pgfpathmoveto{\pgfqpoint{2.544274in}{0.500000in}}%
\pgfpathlineto{\pgfqpoint{2.564692in}{0.500000in}}%
\pgfpathlineto{\pgfqpoint{2.564692in}{0.656883in}}%
\pgfpathlineto{\pgfqpoint{2.544274in}{0.656883in}}%
\pgfpathlineto{\pgfqpoint{2.544274in}{0.500000in}}%
\pgfpathclose%
\pgfusepath{fill}%
\end{pgfscope}%
\begin{pgfscope}%
\pgfpathrectangle{\pgfqpoint{0.750000in}{0.500000in}}{\pgfqpoint{4.650000in}{3.020000in}}%
\pgfusepath{clip}%
\pgfsetbuttcap%
\pgfsetmiterjoin%
\definecolor{currentfill}{rgb}{1.000000,0.000000,0.000000}%
\pgfsetfillcolor{currentfill}%
\pgfsetlinewidth{0.000000pt}%
\definecolor{currentstroke}{rgb}{0.000000,0.000000,0.000000}%
\pgfsetstrokecolor{currentstroke}%
\pgfsetstrokeopacity{0.000000}%
\pgfsetdash{}{0pt}%
\pgfpathmoveto{\pgfqpoint{2.564692in}{0.500000in}}%
\pgfpathlineto{\pgfqpoint{2.585109in}{0.500000in}}%
\pgfpathlineto{\pgfqpoint{2.585109in}{0.510459in}}%
\pgfpathlineto{\pgfqpoint{2.564692in}{0.510459in}}%
\pgfpathlineto{\pgfqpoint{2.564692in}{0.500000in}}%
\pgfpathclose%
\pgfusepath{fill}%
\end{pgfscope}%
\begin{pgfscope}%
\pgfpathrectangle{\pgfqpoint{0.750000in}{0.500000in}}{\pgfqpoint{4.650000in}{3.020000in}}%
\pgfusepath{clip}%
\pgfsetbuttcap%
\pgfsetmiterjoin%
\definecolor{currentfill}{rgb}{1.000000,0.000000,0.000000}%
\pgfsetfillcolor{currentfill}%
\pgfsetlinewidth{0.000000pt}%
\definecolor{currentstroke}{rgb}{0.000000,0.000000,0.000000}%
\pgfsetstrokecolor{currentstroke}%
\pgfsetstrokeopacity{0.000000}%
\pgfsetdash{}{0pt}%
\pgfpathmoveto{\pgfqpoint{2.585109in}{0.500000in}}%
\pgfpathlineto{\pgfqpoint{2.605527in}{0.500000in}}%
\pgfpathlineto{\pgfqpoint{2.605527in}{0.677801in}}%
\pgfpathlineto{\pgfqpoint{2.585109in}{0.677801in}}%
\pgfpathlineto{\pgfqpoint{2.585109in}{0.500000in}}%
\pgfpathclose%
\pgfusepath{fill}%
\end{pgfscope}%
\begin{pgfscope}%
\pgfpathrectangle{\pgfqpoint{0.750000in}{0.500000in}}{\pgfqpoint{4.650000in}{3.020000in}}%
\pgfusepath{clip}%
\pgfsetbuttcap%
\pgfsetmiterjoin%
\definecolor{currentfill}{rgb}{1.000000,0.000000,0.000000}%
\pgfsetfillcolor{currentfill}%
\pgfsetlinewidth{0.000000pt}%
\definecolor{currentstroke}{rgb}{0.000000,0.000000,0.000000}%
\pgfsetstrokecolor{currentstroke}%
\pgfsetstrokeopacity{0.000000}%
\pgfsetdash{}{0pt}%
\pgfpathmoveto{\pgfqpoint{2.605527in}{0.500000in}}%
\pgfpathlineto{\pgfqpoint{2.625945in}{0.500000in}}%
\pgfpathlineto{\pgfqpoint{2.625945in}{0.541835in}}%
\pgfpathlineto{\pgfqpoint{2.605527in}{0.541835in}}%
\pgfpathlineto{\pgfqpoint{2.605527in}{0.500000in}}%
\pgfpathclose%
\pgfusepath{fill}%
\end{pgfscope}%
\begin{pgfscope}%
\pgfpathrectangle{\pgfqpoint{0.750000in}{0.500000in}}{\pgfqpoint{4.650000in}{3.020000in}}%
\pgfusepath{clip}%
\pgfsetbuttcap%
\pgfsetmiterjoin%
\definecolor{currentfill}{rgb}{1.000000,0.000000,0.000000}%
\pgfsetfillcolor{currentfill}%
\pgfsetlinewidth{0.000000pt}%
\definecolor{currentstroke}{rgb}{0.000000,0.000000,0.000000}%
\pgfsetstrokecolor{currentstroke}%
\pgfsetstrokeopacity{0.000000}%
\pgfsetdash{}{0pt}%
\pgfpathmoveto{\pgfqpoint{2.625945in}{0.500000in}}%
\pgfpathlineto{\pgfqpoint{2.646363in}{0.500000in}}%
\pgfpathlineto{\pgfqpoint{2.646363in}{0.541835in}}%
\pgfpathlineto{\pgfqpoint{2.625945in}{0.541835in}}%
\pgfpathlineto{\pgfqpoint{2.625945in}{0.500000in}}%
\pgfpathclose%
\pgfusepath{fill}%
\end{pgfscope}%
\begin{pgfscope}%
\pgfpathrectangle{\pgfqpoint{0.750000in}{0.500000in}}{\pgfqpoint{4.650000in}{3.020000in}}%
\pgfusepath{clip}%
\pgfsetbuttcap%
\pgfsetmiterjoin%
\definecolor{currentfill}{rgb}{1.000000,0.000000,0.000000}%
\pgfsetfillcolor{currentfill}%
\pgfsetlinewidth{0.000000pt}%
\definecolor{currentstroke}{rgb}{0.000000,0.000000,0.000000}%
\pgfsetstrokecolor{currentstroke}%
\pgfsetstrokeopacity{0.000000}%
\pgfsetdash{}{0pt}%
\pgfpathmoveto{\pgfqpoint{2.646363in}{0.500000in}}%
\pgfpathlineto{\pgfqpoint{2.666781in}{0.500000in}}%
\pgfpathlineto{\pgfqpoint{2.666781in}{0.918355in}}%
\pgfpathlineto{\pgfqpoint{2.646363in}{0.918355in}}%
\pgfpathlineto{\pgfqpoint{2.646363in}{0.500000in}}%
\pgfpathclose%
\pgfusepath{fill}%
\end{pgfscope}%
\begin{pgfscope}%
\pgfpathrectangle{\pgfqpoint{0.750000in}{0.500000in}}{\pgfqpoint{4.650000in}{3.020000in}}%
\pgfusepath{clip}%
\pgfsetbuttcap%
\pgfsetmiterjoin%
\definecolor{currentfill}{rgb}{1.000000,0.000000,0.000000}%
\pgfsetfillcolor{currentfill}%
\pgfsetlinewidth{0.000000pt}%
\definecolor{currentstroke}{rgb}{0.000000,0.000000,0.000000}%
\pgfsetstrokecolor{currentstroke}%
\pgfsetstrokeopacity{0.000000}%
\pgfsetdash{}{0pt}%
\pgfpathmoveto{\pgfqpoint{2.666781in}{0.500000in}}%
\pgfpathlineto{\pgfqpoint{2.687198in}{0.500000in}}%
\pgfpathlineto{\pgfqpoint{2.687198in}{0.541835in}}%
\pgfpathlineto{\pgfqpoint{2.666781in}{0.541835in}}%
\pgfpathlineto{\pgfqpoint{2.666781in}{0.500000in}}%
\pgfpathclose%
\pgfusepath{fill}%
\end{pgfscope}%
\begin{pgfscope}%
\pgfpathrectangle{\pgfqpoint{0.750000in}{0.500000in}}{\pgfqpoint{4.650000in}{3.020000in}}%
\pgfusepath{clip}%
\pgfsetbuttcap%
\pgfsetmiterjoin%
\definecolor{currentfill}{rgb}{1.000000,0.000000,0.000000}%
\pgfsetfillcolor{currentfill}%
\pgfsetlinewidth{0.000000pt}%
\definecolor{currentstroke}{rgb}{0.000000,0.000000,0.000000}%
\pgfsetstrokecolor{currentstroke}%
\pgfsetstrokeopacity{0.000000}%
\pgfsetdash{}{0pt}%
\pgfpathmoveto{\pgfqpoint{2.687198in}{0.500000in}}%
\pgfpathlineto{\pgfqpoint{2.707616in}{0.500000in}}%
\pgfpathlineto{\pgfqpoint{2.707616in}{0.907896in}}%
\pgfpathlineto{\pgfqpoint{2.687198in}{0.907896in}}%
\pgfpathlineto{\pgfqpoint{2.687198in}{0.500000in}}%
\pgfpathclose%
\pgfusepath{fill}%
\end{pgfscope}%
\begin{pgfscope}%
\pgfpathrectangle{\pgfqpoint{0.750000in}{0.500000in}}{\pgfqpoint{4.650000in}{3.020000in}}%
\pgfusepath{clip}%
\pgfsetbuttcap%
\pgfsetmiterjoin%
\definecolor{currentfill}{rgb}{1.000000,0.000000,0.000000}%
\pgfsetfillcolor{currentfill}%
\pgfsetlinewidth{0.000000pt}%
\definecolor{currentstroke}{rgb}{0.000000,0.000000,0.000000}%
\pgfsetstrokecolor{currentstroke}%
\pgfsetstrokeopacity{0.000000}%
\pgfsetdash{}{0pt}%
\pgfpathmoveto{\pgfqpoint{2.707616in}{0.500000in}}%
\pgfpathlineto{\pgfqpoint{2.728034in}{0.500000in}}%
\pgfpathlineto{\pgfqpoint{2.728034in}{0.520918in}}%
\pgfpathlineto{\pgfqpoint{2.707616in}{0.520918in}}%
\pgfpathlineto{\pgfqpoint{2.707616in}{0.500000in}}%
\pgfpathclose%
\pgfusepath{fill}%
\end{pgfscope}%
\begin{pgfscope}%
\pgfpathrectangle{\pgfqpoint{0.750000in}{0.500000in}}{\pgfqpoint{4.650000in}{3.020000in}}%
\pgfusepath{clip}%
\pgfsetbuttcap%
\pgfsetmiterjoin%
\definecolor{currentfill}{rgb}{1.000000,0.000000,0.000000}%
\pgfsetfillcolor{currentfill}%
\pgfsetlinewidth{0.000000pt}%
\definecolor{currentstroke}{rgb}{0.000000,0.000000,0.000000}%
\pgfsetstrokecolor{currentstroke}%
\pgfsetstrokeopacity{0.000000}%
\pgfsetdash{}{0pt}%
\pgfpathmoveto{\pgfqpoint{2.728034in}{0.500000in}}%
\pgfpathlineto{\pgfqpoint{2.748452in}{0.500000in}}%
\pgfpathlineto{\pgfqpoint{2.748452in}{0.604589in}}%
\pgfpathlineto{\pgfqpoint{2.728034in}{0.604589in}}%
\pgfpathlineto{\pgfqpoint{2.728034in}{0.500000in}}%
\pgfpathclose%
\pgfusepath{fill}%
\end{pgfscope}%
\begin{pgfscope}%
\pgfpathrectangle{\pgfqpoint{0.750000in}{0.500000in}}{\pgfqpoint{4.650000in}{3.020000in}}%
\pgfusepath{clip}%
\pgfsetbuttcap%
\pgfsetmiterjoin%
\definecolor{currentfill}{rgb}{1.000000,0.000000,0.000000}%
\pgfsetfillcolor{currentfill}%
\pgfsetlinewidth{0.000000pt}%
\definecolor{currentstroke}{rgb}{0.000000,0.000000,0.000000}%
\pgfsetstrokecolor{currentstroke}%
\pgfsetstrokeopacity{0.000000}%
\pgfsetdash{}{0pt}%
\pgfpathmoveto{\pgfqpoint{2.748452in}{0.500000in}}%
\pgfpathlineto{\pgfqpoint{2.768870in}{0.500000in}}%
\pgfpathlineto{\pgfqpoint{2.768870in}{1.012485in}}%
\pgfpathlineto{\pgfqpoint{2.748452in}{1.012485in}}%
\pgfpathlineto{\pgfqpoint{2.748452in}{0.500000in}}%
\pgfpathclose%
\pgfusepath{fill}%
\end{pgfscope}%
\begin{pgfscope}%
\pgfpathrectangle{\pgfqpoint{0.750000in}{0.500000in}}{\pgfqpoint{4.650000in}{3.020000in}}%
\pgfusepath{clip}%
\pgfsetbuttcap%
\pgfsetmiterjoin%
\definecolor{currentfill}{rgb}{1.000000,0.000000,0.000000}%
\pgfsetfillcolor{currentfill}%
\pgfsetlinewidth{0.000000pt}%
\definecolor{currentstroke}{rgb}{0.000000,0.000000,0.000000}%
\pgfsetstrokecolor{currentstroke}%
\pgfsetstrokeopacity{0.000000}%
\pgfsetdash{}{0pt}%
\pgfpathmoveto{\pgfqpoint{2.768870in}{0.500000in}}%
\pgfpathlineto{\pgfqpoint{2.789287in}{0.500000in}}%
\pgfpathlineto{\pgfqpoint{2.789287in}{0.573212in}}%
\pgfpathlineto{\pgfqpoint{2.768870in}{0.573212in}}%
\pgfpathlineto{\pgfqpoint{2.768870in}{0.500000in}}%
\pgfpathclose%
\pgfusepath{fill}%
\end{pgfscope}%
\begin{pgfscope}%
\pgfpathrectangle{\pgfqpoint{0.750000in}{0.500000in}}{\pgfqpoint{4.650000in}{3.020000in}}%
\pgfusepath{clip}%
\pgfsetbuttcap%
\pgfsetmiterjoin%
\definecolor{currentfill}{rgb}{1.000000,0.000000,0.000000}%
\pgfsetfillcolor{currentfill}%
\pgfsetlinewidth{0.000000pt}%
\definecolor{currentstroke}{rgb}{0.000000,0.000000,0.000000}%
\pgfsetstrokecolor{currentstroke}%
\pgfsetstrokeopacity{0.000000}%
\pgfsetdash{}{0pt}%
\pgfpathmoveto{\pgfqpoint{2.789287in}{0.500000in}}%
\pgfpathlineto{\pgfqpoint{2.809705in}{0.500000in}}%
\pgfpathlineto{\pgfqpoint{2.809705in}{1.002026in}}%
\pgfpathlineto{\pgfqpoint{2.789287in}{1.002026in}}%
\pgfpathlineto{\pgfqpoint{2.789287in}{0.500000in}}%
\pgfpathclose%
\pgfusepath{fill}%
\end{pgfscope}%
\begin{pgfscope}%
\pgfpathrectangle{\pgfqpoint{0.750000in}{0.500000in}}{\pgfqpoint{4.650000in}{3.020000in}}%
\pgfusepath{clip}%
\pgfsetbuttcap%
\pgfsetmiterjoin%
\definecolor{currentfill}{rgb}{1.000000,0.000000,0.000000}%
\pgfsetfillcolor{currentfill}%
\pgfsetlinewidth{0.000000pt}%
\definecolor{currentstroke}{rgb}{0.000000,0.000000,0.000000}%
\pgfsetstrokecolor{currentstroke}%
\pgfsetstrokeopacity{0.000000}%
\pgfsetdash{}{0pt}%
\pgfpathmoveto{\pgfqpoint{2.809705in}{0.500000in}}%
\pgfpathlineto{\pgfqpoint{2.830123in}{0.500000in}}%
\pgfpathlineto{\pgfqpoint{2.830123in}{0.625506in}}%
\pgfpathlineto{\pgfqpoint{2.809705in}{0.625506in}}%
\pgfpathlineto{\pgfqpoint{2.809705in}{0.500000in}}%
\pgfpathclose%
\pgfusepath{fill}%
\end{pgfscope}%
\begin{pgfscope}%
\pgfpathrectangle{\pgfqpoint{0.750000in}{0.500000in}}{\pgfqpoint{4.650000in}{3.020000in}}%
\pgfusepath{clip}%
\pgfsetbuttcap%
\pgfsetmiterjoin%
\definecolor{currentfill}{rgb}{1.000000,0.000000,0.000000}%
\pgfsetfillcolor{currentfill}%
\pgfsetlinewidth{0.000000pt}%
\definecolor{currentstroke}{rgb}{0.000000,0.000000,0.000000}%
\pgfsetstrokecolor{currentstroke}%
\pgfsetstrokeopacity{0.000000}%
\pgfsetdash{}{0pt}%
\pgfpathmoveto{\pgfqpoint{2.830123in}{0.500000in}}%
\pgfpathlineto{\pgfqpoint{2.850541in}{0.500000in}}%
\pgfpathlineto{\pgfqpoint{2.850541in}{0.615048in}}%
\pgfpathlineto{\pgfqpoint{2.830123in}{0.615048in}}%
\pgfpathlineto{\pgfqpoint{2.830123in}{0.500000in}}%
\pgfpathclose%
\pgfusepath{fill}%
\end{pgfscope}%
\begin{pgfscope}%
\pgfpathrectangle{\pgfqpoint{0.750000in}{0.500000in}}{\pgfqpoint{4.650000in}{3.020000in}}%
\pgfusepath{clip}%
\pgfsetbuttcap%
\pgfsetmiterjoin%
\definecolor{currentfill}{rgb}{1.000000,0.000000,0.000000}%
\pgfsetfillcolor{currentfill}%
\pgfsetlinewidth{0.000000pt}%
\definecolor{currentstroke}{rgb}{0.000000,0.000000,0.000000}%
\pgfsetstrokecolor{currentstroke}%
\pgfsetstrokeopacity{0.000000}%
\pgfsetdash{}{0pt}%
\pgfpathmoveto{\pgfqpoint{2.850541in}{0.500000in}}%
\pgfpathlineto{\pgfqpoint{2.870959in}{0.500000in}}%
\pgfpathlineto{\pgfqpoint{2.870959in}{1.284416in}}%
\pgfpathlineto{\pgfqpoint{2.850541in}{1.284416in}}%
\pgfpathlineto{\pgfqpoint{2.850541in}{0.500000in}}%
\pgfpathclose%
\pgfusepath{fill}%
\end{pgfscope}%
\begin{pgfscope}%
\pgfpathrectangle{\pgfqpoint{0.750000in}{0.500000in}}{\pgfqpoint{4.650000in}{3.020000in}}%
\pgfusepath{clip}%
\pgfsetbuttcap%
\pgfsetmiterjoin%
\definecolor{currentfill}{rgb}{1.000000,0.000000,0.000000}%
\pgfsetfillcolor{currentfill}%
\pgfsetlinewidth{0.000000pt}%
\definecolor{currentstroke}{rgb}{0.000000,0.000000,0.000000}%
\pgfsetstrokecolor{currentstroke}%
\pgfsetstrokeopacity{0.000000}%
\pgfsetdash{}{0pt}%
\pgfpathmoveto{\pgfqpoint{2.870959in}{0.500000in}}%
\pgfpathlineto{\pgfqpoint{2.891377in}{0.500000in}}%
\pgfpathlineto{\pgfqpoint{2.891377in}{0.667342in}}%
\pgfpathlineto{\pgfqpoint{2.870959in}{0.667342in}}%
\pgfpathlineto{\pgfqpoint{2.870959in}{0.500000in}}%
\pgfpathclose%
\pgfusepath{fill}%
\end{pgfscope}%
\begin{pgfscope}%
\pgfpathrectangle{\pgfqpoint{0.750000in}{0.500000in}}{\pgfqpoint{4.650000in}{3.020000in}}%
\pgfusepath{clip}%
\pgfsetbuttcap%
\pgfsetmiterjoin%
\definecolor{currentfill}{rgb}{1.000000,0.000000,0.000000}%
\pgfsetfillcolor{currentfill}%
\pgfsetlinewidth{0.000000pt}%
\definecolor{currentstroke}{rgb}{0.000000,0.000000,0.000000}%
\pgfsetstrokecolor{currentstroke}%
\pgfsetstrokeopacity{0.000000}%
\pgfsetdash{}{0pt}%
\pgfpathmoveto{\pgfqpoint{2.891377in}{0.500000in}}%
\pgfpathlineto{\pgfqpoint{2.911794in}{0.500000in}}%
\pgfpathlineto{\pgfqpoint{2.911794in}{1.357628in}}%
\pgfpathlineto{\pgfqpoint{2.891377in}{1.357628in}}%
\pgfpathlineto{\pgfqpoint{2.891377in}{0.500000in}}%
\pgfpathclose%
\pgfusepath{fill}%
\end{pgfscope}%
\begin{pgfscope}%
\pgfpathrectangle{\pgfqpoint{0.750000in}{0.500000in}}{\pgfqpoint{4.650000in}{3.020000in}}%
\pgfusepath{clip}%
\pgfsetbuttcap%
\pgfsetmiterjoin%
\definecolor{currentfill}{rgb}{1.000000,0.000000,0.000000}%
\pgfsetfillcolor{currentfill}%
\pgfsetlinewidth{0.000000pt}%
\definecolor{currentstroke}{rgb}{0.000000,0.000000,0.000000}%
\pgfsetstrokecolor{currentstroke}%
\pgfsetstrokeopacity{0.000000}%
\pgfsetdash{}{0pt}%
\pgfpathmoveto{\pgfqpoint{2.911794in}{0.500000in}}%
\pgfpathlineto{\pgfqpoint{2.932212in}{0.500000in}}%
\pgfpathlineto{\pgfqpoint{2.932212in}{0.730095in}}%
\pgfpathlineto{\pgfqpoint{2.911794in}{0.730095in}}%
\pgfpathlineto{\pgfqpoint{2.911794in}{0.500000in}}%
\pgfpathclose%
\pgfusepath{fill}%
\end{pgfscope}%
\begin{pgfscope}%
\pgfpathrectangle{\pgfqpoint{0.750000in}{0.500000in}}{\pgfqpoint{4.650000in}{3.020000in}}%
\pgfusepath{clip}%
\pgfsetbuttcap%
\pgfsetmiterjoin%
\definecolor{currentfill}{rgb}{1.000000,0.000000,0.000000}%
\pgfsetfillcolor{currentfill}%
\pgfsetlinewidth{0.000000pt}%
\definecolor{currentstroke}{rgb}{0.000000,0.000000,0.000000}%
\pgfsetstrokecolor{currentstroke}%
\pgfsetstrokeopacity{0.000000}%
\pgfsetdash{}{0pt}%
\pgfpathmoveto{\pgfqpoint{2.932212in}{0.500000in}}%
\pgfpathlineto{\pgfqpoint{2.952630in}{0.500000in}}%
\pgfpathlineto{\pgfqpoint{2.952630in}{0.730095in}}%
\pgfpathlineto{\pgfqpoint{2.932212in}{0.730095in}}%
\pgfpathlineto{\pgfqpoint{2.932212in}{0.500000in}}%
\pgfpathclose%
\pgfusepath{fill}%
\end{pgfscope}%
\begin{pgfscope}%
\pgfpathrectangle{\pgfqpoint{0.750000in}{0.500000in}}{\pgfqpoint{4.650000in}{3.020000in}}%
\pgfusepath{clip}%
\pgfsetbuttcap%
\pgfsetmiterjoin%
\definecolor{currentfill}{rgb}{1.000000,0.000000,0.000000}%
\pgfsetfillcolor{currentfill}%
\pgfsetlinewidth{0.000000pt}%
\definecolor{currentstroke}{rgb}{0.000000,0.000000,0.000000}%
\pgfsetstrokecolor{currentstroke}%
\pgfsetstrokeopacity{0.000000}%
\pgfsetdash{}{0pt}%
\pgfpathmoveto{\pgfqpoint{2.952630in}{0.500000in}}%
\pgfpathlineto{\pgfqpoint{2.973048in}{0.500000in}}%
\pgfpathlineto{\pgfqpoint{2.973048in}{1.796900in}}%
\pgfpathlineto{\pgfqpoint{2.952630in}{1.796900in}}%
\pgfpathlineto{\pgfqpoint{2.952630in}{0.500000in}}%
\pgfpathclose%
\pgfusepath{fill}%
\end{pgfscope}%
\begin{pgfscope}%
\pgfpathrectangle{\pgfqpoint{0.750000in}{0.500000in}}{\pgfqpoint{4.650000in}{3.020000in}}%
\pgfusepath{clip}%
\pgfsetbuttcap%
\pgfsetmiterjoin%
\definecolor{currentfill}{rgb}{1.000000,0.000000,0.000000}%
\pgfsetfillcolor{currentfill}%
\pgfsetlinewidth{0.000000pt}%
\definecolor{currentstroke}{rgb}{0.000000,0.000000,0.000000}%
\pgfsetstrokecolor{currentstroke}%
\pgfsetstrokeopacity{0.000000}%
\pgfsetdash{}{0pt}%
\pgfpathmoveto{\pgfqpoint{2.973048in}{0.500000in}}%
\pgfpathlineto{\pgfqpoint{2.993466in}{0.500000in}}%
\pgfpathlineto{\pgfqpoint{2.993466in}{0.751013in}}%
\pgfpathlineto{\pgfqpoint{2.973048in}{0.751013in}}%
\pgfpathlineto{\pgfqpoint{2.973048in}{0.500000in}}%
\pgfpathclose%
\pgfusepath{fill}%
\end{pgfscope}%
\begin{pgfscope}%
\pgfpathrectangle{\pgfqpoint{0.750000in}{0.500000in}}{\pgfqpoint{4.650000in}{3.020000in}}%
\pgfusepath{clip}%
\pgfsetbuttcap%
\pgfsetmiterjoin%
\definecolor{currentfill}{rgb}{1.000000,0.000000,0.000000}%
\pgfsetfillcolor{currentfill}%
\pgfsetlinewidth{0.000000pt}%
\definecolor{currentstroke}{rgb}{0.000000,0.000000,0.000000}%
\pgfsetstrokecolor{currentstroke}%
\pgfsetstrokeopacity{0.000000}%
\pgfsetdash{}{0pt}%
\pgfpathmoveto{\pgfqpoint{2.993466in}{0.500000in}}%
\pgfpathlineto{\pgfqpoint{3.013883in}{0.500000in}}%
\pgfpathlineto{\pgfqpoint{3.013883in}{1.932866in}}%
\pgfpathlineto{\pgfqpoint{2.993466in}{1.932866in}}%
\pgfpathlineto{\pgfqpoint{2.993466in}{0.500000in}}%
\pgfpathclose%
\pgfusepath{fill}%
\end{pgfscope}%
\begin{pgfscope}%
\pgfpathrectangle{\pgfqpoint{0.750000in}{0.500000in}}{\pgfqpoint{4.650000in}{3.020000in}}%
\pgfusepath{clip}%
\pgfsetbuttcap%
\pgfsetmiterjoin%
\definecolor{currentfill}{rgb}{1.000000,0.000000,0.000000}%
\pgfsetfillcolor{currentfill}%
\pgfsetlinewidth{0.000000pt}%
\definecolor{currentstroke}{rgb}{0.000000,0.000000,0.000000}%
\pgfsetstrokecolor{currentstroke}%
\pgfsetstrokeopacity{0.000000}%
\pgfsetdash{}{0pt}%
\pgfpathmoveto{\pgfqpoint{3.013883in}{0.500000in}}%
\pgfpathlineto{\pgfqpoint{3.034301in}{0.500000in}}%
\pgfpathlineto{\pgfqpoint{3.034301in}{0.751013in}}%
\pgfpathlineto{\pgfqpoint{3.013883in}{0.751013in}}%
\pgfpathlineto{\pgfqpoint{3.013883in}{0.500000in}}%
\pgfpathclose%
\pgfusepath{fill}%
\end{pgfscope}%
\begin{pgfscope}%
\pgfpathrectangle{\pgfqpoint{0.750000in}{0.500000in}}{\pgfqpoint{4.650000in}{3.020000in}}%
\pgfusepath{clip}%
\pgfsetbuttcap%
\pgfsetmiterjoin%
\definecolor{currentfill}{rgb}{1.000000,0.000000,0.000000}%
\pgfsetfillcolor{currentfill}%
\pgfsetlinewidth{0.000000pt}%
\definecolor{currentstroke}{rgb}{0.000000,0.000000,0.000000}%
\pgfsetstrokecolor{currentstroke}%
\pgfsetstrokeopacity{0.000000}%
\pgfsetdash{}{0pt}%
\pgfpathmoveto{\pgfqpoint{3.034301in}{0.500000in}}%
\pgfpathlineto{\pgfqpoint{3.054719in}{0.500000in}}%
\pgfpathlineto{\pgfqpoint{3.054719in}{0.792848in}}%
\pgfpathlineto{\pgfqpoint{3.034301in}{0.792848in}}%
\pgfpathlineto{\pgfqpoint{3.034301in}{0.500000in}}%
\pgfpathclose%
\pgfusepath{fill}%
\end{pgfscope}%
\begin{pgfscope}%
\pgfpathrectangle{\pgfqpoint{0.750000in}{0.500000in}}{\pgfqpoint{4.650000in}{3.020000in}}%
\pgfusepath{clip}%
\pgfsetbuttcap%
\pgfsetmiterjoin%
\definecolor{currentfill}{rgb}{1.000000,0.000000,0.000000}%
\pgfsetfillcolor{currentfill}%
\pgfsetlinewidth{0.000000pt}%
\definecolor{currentstroke}{rgb}{0.000000,0.000000,0.000000}%
\pgfsetstrokecolor{currentstroke}%
\pgfsetstrokeopacity{0.000000}%
\pgfsetdash{}{0pt}%
\pgfpathmoveto{\pgfqpoint{3.054719in}{0.500000in}}%
\pgfpathlineto{\pgfqpoint{3.075137in}{0.500000in}}%
\pgfpathlineto{\pgfqpoint{3.075137in}{2.288468in}}%
\pgfpathlineto{\pgfqpoint{3.054719in}{2.288468in}}%
\pgfpathlineto{\pgfqpoint{3.054719in}{0.500000in}}%
\pgfpathclose%
\pgfusepath{fill}%
\end{pgfscope}%
\begin{pgfscope}%
\pgfpathrectangle{\pgfqpoint{0.750000in}{0.500000in}}{\pgfqpoint{4.650000in}{3.020000in}}%
\pgfusepath{clip}%
\pgfsetbuttcap%
\pgfsetmiterjoin%
\definecolor{currentfill}{rgb}{1.000000,0.000000,0.000000}%
\pgfsetfillcolor{currentfill}%
\pgfsetlinewidth{0.000000pt}%
\definecolor{currentstroke}{rgb}{0.000000,0.000000,0.000000}%
\pgfsetstrokecolor{currentstroke}%
\pgfsetstrokeopacity{0.000000}%
\pgfsetdash{}{0pt}%
\pgfpathmoveto{\pgfqpoint{3.075137in}{0.500000in}}%
\pgfpathlineto{\pgfqpoint{3.095555in}{0.500000in}}%
\pgfpathlineto{\pgfqpoint{3.095555in}{0.782390in}}%
\pgfpathlineto{\pgfqpoint{3.075137in}{0.782390in}}%
\pgfpathlineto{\pgfqpoint{3.075137in}{0.500000in}}%
\pgfpathclose%
\pgfusepath{fill}%
\end{pgfscope}%
\begin{pgfscope}%
\pgfpathrectangle{\pgfqpoint{0.750000in}{0.500000in}}{\pgfqpoint{4.650000in}{3.020000in}}%
\pgfusepath{clip}%
\pgfsetbuttcap%
\pgfsetmiterjoin%
\definecolor{currentfill}{rgb}{1.000000,0.000000,0.000000}%
\pgfsetfillcolor{currentfill}%
\pgfsetlinewidth{0.000000pt}%
\definecolor{currentstroke}{rgb}{0.000000,0.000000,0.000000}%
\pgfsetstrokecolor{currentstroke}%
\pgfsetstrokeopacity{0.000000}%
\pgfsetdash{}{0pt}%
\pgfpathmoveto{\pgfqpoint{3.095555in}{0.500000in}}%
\pgfpathlineto{\pgfqpoint{3.115972in}{0.500000in}}%
\pgfpathlineto{\pgfqpoint{3.115972in}{2.319844in}}%
\pgfpathlineto{\pgfqpoint{3.095555in}{2.319844in}}%
\pgfpathlineto{\pgfqpoint{3.095555in}{0.500000in}}%
\pgfpathclose%
\pgfusepath{fill}%
\end{pgfscope}%
\begin{pgfscope}%
\pgfpathrectangle{\pgfqpoint{0.750000in}{0.500000in}}{\pgfqpoint{4.650000in}{3.020000in}}%
\pgfusepath{clip}%
\pgfsetbuttcap%
\pgfsetmiterjoin%
\definecolor{currentfill}{rgb}{1.000000,0.000000,0.000000}%
\pgfsetfillcolor{currentfill}%
\pgfsetlinewidth{0.000000pt}%
\definecolor{currentstroke}{rgb}{0.000000,0.000000,0.000000}%
\pgfsetstrokecolor{currentstroke}%
\pgfsetstrokeopacity{0.000000}%
\pgfsetdash{}{0pt}%
\pgfpathmoveto{\pgfqpoint{3.115972in}{0.500000in}}%
\pgfpathlineto{\pgfqpoint{3.136390in}{0.500000in}}%
\pgfpathlineto{\pgfqpoint{3.136390in}{0.813766in}}%
\pgfpathlineto{\pgfqpoint{3.115972in}{0.813766in}}%
\pgfpathlineto{\pgfqpoint{3.115972in}{0.500000in}}%
\pgfpathclose%
\pgfusepath{fill}%
\end{pgfscope}%
\begin{pgfscope}%
\pgfpathrectangle{\pgfqpoint{0.750000in}{0.500000in}}{\pgfqpoint{4.650000in}{3.020000in}}%
\pgfusepath{clip}%
\pgfsetbuttcap%
\pgfsetmiterjoin%
\definecolor{currentfill}{rgb}{1.000000,0.000000,0.000000}%
\pgfsetfillcolor{currentfill}%
\pgfsetlinewidth{0.000000pt}%
\definecolor{currentstroke}{rgb}{0.000000,0.000000,0.000000}%
\pgfsetstrokecolor{currentstroke}%
\pgfsetstrokeopacity{0.000000}%
\pgfsetdash{}{0pt}%
\pgfpathmoveto{\pgfqpoint{3.136390in}{0.500000in}}%
\pgfpathlineto{\pgfqpoint{3.156808in}{0.500000in}}%
\pgfpathlineto{\pgfqpoint{3.156808in}{0.949732in}}%
\pgfpathlineto{\pgfqpoint{3.136390in}{0.949732in}}%
\pgfpathlineto{\pgfqpoint{3.136390in}{0.500000in}}%
\pgfpathclose%
\pgfusepath{fill}%
\end{pgfscope}%
\begin{pgfscope}%
\pgfpathrectangle{\pgfqpoint{0.750000in}{0.500000in}}{\pgfqpoint{4.650000in}{3.020000in}}%
\pgfusepath{clip}%
\pgfsetbuttcap%
\pgfsetmiterjoin%
\definecolor{currentfill}{rgb}{1.000000,0.000000,0.000000}%
\pgfsetfillcolor{currentfill}%
\pgfsetlinewidth{0.000000pt}%
\definecolor{currentstroke}{rgb}{0.000000,0.000000,0.000000}%
\pgfsetstrokecolor{currentstroke}%
\pgfsetstrokeopacity{0.000000}%
\pgfsetdash{}{0pt}%
\pgfpathmoveto{\pgfqpoint{3.156808in}{0.500000in}}%
\pgfpathlineto{\pgfqpoint{3.177226in}{0.500000in}}%
\pgfpathlineto{\pgfqpoint{3.177226in}{2.403515in}}%
\pgfpathlineto{\pgfqpoint{3.156808in}{2.403515in}}%
\pgfpathlineto{\pgfqpoint{3.156808in}{0.500000in}}%
\pgfpathclose%
\pgfusepath{fill}%
\end{pgfscope}%
\begin{pgfscope}%
\pgfpathrectangle{\pgfqpoint{0.750000in}{0.500000in}}{\pgfqpoint{4.650000in}{3.020000in}}%
\pgfusepath{clip}%
\pgfsetbuttcap%
\pgfsetmiterjoin%
\definecolor{currentfill}{rgb}{1.000000,0.000000,0.000000}%
\pgfsetfillcolor{currentfill}%
\pgfsetlinewidth{0.000000pt}%
\definecolor{currentstroke}{rgb}{0.000000,0.000000,0.000000}%
\pgfsetstrokecolor{currentstroke}%
\pgfsetstrokeopacity{0.000000}%
\pgfsetdash{}{0pt}%
\pgfpathmoveto{\pgfqpoint{3.177226in}{0.500000in}}%
\pgfpathlineto{\pgfqpoint{3.197644in}{0.500000in}}%
\pgfpathlineto{\pgfqpoint{3.197644in}{0.939273in}}%
\pgfpathlineto{\pgfqpoint{3.177226in}{0.939273in}}%
\pgfpathlineto{\pgfqpoint{3.177226in}{0.500000in}}%
\pgfpathclose%
\pgfusepath{fill}%
\end{pgfscope}%
\begin{pgfscope}%
\pgfpathrectangle{\pgfqpoint{0.750000in}{0.500000in}}{\pgfqpoint{4.650000in}{3.020000in}}%
\pgfusepath{clip}%
\pgfsetbuttcap%
\pgfsetmiterjoin%
\definecolor{currentfill}{rgb}{1.000000,0.000000,0.000000}%
\pgfsetfillcolor{currentfill}%
\pgfsetlinewidth{0.000000pt}%
\definecolor{currentstroke}{rgb}{0.000000,0.000000,0.000000}%
\pgfsetstrokecolor{currentstroke}%
\pgfsetstrokeopacity{0.000000}%
\pgfsetdash{}{0pt}%
\pgfpathmoveto{\pgfqpoint{3.197644in}{0.500000in}}%
\pgfpathlineto{\pgfqpoint{3.218061in}{0.500000in}}%
\pgfpathlineto{\pgfqpoint{3.218061in}{2.351221in}}%
\pgfpathlineto{\pgfqpoint{3.197644in}{2.351221in}}%
\pgfpathlineto{\pgfqpoint{3.197644in}{0.500000in}}%
\pgfpathclose%
\pgfusepath{fill}%
\end{pgfscope}%
\begin{pgfscope}%
\pgfpathrectangle{\pgfqpoint{0.750000in}{0.500000in}}{\pgfqpoint{4.650000in}{3.020000in}}%
\pgfusepath{clip}%
\pgfsetbuttcap%
\pgfsetmiterjoin%
\definecolor{currentfill}{rgb}{1.000000,0.000000,0.000000}%
\pgfsetfillcolor{currentfill}%
\pgfsetlinewidth{0.000000pt}%
\definecolor{currentstroke}{rgb}{0.000000,0.000000,0.000000}%
\pgfsetstrokecolor{currentstroke}%
\pgfsetstrokeopacity{0.000000}%
\pgfsetdash{}{0pt}%
\pgfpathmoveto{\pgfqpoint{3.218061in}{0.500000in}}%
\pgfpathlineto{\pgfqpoint{3.238479in}{0.500000in}}%
\pgfpathlineto{\pgfqpoint{3.238479in}{0.907896in}}%
\pgfpathlineto{\pgfqpoint{3.218061in}{0.907896in}}%
\pgfpathlineto{\pgfqpoint{3.218061in}{0.500000in}}%
\pgfpathclose%
\pgfusepath{fill}%
\end{pgfscope}%
\begin{pgfscope}%
\pgfpathrectangle{\pgfqpoint{0.750000in}{0.500000in}}{\pgfqpoint{4.650000in}{3.020000in}}%
\pgfusepath{clip}%
\pgfsetbuttcap%
\pgfsetmiterjoin%
\definecolor{currentfill}{rgb}{1.000000,0.000000,0.000000}%
\pgfsetfillcolor{currentfill}%
\pgfsetlinewidth{0.000000pt}%
\definecolor{currentstroke}{rgb}{0.000000,0.000000,0.000000}%
\pgfsetstrokecolor{currentstroke}%
\pgfsetstrokeopacity{0.000000}%
\pgfsetdash{}{0pt}%
\pgfpathmoveto{\pgfqpoint{3.238479in}{0.500000in}}%
\pgfpathlineto{\pgfqpoint{3.258897in}{0.500000in}}%
\pgfpathlineto{\pgfqpoint{3.258897in}{2.769576in}}%
\pgfpathlineto{\pgfqpoint{3.238479in}{2.769576in}}%
\pgfpathlineto{\pgfqpoint{3.238479in}{0.500000in}}%
\pgfpathclose%
\pgfusepath{fill}%
\end{pgfscope}%
\begin{pgfscope}%
\pgfpathrectangle{\pgfqpoint{0.750000in}{0.500000in}}{\pgfqpoint{4.650000in}{3.020000in}}%
\pgfusepath{clip}%
\pgfsetbuttcap%
\pgfsetmiterjoin%
\definecolor{currentfill}{rgb}{1.000000,0.000000,0.000000}%
\pgfsetfillcolor{currentfill}%
\pgfsetlinewidth{0.000000pt}%
\definecolor{currentstroke}{rgb}{0.000000,0.000000,0.000000}%
\pgfsetstrokecolor{currentstroke}%
\pgfsetstrokeopacity{0.000000}%
\pgfsetdash{}{0pt}%
\pgfpathmoveto{\pgfqpoint{3.258897in}{0.500000in}}%
\pgfpathlineto{\pgfqpoint{3.279315in}{0.500000in}}%
\pgfpathlineto{\pgfqpoint{3.279315in}{1.294874in}}%
\pgfpathlineto{\pgfqpoint{3.258897in}{1.294874in}}%
\pgfpathlineto{\pgfqpoint{3.258897in}{0.500000in}}%
\pgfpathclose%
\pgfusepath{fill}%
\end{pgfscope}%
\begin{pgfscope}%
\pgfpathrectangle{\pgfqpoint{0.750000in}{0.500000in}}{\pgfqpoint{4.650000in}{3.020000in}}%
\pgfusepath{clip}%
\pgfsetbuttcap%
\pgfsetmiterjoin%
\definecolor{currentfill}{rgb}{1.000000,0.000000,0.000000}%
\pgfsetfillcolor{currentfill}%
\pgfsetlinewidth{0.000000pt}%
\definecolor{currentstroke}{rgb}{0.000000,0.000000,0.000000}%
\pgfsetstrokecolor{currentstroke}%
\pgfsetstrokeopacity{0.000000}%
\pgfsetdash{}{0pt}%
\pgfpathmoveto{\pgfqpoint{3.279315in}{0.500000in}}%
\pgfpathlineto{\pgfqpoint{3.299733in}{0.500000in}}%
\pgfpathlineto{\pgfqpoint{3.299733in}{0.897437in}}%
\pgfpathlineto{\pgfqpoint{3.279315in}{0.897437in}}%
\pgfpathlineto{\pgfqpoint{3.279315in}{0.500000in}}%
\pgfpathclose%
\pgfusepath{fill}%
\end{pgfscope}%
\begin{pgfscope}%
\pgfpathrectangle{\pgfqpoint{0.750000in}{0.500000in}}{\pgfqpoint{4.650000in}{3.020000in}}%
\pgfusepath{clip}%
\pgfsetbuttcap%
\pgfsetmiterjoin%
\definecolor{currentfill}{rgb}{1.000000,0.000000,0.000000}%
\pgfsetfillcolor{currentfill}%
\pgfsetlinewidth{0.000000pt}%
\definecolor{currentstroke}{rgb}{0.000000,0.000000,0.000000}%
\pgfsetstrokecolor{currentstroke}%
\pgfsetstrokeopacity{0.000000}%
\pgfsetdash{}{0pt}%
\pgfpathmoveto{\pgfqpoint{3.299733in}{0.500000in}}%
\pgfpathlineto{\pgfqpoint{3.320150in}{0.500000in}}%
\pgfpathlineto{\pgfqpoint{3.320150in}{3.020589in}}%
\pgfpathlineto{\pgfqpoint{3.299733in}{3.020589in}}%
\pgfpathlineto{\pgfqpoint{3.299733in}{0.500000in}}%
\pgfpathclose%
\pgfusepath{fill}%
\end{pgfscope}%
\begin{pgfscope}%
\pgfpathrectangle{\pgfqpoint{0.750000in}{0.500000in}}{\pgfqpoint{4.650000in}{3.020000in}}%
\pgfusepath{clip}%
\pgfsetbuttcap%
\pgfsetmiterjoin%
\definecolor{currentfill}{rgb}{1.000000,0.000000,0.000000}%
\pgfsetfillcolor{currentfill}%
\pgfsetlinewidth{0.000000pt}%
\definecolor{currentstroke}{rgb}{0.000000,0.000000,0.000000}%
\pgfsetstrokecolor{currentstroke}%
\pgfsetstrokeopacity{0.000000}%
\pgfsetdash{}{0pt}%
\pgfpathmoveto{\pgfqpoint{3.320150in}{0.500000in}}%
\pgfpathlineto{\pgfqpoint{3.340568in}{0.500000in}}%
\pgfpathlineto{\pgfqpoint{3.340568in}{1.002026in}}%
\pgfpathlineto{\pgfqpoint{3.320150in}{1.002026in}}%
\pgfpathlineto{\pgfqpoint{3.320150in}{0.500000in}}%
\pgfpathclose%
\pgfusepath{fill}%
\end{pgfscope}%
\begin{pgfscope}%
\pgfpathrectangle{\pgfqpoint{0.750000in}{0.500000in}}{\pgfqpoint{4.650000in}{3.020000in}}%
\pgfusepath{clip}%
\pgfsetbuttcap%
\pgfsetmiterjoin%
\definecolor{currentfill}{rgb}{1.000000,0.000000,0.000000}%
\pgfsetfillcolor{currentfill}%
\pgfsetlinewidth{0.000000pt}%
\definecolor{currentstroke}{rgb}{0.000000,0.000000,0.000000}%
\pgfsetstrokecolor{currentstroke}%
\pgfsetstrokeopacity{0.000000}%
\pgfsetdash{}{0pt}%
\pgfpathmoveto{\pgfqpoint{3.340568in}{0.500000in}}%
\pgfpathlineto{\pgfqpoint{3.360986in}{0.500000in}}%
\pgfpathlineto{\pgfqpoint{3.360986in}{3.376190in}}%
\pgfpathlineto{\pgfqpoint{3.340568in}{3.376190in}}%
\pgfpathlineto{\pgfqpoint{3.340568in}{0.500000in}}%
\pgfpathclose%
\pgfusepath{fill}%
\end{pgfscope}%
\begin{pgfscope}%
\pgfpathrectangle{\pgfqpoint{0.750000in}{0.500000in}}{\pgfqpoint{4.650000in}{3.020000in}}%
\pgfusepath{clip}%
\pgfsetbuttcap%
\pgfsetmiterjoin%
\definecolor{currentfill}{rgb}{1.000000,0.000000,0.000000}%
\pgfsetfillcolor{currentfill}%
\pgfsetlinewidth{0.000000pt}%
\definecolor{currentstroke}{rgb}{0.000000,0.000000,0.000000}%
\pgfsetstrokecolor{currentstroke}%
\pgfsetstrokeopacity{0.000000}%
\pgfsetdash{}{0pt}%
\pgfpathmoveto{\pgfqpoint{3.360986in}{0.500000in}}%
\pgfpathlineto{\pgfqpoint{3.381404in}{0.500000in}}%
\pgfpathlineto{\pgfqpoint{3.381404in}{1.148450in}}%
\pgfpathlineto{\pgfqpoint{3.360986in}{1.148450in}}%
\pgfpathlineto{\pgfqpoint{3.360986in}{0.500000in}}%
\pgfpathclose%
\pgfusepath{fill}%
\end{pgfscope}%
\begin{pgfscope}%
\pgfpathrectangle{\pgfqpoint{0.750000in}{0.500000in}}{\pgfqpoint{4.650000in}{3.020000in}}%
\pgfusepath{clip}%
\pgfsetbuttcap%
\pgfsetmiterjoin%
\definecolor{currentfill}{rgb}{1.000000,0.000000,0.000000}%
\pgfsetfillcolor{currentfill}%
\pgfsetlinewidth{0.000000pt}%
\definecolor{currentstroke}{rgb}{0.000000,0.000000,0.000000}%
\pgfsetstrokecolor{currentstroke}%
\pgfsetstrokeopacity{0.000000}%
\pgfsetdash{}{0pt}%
\pgfpathmoveto{\pgfqpoint{3.381404in}{0.500000in}}%
\pgfpathlineto{\pgfqpoint{3.401822in}{0.500000in}}%
\pgfpathlineto{\pgfqpoint{3.401822in}{0.981108in}}%
\pgfpathlineto{\pgfqpoint{3.381404in}{0.981108in}}%
\pgfpathlineto{\pgfqpoint{3.381404in}{0.500000in}}%
\pgfpathclose%
\pgfusepath{fill}%
\end{pgfscope}%
\begin{pgfscope}%
\pgfpathrectangle{\pgfqpoint{0.750000in}{0.500000in}}{\pgfqpoint{4.650000in}{3.020000in}}%
\pgfusepath{clip}%
\pgfsetbuttcap%
\pgfsetmiterjoin%
\definecolor{currentfill}{rgb}{1.000000,0.000000,0.000000}%
\pgfsetfillcolor{currentfill}%
\pgfsetlinewidth{0.000000pt}%
\definecolor{currentstroke}{rgb}{0.000000,0.000000,0.000000}%
\pgfsetstrokecolor{currentstroke}%
\pgfsetstrokeopacity{0.000000}%
\pgfsetdash{}{0pt}%
\pgfpathmoveto{\pgfqpoint{3.401822in}{0.500000in}}%
\pgfpathlineto{\pgfqpoint{3.422239in}{0.500000in}}%
\pgfpathlineto{\pgfqpoint{3.422239in}{2.664987in}}%
\pgfpathlineto{\pgfqpoint{3.401822in}{2.664987in}}%
\pgfpathlineto{\pgfqpoint{3.401822in}{0.500000in}}%
\pgfpathclose%
\pgfusepath{fill}%
\end{pgfscope}%
\begin{pgfscope}%
\pgfpathrectangle{\pgfqpoint{0.750000in}{0.500000in}}{\pgfqpoint{4.650000in}{3.020000in}}%
\pgfusepath{clip}%
\pgfsetbuttcap%
\pgfsetmiterjoin%
\definecolor{currentfill}{rgb}{1.000000,0.000000,0.000000}%
\pgfsetfillcolor{currentfill}%
\pgfsetlinewidth{0.000000pt}%
\definecolor{currentstroke}{rgb}{0.000000,0.000000,0.000000}%
\pgfsetstrokecolor{currentstroke}%
\pgfsetstrokeopacity{0.000000}%
\pgfsetdash{}{0pt}%
\pgfpathmoveto{\pgfqpoint{3.422239in}{0.500000in}}%
\pgfpathlineto{\pgfqpoint{3.442657in}{0.500000in}}%
\pgfpathlineto{\pgfqpoint{3.442657in}{0.949732in}}%
\pgfpathlineto{\pgfqpoint{3.422239in}{0.949732in}}%
\pgfpathlineto{\pgfqpoint{3.422239in}{0.500000in}}%
\pgfpathclose%
\pgfusepath{fill}%
\end{pgfscope}%
\begin{pgfscope}%
\pgfpathrectangle{\pgfqpoint{0.750000in}{0.500000in}}{\pgfqpoint{4.650000in}{3.020000in}}%
\pgfusepath{clip}%
\pgfsetbuttcap%
\pgfsetmiterjoin%
\definecolor{currentfill}{rgb}{1.000000,0.000000,0.000000}%
\pgfsetfillcolor{currentfill}%
\pgfsetlinewidth{0.000000pt}%
\definecolor{currentstroke}{rgb}{0.000000,0.000000,0.000000}%
\pgfsetstrokecolor{currentstroke}%
\pgfsetstrokeopacity{0.000000}%
\pgfsetdash{}{0pt}%
\pgfpathmoveto{\pgfqpoint{3.442657in}{0.500000in}}%
\pgfpathlineto{\pgfqpoint{3.463075in}{0.500000in}}%
\pgfpathlineto{\pgfqpoint{3.463075in}{3.072883in}}%
\pgfpathlineto{\pgfqpoint{3.442657in}{3.072883in}}%
\pgfpathlineto{\pgfqpoint{3.442657in}{0.500000in}}%
\pgfpathclose%
\pgfusepath{fill}%
\end{pgfscope}%
\begin{pgfscope}%
\pgfpathrectangle{\pgfqpoint{0.750000in}{0.500000in}}{\pgfqpoint{4.650000in}{3.020000in}}%
\pgfusepath{clip}%
\pgfsetbuttcap%
\pgfsetmiterjoin%
\definecolor{currentfill}{rgb}{1.000000,0.000000,0.000000}%
\pgfsetfillcolor{currentfill}%
\pgfsetlinewidth{0.000000pt}%
\definecolor{currentstroke}{rgb}{0.000000,0.000000,0.000000}%
\pgfsetstrokecolor{currentstroke}%
\pgfsetstrokeopacity{0.000000}%
\pgfsetdash{}{0pt}%
\pgfpathmoveto{\pgfqpoint{3.463075in}{0.500000in}}%
\pgfpathlineto{\pgfqpoint{3.483493in}{0.500000in}}%
\pgfpathlineto{\pgfqpoint{3.483493in}{0.949732in}}%
\pgfpathlineto{\pgfqpoint{3.463075in}{0.949732in}}%
\pgfpathlineto{\pgfqpoint{3.463075in}{0.500000in}}%
\pgfpathclose%
\pgfusepath{fill}%
\end{pgfscope}%
\begin{pgfscope}%
\pgfpathrectangle{\pgfqpoint{0.750000in}{0.500000in}}{\pgfqpoint{4.650000in}{3.020000in}}%
\pgfusepath{clip}%
\pgfsetbuttcap%
\pgfsetmiterjoin%
\definecolor{currentfill}{rgb}{1.000000,0.000000,0.000000}%
\pgfsetfillcolor{currentfill}%
\pgfsetlinewidth{0.000000pt}%
\definecolor{currentstroke}{rgb}{0.000000,0.000000,0.000000}%
\pgfsetstrokecolor{currentstroke}%
\pgfsetstrokeopacity{0.000000}%
\pgfsetdash{}{0pt}%
\pgfpathmoveto{\pgfqpoint{3.483493in}{0.500000in}}%
\pgfpathlineto{\pgfqpoint{3.503911in}{0.500000in}}%
\pgfpathlineto{\pgfqpoint{3.503911in}{0.970649in}}%
\pgfpathlineto{\pgfqpoint{3.483493in}{0.970649in}}%
\pgfpathlineto{\pgfqpoint{3.483493in}{0.500000in}}%
\pgfpathclose%
\pgfusepath{fill}%
\end{pgfscope}%
\begin{pgfscope}%
\pgfpathrectangle{\pgfqpoint{0.750000in}{0.500000in}}{\pgfqpoint{4.650000in}{3.020000in}}%
\pgfusepath{clip}%
\pgfsetbuttcap%
\pgfsetmiterjoin%
\definecolor{currentfill}{rgb}{1.000000,0.000000,0.000000}%
\pgfsetfillcolor{currentfill}%
\pgfsetlinewidth{0.000000pt}%
\definecolor{currentstroke}{rgb}{0.000000,0.000000,0.000000}%
\pgfsetstrokecolor{currentstroke}%
\pgfsetstrokeopacity{0.000000}%
\pgfsetdash{}{0pt}%
\pgfpathmoveto{\pgfqpoint{3.503911in}{0.500000in}}%
\pgfpathlineto{\pgfqpoint{3.524329in}{0.500000in}}%
\pgfpathlineto{\pgfqpoint{3.524329in}{2.476727in}}%
\pgfpathlineto{\pgfqpoint{3.503911in}{2.476727in}}%
\pgfpathlineto{\pgfqpoint{3.503911in}{0.500000in}}%
\pgfpathclose%
\pgfusepath{fill}%
\end{pgfscope}%
\begin{pgfscope}%
\pgfpathrectangle{\pgfqpoint{0.750000in}{0.500000in}}{\pgfqpoint{4.650000in}{3.020000in}}%
\pgfusepath{clip}%
\pgfsetbuttcap%
\pgfsetmiterjoin%
\definecolor{currentfill}{rgb}{1.000000,0.000000,0.000000}%
\pgfsetfillcolor{currentfill}%
\pgfsetlinewidth{0.000000pt}%
\definecolor{currentstroke}{rgb}{0.000000,0.000000,0.000000}%
\pgfsetstrokecolor{currentstroke}%
\pgfsetstrokeopacity{0.000000}%
\pgfsetdash{}{0pt}%
\pgfpathmoveto{\pgfqpoint{3.524329in}{0.500000in}}%
\pgfpathlineto{\pgfqpoint{3.544746in}{0.500000in}}%
\pgfpathlineto{\pgfqpoint{3.544746in}{0.886978in}}%
\pgfpathlineto{\pgfqpoint{3.524329in}{0.886978in}}%
\pgfpathlineto{\pgfqpoint{3.524329in}{0.500000in}}%
\pgfpathclose%
\pgfusepath{fill}%
\end{pgfscope}%
\begin{pgfscope}%
\pgfpathrectangle{\pgfqpoint{0.750000in}{0.500000in}}{\pgfqpoint{4.650000in}{3.020000in}}%
\pgfusepath{clip}%
\pgfsetbuttcap%
\pgfsetmiterjoin%
\definecolor{currentfill}{rgb}{1.000000,0.000000,0.000000}%
\pgfsetfillcolor{currentfill}%
\pgfsetlinewidth{0.000000pt}%
\definecolor{currentstroke}{rgb}{0.000000,0.000000,0.000000}%
\pgfsetstrokecolor{currentstroke}%
\pgfsetstrokeopacity{0.000000}%
\pgfsetdash{}{0pt}%
\pgfpathmoveto{\pgfqpoint{3.544746in}{0.500000in}}%
\pgfpathlineto{\pgfqpoint{3.565164in}{0.500000in}}%
\pgfpathlineto{\pgfqpoint{3.565164in}{2.518563in}}%
\pgfpathlineto{\pgfqpoint{3.544746in}{2.518563in}}%
\pgfpathlineto{\pgfqpoint{3.544746in}{0.500000in}}%
\pgfpathclose%
\pgfusepath{fill}%
\end{pgfscope}%
\begin{pgfscope}%
\pgfpathrectangle{\pgfqpoint{0.750000in}{0.500000in}}{\pgfqpoint{4.650000in}{3.020000in}}%
\pgfusepath{clip}%
\pgfsetbuttcap%
\pgfsetmiterjoin%
\definecolor{currentfill}{rgb}{1.000000,0.000000,0.000000}%
\pgfsetfillcolor{currentfill}%
\pgfsetlinewidth{0.000000pt}%
\definecolor{currentstroke}{rgb}{0.000000,0.000000,0.000000}%
\pgfsetstrokecolor{currentstroke}%
\pgfsetstrokeopacity{0.000000}%
\pgfsetdash{}{0pt}%
\pgfpathmoveto{\pgfqpoint{3.565164in}{0.500000in}}%
\pgfpathlineto{\pgfqpoint{3.585582in}{0.500000in}}%
\pgfpathlineto{\pgfqpoint{3.585582in}{0.845143in}}%
\pgfpathlineto{\pgfqpoint{3.565164in}{0.845143in}}%
\pgfpathlineto{\pgfqpoint{3.565164in}{0.500000in}}%
\pgfpathclose%
\pgfusepath{fill}%
\end{pgfscope}%
\begin{pgfscope}%
\pgfpathrectangle{\pgfqpoint{0.750000in}{0.500000in}}{\pgfqpoint{4.650000in}{3.020000in}}%
\pgfusepath{clip}%
\pgfsetbuttcap%
\pgfsetmiterjoin%
\definecolor{currentfill}{rgb}{1.000000,0.000000,0.000000}%
\pgfsetfillcolor{currentfill}%
\pgfsetlinewidth{0.000000pt}%
\definecolor{currentstroke}{rgb}{0.000000,0.000000,0.000000}%
\pgfsetstrokecolor{currentstroke}%
\pgfsetstrokeopacity{0.000000}%
\pgfsetdash{}{0pt}%
\pgfpathmoveto{\pgfqpoint{3.585582in}{0.500000in}}%
\pgfpathlineto{\pgfqpoint{3.606000in}{0.500000in}}%
\pgfpathlineto{\pgfqpoint{3.606000in}{0.813766in}}%
\pgfpathlineto{\pgfqpoint{3.585582in}{0.813766in}}%
\pgfpathlineto{\pgfqpoint{3.585582in}{0.500000in}}%
\pgfpathclose%
\pgfusepath{fill}%
\end{pgfscope}%
\begin{pgfscope}%
\pgfpathrectangle{\pgfqpoint{0.750000in}{0.500000in}}{\pgfqpoint{4.650000in}{3.020000in}}%
\pgfusepath{clip}%
\pgfsetbuttcap%
\pgfsetmiterjoin%
\definecolor{currentfill}{rgb}{1.000000,0.000000,0.000000}%
\pgfsetfillcolor{currentfill}%
\pgfsetlinewidth{0.000000pt}%
\definecolor{currentstroke}{rgb}{0.000000,0.000000,0.000000}%
\pgfsetstrokecolor{currentstroke}%
\pgfsetstrokeopacity{0.000000}%
\pgfsetdash{}{0pt}%
\pgfpathmoveto{\pgfqpoint{3.606000in}{0.500000in}}%
\pgfpathlineto{\pgfqpoint{3.626418in}{0.500000in}}%
\pgfpathlineto{\pgfqpoint{3.626418in}{2.142043in}}%
\pgfpathlineto{\pgfqpoint{3.606000in}{2.142043in}}%
\pgfpathlineto{\pgfqpoint{3.606000in}{0.500000in}}%
\pgfpathclose%
\pgfusepath{fill}%
\end{pgfscope}%
\begin{pgfscope}%
\pgfpathrectangle{\pgfqpoint{0.750000in}{0.500000in}}{\pgfqpoint{4.650000in}{3.020000in}}%
\pgfusepath{clip}%
\pgfsetbuttcap%
\pgfsetmiterjoin%
\definecolor{currentfill}{rgb}{1.000000,0.000000,0.000000}%
\pgfsetfillcolor{currentfill}%
\pgfsetlinewidth{0.000000pt}%
\definecolor{currentstroke}{rgb}{0.000000,0.000000,0.000000}%
\pgfsetstrokecolor{currentstroke}%
\pgfsetstrokeopacity{0.000000}%
\pgfsetdash{}{0pt}%
\pgfpathmoveto{\pgfqpoint{3.626418in}{0.500000in}}%
\pgfpathlineto{\pgfqpoint{3.646835in}{0.500000in}}%
\pgfpathlineto{\pgfqpoint{3.646835in}{0.803307in}}%
\pgfpathlineto{\pgfqpoint{3.626418in}{0.803307in}}%
\pgfpathlineto{\pgfqpoint{3.626418in}{0.500000in}}%
\pgfpathclose%
\pgfusepath{fill}%
\end{pgfscope}%
\begin{pgfscope}%
\pgfpathrectangle{\pgfqpoint{0.750000in}{0.500000in}}{\pgfqpoint{4.650000in}{3.020000in}}%
\pgfusepath{clip}%
\pgfsetbuttcap%
\pgfsetmiterjoin%
\definecolor{currentfill}{rgb}{1.000000,0.000000,0.000000}%
\pgfsetfillcolor{currentfill}%
\pgfsetlinewidth{0.000000pt}%
\definecolor{currentstroke}{rgb}{0.000000,0.000000,0.000000}%
\pgfsetstrokecolor{currentstroke}%
\pgfsetstrokeopacity{0.000000}%
\pgfsetdash{}{0pt}%
\pgfpathmoveto{\pgfqpoint{3.646835in}{0.500000in}}%
\pgfpathlineto{\pgfqpoint{3.667253in}{0.500000in}}%
\pgfpathlineto{\pgfqpoint{3.667253in}{2.298926in}}%
\pgfpathlineto{\pgfqpoint{3.646835in}{2.298926in}}%
\pgfpathlineto{\pgfqpoint{3.646835in}{0.500000in}}%
\pgfpathclose%
\pgfusepath{fill}%
\end{pgfscope}%
\begin{pgfscope}%
\pgfpathrectangle{\pgfqpoint{0.750000in}{0.500000in}}{\pgfqpoint{4.650000in}{3.020000in}}%
\pgfusepath{clip}%
\pgfsetbuttcap%
\pgfsetmiterjoin%
\definecolor{currentfill}{rgb}{1.000000,0.000000,0.000000}%
\pgfsetfillcolor{currentfill}%
\pgfsetlinewidth{0.000000pt}%
\definecolor{currentstroke}{rgb}{0.000000,0.000000,0.000000}%
\pgfsetstrokecolor{currentstroke}%
\pgfsetstrokeopacity{0.000000}%
\pgfsetdash{}{0pt}%
\pgfpathmoveto{\pgfqpoint{3.667253in}{0.500000in}}%
\pgfpathlineto{\pgfqpoint{3.687671in}{0.500000in}}%
\pgfpathlineto{\pgfqpoint{3.687671in}{0.792848in}}%
\pgfpathlineto{\pgfqpoint{3.667253in}{0.792848in}}%
\pgfpathlineto{\pgfqpoint{3.667253in}{0.500000in}}%
\pgfpathclose%
\pgfusepath{fill}%
\end{pgfscope}%
\begin{pgfscope}%
\pgfpathrectangle{\pgfqpoint{0.750000in}{0.500000in}}{\pgfqpoint{4.650000in}{3.020000in}}%
\pgfusepath{clip}%
\pgfsetbuttcap%
\pgfsetmiterjoin%
\definecolor{currentfill}{rgb}{1.000000,0.000000,0.000000}%
\pgfsetfillcolor{currentfill}%
\pgfsetlinewidth{0.000000pt}%
\definecolor{currentstroke}{rgb}{0.000000,0.000000,0.000000}%
\pgfsetstrokecolor{currentstroke}%
\pgfsetstrokeopacity{0.000000}%
\pgfsetdash{}{0pt}%
\pgfpathmoveto{\pgfqpoint{3.687671in}{0.500000in}}%
\pgfpathlineto{\pgfqpoint{3.708089in}{0.500000in}}%
\pgfpathlineto{\pgfqpoint{3.708089in}{0.730095in}}%
\pgfpathlineto{\pgfqpoint{3.687671in}{0.730095in}}%
\pgfpathlineto{\pgfqpoint{3.687671in}{0.500000in}}%
\pgfpathclose%
\pgfusepath{fill}%
\end{pgfscope}%
\begin{pgfscope}%
\pgfpathrectangle{\pgfqpoint{0.750000in}{0.500000in}}{\pgfqpoint{4.650000in}{3.020000in}}%
\pgfusepath{clip}%
\pgfsetbuttcap%
\pgfsetmiterjoin%
\definecolor{currentfill}{rgb}{1.000000,0.000000,0.000000}%
\pgfsetfillcolor{currentfill}%
\pgfsetlinewidth{0.000000pt}%
\definecolor{currentstroke}{rgb}{0.000000,0.000000,0.000000}%
\pgfsetstrokecolor{currentstroke}%
\pgfsetstrokeopacity{0.000000}%
\pgfsetdash{}{0pt}%
\pgfpathmoveto{\pgfqpoint{3.708089in}{0.500000in}}%
\pgfpathlineto{\pgfqpoint{3.728507in}{0.500000in}}%
\pgfpathlineto{\pgfqpoint{3.728507in}{1.838736in}}%
\pgfpathlineto{\pgfqpoint{3.708089in}{1.838736in}}%
\pgfpathlineto{\pgfqpoint{3.708089in}{0.500000in}}%
\pgfpathclose%
\pgfusepath{fill}%
\end{pgfscope}%
\begin{pgfscope}%
\pgfpathrectangle{\pgfqpoint{0.750000in}{0.500000in}}{\pgfqpoint{4.650000in}{3.020000in}}%
\pgfusepath{clip}%
\pgfsetbuttcap%
\pgfsetmiterjoin%
\definecolor{currentfill}{rgb}{1.000000,0.000000,0.000000}%
\pgfsetfillcolor{currentfill}%
\pgfsetlinewidth{0.000000pt}%
\definecolor{currentstroke}{rgb}{0.000000,0.000000,0.000000}%
\pgfsetstrokecolor{currentstroke}%
\pgfsetstrokeopacity{0.000000}%
\pgfsetdash{}{0pt}%
\pgfpathmoveto{\pgfqpoint{3.728507in}{0.500000in}}%
\pgfpathlineto{\pgfqpoint{3.748924in}{0.500000in}}%
\pgfpathlineto{\pgfqpoint{3.748924in}{0.625506in}}%
\pgfpathlineto{\pgfqpoint{3.728507in}{0.625506in}}%
\pgfpathlineto{\pgfqpoint{3.728507in}{0.500000in}}%
\pgfpathclose%
\pgfusepath{fill}%
\end{pgfscope}%
\begin{pgfscope}%
\pgfpathrectangle{\pgfqpoint{0.750000in}{0.500000in}}{\pgfqpoint{4.650000in}{3.020000in}}%
\pgfusepath{clip}%
\pgfsetbuttcap%
\pgfsetmiterjoin%
\definecolor{currentfill}{rgb}{1.000000,0.000000,0.000000}%
\pgfsetfillcolor{currentfill}%
\pgfsetlinewidth{0.000000pt}%
\definecolor{currentstroke}{rgb}{0.000000,0.000000,0.000000}%
\pgfsetstrokecolor{currentstroke}%
\pgfsetstrokeopacity{0.000000}%
\pgfsetdash{}{0pt}%
\pgfpathmoveto{\pgfqpoint{3.748924in}{0.500000in}}%
\pgfpathlineto{\pgfqpoint{3.769342in}{0.500000in}}%
\pgfpathlineto{\pgfqpoint{3.769342in}{1.671394in}}%
\pgfpathlineto{\pgfqpoint{3.748924in}{1.671394in}}%
\pgfpathlineto{\pgfqpoint{3.748924in}{0.500000in}}%
\pgfpathclose%
\pgfusepath{fill}%
\end{pgfscope}%
\begin{pgfscope}%
\pgfpathrectangle{\pgfqpoint{0.750000in}{0.500000in}}{\pgfqpoint{4.650000in}{3.020000in}}%
\pgfusepath{clip}%
\pgfsetbuttcap%
\pgfsetmiterjoin%
\definecolor{currentfill}{rgb}{1.000000,0.000000,0.000000}%
\pgfsetfillcolor{currentfill}%
\pgfsetlinewidth{0.000000pt}%
\definecolor{currentstroke}{rgb}{0.000000,0.000000,0.000000}%
\pgfsetstrokecolor{currentstroke}%
\pgfsetstrokeopacity{0.000000}%
\pgfsetdash{}{0pt}%
\pgfpathmoveto{\pgfqpoint{3.769342in}{0.500000in}}%
\pgfpathlineto{\pgfqpoint{3.789760in}{0.500000in}}%
\pgfpathlineto{\pgfqpoint{3.789760in}{0.635965in}}%
\pgfpathlineto{\pgfqpoint{3.769342in}{0.635965in}}%
\pgfpathlineto{\pgfqpoint{3.769342in}{0.500000in}}%
\pgfpathclose%
\pgfusepath{fill}%
\end{pgfscope}%
\begin{pgfscope}%
\pgfpathrectangle{\pgfqpoint{0.750000in}{0.500000in}}{\pgfqpoint{4.650000in}{3.020000in}}%
\pgfusepath{clip}%
\pgfsetbuttcap%
\pgfsetmiterjoin%
\definecolor{currentfill}{rgb}{1.000000,0.000000,0.000000}%
\pgfsetfillcolor{currentfill}%
\pgfsetlinewidth{0.000000pt}%
\definecolor{currentstroke}{rgb}{0.000000,0.000000,0.000000}%
\pgfsetstrokecolor{currentstroke}%
\pgfsetstrokeopacity{0.000000}%
\pgfsetdash{}{0pt}%
\pgfpathmoveto{\pgfqpoint{3.789760in}{0.500000in}}%
\pgfpathlineto{\pgfqpoint{3.810178in}{0.500000in}}%
\pgfpathlineto{\pgfqpoint{3.810178in}{0.656883in}}%
\pgfpathlineto{\pgfqpoint{3.789760in}{0.656883in}}%
\pgfpathlineto{\pgfqpoint{3.789760in}{0.500000in}}%
\pgfpathclose%
\pgfusepath{fill}%
\end{pgfscope}%
\begin{pgfscope}%
\pgfpathrectangle{\pgfqpoint{0.750000in}{0.500000in}}{\pgfqpoint{4.650000in}{3.020000in}}%
\pgfusepath{clip}%
\pgfsetbuttcap%
\pgfsetmiterjoin%
\definecolor{currentfill}{rgb}{1.000000,0.000000,0.000000}%
\pgfsetfillcolor{currentfill}%
\pgfsetlinewidth{0.000000pt}%
\definecolor{currentstroke}{rgb}{0.000000,0.000000,0.000000}%
\pgfsetstrokecolor{currentstroke}%
\pgfsetstrokeopacity{0.000000}%
\pgfsetdash{}{0pt}%
\pgfpathmoveto{\pgfqpoint{3.810178in}{0.500000in}}%
\pgfpathlineto{\pgfqpoint{3.830596in}{0.500000in}}%
\pgfpathlineto{\pgfqpoint{3.830596in}{1.472675in}}%
\pgfpathlineto{\pgfqpoint{3.810178in}{1.472675in}}%
\pgfpathlineto{\pgfqpoint{3.810178in}{0.500000in}}%
\pgfpathclose%
\pgfusepath{fill}%
\end{pgfscope}%
\begin{pgfscope}%
\pgfpathrectangle{\pgfqpoint{0.750000in}{0.500000in}}{\pgfqpoint{4.650000in}{3.020000in}}%
\pgfusepath{clip}%
\pgfsetbuttcap%
\pgfsetmiterjoin%
\definecolor{currentfill}{rgb}{1.000000,0.000000,0.000000}%
\pgfsetfillcolor{currentfill}%
\pgfsetlinewidth{0.000000pt}%
\definecolor{currentstroke}{rgb}{0.000000,0.000000,0.000000}%
\pgfsetstrokecolor{currentstroke}%
\pgfsetstrokeopacity{0.000000}%
\pgfsetdash{}{0pt}%
\pgfpathmoveto{\pgfqpoint{3.830596in}{0.500000in}}%
\pgfpathlineto{\pgfqpoint{3.851013in}{0.500000in}}%
\pgfpathlineto{\pgfqpoint{3.851013in}{0.615048in}}%
\pgfpathlineto{\pgfqpoint{3.830596in}{0.615048in}}%
\pgfpathlineto{\pgfqpoint{3.830596in}{0.500000in}}%
\pgfpathclose%
\pgfusepath{fill}%
\end{pgfscope}%
\begin{pgfscope}%
\pgfpathrectangle{\pgfqpoint{0.750000in}{0.500000in}}{\pgfqpoint{4.650000in}{3.020000in}}%
\pgfusepath{clip}%
\pgfsetbuttcap%
\pgfsetmiterjoin%
\definecolor{currentfill}{rgb}{1.000000,0.000000,0.000000}%
\pgfsetfillcolor{currentfill}%
\pgfsetlinewidth{0.000000pt}%
\definecolor{currentstroke}{rgb}{0.000000,0.000000,0.000000}%
\pgfsetstrokecolor{currentstroke}%
\pgfsetstrokeopacity{0.000000}%
\pgfsetdash{}{0pt}%
\pgfpathmoveto{\pgfqpoint{3.851013in}{0.500000in}}%
\pgfpathlineto{\pgfqpoint{3.871431in}{0.500000in}}%
\pgfpathlineto{\pgfqpoint{3.871431in}{1.420381in}}%
\pgfpathlineto{\pgfqpoint{3.851013in}{1.420381in}}%
\pgfpathlineto{\pgfqpoint{3.851013in}{0.500000in}}%
\pgfpathclose%
\pgfusepath{fill}%
\end{pgfscope}%
\begin{pgfscope}%
\pgfpathrectangle{\pgfqpoint{0.750000in}{0.500000in}}{\pgfqpoint{4.650000in}{3.020000in}}%
\pgfusepath{clip}%
\pgfsetbuttcap%
\pgfsetmiterjoin%
\definecolor{currentfill}{rgb}{1.000000,0.000000,0.000000}%
\pgfsetfillcolor{currentfill}%
\pgfsetlinewidth{0.000000pt}%
\definecolor{currentstroke}{rgb}{0.000000,0.000000,0.000000}%
\pgfsetstrokecolor{currentstroke}%
\pgfsetstrokeopacity{0.000000}%
\pgfsetdash{}{0pt}%
\pgfpathmoveto{\pgfqpoint{3.871431in}{0.500000in}}%
\pgfpathlineto{\pgfqpoint{3.891849in}{0.500000in}}%
\pgfpathlineto{\pgfqpoint{3.891849in}{0.552294in}}%
\pgfpathlineto{\pgfqpoint{3.871431in}{0.552294in}}%
\pgfpathlineto{\pgfqpoint{3.871431in}{0.500000in}}%
\pgfpathclose%
\pgfusepath{fill}%
\end{pgfscope}%
\begin{pgfscope}%
\pgfpathrectangle{\pgfqpoint{0.750000in}{0.500000in}}{\pgfqpoint{4.650000in}{3.020000in}}%
\pgfusepath{clip}%
\pgfsetbuttcap%
\pgfsetmiterjoin%
\definecolor{currentfill}{rgb}{1.000000,0.000000,0.000000}%
\pgfsetfillcolor{currentfill}%
\pgfsetlinewidth{0.000000pt}%
\definecolor{currentstroke}{rgb}{0.000000,0.000000,0.000000}%
\pgfsetstrokecolor{currentstroke}%
\pgfsetstrokeopacity{0.000000}%
\pgfsetdash{}{0pt}%
\pgfpathmoveto{\pgfqpoint{3.891849in}{0.500000in}}%
\pgfpathlineto{\pgfqpoint{3.912267in}{0.500000in}}%
\pgfpathlineto{\pgfqpoint{3.912267in}{0.719636in}}%
\pgfpathlineto{\pgfqpoint{3.891849in}{0.719636in}}%
\pgfpathlineto{\pgfqpoint{3.891849in}{0.500000in}}%
\pgfpathclose%
\pgfusepath{fill}%
\end{pgfscope}%
\begin{pgfscope}%
\pgfpathrectangle{\pgfqpoint{0.750000in}{0.500000in}}{\pgfqpoint{4.650000in}{3.020000in}}%
\pgfusepath{clip}%
\pgfsetbuttcap%
\pgfsetmiterjoin%
\definecolor{currentfill}{rgb}{1.000000,0.000000,0.000000}%
\pgfsetfillcolor{currentfill}%
\pgfsetlinewidth{0.000000pt}%
\definecolor{currentstroke}{rgb}{0.000000,0.000000,0.000000}%
\pgfsetstrokecolor{currentstroke}%
\pgfsetstrokeopacity{0.000000}%
\pgfsetdash{}{0pt}%
\pgfpathmoveto{\pgfqpoint{3.912267in}{0.500000in}}%
\pgfpathlineto{\pgfqpoint{3.932685in}{0.500000in}}%
\pgfpathlineto{\pgfqpoint{3.932685in}{0.918355in}}%
\pgfpathlineto{\pgfqpoint{3.912267in}{0.918355in}}%
\pgfpathlineto{\pgfqpoint{3.912267in}{0.500000in}}%
\pgfpathclose%
\pgfusepath{fill}%
\end{pgfscope}%
\begin{pgfscope}%
\pgfpathrectangle{\pgfqpoint{0.750000in}{0.500000in}}{\pgfqpoint{4.650000in}{3.020000in}}%
\pgfusepath{clip}%
\pgfsetbuttcap%
\pgfsetmiterjoin%
\definecolor{currentfill}{rgb}{1.000000,0.000000,0.000000}%
\pgfsetfillcolor{currentfill}%
\pgfsetlinewidth{0.000000pt}%
\definecolor{currentstroke}{rgb}{0.000000,0.000000,0.000000}%
\pgfsetstrokecolor{currentstroke}%
\pgfsetstrokeopacity{0.000000}%
\pgfsetdash{}{0pt}%
\pgfpathmoveto{\pgfqpoint{3.932685in}{0.500000in}}%
\pgfpathlineto{\pgfqpoint{3.953102in}{0.500000in}}%
\pgfpathlineto{\pgfqpoint{3.953102in}{0.541835in}}%
\pgfpathlineto{\pgfqpoint{3.932685in}{0.541835in}}%
\pgfpathlineto{\pgfqpoint{3.932685in}{0.500000in}}%
\pgfpathclose%
\pgfusepath{fill}%
\end{pgfscope}%
\begin{pgfscope}%
\pgfpathrectangle{\pgfqpoint{0.750000in}{0.500000in}}{\pgfqpoint{4.650000in}{3.020000in}}%
\pgfusepath{clip}%
\pgfsetbuttcap%
\pgfsetmiterjoin%
\definecolor{currentfill}{rgb}{1.000000,0.000000,0.000000}%
\pgfsetfillcolor{currentfill}%
\pgfsetlinewidth{0.000000pt}%
\definecolor{currentstroke}{rgb}{0.000000,0.000000,0.000000}%
\pgfsetstrokecolor{currentstroke}%
\pgfsetstrokeopacity{0.000000}%
\pgfsetdash{}{0pt}%
\pgfpathmoveto{\pgfqpoint{3.953102in}{0.500000in}}%
\pgfpathlineto{\pgfqpoint{3.973520in}{0.500000in}}%
\pgfpathlineto{\pgfqpoint{3.973520in}{1.158909in}}%
\pgfpathlineto{\pgfqpoint{3.953102in}{1.158909in}}%
\pgfpathlineto{\pgfqpoint{3.953102in}{0.500000in}}%
\pgfpathclose%
\pgfusepath{fill}%
\end{pgfscope}%
\begin{pgfscope}%
\pgfpathrectangle{\pgfqpoint{0.750000in}{0.500000in}}{\pgfqpoint{4.650000in}{3.020000in}}%
\pgfusepath{clip}%
\pgfsetbuttcap%
\pgfsetmiterjoin%
\definecolor{currentfill}{rgb}{1.000000,0.000000,0.000000}%
\pgfsetfillcolor{currentfill}%
\pgfsetlinewidth{0.000000pt}%
\definecolor{currentstroke}{rgb}{0.000000,0.000000,0.000000}%
\pgfsetstrokecolor{currentstroke}%
\pgfsetstrokeopacity{0.000000}%
\pgfsetdash{}{0pt}%
\pgfpathmoveto{\pgfqpoint{3.973520in}{0.500000in}}%
\pgfpathlineto{\pgfqpoint{3.993938in}{0.500000in}}%
\pgfpathlineto{\pgfqpoint{3.993938in}{0.562753in}}%
\pgfpathlineto{\pgfqpoint{3.973520in}{0.562753in}}%
\pgfpathlineto{\pgfqpoint{3.973520in}{0.500000in}}%
\pgfpathclose%
\pgfusepath{fill}%
\end{pgfscope}%
\begin{pgfscope}%
\pgfpathrectangle{\pgfqpoint{0.750000in}{0.500000in}}{\pgfqpoint{4.650000in}{3.020000in}}%
\pgfusepath{clip}%
\pgfsetbuttcap%
\pgfsetmiterjoin%
\definecolor{currentfill}{rgb}{1.000000,0.000000,0.000000}%
\pgfsetfillcolor{currentfill}%
\pgfsetlinewidth{0.000000pt}%
\definecolor{currentstroke}{rgb}{0.000000,0.000000,0.000000}%
\pgfsetstrokecolor{currentstroke}%
\pgfsetstrokeopacity{0.000000}%
\pgfsetdash{}{0pt}%
\pgfpathmoveto{\pgfqpoint{3.993938in}{0.500000in}}%
\pgfpathlineto{\pgfqpoint{4.014356in}{0.500000in}}%
\pgfpathlineto{\pgfqpoint{4.014356in}{0.897437in}}%
\pgfpathlineto{\pgfqpoint{3.993938in}{0.897437in}}%
\pgfpathlineto{\pgfqpoint{3.993938in}{0.500000in}}%
\pgfpathclose%
\pgfusepath{fill}%
\end{pgfscope}%
\begin{pgfscope}%
\pgfpathrectangle{\pgfqpoint{0.750000in}{0.500000in}}{\pgfqpoint{4.650000in}{3.020000in}}%
\pgfusepath{clip}%
\pgfsetbuttcap%
\pgfsetmiterjoin%
\definecolor{currentfill}{rgb}{1.000000,0.000000,0.000000}%
\pgfsetfillcolor{currentfill}%
\pgfsetlinewidth{0.000000pt}%
\definecolor{currentstroke}{rgb}{0.000000,0.000000,0.000000}%
\pgfsetstrokecolor{currentstroke}%
\pgfsetstrokeopacity{0.000000}%
\pgfsetdash{}{0pt}%
\pgfpathmoveto{\pgfqpoint{4.014356in}{0.500000in}}%
\pgfpathlineto{\pgfqpoint{4.034774in}{0.500000in}}%
\pgfpathlineto{\pgfqpoint{4.034774in}{0.520918in}}%
\pgfpathlineto{\pgfqpoint{4.014356in}{0.520918in}}%
\pgfpathlineto{\pgfqpoint{4.014356in}{0.500000in}}%
\pgfpathclose%
\pgfusepath{fill}%
\end{pgfscope}%
\begin{pgfscope}%
\pgfpathrectangle{\pgfqpoint{0.750000in}{0.500000in}}{\pgfqpoint{4.650000in}{3.020000in}}%
\pgfusepath{clip}%
\pgfsetbuttcap%
\pgfsetmiterjoin%
\definecolor{currentfill}{rgb}{1.000000,0.000000,0.000000}%
\pgfsetfillcolor{currentfill}%
\pgfsetlinewidth{0.000000pt}%
\definecolor{currentstroke}{rgb}{0.000000,0.000000,0.000000}%
\pgfsetstrokecolor{currentstroke}%
\pgfsetstrokeopacity{0.000000}%
\pgfsetdash{}{0pt}%
\pgfpathmoveto{\pgfqpoint{4.034774in}{0.500000in}}%
\pgfpathlineto{\pgfqpoint{4.055192in}{0.500000in}}%
\pgfpathlineto{\pgfqpoint{4.055192in}{0.541835in}}%
\pgfpathlineto{\pgfqpoint{4.034774in}{0.541835in}}%
\pgfpathlineto{\pgfqpoint{4.034774in}{0.500000in}}%
\pgfpathclose%
\pgfusepath{fill}%
\end{pgfscope}%
\begin{pgfscope}%
\pgfpathrectangle{\pgfqpoint{0.750000in}{0.500000in}}{\pgfqpoint{4.650000in}{3.020000in}}%
\pgfusepath{clip}%
\pgfsetbuttcap%
\pgfsetmiterjoin%
\definecolor{currentfill}{rgb}{1.000000,0.000000,0.000000}%
\pgfsetfillcolor{currentfill}%
\pgfsetlinewidth{0.000000pt}%
\definecolor{currentstroke}{rgb}{0.000000,0.000000,0.000000}%
\pgfsetstrokecolor{currentstroke}%
\pgfsetstrokeopacity{0.000000}%
\pgfsetdash{}{0pt}%
\pgfpathmoveto{\pgfqpoint{4.055192in}{0.500000in}}%
\pgfpathlineto{\pgfqpoint{4.075609in}{0.500000in}}%
\pgfpathlineto{\pgfqpoint{4.075609in}{0.939273in}}%
\pgfpathlineto{\pgfqpoint{4.055192in}{0.939273in}}%
\pgfpathlineto{\pgfqpoint{4.055192in}{0.500000in}}%
\pgfpathclose%
\pgfusepath{fill}%
\end{pgfscope}%
\begin{pgfscope}%
\pgfpathrectangle{\pgfqpoint{0.750000in}{0.500000in}}{\pgfqpoint{4.650000in}{3.020000in}}%
\pgfusepath{clip}%
\pgfsetbuttcap%
\pgfsetmiterjoin%
\definecolor{currentfill}{rgb}{1.000000,0.000000,0.000000}%
\pgfsetfillcolor{currentfill}%
\pgfsetlinewidth{0.000000pt}%
\definecolor{currentstroke}{rgb}{0.000000,0.000000,0.000000}%
\pgfsetstrokecolor{currentstroke}%
\pgfsetstrokeopacity{0.000000}%
\pgfsetdash{}{0pt}%
\pgfpathmoveto{\pgfqpoint{4.075609in}{0.500000in}}%
\pgfpathlineto{\pgfqpoint{4.096027in}{0.500000in}}%
\pgfpathlineto{\pgfqpoint{4.096027in}{0.531377in}}%
\pgfpathlineto{\pgfqpoint{4.075609in}{0.531377in}}%
\pgfpathlineto{\pgfqpoint{4.075609in}{0.500000in}}%
\pgfpathclose%
\pgfusepath{fill}%
\end{pgfscope}%
\begin{pgfscope}%
\pgfpathrectangle{\pgfqpoint{0.750000in}{0.500000in}}{\pgfqpoint{4.650000in}{3.020000in}}%
\pgfusepath{clip}%
\pgfsetbuttcap%
\pgfsetmiterjoin%
\definecolor{currentfill}{rgb}{1.000000,0.000000,0.000000}%
\pgfsetfillcolor{currentfill}%
\pgfsetlinewidth{0.000000pt}%
\definecolor{currentstroke}{rgb}{0.000000,0.000000,0.000000}%
\pgfsetstrokecolor{currentstroke}%
\pgfsetstrokeopacity{0.000000}%
\pgfsetdash{}{0pt}%
\pgfpathmoveto{\pgfqpoint{4.096027in}{0.500000in}}%
\pgfpathlineto{\pgfqpoint{4.116445in}{0.500000in}}%
\pgfpathlineto{\pgfqpoint{4.116445in}{0.646424in}}%
\pgfpathlineto{\pgfqpoint{4.096027in}{0.646424in}}%
\pgfpathlineto{\pgfqpoint{4.096027in}{0.500000in}}%
\pgfpathclose%
\pgfusepath{fill}%
\end{pgfscope}%
\begin{pgfscope}%
\pgfpathrectangle{\pgfqpoint{0.750000in}{0.500000in}}{\pgfqpoint{4.650000in}{3.020000in}}%
\pgfusepath{clip}%
\pgfsetbuttcap%
\pgfsetmiterjoin%
\definecolor{currentfill}{rgb}{1.000000,0.000000,0.000000}%
\pgfsetfillcolor{currentfill}%
\pgfsetlinewidth{0.000000pt}%
\definecolor{currentstroke}{rgb}{0.000000,0.000000,0.000000}%
\pgfsetstrokecolor{currentstroke}%
\pgfsetstrokeopacity{0.000000}%
\pgfsetdash{}{0pt}%
\pgfpathmoveto{\pgfqpoint{4.116445in}{0.500000in}}%
\pgfpathlineto{\pgfqpoint{4.136863in}{0.500000in}}%
\pgfpathlineto{\pgfqpoint{4.136863in}{0.520918in}}%
\pgfpathlineto{\pgfqpoint{4.116445in}{0.520918in}}%
\pgfpathlineto{\pgfqpoint{4.116445in}{0.500000in}}%
\pgfpathclose%
\pgfusepath{fill}%
\end{pgfscope}%
\begin{pgfscope}%
\pgfpathrectangle{\pgfqpoint{0.750000in}{0.500000in}}{\pgfqpoint{4.650000in}{3.020000in}}%
\pgfusepath{clip}%
\pgfsetbuttcap%
\pgfsetmiterjoin%
\definecolor{currentfill}{rgb}{1.000000,0.000000,0.000000}%
\pgfsetfillcolor{currentfill}%
\pgfsetlinewidth{0.000000pt}%
\definecolor{currentstroke}{rgb}{0.000000,0.000000,0.000000}%
\pgfsetstrokecolor{currentstroke}%
\pgfsetstrokeopacity{0.000000}%
\pgfsetdash{}{0pt}%
\pgfpathmoveto{\pgfqpoint{4.136863in}{0.500000in}}%
\pgfpathlineto{\pgfqpoint{4.157281in}{0.500000in}}%
\pgfpathlineto{\pgfqpoint{4.157281in}{0.500000in}}%
\pgfpathlineto{\pgfqpoint{4.136863in}{0.500000in}}%
\pgfpathlineto{\pgfqpoint{4.136863in}{0.500000in}}%
\pgfpathclose%
\pgfusepath{fill}%
\end{pgfscope}%
\begin{pgfscope}%
\pgfpathrectangle{\pgfqpoint{0.750000in}{0.500000in}}{\pgfqpoint{4.650000in}{3.020000in}}%
\pgfusepath{clip}%
\pgfsetbuttcap%
\pgfsetmiterjoin%
\definecolor{currentfill}{rgb}{1.000000,0.000000,0.000000}%
\pgfsetfillcolor{currentfill}%
\pgfsetlinewidth{0.000000pt}%
\definecolor{currentstroke}{rgb}{0.000000,0.000000,0.000000}%
\pgfsetstrokecolor{currentstroke}%
\pgfsetstrokeopacity{0.000000}%
\pgfsetdash{}{0pt}%
\pgfpathmoveto{\pgfqpoint{4.157281in}{0.500000in}}%
\pgfpathlineto{\pgfqpoint{4.177698in}{0.500000in}}%
\pgfpathlineto{\pgfqpoint{4.177698in}{0.677801in}}%
\pgfpathlineto{\pgfqpoint{4.157281in}{0.677801in}}%
\pgfpathlineto{\pgfqpoint{4.157281in}{0.500000in}}%
\pgfpathclose%
\pgfusepath{fill}%
\end{pgfscope}%
\begin{pgfscope}%
\pgfpathrectangle{\pgfqpoint{0.750000in}{0.500000in}}{\pgfqpoint{4.650000in}{3.020000in}}%
\pgfusepath{clip}%
\pgfsetbuttcap%
\pgfsetmiterjoin%
\definecolor{currentfill}{rgb}{1.000000,0.000000,0.000000}%
\pgfsetfillcolor{currentfill}%
\pgfsetlinewidth{0.000000pt}%
\definecolor{currentstroke}{rgb}{0.000000,0.000000,0.000000}%
\pgfsetstrokecolor{currentstroke}%
\pgfsetstrokeopacity{0.000000}%
\pgfsetdash{}{0pt}%
\pgfpathmoveto{\pgfqpoint{4.177698in}{0.500000in}}%
\pgfpathlineto{\pgfqpoint{4.198116in}{0.500000in}}%
\pgfpathlineto{\pgfqpoint{4.198116in}{0.510459in}}%
\pgfpathlineto{\pgfqpoint{4.177698in}{0.510459in}}%
\pgfpathlineto{\pgfqpoint{4.177698in}{0.500000in}}%
\pgfpathclose%
\pgfusepath{fill}%
\end{pgfscope}%
\begin{pgfscope}%
\pgfpathrectangle{\pgfqpoint{0.750000in}{0.500000in}}{\pgfqpoint{4.650000in}{3.020000in}}%
\pgfusepath{clip}%
\pgfsetbuttcap%
\pgfsetmiterjoin%
\definecolor{currentfill}{rgb}{1.000000,0.000000,0.000000}%
\pgfsetfillcolor{currentfill}%
\pgfsetlinewidth{0.000000pt}%
\definecolor{currentstroke}{rgb}{0.000000,0.000000,0.000000}%
\pgfsetstrokecolor{currentstroke}%
\pgfsetstrokeopacity{0.000000}%
\pgfsetdash{}{0pt}%
\pgfpathmoveto{\pgfqpoint{4.198116in}{0.500000in}}%
\pgfpathlineto{\pgfqpoint{4.218534in}{0.500000in}}%
\pgfpathlineto{\pgfqpoint{4.218534in}{0.656883in}}%
\pgfpathlineto{\pgfqpoint{4.198116in}{0.656883in}}%
\pgfpathlineto{\pgfqpoint{4.198116in}{0.500000in}}%
\pgfpathclose%
\pgfusepath{fill}%
\end{pgfscope}%
\begin{pgfscope}%
\pgfpathrectangle{\pgfqpoint{0.750000in}{0.500000in}}{\pgfqpoint{4.650000in}{3.020000in}}%
\pgfusepath{clip}%
\pgfsetbuttcap%
\pgfsetmiterjoin%
\definecolor{currentfill}{rgb}{1.000000,0.000000,0.000000}%
\pgfsetfillcolor{currentfill}%
\pgfsetlinewidth{0.000000pt}%
\definecolor{currentstroke}{rgb}{0.000000,0.000000,0.000000}%
\pgfsetstrokecolor{currentstroke}%
\pgfsetstrokeopacity{0.000000}%
\pgfsetdash{}{0pt}%
\pgfpathmoveto{\pgfqpoint{4.218534in}{0.500000in}}%
\pgfpathlineto{\pgfqpoint{4.238952in}{0.500000in}}%
\pgfpathlineto{\pgfqpoint{4.238952in}{0.520918in}}%
\pgfpathlineto{\pgfqpoint{4.218534in}{0.520918in}}%
\pgfpathlineto{\pgfqpoint{4.218534in}{0.500000in}}%
\pgfpathclose%
\pgfusepath{fill}%
\end{pgfscope}%
\begin{pgfscope}%
\pgfpathrectangle{\pgfqpoint{0.750000in}{0.500000in}}{\pgfqpoint{4.650000in}{3.020000in}}%
\pgfusepath{clip}%
\pgfsetbuttcap%
\pgfsetmiterjoin%
\definecolor{currentfill}{rgb}{1.000000,0.000000,0.000000}%
\pgfsetfillcolor{currentfill}%
\pgfsetlinewidth{0.000000pt}%
\definecolor{currentstroke}{rgb}{0.000000,0.000000,0.000000}%
\pgfsetstrokecolor{currentstroke}%
\pgfsetstrokeopacity{0.000000}%
\pgfsetdash{}{0pt}%
\pgfpathmoveto{\pgfqpoint{4.238952in}{0.500000in}}%
\pgfpathlineto{\pgfqpoint{4.259370in}{0.500000in}}%
\pgfpathlineto{\pgfqpoint{4.259370in}{0.520918in}}%
\pgfpathlineto{\pgfqpoint{4.238952in}{0.520918in}}%
\pgfpathlineto{\pgfqpoint{4.238952in}{0.500000in}}%
\pgfpathclose%
\pgfusepath{fill}%
\end{pgfscope}%
\begin{pgfscope}%
\pgfpathrectangle{\pgfqpoint{0.750000in}{0.500000in}}{\pgfqpoint{4.650000in}{3.020000in}}%
\pgfusepath{clip}%
\pgfsetbuttcap%
\pgfsetmiterjoin%
\definecolor{currentfill}{rgb}{1.000000,0.000000,0.000000}%
\pgfsetfillcolor{currentfill}%
\pgfsetlinewidth{0.000000pt}%
\definecolor{currentstroke}{rgb}{0.000000,0.000000,0.000000}%
\pgfsetstrokecolor{currentstroke}%
\pgfsetstrokeopacity{0.000000}%
\pgfsetdash{}{0pt}%
\pgfpathmoveto{\pgfqpoint{4.259370in}{0.500000in}}%
\pgfpathlineto{\pgfqpoint{4.279787in}{0.500000in}}%
\pgfpathlineto{\pgfqpoint{4.279787in}{0.615048in}}%
\pgfpathlineto{\pgfqpoint{4.259370in}{0.615048in}}%
\pgfpathlineto{\pgfqpoint{4.259370in}{0.500000in}}%
\pgfpathclose%
\pgfusepath{fill}%
\end{pgfscope}%
\begin{pgfscope}%
\pgfpathrectangle{\pgfqpoint{0.750000in}{0.500000in}}{\pgfqpoint{4.650000in}{3.020000in}}%
\pgfusepath{clip}%
\pgfsetbuttcap%
\pgfsetmiterjoin%
\definecolor{currentfill}{rgb}{1.000000,0.000000,0.000000}%
\pgfsetfillcolor{currentfill}%
\pgfsetlinewidth{0.000000pt}%
\definecolor{currentstroke}{rgb}{0.000000,0.000000,0.000000}%
\pgfsetstrokecolor{currentstroke}%
\pgfsetstrokeopacity{0.000000}%
\pgfsetdash{}{0pt}%
\pgfpathmoveto{\pgfqpoint{4.279787in}{0.500000in}}%
\pgfpathlineto{\pgfqpoint{4.300205in}{0.500000in}}%
\pgfpathlineto{\pgfqpoint{4.300205in}{0.500000in}}%
\pgfpathlineto{\pgfqpoint{4.279787in}{0.500000in}}%
\pgfpathlineto{\pgfqpoint{4.279787in}{0.500000in}}%
\pgfpathclose%
\pgfusepath{fill}%
\end{pgfscope}%
\begin{pgfscope}%
\pgfpathrectangle{\pgfqpoint{0.750000in}{0.500000in}}{\pgfqpoint{4.650000in}{3.020000in}}%
\pgfusepath{clip}%
\pgfsetbuttcap%
\pgfsetmiterjoin%
\definecolor{currentfill}{rgb}{1.000000,0.000000,0.000000}%
\pgfsetfillcolor{currentfill}%
\pgfsetlinewidth{0.000000pt}%
\definecolor{currentstroke}{rgb}{0.000000,0.000000,0.000000}%
\pgfsetstrokecolor{currentstroke}%
\pgfsetstrokeopacity{0.000000}%
\pgfsetdash{}{0pt}%
\pgfpathmoveto{\pgfqpoint{4.300205in}{0.500000in}}%
\pgfpathlineto{\pgfqpoint{4.320623in}{0.500000in}}%
\pgfpathlineto{\pgfqpoint{4.320623in}{0.520918in}}%
\pgfpathlineto{\pgfqpoint{4.300205in}{0.520918in}}%
\pgfpathlineto{\pgfqpoint{4.300205in}{0.500000in}}%
\pgfpathclose%
\pgfusepath{fill}%
\end{pgfscope}%
\begin{pgfscope}%
\pgfpathrectangle{\pgfqpoint{0.750000in}{0.500000in}}{\pgfqpoint{4.650000in}{3.020000in}}%
\pgfusepath{clip}%
\pgfsetbuttcap%
\pgfsetmiterjoin%
\definecolor{currentfill}{rgb}{1.000000,0.000000,0.000000}%
\pgfsetfillcolor{currentfill}%
\pgfsetlinewidth{0.000000pt}%
\definecolor{currentstroke}{rgb}{0.000000,0.000000,0.000000}%
\pgfsetstrokecolor{currentstroke}%
\pgfsetstrokeopacity{0.000000}%
\pgfsetdash{}{0pt}%
\pgfpathmoveto{\pgfqpoint{4.320623in}{0.500000in}}%
\pgfpathlineto{\pgfqpoint{4.341041in}{0.500000in}}%
\pgfpathlineto{\pgfqpoint{4.341041in}{0.510459in}}%
\pgfpathlineto{\pgfqpoint{4.320623in}{0.510459in}}%
\pgfpathlineto{\pgfqpoint{4.320623in}{0.500000in}}%
\pgfpathclose%
\pgfusepath{fill}%
\end{pgfscope}%
\begin{pgfscope}%
\pgfpathrectangle{\pgfqpoint{0.750000in}{0.500000in}}{\pgfqpoint{4.650000in}{3.020000in}}%
\pgfusepath{clip}%
\pgfsetbuttcap%
\pgfsetmiterjoin%
\definecolor{currentfill}{rgb}{1.000000,0.000000,0.000000}%
\pgfsetfillcolor{currentfill}%
\pgfsetlinewidth{0.000000pt}%
\definecolor{currentstroke}{rgb}{0.000000,0.000000,0.000000}%
\pgfsetstrokecolor{currentstroke}%
\pgfsetstrokeopacity{0.000000}%
\pgfsetdash{}{0pt}%
\pgfpathmoveto{\pgfqpoint{4.341041in}{0.500000in}}%
\pgfpathlineto{\pgfqpoint{4.361459in}{0.500000in}}%
\pgfpathlineto{\pgfqpoint{4.361459in}{0.500000in}}%
\pgfpathlineto{\pgfqpoint{4.341041in}{0.500000in}}%
\pgfpathlineto{\pgfqpoint{4.341041in}{0.500000in}}%
\pgfpathclose%
\pgfusepath{fill}%
\end{pgfscope}%
\begin{pgfscope}%
\pgfpathrectangle{\pgfqpoint{0.750000in}{0.500000in}}{\pgfqpoint{4.650000in}{3.020000in}}%
\pgfusepath{clip}%
\pgfsetbuttcap%
\pgfsetmiterjoin%
\definecolor{currentfill}{rgb}{1.000000,0.000000,0.000000}%
\pgfsetfillcolor{currentfill}%
\pgfsetlinewidth{0.000000pt}%
\definecolor{currentstroke}{rgb}{0.000000,0.000000,0.000000}%
\pgfsetstrokecolor{currentstroke}%
\pgfsetstrokeopacity{0.000000}%
\pgfsetdash{}{0pt}%
\pgfpathmoveto{\pgfqpoint{4.361459in}{0.500000in}}%
\pgfpathlineto{\pgfqpoint{4.381876in}{0.500000in}}%
\pgfpathlineto{\pgfqpoint{4.381876in}{0.520918in}}%
\pgfpathlineto{\pgfqpoint{4.361459in}{0.520918in}}%
\pgfpathlineto{\pgfqpoint{4.361459in}{0.500000in}}%
\pgfpathclose%
\pgfusepath{fill}%
\end{pgfscope}%
\begin{pgfscope}%
\pgfpathrectangle{\pgfqpoint{0.750000in}{0.500000in}}{\pgfqpoint{4.650000in}{3.020000in}}%
\pgfusepath{clip}%
\pgfsetbuttcap%
\pgfsetmiterjoin%
\definecolor{currentfill}{rgb}{1.000000,0.000000,0.000000}%
\pgfsetfillcolor{currentfill}%
\pgfsetlinewidth{0.000000pt}%
\definecolor{currentstroke}{rgb}{0.000000,0.000000,0.000000}%
\pgfsetstrokecolor{currentstroke}%
\pgfsetstrokeopacity{0.000000}%
\pgfsetdash{}{0pt}%
\pgfpathmoveto{\pgfqpoint{4.381876in}{0.500000in}}%
\pgfpathlineto{\pgfqpoint{4.402294in}{0.500000in}}%
\pgfpathlineto{\pgfqpoint{4.402294in}{0.500000in}}%
\pgfpathlineto{\pgfqpoint{4.381876in}{0.500000in}}%
\pgfpathlineto{\pgfqpoint{4.381876in}{0.500000in}}%
\pgfpathclose%
\pgfusepath{fill}%
\end{pgfscope}%
\begin{pgfscope}%
\pgfpathrectangle{\pgfqpoint{0.750000in}{0.500000in}}{\pgfqpoint{4.650000in}{3.020000in}}%
\pgfusepath{clip}%
\pgfsetbuttcap%
\pgfsetmiterjoin%
\definecolor{currentfill}{rgb}{1.000000,0.000000,0.000000}%
\pgfsetfillcolor{currentfill}%
\pgfsetlinewidth{0.000000pt}%
\definecolor{currentstroke}{rgb}{0.000000,0.000000,0.000000}%
\pgfsetstrokecolor{currentstroke}%
\pgfsetstrokeopacity{0.000000}%
\pgfsetdash{}{0pt}%
\pgfpathmoveto{\pgfqpoint{4.402294in}{0.500000in}}%
\pgfpathlineto{\pgfqpoint{4.422712in}{0.500000in}}%
\pgfpathlineto{\pgfqpoint{4.422712in}{0.510459in}}%
\pgfpathlineto{\pgfqpoint{4.402294in}{0.510459in}}%
\pgfpathlineto{\pgfqpoint{4.402294in}{0.500000in}}%
\pgfpathclose%
\pgfusepath{fill}%
\end{pgfscope}%
\begin{pgfscope}%
\pgfpathrectangle{\pgfqpoint{0.750000in}{0.500000in}}{\pgfqpoint{4.650000in}{3.020000in}}%
\pgfusepath{clip}%
\pgfsetbuttcap%
\pgfsetmiterjoin%
\definecolor{currentfill}{rgb}{1.000000,0.000000,0.000000}%
\pgfsetfillcolor{currentfill}%
\pgfsetlinewidth{0.000000pt}%
\definecolor{currentstroke}{rgb}{0.000000,0.000000,0.000000}%
\pgfsetstrokecolor{currentstroke}%
\pgfsetstrokeopacity{0.000000}%
\pgfsetdash{}{0pt}%
\pgfpathmoveto{\pgfqpoint{4.422712in}{0.500000in}}%
\pgfpathlineto{\pgfqpoint{4.443130in}{0.500000in}}%
\pgfpathlineto{\pgfqpoint{4.443130in}{0.500000in}}%
\pgfpathlineto{\pgfqpoint{4.422712in}{0.500000in}}%
\pgfpathlineto{\pgfqpoint{4.422712in}{0.500000in}}%
\pgfpathclose%
\pgfusepath{fill}%
\end{pgfscope}%
\begin{pgfscope}%
\pgfpathrectangle{\pgfqpoint{0.750000in}{0.500000in}}{\pgfqpoint{4.650000in}{3.020000in}}%
\pgfusepath{clip}%
\pgfsetbuttcap%
\pgfsetmiterjoin%
\definecolor{currentfill}{rgb}{1.000000,0.000000,0.000000}%
\pgfsetfillcolor{currentfill}%
\pgfsetlinewidth{0.000000pt}%
\definecolor{currentstroke}{rgb}{0.000000,0.000000,0.000000}%
\pgfsetstrokecolor{currentstroke}%
\pgfsetstrokeopacity{0.000000}%
\pgfsetdash{}{0pt}%
\pgfpathmoveto{\pgfqpoint{4.443130in}{0.500000in}}%
\pgfpathlineto{\pgfqpoint{4.463548in}{0.500000in}}%
\pgfpathlineto{\pgfqpoint{4.463548in}{0.500000in}}%
\pgfpathlineto{\pgfqpoint{4.443130in}{0.500000in}}%
\pgfpathlineto{\pgfqpoint{4.443130in}{0.500000in}}%
\pgfpathclose%
\pgfusepath{fill}%
\end{pgfscope}%
\begin{pgfscope}%
\pgfpathrectangle{\pgfqpoint{0.750000in}{0.500000in}}{\pgfqpoint{4.650000in}{3.020000in}}%
\pgfusepath{clip}%
\pgfsetbuttcap%
\pgfsetmiterjoin%
\definecolor{currentfill}{rgb}{1.000000,0.000000,0.000000}%
\pgfsetfillcolor{currentfill}%
\pgfsetlinewidth{0.000000pt}%
\definecolor{currentstroke}{rgb}{0.000000,0.000000,0.000000}%
\pgfsetstrokecolor{currentstroke}%
\pgfsetstrokeopacity{0.000000}%
\pgfsetdash{}{0pt}%
\pgfpathmoveto{\pgfqpoint{4.463548in}{0.500000in}}%
\pgfpathlineto{\pgfqpoint{4.483965in}{0.500000in}}%
\pgfpathlineto{\pgfqpoint{4.483965in}{0.520918in}}%
\pgfpathlineto{\pgfqpoint{4.463548in}{0.520918in}}%
\pgfpathlineto{\pgfqpoint{4.463548in}{0.500000in}}%
\pgfpathclose%
\pgfusepath{fill}%
\end{pgfscope}%
\begin{pgfscope}%
\pgfpathrectangle{\pgfqpoint{0.750000in}{0.500000in}}{\pgfqpoint{4.650000in}{3.020000in}}%
\pgfusepath{clip}%
\pgfsetbuttcap%
\pgfsetmiterjoin%
\definecolor{currentfill}{rgb}{1.000000,0.000000,0.000000}%
\pgfsetfillcolor{currentfill}%
\pgfsetlinewidth{0.000000pt}%
\definecolor{currentstroke}{rgb}{0.000000,0.000000,0.000000}%
\pgfsetstrokecolor{currentstroke}%
\pgfsetstrokeopacity{0.000000}%
\pgfsetdash{}{0pt}%
\pgfpathmoveto{\pgfqpoint{4.483965in}{0.500000in}}%
\pgfpathlineto{\pgfqpoint{4.504383in}{0.500000in}}%
\pgfpathlineto{\pgfqpoint{4.504383in}{0.500000in}}%
\pgfpathlineto{\pgfqpoint{4.483965in}{0.500000in}}%
\pgfpathlineto{\pgfqpoint{4.483965in}{0.500000in}}%
\pgfpathclose%
\pgfusepath{fill}%
\end{pgfscope}%
\begin{pgfscope}%
\pgfpathrectangle{\pgfqpoint{0.750000in}{0.500000in}}{\pgfqpoint{4.650000in}{3.020000in}}%
\pgfusepath{clip}%
\pgfsetbuttcap%
\pgfsetmiterjoin%
\definecolor{currentfill}{rgb}{1.000000,0.000000,0.000000}%
\pgfsetfillcolor{currentfill}%
\pgfsetlinewidth{0.000000pt}%
\definecolor{currentstroke}{rgb}{0.000000,0.000000,0.000000}%
\pgfsetstrokecolor{currentstroke}%
\pgfsetstrokeopacity{0.000000}%
\pgfsetdash{}{0pt}%
\pgfpathmoveto{\pgfqpoint{4.504383in}{0.500000in}}%
\pgfpathlineto{\pgfqpoint{4.524801in}{0.500000in}}%
\pgfpathlineto{\pgfqpoint{4.524801in}{0.531377in}}%
\pgfpathlineto{\pgfqpoint{4.504383in}{0.531377in}}%
\pgfpathlineto{\pgfqpoint{4.504383in}{0.500000in}}%
\pgfpathclose%
\pgfusepath{fill}%
\end{pgfscope}%
\begin{pgfscope}%
\pgfpathrectangle{\pgfqpoint{0.750000in}{0.500000in}}{\pgfqpoint{4.650000in}{3.020000in}}%
\pgfusepath{clip}%
\pgfsetbuttcap%
\pgfsetmiterjoin%
\definecolor{currentfill}{rgb}{1.000000,0.000000,0.000000}%
\pgfsetfillcolor{currentfill}%
\pgfsetlinewidth{0.000000pt}%
\definecolor{currentstroke}{rgb}{0.000000,0.000000,0.000000}%
\pgfsetstrokecolor{currentstroke}%
\pgfsetstrokeopacity{0.000000}%
\pgfsetdash{}{0pt}%
\pgfpathmoveto{\pgfqpoint{4.524801in}{0.500000in}}%
\pgfpathlineto{\pgfqpoint{4.545219in}{0.500000in}}%
\pgfpathlineto{\pgfqpoint{4.545219in}{0.500000in}}%
\pgfpathlineto{\pgfqpoint{4.524801in}{0.500000in}}%
\pgfpathlineto{\pgfqpoint{4.524801in}{0.500000in}}%
\pgfpathclose%
\pgfusepath{fill}%
\end{pgfscope}%
\begin{pgfscope}%
\pgfpathrectangle{\pgfqpoint{0.750000in}{0.500000in}}{\pgfqpoint{4.650000in}{3.020000in}}%
\pgfusepath{clip}%
\pgfsetbuttcap%
\pgfsetmiterjoin%
\definecolor{currentfill}{rgb}{1.000000,0.000000,0.000000}%
\pgfsetfillcolor{currentfill}%
\pgfsetlinewidth{0.000000pt}%
\definecolor{currentstroke}{rgb}{0.000000,0.000000,0.000000}%
\pgfsetstrokecolor{currentstroke}%
\pgfsetstrokeopacity{0.000000}%
\pgfsetdash{}{0pt}%
\pgfpathmoveto{\pgfqpoint{4.545219in}{0.500000in}}%
\pgfpathlineto{\pgfqpoint{4.565637in}{0.500000in}}%
\pgfpathlineto{\pgfqpoint{4.565637in}{0.500000in}}%
\pgfpathlineto{\pgfqpoint{4.545219in}{0.500000in}}%
\pgfpathlineto{\pgfqpoint{4.545219in}{0.500000in}}%
\pgfpathclose%
\pgfusepath{fill}%
\end{pgfscope}%
\begin{pgfscope}%
\pgfpathrectangle{\pgfqpoint{0.750000in}{0.500000in}}{\pgfqpoint{4.650000in}{3.020000in}}%
\pgfusepath{clip}%
\pgfsetbuttcap%
\pgfsetmiterjoin%
\definecolor{currentfill}{rgb}{1.000000,0.000000,0.000000}%
\pgfsetfillcolor{currentfill}%
\pgfsetlinewidth{0.000000pt}%
\definecolor{currentstroke}{rgb}{0.000000,0.000000,0.000000}%
\pgfsetstrokecolor{currentstroke}%
\pgfsetstrokeopacity{0.000000}%
\pgfsetdash{}{0pt}%
\pgfpathmoveto{\pgfqpoint{4.565637in}{0.500000in}}%
\pgfpathlineto{\pgfqpoint{4.586054in}{0.500000in}}%
\pgfpathlineto{\pgfqpoint{4.586054in}{0.520918in}}%
\pgfpathlineto{\pgfqpoint{4.565637in}{0.520918in}}%
\pgfpathlineto{\pgfqpoint{4.565637in}{0.500000in}}%
\pgfpathclose%
\pgfusepath{fill}%
\end{pgfscope}%
\begin{pgfscope}%
\pgfpathrectangle{\pgfqpoint{0.750000in}{0.500000in}}{\pgfqpoint{4.650000in}{3.020000in}}%
\pgfusepath{clip}%
\pgfsetbuttcap%
\pgfsetmiterjoin%
\definecolor{currentfill}{rgb}{1.000000,0.000000,0.000000}%
\pgfsetfillcolor{currentfill}%
\pgfsetlinewidth{0.000000pt}%
\definecolor{currentstroke}{rgb}{0.000000,0.000000,0.000000}%
\pgfsetstrokecolor{currentstroke}%
\pgfsetstrokeopacity{0.000000}%
\pgfsetdash{}{0pt}%
\pgfpathmoveto{\pgfqpoint{4.586054in}{0.500000in}}%
\pgfpathlineto{\pgfqpoint{4.606472in}{0.500000in}}%
\pgfpathlineto{\pgfqpoint{4.606472in}{0.500000in}}%
\pgfpathlineto{\pgfqpoint{4.586054in}{0.500000in}}%
\pgfpathlineto{\pgfqpoint{4.586054in}{0.500000in}}%
\pgfpathclose%
\pgfusepath{fill}%
\end{pgfscope}%
\begin{pgfscope}%
\pgfpathrectangle{\pgfqpoint{0.750000in}{0.500000in}}{\pgfqpoint{4.650000in}{3.020000in}}%
\pgfusepath{clip}%
\pgfsetbuttcap%
\pgfsetmiterjoin%
\definecolor{currentfill}{rgb}{1.000000,0.000000,0.000000}%
\pgfsetfillcolor{currentfill}%
\pgfsetlinewidth{0.000000pt}%
\definecolor{currentstroke}{rgb}{0.000000,0.000000,0.000000}%
\pgfsetstrokecolor{currentstroke}%
\pgfsetstrokeopacity{0.000000}%
\pgfsetdash{}{0pt}%
\pgfpathmoveto{\pgfqpoint{4.606472in}{0.500000in}}%
\pgfpathlineto{\pgfqpoint{4.626890in}{0.500000in}}%
\pgfpathlineto{\pgfqpoint{4.626890in}{0.520918in}}%
\pgfpathlineto{\pgfqpoint{4.606472in}{0.520918in}}%
\pgfpathlineto{\pgfqpoint{4.606472in}{0.500000in}}%
\pgfpathclose%
\pgfusepath{fill}%
\end{pgfscope}%
\begin{pgfscope}%
\pgfpathrectangle{\pgfqpoint{0.750000in}{0.500000in}}{\pgfqpoint{4.650000in}{3.020000in}}%
\pgfusepath{clip}%
\pgfsetbuttcap%
\pgfsetmiterjoin%
\definecolor{currentfill}{rgb}{1.000000,0.000000,0.000000}%
\pgfsetfillcolor{currentfill}%
\pgfsetlinewidth{0.000000pt}%
\definecolor{currentstroke}{rgb}{0.000000,0.000000,0.000000}%
\pgfsetstrokecolor{currentstroke}%
\pgfsetstrokeopacity{0.000000}%
\pgfsetdash{}{0pt}%
\pgfpathmoveto{\pgfqpoint{4.626890in}{0.500000in}}%
\pgfpathlineto{\pgfqpoint{4.647308in}{0.500000in}}%
\pgfpathlineto{\pgfqpoint{4.647308in}{0.500000in}}%
\pgfpathlineto{\pgfqpoint{4.626890in}{0.500000in}}%
\pgfpathlineto{\pgfqpoint{4.626890in}{0.500000in}}%
\pgfpathclose%
\pgfusepath{fill}%
\end{pgfscope}%
\begin{pgfscope}%
\pgfpathrectangle{\pgfqpoint{0.750000in}{0.500000in}}{\pgfqpoint{4.650000in}{3.020000in}}%
\pgfusepath{clip}%
\pgfsetbuttcap%
\pgfsetmiterjoin%
\definecolor{currentfill}{rgb}{1.000000,0.000000,0.000000}%
\pgfsetfillcolor{currentfill}%
\pgfsetlinewidth{0.000000pt}%
\definecolor{currentstroke}{rgb}{0.000000,0.000000,0.000000}%
\pgfsetstrokecolor{currentstroke}%
\pgfsetstrokeopacity{0.000000}%
\pgfsetdash{}{0pt}%
\pgfpathmoveto{\pgfqpoint{4.647308in}{0.500000in}}%
\pgfpathlineto{\pgfqpoint{4.667726in}{0.500000in}}%
\pgfpathlineto{\pgfqpoint{4.667726in}{0.510459in}}%
\pgfpathlineto{\pgfqpoint{4.647308in}{0.510459in}}%
\pgfpathlineto{\pgfqpoint{4.647308in}{0.500000in}}%
\pgfpathclose%
\pgfusepath{fill}%
\end{pgfscope}%
\begin{pgfscope}%
\pgfpathrectangle{\pgfqpoint{0.750000in}{0.500000in}}{\pgfqpoint{4.650000in}{3.020000in}}%
\pgfusepath{clip}%
\pgfsetbuttcap%
\pgfsetmiterjoin%
\definecolor{currentfill}{rgb}{0.000000,0.500000,0.000000}%
\pgfsetfillcolor{currentfill}%
\pgfsetlinewidth{0.000000pt}%
\definecolor{currentstroke}{rgb}{0.000000,0.000000,0.000000}%
\pgfsetstrokecolor{currentstroke}%
\pgfsetstrokeopacity{0.000000}%
\pgfsetdash{}{0pt}%
\pgfpathmoveto{\pgfqpoint{0.961364in}{0.500000in}}%
\pgfpathlineto{\pgfqpoint{0.994389in}{0.500000in}}%
\pgfpathlineto{\pgfqpoint{0.994389in}{0.510459in}}%
\pgfpathlineto{\pgfqpoint{0.961364in}{0.510459in}}%
\pgfpathlineto{\pgfqpoint{0.961364in}{0.500000in}}%
\pgfpathclose%
\pgfusepath{fill}%
\end{pgfscope}%
\begin{pgfscope}%
\pgfpathrectangle{\pgfqpoint{0.750000in}{0.500000in}}{\pgfqpoint{4.650000in}{3.020000in}}%
\pgfusepath{clip}%
\pgfsetbuttcap%
\pgfsetmiterjoin%
\definecolor{currentfill}{rgb}{0.000000,0.500000,0.000000}%
\pgfsetfillcolor{currentfill}%
\pgfsetlinewidth{0.000000pt}%
\definecolor{currentstroke}{rgb}{0.000000,0.000000,0.000000}%
\pgfsetstrokecolor{currentstroke}%
\pgfsetstrokeopacity{0.000000}%
\pgfsetdash{}{0pt}%
\pgfpathmoveto{\pgfqpoint{0.994389in}{0.500000in}}%
\pgfpathlineto{\pgfqpoint{1.027415in}{0.500000in}}%
\pgfpathlineto{\pgfqpoint{1.027415in}{0.500000in}}%
\pgfpathlineto{\pgfqpoint{0.994389in}{0.500000in}}%
\pgfpathlineto{\pgfqpoint{0.994389in}{0.500000in}}%
\pgfpathclose%
\pgfusepath{fill}%
\end{pgfscope}%
\begin{pgfscope}%
\pgfpathrectangle{\pgfqpoint{0.750000in}{0.500000in}}{\pgfqpoint{4.650000in}{3.020000in}}%
\pgfusepath{clip}%
\pgfsetbuttcap%
\pgfsetmiterjoin%
\definecolor{currentfill}{rgb}{0.000000,0.500000,0.000000}%
\pgfsetfillcolor{currentfill}%
\pgfsetlinewidth{0.000000pt}%
\definecolor{currentstroke}{rgb}{0.000000,0.000000,0.000000}%
\pgfsetstrokecolor{currentstroke}%
\pgfsetstrokeopacity{0.000000}%
\pgfsetdash{}{0pt}%
\pgfpathmoveto{\pgfqpoint{1.027415in}{0.500000in}}%
\pgfpathlineto{\pgfqpoint{1.060440in}{0.500000in}}%
\pgfpathlineto{\pgfqpoint{1.060440in}{0.500000in}}%
\pgfpathlineto{\pgfqpoint{1.027415in}{0.500000in}}%
\pgfpathlineto{\pgfqpoint{1.027415in}{0.500000in}}%
\pgfpathclose%
\pgfusepath{fill}%
\end{pgfscope}%
\begin{pgfscope}%
\pgfpathrectangle{\pgfqpoint{0.750000in}{0.500000in}}{\pgfqpoint{4.650000in}{3.020000in}}%
\pgfusepath{clip}%
\pgfsetbuttcap%
\pgfsetmiterjoin%
\definecolor{currentfill}{rgb}{0.000000,0.500000,0.000000}%
\pgfsetfillcolor{currentfill}%
\pgfsetlinewidth{0.000000pt}%
\definecolor{currentstroke}{rgb}{0.000000,0.000000,0.000000}%
\pgfsetstrokecolor{currentstroke}%
\pgfsetstrokeopacity{0.000000}%
\pgfsetdash{}{0pt}%
\pgfpathmoveto{\pgfqpoint{1.060440in}{0.500000in}}%
\pgfpathlineto{\pgfqpoint{1.093466in}{0.500000in}}%
\pgfpathlineto{\pgfqpoint{1.093466in}{0.500000in}}%
\pgfpathlineto{\pgfqpoint{1.060440in}{0.500000in}}%
\pgfpathlineto{\pgfqpoint{1.060440in}{0.500000in}}%
\pgfpathclose%
\pgfusepath{fill}%
\end{pgfscope}%
\begin{pgfscope}%
\pgfpathrectangle{\pgfqpoint{0.750000in}{0.500000in}}{\pgfqpoint{4.650000in}{3.020000in}}%
\pgfusepath{clip}%
\pgfsetbuttcap%
\pgfsetmiterjoin%
\definecolor{currentfill}{rgb}{0.000000,0.500000,0.000000}%
\pgfsetfillcolor{currentfill}%
\pgfsetlinewidth{0.000000pt}%
\definecolor{currentstroke}{rgb}{0.000000,0.000000,0.000000}%
\pgfsetstrokecolor{currentstroke}%
\pgfsetstrokeopacity{0.000000}%
\pgfsetdash{}{0pt}%
\pgfpathmoveto{\pgfqpoint{1.093466in}{0.500000in}}%
\pgfpathlineto{\pgfqpoint{1.126491in}{0.500000in}}%
\pgfpathlineto{\pgfqpoint{1.126491in}{0.500000in}}%
\pgfpathlineto{\pgfqpoint{1.093466in}{0.500000in}}%
\pgfpathlineto{\pgfqpoint{1.093466in}{0.500000in}}%
\pgfpathclose%
\pgfusepath{fill}%
\end{pgfscope}%
\begin{pgfscope}%
\pgfpathrectangle{\pgfqpoint{0.750000in}{0.500000in}}{\pgfqpoint{4.650000in}{3.020000in}}%
\pgfusepath{clip}%
\pgfsetbuttcap%
\pgfsetmiterjoin%
\definecolor{currentfill}{rgb}{0.000000,0.500000,0.000000}%
\pgfsetfillcolor{currentfill}%
\pgfsetlinewidth{0.000000pt}%
\definecolor{currentstroke}{rgb}{0.000000,0.000000,0.000000}%
\pgfsetstrokecolor{currentstroke}%
\pgfsetstrokeopacity{0.000000}%
\pgfsetdash{}{0pt}%
\pgfpathmoveto{\pgfqpoint{1.126491in}{0.500000in}}%
\pgfpathlineto{\pgfqpoint{1.159517in}{0.500000in}}%
\pgfpathlineto{\pgfqpoint{1.159517in}{0.500000in}}%
\pgfpathlineto{\pgfqpoint{1.126491in}{0.500000in}}%
\pgfpathlineto{\pgfqpoint{1.126491in}{0.500000in}}%
\pgfpathclose%
\pgfusepath{fill}%
\end{pgfscope}%
\begin{pgfscope}%
\pgfpathrectangle{\pgfqpoint{0.750000in}{0.500000in}}{\pgfqpoint{4.650000in}{3.020000in}}%
\pgfusepath{clip}%
\pgfsetbuttcap%
\pgfsetmiterjoin%
\definecolor{currentfill}{rgb}{0.000000,0.500000,0.000000}%
\pgfsetfillcolor{currentfill}%
\pgfsetlinewidth{0.000000pt}%
\definecolor{currentstroke}{rgb}{0.000000,0.000000,0.000000}%
\pgfsetstrokecolor{currentstroke}%
\pgfsetstrokeopacity{0.000000}%
\pgfsetdash{}{0pt}%
\pgfpathmoveto{\pgfqpoint{1.159517in}{0.500000in}}%
\pgfpathlineto{\pgfqpoint{1.192543in}{0.500000in}}%
\pgfpathlineto{\pgfqpoint{1.192543in}{0.500000in}}%
\pgfpathlineto{\pgfqpoint{1.159517in}{0.500000in}}%
\pgfpathlineto{\pgfqpoint{1.159517in}{0.500000in}}%
\pgfpathclose%
\pgfusepath{fill}%
\end{pgfscope}%
\begin{pgfscope}%
\pgfpathrectangle{\pgfqpoint{0.750000in}{0.500000in}}{\pgfqpoint{4.650000in}{3.020000in}}%
\pgfusepath{clip}%
\pgfsetbuttcap%
\pgfsetmiterjoin%
\definecolor{currentfill}{rgb}{0.000000,0.500000,0.000000}%
\pgfsetfillcolor{currentfill}%
\pgfsetlinewidth{0.000000pt}%
\definecolor{currentstroke}{rgb}{0.000000,0.000000,0.000000}%
\pgfsetstrokecolor{currentstroke}%
\pgfsetstrokeopacity{0.000000}%
\pgfsetdash{}{0pt}%
\pgfpathmoveto{\pgfqpoint{1.192543in}{0.500000in}}%
\pgfpathlineto{\pgfqpoint{1.225568in}{0.500000in}}%
\pgfpathlineto{\pgfqpoint{1.225568in}{0.500000in}}%
\pgfpathlineto{\pgfqpoint{1.192543in}{0.500000in}}%
\pgfpathlineto{\pgfqpoint{1.192543in}{0.500000in}}%
\pgfpathclose%
\pgfusepath{fill}%
\end{pgfscope}%
\begin{pgfscope}%
\pgfpathrectangle{\pgfqpoint{0.750000in}{0.500000in}}{\pgfqpoint{4.650000in}{3.020000in}}%
\pgfusepath{clip}%
\pgfsetbuttcap%
\pgfsetmiterjoin%
\definecolor{currentfill}{rgb}{0.000000,0.500000,0.000000}%
\pgfsetfillcolor{currentfill}%
\pgfsetlinewidth{0.000000pt}%
\definecolor{currentstroke}{rgb}{0.000000,0.000000,0.000000}%
\pgfsetstrokecolor{currentstroke}%
\pgfsetstrokeopacity{0.000000}%
\pgfsetdash{}{0pt}%
\pgfpathmoveto{\pgfqpoint{1.225568in}{0.500000in}}%
\pgfpathlineto{\pgfqpoint{1.258594in}{0.500000in}}%
\pgfpathlineto{\pgfqpoint{1.258594in}{0.500000in}}%
\pgfpathlineto{\pgfqpoint{1.225568in}{0.500000in}}%
\pgfpathlineto{\pgfqpoint{1.225568in}{0.500000in}}%
\pgfpathclose%
\pgfusepath{fill}%
\end{pgfscope}%
\begin{pgfscope}%
\pgfpathrectangle{\pgfqpoint{0.750000in}{0.500000in}}{\pgfqpoint{4.650000in}{3.020000in}}%
\pgfusepath{clip}%
\pgfsetbuttcap%
\pgfsetmiterjoin%
\definecolor{currentfill}{rgb}{0.000000,0.500000,0.000000}%
\pgfsetfillcolor{currentfill}%
\pgfsetlinewidth{0.000000pt}%
\definecolor{currentstroke}{rgb}{0.000000,0.000000,0.000000}%
\pgfsetstrokecolor{currentstroke}%
\pgfsetstrokeopacity{0.000000}%
\pgfsetdash{}{0pt}%
\pgfpathmoveto{\pgfqpoint{1.258594in}{0.500000in}}%
\pgfpathlineto{\pgfqpoint{1.291619in}{0.500000in}}%
\pgfpathlineto{\pgfqpoint{1.291619in}{0.500000in}}%
\pgfpathlineto{\pgfqpoint{1.258594in}{0.500000in}}%
\pgfpathlineto{\pgfqpoint{1.258594in}{0.500000in}}%
\pgfpathclose%
\pgfusepath{fill}%
\end{pgfscope}%
\begin{pgfscope}%
\pgfpathrectangle{\pgfqpoint{0.750000in}{0.500000in}}{\pgfqpoint{4.650000in}{3.020000in}}%
\pgfusepath{clip}%
\pgfsetbuttcap%
\pgfsetmiterjoin%
\definecolor{currentfill}{rgb}{0.000000,0.500000,0.000000}%
\pgfsetfillcolor{currentfill}%
\pgfsetlinewidth{0.000000pt}%
\definecolor{currentstroke}{rgb}{0.000000,0.000000,0.000000}%
\pgfsetstrokecolor{currentstroke}%
\pgfsetstrokeopacity{0.000000}%
\pgfsetdash{}{0pt}%
\pgfpathmoveto{\pgfqpoint{1.291619in}{0.500000in}}%
\pgfpathlineto{\pgfqpoint{1.324645in}{0.500000in}}%
\pgfpathlineto{\pgfqpoint{1.324645in}{0.500000in}}%
\pgfpathlineto{\pgfqpoint{1.291619in}{0.500000in}}%
\pgfpathlineto{\pgfqpoint{1.291619in}{0.500000in}}%
\pgfpathclose%
\pgfusepath{fill}%
\end{pgfscope}%
\begin{pgfscope}%
\pgfpathrectangle{\pgfqpoint{0.750000in}{0.500000in}}{\pgfqpoint{4.650000in}{3.020000in}}%
\pgfusepath{clip}%
\pgfsetbuttcap%
\pgfsetmiterjoin%
\definecolor{currentfill}{rgb}{0.000000,0.500000,0.000000}%
\pgfsetfillcolor{currentfill}%
\pgfsetlinewidth{0.000000pt}%
\definecolor{currentstroke}{rgb}{0.000000,0.000000,0.000000}%
\pgfsetstrokecolor{currentstroke}%
\pgfsetstrokeopacity{0.000000}%
\pgfsetdash{}{0pt}%
\pgfpathmoveto{\pgfqpoint{1.324645in}{0.500000in}}%
\pgfpathlineto{\pgfqpoint{1.357670in}{0.500000in}}%
\pgfpathlineto{\pgfqpoint{1.357670in}{0.500000in}}%
\pgfpathlineto{\pgfqpoint{1.324645in}{0.500000in}}%
\pgfpathlineto{\pgfqpoint{1.324645in}{0.500000in}}%
\pgfpathclose%
\pgfusepath{fill}%
\end{pgfscope}%
\begin{pgfscope}%
\pgfpathrectangle{\pgfqpoint{0.750000in}{0.500000in}}{\pgfqpoint{4.650000in}{3.020000in}}%
\pgfusepath{clip}%
\pgfsetbuttcap%
\pgfsetmiterjoin%
\definecolor{currentfill}{rgb}{0.000000,0.500000,0.000000}%
\pgfsetfillcolor{currentfill}%
\pgfsetlinewidth{0.000000pt}%
\definecolor{currentstroke}{rgb}{0.000000,0.000000,0.000000}%
\pgfsetstrokecolor{currentstroke}%
\pgfsetstrokeopacity{0.000000}%
\pgfsetdash{}{0pt}%
\pgfpathmoveto{\pgfqpoint{1.357670in}{0.500000in}}%
\pgfpathlineto{\pgfqpoint{1.390696in}{0.500000in}}%
\pgfpathlineto{\pgfqpoint{1.390696in}{0.510459in}}%
\pgfpathlineto{\pgfqpoint{1.357670in}{0.510459in}}%
\pgfpathlineto{\pgfqpoint{1.357670in}{0.500000in}}%
\pgfpathclose%
\pgfusepath{fill}%
\end{pgfscope}%
\begin{pgfscope}%
\pgfpathrectangle{\pgfqpoint{0.750000in}{0.500000in}}{\pgfqpoint{4.650000in}{3.020000in}}%
\pgfusepath{clip}%
\pgfsetbuttcap%
\pgfsetmiterjoin%
\definecolor{currentfill}{rgb}{0.000000,0.500000,0.000000}%
\pgfsetfillcolor{currentfill}%
\pgfsetlinewidth{0.000000pt}%
\definecolor{currentstroke}{rgb}{0.000000,0.000000,0.000000}%
\pgfsetstrokecolor{currentstroke}%
\pgfsetstrokeopacity{0.000000}%
\pgfsetdash{}{0pt}%
\pgfpathmoveto{\pgfqpoint{1.390696in}{0.500000in}}%
\pgfpathlineto{\pgfqpoint{1.423722in}{0.500000in}}%
\pgfpathlineto{\pgfqpoint{1.423722in}{0.500000in}}%
\pgfpathlineto{\pgfqpoint{1.390696in}{0.500000in}}%
\pgfpathlineto{\pgfqpoint{1.390696in}{0.500000in}}%
\pgfpathclose%
\pgfusepath{fill}%
\end{pgfscope}%
\begin{pgfscope}%
\pgfpathrectangle{\pgfqpoint{0.750000in}{0.500000in}}{\pgfqpoint{4.650000in}{3.020000in}}%
\pgfusepath{clip}%
\pgfsetbuttcap%
\pgfsetmiterjoin%
\definecolor{currentfill}{rgb}{0.000000,0.500000,0.000000}%
\pgfsetfillcolor{currentfill}%
\pgfsetlinewidth{0.000000pt}%
\definecolor{currentstroke}{rgb}{0.000000,0.000000,0.000000}%
\pgfsetstrokecolor{currentstroke}%
\pgfsetstrokeopacity{0.000000}%
\pgfsetdash{}{0pt}%
\pgfpathmoveto{\pgfqpoint{1.423722in}{0.500000in}}%
\pgfpathlineto{\pgfqpoint{1.456747in}{0.500000in}}%
\pgfpathlineto{\pgfqpoint{1.456747in}{0.500000in}}%
\pgfpathlineto{\pgfqpoint{1.423722in}{0.500000in}}%
\pgfpathlineto{\pgfqpoint{1.423722in}{0.500000in}}%
\pgfpathclose%
\pgfusepath{fill}%
\end{pgfscope}%
\begin{pgfscope}%
\pgfpathrectangle{\pgfqpoint{0.750000in}{0.500000in}}{\pgfqpoint{4.650000in}{3.020000in}}%
\pgfusepath{clip}%
\pgfsetbuttcap%
\pgfsetmiterjoin%
\definecolor{currentfill}{rgb}{0.000000,0.500000,0.000000}%
\pgfsetfillcolor{currentfill}%
\pgfsetlinewidth{0.000000pt}%
\definecolor{currentstroke}{rgb}{0.000000,0.000000,0.000000}%
\pgfsetstrokecolor{currentstroke}%
\pgfsetstrokeopacity{0.000000}%
\pgfsetdash{}{0pt}%
\pgfpathmoveto{\pgfqpoint{1.456747in}{0.500000in}}%
\pgfpathlineto{\pgfqpoint{1.489773in}{0.500000in}}%
\pgfpathlineto{\pgfqpoint{1.489773in}{0.500000in}}%
\pgfpathlineto{\pgfqpoint{1.456747in}{0.500000in}}%
\pgfpathlineto{\pgfqpoint{1.456747in}{0.500000in}}%
\pgfpathclose%
\pgfusepath{fill}%
\end{pgfscope}%
\begin{pgfscope}%
\pgfpathrectangle{\pgfqpoint{0.750000in}{0.500000in}}{\pgfqpoint{4.650000in}{3.020000in}}%
\pgfusepath{clip}%
\pgfsetbuttcap%
\pgfsetmiterjoin%
\definecolor{currentfill}{rgb}{0.000000,0.500000,0.000000}%
\pgfsetfillcolor{currentfill}%
\pgfsetlinewidth{0.000000pt}%
\definecolor{currentstroke}{rgb}{0.000000,0.000000,0.000000}%
\pgfsetstrokecolor{currentstroke}%
\pgfsetstrokeopacity{0.000000}%
\pgfsetdash{}{0pt}%
\pgfpathmoveto{\pgfqpoint{1.489773in}{0.500000in}}%
\pgfpathlineto{\pgfqpoint{1.522798in}{0.500000in}}%
\pgfpathlineto{\pgfqpoint{1.522798in}{0.500000in}}%
\pgfpathlineto{\pgfqpoint{1.489773in}{0.500000in}}%
\pgfpathlineto{\pgfqpoint{1.489773in}{0.500000in}}%
\pgfpathclose%
\pgfusepath{fill}%
\end{pgfscope}%
\begin{pgfscope}%
\pgfpathrectangle{\pgfqpoint{0.750000in}{0.500000in}}{\pgfqpoint{4.650000in}{3.020000in}}%
\pgfusepath{clip}%
\pgfsetbuttcap%
\pgfsetmiterjoin%
\definecolor{currentfill}{rgb}{0.000000,0.500000,0.000000}%
\pgfsetfillcolor{currentfill}%
\pgfsetlinewidth{0.000000pt}%
\definecolor{currentstroke}{rgb}{0.000000,0.000000,0.000000}%
\pgfsetstrokecolor{currentstroke}%
\pgfsetstrokeopacity{0.000000}%
\pgfsetdash{}{0pt}%
\pgfpathmoveto{\pgfqpoint{1.522798in}{0.500000in}}%
\pgfpathlineto{\pgfqpoint{1.555824in}{0.500000in}}%
\pgfpathlineto{\pgfqpoint{1.555824in}{0.510459in}}%
\pgfpathlineto{\pgfqpoint{1.522798in}{0.510459in}}%
\pgfpathlineto{\pgfqpoint{1.522798in}{0.500000in}}%
\pgfpathclose%
\pgfusepath{fill}%
\end{pgfscope}%
\begin{pgfscope}%
\pgfpathrectangle{\pgfqpoint{0.750000in}{0.500000in}}{\pgfqpoint{4.650000in}{3.020000in}}%
\pgfusepath{clip}%
\pgfsetbuttcap%
\pgfsetmiterjoin%
\definecolor{currentfill}{rgb}{0.000000,0.500000,0.000000}%
\pgfsetfillcolor{currentfill}%
\pgfsetlinewidth{0.000000pt}%
\definecolor{currentstroke}{rgb}{0.000000,0.000000,0.000000}%
\pgfsetstrokecolor{currentstroke}%
\pgfsetstrokeopacity{0.000000}%
\pgfsetdash{}{0pt}%
\pgfpathmoveto{\pgfqpoint{1.555824in}{0.500000in}}%
\pgfpathlineto{\pgfqpoint{1.588849in}{0.500000in}}%
\pgfpathlineto{\pgfqpoint{1.588849in}{0.510459in}}%
\pgfpathlineto{\pgfqpoint{1.555824in}{0.510459in}}%
\pgfpathlineto{\pgfqpoint{1.555824in}{0.500000in}}%
\pgfpathclose%
\pgfusepath{fill}%
\end{pgfscope}%
\begin{pgfscope}%
\pgfpathrectangle{\pgfqpoint{0.750000in}{0.500000in}}{\pgfqpoint{4.650000in}{3.020000in}}%
\pgfusepath{clip}%
\pgfsetbuttcap%
\pgfsetmiterjoin%
\definecolor{currentfill}{rgb}{0.000000,0.500000,0.000000}%
\pgfsetfillcolor{currentfill}%
\pgfsetlinewidth{0.000000pt}%
\definecolor{currentstroke}{rgb}{0.000000,0.000000,0.000000}%
\pgfsetstrokecolor{currentstroke}%
\pgfsetstrokeopacity{0.000000}%
\pgfsetdash{}{0pt}%
\pgfpathmoveto{\pgfqpoint{1.588849in}{0.500000in}}%
\pgfpathlineto{\pgfqpoint{1.621875in}{0.500000in}}%
\pgfpathlineto{\pgfqpoint{1.621875in}{0.510459in}}%
\pgfpathlineto{\pgfqpoint{1.588849in}{0.510459in}}%
\pgfpathlineto{\pgfqpoint{1.588849in}{0.500000in}}%
\pgfpathclose%
\pgfusepath{fill}%
\end{pgfscope}%
\begin{pgfscope}%
\pgfpathrectangle{\pgfqpoint{0.750000in}{0.500000in}}{\pgfqpoint{4.650000in}{3.020000in}}%
\pgfusepath{clip}%
\pgfsetbuttcap%
\pgfsetmiterjoin%
\definecolor{currentfill}{rgb}{0.000000,0.500000,0.000000}%
\pgfsetfillcolor{currentfill}%
\pgfsetlinewidth{0.000000pt}%
\definecolor{currentstroke}{rgb}{0.000000,0.000000,0.000000}%
\pgfsetstrokecolor{currentstroke}%
\pgfsetstrokeopacity{0.000000}%
\pgfsetdash{}{0pt}%
\pgfpathmoveto{\pgfqpoint{1.621875in}{0.500000in}}%
\pgfpathlineto{\pgfqpoint{1.654901in}{0.500000in}}%
\pgfpathlineto{\pgfqpoint{1.654901in}{0.500000in}}%
\pgfpathlineto{\pgfqpoint{1.621875in}{0.500000in}}%
\pgfpathlineto{\pgfqpoint{1.621875in}{0.500000in}}%
\pgfpathclose%
\pgfusepath{fill}%
\end{pgfscope}%
\begin{pgfscope}%
\pgfpathrectangle{\pgfqpoint{0.750000in}{0.500000in}}{\pgfqpoint{4.650000in}{3.020000in}}%
\pgfusepath{clip}%
\pgfsetbuttcap%
\pgfsetmiterjoin%
\definecolor{currentfill}{rgb}{0.000000,0.500000,0.000000}%
\pgfsetfillcolor{currentfill}%
\pgfsetlinewidth{0.000000pt}%
\definecolor{currentstroke}{rgb}{0.000000,0.000000,0.000000}%
\pgfsetstrokecolor{currentstroke}%
\pgfsetstrokeopacity{0.000000}%
\pgfsetdash{}{0pt}%
\pgfpathmoveto{\pgfqpoint{1.654901in}{0.500000in}}%
\pgfpathlineto{\pgfqpoint{1.687926in}{0.500000in}}%
\pgfpathlineto{\pgfqpoint{1.687926in}{0.500000in}}%
\pgfpathlineto{\pgfqpoint{1.654901in}{0.500000in}}%
\pgfpathlineto{\pgfqpoint{1.654901in}{0.500000in}}%
\pgfpathclose%
\pgfusepath{fill}%
\end{pgfscope}%
\begin{pgfscope}%
\pgfpathrectangle{\pgfqpoint{0.750000in}{0.500000in}}{\pgfqpoint{4.650000in}{3.020000in}}%
\pgfusepath{clip}%
\pgfsetbuttcap%
\pgfsetmiterjoin%
\definecolor{currentfill}{rgb}{0.000000,0.500000,0.000000}%
\pgfsetfillcolor{currentfill}%
\pgfsetlinewidth{0.000000pt}%
\definecolor{currentstroke}{rgb}{0.000000,0.000000,0.000000}%
\pgfsetstrokecolor{currentstroke}%
\pgfsetstrokeopacity{0.000000}%
\pgfsetdash{}{0pt}%
\pgfpathmoveto{\pgfqpoint{1.687926in}{0.500000in}}%
\pgfpathlineto{\pgfqpoint{1.720952in}{0.500000in}}%
\pgfpathlineto{\pgfqpoint{1.720952in}{0.510459in}}%
\pgfpathlineto{\pgfqpoint{1.687926in}{0.510459in}}%
\pgfpathlineto{\pgfqpoint{1.687926in}{0.500000in}}%
\pgfpathclose%
\pgfusepath{fill}%
\end{pgfscope}%
\begin{pgfscope}%
\pgfpathrectangle{\pgfqpoint{0.750000in}{0.500000in}}{\pgfqpoint{4.650000in}{3.020000in}}%
\pgfusepath{clip}%
\pgfsetbuttcap%
\pgfsetmiterjoin%
\definecolor{currentfill}{rgb}{0.000000,0.500000,0.000000}%
\pgfsetfillcolor{currentfill}%
\pgfsetlinewidth{0.000000pt}%
\definecolor{currentstroke}{rgb}{0.000000,0.000000,0.000000}%
\pgfsetstrokecolor{currentstroke}%
\pgfsetstrokeopacity{0.000000}%
\pgfsetdash{}{0pt}%
\pgfpathmoveto{\pgfqpoint{1.720952in}{0.500000in}}%
\pgfpathlineto{\pgfqpoint{1.753977in}{0.500000in}}%
\pgfpathlineto{\pgfqpoint{1.753977in}{0.510459in}}%
\pgfpathlineto{\pgfqpoint{1.720952in}{0.510459in}}%
\pgfpathlineto{\pgfqpoint{1.720952in}{0.500000in}}%
\pgfpathclose%
\pgfusepath{fill}%
\end{pgfscope}%
\begin{pgfscope}%
\pgfpathrectangle{\pgfqpoint{0.750000in}{0.500000in}}{\pgfqpoint{4.650000in}{3.020000in}}%
\pgfusepath{clip}%
\pgfsetbuttcap%
\pgfsetmiterjoin%
\definecolor{currentfill}{rgb}{0.000000,0.500000,0.000000}%
\pgfsetfillcolor{currentfill}%
\pgfsetlinewidth{0.000000pt}%
\definecolor{currentstroke}{rgb}{0.000000,0.000000,0.000000}%
\pgfsetstrokecolor{currentstroke}%
\pgfsetstrokeopacity{0.000000}%
\pgfsetdash{}{0pt}%
\pgfpathmoveto{\pgfqpoint{1.753977in}{0.500000in}}%
\pgfpathlineto{\pgfqpoint{1.787003in}{0.500000in}}%
\pgfpathlineto{\pgfqpoint{1.787003in}{0.520918in}}%
\pgfpathlineto{\pgfqpoint{1.753977in}{0.520918in}}%
\pgfpathlineto{\pgfqpoint{1.753977in}{0.500000in}}%
\pgfpathclose%
\pgfusepath{fill}%
\end{pgfscope}%
\begin{pgfscope}%
\pgfpathrectangle{\pgfqpoint{0.750000in}{0.500000in}}{\pgfqpoint{4.650000in}{3.020000in}}%
\pgfusepath{clip}%
\pgfsetbuttcap%
\pgfsetmiterjoin%
\definecolor{currentfill}{rgb}{0.000000,0.500000,0.000000}%
\pgfsetfillcolor{currentfill}%
\pgfsetlinewidth{0.000000pt}%
\definecolor{currentstroke}{rgb}{0.000000,0.000000,0.000000}%
\pgfsetstrokecolor{currentstroke}%
\pgfsetstrokeopacity{0.000000}%
\pgfsetdash{}{0pt}%
\pgfpathmoveto{\pgfqpoint{1.787003in}{0.500000in}}%
\pgfpathlineto{\pgfqpoint{1.820028in}{0.500000in}}%
\pgfpathlineto{\pgfqpoint{1.820028in}{0.510459in}}%
\pgfpathlineto{\pgfqpoint{1.787003in}{0.510459in}}%
\pgfpathlineto{\pgfqpoint{1.787003in}{0.500000in}}%
\pgfpathclose%
\pgfusepath{fill}%
\end{pgfscope}%
\begin{pgfscope}%
\pgfpathrectangle{\pgfqpoint{0.750000in}{0.500000in}}{\pgfqpoint{4.650000in}{3.020000in}}%
\pgfusepath{clip}%
\pgfsetbuttcap%
\pgfsetmiterjoin%
\definecolor{currentfill}{rgb}{0.000000,0.500000,0.000000}%
\pgfsetfillcolor{currentfill}%
\pgfsetlinewidth{0.000000pt}%
\definecolor{currentstroke}{rgb}{0.000000,0.000000,0.000000}%
\pgfsetstrokecolor{currentstroke}%
\pgfsetstrokeopacity{0.000000}%
\pgfsetdash{}{0pt}%
\pgfpathmoveto{\pgfqpoint{1.820028in}{0.500000in}}%
\pgfpathlineto{\pgfqpoint{1.853054in}{0.500000in}}%
\pgfpathlineto{\pgfqpoint{1.853054in}{0.510459in}}%
\pgfpathlineto{\pgfqpoint{1.820028in}{0.510459in}}%
\pgfpathlineto{\pgfqpoint{1.820028in}{0.500000in}}%
\pgfpathclose%
\pgfusepath{fill}%
\end{pgfscope}%
\begin{pgfscope}%
\pgfpathrectangle{\pgfqpoint{0.750000in}{0.500000in}}{\pgfqpoint{4.650000in}{3.020000in}}%
\pgfusepath{clip}%
\pgfsetbuttcap%
\pgfsetmiterjoin%
\definecolor{currentfill}{rgb}{0.000000,0.500000,0.000000}%
\pgfsetfillcolor{currentfill}%
\pgfsetlinewidth{0.000000pt}%
\definecolor{currentstroke}{rgb}{0.000000,0.000000,0.000000}%
\pgfsetstrokecolor{currentstroke}%
\pgfsetstrokeopacity{0.000000}%
\pgfsetdash{}{0pt}%
\pgfpathmoveto{\pgfqpoint{1.853054in}{0.500000in}}%
\pgfpathlineto{\pgfqpoint{1.886080in}{0.500000in}}%
\pgfpathlineto{\pgfqpoint{1.886080in}{0.500000in}}%
\pgfpathlineto{\pgfqpoint{1.853054in}{0.500000in}}%
\pgfpathlineto{\pgfqpoint{1.853054in}{0.500000in}}%
\pgfpathclose%
\pgfusepath{fill}%
\end{pgfscope}%
\begin{pgfscope}%
\pgfpathrectangle{\pgfqpoint{0.750000in}{0.500000in}}{\pgfqpoint{4.650000in}{3.020000in}}%
\pgfusepath{clip}%
\pgfsetbuttcap%
\pgfsetmiterjoin%
\definecolor{currentfill}{rgb}{0.000000,0.500000,0.000000}%
\pgfsetfillcolor{currentfill}%
\pgfsetlinewidth{0.000000pt}%
\definecolor{currentstroke}{rgb}{0.000000,0.000000,0.000000}%
\pgfsetstrokecolor{currentstroke}%
\pgfsetstrokeopacity{0.000000}%
\pgfsetdash{}{0pt}%
\pgfpathmoveto{\pgfqpoint{1.886080in}{0.500000in}}%
\pgfpathlineto{\pgfqpoint{1.919105in}{0.500000in}}%
\pgfpathlineto{\pgfqpoint{1.919105in}{0.510459in}}%
\pgfpathlineto{\pgfqpoint{1.886080in}{0.510459in}}%
\pgfpathlineto{\pgfqpoint{1.886080in}{0.500000in}}%
\pgfpathclose%
\pgfusepath{fill}%
\end{pgfscope}%
\begin{pgfscope}%
\pgfpathrectangle{\pgfqpoint{0.750000in}{0.500000in}}{\pgfqpoint{4.650000in}{3.020000in}}%
\pgfusepath{clip}%
\pgfsetbuttcap%
\pgfsetmiterjoin%
\definecolor{currentfill}{rgb}{0.000000,0.500000,0.000000}%
\pgfsetfillcolor{currentfill}%
\pgfsetlinewidth{0.000000pt}%
\definecolor{currentstroke}{rgb}{0.000000,0.000000,0.000000}%
\pgfsetstrokecolor{currentstroke}%
\pgfsetstrokeopacity{0.000000}%
\pgfsetdash{}{0pt}%
\pgfpathmoveto{\pgfqpoint{1.919105in}{0.500000in}}%
\pgfpathlineto{\pgfqpoint{1.952131in}{0.500000in}}%
\pgfpathlineto{\pgfqpoint{1.952131in}{0.510459in}}%
\pgfpathlineto{\pgfqpoint{1.919105in}{0.510459in}}%
\pgfpathlineto{\pgfqpoint{1.919105in}{0.500000in}}%
\pgfpathclose%
\pgfusepath{fill}%
\end{pgfscope}%
\begin{pgfscope}%
\pgfpathrectangle{\pgfqpoint{0.750000in}{0.500000in}}{\pgfqpoint{4.650000in}{3.020000in}}%
\pgfusepath{clip}%
\pgfsetbuttcap%
\pgfsetmiterjoin%
\definecolor{currentfill}{rgb}{0.000000,0.500000,0.000000}%
\pgfsetfillcolor{currentfill}%
\pgfsetlinewidth{0.000000pt}%
\definecolor{currentstroke}{rgb}{0.000000,0.000000,0.000000}%
\pgfsetstrokecolor{currentstroke}%
\pgfsetstrokeopacity{0.000000}%
\pgfsetdash{}{0pt}%
\pgfpathmoveto{\pgfqpoint{1.952131in}{0.500000in}}%
\pgfpathlineto{\pgfqpoint{1.985156in}{0.500000in}}%
\pgfpathlineto{\pgfqpoint{1.985156in}{0.520918in}}%
\pgfpathlineto{\pgfqpoint{1.952131in}{0.520918in}}%
\pgfpathlineto{\pgfqpoint{1.952131in}{0.500000in}}%
\pgfpathclose%
\pgfusepath{fill}%
\end{pgfscope}%
\begin{pgfscope}%
\pgfpathrectangle{\pgfqpoint{0.750000in}{0.500000in}}{\pgfqpoint{4.650000in}{3.020000in}}%
\pgfusepath{clip}%
\pgfsetbuttcap%
\pgfsetmiterjoin%
\definecolor{currentfill}{rgb}{0.000000,0.500000,0.000000}%
\pgfsetfillcolor{currentfill}%
\pgfsetlinewidth{0.000000pt}%
\definecolor{currentstroke}{rgb}{0.000000,0.000000,0.000000}%
\pgfsetstrokecolor{currentstroke}%
\pgfsetstrokeopacity{0.000000}%
\pgfsetdash{}{0pt}%
\pgfpathmoveto{\pgfqpoint{1.985156in}{0.500000in}}%
\pgfpathlineto{\pgfqpoint{2.018182in}{0.500000in}}%
\pgfpathlineto{\pgfqpoint{2.018182in}{0.541835in}}%
\pgfpathlineto{\pgfqpoint{1.985156in}{0.541835in}}%
\pgfpathlineto{\pgfqpoint{1.985156in}{0.500000in}}%
\pgfpathclose%
\pgfusepath{fill}%
\end{pgfscope}%
\begin{pgfscope}%
\pgfpathrectangle{\pgfqpoint{0.750000in}{0.500000in}}{\pgfqpoint{4.650000in}{3.020000in}}%
\pgfusepath{clip}%
\pgfsetbuttcap%
\pgfsetmiterjoin%
\definecolor{currentfill}{rgb}{0.000000,0.500000,0.000000}%
\pgfsetfillcolor{currentfill}%
\pgfsetlinewidth{0.000000pt}%
\definecolor{currentstroke}{rgb}{0.000000,0.000000,0.000000}%
\pgfsetstrokecolor{currentstroke}%
\pgfsetstrokeopacity{0.000000}%
\pgfsetdash{}{0pt}%
\pgfpathmoveto{\pgfqpoint{2.018182in}{0.500000in}}%
\pgfpathlineto{\pgfqpoint{2.051207in}{0.500000in}}%
\pgfpathlineto{\pgfqpoint{2.051207in}{0.552294in}}%
\pgfpathlineto{\pgfqpoint{2.018182in}{0.552294in}}%
\pgfpathlineto{\pgfqpoint{2.018182in}{0.500000in}}%
\pgfpathclose%
\pgfusepath{fill}%
\end{pgfscope}%
\begin{pgfscope}%
\pgfpathrectangle{\pgfqpoint{0.750000in}{0.500000in}}{\pgfqpoint{4.650000in}{3.020000in}}%
\pgfusepath{clip}%
\pgfsetbuttcap%
\pgfsetmiterjoin%
\definecolor{currentfill}{rgb}{0.000000,0.500000,0.000000}%
\pgfsetfillcolor{currentfill}%
\pgfsetlinewidth{0.000000pt}%
\definecolor{currentstroke}{rgb}{0.000000,0.000000,0.000000}%
\pgfsetstrokecolor{currentstroke}%
\pgfsetstrokeopacity{0.000000}%
\pgfsetdash{}{0pt}%
\pgfpathmoveto{\pgfqpoint{2.051207in}{0.500000in}}%
\pgfpathlineto{\pgfqpoint{2.084233in}{0.500000in}}%
\pgfpathlineto{\pgfqpoint{2.084233in}{0.520918in}}%
\pgfpathlineto{\pgfqpoint{2.051207in}{0.520918in}}%
\pgfpathlineto{\pgfqpoint{2.051207in}{0.500000in}}%
\pgfpathclose%
\pgfusepath{fill}%
\end{pgfscope}%
\begin{pgfscope}%
\pgfpathrectangle{\pgfqpoint{0.750000in}{0.500000in}}{\pgfqpoint{4.650000in}{3.020000in}}%
\pgfusepath{clip}%
\pgfsetbuttcap%
\pgfsetmiterjoin%
\definecolor{currentfill}{rgb}{0.000000,0.500000,0.000000}%
\pgfsetfillcolor{currentfill}%
\pgfsetlinewidth{0.000000pt}%
\definecolor{currentstroke}{rgb}{0.000000,0.000000,0.000000}%
\pgfsetstrokecolor{currentstroke}%
\pgfsetstrokeopacity{0.000000}%
\pgfsetdash{}{0pt}%
\pgfpathmoveto{\pgfqpoint{2.084233in}{0.500000in}}%
\pgfpathlineto{\pgfqpoint{2.117259in}{0.500000in}}%
\pgfpathlineto{\pgfqpoint{2.117259in}{0.552294in}}%
\pgfpathlineto{\pgfqpoint{2.084233in}{0.552294in}}%
\pgfpathlineto{\pgfqpoint{2.084233in}{0.500000in}}%
\pgfpathclose%
\pgfusepath{fill}%
\end{pgfscope}%
\begin{pgfscope}%
\pgfpathrectangle{\pgfqpoint{0.750000in}{0.500000in}}{\pgfqpoint{4.650000in}{3.020000in}}%
\pgfusepath{clip}%
\pgfsetbuttcap%
\pgfsetmiterjoin%
\definecolor{currentfill}{rgb}{0.000000,0.500000,0.000000}%
\pgfsetfillcolor{currentfill}%
\pgfsetlinewidth{0.000000pt}%
\definecolor{currentstroke}{rgb}{0.000000,0.000000,0.000000}%
\pgfsetstrokecolor{currentstroke}%
\pgfsetstrokeopacity{0.000000}%
\pgfsetdash{}{0pt}%
\pgfpathmoveto{\pgfqpoint{2.117259in}{0.500000in}}%
\pgfpathlineto{\pgfqpoint{2.150284in}{0.500000in}}%
\pgfpathlineto{\pgfqpoint{2.150284in}{0.552294in}}%
\pgfpathlineto{\pgfqpoint{2.117259in}{0.552294in}}%
\pgfpathlineto{\pgfqpoint{2.117259in}{0.500000in}}%
\pgfpathclose%
\pgfusepath{fill}%
\end{pgfscope}%
\begin{pgfscope}%
\pgfpathrectangle{\pgfqpoint{0.750000in}{0.500000in}}{\pgfqpoint{4.650000in}{3.020000in}}%
\pgfusepath{clip}%
\pgfsetbuttcap%
\pgfsetmiterjoin%
\definecolor{currentfill}{rgb}{0.000000,0.500000,0.000000}%
\pgfsetfillcolor{currentfill}%
\pgfsetlinewidth{0.000000pt}%
\definecolor{currentstroke}{rgb}{0.000000,0.000000,0.000000}%
\pgfsetstrokecolor{currentstroke}%
\pgfsetstrokeopacity{0.000000}%
\pgfsetdash{}{0pt}%
\pgfpathmoveto{\pgfqpoint{2.150284in}{0.500000in}}%
\pgfpathlineto{\pgfqpoint{2.183310in}{0.500000in}}%
\pgfpathlineto{\pgfqpoint{2.183310in}{0.594130in}}%
\pgfpathlineto{\pgfqpoint{2.150284in}{0.594130in}}%
\pgfpathlineto{\pgfqpoint{2.150284in}{0.500000in}}%
\pgfpathclose%
\pgfusepath{fill}%
\end{pgfscope}%
\begin{pgfscope}%
\pgfpathrectangle{\pgfqpoint{0.750000in}{0.500000in}}{\pgfqpoint{4.650000in}{3.020000in}}%
\pgfusepath{clip}%
\pgfsetbuttcap%
\pgfsetmiterjoin%
\definecolor{currentfill}{rgb}{0.000000,0.500000,0.000000}%
\pgfsetfillcolor{currentfill}%
\pgfsetlinewidth{0.000000pt}%
\definecolor{currentstroke}{rgb}{0.000000,0.000000,0.000000}%
\pgfsetstrokecolor{currentstroke}%
\pgfsetstrokeopacity{0.000000}%
\pgfsetdash{}{0pt}%
\pgfpathmoveto{\pgfqpoint{2.183310in}{0.500000in}}%
\pgfpathlineto{\pgfqpoint{2.216335in}{0.500000in}}%
\pgfpathlineto{\pgfqpoint{2.216335in}{0.604589in}}%
\pgfpathlineto{\pgfqpoint{2.183310in}{0.604589in}}%
\pgfpathlineto{\pgfqpoint{2.183310in}{0.500000in}}%
\pgfpathclose%
\pgfusepath{fill}%
\end{pgfscope}%
\begin{pgfscope}%
\pgfpathrectangle{\pgfqpoint{0.750000in}{0.500000in}}{\pgfqpoint{4.650000in}{3.020000in}}%
\pgfusepath{clip}%
\pgfsetbuttcap%
\pgfsetmiterjoin%
\definecolor{currentfill}{rgb}{0.000000,0.500000,0.000000}%
\pgfsetfillcolor{currentfill}%
\pgfsetlinewidth{0.000000pt}%
\definecolor{currentstroke}{rgb}{0.000000,0.000000,0.000000}%
\pgfsetstrokecolor{currentstroke}%
\pgfsetstrokeopacity{0.000000}%
\pgfsetdash{}{0pt}%
\pgfpathmoveto{\pgfqpoint{2.216335in}{0.500000in}}%
\pgfpathlineto{\pgfqpoint{2.249361in}{0.500000in}}%
\pgfpathlineto{\pgfqpoint{2.249361in}{0.594130in}}%
\pgfpathlineto{\pgfqpoint{2.216335in}{0.594130in}}%
\pgfpathlineto{\pgfqpoint{2.216335in}{0.500000in}}%
\pgfpathclose%
\pgfusepath{fill}%
\end{pgfscope}%
\begin{pgfscope}%
\pgfpathrectangle{\pgfqpoint{0.750000in}{0.500000in}}{\pgfqpoint{4.650000in}{3.020000in}}%
\pgfusepath{clip}%
\pgfsetbuttcap%
\pgfsetmiterjoin%
\definecolor{currentfill}{rgb}{0.000000,0.500000,0.000000}%
\pgfsetfillcolor{currentfill}%
\pgfsetlinewidth{0.000000pt}%
\definecolor{currentstroke}{rgb}{0.000000,0.000000,0.000000}%
\pgfsetstrokecolor{currentstroke}%
\pgfsetstrokeopacity{0.000000}%
\pgfsetdash{}{0pt}%
\pgfpathmoveto{\pgfqpoint{2.249361in}{0.500000in}}%
\pgfpathlineto{\pgfqpoint{2.282386in}{0.500000in}}%
\pgfpathlineto{\pgfqpoint{2.282386in}{0.635965in}}%
\pgfpathlineto{\pgfqpoint{2.249361in}{0.635965in}}%
\pgfpathlineto{\pgfqpoint{2.249361in}{0.500000in}}%
\pgfpathclose%
\pgfusepath{fill}%
\end{pgfscope}%
\begin{pgfscope}%
\pgfpathrectangle{\pgfqpoint{0.750000in}{0.500000in}}{\pgfqpoint{4.650000in}{3.020000in}}%
\pgfusepath{clip}%
\pgfsetbuttcap%
\pgfsetmiterjoin%
\definecolor{currentfill}{rgb}{0.000000,0.500000,0.000000}%
\pgfsetfillcolor{currentfill}%
\pgfsetlinewidth{0.000000pt}%
\definecolor{currentstroke}{rgb}{0.000000,0.000000,0.000000}%
\pgfsetstrokecolor{currentstroke}%
\pgfsetstrokeopacity{0.000000}%
\pgfsetdash{}{0pt}%
\pgfpathmoveto{\pgfqpoint{2.282386in}{0.500000in}}%
\pgfpathlineto{\pgfqpoint{2.315412in}{0.500000in}}%
\pgfpathlineto{\pgfqpoint{2.315412in}{0.541835in}}%
\pgfpathlineto{\pgfqpoint{2.282386in}{0.541835in}}%
\pgfpathlineto{\pgfqpoint{2.282386in}{0.500000in}}%
\pgfpathclose%
\pgfusepath{fill}%
\end{pgfscope}%
\begin{pgfscope}%
\pgfpathrectangle{\pgfqpoint{0.750000in}{0.500000in}}{\pgfqpoint{4.650000in}{3.020000in}}%
\pgfusepath{clip}%
\pgfsetbuttcap%
\pgfsetmiterjoin%
\definecolor{currentfill}{rgb}{0.000000,0.500000,0.000000}%
\pgfsetfillcolor{currentfill}%
\pgfsetlinewidth{0.000000pt}%
\definecolor{currentstroke}{rgb}{0.000000,0.000000,0.000000}%
\pgfsetstrokecolor{currentstroke}%
\pgfsetstrokeopacity{0.000000}%
\pgfsetdash{}{0pt}%
\pgfpathmoveto{\pgfqpoint{2.315412in}{0.500000in}}%
\pgfpathlineto{\pgfqpoint{2.348437in}{0.500000in}}%
\pgfpathlineto{\pgfqpoint{2.348437in}{0.625506in}}%
\pgfpathlineto{\pgfqpoint{2.315412in}{0.625506in}}%
\pgfpathlineto{\pgfqpoint{2.315412in}{0.500000in}}%
\pgfpathclose%
\pgfusepath{fill}%
\end{pgfscope}%
\begin{pgfscope}%
\pgfpathrectangle{\pgfqpoint{0.750000in}{0.500000in}}{\pgfqpoint{4.650000in}{3.020000in}}%
\pgfusepath{clip}%
\pgfsetbuttcap%
\pgfsetmiterjoin%
\definecolor{currentfill}{rgb}{0.000000,0.500000,0.000000}%
\pgfsetfillcolor{currentfill}%
\pgfsetlinewidth{0.000000pt}%
\definecolor{currentstroke}{rgb}{0.000000,0.000000,0.000000}%
\pgfsetstrokecolor{currentstroke}%
\pgfsetstrokeopacity{0.000000}%
\pgfsetdash{}{0pt}%
\pgfpathmoveto{\pgfqpoint{2.348438in}{0.500000in}}%
\pgfpathlineto{\pgfqpoint{2.381463in}{0.500000in}}%
\pgfpathlineto{\pgfqpoint{2.381463in}{0.688260in}}%
\pgfpathlineto{\pgfqpoint{2.348438in}{0.688260in}}%
\pgfpathlineto{\pgfqpoint{2.348438in}{0.500000in}}%
\pgfpathclose%
\pgfusepath{fill}%
\end{pgfscope}%
\begin{pgfscope}%
\pgfpathrectangle{\pgfqpoint{0.750000in}{0.500000in}}{\pgfqpoint{4.650000in}{3.020000in}}%
\pgfusepath{clip}%
\pgfsetbuttcap%
\pgfsetmiterjoin%
\definecolor{currentfill}{rgb}{0.000000,0.500000,0.000000}%
\pgfsetfillcolor{currentfill}%
\pgfsetlinewidth{0.000000pt}%
\definecolor{currentstroke}{rgb}{0.000000,0.000000,0.000000}%
\pgfsetstrokecolor{currentstroke}%
\pgfsetstrokeopacity{0.000000}%
\pgfsetdash{}{0pt}%
\pgfpathmoveto{\pgfqpoint{2.381463in}{0.500000in}}%
\pgfpathlineto{\pgfqpoint{2.414489in}{0.500000in}}%
\pgfpathlineto{\pgfqpoint{2.414489in}{0.625506in}}%
\pgfpathlineto{\pgfqpoint{2.381463in}{0.625506in}}%
\pgfpathlineto{\pgfqpoint{2.381463in}{0.500000in}}%
\pgfpathclose%
\pgfusepath{fill}%
\end{pgfscope}%
\begin{pgfscope}%
\pgfpathrectangle{\pgfqpoint{0.750000in}{0.500000in}}{\pgfqpoint{4.650000in}{3.020000in}}%
\pgfusepath{clip}%
\pgfsetbuttcap%
\pgfsetmiterjoin%
\definecolor{currentfill}{rgb}{0.000000,0.500000,0.000000}%
\pgfsetfillcolor{currentfill}%
\pgfsetlinewidth{0.000000pt}%
\definecolor{currentstroke}{rgb}{0.000000,0.000000,0.000000}%
\pgfsetstrokecolor{currentstroke}%
\pgfsetstrokeopacity{0.000000}%
\pgfsetdash{}{0pt}%
\pgfpathmoveto{\pgfqpoint{2.414489in}{0.500000in}}%
\pgfpathlineto{\pgfqpoint{2.447514in}{0.500000in}}%
\pgfpathlineto{\pgfqpoint{2.447514in}{0.688260in}}%
\pgfpathlineto{\pgfqpoint{2.414489in}{0.688260in}}%
\pgfpathlineto{\pgfqpoint{2.414489in}{0.500000in}}%
\pgfpathclose%
\pgfusepath{fill}%
\end{pgfscope}%
\begin{pgfscope}%
\pgfpathrectangle{\pgfqpoint{0.750000in}{0.500000in}}{\pgfqpoint{4.650000in}{3.020000in}}%
\pgfusepath{clip}%
\pgfsetbuttcap%
\pgfsetmiterjoin%
\definecolor{currentfill}{rgb}{0.000000,0.500000,0.000000}%
\pgfsetfillcolor{currentfill}%
\pgfsetlinewidth{0.000000pt}%
\definecolor{currentstroke}{rgb}{0.000000,0.000000,0.000000}%
\pgfsetstrokecolor{currentstroke}%
\pgfsetstrokeopacity{0.000000}%
\pgfsetdash{}{0pt}%
\pgfpathmoveto{\pgfqpoint{2.447514in}{0.500000in}}%
\pgfpathlineto{\pgfqpoint{2.480540in}{0.500000in}}%
\pgfpathlineto{\pgfqpoint{2.480540in}{0.677801in}}%
\pgfpathlineto{\pgfqpoint{2.447514in}{0.677801in}}%
\pgfpathlineto{\pgfqpoint{2.447514in}{0.500000in}}%
\pgfpathclose%
\pgfusepath{fill}%
\end{pgfscope}%
\begin{pgfscope}%
\pgfpathrectangle{\pgfqpoint{0.750000in}{0.500000in}}{\pgfqpoint{4.650000in}{3.020000in}}%
\pgfusepath{clip}%
\pgfsetbuttcap%
\pgfsetmiterjoin%
\definecolor{currentfill}{rgb}{0.000000,0.500000,0.000000}%
\pgfsetfillcolor{currentfill}%
\pgfsetlinewidth{0.000000pt}%
\definecolor{currentstroke}{rgb}{0.000000,0.000000,0.000000}%
\pgfsetstrokecolor{currentstroke}%
\pgfsetstrokeopacity{0.000000}%
\pgfsetdash{}{0pt}%
\pgfpathmoveto{\pgfqpoint{2.480540in}{0.500000in}}%
\pgfpathlineto{\pgfqpoint{2.513565in}{0.500000in}}%
\pgfpathlineto{\pgfqpoint{2.513565in}{0.761472in}}%
\pgfpathlineto{\pgfqpoint{2.480540in}{0.761472in}}%
\pgfpathlineto{\pgfqpoint{2.480540in}{0.500000in}}%
\pgfpathclose%
\pgfusepath{fill}%
\end{pgfscope}%
\begin{pgfscope}%
\pgfpathrectangle{\pgfqpoint{0.750000in}{0.500000in}}{\pgfqpoint{4.650000in}{3.020000in}}%
\pgfusepath{clip}%
\pgfsetbuttcap%
\pgfsetmiterjoin%
\definecolor{currentfill}{rgb}{0.000000,0.500000,0.000000}%
\pgfsetfillcolor{currentfill}%
\pgfsetlinewidth{0.000000pt}%
\definecolor{currentstroke}{rgb}{0.000000,0.000000,0.000000}%
\pgfsetstrokecolor{currentstroke}%
\pgfsetstrokeopacity{0.000000}%
\pgfsetdash{}{0pt}%
\pgfpathmoveto{\pgfqpoint{2.513565in}{0.500000in}}%
\pgfpathlineto{\pgfqpoint{2.546591in}{0.500000in}}%
\pgfpathlineto{\pgfqpoint{2.546591in}{0.771931in}}%
\pgfpathlineto{\pgfqpoint{2.513565in}{0.771931in}}%
\pgfpathlineto{\pgfqpoint{2.513565in}{0.500000in}}%
\pgfpathclose%
\pgfusepath{fill}%
\end{pgfscope}%
\begin{pgfscope}%
\pgfpathrectangle{\pgfqpoint{0.750000in}{0.500000in}}{\pgfqpoint{4.650000in}{3.020000in}}%
\pgfusepath{clip}%
\pgfsetbuttcap%
\pgfsetmiterjoin%
\definecolor{currentfill}{rgb}{0.000000,0.500000,0.000000}%
\pgfsetfillcolor{currentfill}%
\pgfsetlinewidth{0.000000pt}%
\definecolor{currentstroke}{rgb}{0.000000,0.000000,0.000000}%
\pgfsetstrokecolor{currentstroke}%
\pgfsetstrokeopacity{0.000000}%
\pgfsetdash{}{0pt}%
\pgfpathmoveto{\pgfqpoint{2.546591in}{0.500000in}}%
\pgfpathlineto{\pgfqpoint{2.579616in}{0.500000in}}%
\pgfpathlineto{\pgfqpoint{2.579616in}{0.761472in}}%
\pgfpathlineto{\pgfqpoint{2.546591in}{0.761472in}}%
\pgfpathlineto{\pgfqpoint{2.546591in}{0.500000in}}%
\pgfpathclose%
\pgfusepath{fill}%
\end{pgfscope}%
\begin{pgfscope}%
\pgfpathrectangle{\pgfqpoint{0.750000in}{0.500000in}}{\pgfqpoint{4.650000in}{3.020000in}}%
\pgfusepath{clip}%
\pgfsetbuttcap%
\pgfsetmiterjoin%
\definecolor{currentfill}{rgb}{0.000000,0.500000,0.000000}%
\pgfsetfillcolor{currentfill}%
\pgfsetlinewidth{0.000000pt}%
\definecolor{currentstroke}{rgb}{0.000000,0.000000,0.000000}%
\pgfsetstrokecolor{currentstroke}%
\pgfsetstrokeopacity{0.000000}%
\pgfsetdash{}{0pt}%
\pgfpathmoveto{\pgfqpoint{2.579616in}{0.500000in}}%
\pgfpathlineto{\pgfqpoint{2.612642in}{0.500000in}}%
\pgfpathlineto{\pgfqpoint{2.612642in}{0.907896in}}%
\pgfpathlineto{\pgfqpoint{2.579616in}{0.907896in}}%
\pgfpathlineto{\pgfqpoint{2.579616in}{0.500000in}}%
\pgfpathclose%
\pgfusepath{fill}%
\end{pgfscope}%
\begin{pgfscope}%
\pgfpathrectangle{\pgfqpoint{0.750000in}{0.500000in}}{\pgfqpoint{4.650000in}{3.020000in}}%
\pgfusepath{clip}%
\pgfsetbuttcap%
\pgfsetmiterjoin%
\definecolor{currentfill}{rgb}{0.000000,0.500000,0.000000}%
\pgfsetfillcolor{currentfill}%
\pgfsetlinewidth{0.000000pt}%
\definecolor{currentstroke}{rgb}{0.000000,0.000000,0.000000}%
\pgfsetstrokecolor{currentstroke}%
\pgfsetstrokeopacity{0.000000}%
\pgfsetdash{}{0pt}%
\pgfpathmoveto{\pgfqpoint{2.612642in}{0.500000in}}%
\pgfpathlineto{\pgfqpoint{2.645668in}{0.500000in}}%
\pgfpathlineto{\pgfqpoint{2.645668in}{0.928814in}}%
\pgfpathlineto{\pgfqpoint{2.612642in}{0.928814in}}%
\pgfpathlineto{\pgfqpoint{2.612642in}{0.500000in}}%
\pgfpathclose%
\pgfusepath{fill}%
\end{pgfscope}%
\begin{pgfscope}%
\pgfpathrectangle{\pgfqpoint{0.750000in}{0.500000in}}{\pgfqpoint{4.650000in}{3.020000in}}%
\pgfusepath{clip}%
\pgfsetbuttcap%
\pgfsetmiterjoin%
\definecolor{currentfill}{rgb}{0.000000,0.500000,0.000000}%
\pgfsetfillcolor{currentfill}%
\pgfsetlinewidth{0.000000pt}%
\definecolor{currentstroke}{rgb}{0.000000,0.000000,0.000000}%
\pgfsetstrokecolor{currentstroke}%
\pgfsetstrokeopacity{0.000000}%
\pgfsetdash{}{0pt}%
\pgfpathmoveto{\pgfqpoint{2.645668in}{0.500000in}}%
\pgfpathlineto{\pgfqpoint{2.678693in}{0.500000in}}%
\pgfpathlineto{\pgfqpoint{2.678693in}{1.127532in}}%
\pgfpathlineto{\pgfqpoint{2.645668in}{1.127532in}}%
\pgfpathlineto{\pgfqpoint{2.645668in}{0.500000in}}%
\pgfpathclose%
\pgfusepath{fill}%
\end{pgfscope}%
\begin{pgfscope}%
\pgfpathrectangle{\pgfqpoint{0.750000in}{0.500000in}}{\pgfqpoint{4.650000in}{3.020000in}}%
\pgfusepath{clip}%
\pgfsetbuttcap%
\pgfsetmiterjoin%
\definecolor{currentfill}{rgb}{0.000000,0.500000,0.000000}%
\pgfsetfillcolor{currentfill}%
\pgfsetlinewidth{0.000000pt}%
\definecolor{currentstroke}{rgb}{0.000000,0.000000,0.000000}%
\pgfsetstrokecolor{currentstroke}%
\pgfsetstrokeopacity{0.000000}%
\pgfsetdash{}{0pt}%
\pgfpathmoveto{\pgfqpoint{2.678693in}{0.500000in}}%
\pgfpathlineto{\pgfqpoint{2.711719in}{0.500000in}}%
\pgfpathlineto{\pgfqpoint{2.711719in}{1.043861in}}%
\pgfpathlineto{\pgfqpoint{2.678693in}{1.043861in}}%
\pgfpathlineto{\pgfqpoint{2.678693in}{0.500000in}}%
\pgfpathclose%
\pgfusepath{fill}%
\end{pgfscope}%
\begin{pgfscope}%
\pgfpathrectangle{\pgfqpoint{0.750000in}{0.500000in}}{\pgfqpoint{4.650000in}{3.020000in}}%
\pgfusepath{clip}%
\pgfsetbuttcap%
\pgfsetmiterjoin%
\definecolor{currentfill}{rgb}{0.000000,0.500000,0.000000}%
\pgfsetfillcolor{currentfill}%
\pgfsetlinewidth{0.000000pt}%
\definecolor{currentstroke}{rgb}{0.000000,0.000000,0.000000}%
\pgfsetstrokecolor{currentstroke}%
\pgfsetstrokeopacity{0.000000}%
\pgfsetdash{}{0pt}%
\pgfpathmoveto{\pgfqpoint{2.711719in}{0.500000in}}%
\pgfpathlineto{\pgfqpoint{2.744744in}{0.500000in}}%
\pgfpathlineto{\pgfqpoint{2.744744in}{1.169368in}}%
\pgfpathlineto{\pgfqpoint{2.711719in}{1.169368in}}%
\pgfpathlineto{\pgfqpoint{2.711719in}{0.500000in}}%
\pgfpathclose%
\pgfusepath{fill}%
\end{pgfscope}%
\begin{pgfscope}%
\pgfpathrectangle{\pgfqpoint{0.750000in}{0.500000in}}{\pgfqpoint{4.650000in}{3.020000in}}%
\pgfusepath{clip}%
\pgfsetbuttcap%
\pgfsetmiterjoin%
\definecolor{currentfill}{rgb}{0.000000,0.500000,0.000000}%
\pgfsetfillcolor{currentfill}%
\pgfsetlinewidth{0.000000pt}%
\definecolor{currentstroke}{rgb}{0.000000,0.000000,0.000000}%
\pgfsetstrokecolor{currentstroke}%
\pgfsetstrokeopacity{0.000000}%
\pgfsetdash{}{0pt}%
\pgfpathmoveto{\pgfqpoint{2.744744in}{0.500000in}}%
\pgfpathlineto{\pgfqpoint{2.777770in}{0.500000in}}%
\pgfpathlineto{\pgfqpoint{2.777770in}{1.148450in}}%
\pgfpathlineto{\pgfqpoint{2.744744in}{1.148450in}}%
\pgfpathlineto{\pgfqpoint{2.744744in}{0.500000in}}%
\pgfpathclose%
\pgfusepath{fill}%
\end{pgfscope}%
\begin{pgfscope}%
\pgfpathrectangle{\pgfqpoint{0.750000in}{0.500000in}}{\pgfqpoint{4.650000in}{3.020000in}}%
\pgfusepath{clip}%
\pgfsetbuttcap%
\pgfsetmiterjoin%
\definecolor{currentfill}{rgb}{0.000000,0.500000,0.000000}%
\pgfsetfillcolor{currentfill}%
\pgfsetlinewidth{0.000000pt}%
\definecolor{currentstroke}{rgb}{0.000000,0.000000,0.000000}%
\pgfsetstrokecolor{currentstroke}%
\pgfsetstrokeopacity{0.000000}%
\pgfsetdash{}{0pt}%
\pgfpathmoveto{\pgfqpoint{2.777770in}{0.500000in}}%
\pgfpathlineto{\pgfqpoint{2.810795in}{0.500000in}}%
\pgfpathlineto{\pgfqpoint{2.810795in}{1.169368in}}%
\pgfpathlineto{\pgfqpoint{2.777770in}{1.169368in}}%
\pgfpathlineto{\pgfqpoint{2.777770in}{0.500000in}}%
\pgfpathclose%
\pgfusepath{fill}%
\end{pgfscope}%
\begin{pgfscope}%
\pgfpathrectangle{\pgfqpoint{0.750000in}{0.500000in}}{\pgfqpoint{4.650000in}{3.020000in}}%
\pgfusepath{clip}%
\pgfsetbuttcap%
\pgfsetmiterjoin%
\definecolor{currentfill}{rgb}{0.000000,0.500000,0.000000}%
\pgfsetfillcolor{currentfill}%
\pgfsetlinewidth{0.000000pt}%
\definecolor{currentstroke}{rgb}{0.000000,0.000000,0.000000}%
\pgfsetstrokecolor{currentstroke}%
\pgfsetstrokeopacity{0.000000}%
\pgfsetdash{}{0pt}%
\pgfpathmoveto{\pgfqpoint{2.810795in}{0.500000in}}%
\pgfpathlineto{\pgfqpoint{2.843821in}{0.500000in}}%
\pgfpathlineto{\pgfqpoint{2.843821in}{1.200745in}}%
\pgfpathlineto{\pgfqpoint{2.810795in}{1.200745in}}%
\pgfpathlineto{\pgfqpoint{2.810795in}{0.500000in}}%
\pgfpathclose%
\pgfusepath{fill}%
\end{pgfscope}%
\begin{pgfscope}%
\pgfpathrectangle{\pgfqpoint{0.750000in}{0.500000in}}{\pgfqpoint{4.650000in}{3.020000in}}%
\pgfusepath{clip}%
\pgfsetbuttcap%
\pgfsetmiterjoin%
\definecolor{currentfill}{rgb}{0.000000,0.500000,0.000000}%
\pgfsetfillcolor{currentfill}%
\pgfsetlinewidth{0.000000pt}%
\definecolor{currentstroke}{rgb}{0.000000,0.000000,0.000000}%
\pgfsetstrokecolor{currentstroke}%
\pgfsetstrokeopacity{0.000000}%
\pgfsetdash{}{0pt}%
\pgfpathmoveto{\pgfqpoint{2.843821in}{0.500000in}}%
\pgfpathlineto{\pgfqpoint{2.876847in}{0.500000in}}%
\pgfpathlineto{\pgfqpoint{2.876847in}{1.357628in}}%
\pgfpathlineto{\pgfqpoint{2.843821in}{1.357628in}}%
\pgfpathlineto{\pgfqpoint{2.843821in}{0.500000in}}%
\pgfpathclose%
\pgfusepath{fill}%
\end{pgfscope}%
\begin{pgfscope}%
\pgfpathrectangle{\pgfqpoint{0.750000in}{0.500000in}}{\pgfqpoint{4.650000in}{3.020000in}}%
\pgfusepath{clip}%
\pgfsetbuttcap%
\pgfsetmiterjoin%
\definecolor{currentfill}{rgb}{0.000000,0.500000,0.000000}%
\pgfsetfillcolor{currentfill}%
\pgfsetlinewidth{0.000000pt}%
\definecolor{currentstroke}{rgb}{0.000000,0.000000,0.000000}%
\pgfsetstrokecolor{currentstroke}%
\pgfsetstrokeopacity{0.000000}%
\pgfsetdash{}{0pt}%
\pgfpathmoveto{\pgfqpoint{2.876847in}{0.500000in}}%
\pgfpathlineto{\pgfqpoint{2.909872in}{0.500000in}}%
\pgfpathlineto{\pgfqpoint{2.909872in}{1.347169in}}%
\pgfpathlineto{\pgfqpoint{2.876847in}{1.347169in}}%
\pgfpathlineto{\pgfqpoint{2.876847in}{0.500000in}}%
\pgfpathclose%
\pgfusepath{fill}%
\end{pgfscope}%
\begin{pgfscope}%
\pgfpathrectangle{\pgfqpoint{0.750000in}{0.500000in}}{\pgfqpoint{4.650000in}{3.020000in}}%
\pgfusepath{clip}%
\pgfsetbuttcap%
\pgfsetmiterjoin%
\definecolor{currentfill}{rgb}{0.000000,0.500000,0.000000}%
\pgfsetfillcolor{currentfill}%
\pgfsetlinewidth{0.000000pt}%
\definecolor{currentstroke}{rgb}{0.000000,0.000000,0.000000}%
\pgfsetstrokecolor{currentstroke}%
\pgfsetstrokeopacity{0.000000}%
\pgfsetdash{}{0pt}%
\pgfpathmoveto{\pgfqpoint{2.909872in}{0.500000in}}%
\pgfpathlineto{\pgfqpoint{2.942898in}{0.500000in}}%
\pgfpathlineto{\pgfqpoint{2.942898in}{1.629558in}}%
\pgfpathlineto{\pgfqpoint{2.909872in}{1.629558in}}%
\pgfpathlineto{\pgfqpoint{2.909872in}{0.500000in}}%
\pgfpathclose%
\pgfusepath{fill}%
\end{pgfscope}%
\begin{pgfscope}%
\pgfpathrectangle{\pgfqpoint{0.750000in}{0.500000in}}{\pgfqpoint{4.650000in}{3.020000in}}%
\pgfusepath{clip}%
\pgfsetbuttcap%
\pgfsetmiterjoin%
\definecolor{currentfill}{rgb}{0.000000,0.500000,0.000000}%
\pgfsetfillcolor{currentfill}%
\pgfsetlinewidth{0.000000pt}%
\definecolor{currentstroke}{rgb}{0.000000,0.000000,0.000000}%
\pgfsetstrokecolor{currentstroke}%
\pgfsetstrokeopacity{0.000000}%
\pgfsetdash{}{0pt}%
\pgfpathmoveto{\pgfqpoint{2.942898in}{0.500000in}}%
\pgfpathlineto{\pgfqpoint{2.975923in}{0.500000in}}%
\pgfpathlineto{\pgfqpoint{2.975923in}{1.545887in}}%
\pgfpathlineto{\pgfqpoint{2.942898in}{1.545887in}}%
\pgfpathlineto{\pgfqpoint{2.942898in}{0.500000in}}%
\pgfpathclose%
\pgfusepath{fill}%
\end{pgfscope}%
\begin{pgfscope}%
\pgfpathrectangle{\pgfqpoint{0.750000in}{0.500000in}}{\pgfqpoint{4.650000in}{3.020000in}}%
\pgfusepath{clip}%
\pgfsetbuttcap%
\pgfsetmiterjoin%
\definecolor{currentfill}{rgb}{0.000000,0.500000,0.000000}%
\pgfsetfillcolor{currentfill}%
\pgfsetlinewidth{0.000000pt}%
\definecolor{currentstroke}{rgb}{0.000000,0.000000,0.000000}%
\pgfsetstrokecolor{currentstroke}%
\pgfsetstrokeopacity{0.000000}%
\pgfsetdash{}{0pt}%
\pgfpathmoveto{\pgfqpoint{2.975923in}{0.500000in}}%
\pgfpathlineto{\pgfqpoint{3.008949in}{0.500000in}}%
\pgfpathlineto{\pgfqpoint{3.008949in}{1.284416in}}%
\pgfpathlineto{\pgfqpoint{2.975923in}{1.284416in}}%
\pgfpathlineto{\pgfqpoint{2.975923in}{0.500000in}}%
\pgfpathclose%
\pgfusepath{fill}%
\end{pgfscope}%
\begin{pgfscope}%
\pgfpathrectangle{\pgfqpoint{0.750000in}{0.500000in}}{\pgfqpoint{4.650000in}{3.020000in}}%
\pgfusepath{clip}%
\pgfsetbuttcap%
\pgfsetmiterjoin%
\definecolor{currentfill}{rgb}{0.000000,0.500000,0.000000}%
\pgfsetfillcolor{currentfill}%
\pgfsetlinewidth{0.000000pt}%
\definecolor{currentstroke}{rgb}{0.000000,0.000000,0.000000}%
\pgfsetstrokecolor{currentstroke}%
\pgfsetstrokeopacity{0.000000}%
\pgfsetdash{}{0pt}%
\pgfpathmoveto{\pgfqpoint{3.008949in}{0.500000in}}%
\pgfpathlineto{\pgfqpoint{3.041974in}{0.500000in}}%
\pgfpathlineto{\pgfqpoint{3.041974in}{1.305333in}}%
\pgfpathlineto{\pgfqpoint{3.008949in}{1.305333in}}%
\pgfpathlineto{\pgfqpoint{3.008949in}{0.500000in}}%
\pgfpathclose%
\pgfusepath{fill}%
\end{pgfscope}%
\begin{pgfscope}%
\pgfpathrectangle{\pgfqpoint{0.750000in}{0.500000in}}{\pgfqpoint{4.650000in}{3.020000in}}%
\pgfusepath{clip}%
\pgfsetbuttcap%
\pgfsetmiterjoin%
\definecolor{currentfill}{rgb}{0.000000,0.500000,0.000000}%
\pgfsetfillcolor{currentfill}%
\pgfsetlinewidth{0.000000pt}%
\definecolor{currentstroke}{rgb}{0.000000,0.000000,0.000000}%
\pgfsetstrokecolor{currentstroke}%
\pgfsetstrokeopacity{0.000000}%
\pgfsetdash{}{0pt}%
\pgfpathmoveto{\pgfqpoint{3.041974in}{0.500000in}}%
\pgfpathlineto{\pgfqpoint{3.075000in}{0.500000in}}%
\pgfpathlineto{\pgfqpoint{3.075000in}{1.828277in}}%
\pgfpathlineto{\pgfqpoint{3.041974in}{1.828277in}}%
\pgfpathlineto{\pgfqpoint{3.041974in}{0.500000in}}%
\pgfpathclose%
\pgfusepath{fill}%
\end{pgfscope}%
\begin{pgfscope}%
\pgfpathrectangle{\pgfqpoint{0.750000in}{0.500000in}}{\pgfqpoint{4.650000in}{3.020000in}}%
\pgfusepath{clip}%
\pgfsetbuttcap%
\pgfsetmiterjoin%
\definecolor{currentfill}{rgb}{0.000000,0.500000,0.000000}%
\pgfsetfillcolor{currentfill}%
\pgfsetlinewidth{0.000000pt}%
\definecolor{currentstroke}{rgb}{0.000000,0.000000,0.000000}%
\pgfsetstrokecolor{currentstroke}%
\pgfsetstrokeopacity{0.000000}%
\pgfsetdash{}{0pt}%
\pgfpathmoveto{\pgfqpoint{3.075000in}{0.500000in}}%
\pgfpathlineto{\pgfqpoint{3.108026in}{0.500000in}}%
\pgfpathlineto{\pgfqpoint{3.108026in}{1.692312in}}%
\pgfpathlineto{\pgfqpoint{3.075000in}{1.692312in}}%
\pgfpathlineto{\pgfqpoint{3.075000in}{0.500000in}}%
\pgfpathclose%
\pgfusepath{fill}%
\end{pgfscope}%
\begin{pgfscope}%
\pgfpathrectangle{\pgfqpoint{0.750000in}{0.500000in}}{\pgfqpoint{4.650000in}{3.020000in}}%
\pgfusepath{clip}%
\pgfsetbuttcap%
\pgfsetmiterjoin%
\definecolor{currentfill}{rgb}{0.000000,0.500000,0.000000}%
\pgfsetfillcolor{currentfill}%
\pgfsetlinewidth{0.000000pt}%
\definecolor{currentstroke}{rgb}{0.000000,0.000000,0.000000}%
\pgfsetstrokecolor{currentstroke}%
\pgfsetstrokeopacity{0.000000}%
\pgfsetdash{}{0pt}%
\pgfpathmoveto{\pgfqpoint{3.108026in}{0.500000in}}%
\pgfpathlineto{\pgfqpoint{3.141051in}{0.500000in}}%
\pgfpathlineto{\pgfqpoint{3.141051in}{1.545887in}}%
\pgfpathlineto{\pgfqpoint{3.108026in}{1.545887in}}%
\pgfpathlineto{\pgfqpoint{3.108026in}{0.500000in}}%
\pgfpathclose%
\pgfusepath{fill}%
\end{pgfscope}%
\begin{pgfscope}%
\pgfpathrectangle{\pgfqpoint{0.750000in}{0.500000in}}{\pgfqpoint{4.650000in}{3.020000in}}%
\pgfusepath{clip}%
\pgfsetbuttcap%
\pgfsetmiterjoin%
\definecolor{currentfill}{rgb}{0.000000,0.500000,0.000000}%
\pgfsetfillcolor{currentfill}%
\pgfsetlinewidth{0.000000pt}%
\definecolor{currentstroke}{rgb}{0.000000,0.000000,0.000000}%
\pgfsetstrokecolor{currentstroke}%
\pgfsetstrokeopacity{0.000000}%
\pgfsetdash{}{0pt}%
\pgfpathmoveto{\pgfqpoint{3.141051in}{0.500000in}}%
\pgfpathlineto{\pgfqpoint{3.174077in}{0.500000in}}%
\pgfpathlineto{\pgfqpoint{3.174077in}{1.807359in}}%
\pgfpathlineto{\pgfqpoint{3.141051in}{1.807359in}}%
\pgfpathlineto{\pgfqpoint{3.141051in}{0.500000in}}%
\pgfpathclose%
\pgfusepath{fill}%
\end{pgfscope}%
\begin{pgfscope}%
\pgfpathrectangle{\pgfqpoint{0.750000in}{0.500000in}}{\pgfqpoint{4.650000in}{3.020000in}}%
\pgfusepath{clip}%
\pgfsetbuttcap%
\pgfsetmiterjoin%
\definecolor{currentfill}{rgb}{0.000000,0.500000,0.000000}%
\pgfsetfillcolor{currentfill}%
\pgfsetlinewidth{0.000000pt}%
\definecolor{currentstroke}{rgb}{0.000000,0.000000,0.000000}%
\pgfsetstrokecolor{currentstroke}%
\pgfsetstrokeopacity{0.000000}%
\pgfsetdash{}{0pt}%
\pgfpathmoveto{\pgfqpoint{3.174077in}{0.500000in}}%
\pgfpathlineto{\pgfqpoint{3.207102in}{0.500000in}}%
\pgfpathlineto{\pgfqpoint{3.207102in}{1.880571in}}%
\pgfpathlineto{\pgfqpoint{3.174077in}{1.880571in}}%
\pgfpathlineto{\pgfqpoint{3.174077in}{0.500000in}}%
\pgfpathclose%
\pgfusepath{fill}%
\end{pgfscope}%
\begin{pgfscope}%
\pgfpathrectangle{\pgfqpoint{0.750000in}{0.500000in}}{\pgfqpoint{4.650000in}{3.020000in}}%
\pgfusepath{clip}%
\pgfsetbuttcap%
\pgfsetmiterjoin%
\definecolor{currentfill}{rgb}{0.000000,0.500000,0.000000}%
\pgfsetfillcolor{currentfill}%
\pgfsetlinewidth{0.000000pt}%
\definecolor{currentstroke}{rgb}{0.000000,0.000000,0.000000}%
\pgfsetstrokecolor{currentstroke}%
\pgfsetstrokeopacity{0.000000}%
\pgfsetdash{}{0pt}%
\pgfpathmoveto{\pgfqpoint{3.207102in}{0.500000in}}%
\pgfpathlineto{\pgfqpoint{3.240128in}{0.500000in}}%
\pgfpathlineto{\pgfqpoint{3.240128in}{1.796900in}}%
\pgfpathlineto{\pgfqpoint{3.207102in}{1.796900in}}%
\pgfpathlineto{\pgfqpoint{3.207102in}{0.500000in}}%
\pgfpathclose%
\pgfusepath{fill}%
\end{pgfscope}%
\begin{pgfscope}%
\pgfpathrectangle{\pgfqpoint{0.750000in}{0.500000in}}{\pgfqpoint{4.650000in}{3.020000in}}%
\pgfusepath{clip}%
\pgfsetbuttcap%
\pgfsetmiterjoin%
\definecolor{currentfill}{rgb}{0.000000,0.500000,0.000000}%
\pgfsetfillcolor{currentfill}%
\pgfsetlinewidth{0.000000pt}%
\definecolor{currentstroke}{rgb}{0.000000,0.000000,0.000000}%
\pgfsetstrokecolor{currentstroke}%
\pgfsetstrokeopacity{0.000000}%
\pgfsetdash{}{0pt}%
\pgfpathmoveto{\pgfqpoint{3.240128in}{0.500000in}}%
\pgfpathlineto{\pgfqpoint{3.273153in}{0.500000in}}%
\pgfpathlineto{\pgfqpoint{3.273153in}{1.859654in}}%
\pgfpathlineto{\pgfqpoint{3.240128in}{1.859654in}}%
\pgfpathlineto{\pgfqpoint{3.240128in}{0.500000in}}%
\pgfpathclose%
\pgfusepath{fill}%
\end{pgfscope}%
\begin{pgfscope}%
\pgfpathrectangle{\pgfqpoint{0.750000in}{0.500000in}}{\pgfqpoint{4.650000in}{3.020000in}}%
\pgfusepath{clip}%
\pgfsetbuttcap%
\pgfsetmiterjoin%
\definecolor{currentfill}{rgb}{0.000000,0.500000,0.000000}%
\pgfsetfillcolor{currentfill}%
\pgfsetlinewidth{0.000000pt}%
\definecolor{currentstroke}{rgb}{0.000000,0.000000,0.000000}%
\pgfsetstrokecolor{currentstroke}%
\pgfsetstrokeopacity{0.000000}%
\pgfsetdash{}{0pt}%
\pgfpathmoveto{\pgfqpoint{3.273153in}{0.500000in}}%
\pgfpathlineto{\pgfqpoint{3.306179in}{0.500000in}}%
\pgfpathlineto{\pgfqpoint{3.306179in}{2.016537in}}%
\pgfpathlineto{\pgfqpoint{3.273153in}{2.016537in}}%
\pgfpathlineto{\pgfqpoint{3.273153in}{0.500000in}}%
\pgfpathclose%
\pgfusepath{fill}%
\end{pgfscope}%
\begin{pgfscope}%
\pgfpathrectangle{\pgfqpoint{0.750000in}{0.500000in}}{\pgfqpoint{4.650000in}{3.020000in}}%
\pgfusepath{clip}%
\pgfsetbuttcap%
\pgfsetmiterjoin%
\definecolor{currentfill}{rgb}{0.000000,0.500000,0.000000}%
\pgfsetfillcolor{currentfill}%
\pgfsetlinewidth{0.000000pt}%
\definecolor{currentstroke}{rgb}{0.000000,0.000000,0.000000}%
\pgfsetstrokecolor{currentstroke}%
\pgfsetstrokeopacity{0.000000}%
\pgfsetdash{}{0pt}%
\pgfpathmoveto{\pgfqpoint{3.306179in}{0.500000in}}%
\pgfpathlineto{\pgfqpoint{3.339205in}{0.500000in}}%
\pgfpathlineto{\pgfqpoint{3.339205in}{2.142043in}}%
\pgfpathlineto{\pgfqpoint{3.306179in}{2.142043in}}%
\pgfpathlineto{\pgfqpoint{3.306179in}{0.500000in}}%
\pgfpathclose%
\pgfusepath{fill}%
\end{pgfscope}%
\begin{pgfscope}%
\pgfpathrectangle{\pgfqpoint{0.750000in}{0.500000in}}{\pgfqpoint{4.650000in}{3.020000in}}%
\pgfusepath{clip}%
\pgfsetbuttcap%
\pgfsetmiterjoin%
\definecolor{currentfill}{rgb}{0.000000,0.500000,0.000000}%
\pgfsetfillcolor{currentfill}%
\pgfsetlinewidth{0.000000pt}%
\definecolor{currentstroke}{rgb}{0.000000,0.000000,0.000000}%
\pgfsetstrokecolor{currentstroke}%
\pgfsetstrokeopacity{0.000000}%
\pgfsetdash{}{0pt}%
\pgfpathmoveto{\pgfqpoint{3.339205in}{0.500000in}}%
\pgfpathlineto{\pgfqpoint{3.372230in}{0.500000in}}%
\pgfpathlineto{\pgfqpoint{3.372230in}{2.330303in}}%
\pgfpathlineto{\pgfqpoint{3.339205in}{2.330303in}}%
\pgfpathlineto{\pgfqpoint{3.339205in}{0.500000in}}%
\pgfpathclose%
\pgfusepath{fill}%
\end{pgfscope}%
\begin{pgfscope}%
\pgfpathrectangle{\pgfqpoint{0.750000in}{0.500000in}}{\pgfqpoint{4.650000in}{3.020000in}}%
\pgfusepath{clip}%
\pgfsetbuttcap%
\pgfsetmiterjoin%
\definecolor{currentfill}{rgb}{0.000000,0.500000,0.000000}%
\pgfsetfillcolor{currentfill}%
\pgfsetlinewidth{0.000000pt}%
\definecolor{currentstroke}{rgb}{0.000000,0.000000,0.000000}%
\pgfsetstrokecolor{currentstroke}%
\pgfsetstrokeopacity{0.000000}%
\pgfsetdash{}{0pt}%
\pgfpathmoveto{\pgfqpoint{3.372230in}{0.500000in}}%
\pgfpathlineto{\pgfqpoint{3.405256in}{0.500000in}}%
\pgfpathlineto{\pgfqpoint{3.405256in}{2.382597in}}%
\pgfpathlineto{\pgfqpoint{3.372230in}{2.382597in}}%
\pgfpathlineto{\pgfqpoint{3.372230in}{0.500000in}}%
\pgfpathclose%
\pgfusepath{fill}%
\end{pgfscope}%
\begin{pgfscope}%
\pgfpathrectangle{\pgfqpoint{0.750000in}{0.500000in}}{\pgfqpoint{4.650000in}{3.020000in}}%
\pgfusepath{clip}%
\pgfsetbuttcap%
\pgfsetmiterjoin%
\definecolor{currentfill}{rgb}{0.000000,0.500000,0.000000}%
\pgfsetfillcolor{currentfill}%
\pgfsetlinewidth{0.000000pt}%
\definecolor{currentstroke}{rgb}{0.000000,0.000000,0.000000}%
\pgfsetstrokecolor{currentstroke}%
\pgfsetstrokeopacity{0.000000}%
\pgfsetdash{}{0pt}%
\pgfpathmoveto{\pgfqpoint{3.405256in}{0.500000in}}%
\pgfpathlineto{\pgfqpoint{3.438281in}{0.500000in}}%
\pgfpathlineto{\pgfqpoint{3.438281in}{2.131584in}}%
\pgfpathlineto{\pgfqpoint{3.405256in}{2.131584in}}%
\pgfpathlineto{\pgfqpoint{3.405256in}{0.500000in}}%
\pgfpathclose%
\pgfusepath{fill}%
\end{pgfscope}%
\begin{pgfscope}%
\pgfpathrectangle{\pgfqpoint{0.750000in}{0.500000in}}{\pgfqpoint{4.650000in}{3.020000in}}%
\pgfusepath{clip}%
\pgfsetbuttcap%
\pgfsetmiterjoin%
\definecolor{currentfill}{rgb}{0.000000,0.500000,0.000000}%
\pgfsetfillcolor{currentfill}%
\pgfsetlinewidth{0.000000pt}%
\definecolor{currentstroke}{rgb}{0.000000,0.000000,0.000000}%
\pgfsetstrokecolor{currentstroke}%
\pgfsetstrokeopacity{0.000000}%
\pgfsetdash{}{0pt}%
\pgfpathmoveto{\pgfqpoint{3.438281in}{0.500000in}}%
\pgfpathlineto{\pgfqpoint{3.471307in}{0.500000in}}%
\pgfpathlineto{\pgfqpoint{3.471307in}{2.079290in}}%
\pgfpathlineto{\pgfqpoint{3.438281in}{2.079290in}}%
\pgfpathlineto{\pgfqpoint{3.438281in}{0.500000in}}%
\pgfpathclose%
\pgfusepath{fill}%
\end{pgfscope}%
\begin{pgfscope}%
\pgfpathrectangle{\pgfqpoint{0.750000in}{0.500000in}}{\pgfqpoint{4.650000in}{3.020000in}}%
\pgfusepath{clip}%
\pgfsetbuttcap%
\pgfsetmiterjoin%
\definecolor{currentfill}{rgb}{0.000000,0.500000,0.000000}%
\pgfsetfillcolor{currentfill}%
\pgfsetlinewidth{0.000000pt}%
\definecolor{currentstroke}{rgb}{0.000000,0.000000,0.000000}%
\pgfsetstrokecolor{currentstroke}%
\pgfsetstrokeopacity{0.000000}%
\pgfsetdash{}{0pt}%
\pgfpathmoveto{\pgfqpoint{3.471307in}{0.500000in}}%
\pgfpathlineto{\pgfqpoint{3.504332in}{0.500000in}}%
\pgfpathlineto{\pgfqpoint{3.504332in}{1.786442in}}%
\pgfpathlineto{\pgfqpoint{3.471307in}{1.786442in}}%
\pgfpathlineto{\pgfqpoint{3.471307in}{0.500000in}}%
\pgfpathclose%
\pgfusepath{fill}%
\end{pgfscope}%
\begin{pgfscope}%
\pgfpathrectangle{\pgfqpoint{0.750000in}{0.500000in}}{\pgfqpoint{4.650000in}{3.020000in}}%
\pgfusepath{clip}%
\pgfsetbuttcap%
\pgfsetmiterjoin%
\definecolor{currentfill}{rgb}{0.000000,0.500000,0.000000}%
\pgfsetfillcolor{currentfill}%
\pgfsetlinewidth{0.000000pt}%
\definecolor{currentstroke}{rgb}{0.000000,0.000000,0.000000}%
\pgfsetstrokecolor{currentstroke}%
\pgfsetstrokeopacity{0.000000}%
\pgfsetdash{}{0pt}%
\pgfpathmoveto{\pgfqpoint{3.504332in}{0.500000in}}%
\pgfpathlineto{\pgfqpoint{3.537358in}{0.500000in}}%
\pgfpathlineto{\pgfqpoint{3.537358in}{1.786442in}}%
\pgfpathlineto{\pgfqpoint{3.504332in}{1.786442in}}%
\pgfpathlineto{\pgfqpoint{3.504332in}{0.500000in}}%
\pgfpathclose%
\pgfusepath{fill}%
\end{pgfscope}%
\begin{pgfscope}%
\pgfpathrectangle{\pgfqpoint{0.750000in}{0.500000in}}{\pgfqpoint{4.650000in}{3.020000in}}%
\pgfusepath{clip}%
\pgfsetbuttcap%
\pgfsetmiterjoin%
\definecolor{currentfill}{rgb}{0.000000,0.500000,0.000000}%
\pgfsetfillcolor{currentfill}%
\pgfsetlinewidth{0.000000pt}%
\definecolor{currentstroke}{rgb}{0.000000,0.000000,0.000000}%
\pgfsetstrokecolor{currentstroke}%
\pgfsetstrokeopacity{0.000000}%
\pgfsetdash{}{0pt}%
\pgfpathmoveto{\pgfqpoint{3.537358in}{0.500000in}}%
\pgfpathlineto{\pgfqpoint{3.570384in}{0.500000in}}%
\pgfpathlineto{\pgfqpoint{3.570384in}{1.807359in}}%
\pgfpathlineto{\pgfqpoint{3.537358in}{1.807359in}}%
\pgfpathlineto{\pgfqpoint{3.537358in}{0.500000in}}%
\pgfpathclose%
\pgfusepath{fill}%
\end{pgfscope}%
\begin{pgfscope}%
\pgfpathrectangle{\pgfqpoint{0.750000in}{0.500000in}}{\pgfqpoint{4.650000in}{3.020000in}}%
\pgfusepath{clip}%
\pgfsetbuttcap%
\pgfsetmiterjoin%
\definecolor{currentfill}{rgb}{0.000000,0.500000,0.000000}%
\pgfsetfillcolor{currentfill}%
\pgfsetlinewidth{0.000000pt}%
\definecolor{currentstroke}{rgb}{0.000000,0.000000,0.000000}%
\pgfsetstrokecolor{currentstroke}%
\pgfsetstrokeopacity{0.000000}%
\pgfsetdash{}{0pt}%
\pgfpathmoveto{\pgfqpoint{3.570384in}{0.500000in}}%
\pgfpathlineto{\pgfqpoint{3.603409in}{0.500000in}}%
\pgfpathlineto{\pgfqpoint{3.603409in}{1.838736in}}%
\pgfpathlineto{\pgfqpoint{3.570384in}{1.838736in}}%
\pgfpathlineto{\pgfqpoint{3.570384in}{0.500000in}}%
\pgfpathclose%
\pgfusepath{fill}%
\end{pgfscope}%
\begin{pgfscope}%
\pgfpathrectangle{\pgfqpoint{0.750000in}{0.500000in}}{\pgfqpoint{4.650000in}{3.020000in}}%
\pgfusepath{clip}%
\pgfsetbuttcap%
\pgfsetmiterjoin%
\definecolor{currentfill}{rgb}{0.000000,0.500000,0.000000}%
\pgfsetfillcolor{currentfill}%
\pgfsetlinewidth{0.000000pt}%
\definecolor{currentstroke}{rgb}{0.000000,0.000000,0.000000}%
\pgfsetstrokecolor{currentstroke}%
\pgfsetstrokeopacity{0.000000}%
\pgfsetdash{}{0pt}%
\pgfpathmoveto{\pgfqpoint{3.603409in}{0.500000in}}%
\pgfpathlineto{\pgfqpoint{3.636435in}{0.500000in}}%
\pgfpathlineto{\pgfqpoint{3.636435in}{1.671394in}}%
\pgfpathlineto{\pgfqpoint{3.603409in}{1.671394in}}%
\pgfpathlineto{\pgfqpoint{3.603409in}{0.500000in}}%
\pgfpathclose%
\pgfusepath{fill}%
\end{pgfscope}%
\begin{pgfscope}%
\pgfpathrectangle{\pgfqpoint{0.750000in}{0.500000in}}{\pgfqpoint{4.650000in}{3.020000in}}%
\pgfusepath{clip}%
\pgfsetbuttcap%
\pgfsetmiterjoin%
\definecolor{currentfill}{rgb}{0.000000,0.500000,0.000000}%
\pgfsetfillcolor{currentfill}%
\pgfsetlinewidth{0.000000pt}%
\definecolor{currentstroke}{rgb}{0.000000,0.000000,0.000000}%
\pgfsetstrokecolor{currentstroke}%
\pgfsetstrokeopacity{0.000000}%
\pgfsetdash{}{0pt}%
\pgfpathmoveto{\pgfqpoint{3.636435in}{0.500000in}}%
\pgfpathlineto{\pgfqpoint{3.669460in}{0.500000in}}%
\pgfpathlineto{\pgfqpoint{3.669460in}{1.671394in}}%
\pgfpathlineto{\pgfqpoint{3.636435in}{1.671394in}}%
\pgfpathlineto{\pgfqpoint{3.636435in}{0.500000in}}%
\pgfpathclose%
\pgfusepath{fill}%
\end{pgfscope}%
\begin{pgfscope}%
\pgfpathrectangle{\pgfqpoint{0.750000in}{0.500000in}}{\pgfqpoint{4.650000in}{3.020000in}}%
\pgfusepath{clip}%
\pgfsetbuttcap%
\pgfsetmiterjoin%
\definecolor{currentfill}{rgb}{0.000000,0.500000,0.000000}%
\pgfsetfillcolor{currentfill}%
\pgfsetlinewidth{0.000000pt}%
\definecolor{currentstroke}{rgb}{0.000000,0.000000,0.000000}%
\pgfsetstrokecolor{currentstroke}%
\pgfsetstrokeopacity{0.000000}%
\pgfsetdash{}{0pt}%
\pgfpathmoveto{\pgfqpoint{3.669460in}{0.500000in}}%
\pgfpathlineto{\pgfqpoint{3.702486in}{0.500000in}}%
\pgfpathlineto{\pgfqpoint{3.702486in}{1.357628in}}%
\pgfpathlineto{\pgfqpoint{3.669460in}{1.357628in}}%
\pgfpathlineto{\pgfqpoint{3.669460in}{0.500000in}}%
\pgfpathclose%
\pgfusepath{fill}%
\end{pgfscope}%
\begin{pgfscope}%
\pgfpathrectangle{\pgfqpoint{0.750000in}{0.500000in}}{\pgfqpoint{4.650000in}{3.020000in}}%
\pgfusepath{clip}%
\pgfsetbuttcap%
\pgfsetmiterjoin%
\definecolor{currentfill}{rgb}{0.000000,0.500000,0.000000}%
\pgfsetfillcolor{currentfill}%
\pgfsetlinewidth{0.000000pt}%
\definecolor{currentstroke}{rgb}{0.000000,0.000000,0.000000}%
\pgfsetstrokecolor{currentstroke}%
\pgfsetstrokeopacity{0.000000}%
\pgfsetdash{}{0pt}%
\pgfpathmoveto{\pgfqpoint{3.702486in}{0.500000in}}%
\pgfpathlineto{\pgfqpoint{3.735511in}{0.500000in}}%
\pgfpathlineto{\pgfqpoint{3.735511in}{1.504052in}}%
\pgfpathlineto{\pgfqpoint{3.702486in}{1.504052in}}%
\pgfpathlineto{\pgfqpoint{3.702486in}{0.500000in}}%
\pgfpathclose%
\pgfusepath{fill}%
\end{pgfscope}%
\begin{pgfscope}%
\pgfpathrectangle{\pgfqpoint{0.750000in}{0.500000in}}{\pgfqpoint{4.650000in}{3.020000in}}%
\pgfusepath{clip}%
\pgfsetbuttcap%
\pgfsetmiterjoin%
\definecolor{currentfill}{rgb}{0.000000,0.500000,0.000000}%
\pgfsetfillcolor{currentfill}%
\pgfsetlinewidth{0.000000pt}%
\definecolor{currentstroke}{rgb}{0.000000,0.000000,0.000000}%
\pgfsetstrokecolor{currentstroke}%
\pgfsetstrokeopacity{0.000000}%
\pgfsetdash{}{0pt}%
\pgfpathmoveto{\pgfqpoint{3.735511in}{0.500000in}}%
\pgfpathlineto{\pgfqpoint{3.768537in}{0.500000in}}%
\pgfpathlineto{\pgfqpoint{3.768537in}{1.524970in}}%
\pgfpathlineto{\pgfqpoint{3.735511in}{1.524970in}}%
\pgfpathlineto{\pgfqpoint{3.735511in}{0.500000in}}%
\pgfpathclose%
\pgfusepath{fill}%
\end{pgfscope}%
\begin{pgfscope}%
\pgfpathrectangle{\pgfqpoint{0.750000in}{0.500000in}}{\pgfqpoint{4.650000in}{3.020000in}}%
\pgfusepath{clip}%
\pgfsetbuttcap%
\pgfsetmiterjoin%
\definecolor{currentfill}{rgb}{0.000000,0.500000,0.000000}%
\pgfsetfillcolor{currentfill}%
\pgfsetlinewidth{0.000000pt}%
\definecolor{currentstroke}{rgb}{0.000000,0.000000,0.000000}%
\pgfsetstrokecolor{currentstroke}%
\pgfsetstrokeopacity{0.000000}%
\pgfsetdash{}{0pt}%
\pgfpathmoveto{\pgfqpoint{3.768537in}{0.500000in}}%
\pgfpathlineto{\pgfqpoint{3.801563in}{0.500000in}}%
\pgfpathlineto{\pgfqpoint{3.801563in}{1.368087in}}%
\pgfpathlineto{\pgfqpoint{3.768537in}{1.368087in}}%
\pgfpathlineto{\pgfqpoint{3.768537in}{0.500000in}}%
\pgfpathclose%
\pgfusepath{fill}%
\end{pgfscope}%
\begin{pgfscope}%
\pgfpathrectangle{\pgfqpoint{0.750000in}{0.500000in}}{\pgfqpoint{4.650000in}{3.020000in}}%
\pgfusepath{clip}%
\pgfsetbuttcap%
\pgfsetmiterjoin%
\definecolor{currentfill}{rgb}{0.000000,0.500000,0.000000}%
\pgfsetfillcolor{currentfill}%
\pgfsetlinewidth{0.000000pt}%
\definecolor{currentstroke}{rgb}{0.000000,0.000000,0.000000}%
\pgfsetstrokecolor{currentstroke}%
\pgfsetstrokeopacity{0.000000}%
\pgfsetdash{}{0pt}%
\pgfpathmoveto{\pgfqpoint{3.801563in}{0.500000in}}%
\pgfpathlineto{\pgfqpoint{3.834588in}{0.500000in}}%
\pgfpathlineto{\pgfqpoint{3.834588in}{1.232121in}}%
\pgfpathlineto{\pgfqpoint{3.801563in}{1.232121in}}%
\pgfpathlineto{\pgfqpoint{3.801563in}{0.500000in}}%
\pgfpathclose%
\pgfusepath{fill}%
\end{pgfscope}%
\begin{pgfscope}%
\pgfpathrectangle{\pgfqpoint{0.750000in}{0.500000in}}{\pgfqpoint{4.650000in}{3.020000in}}%
\pgfusepath{clip}%
\pgfsetbuttcap%
\pgfsetmiterjoin%
\definecolor{currentfill}{rgb}{0.000000,0.500000,0.000000}%
\pgfsetfillcolor{currentfill}%
\pgfsetlinewidth{0.000000pt}%
\definecolor{currentstroke}{rgb}{0.000000,0.000000,0.000000}%
\pgfsetstrokecolor{currentstroke}%
\pgfsetstrokeopacity{0.000000}%
\pgfsetdash{}{0pt}%
\pgfpathmoveto{\pgfqpoint{3.834588in}{0.500000in}}%
\pgfpathlineto{\pgfqpoint{3.867614in}{0.500000in}}%
\pgfpathlineto{\pgfqpoint{3.867614in}{1.399463in}}%
\pgfpathlineto{\pgfqpoint{3.834588in}{1.399463in}}%
\pgfpathlineto{\pgfqpoint{3.834588in}{0.500000in}}%
\pgfpathclose%
\pgfusepath{fill}%
\end{pgfscope}%
\begin{pgfscope}%
\pgfpathrectangle{\pgfqpoint{0.750000in}{0.500000in}}{\pgfqpoint{4.650000in}{3.020000in}}%
\pgfusepath{clip}%
\pgfsetbuttcap%
\pgfsetmiterjoin%
\definecolor{currentfill}{rgb}{0.000000,0.500000,0.000000}%
\pgfsetfillcolor{currentfill}%
\pgfsetlinewidth{0.000000pt}%
\definecolor{currentstroke}{rgb}{0.000000,0.000000,0.000000}%
\pgfsetstrokecolor{currentstroke}%
\pgfsetstrokeopacity{0.000000}%
\pgfsetdash{}{0pt}%
\pgfpathmoveto{\pgfqpoint{3.867614in}{0.500000in}}%
\pgfpathlineto{\pgfqpoint{3.900639in}{0.500000in}}%
\pgfpathlineto{\pgfqpoint{3.900639in}{1.190286in}}%
\pgfpathlineto{\pgfqpoint{3.867614in}{1.190286in}}%
\pgfpathlineto{\pgfqpoint{3.867614in}{0.500000in}}%
\pgfpathclose%
\pgfusepath{fill}%
\end{pgfscope}%
\begin{pgfscope}%
\pgfpathrectangle{\pgfqpoint{0.750000in}{0.500000in}}{\pgfqpoint{4.650000in}{3.020000in}}%
\pgfusepath{clip}%
\pgfsetbuttcap%
\pgfsetmiterjoin%
\definecolor{currentfill}{rgb}{0.000000,0.500000,0.000000}%
\pgfsetfillcolor{currentfill}%
\pgfsetlinewidth{0.000000pt}%
\definecolor{currentstroke}{rgb}{0.000000,0.000000,0.000000}%
\pgfsetstrokecolor{currentstroke}%
\pgfsetstrokeopacity{0.000000}%
\pgfsetdash{}{0pt}%
\pgfpathmoveto{\pgfqpoint{3.900639in}{0.500000in}}%
\pgfpathlineto{\pgfqpoint{3.933665in}{0.500000in}}%
\pgfpathlineto{\pgfqpoint{3.933665in}{1.190286in}}%
\pgfpathlineto{\pgfqpoint{3.900639in}{1.190286in}}%
\pgfpathlineto{\pgfqpoint{3.900639in}{0.500000in}}%
\pgfpathclose%
\pgfusepath{fill}%
\end{pgfscope}%
\begin{pgfscope}%
\pgfpathrectangle{\pgfqpoint{0.750000in}{0.500000in}}{\pgfqpoint{4.650000in}{3.020000in}}%
\pgfusepath{clip}%
\pgfsetbuttcap%
\pgfsetmiterjoin%
\definecolor{currentfill}{rgb}{0.000000,0.500000,0.000000}%
\pgfsetfillcolor{currentfill}%
\pgfsetlinewidth{0.000000pt}%
\definecolor{currentstroke}{rgb}{0.000000,0.000000,0.000000}%
\pgfsetstrokecolor{currentstroke}%
\pgfsetstrokeopacity{0.000000}%
\pgfsetdash{}{0pt}%
\pgfpathmoveto{\pgfqpoint{3.933665in}{0.500000in}}%
\pgfpathlineto{\pgfqpoint{3.966690in}{0.500000in}}%
\pgfpathlineto{\pgfqpoint{3.966690in}{1.096156in}}%
\pgfpathlineto{\pgfqpoint{3.933665in}{1.096156in}}%
\pgfpathlineto{\pgfqpoint{3.933665in}{0.500000in}}%
\pgfpathclose%
\pgfusepath{fill}%
\end{pgfscope}%
\begin{pgfscope}%
\pgfpathrectangle{\pgfqpoint{0.750000in}{0.500000in}}{\pgfqpoint{4.650000in}{3.020000in}}%
\pgfusepath{clip}%
\pgfsetbuttcap%
\pgfsetmiterjoin%
\definecolor{currentfill}{rgb}{0.000000,0.500000,0.000000}%
\pgfsetfillcolor{currentfill}%
\pgfsetlinewidth{0.000000pt}%
\definecolor{currentstroke}{rgb}{0.000000,0.000000,0.000000}%
\pgfsetstrokecolor{currentstroke}%
\pgfsetstrokeopacity{0.000000}%
\pgfsetdash{}{0pt}%
\pgfpathmoveto{\pgfqpoint{3.966690in}{0.500000in}}%
\pgfpathlineto{\pgfqpoint{3.999716in}{0.500000in}}%
\pgfpathlineto{\pgfqpoint{3.999716in}{1.033403in}}%
\pgfpathlineto{\pgfqpoint{3.966690in}{1.033403in}}%
\pgfpathlineto{\pgfqpoint{3.966690in}{0.500000in}}%
\pgfpathclose%
\pgfusepath{fill}%
\end{pgfscope}%
\begin{pgfscope}%
\pgfpathrectangle{\pgfqpoint{0.750000in}{0.500000in}}{\pgfqpoint{4.650000in}{3.020000in}}%
\pgfusepath{clip}%
\pgfsetbuttcap%
\pgfsetmiterjoin%
\definecolor{currentfill}{rgb}{0.000000,0.500000,0.000000}%
\pgfsetfillcolor{currentfill}%
\pgfsetlinewidth{0.000000pt}%
\definecolor{currentstroke}{rgb}{0.000000,0.000000,0.000000}%
\pgfsetstrokecolor{currentstroke}%
\pgfsetstrokeopacity{0.000000}%
\pgfsetdash{}{0pt}%
\pgfpathmoveto{\pgfqpoint{3.999716in}{0.500000in}}%
\pgfpathlineto{\pgfqpoint{4.032741in}{0.500000in}}%
\pgfpathlineto{\pgfqpoint{4.032741in}{1.075238in}}%
\pgfpathlineto{\pgfqpoint{3.999716in}{1.075238in}}%
\pgfpathlineto{\pgfqpoint{3.999716in}{0.500000in}}%
\pgfpathclose%
\pgfusepath{fill}%
\end{pgfscope}%
\begin{pgfscope}%
\pgfpathrectangle{\pgfqpoint{0.750000in}{0.500000in}}{\pgfqpoint{4.650000in}{3.020000in}}%
\pgfusepath{clip}%
\pgfsetbuttcap%
\pgfsetmiterjoin%
\definecolor{currentfill}{rgb}{0.000000,0.500000,0.000000}%
\pgfsetfillcolor{currentfill}%
\pgfsetlinewidth{0.000000pt}%
\definecolor{currentstroke}{rgb}{0.000000,0.000000,0.000000}%
\pgfsetstrokecolor{currentstroke}%
\pgfsetstrokeopacity{0.000000}%
\pgfsetdash{}{0pt}%
\pgfpathmoveto{\pgfqpoint{4.032741in}{0.500000in}}%
\pgfpathlineto{\pgfqpoint{4.065767in}{0.500000in}}%
\pgfpathlineto{\pgfqpoint{4.065767in}{1.106615in}}%
\pgfpathlineto{\pgfqpoint{4.032741in}{1.106615in}}%
\pgfpathlineto{\pgfqpoint{4.032741in}{0.500000in}}%
\pgfpathclose%
\pgfusepath{fill}%
\end{pgfscope}%
\begin{pgfscope}%
\pgfpathrectangle{\pgfqpoint{0.750000in}{0.500000in}}{\pgfqpoint{4.650000in}{3.020000in}}%
\pgfusepath{clip}%
\pgfsetbuttcap%
\pgfsetmiterjoin%
\definecolor{currentfill}{rgb}{0.000000,0.500000,0.000000}%
\pgfsetfillcolor{currentfill}%
\pgfsetlinewidth{0.000000pt}%
\definecolor{currentstroke}{rgb}{0.000000,0.000000,0.000000}%
\pgfsetstrokecolor{currentstroke}%
\pgfsetstrokeopacity{0.000000}%
\pgfsetdash{}{0pt}%
\pgfpathmoveto{\pgfqpoint{4.065767in}{0.500000in}}%
\pgfpathlineto{\pgfqpoint{4.098793in}{0.500000in}}%
\pgfpathlineto{\pgfqpoint{4.098793in}{0.949732in}}%
\pgfpathlineto{\pgfqpoint{4.065767in}{0.949732in}}%
\pgfpathlineto{\pgfqpoint{4.065767in}{0.500000in}}%
\pgfpathclose%
\pgfusepath{fill}%
\end{pgfscope}%
\begin{pgfscope}%
\pgfpathrectangle{\pgfqpoint{0.750000in}{0.500000in}}{\pgfqpoint{4.650000in}{3.020000in}}%
\pgfusepath{clip}%
\pgfsetbuttcap%
\pgfsetmiterjoin%
\definecolor{currentfill}{rgb}{0.000000,0.500000,0.000000}%
\pgfsetfillcolor{currentfill}%
\pgfsetlinewidth{0.000000pt}%
\definecolor{currentstroke}{rgb}{0.000000,0.000000,0.000000}%
\pgfsetstrokecolor{currentstroke}%
\pgfsetstrokeopacity{0.000000}%
\pgfsetdash{}{0pt}%
\pgfpathmoveto{\pgfqpoint{4.098793in}{0.500000in}}%
\pgfpathlineto{\pgfqpoint{4.131818in}{0.500000in}}%
\pgfpathlineto{\pgfqpoint{4.131818in}{0.834684in}}%
\pgfpathlineto{\pgfqpoint{4.098793in}{0.834684in}}%
\pgfpathlineto{\pgfqpoint{4.098793in}{0.500000in}}%
\pgfpathclose%
\pgfusepath{fill}%
\end{pgfscope}%
\begin{pgfscope}%
\pgfpathrectangle{\pgfqpoint{0.750000in}{0.500000in}}{\pgfqpoint{4.650000in}{3.020000in}}%
\pgfusepath{clip}%
\pgfsetbuttcap%
\pgfsetmiterjoin%
\definecolor{currentfill}{rgb}{0.000000,0.500000,0.000000}%
\pgfsetfillcolor{currentfill}%
\pgfsetlinewidth{0.000000pt}%
\definecolor{currentstroke}{rgb}{0.000000,0.000000,0.000000}%
\pgfsetstrokecolor{currentstroke}%
\pgfsetstrokeopacity{0.000000}%
\pgfsetdash{}{0pt}%
\pgfpathmoveto{\pgfqpoint{4.131818in}{0.500000in}}%
\pgfpathlineto{\pgfqpoint{4.164844in}{0.500000in}}%
\pgfpathlineto{\pgfqpoint{4.164844in}{0.866061in}}%
\pgfpathlineto{\pgfqpoint{4.131818in}{0.866061in}}%
\pgfpathlineto{\pgfqpoint{4.131818in}{0.500000in}}%
\pgfpathclose%
\pgfusepath{fill}%
\end{pgfscope}%
\begin{pgfscope}%
\pgfpathrectangle{\pgfqpoint{0.750000in}{0.500000in}}{\pgfqpoint{4.650000in}{3.020000in}}%
\pgfusepath{clip}%
\pgfsetbuttcap%
\pgfsetmiterjoin%
\definecolor{currentfill}{rgb}{0.000000,0.500000,0.000000}%
\pgfsetfillcolor{currentfill}%
\pgfsetlinewidth{0.000000pt}%
\definecolor{currentstroke}{rgb}{0.000000,0.000000,0.000000}%
\pgfsetstrokecolor{currentstroke}%
\pgfsetstrokeopacity{0.000000}%
\pgfsetdash{}{0pt}%
\pgfpathmoveto{\pgfqpoint{4.164844in}{0.500000in}}%
\pgfpathlineto{\pgfqpoint{4.197869in}{0.500000in}}%
\pgfpathlineto{\pgfqpoint{4.197869in}{0.866061in}}%
\pgfpathlineto{\pgfqpoint{4.164844in}{0.866061in}}%
\pgfpathlineto{\pgfqpoint{4.164844in}{0.500000in}}%
\pgfpathclose%
\pgfusepath{fill}%
\end{pgfscope}%
\begin{pgfscope}%
\pgfpathrectangle{\pgfqpoint{0.750000in}{0.500000in}}{\pgfqpoint{4.650000in}{3.020000in}}%
\pgfusepath{clip}%
\pgfsetbuttcap%
\pgfsetmiterjoin%
\definecolor{currentfill}{rgb}{0.000000,0.500000,0.000000}%
\pgfsetfillcolor{currentfill}%
\pgfsetlinewidth{0.000000pt}%
\definecolor{currentstroke}{rgb}{0.000000,0.000000,0.000000}%
\pgfsetstrokecolor{currentstroke}%
\pgfsetstrokeopacity{0.000000}%
\pgfsetdash{}{0pt}%
\pgfpathmoveto{\pgfqpoint{4.197869in}{0.500000in}}%
\pgfpathlineto{\pgfqpoint{4.230895in}{0.500000in}}%
\pgfpathlineto{\pgfqpoint{4.230895in}{0.845143in}}%
\pgfpathlineto{\pgfqpoint{4.197869in}{0.845143in}}%
\pgfpathlineto{\pgfqpoint{4.197869in}{0.500000in}}%
\pgfpathclose%
\pgfusepath{fill}%
\end{pgfscope}%
\begin{pgfscope}%
\pgfpathrectangle{\pgfqpoint{0.750000in}{0.500000in}}{\pgfqpoint{4.650000in}{3.020000in}}%
\pgfusepath{clip}%
\pgfsetbuttcap%
\pgfsetmiterjoin%
\definecolor{currentfill}{rgb}{0.000000,0.500000,0.000000}%
\pgfsetfillcolor{currentfill}%
\pgfsetlinewidth{0.000000pt}%
\definecolor{currentstroke}{rgb}{0.000000,0.000000,0.000000}%
\pgfsetstrokecolor{currentstroke}%
\pgfsetstrokeopacity{0.000000}%
\pgfsetdash{}{0pt}%
\pgfpathmoveto{\pgfqpoint{4.230895in}{0.500000in}}%
\pgfpathlineto{\pgfqpoint{4.263920in}{0.500000in}}%
\pgfpathlineto{\pgfqpoint{4.263920in}{0.719636in}}%
\pgfpathlineto{\pgfqpoint{4.230895in}{0.719636in}}%
\pgfpathlineto{\pgfqpoint{4.230895in}{0.500000in}}%
\pgfpathclose%
\pgfusepath{fill}%
\end{pgfscope}%
\begin{pgfscope}%
\pgfpathrectangle{\pgfqpoint{0.750000in}{0.500000in}}{\pgfqpoint{4.650000in}{3.020000in}}%
\pgfusepath{clip}%
\pgfsetbuttcap%
\pgfsetmiterjoin%
\definecolor{currentfill}{rgb}{0.000000,0.500000,0.000000}%
\pgfsetfillcolor{currentfill}%
\pgfsetlinewidth{0.000000pt}%
\definecolor{currentstroke}{rgb}{0.000000,0.000000,0.000000}%
\pgfsetstrokecolor{currentstroke}%
\pgfsetstrokeopacity{0.000000}%
\pgfsetdash{}{0pt}%
\pgfpathmoveto{\pgfqpoint{4.263920in}{0.500000in}}%
\pgfpathlineto{\pgfqpoint{4.296946in}{0.500000in}}%
\pgfpathlineto{\pgfqpoint{4.296946in}{0.730095in}}%
\pgfpathlineto{\pgfqpoint{4.263920in}{0.730095in}}%
\pgfpathlineto{\pgfqpoint{4.263920in}{0.500000in}}%
\pgfpathclose%
\pgfusepath{fill}%
\end{pgfscope}%
\begin{pgfscope}%
\pgfpathrectangle{\pgfqpoint{0.750000in}{0.500000in}}{\pgfqpoint{4.650000in}{3.020000in}}%
\pgfusepath{clip}%
\pgfsetbuttcap%
\pgfsetmiterjoin%
\definecolor{currentfill}{rgb}{0.000000,0.500000,0.000000}%
\pgfsetfillcolor{currentfill}%
\pgfsetlinewidth{0.000000pt}%
\definecolor{currentstroke}{rgb}{0.000000,0.000000,0.000000}%
\pgfsetstrokecolor{currentstroke}%
\pgfsetstrokeopacity{0.000000}%
\pgfsetdash{}{0pt}%
\pgfpathmoveto{\pgfqpoint{4.296946in}{0.500000in}}%
\pgfpathlineto{\pgfqpoint{4.329972in}{0.500000in}}%
\pgfpathlineto{\pgfqpoint{4.329972in}{0.740554in}}%
\pgfpathlineto{\pgfqpoint{4.296946in}{0.740554in}}%
\pgfpathlineto{\pgfqpoint{4.296946in}{0.500000in}}%
\pgfpathclose%
\pgfusepath{fill}%
\end{pgfscope}%
\begin{pgfscope}%
\pgfpathrectangle{\pgfqpoint{0.750000in}{0.500000in}}{\pgfqpoint{4.650000in}{3.020000in}}%
\pgfusepath{clip}%
\pgfsetbuttcap%
\pgfsetmiterjoin%
\definecolor{currentfill}{rgb}{0.000000,0.500000,0.000000}%
\pgfsetfillcolor{currentfill}%
\pgfsetlinewidth{0.000000pt}%
\definecolor{currentstroke}{rgb}{0.000000,0.000000,0.000000}%
\pgfsetstrokecolor{currentstroke}%
\pgfsetstrokeopacity{0.000000}%
\pgfsetdash{}{0pt}%
\pgfpathmoveto{\pgfqpoint{4.329972in}{0.500000in}}%
\pgfpathlineto{\pgfqpoint{4.362997in}{0.500000in}}%
\pgfpathlineto{\pgfqpoint{4.362997in}{0.688260in}}%
\pgfpathlineto{\pgfqpoint{4.329972in}{0.688260in}}%
\pgfpathlineto{\pgfqpoint{4.329972in}{0.500000in}}%
\pgfpathclose%
\pgfusepath{fill}%
\end{pgfscope}%
\begin{pgfscope}%
\pgfpathrectangle{\pgfqpoint{0.750000in}{0.500000in}}{\pgfqpoint{4.650000in}{3.020000in}}%
\pgfusepath{clip}%
\pgfsetbuttcap%
\pgfsetmiterjoin%
\definecolor{currentfill}{rgb}{0.000000,0.500000,0.000000}%
\pgfsetfillcolor{currentfill}%
\pgfsetlinewidth{0.000000pt}%
\definecolor{currentstroke}{rgb}{0.000000,0.000000,0.000000}%
\pgfsetstrokecolor{currentstroke}%
\pgfsetstrokeopacity{0.000000}%
\pgfsetdash{}{0pt}%
\pgfpathmoveto{\pgfqpoint{4.362997in}{0.500000in}}%
\pgfpathlineto{\pgfqpoint{4.396023in}{0.500000in}}%
\pgfpathlineto{\pgfqpoint{4.396023in}{0.677801in}}%
\pgfpathlineto{\pgfqpoint{4.362997in}{0.677801in}}%
\pgfpathlineto{\pgfqpoint{4.362997in}{0.500000in}}%
\pgfpathclose%
\pgfusepath{fill}%
\end{pgfscope}%
\begin{pgfscope}%
\pgfpathrectangle{\pgfqpoint{0.750000in}{0.500000in}}{\pgfqpoint{4.650000in}{3.020000in}}%
\pgfusepath{clip}%
\pgfsetbuttcap%
\pgfsetmiterjoin%
\definecolor{currentfill}{rgb}{0.000000,0.500000,0.000000}%
\pgfsetfillcolor{currentfill}%
\pgfsetlinewidth{0.000000pt}%
\definecolor{currentstroke}{rgb}{0.000000,0.000000,0.000000}%
\pgfsetstrokecolor{currentstroke}%
\pgfsetstrokeopacity{0.000000}%
\pgfsetdash{}{0pt}%
\pgfpathmoveto{\pgfqpoint{4.396023in}{0.500000in}}%
\pgfpathlineto{\pgfqpoint{4.429048in}{0.500000in}}%
\pgfpathlineto{\pgfqpoint{4.429048in}{0.656883in}}%
\pgfpathlineto{\pgfqpoint{4.396023in}{0.656883in}}%
\pgfpathlineto{\pgfqpoint{4.396023in}{0.500000in}}%
\pgfpathclose%
\pgfusepath{fill}%
\end{pgfscope}%
\begin{pgfscope}%
\pgfpathrectangle{\pgfqpoint{0.750000in}{0.500000in}}{\pgfqpoint{4.650000in}{3.020000in}}%
\pgfusepath{clip}%
\pgfsetbuttcap%
\pgfsetmiterjoin%
\definecolor{currentfill}{rgb}{0.000000,0.500000,0.000000}%
\pgfsetfillcolor{currentfill}%
\pgfsetlinewidth{0.000000pt}%
\definecolor{currentstroke}{rgb}{0.000000,0.000000,0.000000}%
\pgfsetstrokecolor{currentstroke}%
\pgfsetstrokeopacity{0.000000}%
\pgfsetdash{}{0pt}%
\pgfpathmoveto{\pgfqpoint{4.429048in}{0.500000in}}%
\pgfpathlineto{\pgfqpoint{4.462074in}{0.500000in}}%
\pgfpathlineto{\pgfqpoint{4.462074in}{0.604589in}}%
\pgfpathlineto{\pgfqpoint{4.429048in}{0.604589in}}%
\pgfpathlineto{\pgfqpoint{4.429048in}{0.500000in}}%
\pgfpathclose%
\pgfusepath{fill}%
\end{pgfscope}%
\begin{pgfscope}%
\pgfpathrectangle{\pgfqpoint{0.750000in}{0.500000in}}{\pgfqpoint{4.650000in}{3.020000in}}%
\pgfusepath{clip}%
\pgfsetbuttcap%
\pgfsetmiterjoin%
\definecolor{currentfill}{rgb}{0.000000,0.500000,0.000000}%
\pgfsetfillcolor{currentfill}%
\pgfsetlinewidth{0.000000pt}%
\definecolor{currentstroke}{rgb}{0.000000,0.000000,0.000000}%
\pgfsetstrokecolor{currentstroke}%
\pgfsetstrokeopacity{0.000000}%
\pgfsetdash{}{0pt}%
\pgfpathmoveto{\pgfqpoint{4.462074in}{0.500000in}}%
\pgfpathlineto{\pgfqpoint{4.495099in}{0.500000in}}%
\pgfpathlineto{\pgfqpoint{4.495099in}{0.688260in}}%
\pgfpathlineto{\pgfqpoint{4.462074in}{0.688260in}}%
\pgfpathlineto{\pgfqpoint{4.462074in}{0.500000in}}%
\pgfpathclose%
\pgfusepath{fill}%
\end{pgfscope}%
\begin{pgfscope}%
\pgfpathrectangle{\pgfqpoint{0.750000in}{0.500000in}}{\pgfqpoint{4.650000in}{3.020000in}}%
\pgfusepath{clip}%
\pgfsetbuttcap%
\pgfsetmiterjoin%
\definecolor{currentfill}{rgb}{0.000000,0.500000,0.000000}%
\pgfsetfillcolor{currentfill}%
\pgfsetlinewidth{0.000000pt}%
\definecolor{currentstroke}{rgb}{0.000000,0.000000,0.000000}%
\pgfsetstrokecolor{currentstroke}%
\pgfsetstrokeopacity{0.000000}%
\pgfsetdash{}{0pt}%
\pgfpathmoveto{\pgfqpoint{4.495099in}{0.500000in}}%
\pgfpathlineto{\pgfqpoint{4.528125in}{0.500000in}}%
\pgfpathlineto{\pgfqpoint{4.528125in}{0.604589in}}%
\pgfpathlineto{\pgfqpoint{4.495099in}{0.604589in}}%
\pgfpathlineto{\pgfqpoint{4.495099in}{0.500000in}}%
\pgfpathclose%
\pgfusepath{fill}%
\end{pgfscope}%
\begin{pgfscope}%
\pgfpathrectangle{\pgfqpoint{0.750000in}{0.500000in}}{\pgfqpoint{4.650000in}{3.020000in}}%
\pgfusepath{clip}%
\pgfsetbuttcap%
\pgfsetmiterjoin%
\definecolor{currentfill}{rgb}{0.000000,0.500000,0.000000}%
\pgfsetfillcolor{currentfill}%
\pgfsetlinewidth{0.000000pt}%
\definecolor{currentstroke}{rgb}{0.000000,0.000000,0.000000}%
\pgfsetstrokecolor{currentstroke}%
\pgfsetstrokeopacity{0.000000}%
\pgfsetdash{}{0pt}%
\pgfpathmoveto{\pgfqpoint{4.528125in}{0.500000in}}%
\pgfpathlineto{\pgfqpoint{4.561151in}{0.500000in}}%
\pgfpathlineto{\pgfqpoint{4.561151in}{0.594130in}}%
\pgfpathlineto{\pgfqpoint{4.528125in}{0.594130in}}%
\pgfpathlineto{\pgfqpoint{4.528125in}{0.500000in}}%
\pgfpathclose%
\pgfusepath{fill}%
\end{pgfscope}%
\begin{pgfscope}%
\pgfpathrectangle{\pgfqpoint{0.750000in}{0.500000in}}{\pgfqpoint{4.650000in}{3.020000in}}%
\pgfusepath{clip}%
\pgfsetbuttcap%
\pgfsetmiterjoin%
\definecolor{currentfill}{rgb}{0.000000,0.500000,0.000000}%
\pgfsetfillcolor{currentfill}%
\pgfsetlinewidth{0.000000pt}%
\definecolor{currentstroke}{rgb}{0.000000,0.000000,0.000000}%
\pgfsetstrokecolor{currentstroke}%
\pgfsetstrokeopacity{0.000000}%
\pgfsetdash{}{0pt}%
\pgfpathmoveto{\pgfqpoint{4.561151in}{0.500000in}}%
\pgfpathlineto{\pgfqpoint{4.594176in}{0.500000in}}%
\pgfpathlineto{\pgfqpoint{4.594176in}{0.531377in}}%
\pgfpathlineto{\pgfqpoint{4.561151in}{0.531377in}}%
\pgfpathlineto{\pgfqpoint{4.561151in}{0.500000in}}%
\pgfpathclose%
\pgfusepath{fill}%
\end{pgfscope}%
\begin{pgfscope}%
\pgfpathrectangle{\pgfqpoint{0.750000in}{0.500000in}}{\pgfqpoint{4.650000in}{3.020000in}}%
\pgfusepath{clip}%
\pgfsetbuttcap%
\pgfsetmiterjoin%
\definecolor{currentfill}{rgb}{0.000000,0.500000,0.000000}%
\pgfsetfillcolor{currentfill}%
\pgfsetlinewidth{0.000000pt}%
\definecolor{currentstroke}{rgb}{0.000000,0.000000,0.000000}%
\pgfsetstrokecolor{currentstroke}%
\pgfsetstrokeopacity{0.000000}%
\pgfsetdash{}{0pt}%
\pgfpathmoveto{\pgfqpoint{4.594176in}{0.500000in}}%
\pgfpathlineto{\pgfqpoint{4.627202in}{0.500000in}}%
\pgfpathlineto{\pgfqpoint{4.627202in}{0.573212in}}%
\pgfpathlineto{\pgfqpoint{4.594176in}{0.573212in}}%
\pgfpathlineto{\pgfqpoint{4.594176in}{0.500000in}}%
\pgfpathclose%
\pgfusepath{fill}%
\end{pgfscope}%
\begin{pgfscope}%
\pgfpathrectangle{\pgfqpoint{0.750000in}{0.500000in}}{\pgfqpoint{4.650000in}{3.020000in}}%
\pgfusepath{clip}%
\pgfsetbuttcap%
\pgfsetmiterjoin%
\definecolor{currentfill}{rgb}{0.000000,0.500000,0.000000}%
\pgfsetfillcolor{currentfill}%
\pgfsetlinewidth{0.000000pt}%
\definecolor{currentstroke}{rgb}{0.000000,0.000000,0.000000}%
\pgfsetstrokecolor{currentstroke}%
\pgfsetstrokeopacity{0.000000}%
\pgfsetdash{}{0pt}%
\pgfpathmoveto{\pgfqpoint{4.627202in}{0.500000in}}%
\pgfpathlineto{\pgfqpoint{4.660227in}{0.500000in}}%
\pgfpathlineto{\pgfqpoint{4.660227in}{0.531377in}}%
\pgfpathlineto{\pgfqpoint{4.627202in}{0.531377in}}%
\pgfpathlineto{\pgfqpoint{4.627202in}{0.500000in}}%
\pgfpathclose%
\pgfusepath{fill}%
\end{pgfscope}%
\begin{pgfscope}%
\pgfpathrectangle{\pgfqpoint{0.750000in}{0.500000in}}{\pgfqpoint{4.650000in}{3.020000in}}%
\pgfusepath{clip}%
\pgfsetbuttcap%
\pgfsetmiterjoin%
\definecolor{currentfill}{rgb}{0.000000,0.500000,0.000000}%
\pgfsetfillcolor{currentfill}%
\pgfsetlinewidth{0.000000pt}%
\definecolor{currentstroke}{rgb}{0.000000,0.000000,0.000000}%
\pgfsetstrokecolor{currentstroke}%
\pgfsetstrokeopacity{0.000000}%
\pgfsetdash{}{0pt}%
\pgfpathmoveto{\pgfqpoint{4.660227in}{0.500000in}}%
\pgfpathlineto{\pgfqpoint{4.693253in}{0.500000in}}%
\pgfpathlineto{\pgfqpoint{4.693253in}{0.552294in}}%
\pgfpathlineto{\pgfqpoint{4.660227in}{0.552294in}}%
\pgfpathlineto{\pgfqpoint{4.660227in}{0.500000in}}%
\pgfpathclose%
\pgfusepath{fill}%
\end{pgfscope}%
\begin{pgfscope}%
\pgfpathrectangle{\pgfqpoint{0.750000in}{0.500000in}}{\pgfqpoint{4.650000in}{3.020000in}}%
\pgfusepath{clip}%
\pgfsetbuttcap%
\pgfsetmiterjoin%
\definecolor{currentfill}{rgb}{0.000000,0.500000,0.000000}%
\pgfsetfillcolor{currentfill}%
\pgfsetlinewidth{0.000000pt}%
\definecolor{currentstroke}{rgb}{0.000000,0.000000,0.000000}%
\pgfsetstrokecolor{currentstroke}%
\pgfsetstrokeopacity{0.000000}%
\pgfsetdash{}{0pt}%
\pgfpathmoveto{\pgfqpoint{4.693253in}{0.500000in}}%
\pgfpathlineto{\pgfqpoint{4.726278in}{0.500000in}}%
\pgfpathlineto{\pgfqpoint{4.726278in}{0.531377in}}%
\pgfpathlineto{\pgfqpoint{4.693253in}{0.531377in}}%
\pgfpathlineto{\pgfqpoint{4.693253in}{0.500000in}}%
\pgfpathclose%
\pgfusepath{fill}%
\end{pgfscope}%
\begin{pgfscope}%
\pgfpathrectangle{\pgfqpoint{0.750000in}{0.500000in}}{\pgfqpoint{4.650000in}{3.020000in}}%
\pgfusepath{clip}%
\pgfsetbuttcap%
\pgfsetmiterjoin%
\definecolor{currentfill}{rgb}{0.000000,0.500000,0.000000}%
\pgfsetfillcolor{currentfill}%
\pgfsetlinewidth{0.000000pt}%
\definecolor{currentstroke}{rgb}{0.000000,0.000000,0.000000}%
\pgfsetstrokecolor{currentstroke}%
\pgfsetstrokeopacity{0.000000}%
\pgfsetdash{}{0pt}%
\pgfpathmoveto{\pgfqpoint{4.726278in}{0.500000in}}%
\pgfpathlineto{\pgfqpoint{4.759304in}{0.500000in}}%
\pgfpathlineto{\pgfqpoint{4.759304in}{0.520918in}}%
\pgfpathlineto{\pgfqpoint{4.726278in}{0.520918in}}%
\pgfpathlineto{\pgfqpoint{4.726278in}{0.500000in}}%
\pgfpathclose%
\pgfusepath{fill}%
\end{pgfscope}%
\begin{pgfscope}%
\pgfpathrectangle{\pgfqpoint{0.750000in}{0.500000in}}{\pgfqpoint{4.650000in}{3.020000in}}%
\pgfusepath{clip}%
\pgfsetbuttcap%
\pgfsetmiterjoin%
\definecolor{currentfill}{rgb}{0.000000,0.500000,0.000000}%
\pgfsetfillcolor{currentfill}%
\pgfsetlinewidth{0.000000pt}%
\definecolor{currentstroke}{rgb}{0.000000,0.000000,0.000000}%
\pgfsetstrokecolor{currentstroke}%
\pgfsetstrokeopacity{0.000000}%
\pgfsetdash{}{0pt}%
\pgfpathmoveto{\pgfqpoint{4.759304in}{0.500000in}}%
\pgfpathlineto{\pgfqpoint{4.792330in}{0.500000in}}%
\pgfpathlineto{\pgfqpoint{4.792330in}{0.573212in}}%
\pgfpathlineto{\pgfqpoint{4.759304in}{0.573212in}}%
\pgfpathlineto{\pgfqpoint{4.759304in}{0.500000in}}%
\pgfpathclose%
\pgfusepath{fill}%
\end{pgfscope}%
\begin{pgfscope}%
\pgfpathrectangle{\pgfqpoint{0.750000in}{0.500000in}}{\pgfqpoint{4.650000in}{3.020000in}}%
\pgfusepath{clip}%
\pgfsetbuttcap%
\pgfsetmiterjoin%
\definecolor{currentfill}{rgb}{0.000000,0.500000,0.000000}%
\pgfsetfillcolor{currentfill}%
\pgfsetlinewidth{0.000000pt}%
\definecolor{currentstroke}{rgb}{0.000000,0.000000,0.000000}%
\pgfsetstrokecolor{currentstroke}%
\pgfsetstrokeopacity{0.000000}%
\pgfsetdash{}{0pt}%
\pgfpathmoveto{\pgfqpoint{4.792330in}{0.500000in}}%
\pgfpathlineto{\pgfqpoint{4.825355in}{0.500000in}}%
\pgfpathlineto{\pgfqpoint{4.825355in}{0.531377in}}%
\pgfpathlineto{\pgfqpoint{4.792330in}{0.531377in}}%
\pgfpathlineto{\pgfqpoint{4.792330in}{0.500000in}}%
\pgfpathclose%
\pgfusepath{fill}%
\end{pgfscope}%
\begin{pgfscope}%
\pgfpathrectangle{\pgfqpoint{0.750000in}{0.500000in}}{\pgfqpoint{4.650000in}{3.020000in}}%
\pgfusepath{clip}%
\pgfsetbuttcap%
\pgfsetmiterjoin%
\definecolor{currentfill}{rgb}{0.000000,0.500000,0.000000}%
\pgfsetfillcolor{currentfill}%
\pgfsetlinewidth{0.000000pt}%
\definecolor{currentstroke}{rgb}{0.000000,0.000000,0.000000}%
\pgfsetstrokecolor{currentstroke}%
\pgfsetstrokeopacity{0.000000}%
\pgfsetdash{}{0pt}%
\pgfpathmoveto{\pgfqpoint{4.825355in}{0.500000in}}%
\pgfpathlineto{\pgfqpoint{4.858381in}{0.500000in}}%
\pgfpathlineto{\pgfqpoint{4.858381in}{0.541835in}}%
\pgfpathlineto{\pgfqpoint{4.825355in}{0.541835in}}%
\pgfpathlineto{\pgfqpoint{4.825355in}{0.500000in}}%
\pgfpathclose%
\pgfusepath{fill}%
\end{pgfscope}%
\begin{pgfscope}%
\pgfpathrectangle{\pgfqpoint{0.750000in}{0.500000in}}{\pgfqpoint{4.650000in}{3.020000in}}%
\pgfusepath{clip}%
\pgfsetbuttcap%
\pgfsetmiterjoin%
\definecolor{currentfill}{rgb}{0.000000,0.500000,0.000000}%
\pgfsetfillcolor{currentfill}%
\pgfsetlinewidth{0.000000pt}%
\definecolor{currentstroke}{rgb}{0.000000,0.000000,0.000000}%
\pgfsetstrokecolor{currentstroke}%
\pgfsetstrokeopacity{0.000000}%
\pgfsetdash{}{0pt}%
\pgfpathmoveto{\pgfqpoint{4.858381in}{0.500000in}}%
\pgfpathlineto{\pgfqpoint{4.891406in}{0.500000in}}%
\pgfpathlineto{\pgfqpoint{4.891406in}{0.531377in}}%
\pgfpathlineto{\pgfqpoint{4.858381in}{0.531377in}}%
\pgfpathlineto{\pgfqpoint{4.858381in}{0.500000in}}%
\pgfpathclose%
\pgfusepath{fill}%
\end{pgfscope}%
\begin{pgfscope}%
\pgfpathrectangle{\pgfqpoint{0.750000in}{0.500000in}}{\pgfqpoint{4.650000in}{3.020000in}}%
\pgfusepath{clip}%
\pgfsetbuttcap%
\pgfsetmiterjoin%
\definecolor{currentfill}{rgb}{0.000000,0.500000,0.000000}%
\pgfsetfillcolor{currentfill}%
\pgfsetlinewidth{0.000000pt}%
\definecolor{currentstroke}{rgb}{0.000000,0.000000,0.000000}%
\pgfsetstrokecolor{currentstroke}%
\pgfsetstrokeopacity{0.000000}%
\pgfsetdash{}{0pt}%
\pgfpathmoveto{\pgfqpoint{4.891406in}{0.500000in}}%
\pgfpathlineto{\pgfqpoint{4.924432in}{0.500000in}}%
\pgfpathlineto{\pgfqpoint{4.924432in}{0.500000in}}%
\pgfpathlineto{\pgfqpoint{4.891406in}{0.500000in}}%
\pgfpathlineto{\pgfqpoint{4.891406in}{0.500000in}}%
\pgfpathclose%
\pgfusepath{fill}%
\end{pgfscope}%
\begin{pgfscope}%
\pgfpathrectangle{\pgfqpoint{0.750000in}{0.500000in}}{\pgfqpoint{4.650000in}{3.020000in}}%
\pgfusepath{clip}%
\pgfsetbuttcap%
\pgfsetmiterjoin%
\definecolor{currentfill}{rgb}{0.000000,0.500000,0.000000}%
\pgfsetfillcolor{currentfill}%
\pgfsetlinewidth{0.000000pt}%
\definecolor{currentstroke}{rgb}{0.000000,0.000000,0.000000}%
\pgfsetstrokecolor{currentstroke}%
\pgfsetstrokeopacity{0.000000}%
\pgfsetdash{}{0pt}%
\pgfpathmoveto{\pgfqpoint{4.924432in}{0.500000in}}%
\pgfpathlineto{\pgfqpoint{4.957457in}{0.500000in}}%
\pgfpathlineto{\pgfqpoint{4.957457in}{0.510459in}}%
\pgfpathlineto{\pgfqpoint{4.924432in}{0.510459in}}%
\pgfpathlineto{\pgfqpoint{4.924432in}{0.500000in}}%
\pgfpathclose%
\pgfusepath{fill}%
\end{pgfscope}%
\begin{pgfscope}%
\pgfpathrectangle{\pgfqpoint{0.750000in}{0.500000in}}{\pgfqpoint{4.650000in}{3.020000in}}%
\pgfusepath{clip}%
\pgfsetbuttcap%
\pgfsetmiterjoin%
\definecolor{currentfill}{rgb}{0.000000,0.500000,0.000000}%
\pgfsetfillcolor{currentfill}%
\pgfsetlinewidth{0.000000pt}%
\definecolor{currentstroke}{rgb}{0.000000,0.000000,0.000000}%
\pgfsetstrokecolor{currentstroke}%
\pgfsetstrokeopacity{0.000000}%
\pgfsetdash{}{0pt}%
\pgfpathmoveto{\pgfqpoint{4.957457in}{0.500000in}}%
\pgfpathlineto{\pgfqpoint{4.990483in}{0.500000in}}%
\pgfpathlineto{\pgfqpoint{4.990483in}{0.500000in}}%
\pgfpathlineto{\pgfqpoint{4.957457in}{0.500000in}}%
\pgfpathlineto{\pgfqpoint{4.957457in}{0.500000in}}%
\pgfpathclose%
\pgfusepath{fill}%
\end{pgfscope}%
\begin{pgfscope}%
\pgfpathrectangle{\pgfqpoint{0.750000in}{0.500000in}}{\pgfqpoint{4.650000in}{3.020000in}}%
\pgfusepath{clip}%
\pgfsetbuttcap%
\pgfsetmiterjoin%
\definecolor{currentfill}{rgb}{0.000000,0.500000,0.000000}%
\pgfsetfillcolor{currentfill}%
\pgfsetlinewidth{0.000000pt}%
\definecolor{currentstroke}{rgb}{0.000000,0.000000,0.000000}%
\pgfsetstrokecolor{currentstroke}%
\pgfsetstrokeopacity{0.000000}%
\pgfsetdash{}{0pt}%
\pgfpathmoveto{\pgfqpoint{4.990483in}{0.500000in}}%
\pgfpathlineto{\pgfqpoint{5.023509in}{0.500000in}}%
\pgfpathlineto{\pgfqpoint{5.023509in}{0.520918in}}%
\pgfpathlineto{\pgfqpoint{4.990483in}{0.520918in}}%
\pgfpathlineto{\pgfqpoint{4.990483in}{0.500000in}}%
\pgfpathclose%
\pgfusepath{fill}%
\end{pgfscope}%
\begin{pgfscope}%
\pgfpathrectangle{\pgfqpoint{0.750000in}{0.500000in}}{\pgfqpoint{4.650000in}{3.020000in}}%
\pgfusepath{clip}%
\pgfsetbuttcap%
\pgfsetmiterjoin%
\definecolor{currentfill}{rgb}{0.000000,0.500000,0.000000}%
\pgfsetfillcolor{currentfill}%
\pgfsetlinewidth{0.000000pt}%
\definecolor{currentstroke}{rgb}{0.000000,0.000000,0.000000}%
\pgfsetstrokecolor{currentstroke}%
\pgfsetstrokeopacity{0.000000}%
\pgfsetdash{}{0pt}%
\pgfpathmoveto{\pgfqpoint{5.023509in}{0.500000in}}%
\pgfpathlineto{\pgfqpoint{5.056534in}{0.500000in}}%
\pgfpathlineto{\pgfqpoint{5.056534in}{0.510459in}}%
\pgfpathlineto{\pgfqpoint{5.023509in}{0.510459in}}%
\pgfpathlineto{\pgfqpoint{5.023509in}{0.500000in}}%
\pgfpathclose%
\pgfusepath{fill}%
\end{pgfscope}%
\begin{pgfscope}%
\pgfpathrectangle{\pgfqpoint{0.750000in}{0.500000in}}{\pgfqpoint{4.650000in}{3.020000in}}%
\pgfusepath{clip}%
\pgfsetbuttcap%
\pgfsetmiterjoin%
\definecolor{currentfill}{rgb}{0.000000,0.500000,0.000000}%
\pgfsetfillcolor{currentfill}%
\pgfsetlinewidth{0.000000pt}%
\definecolor{currentstroke}{rgb}{0.000000,0.000000,0.000000}%
\pgfsetstrokecolor{currentstroke}%
\pgfsetstrokeopacity{0.000000}%
\pgfsetdash{}{0pt}%
\pgfpathmoveto{\pgfqpoint{5.056534in}{0.500000in}}%
\pgfpathlineto{\pgfqpoint{5.089560in}{0.500000in}}%
\pgfpathlineto{\pgfqpoint{5.089560in}{0.500000in}}%
\pgfpathlineto{\pgfqpoint{5.056534in}{0.500000in}}%
\pgfpathlineto{\pgfqpoint{5.056534in}{0.500000in}}%
\pgfpathclose%
\pgfusepath{fill}%
\end{pgfscope}%
\begin{pgfscope}%
\pgfpathrectangle{\pgfqpoint{0.750000in}{0.500000in}}{\pgfqpoint{4.650000in}{3.020000in}}%
\pgfusepath{clip}%
\pgfsetbuttcap%
\pgfsetmiterjoin%
\definecolor{currentfill}{rgb}{0.000000,0.500000,0.000000}%
\pgfsetfillcolor{currentfill}%
\pgfsetlinewidth{0.000000pt}%
\definecolor{currentstroke}{rgb}{0.000000,0.000000,0.000000}%
\pgfsetstrokecolor{currentstroke}%
\pgfsetstrokeopacity{0.000000}%
\pgfsetdash{}{0pt}%
\pgfpathmoveto{\pgfqpoint{5.089560in}{0.500000in}}%
\pgfpathlineto{\pgfqpoint{5.122585in}{0.500000in}}%
\pgfpathlineto{\pgfqpoint{5.122585in}{0.500000in}}%
\pgfpathlineto{\pgfqpoint{5.089560in}{0.500000in}}%
\pgfpathlineto{\pgfqpoint{5.089560in}{0.500000in}}%
\pgfpathclose%
\pgfusepath{fill}%
\end{pgfscope}%
\begin{pgfscope}%
\pgfpathrectangle{\pgfqpoint{0.750000in}{0.500000in}}{\pgfqpoint{4.650000in}{3.020000in}}%
\pgfusepath{clip}%
\pgfsetbuttcap%
\pgfsetmiterjoin%
\definecolor{currentfill}{rgb}{0.000000,0.500000,0.000000}%
\pgfsetfillcolor{currentfill}%
\pgfsetlinewidth{0.000000pt}%
\definecolor{currentstroke}{rgb}{0.000000,0.000000,0.000000}%
\pgfsetstrokecolor{currentstroke}%
\pgfsetstrokeopacity{0.000000}%
\pgfsetdash{}{0pt}%
\pgfpathmoveto{\pgfqpoint{5.122585in}{0.500000in}}%
\pgfpathlineto{\pgfqpoint{5.155611in}{0.500000in}}%
\pgfpathlineto{\pgfqpoint{5.155611in}{0.510459in}}%
\pgfpathlineto{\pgfqpoint{5.122585in}{0.510459in}}%
\pgfpathlineto{\pgfqpoint{5.122585in}{0.500000in}}%
\pgfpathclose%
\pgfusepath{fill}%
\end{pgfscope}%
\begin{pgfscope}%
\pgfpathrectangle{\pgfqpoint{0.750000in}{0.500000in}}{\pgfqpoint{4.650000in}{3.020000in}}%
\pgfusepath{clip}%
\pgfsetbuttcap%
\pgfsetmiterjoin%
\definecolor{currentfill}{rgb}{0.000000,0.500000,0.000000}%
\pgfsetfillcolor{currentfill}%
\pgfsetlinewidth{0.000000pt}%
\definecolor{currentstroke}{rgb}{0.000000,0.000000,0.000000}%
\pgfsetstrokecolor{currentstroke}%
\pgfsetstrokeopacity{0.000000}%
\pgfsetdash{}{0pt}%
\pgfpathmoveto{\pgfqpoint{5.155611in}{0.500000in}}%
\pgfpathlineto{\pgfqpoint{5.188636in}{0.500000in}}%
\pgfpathlineto{\pgfqpoint{5.188636in}{0.510459in}}%
\pgfpathlineto{\pgfqpoint{5.155611in}{0.510459in}}%
\pgfpathlineto{\pgfqpoint{5.155611in}{0.500000in}}%
\pgfpathclose%
\pgfusepath{fill}%
\end{pgfscope}%
\begin{pgfscope}%
\pgfsetbuttcap%
\pgfsetroundjoin%
\definecolor{currentfill}{rgb}{0.000000,0.000000,0.000000}%
\pgfsetfillcolor{currentfill}%
\pgfsetlinewidth{0.803000pt}%
\definecolor{currentstroke}{rgb}{0.000000,0.000000,0.000000}%
\pgfsetstrokecolor{currentstroke}%
\pgfsetdash{}{0pt}%
\pgfsys@defobject{currentmarker}{\pgfqpoint{0.000000in}{-0.048611in}}{\pgfqpoint{0.000000in}{0.000000in}}{%
\pgfpathmoveto{\pgfqpoint{0.000000in}{0.000000in}}%
\pgfpathlineto{\pgfqpoint{0.000000in}{-0.048611in}}%
\pgfusepath{stroke,fill}%
}%
\begin{pgfscope}%
\pgfsys@transformshift{0.782488in}{0.500000in}%
\pgfsys@useobject{currentmarker}{}%
\end{pgfscope}%
\end{pgfscope}%
\begin{pgfscope}%
\definecolor{textcolor}{rgb}{0.000000,0.000000,0.000000}%
\pgfsetstrokecolor{textcolor}%
\pgfsetfillcolor{textcolor}%
\pgftext[x=0.782488in,y=0.402778in,,top]{\color{textcolor}\sffamily\fontsize{10.000000}{12.000000}\selectfont 0.30}%
\end{pgfscope}%
\begin{pgfscope}%
\pgfsetbuttcap%
\pgfsetroundjoin%
\definecolor{currentfill}{rgb}{0.000000,0.000000,0.000000}%
\pgfsetfillcolor{currentfill}%
\pgfsetlinewidth{0.803000pt}%
\definecolor{currentstroke}{rgb}{0.000000,0.000000,0.000000}%
\pgfsetstrokecolor{currentstroke}%
\pgfsetdash{}{0pt}%
\pgfsys@defobject{currentmarker}{\pgfqpoint{0.000000in}{-0.048611in}}{\pgfqpoint{0.000000in}{0.000000in}}{%
\pgfpathmoveto{\pgfqpoint{0.000000in}{0.000000in}}%
\pgfpathlineto{\pgfqpoint{0.000000in}{-0.048611in}}%
\pgfusepath{stroke,fill}%
}%
\begin{pgfscope}%
\pgfsys@transformshift{1.426674in}{0.500000in}%
\pgfsys@useobject{currentmarker}{}%
\end{pgfscope}%
\end{pgfscope}%
\begin{pgfscope}%
\definecolor{textcolor}{rgb}{0.000000,0.000000,0.000000}%
\pgfsetstrokecolor{textcolor}%
\pgfsetfillcolor{textcolor}%
\pgftext[x=1.426674in,y=0.402778in,,top]{\color{textcolor}\sffamily\fontsize{10.000000}{12.000000}\selectfont 0.35}%
\end{pgfscope}%
\begin{pgfscope}%
\pgfsetbuttcap%
\pgfsetroundjoin%
\definecolor{currentfill}{rgb}{0.000000,0.000000,0.000000}%
\pgfsetfillcolor{currentfill}%
\pgfsetlinewidth{0.803000pt}%
\definecolor{currentstroke}{rgb}{0.000000,0.000000,0.000000}%
\pgfsetstrokecolor{currentstroke}%
\pgfsetdash{}{0pt}%
\pgfsys@defobject{currentmarker}{\pgfqpoint{0.000000in}{-0.048611in}}{\pgfqpoint{0.000000in}{0.000000in}}{%
\pgfpathmoveto{\pgfqpoint{0.000000in}{0.000000in}}%
\pgfpathlineto{\pgfqpoint{0.000000in}{-0.048611in}}%
\pgfusepath{stroke,fill}%
}%
\begin{pgfscope}%
\pgfsys@transformshift{2.070859in}{0.500000in}%
\pgfsys@useobject{currentmarker}{}%
\end{pgfscope}%
\end{pgfscope}%
\begin{pgfscope}%
\definecolor{textcolor}{rgb}{0.000000,0.000000,0.000000}%
\pgfsetstrokecolor{textcolor}%
\pgfsetfillcolor{textcolor}%
\pgftext[x=2.070859in,y=0.402778in,,top]{\color{textcolor}\sffamily\fontsize{10.000000}{12.000000}\selectfont 0.40}%
\end{pgfscope}%
\begin{pgfscope}%
\pgfsetbuttcap%
\pgfsetroundjoin%
\definecolor{currentfill}{rgb}{0.000000,0.000000,0.000000}%
\pgfsetfillcolor{currentfill}%
\pgfsetlinewidth{0.803000pt}%
\definecolor{currentstroke}{rgb}{0.000000,0.000000,0.000000}%
\pgfsetstrokecolor{currentstroke}%
\pgfsetdash{}{0pt}%
\pgfsys@defobject{currentmarker}{\pgfqpoint{0.000000in}{-0.048611in}}{\pgfqpoint{0.000000in}{0.000000in}}{%
\pgfpathmoveto{\pgfqpoint{0.000000in}{0.000000in}}%
\pgfpathlineto{\pgfqpoint{0.000000in}{-0.048611in}}%
\pgfusepath{stroke,fill}%
}%
\begin{pgfscope}%
\pgfsys@transformshift{2.715044in}{0.500000in}%
\pgfsys@useobject{currentmarker}{}%
\end{pgfscope}%
\end{pgfscope}%
\begin{pgfscope}%
\definecolor{textcolor}{rgb}{0.000000,0.000000,0.000000}%
\pgfsetstrokecolor{textcolor}%
\pgfsetfillcolor{textcolor}%
\pgftext[x=2.715044in,y=0.402778in,,top]{\color{textcolor}\sffamily\fontsize{10.000000}{12.000000}\selectfont 0.45}%
\end{pgfscope}%
\begin{pgfscope}%
\pgfsetbuttcap%
\pgfsetroundjoin%
\definecolor{currentfill}{rgb}{0.000000,0.000000,0.000000}%
\pgfsetfillcolor{currentfill}%
\pgfsetlinewidth{0.803000pt}%
\definecolor{currentstroke}{rgb}{0.000000,0.000000,0.000000}%
\pgfsetstrokecolor{currentstroke}%
\pgfsetdash{}{0pt}%
\pgfsys@defobject{currentmarker}{\pgfqpoint{0.000000in}{-0.048611in}}{\pgfqpoint{0.000000in}{0.000000in}}{%
\pgfpathmoveto{\pgfqpoint{0.000000in}{0.000000in}}%
\pgfpathlineto{\pgfqpoint{0.000000in}{-0.048611in}}%
\pgfusepath{stroke,fill}%
}%
\begin{pgfscope}%
\pgfsys@transformshift{3.359230in}{0.500000in}%
\pgfsys@useobject{currentmarker}{}%
\end{pgfscope}%
\end{pgfscope}%
\begin{pgfscope}%
\definecolor{textcolor}{rgb}{0.000000,0.000000,0.000000}%
\pgfsetstrokecolor{textcolor}%
\pgfsetfillcolor{textcolor}%
\pgftext[x=3.359230in,y=0.402778in,,top]{\color{textcolor}\sffamily\fontsize{10.000000}{12.000000}\selectfont 0.50}%
\end{pgfscope}%
\begin{pgfscope}%
\pgfsetbuttcap%
\pgfsetroundjoin%
\definecolor{currentfill}{rgb}{0.000000,0.000000,0.000000}%
\pgfsetfillcolor{currentfill}%
\pgfsetlinewidth{0.803000pt}%
\definecolor{currentstroke}{rgb}{0.000000,0.000000,0.000000}%
\pgfsetstrokecolor{currentstroke}%
\pgfsetdash{}{0pt}%
\pgfsys@defobject{currentmarker}{\pgfqpoint{0.000000in}{-0.048611in}}{\pgfqpoint{0.000000in}{0.000000in}}{%
\pgfpathmoveto{\pgfqpoint{0.000000in}{0.000000in}}%
\pgfpathlineto{\pgfqpoint{0.000000in}{-0.048611in}}%
\pgfusepath{stroke,fill}%
}%
\begin{pgfscope}%
\pgfsys@transformshift{4.003415in}{0.500000in}%
\pgfsys@useobject{currentmarker}{}%
\end{pgfscope}%
\end{pgfscope}%
\begin{pgfscope}%
\definecolor{textcolor}{rgb}{0.000000,0.000000,0.000000}%
\pgfsetstrokecolor{textcolor}%
\pgfsetfillcolor{textcolor}%
\pgftext[x=4.003415in,y=0.402778in,,top]{\color{textcolor}\sffamily\fontsize{10.000000}{12.000000}\selectfont 0.55}%
\end{pgfscope}%
\begin{pgfscope}%
\pgfsetbuttcap%
\pgfsetroundjoin%
\definecolor{currentfill}{rgb}{0.000000,0.000000,0.000000}%
\pgfsetfillcolor{currentfill}%
\pgfsetlinewidth{0.803000pt}%
\definecolor{currentstroke}{rgb}{0.000000,0.000000,0.000000}%
\pgfsetstrokecolor{currentstroke}%
\pgfsetdash{}{0pt}%
\pgfsys@defobject{currentmarker}{\pgfqpoint{0.000000in}{-0.048611in}}{\pgfqpoint{0.000000in}{0.000000in}}{%
\pgfpathmoveto{\pgfqpoint{0.000000in}{0.000000in}}%
\pgfpathlineto{\pgfqpoint{0.000000in}{-0.048611in}}%
\pgfusepath{stroke,fill}%
}%
\begin{pgfscope}%
\pgfsys@transformshift{4.647600in}{0.500000in}%
\pgfsys@useobject{currentmarker}{}%
\end{pgfscope}%
\end{pgfscope}%
\begin{pgfscope}%
\definecolor{textcolor}{rgb}{0.000000,0.000000,0.000000}%
\pgfsetstrokecolor{textcolor}%
\pgfsetfillcolor{textcolor}%
\pgftext[x=4.647600in,y=0.402778in,,top]{\color{textcolor}\sffamily\fontsize{10.000000}{12.000000}\selectfont 0.60}%
\end{pgfscope}%
\begin{pgfscope}%
\pgfsetbuttcap%
\pgfsetroundjoin%
\definecolor{currentfill}{rgb}{0.000000,0.000000,0.000000}%
\pgfsetfillcolor{currentfill}%
\pgfsetlinewidth{0.803000pt}%
\definecolor{currentstroke}{rgb}{0.000000,0.000000,0.000000}%
\pgfsetstrokecolor{currentstroke}%
\pgfsetdash{}{0pt}%
\pgfsys@defobject{currentmarker}{\pgfqpoint{0.000000in}{-0.048611in}}{\pgfqpoint{0.000000in}{0.000000in}}{%
\pgfpathmoveto{\pgfqpoint{0.000000in}{0.000000in}}%
\pgfpathlineto{\pgfqpoint{0.000000in}{-0.048611in}}%
\pgfusepath{stroke,fill}%
}%
\begin{pgfscope}%
\pgfsys@transformshift{5.291786in}{0.500000in}%
\pgfsys@useobject{currentmarker}{}%
\end{pgfscope}%
\end{pgfscope}%
\begin{pgfscope}%
\definecolor{textcolor}{rgb}{0.000000,0.000000,0.000000}%
\pgfsetstrokecolor{textcolor}%
\pgfsetfillcolor{textcolor}%
\pgftext[x=5.291786in,y=0.402778in,,top]{\color{textcolor}\sffamily\fontsize{10.000000}{12.000000}\selectfont 0.65}%
\end{pgfscope}%
\begin{pgfscope}%
\definecolor{textcolor}{rgb}{0.000000,0.000000,0.000000}%
\pgfsetstrokecolor{textcolor}%
\pgfsetfillcolor{textcolor}%
\pgftext[x=3.075000in,y=0.212809in,,top]{\color{textcolor}\sffamily\fontsize{10.000000}{12.000000}\selectfont loss}%
\end{pgfscope}%
\begin{pgfscope}%
\pgfsetbuttcap%
\pgfsetroundjoin%
\definecolor{currentfill}{rgb}{0.000000,0.000000,0.000000}%
\pgfsetfillcolor{currentfill}%
\pgfsetlinewidth{0.803000pt}%
\definecolor{currentstroke}{rgb}{0.000000,0.000000,0.000000}%
\pgfsetstrokecolor{currentstroke}%
\pgfsetdash{}{0pt}%
\pgfsys@defobject{currentmarker}{\pgfqpoint{-0.048611in}{0.000000in}}{\pgfqpoint{-0.000000in}{0.000000in}}{%
\pgfpathmoveto{\pgfqpoint{-0.000000in}{0.000000in}}%
\pgfpathlineto{\pgfqpoint{-0.048611in}{0.000000in}}%
\pgfusepath{stroke,fill}%
}%
\begin{pgfscope}%
\pgfsys@transformshift{0.750000in}{0.500000in}%
\pgfsys@useobject{currentmarker}{}%
\end{pgfscope}%
\end{pgfscope}%
\begin{pgfscope}%
\definecolor{textcolor}{rgb}{0.000000,0.000000,0.000000}%
\pgfsetstrokecolor{textcolor}%
\pgfsetfillcolor{textcolor}%
\pgftext[x=0.564412in, y=0.447238in, left, base]{\color{textcolor}\sffamily\fontsize{10.000000}{12.000000}\selectfont 0}%
\end{pgfscope}%
\begin{pgfscope}%
\pgfsetbuttcap%
\pgfsetroundjoin%
\definecolor{currentfill}{rgb}{0.000000,0.000000,0.000000}%
\pgfsetfillcolor{currentfill}%
\pgfsetlinewidth{0.803000pt}%
\definecolor{currentstroke}{rgb}{0.000000,0.000000,0.000000}%
\pgfsetstrokecolor{currentstroke}%
\pgfsetdash{}{0pt}%
\pgfsys@defobject{currentmarker}{\pgfqpoint{-0.048611in}{0.000000in}}{\pgfqpoint{-0.000000in}{0.000000in}}{%
\pgfpathmoveto{\pgfqpoint{-0.000000in}{0.000000in}}%
\pgfpathlineto{\pgfqpoint{-0.048611in}{0.000000in}}%
\pgfusepath{stroke,fill}%
}%
\begin{pgfscope}%
\pgfsys@transformshift{0.750000in}{1.022944in}%
\pgfsys@useobject{currentmarker}{}%
\end{pgfscope}%
\end{pgfscope}%
\begin{pgfscope}%
\definecolor{textcolor}{rgb}{0.000000,0.000000,0.000000}%
\pgfsetstrokecolor{textcolor}%
\pgfsetfillcolor{textcolor}%
\pgftext[x=0.476047in, y=0.970182in, left, base]{\color{textcolor}\sffamily\fontsize{10.000000}{12.000000}\selectfont 50}%
\end{pgfscope}%
\begin{pgfscope}%
\pgfsetbuttcap%
\pgfsetroundjoin%
\definecolor{currentfill}{rgb}{0.000000,0.000000,0.000000}%
\pgfsetfillcolor{currentfill}%
\pgfsetlinewidth{0.803000pt}%
\definecolor{currentstroke}{rgb}{0.000000,0.000000,0.000000}%
\pgfsetstrokecolor{currentstroke}%
\pgfsetdash{}{0pt}%
\pgfsys@defobject{currentmarker}{\pgfqpoint{-0.048611in}{0.000000in}}{\pgfqpoint{-0.000000in}{0.000000in}}{%
\pgfpathmoveto{\pgfqpoint{-0.000000in}{0.000000in}}%
\pgfpathlineto{\pgfqpoint{-0.048611in}{0.000000in}}%
\pgfusepath{stroke,fill}%
}%
\begin{pgfscope}%
\pgfsys@transformshift{0.750000in}{1.545887in}%
\pgfsys@useobject{currentmarker}{}%
\end{pgfscope}%
\end{pgfscope}%
\begin{pgfscope}%
\definecolor{textcolor}{rgb}{0.000000,0.000000,0.000000}%
\pgfsetstrokecolor{textcolor}%
\pgfsetfillcolor{textcolor}%
\pgftext[x=0.387682in, y=1.493126in, left, base]{\color{textcolor}\sffamily\fontsize{10.000000}{12.000000}\selectfont 100}%
\end{pgfscope}%
\begin{pgfscope}%
\pgfsetbuttcap%
\pgfsetroundjoin%
\definecolor{currentfill}{rgb}{0.000000,0.000000,0.000000}%
\pgfsetfillcolor{currentfill}%
\pgfsetlinewidth{0.803000pt}%
\definecolor{currentstroke}{rgb}{0.000000,0.000000,0.000000}%
\pgfsetstrokecolor{currentstroke}%
\pgfsetdash{}{0pt}%
\pgfsys@defobject{currentmarker}{\pgfqpoint{-0.048611in}{0.000000in}}{\pgfqpoint{-0.000000in}{0.000000in}}{%
\pgfpathmoveto{\pgfqpoint{-0.000000in}{0.000000in}}%
\pgfpathlineto{\pgfqpoint{-0.048611in}{0.000000in}}%
\pgfusepath{stroke,fill}%
}%
\begin{pgfscope}%
\pgfsys@transformshift{0.750000in}{2.068831in}%
\pgfsys@useobject{currentmarker}{}%
\end{pgfscope}%
\end{pgfscope}%
\begin{pgfscope}%
\definecolor{textcolor}{rgb}{0.000000,0.000000,0.000000}%
\pgfsetstrokecolor{textcolor}%
\pgfsetfillcolor{textcolor}%
\pgftext[x=0.387682in, y=2.016070in, left, base]{\color{textcolor}\sffamily\fontsize{10.000000}{12.000000}\selectfont 150}%
\end{pgfscope}%
\begin{pgfscope}%
\pgfsetbuttcap%
\pgfsetroundjoin%
\definecolor{currentfill}{rgb}{0.000000,0.000000,0.000000}%
\pgfsetfillcolor{currentfill}%
\pgfsetlinewidth{0.803000pt}%
\definecolor{currentstroke}{rgb}{0.000000,0.000000,0.000000}%
\pgfsetstrokecolor{currentstroke}%
\pgfsetdash{}{0pt}%
\pgfsys@defobject{currentmarker}{\pgfqpoint{-0.048611in}{0.000000in}}{\pgfqpoint{-0.000000in}{0.000000in}}{%
\pgfpathmoveto{\pgfqpoint{-0.000000in}{0.000000in}}%
\pgfpathlineto{\pgfqpoint{-0.048611in}{0.000000in}}%
\pgfusepath{stroke,fill}%
}%
\begin{pgfscope}%
\pgfsys@transformshift{0.750000in}{2.591775in}%
\pgfsys@useobject{currentmarker}{}%
\end{pgfscope}%
\end{pgfscope}%
\begin{pgfscope}%
\definecolor{textcolor}{rgb}{0.000000,0.000000,0.000000}%
\pgfsetstrokecolor{textcolor}%
\pgfsetfillcolor{textcolor}%
\pgftext[x=0.387682in, y=2.539013in, left, base]{\color{textcolor}\sffamily\fontsize{10.000000}{12.000000}\selectfont 200}%
\end{pgfscope}%
\begin{pgfscope}%
\pgfsetbuttcap%
\pgfsetroundjoin%
\definecolor{currentfill}{rgb}{0.000000,0.000000,0.000000}%
\pgfsetfillcolor{currentfill}%
\pgfsetlinewidth{0.803000pt}%
\definecolor{currentstroke}{rgb}{0.000000,0.000000,0.000000}%
\pgfsetstrokecolor{currentstroke}%
\pgfsetdash{}{0pt}%
\pgfsys@defobject{currentmarker}{\pgfqpoint{-0.048611in}{0.000000in}}{\pgfqpoint{-0.000000in}{0.000000in}}{%
\pgfpathmoveto{\pgfqpoint{-0.000000in}{0.000000in}}%
\pgfpathlineto{\pgfqpoint{-0.048611in}{0.000000in}}%
\pgfusepath{stroke,fill}%
}%
\begin{pgfscope}%
\pgfsys@transformshift{0.750000in}{3.114719in}%
\pgfsys@useobject{currentmarker}{}%
\end{pgfscope}%
\end{pgfscope}%
\begin{pgfscope}%
\definecolor{textcolor}{rgb}{0.000000,0.000000,0.000000}%
\pgfsetstrokecolor{textcolor}%
\pgfsetfillcolor{textcolor}%
\pgftext[x=0.387682in, y=3.061957in, left, base]{\color{textcolor}\sffamily\fontsize{10.000000}{12.000000}\selectfont 250}%
\end{pgfscope}%
\begin{pgfscope}%
\definecolor{textcolor}{rgb}{0.000000,0.000000,0.000000}%
\pgfsetstrokecolor{textcolor}%
\pgfsetfillcolor{textcolor}%
\pgftext[x=0.332126in,y=2.010000in,,bottom,rotate=90.000000]{\color{textcolor}\sffamily\fontsize{10.000000}{12.000000}\selectfont count}%
\end{pgfscope}%
\begin{pgfscope}%
\pgfsetrectcap%
\pgfsetmiterjoin%
\pgfsetlinewidth{0.803000pt}%
\definecolor{currentstroke}{rgb}{0.000000,0.000000,0.000000}%
\pgfsetstrokecolor{currentstroke}%
\pgfsetdash{}{0pt}%
\pgfpathmoveto{\pgfqpoint{0.750000in}{0.500000in}}%
\pgfpathlineto{\pgfqpoint{0.750000in}{3.520000in}}%
\pgfusepath{stroke}%
\end{pgfscope}%
\begin{pgfscope}%
\pgfsetrectcap%
\pgfsetmiterjoin%
\pgfsetlinewidth{0.803000pt}%
\definecolor{currentstroke}{rgb}{0.000000,0.000000,0.000000}%
\pgfsetstrokecolor{currentstroke}%
\pgfsetdash{}{0pt}%
\pgfpathmoveto{\pgfqpoint{5.400000in}{0.500000in}}%
\pgfpathlineto{\pgfqpoint{5.400000in}{3.520000in}}%
\pgfusepath{stroke}%
\end{pgfscope}%
\begin{pgfscope}%
\pgfsetrectcap%
\pgfsetmiterjoin%
\pgfsetlinewidth{0.803000pt}%
\definecolor{currentstroke}{rgb}{0.000000,0.000000,0.000000}%
\pgfsetstrokecolor{currentstroke}%
\pgfsetdash{}{0pt}%
\pgfpathmoveto{\pgfqpoint{0.750000in}{0.500000in}}%
\pgfpathlineto{\pgfqpoint{5.400000in}{0.500000in}}%
\pgfusepath{stroke}%
\end{pgfscope}%
\begin{pgfscope}%
\pgfsetrectcap%
\pgfsetmiterjoin%
\pgfsetlinewidth{0.803000pt}%
\definecolor{currentstroke}{rgb}{0.000000,0.000000,0.000000}%
\pgfsetstrokecolor{currentstroke}%
\pgfsetdash{}{0pt}%
\pgfpathmoveto{\pgfqpoint{0.750000in}{3.520000in}}%
\pgfpathlineto{\pgfqpoint{5.400000in}{3.520000in}}%
\pgfusepath{stroke}%
\end{pgfscope}%
\begin{pgfscope}%
\definecolor{textcolor}{rgb}{0.000000,0.000000,0.000000}%
\pgfsetstrokecolor{textcolor}%
\pgfsetfillcolor{textcolor}%
\pgftext[x=3.075000in,y=3.603333in,,base]{\color{textcolor}\sffamily\fontsize{12.000000}{14.400000}\selectfont loss over time for Rijndael's MixColumns function}%
\end{pgfscope}%
\begin{pgfscope}%
\pgfsetbuttcap%
\pgfsetmiterjoin%
\definecolor{currentfill}{rgb}{1.000000,1.000000,1.000000}%
\pgfsetfillcolor{currentfill}%
\pgfsetfillopacity{0.800000}%
\pgfsetlinewidth{1.003750pt}%
\definecolor{currentstroke}{rgb}{0.800000,0.800000,0.800000}%
\pgfsetstrokecolor{currentstroke}%
\pgfsetstrokeopacity{0.800000}%
\pgfsetdash{}{0pt}%
\pgfpathmoveto{\pgfqpoint{4.562381in}{3.001174in}}%
\pgfpathlineto{\pgfqpoint{5.302778in}{3.001174in}}%
\pgfpathquadraticcurveto{\pgfqpoint{5.330556in}{3.001174in}}{\pgfqpoint{5.330556in}{3.028952in}}%
\pgfpathlineto{\pgfqpoint{5.330556in}{3.422778in}}%
\pgfpathquadraticcurveto{\pgfqpoint{5.330556in}{3.450556in}}{\pgfqpoint{5.302778in}{3.450556in}}%
\pgfpathlineto{\pgfqpoint{4.562381in}{3.450556in}}%
\pgfpathquadraticcurveto{\pgfqpoint{4.534603in}{3.450556in}}{\pgfqpoint{4.534603in}{3.422778in}}%
\pgfpathlineto{\pgfqpoint{4.534603in}{3.028952in}}%
\pgfpathquadraticcurveto{\pgfqpoint{4.534603in}{3.001174in}}{\pgfqpoint{4.562381in}{3.001174in}}%
\pgfpathlineto{\pgfqpoint{4.562381in}{3.001174in}}%
\pgfpathclose%
\pgfusepath{stroke,fill}%
\end{pgfscope}%
\begin{pgfscope}%
\pgfsetbuttcap%
\pgfsetmiterjoin%
\definecolor{currentfill}{rgb}{1.000000,0.000000,0.000000}%
\pgfsetfillcolor{currentfill}%
\pgfsetlinewidth{0.000000pt}%
\definecolor{currentstroke}{rgb}{0.000000,0.000000,0.000000}%
\pgfsetstrokecolor{currentstroke}%
\pgfsetstrokeopacity{0.000000}%
\pgfsetdash{}{0pt}%
\pgfpathmoveto{\pgfqpoint{4.590158in}{3.289477in}}%
\pgfpathlineto{\pgfqpoint{4.867936in}{3.289477in}}%
\pgfpathlineto{\pgfqpoint{4.867936in}{3.386699in}}%
\pgfpathlineto{\pgfqpoint{4.590158in}{3.386699in}}%
\pgfpathlineto{\pgfqpoint{4.590158in}{3.289477in}}%
\pgfpathclose%
\pgfusepath{fill}%
\end{pgfscope}%
\begin{pgfscope}%
\definecolor{textcolor}{rgb}{0.000000,0.000000,0.000000}%
\pgfsetstrokecolor{textcolor}%
\pgfsetfillcolor{textcolor}%
\pgftext[x=4.979047in,y=3.289477in,left,base]{\color{textcolor}\sffamily\fontsize{10.000000}{12.000000}\selectfont SNN}%
\end{pgfscope}%
\begin{pgfscope}%
\pgfsetbuttcap%
\pgfsetmiterjoin%
\definecolor{currentfill}{rgb}{0.000000,0.500000,0.000000}%
\pgfsetfillcolor{currentfill}%
\pgfsetlinewidth{0.000000pt}%
\definecolor{currentstroke}{rgb}{0.000000,0.000000,0.000000}%
\pgfsetstrokecolor{currentstroke}%
\pgfsetstrokeopacity{0.000000}%
\pgfsetdash{}{0pt}%
\pgfpathmoveto{\pgfqpoint{4.590158in}{3.085620in}}%
\pgfpathlineto{\pgfqpoint{4.867936in}{3.085620in}}%
\pgfpathlineto{\pgfqpoint{4.867936in}{3.182842in}}%
\pgfpathlineto{\pgfqpoint{4.590158in}{3.182842in}}%
\pgfpathlineto{\pgfqpoint{4.590158in}{3.085620in}}%
\pgfpathclose%
\pgfusepath{fill}%
\end{pgfscope}%
\begin{pgfscope}%
\definecolor{textcolor}{rgb}{0.000000,0.000000,0.000000}%
\pgfsetstrokecolor{textcolor}%
\pgfsetfillcolor{textcolor}%
\pgftext[x=4.979047in,y=3.085620in,left,base]{\color{textcolor}\sffamily\fontsize{10.000000}{12.000000}\selectfont NN}%
\end{pgfscope}%
\end{pgfpicture}%
\makeatother%
\endgroup%

    \caption{Caption}
    \label{fig:my_label}
\end{figure}

\begin{figure}
\input{./plots/Rijndael_MixColumns_function_binomial.pgf}
    \caption{Caption}
    \label{fig:my_label}
\end{figure}

\printbiliography

\end{document}
