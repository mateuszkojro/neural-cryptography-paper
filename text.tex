\documentclass{artikel3}
\usepackage[utf8]{inputenc}
\usepackage[margin=1in]{geometry}
\usepackage[titletoc,title]{appendix}
\usepackage[T1]{fontenc}
\usepackage{lmodern}
\usepackage{amsmath,amsfonts,amssymb,mathtools}
\usepackage{graphicx,float}
\usepackage{minted}
\usepackage{circuitikz}
\usemintedstyle{trac}
\title{My text title}
\author{Mateusz Kojro}
\date{\today}

\begin{document}

\maketitle

% Abstract
\begin{abstract}
    Add your abstract here.
\end{abstract}


\section{Some circuit}

\begin{circuitikz} \draw
    (2,0)to[battery, l_=$\epsilon$](4,0)
    (4,0)to[R=$R_1$](4,2)
    (2,0)to[R=$R_2$,*-*](2,4)
    (0,0)to[R=$R_3$](0,2)
    (0,2)--(2,4)
    (4,2)--(2,4)
    (0,0)--(2,0)
    ;
    \label{circuit}
\end{circuitikz}

\section{Hello nice code}
\begin{center}
    \begin{minted}[frame=single,]{cpp}
#include <iostream>
int main() { // Thats a main function
    std::cout << "Hello world!" << std::endl;    
    return 0;
}
\end{minted}
\end{center}

% Introduction and Overview
\section{Introduction and Overview}
Add your introduction and overview here.

% Example Subsection
\subsection{Subsection Title}
This is a subsection.

% Example Subsubsection
\subsubsection{Subsubsection Title}
This is a subsubsection.

%  Theoretical Background
\section{Theoretical Background}
Add your theoretical background here. Some example text: As we learned from our textbook \cite{kutz_2013}, Fourier introduced the concept of representing a given function $f(x)$ by a trigonometric series of sines and cosines:
\begin{equation}
    f(x) = \frac{a_0}{2} + \sum_{i=1}^\infty \left(a_n\cos{nx} + b_n\sin{nx}\right) \quad x \in (-\pi,\pi].
    \label{eqn:fourierseries}
\end{equation}
You can reference numbered equations, figures, tables, algorithms, and code like this: Equation~\ref{eqn:fourierseries}, etc.

% Algorithm Implementation and Development
\section{Algorithm Implementation and Development}
Add your algorithm implementation and development here. See Algorithm~\ref{alg:example} for how to include an algorithm in your document. This is how to make an \textit{ordered} list:
\begin{enumerate}
    \item Fluffy swallowed a marble.
    \item I took Fluffy to the vet.
    \item They took an ultrasound of Fluffy's intestines.
\end{enumerate}

% Computational Results
\section{Computational Results}
\begin{table}
    \centering
    \begin{tabular}{rll}
           & Name             & Years        \\
        \hline
        1  & Frosty           & 1922-1930    \\
        2  & Frosty II        & 1930-1936    \\
        3  & Wasky            & 1946         \\
        4  & Wasky II         & 1947         \\
        5  & Ski              & 1954         \\
        6  & Denali           & 1958         \\
        7  & King Chinook     & 1959-1968    \\
        8  & Regent Denali    & 1969         \\
        9  & Sundodger Denali & 1981-1992    \\
        10 & King Redoubt     & 1992-1998    \\
        11 & Prince Redoubt   & 1998         \\
        12 & Spirit           & 1999-2008    \\
        13 & Dubs I           & 2009-2018    \\
        14 & Dubs II          & 2018-Present
    \end{tabular}
    \caption{UW mascots as described in \cite{washington_huskies}.}
    \label{tab:mascots}
\end{table}


% Summary and Conclusions
\section{Summary and Conclusions}
Add your summary and conclusions here.


\end{document}
